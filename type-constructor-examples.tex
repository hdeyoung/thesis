% arara: pdflatex
% arara: pdflatex
\documentclass[tufte-handout]{tufte-thesis}

\usepackage{logic}

\begin{document}

\title{Session Type Constructors: Examples}
\author{Henry DeYoung}
\maketitle

\section{Language of $a^nb^{n+1}$}

Let $L_0$ be the language $\{ a^nb^{n+1} \mid n \geq 0 \}$.
I claim that we may represent this language as a session type $S$ where
\begin{equation*}
  \begin{lgathered}
    S = Q(\plus*{b\colon \one}) \\
    Q(\kappa_b) = \plus*{a\colon Q(\plus*{b\colon \kappa_b}), b\colon \kappa_b}
  \end{lgathered}
\end{equation*}
I claim that the corresponding BPA is:
\begin{equation*}
  \begin{lgathered}
    S = Q_b\,b \\
    Q_b = a\,Q_b\,b + b
  \end{lgathered}
\end{equation*}

\section{Language of $[^na]^na \cup [^nb]^nb$}

Let $L_3$ be the language $\{ [^na]^na \mid n \geq 0 \} \cup \{ [^nb]^nb \mid n \geq 0 \}$.
From what I recall, the session type that we wrote (modulo some renaming) was:
\begin{equation*}
  \begin{lgathered}
    S = Q(\plus*{a\colon \one} , \plus*{b\colon \one}) \\
    Q(\kappa_a, \kappa_b) = \plus*{[\colon Q(\plus*{]\colon \kappa_a}, \plus*{]\colon \kappa_b}), a\colon \kappa_a, b\colon \kappa_b}
  \end{lgathered}
\end{equation*}
I think that the BPA that we wrote (again, modulo some renaming) was:
\begin{equation*}
  \begin{lgathered}
    S = Q_a\,a + Q_b\,b \\
    Q_a = [\,Q_a\,] + a \\
    Q_a = [\,Q_b\,] + b
  \end{lgathered}
\end{equation*}

\section{}

The example on page 34 in Solomon's paper corresponds to the following type, I think.
\begin{equation*}
  \begin{lgathered}
    t(x_1, x_2) = \plus*{a\colon t(\plus*{a\colon \mathit{integer}, b\colon x_1, c\colon \mathit{real}},
                                   \plus*{a\colon \mathit{integer}, b\colon \mathit{real}, c\colon x_2}) ,
                         b\colon x_1,
                         c\colon x_2}
  \end{lgathered}
\end{equation*}
Based on the grammar that Solomon gives on page 34, I would expect the following BPA to correspond to his $Address(integer, t)$.
\begin{equation*}
  \begin{lgathered}
    t = a\,t + a\,t_b\,a + a\,t_c\,a \\
    t_b = a\,t_b\,b + b \\
    t_c = a\,t_c\,c + c
  \end{lgathered}
\end{equation*}
This BPA does correspond to the language $\{ a^mb^na \mid m \geq n \geq 1 \} \cup \{ a^mc^na \mid m \geq n \geq 1 \}$ because $t_b$ generates the language $\{ a^ib^{i+1} \mid i \geq 0 \}$ and $t_c$ generates the language $\{ a^ic^{i+1} \mid i \geq 0 \}$.

On page 34, Solomon says that 
the nonterminals $t_b$ and $t_c$ are constructed as ``the set of paths from the root of partially expanded versions of $t$ to an occurrence of $x_1$'' and $x_2$, repectively.
I think that sentence of Solomon's is the key to slicing a session type into BPA nonterminals.

What I don't yet understand is what $\mathit{Address}(\mathit{integer},t)$ has to do with $t$ overall.

\end{document}
