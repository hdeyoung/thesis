\chapter{Session-typed processes}\label{ch:session-typed-procs}

\begin{itemize}
\item Connections to SILL
\end{itemize}

\section{Singleton linear logic}\label{sec:singleton-logic}

\begin{syntax*}
  Propositions &
    A & \alpha \mid A \plus B \mid \zero \mid A \with B \mid \top
  \\
  Contexts & \sctx & A
\end{syntax*}

\begin{inferences}
  \infer[\jrule{CUT}^A]{\slseq{\sctx |- C}}{
    \slseq{\sctx |- A} & \slseq{A |- C}}
  \and
  \infer[\jrule{ID}^A]{\slseq{A |- A}}{}
  \\
  \infer[\rrule{\plus}_1]{\slseq{\sctx |- A \plus B}}{
    \slseq{\sctx |- A}}
  \and
  \infer[\rrule{\plus}_2]{\slseq{\sctx |- A \plus B}}{
    \slseq{\sctx |- B}}
  \and
  \infer[\lrule{\plus}]{\slseq{A \plus B |- C}}{
    \slseq{A |- C} & \slseq{B |- C}}
  \\
  \text{(no $\rrule{\zero}$ rule)}
  \and
  \infer[\lrule{\zero}]{\slseq{\zero |- C}}{}
  \\
  \infer[\rrule{\with}]{\slseq{\sctx |- A \with B}}{
    \slseq{\sctx |- A} & \slseq{\sctx |- B}}
  \and
  \infer[\lrule{\with}_1]{\slseq{A \with B |- C}}{
    \slseq{A |- C}}
  \and
  \infer[\lrule{\with}_2]{\slseq{A \with B |- C}}{
    \slseq{B |- C}}
  \\
  \infer[\rrule{\top}]{\slseq{\sctx |- \top}}{}
  \and
  \text{(no $\lrule{\top}$ rule)}
\end{inferences}

This is not the additive fragment of linear logic.

\begin{theorem}
  The $\jrule{CUT}^A$ and $\jrule{ID}^A$ rules are admissible in the system having
  \begin{equation*}
    \infer[\normalfont\jrule{ID}^{\alpha}]{\slseq{\alpha |- \alpha}}{}
  \end{equation*}
  as its only judgmental rule.
\end{theorem}


\subsection{A subsingleton extension}

Multiplicative units


\section{Session-typed processes: A Curry--Howard interpretation of singleton linear logic}

\begin{syntax*}
  Session types &
    A & \alpha \mid \plus*[sub=_{\ell \in L}]{\ell:A_{\ell}} \mid \with*[sub=_{\ell \in L}]{\ell:A_{\ell}}
  \\
  Process terms &
    P,Q & \spawn{P}{Q} \mid \fwd
            \begin{array}[t]{@{{} \mid {}}l@{}}
              \selectR{\kay} \mid \caseL[\ell \in L]{\ell => Q_{\ell}} \\
              \caseR[\ell \in L]{\ell => P_{\ell}} \mid \selectL{\kay}
            \end{array}
\end{syntax*}

\begin{inferences}
  \infer[\jrule{CUT}^A]{\slof{\sctx |- \spawn{P}{Q} : C}}{
    \slof{\sctx |- P : A} & \slof{A |- Q : C}}
  \and
  \infer[\jrule{ID}^A]{\slof{A |- \fwd : A}}{}
  \\
  \infer[\rrule{\plus}]{\slof{A_{\kay} |- \selectR{\kay} : \plus*[sub=_{\ell \in L}]{\ell:A_{\ell}}}}{
    \text{($\kay \in L$)}}
  \and
  \infer[\lrule{\plus}]{\slof{\plus*[sub=_{\ell \in L}]{\ell:A_{\ell}} |- \caseL[\ell \in L]{\ell => Q_{\ell}} : C}}{
    \multipremise{\ell \in L}{\slof{A_{\ell} |- Q_{\ell} : C}}}
  \\
  \infer[\rrule{\with}]{\slof{\sctx |- \caseR[\ell \in L]{\ell => P_{\ell}} : \with*[sub=_{\ell \in L}]{\ell:A_{\ell}}}}{
    \multipremise{\ell \in L}{\slof{\sctx |- P_{\ell} : A_{\ell}}}}
  \and
  \infer[\lrule{\with}]{\slof{\with*[sub=_{\ell \in L}]{\ell:A_{\ell}} |- \selectL{\kay} : C}}{
    \text{($\kay \in L$)}}
\end{inferences}

Discussion of cut reduction vs.\ admissibility

\begin{syntax*}
  Configurations &
    \cnf & \cnfe \mid (\cnf_1 \cc \cnf_2) \mid P
\end{syntax*}

\begin{inferences}
  \infer[\jrule{CUT}^{\sctx'}]{\slcof{\sctx |- \cnf_1 \cc \cnf_2 : \sctx''}}{
    \slcof{\sctx |- \cnf_1 : \sctx'} & \slcof{\sctx' |- \cnf_2 : \sctx''}}
  \and
  \infer[\jrule{ID}^{\sctx}]{\slcof{\sctx |- \cnfe : \sctx}}{}
  \and
  \infer[\jrule{PROC}]{\slcof{\sctx |- P : A}}{
    \slof{\sctx |- P : A}}
\end{inferences}

\begin{inferences}
  \infer{\spawn{P}{Q} \reduces P \cc Q}{}
  \and
  \infer{\fwd \reduces \cnfe}{}
  \\
  \infer{\selectR{\kay} \cc \caseL[\ell \in L]{\ell => Q_{\ell}} \reduces Q_{\kay}}{}
  \and
  \infer{\caseR[\ell \in L]{\ell => P_{\ell}} \cc \selectL{\kay} \reduces P_{\kay}}{}
\end{inferences}

\begin{itemize}
\item SSOS is an ordered rewriting specification.  (How does this work with definitions?)
\end{itemize}

\begin{equation*}
  \begin{lgathered}
    \proc{\spawn{P}{Q}} \defd \proc{P} \fuse \proc{Q} \\
    \proc{\fwd} \defd \one \\
    \proc{\caseL[\ell \in L]{\ell => Q_{\ell}}} \defd \bigwith_{\ell \in L}\bigl(\proc{\selectR{\ell}} \limp \proc{Q_{\ell}}\bigr) \\
    \proc{\caseR[\ell \in L]{\ell => P_{\ell}}} \defd \bigwith_{\ell \in L}\bigl(\proc{P_{\ell}} \pmir \proc{\selectL{\ell}}\bigr)
  \end{lgathered}
\end{equation*}

\subsection{Example: Binary counter}

\begin{equation*}
  \proc{b_1} \defd (\msgL{i} \fuse \proc{b_0} \pmir \msgL{i}) \with (\proc{b_0} \fuse \msgR{s} \pmir \msgL{d})
\end{equation*}

\subsection{Example: \Aclp*{DFA}}

Contrast with inability to express \acp{NFA} (languages vs.\ operational semantics)

%%% Local Variables:
%%% mode: latex
%%% TeX-master: "thesis"
%%% End:
