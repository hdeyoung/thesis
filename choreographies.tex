\chapter{A formula-as-process interpretation of ordered rewriting}\label{ch:formula-as-process}

As demonstrated in \cref{ch:string-rewriting,ch:ordered-rewriting}, string rewriting and the more expressive (focused) ordered rewriting are suitable frameworks for describing the dynamics of concurrent systems whose components have a monoidal structure.
But, as formulated thus far, these rewriting descriptions are only abstract specifications -- they lack the clear notion of local communication and decentralized execution

String rewriting axioms, $w \reduces w'$, and ordered rewriting steps\fixnote{wc}, $\octx \reduces \octx'$, are strictly global in their phrasing, stating merely that any substate of the form $w$ or $\octx$ may be transformed, \foreigntext{en masse}, into $w'$ or $\octx'$.
Because nothing at all is said about how this transformation is achieved, the rewritings specified by these frameworks are [...].
It might as well be assumed that a meta-level actor is responsible for coordinating and conducting the rewriting, with the (sub-)states and their constituent symbols or propositions as mere passive objects.

As outlined in \cref{ch:introduction}, our eventual goal is to relate these rewriting-based specifications to (session-typed) message-passing concurrent processes.
With an eye on this goal, the contributions of this \lcnamecref{ch:formula-as-process} push ordered rewriting toward a lower level of abstraction.

First, in \cref{sec:formula-as-process:interpretation}, we refine the focused ordered rewriting framework of the previous \lcnamecref{ch:ordered-rewriting} into one that can be cleanly given a \emph{formula-as-process} interpretation\autocites{Miller:??}{Cervesato+Scedrov:IC09}.
Under this interpretation, atomic propositions may be viewed as messages; (compound) propositions, as processes; contexts, as configurations of those processes; and rewriting, as message-passing communication [among the processes in a configuration].\fixnote{fix}
Somewhat surprisingly, only two small tweaks (and one optional addition) to the structure of propositions are needed to make this formula-as-process interpretation viable.

Second, with the new perspective that this formula-as-process interpretation supplies, we can now understand ordered rewriting in terms of local interactions.
\Cref{sec:formula-as-process:interaction} presents a labeled transition semantics for

\ac{FOR}
\Ac{FOR}

Finally, in \cref{sec:formula-as-process:choreographies}, we introduce a notion of \emph{choreography} that relates a string rewriting specification to 


\newthought{In process calculi} like the $\pi$-calculus, a reduction semantics often plays a lesser role than a labeled transition semantics does.
But here the roles are reversed, with the labeled transition semantics taking a back seat, because of the reduction semantics's clearer connection to rewriting.

The other contribution of this \lcnamecref{ch:choreographies} is the idea of choreographing string rewriting specifications into this formula-as-process refinement of ordered rewriting.

This formula-as-process interpretation nudges ordered rewriting away from a state-transformation model of concurrency toward a process-based model.
Computation is still driven by derivability



\newthought{As shown in} \cref{ch:string-rewriting}, string rewriting is a suitable framework for describing the dynamics of concurrent systems whose components have a monoidal structure.
But these string rewriting descriptions are only abstract specifications from a state-transformation perspective -- they lack a clear notion of local, decentralized execution and therefore give only a global characterization of the interactions between components.

String rewriting axioms $w \reduces w'$ are strictly global in their phrasing, stating merely that any substring of the form $w$ may be replaced, en masse, with $w'$ -- nothing is said about how this replacement is achieved.
The string rewriting framework therefore implicitly suggests that a meta-level actor is responsible for coordinating and conducting the rewriting, with substrings and their constituent symbols as mere passive accessories\fixnote{word choice?}.

In this \lcnamecref{ch:choreographies}, our goal is to move toward a lower level of abstraction.


As an example, recall from \cref{ch:string-rewriting} the string rewriting specification of a system that may transform strings that end with $b$ into the empty string:
\begin{equation}
  \infer{a \wc b \reduces b}{}
  \qquad
  \infer{b \reduces \emp}{}
\end{equation}


Accordingly, we refine the focused ordered rewriting framework of the previous \lcnamecref{ch:ordered-rewriting} into one that can be given a \vocab{formula-as-process} interpretation in which [ordered] rewriting faithfully represents message-passing communication among processes that are arranged in a linear topology.
Then, in \cref{??}, we describe how a string rewriting specification may be transformed into an ordered rewriting \vocab{choreography}.

In this \lcnamecref{ch:choreographies}, we refine the focused ordered rewriting framework of the previous \lcnamecref{ch:ordered-rewriting} into one that can be given a \vocab{formula-as-process} interpretation in which rewriting faithfully represents message-passing communication among 

As argued 

Then we show how, given a mapping of symbols to either [...], choreographies can be generated from a string rewriting specification.



\section{Refining ordered rewriting: A formula-as-process interpretation}

In this section, we present a \emph{formula-as-process} interpretation of focused ordered rewriting in which propositions are viewed as message-passing processes.

More specifically,
% under this formula-as-process interpretation,
positive atoms, $\p{a}$, may be viewed as messages, and negative propositions, $\n{A}$, as processes that receive and react to those messages.
Ordered contexts, $\np{\octx}$, which consist of negative propositions and positive atoms, are then linear-topology configurations of processes and messages.
And positive propositions, $\p{A}$, which reify ordered contexts as propositions, are then processes that reify configurations.
Lastly, the rewriting relation, $\reduces$, is viewed as a reduction semantics for message-passing communication among the processes in a configuration.

Perhaps surprisingly, only three small tweaks to the structure of propositions are needed to make this formula-as-process reading viable.
\begin{itemize}
\item
  The positive atoms are now partitioned into two classes, left- and right-directed atoms, to allow us to identify the direction in which a message is flowing.

\item
  The left- and right-handed implications are now restricted to have atomic premises with a matching\fixnote{?} direction.
  Instead of the more general $\p{A} \limp \n{B}$ and $\n{B} \pmir \p{A}$, now only implications of the forms $\p{\atmR{a}} \limp \n{B}$ and $\n{B} \pmir \p{\atmL{a}}$ are permitted.
  These are more cleanly interpreted as inputs of single messages.

\item 
  Recursively defined negative propositions, $\n{\defp{p}} \defd \n{A}$, are introduced to correspond to recursive processes.
\end{itemize}

\subsection{Left- and right-directed atoms as directed messages}

The first tweak to the structure of propositions is that the positive atoms are now partitioned into two classes: left- and right-directed atoms.
These directions, which we denote by an arrow placed below the atom, indicate the direction in which the corresponding message flows.
Left-directed atoms, $\p{\atmL{a}}$, are messages that are being sent to the left; right-directed atoms, $\p{\atmR{a}}$, are messages that are being sent to the right.

\subsection{Implications restricted to atomic premises as input processes}

\subsection{Recursively defined negative propositions}

Recall from \cref{ch:ordered-rewriting} that \acl{FOR} is terminating: for all ordered contexts $\np{\octx}$, every rewriting sequence from $\np{\octx}$ is finite~\parencref{thm:ordered-rewriting:??}.
Although a seemingly pleasant property, termination significantly limits the expressiveness of \acl{FOR}.
For example, without unbounded rewriting, we cannot even describe producer--consumer systems or finite automata.

As the proof of termination shows, rewriting is bounded precisely because contexts consist of finitely many \emph{finite} propositions.
In multiset and ordered rewriting, unbounded behavior is traditionally introduced by way of persistent propositions that may be replicated as much as needed\autocites{??}{Polakow:CMU??}{Simmons:CMU12}.

However, another option -- and the one that we pursue here -- is to permit circular negative propositions in the form of mutually recursive definitions, $\n{\defp{p}} \defd \n{A}$, where the grammar of negative propositions now includes these recursively defined propositions:
\begin{equation*}
  \n{A}, \n{B} \Coloneqq \p{\atmR{a}} \limp \n{B} \mid \n{B} \pmir \p{\atmL{a}} \mid \n{A} \with \n{B} \mid \top \mid \n{\defp{p}}
  \,.
\end{equation*}
Sequent calculi with definitions of this kind have previously been studied\autocites{Hallnas:??}{Erikkson:??}{Schroeder-Heister:??}{McDowell+Miller:??}{Tiu+Momigliano:??}, but, to the best of our knowledge, the use of recursive definitions in the context of logically motivated rewriting systems is new.

To rule out definitions like $\n{\defp{p}} \defd \n{\defp{p}}$ that do not correspond to sensible infinite propositions and also sensible but inessential definitions like $\n{\defp{p}} \defd \n{\defp{q}}$, we require that all definitions be \emph{contractive}\autocite{??} -- \ie, that the body of each recursive definition begin with a logical connective (or logical constant or atomic proposition) at the top level.

Recursive definitions are collected into a signature, $\orsig$, that indexes the rewriting relations: $\reduces_{\orsig}$ and $\Reduces_{\orsig}$.%
\footnote{We nearly always elide the index, as it is usually clear from context.}
Syntactically, these signatures are given by
\begin{equation*}
  \orsig \Coloneqq \orsige \mid \orsig, (\n{\defp{p}} \defd \n{A})
  \,.
\end{equation*}

\newthought{By analogy with} recursive types from functional programming\autocite{??}, we must now decide whether to treat definitions \emph{iso}\-recursively or \emph{equi}\-recursively.
Under an equirecursive treatment, definitions may be silently unrolled or rolled at will;
in other words, $\n{\defp{p}}$ is literally \emph{equal} to its unrolling: $\n{\defp{p}} = \n{A}$.
In contrast, under an isorecursive treatment, unrolling a recursively defined proposition would count only as an explicit step of rewriting: $\n{\defp{p}} \neq \n{A}$ but 
\marginnote{%
  $\infer[\lrule{\defd}]{\lfocus{\atmR{\octx}_L}{\n{\defp{p}}}{\atmL{\octx}_R}{_{\orsig} \p{C}}}{
    \text{($(\n{\defp{p}} \defd \n{A}) \in \orsig$)} &
    \lfocus{\atmR{\octx}_L}{\n{A}}{\atmL{\octx}_R}{_{\orsig} \p{C}}}$}

We choose an equirecursive treatment of definitions because the accompanying generous notion of equality helps to minimize the overhead of recursively defined propositions.
As a simple example, under the equirecursive definition $\n{\defp{p}} \defd \p{\atmR{a}} \limp \up \dn \n{\defp{p}}$, we have the trace
\begin{equation*}
  \p{\atmR{a}} \oc \p{\atmR{a}} \oc \n{\defp{p}}
    = \p{\atmR{a}} \oc \p{\atmR{a}} \oc (\p{\atmR{a}} \limp \up \dn \n{\defp{p}})
    \reduces \p{\atmR{a}} \oc \n{\defp{p}}
\end{equation*}

However, because the left-focus judgment is defined inductively, not coinductively, there are some recursively defined negative propositions that cannot successfully be put into focus.
Under the definition $\n{\defp{p}} \defd \p{\atmR{a}} \limp \n{\defp{p}}$, for example, there are no contexts $\atmR{\octx}_L$ and $\atmL{\octx}_R$ and positive consequent $\p{C}$ for which $\lfocus{\atmR{\octx}_L}{\n{\defp{p}}}{\atmL{\octx}_R}{\p{C}}$ is derivable.
To derive a left-focus judgment on $\n{\defp{p}}$, the finite context $\atmR{\octx}_L$ would need to hold an infinite stream of $\p{\atmR{a}}$ atoms, which is impossible in an inductively defined, and hence finite, context.

However, by inserting $\up \dn$ as a double shift to blur focus -- in a way similar to how double shifts were used in the embedding of unfocused rewriting~\parencref{sec:ordered-rewriting:??} -- the definition can be revised to one that admits a left-focus judgment.
Specificaly, if $\n{\defp{p}} \defd \p{\atmR{a}} \limp \up \dn \n{\defp{p}}$, then $\lfocus{\p{\atmR{a}}}{\n{\defp{p}}}{}{\dn \n{\defp{p}}}$ is derivable, and so $\p{\atmR{a}} \oc \n{\defp{p}} \reduces \n{\defp{p}}$.


\section{}


This interpretation is summarized in the adjacent \lcnamecref{fig:choreographies:propctx-table}.%
\begin{margintable}
  \begin{center}
    \begin{tabular}{@{}l@{\enspace\ }>{\itshape}l@{}}
      $\p{\atmL{a}}$ & left-directed message \\
      $\p{\atmR{a}}$ & right-directed message \\
      $\n{A}$ & message-passing process \\
      $\np{\octx}$ & process configuration \\
      $\p{A}$ & configuration reified as a process
    \end{tabular}
  \end{center}
  \caption{A formula-as-process interpretation of polarized ordered propositions and contexts}\label{fig:choreographies:propctx-table}
\end{margintable}%

\newthought{}
Under the formula-as-process interpretation, the rewriting judgment, $\np{\octx} \reduces \np{\octx'{}}$, is viewed as message-passing communication within the process configuration $\np{\octx}$.


\begin{equation*}
  \n{A}, \n{B} \Coloneqq \p{\atmR{a}} \limp \up \p{B} \mid \up \p{B} \pmir \p{\atmL{a}} \mid \n{A} \with \n{B} \mid \top \mid \up \p{A}
\end{equation*}

\begin{equation*}
  \p{A}, \p{B} \Coloneqq \p{\atmL{a}} \mid \p{\atmR{a}} \mid \p{A} \fuse \p{B} \mid \one \mid \dn \n{A}
\end{equation*}



\begin{align*}
  \n{A} &\Coloneqq \p{\atmR{a}} \limp \n{B} \mid \n{B} \pmir \p{\atmL{a}} \mid \n{A} \with \n{B} \mid \top \mid \up \p{A} \mid \n{\defp{p}}
\end{align*}

\begin{margintable}
  \begin{center}
    \begin{tabular}{@{}r@{\enspace}>{\itshape}l@{}}
      $\p{\atmR{a}} \limp \n{B}$ & receive message $\p{\atmR{a}}$ from the right \\
      $\n{B} \pmir \p{\atmL{a}}$ & receive message $\p{\atmL{a}}$ from the left \\
      $\n{A} \with \n{B}$ & nondeterministic branching \\% continue as $\n{A}$ or $\n{B}$ \\
      $\top\hphantom{\n{}}$ & \\
      $\up \p{A}$ & \\
      $\n{\defp{p}}$ & call a recursively defined process
    \end{tabular}
  \end{center}
  \caption{A formula-as-process interpretation of negative propositions}\label{fig:choreographies:negprop-table}
\end{margintable}

\begin{margintable}
  \begin{center}
    \begin{tabular}{@{}r@{\enspace}>{\itshape}l@{}}
      $\np{\octx}_1 \oc \np{\octx}_2$ & parallel composition of configurations \\
      $(\octxe)$ & empty configuration \\
      $\n{A}$ & single-process configuration \\
      $\p{\atmL{a}}$ & left-directed message \\
      $\p{\atmR{a}}$ & right-directed message
    \end{tabular}
  \end{center}
  \caption{A formula-as-process interpretation of contexts}\label{fig:choreographies:ctxprop-table}
\end{margintable}

\begin{margintable}
  \begin{center}
    \begin{tabular}{@{}r@{\enspace}>{\itshape}l@{}}
      $\p{\atmL{a}}$ & left-directed message \\
      $\p{\atmR{a}}$ & right-directed message \\
      $\p{A} \fuse \p{B}$ & parallel composition of $\p{A}$ and $\p{B}$ \\
      $\one\hphantom{\p{}}$ & forwarding process \\
      $\dn \n{A}$ & 
    \end{tabular}
  \end{center}
  \caption{A formula-as-process interpretation of positive propositions}\label{fig:choreographies:posprop-table}
\end{margintable}

In focused ordered rewriting, ordered contexts consist of positive atoms, $\p{a}$, and negative propositions, $\n{A}$.
Under the formula-as-process interpretation, positive atoms will be viewed as messages, and negative propositions will be viewed as processes that receive and react to those messages.
Ordered contexts are then configurations of processes and messages, arranged in a linear topology.





Positive atoms, $\p{a}$


Negative propositions, $\n{A}$, are interpreted as message-passing processes, with positive atoms, $\p{a}$, as messages passed between them.
Ordered contexts, $\octx$, are then configurations of processes and messages arranged in a linear topology.
Finally, the positive propositions, $\p{A}$, reify ordered contexts, and so they can be interpreted as process expressions that reify process configurations.

The ordered implications $\p{A} \limp \n{B}$ and $\n{B} \pmir \p{A}$ are restricted to $\p{a} \limp \n{B}$ and $\n{B} \pmir \p{a}$, respectively, so that they may cleanly be interpreted as processes that input a message $\p{a}$ from the left and right, respectively.


Each positive atom $\p{a}$ is assigned a direction, either $\atmL{a}$ or $\atmR{a}$, that indicates 

$\atmL{a}$ and $\atmR{a}$; and $\p{A} \limp \n{B}$ restricted to $\atmR{a} \limp \n{B}$ and similarly for right-handed implication.

\subsection{Formula-as-process}

Discuss here?

Example of $\proc{b} \defd (\atmR{a} \limp \up \dn \proc{b}) \with \up \one$, without explicitly relating it to the specification.

\subsection{Focused ordered rewriting, revisited}

The two changes introduced by the formula-as-process interpretation -- atom directions and atomic premises for implications -- trickle down to the focused ordered rewriting framework.

First, because each positive atom is now marked with a direction, the $\jrule{ID}\smash{^{\p{a}}}$ rule%
\marginnote{\qquad%
  $\infer[\jrule{ID}\smash{^{\p{a}}}]{\rfocus{\p{a}}{\p{a}}}{}$%
}
that was previously part of the right-focus judgment's definition is replaced by two similar rules:
\begin{equation*}
  \infer[\jrule{ID}\smash{^{\atmR{a}}}]{\rfocus{\atmR{a}}{\atmR{a}}}{}
  \qquad\text{and}\qquad
  \infer[\jrule{ID}\smash{^{\atmL{a}}}]{\rfocus{\atmL{a}}{\atmL{a}}}{}
  \,.
\end{equation*}
The other right-focusing rules remain unchanged.

Second, because left-handed implications now have only right-directed atoms as premises and right-handed implications have only left-directed atoms as premises, the left-focus judgment and its rules must be revised.
Instead of $\lfocus{\np{\octx}_L}{\n{A}}{\np{\octx}_R}{\p{C}}$, which has arbitrary contexts to the left and right of $\n{A}$, the judgment is now $\lfocus{\atmR{\octx}_L}{\n{A}}{\atmL{\octx}_R}{\p{C}}$ -- the left-hand context consists only of right-directed atoms, hence $\atmR{\octx}_L$; symmetrically, the right-hand context consists only of left-directed atoms, hence $\atmL{\octx}_R$.

With the restriction to atomic premises, we need to reconsider the left-focus rules for the left- and right-handed implications.
By inversion under the previous set of left-focus rules, any derivation focused on $\atmR{a} \limp \n{B}$ would end with
\begin{equation*}
  \infer[\lrule{\limp}]{\lfocus{\atmR{\octx}_L \oc \atmR{a}}{\atmR{a} \limp \n{B}}{\atmL{\octx}_R}{\p{C}}}{
    \infer[\jrule{ID}\smash{^{\atmR{a}}}]{\rfocus{\atmR{a}}{\atmR{a}}}{} &
    \lfocus{\atmR{\octx}_L}{\n{B}}{\atmL{\octx}_R}{\p{C}}}
  %
  \qquad
  %
  \infer[\lrule{\limp}']{\lfocus{\atmR{\octx}_L \oc \atmR{a}}{\atmR{a} \limp \n{B}}{\atmL{\octx}_R}{\p{C}}}{
    \lfocus{\atmR{\octx}_L}{\n{B}}{\atmL{\octx}_R}{\p{C}}}
\end{equation*}
By similar reasoning, we arrive at a $\lrule{\pmir}'$ rule, as well:
\begin{equation*}
  \infer[\lrule{\pmir}']{\lfocus{\atmR{\octx}_L}{\n{B} \pmir \atmL{a}}{\atmL{a} \oc \atmL{\octx}_R}{\p{C}}}{
    \lfocus{\atmR{\octx}_L}{\n{B}}{\atmL{\octx}_R}{\p{C}}}
\end{equation*}
The other rules for the left-focus judgment remain unchanged, save for the fact that the left- and right-hand context now contain only atoms of the appropriate direction.
\Cref{fig:??} summarizes the revised rules for the right- and left-focus judgments.

Having refined the left-focus judgement to use input message contexts, we may similarly refine the reduction rules, $\jrule{$\reduces$I}$ and $\jrule{$\reduces$C}$:
\begin{inferences}
  \infer[\jrule{$\reduces$I}]{\atmR{\octx}_L \oc \n{A} \oc \atmL{\octx}_R \reduces \np{\octx'{}}}{
    \lfocus{\atmR{\octx}_L}{\n{A}}{\atmL{\octx}_R}{\p{B}} &
    \rfocus{\np{\octx'{}}}{\p{B}}}
  \and
  \infer[\jrule{$\reduces$C}]{\np{\octx}_L \oc \np{\octx} \oc \np{\octx}_R \reduces \np{\octx}_L \oc \np{\octx'{}} \oc \np{\octx}_R}{
    \np{\octx} \reduces \np{\octx'{}}}
\end{inferences}

\subsection{}

In unfocused and focused ordered rewriting of the \ac{OR}~\parencref{??} and \ac{FOR}~\parencref{??} frameworks, the rewriting relation, $\reduces$, described a purely internal operation: $\octx \reduces \octx'$ held independently of any environment that might surround $\octx$.
Ordered rewriting's isolationism was affirmed by its compatibility rule, $\jrule{$\reduces$C}$%
\marginnote{%
$\infer[\jrule{$\reduces$C}]{\octx_L \oc \octx \oc \octx_R \reduces \octx_L \oc \octx' \oc \octx_R}{
  \octx \reduces \octx'}$}%
, which shows that the environment remains unaffected by the rewriting of its [...].

Now that we have a formula-as-process interpretation to ordered rewriting, we should reconsider this strict isolationism.
Under our formula-as-process reading, this isolationist rewriting judgment corresponds to a reduction semantics for processes.
But now, messages, as represented by the directed $\p{\atmL{a}}$ and $\p{\atmR{a}}$ atoms, make it possible to describe the interactions that a configuration $\octx$ offers to its surroundings. 

\subsection{Input transitions}


For the formula-as-process interpretation, we have thus far examined the rewriting judgement, $\octx \reduces \octx'$, and suggested that it represents a kind of reduction semantics for the underlying processes.
% Now that we are ascribing a formula-as-process interpretation to ordered rewriting, this judgment characterizes internal reductions.

A reduction semantics is not the only way to describe the operational semantics.
In process calculi like the $\pi$-calculus, labeled transition systems are frequently used as an alternative to a reduction semantics, particularly when [...].
Now that we are ascribing a formula-as-process interpretation to ordered rewriting, we can similarly conceive of a labeled transition system for ordered contexts.

Interactions [...]
Rather than adopting an explicit judgement for output interactions, we make use of context equality. 
The context $\octx$ outputs messages $\atmL{\octx}_L$ to the left and messages $\atmR{\octx}_R$ to the right exactly when $\octx = \atmL{\octx}_L \oc \octx' \oc \atmR{\octx}_R$ for some context $\octx'$.
We will sometimes refer to the context $\octx'$ here as the \vocab{continuation context} because it represents\fixnote{wc?} the state after the output of $\atmL{\octx}_L$ and $\atmR{\octx}_R$ occurs.

Input interactions do require an explicit judgment, $\ireduces{\atmR{\octx}_L \oc #1 \oc \atmL{\octx}_R}{\octx}{\octx'}$, indicating that upon receiving messages $\atmR{\octx}_L$ from the left and $\atmL{\octx}_R$ from the right, the context $\octx$ evolves to $\octx'$ in a single step.

We think of $\octx$ as the sole innput to the input transition judgement, and $\atmR{\octx}_L$, $\atmL{\octx}_R$, and $\octx'$ as outputs of the judgement.
The input transition judgement asks \enquote{What input messages suffice for $\octx$ to make a transition?}

% Labeled transition systems for process calculi usually are centered around output and input transitions 

At its heart, each transition derives from focusing on a single negative proposition, $\n{A}$, as captured by the [....] rule:
\begin{equation*}
  \infer[?]{\ireduces{\atmR{\octx}_L \oc #1 \oc \atmL{\octx}_R}{\n{A}}{\octx'}}{
    \lfocus{\atmR{\octx}_L}{\n{A}}{\atmL{\octx}_R}{\p{C}} &
    \rfocus{\octx'}{\p{C}}}
  \,.
\end{equation*}
Aside from the change of judgment in the rule's conclusion, this [...] rule is identical to the core [...] rule for reduction.
How can we claim that input transition is distinct from reduction?

The difference is one of input/output modes.
In a reduction $\atmR{\octx}_L \oc \n{A} \oc \atmL{\octx}_R \reduces \np{\octx'{}}$, the entire $\atmR{\octx}_L \oc \n{A} \oc \atmL{\octx}_R$ context is treated as an input to the reduction judgment.
In the input transition $\ireduces{\atmR{\octx}_L \oc #1 \oc \atmL{\octx}_R}{\n{A}}{\np{\octx'{}}}$, on the other hand, the proposition $\n{A}$ is treated as an input to the input transition judgment, but the contexts $\atmR{\octx}_L$, $\atmL{\octx}_R$, and $\np{\octx'{}}$ are all treated as outputs.

In addition to this core input transition rule, 

\begin{equation*}
  \infer[?]{\ireduces{\atmR{\octx}_L \oc #1 \oc \atmL{\octx}_R}{\n{A}}{\np{\octx'{}}}}{
    \lfocus{\atmR{\octx}_L}{\n{A}}{\atmL{\octx}_R}{\p{C}} &
    \rfocus{\np{\octx'{}}}{\p{C}}}
\end{equation*}


Also discuss here?

\begin{inferences}
  \infer{\ireduces{\atmR{\octx}_L \oc #1 \oc \atmL{\octx}_R}{\n{A}}{\octx'}}{
    \lfocus{\atmR{\octx}_L}{\n{A}}{\atmL{\octx}_R}{\p{C}} &
    \rfocus{\octx'}{\p{C}}}
  \\
  \infer{\ireduces{\atmR{\octx}_L \oc #1 \oc \atmL{\octx}_R}{\atmR{a} \oc \octx}{\octx'}}{
    \ireduces{\atmR{\octx}_L \oc \atmR{a} \oc #1 \oc \atmL{\octx}_R}{\octx}{\octx'}}
  \and
  \infer{\ireduces{\atmR{\octx}_L \oc #1 \oc \atmL{\octx}_R}{\octx \oc \atmL{a}}{\octx'}}{
    \ireduces{\atmR{\octx}_L \oc #1 \oc \atmL{a} \oc \atmL{\octx}_R}{\octx}{\octx'}}
  \\
   \infer{\ireduces{#1 \oc \atmL{\octx}_R}{\atmL{a} \oc \octx}{\atmL{a} \oc \octx'}}{
    \ireduces{#1 \oc \atmL{\octx}_R}{\octx}{\octx'}}
    \and
   \infer{\ireduces{#1 \oc \atmL{\octx}_R}{\atmR{a} \oc \octx}{\atmR{a} \oc \octx'}}{
    \ireduces{#1 \oc \atmL{\octx}_R}{\octx}{\octx'}}
  \and
   \infer{\ireduces{#1 \oc \atmL{\octx}_R}{\n{A} \oc \octx}{\n{A} \oc \octx'}}{
    \ireduces{#1 \oc \atmL{\octx}_R}{\octx}{\octx'}}
  \\
  \infer{\ireduces{\atmR{\octx}_L \oc #1}{\octx \oc \atmL{a}}{\octx' \oc \atmL{a}}}{
    \ireduces{\atmR{\octx}_L \oc #1}{\octx}{\octx'}}
  \and
  \infer{\ireduces{\atmR{\octx}_L \oc #1}{\octx \oc \atmR{a}}{\octx' \oc \atmR{a}}}{
    \ireduces{\atmR{\octx}_L \oc #1}{\octx}{\octx'}}
  \and
  \infer{\ireduces{\atmR{\octx}_L \oc #1}{\octx \oc \n{A}}{\octx' \oc \n{A}}}{
    \ireduces{\atmR{\octx}_L \oc #1}{\octx}{\octx'}}
\end{inferences}

\begin{theorem}
  If $\ireduces{\atmR{\octx}_L \oc #1 \oc \atmL{\octx}_R}{\octx}{\octx'}$, then $\atmR{\octx}_L \oc \octx \oc \atmL{\octx}_R \reduces \octx'$.
  Conversely, if $\octx \reduces \octx'$, then $\ireduces{#1}{\octx}{\octx'}$.
\end{theorem}
\begin{proof}
  The two claims are proved by induction on the structure of the given input transition and reduction, respectively, after first proving an easy lemma:
  \begin{itemize}
  \item If $\ireduces{\atmR{\octx}_L \oc #1 \oc \atmL{\octx}_R}{\octx}{\octx'}$, then $\ireduces{#1}{\atmR{\octx}_L \oc \octx \oc \atmL{\octx}_R}{\octx'}$.
  \qedhere
  \end{itemize}
\end{proof}

\begin{equation*}
  \infer{\octx_L \oc \octx \oc \octx_R \reduces \octx'_L \oc \octx' \oc \octx'_R}{
    \octx_L = \octx'_L \oc \atmR{\octx}_L &
    \ireduces{\atmR{\octx}_L \oc #1 \oc \atmL{\octx}_R}{\octx}{\octx'} &
    \atmL{\octx}_R \oc \octx'_R = \octx_R}
\end{equation*}

\section{}

To choreograph a string rewriting specification, we would like to assign one, and only one, role to each symbol $a \in \sralph$: in the choreography, each symbol $a$ becomes either a message, $\atmL{a}$ or $\atmR{a}$, or a recursively defined process, $\defp{a}$.
A monoid homomorphism from strings to ordered contexts that satisfies this condition is called a \vocab{[...] assignment}.

For example, recall from \cref{ch:string-rewriting} the string rewriting specification of a system that can rewrite strings over $\sralph = \Set{a,b}$ into the empty string if the initial string ends in $b$;
that specification used axioms
\begin{equation*}
  \srsig = (a \wc b \reduces b) , (b \reduces \emp)
  \,.
\end{equation*}
The monoid homomorphism $\theta$ such that $\theta(a) = \atmR{a}$ and $\theta(b) = \defp{b}$ is a [...] assignment for this specification.

When applied to the specification's axioms, the [...] assignment $\theta$ induces the rewriting steps
\begin{equation*}
  \atmR{a} \oc \defp{b} \reduces \defp{b}
  \quad\text{and}\quad
  \defp{b} \reduces (\octxe)
  \,.
\end{equation*}
The [...] assignment $\theta$ yields a well-specified choreography because we can solve these induced rewritings for $\defp{b}$, determining a definition for $\defp{b}$ that makes these -- and only these -- rewriting steps derivable.
Here, that solution is $\defp{b} \defd (\atmR{a} \limp \up \dn \defp{b}) \with \up \one$.
\begin{equation*}
  \atmR{a} \oc \bigl((\atmR{a} \limp \up \dn \defp{b}) \with \up \one\bigr) \reduces \defp{b}
  \quad\text{and}\quad
  (\atmR{a} \limp \up \dn \defp{b}) \with \up \one \reduces (\octxe)
  \,.
\end{equation*}
and because $\octx_L \oc \defp{b} \oc \octx_R \reduces \octx'$ implies either 
\begin{itemize}
\item $\octx_L = \octx'_L \oc \atmR{a}$ and $\octx' = \octx'_L \oc \defp{b} \oc \octx_R$ for some $\octx'_L$;
\item $\octx' = \octx_L \oc \octx_R$;
\item $\octx_L \reduces \octx'_L$ and $\octx' = \octx'_L \oc \defp{b} \oc \octx_R$ for some $\octx'_L$; or
\item $\octx_R \reduces \octx'_R$ and $\octx' = \octx_L \oc \defp{b} \oc \octx'_R$ for some $\octx'_R$.
\end{itemize}

More generally, a [...] assignment $\theta$ yields a well-specified choreography for a specification over alphabet $\sralph$ with axioms $\srsig$ if the induced ordered rewriting steps $\theta(\srsig)$ are solvable.

Not all [...] assignments yield well-specified choreographies.
This happens when there is no solution for the recursively defined propositionsthat makes all of the induced rewritings derivable.
\begin{itemize}
\item
  \emph{Each induced rewriting must have at least one process in its premise\fixnote{wc}.}
  For example, the [...] assignments $\theta'$ such that either $\theta'(b) = \atmL{b}$ or $\theta'(b) = \atmR{b}$ holds do \emph{not} yield well-specified choreographies.
  From the string rewriting axiom $b \reduces \emp$, the [...] assignment $\theta'$ induces either $\atmL{b} \reduces (\octxe)$ or $\atmR{b} \reduces (\octxe)$, and there is no solution that makes either of these induced ordered rewritings derivable.

\item
  \emph{Each induced rewriting must have at most one process in its premise\fixnote{wc}.}
  For example, the [...] assignment $\theta'$ such that $\theta'(a) = \defp{a}$ and $\theta'(b) = \defp{b}$ hold does not yield a well-specified choreography because there is no solution that makes $\defp{a} \oc \defp{b} \reduces \defp{b}$ derivable.

\item
  \emph{Each message in the premises of induced rewritings must be flowing toward that premise's process\fixnote{wc}.}
  For example, the [...] assignment $\theta'$ such that $\theta'(a) = \atmL{a}$ and $\theta'(b) = \defp{b}$ hold does not yield a well-specified choreography because there is no solution that makes $\atmL{a} \oc \defp{b} \reduces \defp{b}$ derivable.
  In \ac{PFOR} there is no process $\defp{b}$ that can receive a message, like $\atmL{a}$, that is flowing away.
\end{itemize}


For the choreography to be well-specified, this [...] assignment must induce from the string rewriting specification's axioms a collection of locally achievable  ordered rewriting steps\fixnote{reductions}.
If the ordered rewriting steps induced by the [...] assignment cannot be achieved by local communication, then the choreography is not well-specified.

For example, recall from \cref{ch:string-rewriting} the string rewriting specification of a system that can rewrite strings over $\sralph = \Set{a,b}$ into the empty string if the initial string ends in $b$;
that specification used axioms
\begin{equation*}
  \srsig = (a \wc b \reduces b) , (b \reduces \emp)
  \,.
\end{equation*}

So, to choreograph this specification, we must choose an assignment of roles -- either message or process -- to symbols $a$ and $b$ --
let's choose $a \mapsto \atmR{a}$ and $b \mapsto \defp{b}$.
From the axioms $\srsig$, this assignment induces the rewritings
\begin{equation*}
  \atmR{a} \oc \defp{b} \reduces \defp{b}
  \quad\text{and}\quad
  \defp{b} \reduces (\octxe)
  \,.
\end{equation*}
Are these reductions achievable by purely local communication?
Because our formula-as-process interpretation of ordered rewriting ensures that all communication is local, we need only verify that there is a solution for $\defp{b}$ [...].

Any solution for $\defp{b}$ must be consistent with $\atmR{a} \limp \up \dn \defp{b}$ so that $\atmR{a} \oc \defp{b} \reduces \defp{b}$ is derivable.
Furthermore, any solution for $\defp{b}$ must be consistent with $\up \one$ so that $\defp{b} \reduces \octxe$ is derivable.
The least such solution is
\begin{equation*}
  \defp{b} \defd (\atmR{a} \limp \up \dn \defp{b}) \with \up \one
  \,,
\end{equation*}
It indeed validates the required reductions,
\begin{gather*}
  \atmR{a} \oc \defp{b} = \atmR{a} \oc \bigl((\atmR{a} \limp \up \dn \defp{b}) \with \up \one\bigr) \reduces \defp{b} \\
  \defp{b} = \atmR{a} \oc \bigl((\atmR{a} \limp \up \dn \defp{b}) \with \up \one\bigr) \reduces (\octxe)
  \,,
\end{gather*}
and only the required reductions:
\begin{quotation}
  If $\octx_L \oc \defp{b} \oc \octx_R \reduces \octx'$, then either:
  \begin{itemize}
  \item $\octx_L = \octx'_L \oc \atmR{a}$ and $\octx' = \octx'_L \oc \defp{b} \oc \octx_R$, for some $\octx'_L$;
  \item $\octx' = \octx_L \oc \octx_R$;
  \item $\octx_L \reduces \octx'_L$ and $\octx' = \octx'_L \oc \defp{b} \oc \octx_R$, for some $\octx'_L$; or
  \item $\octx_R \reduces \octx'_R$ and $\octx' = \octx_L \oc \defp{b} \oc \octx'_R$, for some $\octx'_R$.
  \end{itemize}
\end{quotation}

\begin{gather*}
  \atmL{a} \oc \defp{b} \reduces \defp{b}
  \quad\text{and}\quad
  \defp{b} \reduces (\octxe)
  \\
  \defp{a} \oc \defp{b} \reduces \defp{b}
  \quad\text{and}\quad
  \defp{b} \reduces (\octxe)
  \\
  \defp{a} \oc \atmL{b} \reduces \atmL{b}
  \quad\text{and}\quad
  \atmL{b} \reduces (\octxe)
\end{gather*}



To be well-specified, 

An \emph{[...] assignment} $\theta$ is a monoid homomorphism from strings to ordered contexts that injectively maps each symbol $a \in \sralph$ to either a message, $\atmL{a}$ or $\atmR{a}$, or a recursively defined process, $\defp{a}$.%
\footnote{Injectivity keeps $\theta$ from identifying distinct symbols.}

Given an [...] assignment $\theta$, a string rewriting specification's axioms induce rewriting steps that must hold if the specification is to have a choreography.
For each axiom $w \reduces w' \in \srsig$, the [...] assignment $\theta$ induces a requirement that a faithful choreography must satisfy the rewriting step $\theta(w) \reduces \theta(w')$.

\section{Constructing a choreography from a specification}

For an example of this procedure, let's construct a choreography for the string rewriting specification of the system from \cref{ch:string-rewriting} that can rewrite strings over $\ialph = \Set{a,b}$ into the empty string.
Recall that that specification consisted of the axioms
\begin{equation*}
  \srsig = (a \wc b \reduces b) , (b \reduces \emp)
  \,.
\end{equation*}


The first step in constructing a choreography is to choose a \emph{[...] assignment} that maps each symbol to either an atom or recursively defined proposition, [which represent a message or recursively defined process,respectively.]
For example, $\theta = \Set{a \mapsto \atmR{a} , b \mapsto \defp{b}}$ is an [...] assignment that maps $a$ to a right-directed message and $b$ to a process.
Like $\theta$, all [...] assignments must be injective, to keep distinct symbols from becoming identified in the choreography.

Next, we apply the [...] assignment to each of the string rewriting specification's axioms and simultaneously replace the empty string with the empty ordered context.\footnote{Strictly speaking, the monoid operations are also exchanged, but because both are indicated by juxtaposition, this happens silently.}
This results in a collection of ordered rewriting steps that the choreography must satisfy if it is to be a faithful reflection of the string rewriting specification.
Applying $\theta$ to the axioms of \cref{??} yields 
\begin{equation*}
  \atmR{a} \oc \defp{b} \reduces \defp{b}
  \quad\text{and}\quad
  \defp{b} \reduces \octxe
\end{equation*}
as rewritings required of the choreography.

Finally, we solve for the recursively defined propositions that appear in the required rewritings.
In this example, $\defp{b}$ must be consistent with $\atmR{a} \limp \up \dn \defp{b}$ if $\atmR{a} \oc \defp{b} \reduces \defp{b}$ is to be derivable;
$\defp{b}$ must also be consistent with $\up \one$ if $\defp{b} \reduces \octxe$ is to be derivable.
The least such solution is $\defp{b} \defd (\atmR{a} \limp \up \dn \defp{b}) \with \up \one$.
Indeed, under this definition, 
\begin{equation*}
  \atmR{a} \oc \defp{b} \reduces \defp{b}
  \quad\text{and}\quad
  \defp{b} \reduces \octxe
\end{equation*}
are both derivable.

\subsection{}

Not all [...] assignments yield choreographies.
For instance, suppose we had chosen $\theta' = \Set{a \mapsto \defp{a} , b \mapsto \atmR{b}}$ or any other assignment $\theta'$ that maps $b$ to an atom.
Applying $\theta'$ to the second axiom would yield either $\atmR{b} \reduces \octxe$ or $\atmL{b} \reduces \octxe$ as required rewriting steps.
Neither of these make for a valid choreography both of which require a message to be recognized and acted upon by the ether.


\subsection{}

To construct a choreography, we need to find a \emph{choreographing assignment} that consistently localizes each axiom.

An assignment that maps both $a$ and $b$ to messages, such as $\theta = \Set{a \mapsto \atmR{a} , b \mapsto \atmR{b}}$ which results in $\atmR{a} \oc \atmR{b} \reduces \atmR{b}$  and $\atmR{b} \reduces \octxe$,
 

  
As a string rewriting specification, the axioms are interpreted from a global perspective.
For instance, the first axiom states that when the symbols $a \oc b$ occur in that order, they may be rewritten to $b$.
But the axiom does not describe how that rewriting occurs.

With choreographies, we would like to work at a (slightly) lower level of abstraction to describe 


Suppose that we are given a string rewriting specification that consists of axioms $?$ over the rewriting alphabet $\sralph$.
A \vocab{choreographing assignment} is an injection in which each symbol $a \in \sralph$ is mapped to an ordered proposition: either an atomic proposition, $\atmL{a}$ or $\atmR{a}$, or a recursively defined proposition, $\defp{a}$.

Given a choregraphing assignment $\theta$, we may construct a choregraphy from the string rewriting specification.
Intuitively, each axiom is annotated according to $\theta$, and then the resulting [...] are used to construct a family of recursive definitions, one for each $\defp{a}$ in the image of $\theta$.

A choreography is an ordered rewriting specification that simulates the string rewriting specification [...].

Consider the recurring string rewriting specification with axioms
\begin{equation*}
  \infer{a \oc b \reduces b}{}
  \qquad\text{and}\qquad
  \infer{b \reduces \emp}{}
  \:.
\end{equation*}
We must consistently annotate each symbol as either a left-directed atom, right-directed atom, or recursively defined proposition in sucha way that each axiom's premise $w$ has the form $w_1 \wc a \wc w_2$ with 

\begin{equation*}
  \atmR{a} \oc \defp{b} \reduces \defp{b}
  \qquad\text{and}\qquad
  \defp{b} \reduces \octxe
\end{equation*}

Now we must solve for $\defp{b}$, determining a definition $\defp{b} \defd \n{B}$ such that these two rewriting steps are derivable.
For the first step to be derivable, $\defp{b}$ should have a definition that is consistent with $\atmR{a} \limp \up \dn \defp{b}$, for 
\begin{equation*}
  \atmR{a} \oc (\atmR{a} \limp \up \dn \defp{b}) \reduces \defp{b}
\end{equation*}

Consider the choreographing assignment $\theta$ that maps $a$ to the atom $\atmR{a}$ and $b$ to the recursively defined proposition $\defp{b}$.
Upon annotating the above string rewriting axioms according to $\theta$, we arrive at the [...]
\begin{equation*}
  \atmR{a} \oc \dprop{b} \reduces \dprop{b}
  \qquad\text{and}\qquad
  \dprop{b} \reduces \one
  \,.
\end{equation*}

\begin{equation*}
  \dprop{b} \defd \atmR{a} \limp \up \dn \dprop{b}
  \qquad\text{and}\qquad
  \dprop{b} \defd \up \one
  \,,
\end{equation*}
respectively.
Combining these into a single definition that allows a nondeterministic choice between the two, we have
\begin{equation*}
  \dprop{b} \defd (\atmR{a} \limp \up \dn \dprop{b}) \with \up \one
  \,,
\end{equation*}
or $\dprop{b} \defd (\atmR{a} \limp \up \dprop{b}) \with \one$ if the minimally necessary shifts are elided.

By construction, this choreography is adequate with respect to the specification, in the sense that it can simulate each of the specification's possible steps and vice versa.
\begin{itemize}
\item $w \reduces w'$ only if $\theta(w) \reduces \theta(w')$
  For example, just as the string rewriting specification admits $a \oc b \reduces b$, the ordered rewriting choreography admits
  \begin{equation*}
    \theta(a \oc b) = \atmR{a} \oc \defp{b} = \atmR{a} \oc \bigl((\atmR{a} \limp \up \dn \defp{b}) \with \one\bigr) \reduces \defp{b} = \theta(b)
    \,.
  \end{equation*}

\item $\theta(w) \reduces \octx'$ only if $w \reduces \theta^{-1}(\octx')$
  For example, just as the ordered rewriting choreography admits $\theta(b) = \defp{b} \reduces \octxe$, the string rewriting specification admits $b \reduces \emp = \theta^{-1}(\octxe)$.
\end{itemize}
% The choreography can simulate each of the specification's possible rewriting steps: for example, $\theta(a \oc b) = \atmR{a} \oc \dprop{b} \reduces \dprop{b} = \theta(b)$, just as $a \oc b \reduces b$.
% Conversely, each of the choreography's possible rewriting steps can be simulated by the specification: for example, $\theta^{-1}(\atmR{a} \oc \dprop{b}) = a \oc b \reduces b = \theta^{-1}(\dprop{b})$, just as $\atmR{a} \oc \dprop{b} \reduces \dprop{b}$.

\subsection{Formal description}

Given a [...] assignment $\theta$ and a string rewriting signature $\srsig$, the judgment $\chorsig{\theta}{\srsig}{\orsig}$ produces \iacl{OR} signature $\orsig$ that constitutes, together with $\theta$, a well-specified choreography for $\srsig$.
In other words, $\orsig$ is a solution to $\theta(\srsig)$, the rewritings induced from the string rewriting axioms $\srsig$ by $\theta$.
If $\chorsig{\theta}{\srsig}{\orsig}$ is not derivable for any $\orsig$, then the [...] assignment $\theta$ does not yield a well-specified choreography of $\srsig$.

Each axiom $w \reduces w'$ in the string rewriting signature $\srsig$ is processed in turn.
\begin{itemize}
\item First, we verify that the premise $w$ may be expressed as $w = w_1 \wc a \wc w_2$, where:
$\theta$ assigns a process role to $a$ (\ie, $\theta(a) = \defp{a}$ for some $\defp{a}$);
right-directed message roles to all symbols in $w_1$ (\ie, $\theta(w_1) = \atmR{\octx}_L$ for some $\atmR{\octx}_L$);
and left-directed messages roles to all symbols in $w_2$ (\ie, $\theta(w_2) = \atmL{\octx}_R$ for some $\atmL{\octx}_R$).

\item
  Next, we construct the quasi-proposition $\theta(w_1) \limp \up \bigfuse \theta(w') \pmir \theta(w_2)$.

\item
  Last, we inductively 
\end{itemize}

To define this choreographing judgment, we also use an auxiliary judgment that choreographs individual axioms.
Given a [...] assignment $\theta$, strings $w_1$ and $w_2$, and a positive proposition $\p{A}$, the judgment $\chorax{\theta}{w_1}{\p{A}}{w_2}{\n{B}}$ checks that $\theta(w_1)$ and $\theta(w_2)$ consist of only right- and left-directed atoms, respectively, and then produces a negative proposition $\n{B}$ that is morally $\theta(w_1) \limp \up \p{A} \pmir \theta(w_2)$ -- that is, a proposition $\n{B}$ such that $\lfocus{\theta(w_1)}{\n{B}}{\theta(w_2)}{\p{A}}$.

Each of the axioms is processed one-by-one.
From the axiom $w \reduces w'$, a symbol $a$ is nondeterministically selected from the premise $w$;
the selected symbol must be assigned a process role by $\theta$.

For each axiom $w \reduces w'$, we verfify that the premise $w$ may be expressed as $w = w_1 \wc a \wc w_2$, where $\theta$ assigns a process role to $a$, right-directed message roles to all symbols in $w_1$, and left-directed messages roles to all symbols in $w_2$.
Then, for each process $\defp{a}$, all axioms with premises $w_1 \wc a \wc w_2$
\begin{inferences}
  \infer{(\octxe) \limp \up \p{A} \pmir (\octxe) \rightsquigarrow \up \p{A}}{}
  \\
  \infer{(\atmR{\octx}_L \oc \atmR{a}) \limp \up \p{A} \pmir \atmL{\octx}_R \rightsquigarrow \atmR{a} \limp \n{B}}{
    \atmR{\octx}_L \limp \up \p{A} \pmir \atmL{\octx}_R \rightsquigarrow \n{B}}
  \and
  \infer{\atmR{\octx}_L \limp \up \p{A} \pmir (\atmL{a} \oc \atmL{\octx}_R) \rightsquigarrow \n{B} \pmir \atmL{a}}{
    \atmR{\octx}_L \limp \up \p{A} \pmir \atmL{\octx}_R \rightsquigarrow \n{B}}
\end{inferences}
When $\theta(b) = \atmR{b}$, the quasi-proposition $\theta(w_1 \wc b) \limp \up \p{A} \pmir \theta(w_2)$ is equivalent to $\atmR{b} \limp \bigl(\theta(w_1) \limp \up \p{A} \pmir \theta(w_2)\bigr)$.%
\footnote{That $b$ moves from the right of $w_1$ to the left of $\theta(w_1)$ as $\atmR{b}$ is somewhat counterintuitive, but the proof of \cref{??} explains.}
Likewise, when $\theta(b) = \atmL{b}$, the quasi-proposition $\theta(w_1) \limp \up \p{A} \pmir \theta(b \wc w_2)$ is equivalent to $\bigl(\theta(w_1) \limp \up \p{A} \pmir \theta(w_2)\bigr) \pmir \atmL{b}$.
 
\begin{inferences}
  \infer{(\octxe) \limp \up \p{A} \pmir (\octxe) \rightsquigarrow \up \p{A}}{}
  \\
  \infer{(\atmR{\octx}_L \oc \atmR{a}) \limp \up \p{A} \pmir \atmL{\octx}_R \rightsquigarrow \atmR{a} \limp \n{B}}{
    \atmR{\octx}_L \limp \up \p{A} \pmir \atmL{\octx}_R \rightsquigarrow \n{B}}
  \and
  \infer{\atmR{\octx}_L \limp \up \p{A} \pmir (\atmL{a} \oc \atmL{\octx}_R) \rightsquigarrow \n{B} \pmir \atmL{a}}{
    \atmR{\octx}_L \limp \up \p{A} \pmir \atmL{\octx}_R \rightsquigarrow \n{B}}
  \\
  \infer{\chorsig{\theta}{\srsig, (w_1 \wc a \wc w_2 \reduces w')}{\orsig, (\defp{a} \defd \n{A} \with \n{B})}}{
    \begin{array}[b]{@{}c@{}}
      \text{($\theta(w_1) = \atmR{\octx}_L$)} \quad
      \text{($\theta(a) = \defp{a}$)} \quad
      \text{($\theta(w_2) = \atmL{\octx}_R$)} \\
      \atmR{\octx}_L \limp \up \bigfuse \theta(w') \pmir \atmL{\octx}_R \rightsquigarrow \n{B} \quad
      \chorsig{\theta}{\srsig}{\orsig, (\defp{a} \defd \n{A})}
    \end{array}}
  \\
  \infer{\chorsig{\theta}{\srsig, (w_1 \wc a \wc w_2 \reduces w')}{\orsig, (\defp{a} \defd \n{B})}}{
    \begin{array}[b]{@{}c@{}}
      \text{($\theta(w_1) = \atmR{\octx}_L$)} \quad
      \text{($\theta(a) = \defp{a}$)} \quad
      \text{($\theta(w_2) = \atmL{\octx}_R$)} \\
      \atmR{\octx}_L \limp \up \bigfuse \theta(w') \pmir \atmL{\octx}_R \rightsquigarrow \n{B} \quad
      \chorsig{\theta}{\srsig}{\orsig} \quad
      \text{($\defp{a} \notin \dom{\orsig}$)}
    \end{array}}
\end{inferences}

Judgments $\chorsig{\theta}{\srsig}{\orsig}$ and $\chorax{\theta}{w_1}{\n{A}}{w_2}{\n{B}}$.
In both judgments, all terms before the $\chorarrow$ are inputs; all terms after the $\chorarrow$ are outputs.

\begin{inferences}
  \infer{\chorsig{\theta}{\srsig, (w_1 \wc a \wc w_2 \reduces w')}{\orsig, (\defp{a} \defd \n{A} \with \n{B})}}{
    \begin{array}[b]{@{}c@{}}
      \text{($\theta(w_1) = \atmR{\octx}_L$)} \quad
      \text{($\theta(a) = \defp{a}$)} \quad
      \text{($\theta(w_2) = \atmL{\octx}_R$)} \\
      \atmR{\octx}_L \limp \up \bigfuse \theta(w') \pmir \atmL{\octx}_R \rightsquigarrow \n{B} \quad
      \chorsig{\theta}{\srsig}{\orsig, (\defp{a} \defd \n{A})}
    \end{array}}
  \\
  \infer{\chorsig{\theta}{\srsig, (w_1 \wc a \wc w_2 \reduces w')}{\orsig, (\defp{a} \defd \n{B})}}{
    \begin{array}[b]{@{}c@{}}
      \text{($\theta(w_1) = \atmR{\octx}_L$)} \quad
      \text{($\theta(a) = \defp{a}$)} \quad
      \text{($\theta(w_2) = \atmL{\octx}_R$)} \\
      \atmR{\octx}_L \limp \up \bigfuse \theta(w') \pmir \atmL{\octx}_R \rightsquigarrow \n{B} \quad
      \chorsig{\theta}{\srsig}{\orsig} \quad
      \text{($\defp{a} \notin \dom{\orsig}$)}
    \end{array}}
\end{inferences}


% \begin{inferences}
%   \infer{\chorsig{\theta}{\sige}{\sige}}{}
%   \and
%   \infer{\chorsig{\theta}{\sig, w \reduces w'}{\sig', \proc{a} \defd \n{A}_1 \with \n{A}_2(\up \p{C})}}{
%     \chorsig{\theta}{\sig}{\sig'} &
%     \chorax{\theta}{w \reduces w'}{\proc{a}}{\n{A}_2(\Box)}{\p{C}} &
%     \text{($\sig'(\proc{a}) = \n{A}_1$)}}
%   \\
%   \infer{\chorsig{\theta}{\sig, w \reduces w'}{\sig', \proc{a} \defd \n{A}(\up \p{C})}}{
%     \chorsig{\theta}{\sig}{\sig'} &
%     \chorax{\thea}{w \reduces w'}{\proc{a}}{\n{A}(\Box)}{\p{C}} &
%     \text{($\proc{a} \notin \dom{\sig'}$)}}
%   \\
%   \infer{\chorax{\theta}{a \reduces w'}{\proc{a}}{\Box}{\bigfuse \octx'}}{
%     \text{($\theta(a) = \proc{a}$)} &
%     \text{($\theta(w') = \octx'$)}}
%   \and
%   \infer{\chorax{\theta}{b \oc w \reduces w'}{\proc{a}}{\n{A}(\atmR{b} \limp \Box)}{\p{C}}}{
%     \chorax{\theta}{w \reduces w'}{\proc{a}}{\n{A}(\Box)}{\p{C}} &
%     \text{($\theta(b) = \atmR{b}$)}}
%   \\
%   \infer{\chorax{\theta}{w \oc b \reduces w'}{\proc{a}}{\n{A}(\Box \pmir \atmL{b})}{C}}{
%     \chorax{\theta}{w \reduces w'}{\proc{a}}{\n{A}(\Box)}{\p{C}} &
%     \text{($\theta(b) = \atmL{b}$)}}
% \end{inferences}


\begin{lemma}\label{lem:chorax-sound-complete}
  % If $\chorax{\theta}{w_1}{\p{C}}{w_2}{\n{B}}$, then $\lfocus{\theta(w_1)}{\n{B}}{\theta(w_2)}{\p{C}}$.
  If $\atmR{\octx}_L \limp \up \p{A} \pmir \atmL{\octx}_R \rightsquigarrow \n{B}$, then $\lfocus{\atmR{\lctx}_L}{\n{B}}{\atmL{\lctx}_R}{\p{C}}$ if, and only if, $\atmR{\octx}_L = \atmR{\lctx}_L$ and $\atmL{\octx}_R = \atmL{\lctx}_R$ and $\p{A} = \p{C}$.
\end{lemma}
\begin{proof}
  By induction over the structure of the given derivation.

  As an example case, consider
  % \begin{itemize}
  % \item Consider the case in which
  %   \begin{equation*}
  %     \infer{\chorax{\theta}{\emp}{\p{A}}{\emp}{\up \p{A}}}{}
  %     \,.
  %   \end{equation*}
  %   We must show that $\lfocus{\atmR{\octx}_L}{\up \p{A}}{\atmL{\octx}_R}{\p{C}}$ if, and only if, $\atmR{\octx}_L = \atmL{\octx}_R = \theta(\emp)$ and $\p{A} = \p{C}$.
  %   Indeed, the $\lrule{\up}$ rule is the unique rule for left-focusing on $\up \p{A}$, and $\octxe = \theta(\emp)$ because $\theta$ is a monoid homomorphism.
  % 
  % \item Consider the case in which
    \begin{equation*}
      \infer{(\atmR{\octx}_L \oc \atmR{a}) \limp \up \p{A} \pmir \atmL{\octx}_R \rightsquigarrow \atmR{a} \limp \n{B}}{
        \atmR{\octx}_L \limp \up \p{A} \pmir \atmL{\octx}_R \rightsquigarrow \n{B}}
      \,.
    \end{equation*}
    We must show that $\lfocus{\atmR{\lctx}_L}{\atmR{a} \limp \n{B}}{\atmL{\lctx}_R}{\p{C}}$ if, and only if, $\atmR{\octx}_L \oc \atmR{a} = \atmR{\lctx}_L$ and $\atmL{\octx}_R = \atmL{\lctx}_R$ and $\p{A} = \p{C}$.
    Indeed, the $\lrule{\limp}$ rule is the unique rule for left-focusing on $\atmR{a} \limp \n{B}$, so $\lfocus{\atmR{\lctx}_L}{\atmR{a} \limp \n{B}}{\atmL{\lctx}_R}{\p{C}}$ if, and only if, $\atmR{\lctx}_L = \atmR{\lctx}'_L \oc \atmR{a}$ and $\lfocus{\atmR{\lctx}'_L}{\n{B}}{\atmL{\lctx}_R}{\p{C}}$ for some $\atmR{\lctx}'_L$.
    By the inductive hypothesis, $\lfocus{\atmR{\lctx}'_L}{\n{B}}{\atmL{\lctx}_R}{\p{C}}$ if, and only if, $\atmR{\octx}_L = \atmR{\lctx}'_L$ and $\atmL{\octx}_R = \atmL{\lctx}_R$ and $\p{A} = \p{C}$.
    Putting everything together, $\lfocus{\atmR{\lctx}_L}{\atmR{a} \limp \n{B}}{\atmL{\lctx}_R}{\p{C}}$ if, and only if, $\atmR{\octx}_L \oc \atmR{a} = \atmR{\lctx}_L$ and $\atmL{\octx}_R = \atmL{\lctx}_R$ and $\p{A} = \p{C}$, as required.
  % 
  % \item
  %   The case in which
  % \begin{equation*}
  %   \infer{\chorax{\theta}{w_1}{\p{A}}{b \oc w_2}{\n{B} \pmir \atmL{b}}}{
  %     \chorax{\theta}{w_1}{\p{A}}{w_2}{\n{B}} &
  %     \text{($\theta(b) = \atmL{b}$)}}
  % \end{equation*}
  %   is symmtric to the previous one.
  % %
  % \qedhere
  % \end{itemize}
\end{proof}


\begin{lemma}[Weakening]
  If $\octx \reduces_{\orsig} \octx'$ and $\dom{\orsig} \cap \dom{\orsig'} = \emptyset$, then $\octx \reduces_{\orsig, \orsig'} \octx'$.
  Similarly, if $\octx \reduces_{\orsig, (\defp{a} \defd \n{A})} \octx'$ or $\octx \reduces_{\orsig, (\defp{a} \defd \n{B
})} \octx'$, then $\octx \reduces_{\orsig, (\defp{a} \defd \n{A} \with \n{B})} \octx'$.
\end{lemma}
\begin{proof}
  By induction over the structure of the given rewriting step.
\end{proof}

% \begin{lemma}
%   If $(w \reduces w') \in \srsig$ and $\chorsig{\theta}{\srsig}{\orsig}$, then $\theta(w) \reduces_{\orsig} \theta(w')$.
% \end{lemma}
% \begin{proof}
%   By induction over the structure of the given choreographing derivation, $\chorsig{\theta}{\srsig}{\orsig}$.
%   \begin{itemize}
%   \item Consider the case in which $w = w_1 \oc a \oc w_2 \reduces w'$ is the axiom in question and
%     \begin{equation*}
%       \infer{\chorsig{\theta}{\srsig, w_1 \oc a \oc w_2 \reduces w'}{\orsig, \defp{a} \defd \n{A} \with \n{B}}}{
%         \text{($\theta(a) = \defp{a}$)} &
%         \chorax{\theta}{w_1}{\bigfuse \theta(w')}{w_2}{\n{B}} &
%         \chorsig{\theta}{\srsig}{\orsig, \defp{a} = \n{A}}}
%     \end{equation*}
%     It follows from \cref{??} that $\lfocus{\theta(w_1)}{\n{B}}{\theta(w_2)}{\bigfuse \theta(w')}$, and hence $\lfocus{\theta(w_1)}{\n{A} \with \n{B}}{\theta(w_2)}{\bigfuse \theta(w')}$.
%     And because $\rfocus{\theta(w')}{\bigfuse \theta(w')}$ and $\defp{a} \defd \n{A} \with \n{B}$, we have $\theta(w) = \theta(w_1) \oc \defp{a} \oc \theta(w_2) \reduces \theta(w')$.
  
%   \item Consider the case in which
%     \begin{equation*}
%       \infer{\chorsig{\theta}{\sig, v_1 \oc a \oc v_2 \reduces v'}{\sig', \hat{a} \defd \n{A} \with \n{B}}}{
%         \text{($\theta(a) = \hat{a}$)} &
%         \chorax{\theta}{v_1}{\bigfuse \theta(v')}{v_2}{\n{B}} &
%         \chorsig{\theta}{\sig}{\sig'} &
%         \text{($\sig'(\hat{a}) = \n{A}$)}}
%     \end{equation*}
%     and the axiom $w \reduces w'$ comes from $\sig$.
%     By the inductive hypothesis, $\theta(w) \reduces_{\sig'} \theta(w')$.
%     By \cref{??}, $\theta(w) \reduces_{\sig', \hat{a} \defd \n{A} \with \n{B}} \theta(w')$.
  
%   \item Consider the case in which
%     \begin{equation*}
%       \infer{\chorsig{\theta}{\sig, v_1 \oc a \oc v_2 \reduces v'}{\sig', \hat{a} \defd \n{B}}}{
%         \text{($\theta(a) = \hat{a}$)} &
%         \chorax{\theta}{v_1}{\bigfuse \theta(v')}{v_2}{\n{B}} &
%         \chorsig{\theta}{\sig}{\sig'} &
%         \text{($\hat{a} \notin \dom{\sig'}$)}}
%     \end{equation*}
%     and the axiom $w \reduces w'$ comes from $\sig$.
%     By the inductive hypothesis, $\theta(w) \reduces_{\sig'} \theta(w')$.
%     By \cref{??}, $\theta(w) \reduces_{\sig', \hat{a} \defd \n{B}} \theta(w')$.
%   %
%   \qedhere
%   \end{itemize}
% \end{proof}

\begin{theorem}[Completeness]\leavevmode
  If $\chorsig{\theta}{\srsig}{\orsig}$ and $w \reduces_{\srsig} w'$, then $\theta(w) \reduces_{\orsig} \theta(w')$.
\end{theorem}
\begin{proof}
  By simultaneous structural induction on the given choreographing derivation, $\chorsig{\theta}{\srsig}{\orsig}$, and ordered rewriting step, $w \reduces_{\srsig} w'$.
  \begin{itemize}
  \item
    Consider the case in which
    \begin{equation*}
      \chorsig{\theta}{\srsig}{\orsig}
      \qquad\text{and}\qquad
      w =
      \infer[\jrule{$\reduces$C}]{w_1 \wc w_0 \wc w_2 \reduces_{\srsig} w_1 \wc w'_0 \wc w_2}{
        w_0 \reduces_{\srsig} w'_0}
      = w'
      \,.
    \end{equation*}
    By the inductive hypothesis, $\theta(w_0) \reduces_{\orsig} \theta(w'_0)$.
    It follows from ordered rewriting's $\jrule{$\reduces$C}$ rule that
    \begin{equation*}
      \theta(w) = \theta(w_1) \oc \theta(w_0) \oc \theta(w_2) \reduces_{\orsig} \theta(w_1) \oc \theta(w'_0) \oc \theta(w_2) = \theta(w')
      \,.
    \end{equation*}

  \item
    Consider the case in which
    \begin{gather*}
      \infer{\chorsig{\theta}{\srsig_0, (w_1 \wc a \wc w_2 \reduces w')}{\orsig_0, (\defp{a} \defd \n{A} \with \n{B})}}{
        \begin{array}[b]{@{}c@{}}
          \text{($\theta(w_1) = \atmR{\octx}_L$)} \quad
          \text{($\theta(a) = \defp{a}$)} \quad
          \text{($\theta(w_2) = \atmL{\octx}_R$)} \\
          \atmR{\octx}_L \limp \up \bigfuse \theta(w') \pmir \atmL{\octx}_R \rightsquigarrow \n{B} \quad
          \chorsig{\theta}{\srsig_0}{\orsig_0, (\defp{a} \defd \n{A})}
        \end{array}}
    \shortintertext{and}
      w =
      \infer[\jrule{$\reduces$AX}]{w_1 \wc a \wc w_2 \reduces_{\srsig} w'}{}
    \end{gather*}
    where $\srsig = \srsig_0 , (w_1 \wc a \wc w_2 \reduces w')$ and $\orsig = \orsig_0, (\defp{a} \defd \n{A} \with \n{B})$.

    By \cref{lem:chorax-sound-complete}, $\lfocus{\theta(w_1)}{\n{B}}{\theta(w_2)}{\bigfuse \theta(w')}$.
    Upon adding the $\lrule{\with}_2$ rule, $\lfocus{\theta(w_1)}{\n{A} \with \n{B}}{\theta(w_2)}{\bigfuse \theta(w')}$.
    Because $\rfocus{\theta(w')}{\bigfuse \theta(w')}$~\parencref{??}, it follows by the $\jrule{$\reduces$I}$ rule that $\theta(w_1) \oc (\n{A} \with \n{B}) \oc \theta(w_2) \reduces_{\orsig} \theta(w')$, and so $\theta(w) = \theta(w_1) \oc \defp{a} \oc \theta(w_2) = \theta(w_1) \oc (\n{A} \with \n{B}) \oc \theta(w_2) \reduces_{\orsig} \theta(w')$


  \item
    Consider the case in which
    \begin{gather*}
      \infer{\chorsig{\theta}{\srsig_0, (w_1 \wc a \wc w_2 \reduces w')}{\orsig_0, (\defp{a} \defd \n{B})}}{
        \begin{array}[b]{@{}c@{}}
          \text{($\theta(w_1) = \atmR{\octx}_L$)} \quad
          \text{($\theta(a) = \defp{a}$)} \quad
          \text{($\theta(w_2) = \atmL{\octx}_R$)} \\
          \atmR{\octx}_L \limp \up \bigfuse \theta(w') \pmir \atmL{\octx}_R \rightsquigarrow \n{B} \quad
          \chorsig{\theta}{\srsig_0}{\orsig_0} \quad
          \text{($\defp{a} \notin \dom{\orsig_0}$)}
        \end{array}}
    \shortintertext{and}
      w =
      \infer[\jrule{$\reduces$AX}]{w_1 \wc a \wc w_2 \reduces_{\srsig} w'}{}
    \end{gather*}
    where $\srsig = \srsig_0 , (w_1 \wc a \wc w_2 \reduces w')$ and $\orsig = \orsig_0 , (\defp{a} \defd \n{B})$.

    By \cref{lem:chorax-sound-complete}, $\lfocus{\theta(w_1)}{\n{B}}{\theta(w_2)}{\bigfuse \theta(w')}$.
    Because $\rfocus{\theta(w')}{\bigfuse \theta(w')}$, it follows by the $\jrule{$\reduces$I}$ rule that $\theta(w_1) \oc \n{B} \oc \theta(w_2) \reduces_{\orsig} \theta(w')$, and so $\theta(w) = \theta(w_1) \oc \defp{a} \oc \theta(w_2) = \theta(w_1) \oc \n{B} \oc \theta(w_2) \reduces_{\orsig} \theta(w')$.
    
  \item
    Consider the case in which
    \begin{gather*}
      \infer{\chorsig{\theta}{\srsig_0, (v_1 \wc b \wc v_2 \reduces v')}{\orsig_0, (\defp{b} \defd \n{A} \with \n{B})}}{
        \begin{array}[b]{@{}c@{}}
          \text{($\theta(v_1) = \atmR{\octx}_L$)} \quad
          \text{($\theta(b) = \defp{b}$)} \quad
          \text{($\theta(v_2) = \atmL{\octx}_R$)} \\
          \atmR{\octx}_L \limp \up \bigfuse \theta(v') \pmir \atmL{\octx}_R \rightsquigarrow \n{B} \quad
          \chorsig{\theta}{\srsig_0}{\orsig_0, (\defp{b} \defd \n{A})}
        \end{array}}
    \shortintertext{and}
      \infer[\jrule{$\reduces$AX}]{w \reduces_{\srsig} w'}{
        w \reduces w' \in \srsig_0}
    \end{gather*}
    where $\srsig = \srsig_0, (v_1 \wc b \wc v_2 \reduces v')$ and $\orsig = \orsig_0 , (\defp{b} \defd \n{A} \with \n{B})$.

    By the inductive hypothesis, $\theta(w) \reduces_{\orsig_0, (\defp{b} \defd \n{A})} \theta(w')$.
    It follows from the weakening \lcnamecref{??}~\parencref{??} that $\theta(w) \reduces_{\orsig} \theta(w')$, as required.

  \item
    The case in which
    \begin{gather*}
      \infer{\chorsig{\theta}{\srsig_0, (v_1 \wc b \wc v_2 \reduces v')}{\orsig_0, (\defp{b} \defd \n{B})}}{
        \begin{array}[b]{@{}c@{}}
          \text{($\theta(v_1) = \atmR{\octx}_L$)} \quad
          \text{($\theta(b) = \defp{b}$)} \quad
          \text{($\theta(v_2) = \atmL{\octx}_R$)} \\
          \atmR{\octx}_L \limp \up \bigfuse \theta(v') \pmir \atmL{\octx}_R \rightsquigarrow \n{B} \quad
          \chorsig{\theta}{\srsig_0}{\orsig_0} \quad
          \text{($\defp{b} \notin \dom{\orsig_0}$)}
        \end{array}}
    \shortintertext{and}
      \infer[\jrule{$\reduces$AX}]{w \reduces_{\srsig} w'}{
        w \reduces w' \in \srsig_0}
    \end{gather*}
    where $\srsig = \srsig_0, (v_1 \wc b \wc v_2 \reduces v')$ and $\orsig = \orsig_0 , (\defp{b} \defd \n{B})$ is similar to the previous one.
    %
  \qedhere
  \end{itemize}
\end{proof}




% \begin{lemma}
%   If $\chorax{\theta}{w_1}{\p{A}}{w_2}{\n{B}}$ and $\lfocus{\atmR{\octx}_L}{\n{B}}{\atmL{\octx}_R}{\p{C}}$, then $\atmR{\octx}_L = \theta(w_1)$ and $\atmL{\octx}_R = \theta(w_2)$ and $\p{A} = \p{C}$.
% \end{lemma}
% \begin{proof}
%   By induction over the structure of the given choreographing derivation, $\chorax{\theta}{w_1}{\p{A}}{w_2}{\n{B}}$.
%   \begin{itemize}
%   \item
%     Consider the case in which
%   \begin{equation*}
%     \infer{\chorax{\theta}{\emp}{\p{A}}{\emp}{\up \p{A}}}{}
%     \qquad\text{and}\qquad
%     \lfocus{\atmR{\octx}_1}{\up \p{A}}{\atmL{\octx}_2}{\p{C}}
%     \,.
%   \end{equation*}
%   By inversion on the left-focus derivation, $\atmR{\octx}_L = \octxe = \theta(\emp)$ and $\atmL{\octx}_R = \octxe = \theta(\emp)$, as well as $\p{A} = \p{C}$, as required.

%   \item
%     Consider the case in which
%   \begin{equation*}
%     \infer{\chorax{\theta}{w_1 \oc b}{\p{A}}{w_2}{\atmR{b} \limp \n{B}}}{
%       \text{($\theta(b) = \atmR{b}$)} &
%       \chorax{\theta}{w_1}{\p{A}}{w_2}{\n{B}}}
%     \qquad\text{and}\qquad
%     \lfocus{\atmR{\octx}_L}{\atmR{b} \limp \n{B}}{\atmL{\octx}_R}{\p{C}}
%     \,.
%   \end{equation*}
%   By inversion on the left-focus derivation for $\atmR{b} \limp \n{B}$, there exists $\atmR{\octx}'_1$ such that $\atmR{\octx}_L = \atmR{\octx}'_L \oc \atmR{b}$ and $\lfocus{\atmR{\octx}'_L}{\n{B}}{\atmL{\octx}_R}{\p{C}}$.
%   It follows from the inductive hypothesis that $\atmR{\octx}'_L = \theta(w_1)$ and $\atmL{\octx}_R = \theta(w_2)$ and $\p{A} = \p{C}$.
%   So $\atmR{\octx}_L = \theta(w_1) \oc \atmR{b} = \theta(w_1 \oc b)$.

%   \item
%     The case in which
% \begin{equation*}
%     \infer{\chorax{\theta}{w_1 \oc b}{\p{A}}{w_2}{\n{B} \pmir \atmL{b}}}{
%       \text{($\theta(b) = \atmL{b}$)} &
%       \chorax{\theta}{w_1}{\p{A}}{w_2}{\n{B}}}
%     \qquad\text{and}\qquad
%     \lfocus{\atmR{\octx}_L}{\n{B} \pmir \atmL{b}}{\atmL{\octx}_R}{\p{C}}
%   \end{equation*}
%   is symmetric.
%   \qedhere
%   \end{itemize}
% \end{proof}


\begin{lemma}
  If $\chorsig{\theta}{\srsig}{\orsig}$ and $\lfocus{\atmR{\octx}_L}{\defp{a}}{\atmL{\octx}_R}{_{\orsig} \p{C}}$, then there exists an axiom $(w_1 \oc a \oc w_2 \reduces w') \in \srsig$ such that $\atmR{\octx}_L = \theta(w_1)$, $\atmL{\octx}_R = \theta(w_2)$, and $\p{C} = \bigfuse \theta(w')$.
\end{lemma}
\begin{proof}
  By induction over the structure of the given choreographing derivation, $\chorsig{\theta}{\srsig}{\orsig}$.
  \begin{itemize}
  \item
    Consider the case in which
    \begin{gather*}
      \infer{\chorsig{\theta}{\srsig_0, (v_1 \wc b \wc v_2 \reduces v')}{\orsig_0, (\defp{b} \defd \n{A} \with \n{B})}}{
        \begin{array}[b]{@{}c@{}}
          \text{($\theta(v_1) = \atmR{\lctx}_L$)} \quad
          \text{($\theta(b) = \defp{b}$)} \quad
          \text{($\theta(v_2) = \atmL{\lctx}_R$)} \\
          \atmR{\lctx}_L \limp \up \bigfuse \theta(v') \pmir \atmL{\lctx}_R \rightsquigarrow \n{B} \quad
          \chorsig{\theta}{\srsig_0}{\orsig_0, (\defp{b} \defd \n{A})}
        \end{array}}
    \shortintertext{and}
      \lfocus{\atmR{\octx}_L}{\defp{a}}{\atmL{\octx}_R}{_{\orsig} \p{C}}
    \end{gather*}
    where $a \neq b$ and $\srsig = \srsig_0, (v_1 \wc b \wc v_2 \reduces v')$ and $\orsig = \orsig_0, (\defp{b} \defd \n{A} \with \n{B})$.

    By the inductive hypothesis, there exists a string rewriting axiom $(w_1 \wc a \wc w_2 \reduces w') \in \srsig_0$ such that $\atmR{\octx}_L = \theta(w_1)$, $\atmL{\octx}_R = \theta(w_2)$, and $\p{C} = \bigfuse \theta(w')$.
  The same axiom is contained in the signature $\srsig$.

  \item
    Consider the case in which
    \begin{gather*}
      \infer{\chorsig{\theta}{\srsig_0, (v_1 \wc b \wc v_2 \reduces v')}{\orsig_0, (\defp{b} \defd \n{B})}}{
        \begin{array}[b]{@{}c@{}}
          \text{($\theta(v_1) = \atmR{\lctx}_L$)} \quad
          \text{($\theta(b) = \defp{b}$)} \quad
          \text{($\theta(v_2) = \atmL{\lctx}_R$)} \\
          \atmR{\lctx}_L \limp \up \bigfuse \theta(v') \pmir \atmL{\lctx}_R \rightsquigarrow \n{B} \quad
          \chorsig{\theta}{\srsig_0}{\orsig_0} \quad
          \text{($\defp{b} \notin \dom{\orsig_0}$)}
        \end{array}}
    \shortintertext{and}
      \lfocus{\atmR{\octx}_L}{\defp{a}}{\atmL{\octx}_R}{_{\orsig} \p{C}}
    \end{gather*}
    where $a \neq b$ and $\srsig = \srsig_0, (v_1 \wc b \wc v_2 \reduces v')$ and $\orsig = \orsig_0, (\defp{b} \defd \n{B})$.

    By the inductive hypothesis, there exists a string rewriting axiom $(w_1 \wc a \wc w_2 \reduces w') \in \srsig_0$ such that $\atmR{\octx}_L = \theta(w_1)$, $\atmL{\octx}_R = \theta(w_2)$, and $\p{C} = \bigfuse \theta(w')$.
  The same axiom is contained in the signature $\srsig$.

  \item
    Consider the case in which
    \begin{gather*}
      \infer{\chorsig{\theta}{\srsig_0, (w_1 \wc a \wc w_2 \reduces w')}{\orsig_0, (\defp{a} \defd \n{A} \with \n{B})}}{
        \begin{array}[b]{@{}c@{}}
          \text{($\theta(w_1) = \atmR{\lctx}_L$)} \quad
          \text{($\theta(a) = \defp{a}$)} \quad
          \text{($\theta(w_2) = \atmL{\lctx}_R$)} \\
          \atmR{\lctx}_L \limp \up \bigfuse \theta(w') \pmir \atmL{\lctx}_R \rightsquigarrow \n{B} \quad
          \chorsig{\theta}{\srsig_0}{\orsig_0, (\defp{a} \defd \n{A})}
        \end{array}}
    \shortintertext{and}
      \infer[\lrule{\with}_2]{\lfocus{\atmR{\octx}_L}{\defp{a} = \n{A} \with \n{B}}{\atmL{\octx}_R}{_{\orsig} \p{C}}}{
        \lfocus{\atmR{\octx}_L}{\n{B}}{\atmL{\octx}_R}{_{\orsig} \p{C}}}
    \end{gather*}
    where $\srsig = \srsig_0, (w_1 \wc a \wc w_2 \reduces w')$ and $\orsig = \orsig_0, (\defp{a} \defd \n{A} \with \n{B})$.

    By \cref{??}, $\atmR{\octx}_L = \atmR{\lctx}_L = \theta(w_1)$ and $\atmL{\octx}_R = \atmL{\lctx}_R = \theta(w_2)$ and $\p{C} = \bigfuse \theta(w')$.
    And the axiom $w_1 \wc a \wc w_2 \reduces w'$ is contained in the signature $\orsig$.

  \item
    Consider the case in which
    \begin{gather*}
      \infer{\chorsig{\theta}{\srsig_0, (v_1 \wc a \wc v_2 \reduces v')}{\orsig_0, (\defp{a} \defd \n{A} \with \n{B})}}{
        \begin{array}[b]{@{}c@{}}
          \text{($\theta(v_1) = \atmR{\lctx}_L$)} \quad
          \text{($\theta(a) = \defp{a}$)} \quad
          \text{($\theta(v_2) = \atmL{\lctx}_R$)} \\
          \atmR{\lctx}_L \limp \up \bigfuse \theta(v') \pmir \atmL{\lctx}_R \rightsquigarrow \n{B} \quad
          \chorsig{\theta}{\srsig_0}{\orsig_0, (\defp{a} \defd \n{A})}
        \end{array}}
    \shortintertext{and}
      \infer[\lrule{\with}_1]{\lfocus{\atmR{\octx}_L}{\defp{a} = \n{A} \with \n{B}}{\atmL{\octx}_R}{_{\orsig} \p{C}}}{
        \lfocus{\atmR{\octx}_L}{\n{A}}{\atmL{\octx}_R}{_{\orsig} \p{C}}}
    \end{gather*}
    where $\srsig = \srsig_0, (v_1 \wc a \wc v_2 \reduces v')$ and $\orsig = \orsig_0, (\defp{a} \defd \n{A} \with \n{B})$.

    Let $\orsig' = \orsig_0 , (\defp{a} \defd \n{A})$.
    Then $\lfocus{\atmR{\octx}_L}{\defp{a} = \n{A}}{\atmL{\octx}_R}{_{\orsig'} \p{C}}$.
    By inductive hypothesis, there exists an axiom $(w_1 \wc a \wc w_2 \reduces w') \in \srsig_0$ such that $\atmR{\octx}_L = \theta(w_1)$ and $\atmL{\octx}_R = \theta(w_2)$ and $\p{C} = \bigfuse \theta(w')$.
    That same axiom is also contained in the signature $\srsig$.

  \item
    Consider the case in which
    \begin{gather*}
      \infer{\chorsig{\theta}{\srsig_0, (w_1 \wc a \wc w_2 \reduces w')}{\orsig_0, (\defp{a} \defd \n{B})}}{
        \begin{array}[b]{@{}c@{}}
          \text{($\theta(w_1) = \atmR{\lctx}_L$)} \quad
          \text{($\theta(a) = \defp{a}$)} \quad
          \text{($\theta(w_2) = \atmL{\lctx}_R$)} \\
          \atmR{\lctx}_L \limp \up \bigfuse \theta(w') \pmir \atmL{\lctx}_R \rightsquigarrow \n{B} \quad
          \chorsig{\theta}{\srsig_0}{\orsig_0} \quad
          \text{($\defp{a} \notin \dom{\orsig_0}$)}
        \end{array}}
    \shortintertext{and}
      \lfocus{\atmR{\octx}_L}{\defp{a} = \n{B}}{\atmL{\octx}_R}{_{\orsig} \p{C}}
    \end{gather*}
    where $\srsig = \srsig_0, (w_1 \wc a \wc w_2 \reduces w')$ and $\orsig = \orsig_0, (\defp{a} \defd \n{B})$.

    By \cref{??}, $\atmR{\octx}_L = \atmR{\lctx}_L = \theta(w_1)$ and $\atmL{\octx}_R = \atmL{\lctx}_R = \theta(w_2)$ and $\p{C} = \bigfuse \theta(w')$.
    And the axiom $w_1 \wc a \wc w_2 \reduces w'$ is contained in the signature $\orsig$.


  % \item
  %   The case in which
  % \begin{gather*}
  %   \infer{\chorsig{\theta}{\srsig, v_1 \oc b \oc v_2 \reduces v'}{\orsig, \defp{a} \defd \n{A}, \defp{b} \defd \n{B}}}{
  %     \text{($\theta(b) = \defp{b}$)} &
  %     \chorax{\theta}{v_1}{\bigfuse \theta(v')}{v_2}{\n{B}} &
  %     \chorsig{\theta}{\srsig}{\orsig, \defp{a} \defd \n{A}} &
  %     \text{($\defp{b} \notin \dom{\orsig}$)}}
  %   \\\text{and}\\
  %   \lfocus{\atmR{\octx}_L}{\defp{a}}{\atmL{\octx}_R}{\p{C}}
  % \end{gather*}
  % is similar.

  % \item
  % Consider the case in which
  % \begin{gather*}
  %   \infer{\chorsig{\theta}{\srsig, v_1 \oc a \oc v_2 \reduces v'}{\orsig, \defp{a} \defd \n{A}_1 \with \n{A}_2}}{
  %     \text{($\theta(a) = \defp{a}$)} &
  %     \chorax{\theta}{v_1}{\bigfuse \theta(v')}{v_2}{\n{A}_2} &
  %     \chorsig{\theta}{\srsig}{\orsig, \defp{a} \defd \n{A}_1}}
  %   \\\text{and}\\
  %   \lfocus{\atmR{\octx}_L}{\defp{a}}{\atmL{\octx}_R}{\p{C}}
  %   \,.
  % \end{gather*}
  % There are two cases, according to whether the $\lfocus{\atmR{\octx}_L}{\defp{a}}{\atmL{\octx}_R}{\p{C}}$ derivation ends with the $\lrule{\with}_1$ or $\lrule{\with}_2$ rule.
  %   \begin{itemize}
  %   \item If the left-focus derivation ends with the $\lrule{\with}_2$ rule, then $\lfocus{\atmR{\octx}_L}{\n{A}_2}{\atmL{\octx}_R}{\p{C}}$.
  %     Because $\chorax{\theta}{v_1}{\bigfuse \theta(v')}{v_2}{\n{A}_2}$, it follows from \cref{??} that $\atmR{\octx}_L = \theta(v_1)$ and $\atmL{\octx}_R = \theta(v_2)$ and $\p{C} = \bigfuse \theta(v')$.
  %     Choose the axiom $w_1 \oc a \oc w_2 \reduces w'$ to be $v_1 \oc a \oc v_2 \reduces v'$.

  %   \item Otherwise, if the left-focus derivation instead ends with the $\lrule{\with}_1$ rule, then $\lfocus{\atmR{\octx}_L}{\n{A}_1}{\atmL{\octx}_R}{\p{C}}$.
  %     By the inductive hypothesis, $\atmR{\octx}_L = \theta(w_1)$, $\atmL{\octx}_R = \theta(w_2)$, and $\p{C} = \bigfuse \theta(w')$ for some string rewriting axiom $(w_1 \oc a \oc w_2 \reduces w') \in \srsig$.
  %     The same axiom is contained in the signuare $\srsig, v_1 \oc a \oc v_2 \reduces v'$.
  %   \end{itemize}

  % \item
  % Consider the case in which 
  % \begin{equation*}
  %   \infer{\chorsig{\theta}{\srsig, w_1 \oc a \oc w_2 \reduces w'}{\orsig, \defp{a} \defd \n{A}}}{
  %     \text{($\theta(a) = \defp{a}$)} &
  %     \chorax{\theta}{w_1}{\bigfuse \theta(w')}{w_2}{\n{A}} &
  %     \chorsig{\theta}{\srsig}{\orsig} &
  %     \text{($\defp{a} \notin \dom{\orsig}$)}}
  % \end{equation*}
  % Because $\chorax{\theta}{w_1}{\bigfuse \theta(w')}{w_2}{\n{A}_2}$, it follows from \cref{??} that $\atmR{\octx}_L = \theta(w_1)$ and $\atmL{\octx}_R = \theta(w_2)$ and $\p{C} = \bigfuse \theta(w')$.
  %
  \qedhere
  \end{itemize}
\end{proof}

\begin{theorem}[Soundness]
  If $\chorsig{\theta}{\srsig}{\orsig}$ and $\theta(a) = \defp{a}$ and $\octx_L \oc \defp{a} \oc \octx_R \reduces_{\orsig} \octx'$, then either:
  \begin{itemize}
  \item $\octx_L = \octx'_L \oc \theta(w_1)$ and $\octx_R = \theta(w_2) \oc \octx'_R$ and $\octx' = \octx'_L \oc \theta(w') \oc \octx'_R$ for some contexts $\octx'_L$ and $\octx'_R$ and some strings $w_1$, $w_2$, and $w'$ such that $(w_1 \wc a \wc w_2 \reduces w') \in \srsig$ and $\lfocus{\theta(w_1)}{\defp{a}}{\theta(w_2)}{\bigfuse \theta(w')}$;
  \item $\octx_L \reduces_{\orsig} \octx'_L$ for some context $\octx'_L$ such that $\octx' = \octx'_L \oc \defp{a} \oc \octx_R$; or
  \item $\octx_R \reduces_{\orsig} \octx'_R$ for some context $\octx'_R$ such that $\octx' = \octx_L \oc \defp{a} \oc \octx'_R$.
  \end{itemize}
\end{theorem}
\begin{proof}
  As a negative proposition, $\defp{a}$ serves as a barrier for interactions between $\octx_L$ and $\octx_R$ -- in \ac{PFOR}, implications cannot consume negative propositions.
  Thus, any reduction on $\octx_L \oc \defp{a} \oc \octx_R$ must occur within either $\octx_L$ or $\octx_R$ alone or must arise from $\defp{a}$.

  If the reduction on $\octx_L \oc \defp{a} \oc \octx_R$ arises from $\defp{a}$, then it arises from a bipole that begins by focusing on $\defp{a}$.
  In other words, $\octx_L = \octx'_L \oc \atmR{\lctx}_L$ and $\octx_R = \atmL{\lctx}_R \oc \octx'_R$ and $\octx' = \octx'_L \oc \lctx' \oc \octx'_R$ for some contexts $\atmR{\lctx}_L$, $\atmL{\lctx}_R$, and $\lctx'$ and positive proposition $\p{C}$ such that $\lfocus{\atmR{\lctx}_L}{\defp{a}}{\atmL{\lctx}_R}{\p{C}}$ and $\rfocus{\lctx'}{\p{C}}$.
  By \cref{??}, there exists an axiom $(w_1 \wc a \wc w_2 \reduces w') \in \srsig$ such that $\atmR{\lctx}_L = \theta(w_1)$ and $\atmL{\lctx}_R = \theta(w_2)$ and $\p{C} = \bigfuse \theta(w')$.
  It follows that $\lctx' = \theta(w')$.
\end{proof}

\begin{corollary}[Soundness]
  If $\chorsig{\theta}{\srsig}{\orsig}$ and $\theta(w) \reduces_{\orsig} \octx'$, then $\octx' = \theta(w')$ for some $w'$ such that $w \reduces_{\srsig} w'$.
\end{corollary}

% \begin{theorem}[Soundness]
%   If $\chorsig{\theta}{\srsig}{\orsig}$ and $\theta(w) \reduces_{\orsig} \octx'$, then $\octx' = \theta(w')$ for some $w'$ such that $w \reduces_{\srsig} w'$.
% \end{theorem}
% \begin{proof}
%   By induction over the structure of the given ordered rewriting step, $\theta(w) \reduces_{\orsig} \octx'$.
%   \begin{itemize}
%   \item Consider the case in which
%     \begin{equation*}
%       \chorsig{\theta}{\srsig}{\orsig}
%       \qquad\text{and}\qquad
%       \theta(w) = 
%       \infer[\jrule{$\reduces$C}\smash{_{\jrule{L}}}]{\octx_1 \oc \octx_2 \reduces_{\orsig} \octx'_1 \oc \octx_2}{
%         \octx_1 \reduces_{\orsig} \octx'_1}
%       = \octx'
%       \,.
%     \end{equation*}
%     By inversion, $w = w_1 \oc w_2$ for some $w_1$ and $w_2$ such that $\octx_1 = \theta(w_1)$ and $\octx_2 = \theta(w_2)$.
%     From the inductive hypothesis, it follows that there exists a string $w'_1$ such that $w_1 \reduces_{\srsig} w'_1$ and $\octx'_1 = \theta(w'_1)$.
%     Let $w' = w'_1 \oc w_2$, and notice that $w = w_1 \oc w_2 \reduces_{\srsig} w'_1 \oc w_2 = w'$ and $\octx' = \theta(w'_1) \oc \theta(w_2) = \theta(w')$, as required.

%   \item
%     The case in which
%     \begin{equation*}
%       \chorsig{\theta}{\srsig}{\orsig}
%       \qquad\text{and}\qquad
%       \theta(w) = 
%       \infer[\jrule{$\reduces$C}\smash{_{\jrule{R}}}]{\octx_1 \oc \octx_2 \reduces_{\orsig} \octx_1 \oc \octx'_2}{
%         \octx_2 \reduces_{\orsig} \octx'_2}
%       = \octx'
%     \end{equation*}
%     is symmetric.

%   \item
%     Consider the case in which 
%     \begin{equation*}
%       \chorsig{\theta}{\srsig}{\orsig}
%       \qquad\text{and}\qquad
%       \theta(w) =
%       \infer{\atmR{\octx}_1 \oc \n{A} \oc \atmL{\octx}_2 \reduces_{\orsig} \octx'}{
%         \lfocus{\atmR{\octx}_1}{\n{A}}{\atmL{\octx}_2}{\p{C}} &
%         \rfocus{\octx'}{\p{C}}}
%       \,.
%     \end{equation*}
%     The image $\theta(w)$ can contain a negative proposition $\n{A}$ only if $w = w_1 \oc a \oc w_2$ for some $w_1$, $a$, and $w_2$ such that $\theta(a) = \defp{a}$ with $(\defp{a} \defd \n{A}) \in \orsig$.
%     By inversion, both $\theta(w_1) = \atmR{\octx}_1$ and $\theta(w_2) = \atmL{\octx}_2$ must hold.
%     It then follows from \cref{??} that $\p{C} = \bigfuse \theta(w')$ for some string $w'$ such that $(w_1 \oc a \oc w_2 \reduces w') \in \srsig$.
%     Because $\rfocus{\octx'}{\bigfuse \theta(w')}$ only if $\octx' = \theta(w')$~\parencref{??}, the string $w'$ is such that $w = w_1 \oc a \oc w_2 \reduces_{\srsig} w'$, with $\octx' = \theta(w')$.
%   %
%   \qedhere
%   \end{itemize}
% \end{proof}



\subsection{No choreography}

Not all string rewriting specifications admit a choreography.
For example, the specification
\begin{equation*}
  \infer{a \oc b \reduces b}{}
  \qquad
  \infer{a \reduces \emp}{}
  \qquad\text{and}\qquad
  \infer{b \reduces \emp}{}
\end{equation*}
cannot be given a choreography.
More precisely, there is no choreographing assignment $\theta$ such that $\chorsig{\theta}{\sig}{\sig'}$ is derivable for some signature $\sig'$.
For the sake of contradiction, suppose that $\theta$ were such a choreographing assignment.
Then, for the specification's latter two axioms to be choreographable, both $\theta(a) = \proc{a}$ and $\theta(b) = \proc{b}$ must hold.
In that case, however, the specification's first axiom cannot be choreographed properly because $\theta$ maps more than one of the axiom's symbols to recursively defined propositions.


\section{Encoding \aclp*{NFA}}

Recall from \cref{??} our string rewriting specification of how \iac{NFA} processes its input.
Given \iac{NFA} $\aut{A} = (Q, ?, F)$\fixnote{fix} over an input alphabet $\ialph$, the \ac{NFA}'s operational semantics is adequately captured by the following string rewriting axioms:
\begin{equation*}
  \infer{a \oc q \reduces q'_a}{}
  \enspace\text{for each transition $q \nfareduces[a] q'_a$.}
\end{equation*}
\begin{equation*}
  \infer{\emp \oc q \reduces F(q)}{}
  \enspace\text{for each state $q$, where}\enspace
  F(q) = \begin{cases*}
           (\octxe) & if $q \in F$ \\
           \symrej & if $q \notin F$\,.
         \end{cases*}
\end{equation*}
We would like to choreograph this string rewriting specification.
There are, in fact, two distinct choreographies for this specification.

\subsection{A functional choreography}

One possible choreography for this specification interprets each input symbol $a \in \ialph$ as a right-directed atom, $\atmR{a}$; each state $q \in Q$ as a recursively defined proposition, $\defp{q}$; and the end-of-word marker, $\emp$, as a right-directed atom, $\atmR{\emp}$.
In other words, the input string is transmitted as a sequence of messages to a process $\defp{q}$ that tracks the \ac{NFA}'s current state.

% In other words, the \ac{NFA}'s input is treated as a sequence of messages, $\atmR{\emp} \oc \atmR{a}_n \dotsm \atmR{a}_2 \oc \atmR{a}_1$, and the \ac{NFA}'s states are treated as [recursive] processes.

Under this choreographing assignment, the string rewriting axioms become rewriting steps that must be derivable:
\begin{equation*}
  \atmR{a} \oc \defp{q}
    \reduces \defp{q}'_a
  %
  \atmR{\emp} \oc \defp{q}
    \reduces \begin{cases*}
               & if $q \in F$ \\
               & if $q \notin F$
             \end{cases*}
\end{equation*}
These required rewritings are indeed local: each one contains exactly one recursively defined process in its premise, with all remaining propositions in its premise being input messages for that process.
For instance, each $\atmR{a} \oc \defp{q} \reduces \defp{q}'_a$

Solving for each recursively defined proposition, we have one definition, 
\begin{equation*}
  \defp{q} \defd \bigwith_{a \in \ialph} \bigwith_{q\smash{'_a}} (\atmR{a} \limp \defp{q}'_a) \with (\atmR{\emp} \limp \nfa{F}(q))
  \,,
\end{equation*}
for each \ac{NFA} state $q \in Q$.
This choreography might be called \enquote*{functional} because the data, an input string, are represented by messages that are acted on in a function-like way by the current state's process, $\defp{q}$.

\begin{theorem}\label{thm:choreographies:nfa-functional-chorsig}
  $\chorsig{\theta}{\srsig}{\orsig}$ is derivable.
\end{theorem}
%
\noindent
The proof of this \lcnamecref{thm:choreographies:nfa-functional-chorsig} is a completely straightforward, if tedious, formalization of the preceding intuition, along the line of \cref{??}.

\begin{corollary}
  If $q \nfareduces[a] q'_a$, then $\atmR{a} \oc \proc{q} \Reduces \proc{q}'_a$.
  If $\atmR{a} \oc \proc{q} \reduces \octx'$, then $\octx' \Reduces \proc{q}'_a$ for some $q'_a$ that $a$-succeeds $q$.
  If $q \nfareduces[a] q'_a$, then $\atmR{a} \oc \proc{q} \Reduces \proc{q}'_a$.
  If $\atmR{a} \oc \proc{q} \reduces \octx'$, then $\octx' \Reduces \proc{q}'_a$ for some $q'_a$ that $a$-succeeds $q$.
\end{corollary}

\begin{corollary}
  If $q \nfareduces[a] q'_a$, then $\atmR{a} \oc \proc{q} \Reduces \proc{q}'_a$.
  If $\atmR{a} \oc \proc{q} \reduces \octx'$, then $\octx' = \proc{q}'_a$ for some $q'_a$ that $a$-succeeds $q$.
\end{corollary}


\begin{theorem}
  Let $\aut{A} = (...)$ be \iac{DFA} over the input alphabet $\ialph$.
  $q \asim s$ if, and only if, $\proc{q} = \proc{s}$, for all states $q$ and $s$.
\end{theorem}
\begin{proof}
  ...
\end{proof}

But the same does not hold for \acp{NFA}.
\begin{proof}[Counterexample]
  ...
\end{proof}

\subsection{An object-oriented choreography}

The functional choreography for our string rewriting specification of \acp{NFA} is not the only possible choreography.
Instead of treating the input symbols as messages and the states as processes, we may uae a dual assignment: 

\begin{equation*}
  a \wc q \reduces q'_a becomes \defp{a} \oc \atmL{q} \reduces \atmL{q}'_a
  %
  \emp \wc q \reduces F(q) becomes \defp{\emp} \oc \atmL{q} \reduces ?
\end{equation*}
Solving for the recursively defined propositions $\defp{a}$ for each $a \in \ialph$ and $\defp{\emp}$, we have the following definitions:
\begin{equation*}
  \orsig =
  \begin{lgathered}[t]
    \left(
      \defp{a} \defd \bigwith_{q \in Q} \bigwith_{q\smash{'_a}} (\atmL{q}'_a \pmir \atmL{q})
    \right)_{a \in \ialph}
    \,,\\
    \defp{\emp} \defd \bigwith_{q \in Q} (\nfa{F}(q) \pmir \atmL{q})
    \,.
  \end{lgathered}
\end{equation*}

\begin{corollary}
  If $q \nfareduces[a] q'_a$, then $\proc{a} \oc \atmL{q} \Reduces \atmL{q}'_a$.
  If $\proc{a} \oc \atmL{q} \reduces \octx'$, then $\octx' \Reduces \atmL{q}'_a$ for some $q'_a$ that $a$-succeeds $q$.
\end{corollary}

Define a relation on input symbols $a$ and $b$ such that $\proc{a} = \proc{b}$.
Two symbols are then related exactly when they lead to the same successor states.

\section{Binary counters}\label{sec:formula-as-process:counters}

Recall from \cref{??} a string rewriting specification of binary counters, \ie, binary representations of natural numbers equipped with increment and decrement operations:
\begin{inferences}
  \infer{e \wc i \reduces e \wc b_1}{}
  \and
  \infer{b_0 \wc i \reduces b_1}{}
  \and
  \infer{b_1 \wc i \reduces i \wc b_0}{}
  \\
  \infer{e \wc d \reduces z}{}
  \and
  \infer{b_0 \wc d \reduces b'_0}{}
  \and
  \infer{b_1 \wc d \reduces b_0 \wc s}{}
  \\
  \infer{z \wc b'_0 \reduces z}{}
  \and
  \infer{s \wc b'_0 \reduces b_1 \wc s}{}
\end{inferences}
In this \lcnamecref{sec:formula-as-process:counters}, we present 

\subsection{An object-oriented choreography}

\begin{equation*}
  \begin{lgathered}
    \proc{e} \defd (\proc{e} \fuse \proc{b}_1 \pmir \atmL{i}) \with (\atmR{z} \pmir \atmL{d}) \\
    \proc{b}_0 \defd (\up \dn \proc{b}_1 \pmir \atmL{i}) \with (\atmL{d} \fuse \proc{b}'_0 \pmir \atmL{d}) \\
    \proc{b}_1 \defd (\atmL{i} \fuse \proc{b}_0 \pmir \atmL{i}) \with (\proc{b}_0 \fuse \atmR{s} \pmir \atmL{d}) \\
    \proc{b}'_0 \defd (\atmR{z} \limp \atmR{z}) \with (\atmR{s} \limp \proc{b}_1 \fuse \atmR{s})
  \end{lgathered}
\end{equation*}


Is the following result what we want?
\begin{corollary}
  If $\ainc{w}{n}$, then $\theta(w) \Reduces \theta(w')$ and $\aval{w'}{n}$, for some $w'$.
  If $\ainc{w}{n}$ and $\theta(w) \Reduces \octx'$, then $\octx' = \theta(w')$ and $\ainc{w'}{n}$, for some $w'$.
\end{corollary}


\begin{equation*}
  \begin{lgathered}
    \proc{e} \defd (\proc{e} \fuse \proc{b}_1 \pmir \atmL{i}) \with (\proc{z} \pmir \atmL{d}) \\
    \proc{b}_0 \defd (\up \dn \proc{b}_1 \pmir \atmL{i}) \with (\atmL{d} \fuse \atmL{b}'_0 \pmir \atmL{d}) \\
    \proc{b}_1 \defd (\atmL{i} \fuse \proc{b}_0 \pmir \atmL{i}) \with (\proc{b}_0 \fuse \proc{s} \pmir \atmL{d}) \\
    \proc{z} \defd \up \dn \proc{z} \pmir \atmL{b}'_0 \\
    \proc{s} \defd \proc{b}_1 \fuse \proc{s} \pmir \atmL{b}'_0
  \end{lgathered}
\end{equation*}

\begin{verbatim}
  bin = &{ i: bin, d: hdun }
  hdun = &{ x: +{ z: 1, s: bin } }

  1 |- e : bin
  bin |- b0 : bin
  bin |- b1 : bin
  1 |- z : hdun
  bin |- s : hdun
\end{verbatim}

\begin{equation*}
  \begin{lgathered}
    \proc{z} \defd (\up \dn \proc{z} \pmir \atmL{b}'_0) \with (\atmR{z} \pmir \atmL{x}) \\
    \proc{s} \defd (\proc{b}_1 \fuse \proc{s} \pmir \atmL{b}'_0) \with (\atmR{s} \pmir \atmL{x})
  \end{lgathered}
\end{equation*}


\subsection{A functional choreography}

\begin{equation*}
  \begin{lgathered}
    \proc{\imath} \defd (\atmR{e} \limp \atmR{e} \fuse \atmR{b}_1) \with (\atmR{b}_0 \limp \atmR{b}_1) \with (\atmR{b}_1 \limp \proc{\imath} \fuse \atmR{b}_0) \\
    \proc{d} \defd (\atmR{e} \limp \up \dn \proc{z}) \with (\atmR{b}_0 \limp \proc{d} \fuse \atmL{b}'_0) \with (\atmR{b}_1 \limp \atmR{b}_0 \fuse \proc{s}) \\
    \proc{z} \defd \up \dn \proc{z} \pmir \atmL{b}'_0 \\
    \proc{s} \defd \atmR{b}_1 \fuse \proc{s} \pmir \atmL{b}'_0
  \end{lgathered}
\end{equation*}


\begin{equation*}
  \begin{lgathered}
    \proc{\imath} \defd (\atmR{e} \limp \atmR{e} \fuse \atmR{b}_1) \with (\atmR{b}_0 \limp \atmR{b}_1) \with (\atmR{b}_1 \limp \proc{\imath} \fuse \atmR{b}_0) \\
    \proc{d} \defd (\atmR{e} \limp \atmR{z}) \with (\atmR{b}_0 \limp \proc{d} \fuse \proc{b}'_0) \with (\atmR{b}_1 \limp \atmR{b}_0 \fuse \atmR{s}) \\
    \proc{b}'_0 \defd (\atmR{z} \limp \atmR{z}) \with (\atmR{s} \limp \proc{b}_1 \fuse \atmR{s})
  \end{lgathered}
\end{equation*}

%%% Local Variables:
%%% mode: latex
%%% TeX-master: "thesis"
%%% End:
