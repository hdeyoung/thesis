\chapter{Choreographies}\label{ch:choreographies}

As shown in \cref{ch:string-rewriting}, string rewriting is a suitable framework for describing the dynamics\fixnote{?} of concurrent systems whose components form a monoidal structure.
But these [string rewriting] descriptions are [...] high-level specifications, not implementations, because they lack a clear notion of local, decentralized execution give only a global characterization of the interactions between components.

String rewriting axioms $w \reduces w'$ are strictly global in their phrasing, stating merely that any substring of the form $w$ may be replaced, en masse, with $w'$ -- nothing is said about how this replacement is achieved.
The string rewriting framework therefore implicitly suggests that a meta-level actor is responsible for coordinating and conducting the rewriting, with substrings and their constituent symbols as mere passive accessories\fixnote{word choice?}.

In this \lcnamecref{ch:choreographies}, our goal is to move toward a lower level of abstraction.


As an example, recall from \cref{ch:string-rewriting} the string rewriting specification of a system that may transform strings that end with $b$ into the empty string:
\begin{equation}
  \infer{a \wc b \reduces b}{}
  \qquad
  \infer{b \reduces \emp}{}
\end{equation}


Accordingly, we refine the focused ordered rewriting framework of the previous \lcnamecref{ch:ordered-rewriting} into one that can be given a \vocab{formula-as-process} interpretation in which [ordered] rewriting faithfully represents message-passing communication among processes that are arranged in a linear topology.
Then, in \cref{??}, we describe how a string rewriting specification may be transformed into an ordered rewriting \vocab{choreography}.

In this \lcnamecref{ch:choreographies}, we refine the focused ordered rewriting framework of the previous \lcnamecref{ch:ordered-rewriting} into one that can be given a \vocab{formula-as-process} interpretation in which rewriting faithfully represents message-passing communication among 

As argued 

Then we show how, given a mapping of symbols to either [...], choreographies can be generated from a string rewriting specification.

\section{Refining ordered rewriting: A formula-as-process interpretation}

Negative propositions, $\n{A}$, are interpreted as message-passing processes, with positive atoms, $\p{a}$, as messages passed between them.
Ordered contexts, $\octx$, are then configurations of processes and messages arranged in a linear topology.
Finally, the positive propostions, $\p{A}$, reify ordered contexts, and so they can be interpreted as process expressions that reify process configurations.

The ordered implications $\p{A} \limp \n{B}$ and $\n{B} \pmir \p{A}$ are restricted to $\p{a} \limp \n{B}$ and $\n{B} \pmir \p{a}$, respectively, so that they may cleanly be interpreted as processes that input a message $\p{a}$ from the left and right, respectively.


Each positive atom $\p{a}$ is assigned a direction, either $\atmL{a}$ or $\atmR{a}$, that indicates 

$\atmL{a}$ and $\atmR{a}$; and $\p{A} \limp \n{B}$ restricted to $\atmR{a} \limp \n{B}$ and similarly for right-handed implication.

\subsection{Formula-as-process}

Discuss here?

Example of $\proc{b} \defd (\atmR{a} \limp \up \dn \proc{b}) \with \up \one$, without explicitly relating it to the specification.

\subsection{Focused ordered rewriting, revisited}

The two changes introduced by the formula-as-process interpretation -- atom directions and atomic premises for implications -- trickle down to the focused ordered rewriting framework.

Now that each positive atom is marked with a direction, the $\jrule{ID}\smash{^{\p{a}}}$ rule%
\marginnote{\qquad%
  $\infer[\jrule{ID}\smash{^{\p{a}}}]{\rfocus{\p{a}}{\p{a}}}{}$%
}
that was previously part of the right-focus judgment's definition is replaced by two similar rules:
\begin{equation*}
  \infer[\jrule{ID}\smash{^{\atmR{a}}}]{\rfocus{\atmR{a}}{\atmR{a}}}{}
  \qquad\text{and}\qquad
  \infer[\jrule{ID}\smash{^{\atmL{a}}}]{\rfocus{\atmL{a}}{\atmL{a}}}{}
  \,.
\end{equation*}
The other right-focusing rules remain unchanged.

Because left-handed implications have only right-directed atoms as premises and right-handed implications have only left-directed atoms as premises, the left-focus judgment may also be refined to $\lfocus{\atmR{\octx}_L}{\n{A}}{\atmL{\octx}_R}{\p{C}}$ -- the left-hand context now consists only of right-directed atoms, hence $\atmR{\octx}_L$; and the right-hand context now consists only of left-directed atoms, hence $\atmL{\octx}_R$.

\begin{equation*}
  \infer[\lrule{\limp}]{\lfocus{\atmR{\octx}_L \oc \atmR{a}}{\atmR{a} \limp \n{B}}{\atmL{\octx}_R}{\p{C}}}{
    \infer[\jrule{ID}\smash{^{\atmR{a}}}]{\rfocus{\atmR{a}}{\atmR{a}}}{} &
    \lfocus{\atmR{\octx}_L}{\n{B}}{\atmL{\octx}_R}{\p{C}}}
  %
  \qquad
  %
  \infer[\lrule{\limp}']{\lfocus{\atmR{\octx}_L \oc \atmR{a}}{\atmR{a} \limp \n{B}}{\atmL{\octx}_R}{\p{C}}}{
    \lfocus{\atmR{\octx}_L}{\n{B}}{\atmL{\octx}_R}{\p{C}}}
\end{equation*}
By a similar construction, we arrive at a $\lrule{\pmir}'$ rule, as well:
\begin{equation*}
  \infer[\lrule{\pmir}']{\lfocus{\atmR{\octx}_L}{\n{B} \pmir \atmL{a}}{\atmL{a} \oc \atmL{\octx}_R}{\p{C}}}{
    \lfocus{\atmR{\octx}_L}{\n{B}}{\atmL{\octx}_R}{\p{C}}}
\end{equation*}
The other rules for the left-focus judgment remain unchanged, save for the fact that the left- and right-hand context now contain only atoms of the appropriate direction.
\Cref{fig:??} summarizes the revised rules for the right- and left-focus judgments.

\begin{equation*}
  \infer[?]{\atmR{\octx}_L \oc \n{A} \oc \atmL{\octx}_R \reduces \np{\octx'{}}}{
    \lfocus{\atmR{\octx}_L}{\n{A}}{\atmL{\octx}_R}{\p{B}} &
    \rfocus{\np{\octx'{}}}{\p{B}}}
\end{equation*}

\subsection{Input transitions}

Thus far, we have examined only rewriting [...].

Now that we are ascribing a formula-as-process interpretation to ordered rewriting, this judgment characterizes internal reductions.

In process calculi like the $\pi$-calculus, labeled transition systems are frequently used as an alternative to reduction in describing the operational semantics.
Now that we are ascribing a formula-as-process interpretation to ordered rewriting, we can similarly conceive of a labeled transition system for ordered contexts.

$\ireduces{\atmR{\octx}_L \oc #1 \oc \atmL{\octx}_R}{\np{\octx}}{\np{\octx'{}}}$

\begin{equation*}
  \infer[?]{\ireduces{\atmR{\octx}_L \oc #1 \oc \atmL{\octx}_R}{\n{A}}{\np{\octx'{}}}}{
    \lfocus{\atmR{\octx}_L}{\n{A}}{\atmL{\octx}_R}{\p{C}} &
    \rfocus{\np{\octx'{}}}{\p{C}}}
\end{equation*}
Aside from the change of judgment in the rule's conclusion, this [...] rule is identical to the core [...] rule for reduction.
How can we claim that input transition is distinct from reduction?

The difference is one of input/output modes.
In a reduction $\atmR{\octx}_L \oc \n{A} \oc \atmL{\octx}_R \reduces \np{\octx'{}}$, the entire $\atmR{\octx}_L \oc \n{A} \oc \atmL{\octx}_R$ context is treated as an input to the reduction judgment.
In the input transition $\ireduces{\atmR{\octx}_L \oc #1 \oc \atmL{\octx}_R}{\n{A}}{\np{\octx'{}}}$, on the other hand, the proposition $\n{A}$ is treated as an input to the input transition judgment, but the contexts $\atmR{\octx}_L$, $\atmL{\octx}_R$, and $\np{\octx'{}}$ are all treated as outputs.

In addition to this core input transition rule, 

\begin{equation*}
  \infer[?]{\ireduces{\atmR{\octx}_L \oc #1 \oc \atmL{\octx}_R}{\n{A}}{\np{\octx'{}}}}{
    \lfocus{\atmR{\octx}_L}{\n{A}}{\atmL{\octx}_R}{\p{C}} &
    \rfocus{\np{\octx'{}}}{\p{C}}}
\end{equation*}


Also discuss here?

\begin{inferences}
  \infer{\ireduces{\atmR{\octx}_L \oc #1 \oc \atmL{\octx}_R}{\n{A}}{\octx'}}{
    \lfocus{\atmR{\octx}_L}{\n{A}}{\atmL{\octx}_R}{\p{C}} &
    \rfocus{\octx'}{\p{C}}}
  \\
  \infer{\ireduces{\atmR{\octx}_L \oc #1 \oc \atmL{\octx}_R}{\atmR{a} \oc \octx}{\octx'}}{
    \ireduces{\atmR{\octx}_L \oc \atmR{a} \oc #1 \oc \atmL{\octx}_R}{\octx}{\octx'}}
  \and
  \infer{\ireduces{\atmR{\octx}_L \oc #1 \oc \atmL{\octx}_R}{\octx \oc \atmL{a}}{\octx'}}{
    \ireduces{\atmR{\octx}_L \oc #1 \oc \atmL{a} \oc \atmL{\octx}_R}{\octx}{\octx'}}
  \\
  \infer{\ireduces{#1 \oc \atmL{\octx}_R}{\omega \oc \octx}{\omega \oc \octx'}}{
    \ireduces{#1 \oc \atmL{\octx}_R}{\octx}{\octx'}}
  \and
  \infer{\ireduces{\atmR{\octx}_L \oc #1}{\octx \oc \omega}{\octx' \oc \omega}}{
    \ireduces{\atmR{\octx}_L \oc #1}{\octx}{\octx'}}
\end{inferences}

\begin{theorem}
  If $\ireduces{\atmR{\octx}_L \oc #1 \oc \atmL{\octx}_R}{\octx}{\octx'}$, then $\atmR{\octx}_L \oc \octx \oc \atmL{\octx}_R \reduces \octx'$.
  Conversely, if $\octx \reduces \octx'$, then $\ireduces{#1}{\octx}{\octx'}$.
\end{theorem}
\begin{proof}
  The two claims are proved by induction on the structure of the given input transition and reduction, respectively, after first proving an easy lemma:
  \begin{itemize}
  \item If $\ireduces{\atmR{\octx}_L \oc #1 \oc \atmL{\octx}_R}{\octx}{\octx'}$, then $\ireduces{#1}{\atmR{\octx}_L \oc \octx \oc \atmL{\octx}_R}{\octx'}$.
  \qedhere
  \end{itemize}
\end{proof}

\section{Constructing a choreography from a specification}

Recall from \cref{??} the string rewriting specification having axioms
\begin{equation*}
  \infer{a \oc b \reduces b}{}
  \qquad\text{and}\qquad
  \infer{b \reduces \emp}{}
  \:.
\end{equation*}
As a string rewriting specification, the axioms are interpreted from a global perspective.
For instance, the first axiom states that when the symbols $a \oc b$ occur in that order, they may be rewritten to $b$.
But the axiom does not describe how that rewriting occurs.

With choreographies, we would like to work at a (slightly) lower level of abstraction to describe 


Suppose that we are given a string rewriting specification that consists of axioms $?$ over the rewriting alphabet $\sralph$.
A \vocab{choreographing assignment} is an injection in which each symbol $a \in \sralph$ is mapped to an ordered proposition: either an atomic proposition, $\atmL{a}$ or $\atmR{a}$, or a recursively defined proposition, $\defp{a}$.

Given a choregraphing assignment $\theta$, we may construct a choregraphy from the string rewriting specification.
Intuitively, each axiom is annotated according to $\theta$, and then the resulting [...] are used to construct a family of recursive definitions, one for each $\defp{a}$ in the image of $\theta$.

A choreography is an ordered rewriting specification that simulates the string rewriting specification [...].

Consider the recurring string rewriting specification with axioms
\begin{equation*}
  \infer{a \oc b \reduces b}{}
  \qquad\text{and}\qquad
  \infer{b \reduces \emp}{}
  \:.
\end{equation*}
We must consistently annotate each symbol as either a left-directed atom, right-directed atom, or recursively defined proposition in sucha way that each axiom's premise $w$ has the form $w_1 \wc a \wc w_2$ with 

\begin{equation*}
  \atmR{a} \oc \defp{b} \reduces \defp{b}
  \qquad\text{and}\qquad
  \defp{b} \reduces \octxe
\end{equation*}

Now we must solve for $\defp{b}$, determining a definition $\defp{b} \defd \n{B}$ such that these two rewriting steps are derivable.
For the first step to be derivable, $\defp{b}$ should have a definition that is consistent with $\atmR{a} \limp \up \dn \defp{b}$, for 
\begin{equation*}
  \atmR{a} \oc (\atmR{a} \limp \up \dn \defp{b}) \reduces \defp{b}
\end{equation*}

Consider the choreographing assignment $\theta$ that maps $a$ to the atom $\atmR{a}$ and $b$ to the recursively defined proposition $\defp{b}$.
Upon annotating the above string rewriting axioms according to $\theta$, we arrive at the [...]
\begin{equation*}
  \atmR{a} \oc \dprop{b} \reduces \dprop{b}
  \qquad\text{and}\qquad
  \dprop{b} \reduces \one
  \,.
\end{equation*}

\begin{equation*}
  \dprop{b} \defd \atmR{a} \limp \up \dn \dprop{b}
  \qquad\text{and}\qquad
  \dprop{b} \defd \up \one
  \,,
\end{equation*}
respectively.
Combining these into a single definition that allows a nondeterministic choice between the two, we have
\begin{equation*}
  \dprop{b} \defd (\atmR{a} \limp \up \dn \dprop{b}) \with \up \one
  \,,
\end{equation*}
or $\dprop{b} \defd (\atmR{a} \limp \up \dprop{b}) \with \one$ if the minimally necessary shifts are elided.

By construction, this choreography is adequate with respect to the specification, in the sense that it can simulate each of the specification's possible steps and vice versa.
\begin{itemize}
\item $w \reduces w'$ only if $\theta(w) \reduces \theta(w')$
  For example, just as the string rewriting specification admits $a \oc b \reduces b$, the ordered rewriting choreography admits
  \begin{equation*}
    \theta(a \oc b) = \atmR{a} \oc \defdP{b} = \atmR{a} \oc \bigl((\atmR{a} \limp \up \dn \defdP{b}) \with \one\bigr) \reduces \defdP{b} = \theta(b)
    \,.
  \end{equation*}

\item $\theta(w) \reduces \octx'$ only if $w \reduces \theta^{-1}(\octx')$
  For example, just as the ordered rewriting choreography admits $\theta(b) = \defdP{b} \reduces \octxe$, the string rewriting specification admits $b \reduces \emp = \theta^{-1}(\octxe)$.
\end{itemize}
% The choreography can simulate each of the specification's possible rewriting steps: for example, $\theta(a \oc b) = \atmR{a} \oc \dprop{b} \reduces \dprop{b} = \theta(b)$, just as $a \oc b \reduces b$.
% Conversely, each of the choreography's possible rewriting steps can be simulated by the specification: for example, $\theta^{-1}(\atmR{a} \oc \dprop{b}) = a \oc b \reduces b = \theta^{-1}(\dprop{b})$, just as $\atmR{a} \oc \dprop{b} \reduces \dprop{b}$.

\subsection{Formal description}

Judgments $\chorsig{\theta}{\srsig}{\orsig}$ and $\chorax{\theta}{w_1}{\n{A}}{w_2}{\n{B}}$.
In both judgments, all terms before the $\chorarrow$ are inputs; all terms after the $\chorarrow$ are outputs.

\begin{inferences}
  \infer{\chorsig{\theta}{\srsige}{\orsige}}{}
  \\
  \infer{\chorsig{\theta}{\srsig, w_1 \oc a \oc w_2 \reduces w'}{\orsig, \defp{a} \defd \n{A} \with \n{B}}}{
    \text{($\theta(a) = \defp{a}$)} &
    \chorax{\theta}{w_1}{\bigfuse \theta(w')}{w_2}{\n{B}} &
    \chorsig{\theta}{\srsig}{\orsig, \defp{a} \defd \n{A}}}
  \\
  \infer{\chorsig{\theta}{\srsig, w_1 \oc a \oc w_2 \reduces w'}{\orsig, \defp{a} \defd \n{B}}}{
    \text{($\theta(a) = \defp{a}$)} &
    \chorax{\theta}{w_1}{\bigfuse \theta(w')}{w_2}{\n{B}} &
    \chorsig{\theta}{\srsig}{\orsig} &
    \text{($\defp{a} \notin \dom{\orsig}$)}}
  \\
  \infer{\chorax{\theta}{\emp}{\p{A}}{\emp}{\up \p{A}}}{}
  \\
  \infer{\chorax{\theta}{w_1 \oc b}{\p{A}}{w_2}{\atmR{b} \limp \n{B}}}{
    \chorax{\theta}{w_1}{\p{A}}{w_2}{\n{B}} &
    \text{($\theta(b) = \atmR{b}$)}}
  \and
  \infer{\chorax{\theta}{w_1}{\p{A}}{b \oc w_2}{\n{B} \pmir \atmL{b}}}{
    \chorax{\theta}{w_1}{\p{A}}{w_2}{\n{B}} &
    \text{($\theta(b) = \atmL{b}$)}}
\end{inferences}

% \begin{inferences}
%   \infer{\chorsig{\theta}{\sige}{\sige}}{}
%   \and
%   \infer{\chorsig{\theta}{\sig, w \reduces w'}{\sig', \proc{a} \defd \n{A}_1 \with \n{A}_2(\up \p{C})}}{
%     \chorsig{\theta}{\sig}{\sig'} &
%     \chorax{\theta}{w \reduces w'}{\proc{a}}{\n{A}_2(\Box)}{\p{C}} &
%     \text{($\sig'(\proc{a}) = \n{A}_1$)}}
%   \\
%   \infer{\chorsig{\theta}{\sig, w \reduces w'}{\sig', \proc{a} \defd \n{A}(\up \p{C})}}{
%     \chorsig{\theta}{\sig}{\sig'} &
%     \chorax{\thea}{w \reduces w'}{\proc{a}}{\n{A}(\Box)}{\p{C}} &
%     \text{($\proc{a} \notin \dom{\sig'}$)}}
%   \\
%   \infer{\chorax{\theta}{a \reduces w'}{\proc{a}}{\Box}{\bigfuse \octx'}}{
%     \text{($\theta(a) = \proc{a}$)} &
%     \text{($\theta(w') = \octx'$)}}
%   \and
%   \infer{\chorax{\theta}{b \oc w \reduces w'}{\proc{a}}{\n{A}(\atmR{b} \limp \Box)}{\p{C}}}{
%     \chorax{\theta}{w \reduces w'}{\proc{a}}{\n{A}(\Box)}{\p{C}} &
%     \text{($\theta(b) = \atmR{b}$)}}
%   \\
%   \infer{\chorax{\theta}{w \oc b \reduces w'}{\proc{a}}{\n{A}(\Box \pmir \atmL{b})}{C}}{
%     \chorax{\theta}{w \reduces w'}{\proc{a}}{\n{A}(\Box)}{\p{C}} &
%     \text{($\theta(b) = \atmL{b}$)}}
% \end{inferences}


\begin{lemma}\label{lem:chorax-sound-complete}
  % If $\chorax{\theta}{w_1}{\p{C}}{w_2}{\n{B}}$, then $\lfocus{\theta(w_1)}{\n{B}}{\theta(w_2)}{\p{C}}$.
  If $\chorax{\theta}{w_1}{\p{A}}{w_2}{\n{B}}$, then $\lfocus{\atmR{\octx}_L}{\n{B}}{\atmL{\octx}_R}{\p{C}}$ if, and only if, $\atmR{\octx}_L = \theta(w_1)$ and $\atmL{\octx}_R = \theta(w_2)$ and $\p{A} = \p{C}$.
\end{lemma}
\begin{proof}
  Each direction is proved separately by induction over the structure of the given derivation.

  \begin{itemize}
  \item Consider the case in which
    \begin{equation*}
      \infer{\chorax{\theta}{\emp}{\p{A}}{\emp}{\up \p{A}}}{}
      \,.
    \end{equation*}
    We must show that $\lfocus{\atmR{\octx}_L}{\up \p{A}}{\atmL{\octx}_R}{\p{C}}$ if, and only if, $\atmR{\octx}_L = \atmL{\octx}_R = \theta(\emp)$ and $\p{A} = \p{C}$.
    Indeed, the $\lrule{\up}$ rule is the unique rule for left-focusing on $\up \p{A}$, and $\octxe = \theta(\emp)$ because $\theta$ is a monoid homomorphism.

  \item Consider the case in which
  \begin{equation*}
    \infer{\chorax{\theta}{w_1 \oc b}{\p{A}}{w_2}{\atmR{b} \limp \n{B}}}{
      \chorax{\theta}{w_1}{\p{A}}{w_2}{\n{B}} &
      \text{($\theta(b) = \atmR{b}$)}}
    \,.
  \end{equation*}
    We must show that $\lfocus{\atmR{\octx}_L}{\atmR{b} \limp \n{B}}{\atmL{\octx}_R}{\p{C}}$ if, and only if, $\atmR{\octx}_L = \theta(w_1 \oc b)$ and $\atmL{\octx}_R = \theta(w_2)$ and $\p{A} = \p{C}$.
    Indeed, the $\lrule{\limp}$ rule is the unique rule for left-focusing on $\atmR{b} \limp \n{B}$, so $\lfocus{\atmR{\octx}_L}{\atmR{b} \limp \n{B}}{\atmL{\octx}_R}{\p{C}}$ if, and only if, $\atmR{\octx}_L = \atmR{\octx}'_L \oc \atmR{b}$ and $\lfocus{\atmR{\octx}'_L}{\n{B}}{\atmL{\octx}_R}{\p{C}}$ for some $\atmR{\octx}'_L$.
    By the inductive hypothesis, $\lfocus{\atmR{\octx}'_L}{\n{B}}{\atmL{\octx}_R}{\p{C}}$ if, and only if, $\atmR{\octx}'_L = \theta(w_1)$ and $\atmL{\octx}_R = \theta(w_2)$ and $\p{A} = \p{C}$.
    Also, notice that $\theta(w_1) \oc \atmR{b} = \theta(w_1 \wc b)$.
    Putting everything together, $\lfocus{\atmR{\octx}_L}{\atmR{b} \limp \n{B}}{\atmL{\octx}_R}{\p{C}}$ if, and only if, $\atmR{\octx}_L = \theta(w_1 \wc b)$ and $\atmL{\octx}_R = \theta(w_2)$ and $\p{A} = \p{C}$, as required.

  \item
    The case in which
  \begin{equation*}
    \infer{\chorax{\theta}{w_1}{\p{A}}{b \oc w_2}{\n{B} \pmir \atmL{b}}}{
      \chorax{\theta}{w_1}{\p{A}}{w_2}{\n{B}} &
      \text{($\theta(b) = \atmL{b}$)}}
  \end{equation*}
    is symmtric to the previous one.
  %
  \qedhere
  \end{itemize}
\end{proof}


\begin{lemma}[Weakening]
  If $\octx \reduces_{\orsig} \octx'$ and $\dom{\orsig} \cap \dom{\orsig'} = \emptyset$, then $\octx \reduces_{\orsig, \orsig'} \octx'$.
  If $\octx \reduces_{\orsig, \defp{a} \defd \n{A}_1} \octx'$ or $\octx \reduces_{\orsig, \defp{a} \defd \n{A}_2} \octx'$, then $\octx \reduces_{\orsig, \defp{a} \defd \n{A}_1 \with \n{A}_2} \octx'$.
\end{lemma}
\begin{proof}
  By induction over the structure of the given rewriting step.
\end{proof}

\begin{lemma}
  If $(w \reduces w') \in \srsig$ and $\chorsig{\theta}{\srsig}{\orsig}$, then $\theta(w) \reduces_{\orsig} \theta(w')$.
\end{lemma}
\begin{proof}
  By induction over the structure of the given choreographing derivation, $\chorsig{\theta}{\srsig}{\orsig}$.
  \begin{itemize}
  \item Consider the case in which $w = w_1 \oc a \oc w_2 \reduces w'$ is the axiom in question and
    \begin{equation*}
      \infer{\chorsig{\theta}{\srsig, w_1 \oc a \oc w_2 \reduces w'}{\orsig, \defp{a} \defd \n{A} \with \n{B}}}{
        \text{($\theta(a) = \defp{a}$)} &
        \chorax{\theta}{w_1}{\bigfuse \theta(w')}{w_2}{\n{B}} &
        \chorsig{\theta}{\srsig}{\orsig, \defp{a} = \n{A}}}
    \end{equation*}
    It follows from \cref{??} that $\lfocus{\theta(w_1)}{\n{B}}{\theta(w_2)}{\bigfuse \theta(w')}$, and hence $\lfocus{\theta(w_1)}{\n{A} \with \n{B}}{\theta(w_2)}{\bigfuse \theta(w')}$.
    And because $\rfocus{\theta(w')}{\bigfuse \theta(w')}$ and $\defp{a} \defd \n{A} \with \n{B}$, we have $\theta(w) = \theta(w_1) \oc \defp{a} \oc \theta(w_2) \reduces \theta(w')$.
  
  \item Consider the case in which
    \begin{equation*}
      \infer{\chorsig{\theta}{\sig, v_1 \oc a \oc v_2 \reduces v'}{\sig', \hat{a} \defd \n{A} \with \n{B}}}{
        \text{($\theta(a) = \hat{a}$)} &
        \chorax{\theta}{v_1}{\bigfuse \theta(v')}{v_2}{\n{B}} &
        \chorsig{\theta}{\sig}{\sig'} &
        \text{($\sig'(\hat{a}) = \n{A}$)}}
    \end{equation*}
    and the axiom $w \reduces w'$ comes from $\sig$.
    By the inductive hypothesis, $\theta(w) \reduces_{\sig'} \theta(w')$.
    By \cref{??}, $\theta(w) \reduces_{\sig', \hat{a} \defd \n{A} \with \n{B}} \theta(w')$.
  
  \item Consider the case in which
    \begin{equation*}
      \infer{\chorsig{\theta}{\sig, v_1 \oc a \oc v_2 \reduces v'}{\sig', \hat{a} \defd \n{B}}}{
        \text{($\theta(a) = \hat{a}$)} &
        \chorax{\theta}{v_1}{\bigfuse \theta(v')}{v_2}{\n{B}} &
        \chorsig{\theta}{\sig}{\sig'} &
        \text{($\hat{a} \notin \dom{\sig'}$)}}
    \end{equation*}
    and the axiom $w \reduces w'$ comes from $\sig$.
    By the inductive hypothesis, $\theta(w) \reduces_{\sig'} \theta(w')$.
    By \cref{??}, $\theta(w) \reduces_{\sig', \hat{a} \defd \n{B}} \theta(w')$.
  %
  \qedhere
  \end{itemize}
\end{proof}

\begin{theorem}[Completeness]\leavevmode
  If $\chorsig{\theta}{\srsig}{\orsig}$ and $w \reduces_{\srsig} w'$, then $\theta(w) \reduces_{\orsig} \theta(w')$.
\end{theorem}
\begin{proof}
  By simultaneous structural induction on the given choreographing derivation, $\chorsig{\theta}{\srsig}{\orsig}$, and ordered rewriting step, $w \reduces_{\srsig} w'$.
  \begin{itemize}
  \item
    Consider the case in which
    \begin{equation*}
      \chorsig{\theta}{\srsig}{\orsig}
      \qquad\text{and}\qquad
      w =
      \infer[\jrule{$\reduces$C}\smash{_{\jrule{L}}}]{w_1 \oc w_2 \reduces w'_1 \oc w_2}{
        w_1 \reduces w'_1}
      = w'
      \,.
    \end{equation*}
    By the inductive hypothesis, $\theta(w_1) \reduces \theta(w'_1)$.
    It follows from ordered rewriting's $\jrule{$\reduces$C}\smash{_{\jrule{L}}}$ rule that $\theta(w) = \theta(w_1) \oc \theta(w_2) \reduces \theta(w'_1) \oc \theta(w_2) = \theta(w')$.

  \item
    The case in which
    \begin{equation*}
      \chorsig{\theta}{\srsig}{\orsig}
      \qquad\text{and}\qquad
      w =
      \infer[\jrule{$\reduces$C}\smash{_{\jrule{R}}}]{w_1 \oc w_2 \reduces w_1 \oc w'_2}{
        w_2 \reduces w'_2}
      = w'
    \end{equation*}
    is symmetric.

  \item
    Consider the case in which
    \begin{equation*}
      \infer{\chorsig{\theta}{\srsig, w_1 \wc a \wc w_2 \reduces w'}{\orsig, \defp{a} \defd \n{A}}}{
        \text{($\theta(a) = \defp{a}$)} &
        \chorax{\theta}{w_1}{\bigfuse \theta(w')}{w_2}{\n{A}} &
        \chorsig{\theta}{\srsig}{\orsig} &
        \text{($\defp{a} \notin \dom{\orsig}$)}}
      \qquad\text{and}\qquad
      w =
      \infer[\jrule{$\reduces$AX}]{w_1 \wc a \wc w_2 \reduces_{\srsig, w_1 \wc a \wc w_2 \reduces w'} w'}{}
    \end{equation*}
    By \cref{lem:chorax-sound-complete}, $\lfocus{\theta(w_1)}{\n{A}}{\theta(w_2)}{\bigfuse \theta(w')}$.
    Because $\rfocus{\theta(w')}{\bigfuse \theta(w')}$, it follows by the []\fixnote{fix} rule that $\theta(w_1) \oc \n{A} \oc \theta(w_2) \reduces_{\orsig, \defp{a} \defd \n{A}} \theta(w')$, and so $\theta(w) \reduces_{\orsig, \defp{a} \defd \n{A}} \theta(w')$

  \item
    Consider the case in which
    \begin{equation*}
      \infer{\chorsig{\theta}{\srsig, w_1 \wc a \wc w_2 \reduces w'}{\orsig, \defp{a} \defd \n{A}_1 \with \n{A}_2}}{
        \text{($\theta(a) = \defp{a}$)} &
        \chorax{\theta}{w_1}{\bigfuse \theta(w')}{w_2}{\n{A}_2} &
        \chorsig{\theta}{\srsig}{\orsig, \defp{a} \defd \n{A}_1}}
      \qquad\text{and}\qquad
      w =
      \infer[\jrule{$\reduces$AX}]{w_1 \wc a \wc w_2 \reduces_{\srsig, w_1 \wc a \wc w_2 \reduces w'} w'}{}
    \end{equation*}
    By \cref{lem:chorax-sound-complete}, $\lfocus{\theta(w_1)}{\n{A}_2}{\theta(w_2)}{\bigfuse \theta(w')}$.
    Upon adding the $\lrule{\with}_2$ rule, $\lfocus{\theta(w_1)}{\n{A}_1 \with \n{A}_2}{\theta(w_2)}{\bigfuse \theta(w')}$.
    Because $\rfocus{\theta(w')}{\bigfuse \theta(w')}$, it follows by the []\fixnote{fix} rule that $\theta(w_1) \oc (\n{A}_1 \with \n{A}_2) \oc \theta(w_2) \reduces_{\orsig, \defp{a} \defd \n{A}_1 \with \n{A}_2} \theta(w')$, and so $\theta(w) \reduces_{\orsig, \defp{a} \defd \n{A}_1 \with \n{A}_2} \theta(w')$
    
  \item
    Consider the case in which
    \begin{equation*}
      \infer{\chorsig{\theta}{\srsig, v_1 \wc v \wc v_2 \reduces v'}{\orsig, \defp{b} \defd \n{B}}}{
        \text{($\theta(b) = \defp{b}$)} &
        \chorax{\theta}{v_1}{\bigfuse \theta(v')}{v_2}{\n{B}} &
        \chorsig{\theta}{\srsig}{\orsig} &
        \text{($\defp{b} \notin \dom{\orsig}$)}}
      \qquad\text{and}\qquad
      \infer[\jrule{$\reduces$AX}]{w \reduces_{\srsig, v_1 \wc b \wc v_2 \reduces v'} w'}{
        w \reduces w' \in \srsig}
    \end{equation*}
  \end{itemize}
\end{proof}




% \begin{lemma}
%   If $\chorax{\theta}{w_1}{\p{A}}{w_2}{\n{B}}$ and $\lfocus{\atmR{\octx}_L}{\n{B}}{\atmL{\octx}_R}{\p{C}}$, then $\atmR{\octx}_L = \theta(w_1)$ and $\atmL{\octx}_R = \theta(w_2)$ and $\p{A} = \p{C}$.
% \end{lemma}
% \begin{proof}
%   By induction over the structure of the given choreographing derivation, $\chorax{\theta}{w_1}{\p{A}}{w_2}{\n{B}}$.
%   \begin{itemize}
%   \item
%     Consider the case in which
%   \begin{equation*}
%     \infer{\chorax{\theta}{\emp}{\p{A}}{\emp}{\up \p{A}}}{}
%     \qquad\text{and}\qquad
%     \lfocus{\atmR{\octx}_1}{\up \p{A}}{\atmL{\octx}_2}{\p{C}}
%     \,.
%   \end{equation*}
%   By inversion on the left-focus derivation, $\atmR{\octx}_L = \octxe = \theta(\emp)$ and $\atmL{\octx}_R = \octxe = \theta(\emp)$, as well as $\p{A} = \p{C}$, as required.

%   \item
%     Consider the case in which
%   \begin{equation*}
%     \infer{\chorax{\theta}{w_1 \oc b}{\p{A}}{w_2}{\atmR{b} \limp \n{B}}}{
%       \text{($\theta(b) = \atmR{b}$)} &
%       \chorax{\theta}{w_1}{\p{A}}{w_2}{\n{B}}}
%     \qquad\text{and}\qquad
%     \lfocus{\atmR{\octx}_L}{\atmR{b} \limp \n{B}}{\atmL{\octx}_R}{\p{C}}
%     \,.
%   \end{equation*}
%   By inversion on the left-focus derivation for $\atmR{b} \limp \n{B}$, there exists $\atmR{\octx}'_1$ such that $\atmR{\octx}_L = \atmR{\octx}'_L \oc \atmR{b}$ and $\lfocus{\atmR{\octx}'_L}{\n{B}}{\atmL{\octx}_R}{\p{C}}$.
%   It follows from the inductive hypothesis that $\atmR{\octx}'_L = \theta(w_1)$ and $\atmL{\octx}_R = \theta(w_2)$ and $\p{A} = \p{C}$.
%   So $\atmR{\octx}_L = \theta(w_1) \oc \atmR{b} = \theta(w_1 \oc b)$.

%   \item
%     The case in which
% \begin{equation*}
%     \infer{\chorax{\theta}{w_1 \oc b}{\p{A}}{w_2}{\n{B} \pmir \atmL{b}}}{
%       \text{($\theta(b) = \atmL{b}$)} &
%       \chorax{\theta}{w_1}{\p{A}}{w_2}{\n{B}}}
%     \qquad\text{and}\qquad
%     \lfocus{\atmR{\octx}_L}{\n{B} \pmir \atmL{b}}{\atmL{\octx}_R}{\p{C}}
%   \end{equation*}
%   is symmetric.
%   \qedhere
%   \end{itemize}
% \end{proof}


\begin{lemma}
  If $\chorsig{\theta}{\srsig}{\orsig, \defp{a} \defd \n{A}}$ and $\lfocus{\atmR{\octx}_L}{\defp{a}}{\atmL{\octx}_R}{\p{C}}$, then there exists a string rewriting axiom $(w_1 \oc a \oc w_2 \reduces w') \in \srsig$ such that $\atmR{\octx}_L = \theta(w_1)$, $\atmL{\octx}_R = \theta(w_2)$, and $\p{C} = \bigfuse \theta(w')$.
\end{lemma}
\begin{proof}
  By induction over the structure of the given choreographing derivation, $\chorsig{\theta}{\srsig}{\orsig, \defp{a} \defd \n{A} }$.
  \begin{itemize}
  \item
  Consider the case in which
  \begin{gather*}
    \infer{\chorsig{\theta}{\srsig, v_1 \oc b \oc v_2 \reduces v'}{\orsig, \defp{a} \defd \n{A}, \defp{b} \defd \n{B}_1 \with \n{B}_2}}{
      \text{($\theta(b) = \defp{b}$)} &
      \chorax{\theta}{v_1}{\bigfuse \theta(v')}{v_2}{\n{B}_2} &
      \chorsig{\theta}{\srsig}{\orsig, \defp{a} \defd \n{A}, \defp{b} \defd \n{B}_1}}
    \\\text{and}\\
    \lfocus{\atmR{\octx}_L}{\defp{a}}{\atmL{\octx}_R}{\p{C}}
    \,.
  \end{gather*}
  By the inductive hypothesis, there exists a string rewriting axiom $(w_1 \oc a \oc w_2 \reduces w') \in \srsig$ such that $\atmR{\octx}_L = \theta(w_1)$, $\atmL{\octx}_R = \theta(w_2)$, and $\p{C} = \bigfuse \theta(w')$.
  The same axiom is contained in the signature $\srsig, v_1 \oc b \oc v_2 \reduces v'$.

  \item
    The case in which
  \begin{gather*}
    \infer{\chorsig{\theta}{\srsig, v_1 \oc b \oc v_2 \reduces v'}{\orsig, \defp{a} \defd \n{A}, \defp{b} \defd \n{B}}}{
      \text{($\theta(b) = \defp{b}$)} &
      \chorax{\theta}{v_1}{\bigfuse \theta(v')}{v_2}{\n{B}} &
      \chorsig{\theta}{\srsig}{\orsig, \defp{a} \defd \n{A}} &
      \text{($\defp{b} \notin \dom{\orsig}$)}}
    \\\text{and}\\
    \lfocus{\atmR{\octx}_L}{\defp{a}}{\atmL{\octx}_R}{\p{C}}
  \end{gather*}
  is similar.

  \item
  Consider the case in which
  \begin{gather*}
    \infer{\chorsig{\theta}{\srsig, v_1 \oc a \oc v_2 \reduces v'}{\orsig, \defp{a} \defd \n{A}_1 \with \n{A}_2}}{
      \text{($\theta(a) = \defp{a}$)} &
      \chorax{\theta}{v_1}{\bigfuse \theta(v')}{v_2}{\n{A}_2} &
      \chorsig{\theta}{\srsig}{\orsig, \defp{a} \defd \n{A}_1}}
    \\\text{and}\\
    \lfocus{\atmR{\octx}_L}{\defp{a}}{\atmL{\octx}_R}{\p{C}}
    \,.
  \end{gather*}
  There are two cases, according to whether the $\lfocus{\atmR{\octx}_L}{\defp{a}}{\atmL{\octx}_R}{\p{C}}$ derivation ends with the $\lrule{\with}_1$ or $\lrule{\with}_2$ rule.
    \begin{itemize}
    \item If the left-focus derivation ends with the $\lrule{\with}_2$ rule, then $\lfocus{\atmR{\octx}_L}{\n{A}_2}{\atmL{\octx}_R}{\p{C}}$.
      Because $\chorax{\theta}{v_1}{\bigfuse \theta(v')}{v_2}{\n{A}_2}$, it follows from \cref{??} that $\atmR{\octx}_L = \theta(v_1)$ and $\atmL{\octx}_R = \theta(v_2)$ and $\p{C} = \bigfuse \theta(v')$.
      Choose the axiom $w_1 \oc a \oc w_2 \reduces w'$ to be $v_1 \oc a \oc v_2 \reduces v'$.

    \item Otherwise, if the left-focus derivation instead ends with the $\lrule{\with}_1$ rule, then $\lfocus{\atmR{\octx}_L}{\n{A}_1}{\atmL{\octx}_R}{\p{C}}$.
      By the inductive hypothesis, $\atmR{\octx}_L = \theta(w_1)$, $\atmL{\octx}_R = \theta(w_2)$, and $\p{C} = \bigfuse \theta(w')$ for some string rewriting axiom $(w_1 \oc a \oc w_2 \reduces w') \in \srsig$.
      The same axiom is contained in the signuare $\srsig, v_1 \oc a \oc v_2 \reduces v'$.
    \end{itemize}

  \item
  Consider the case in which 
  \begin{equation*}
    \infer{\chorsig{\theta}{\srsig, w_1 \oc a \oc w_2 \reduces w'}{\orsig, \defp{a} \defd \n{A}}}{
      \text{($\theta(a) = \defp{a}$)} &
      \chorax{\theta}{w_1}{\bigfuse \theta(w')}{w_2}{\n{A}} &
      \chorsig{\theta}{\srsig}{\orsig} &
      \text{($\defp{a} \notin \dom{\orsig}$)}}
  \end{equation*}
  Because $\chorax{\theta}{w_1}{\bigfuse \theta(w')}{w_2}{\n{A}_2}$, it follows from \cref{??} that $\atmR{\octx}_L = \theta(w_1)$ and $\atmL{\octx}_R = \theta(w_2)$ and $\p{C} = \bigfuse \theta(w')$.
  %
  \qedhere
  \end{itemize}
\end{proof}


\begin{theorem}[Soundness]
  If $\chorsig{\theta}{\srsig}{\orsig}$ and $\theta(w) \reduces_{\orsig} \octx'$, then $\octx' = \theta(w')$ for some $w'$ such that $w \reduces_{\srsig} w'$.
\end{theorem}
\begin{proof}
  By induction over the structure of the given ordered rewriting step, $\theta(w) \reduces_{\orsig} \octx'$.
  \begin{itemize}
  \item Consider the case in which
    \begin{equation*}
      \chorsig{\theta}{\srsig}{\orsig}
      \qquad\text{and}\qquad
      \theta(w) = 
      \infer[\jrule{$\reduces$C}\smash{_{\jrule{L}}}]{\octx_1 \oc \octx_2 \reduces_{\orsig} \octx'_1 \oc \octx_2}{
        \octx_1 \reduces_{\orsig} \octx'_1}
      = \octx'
      \,.
    \end{equation*}
    By inversion, $w = w_1 \oc w_2$ for some $w_1$ and $w_2$ such that $\octx_1 = \theta(w_1)$ and $\octx_2 = \theta(w_2)$.
    From the inductive hypothesis, it follows that there exists a string $w'_1$ such that $w_1 \reduces_{\srsig} w'_1$ and $\octx'_1 = \theta(w'_1)$.
    Let $w' = w'_1 \oc w_2$, and notice that $w = w_1 \oc w_2 \reduces_{\srsig} w'_1 \oc w_2 = w'$ and $\octx' = \theta(w'_1) \oc \theta(w_2) = \theta(w')$, as required.

  \item
    The case in which
    \begin{equation*}
      \chorsig{\theta}{\srsig}{\orsig}
      \qquad\text{and}\qquad
      \theta(w) = 
      \infer[\jrule{$\reduces$C}\smash{_{\jrule{R}}}]{\octx_1 \oc \octx_2 \reduces_{\orsig} \octx_1 \oc \octx'_2}{
        \octx_2 \reduces_{\orsig} \octx'_2}
      = \octx'
    \end{equation*}
    is symmetric.

  \item
    Consider the case in which 
    \begin{equation*}
      \chorsig{\theta}{\srsig}{\orsig}
      \qquad\text{and}\qquad
      \theta(w) =
      \infer{\atmR{\octx}_1 \oc \n{A} \oc \atmL{\octx}_2 \reduces_{\orsig} \octx'}{
        \lfocus{\atmR{\octx}_1}{\n{A}}{\atmL{\octx}_2}{\p{C}} &
        \rfocus{\octx'}{\p{C}}}
      \,.
    \end{equation*}
    The image $\theta(w)$ can contain a negative proposition $\n{A}$ only if $w = w_1 \oc a \oc w_2$ for some $w_1$, $a$, and $w_2$ such that $\theta(a) = \defp{a}$ with $(\defp{a} \defd \n{A}) \in \orsig$.
    By inversion, both $\theta(w_1) = \atmR{\octx}_1$ and $\theta(w_2) = \atmL{\octx}_2$ must hold.
    It then follows from \cref{??} that $\p{C} = \bigfuse \theta(w')$ for some string $w'$ such that $(w_1 \oc a \oc w_2 \reduces w') \in \srsig$.
    Because $\rfocus{\octx'}{\bigfuse \theta(w')}$ only if $\octx' = \theta(w')$~\parencref{??}, the string $w'$ is such that $w = w_1 \oc a \oc w_2 \reduces_{\srsig} w'$, with $\octx' = \theta(w')$.
  %
  \qedhere
  \end{itemize}
\end{proof}



\subsection{No choreography}

Not all string rewriting specifications admit a choreography.
For example, the specification
\begin{equation*}
  \infer{a \oc b \reduces b}{}
  \qquad
  \infer{a \reduces \emp}{}
  \qquad\text{and}\qquad
  \infer{b \reduces \emp}{}
\end{equation*}
cannot be given a choreography.
More precisely, there is no choreographing assignment $\theta$ such that $\chorsig{\theta}{\sig}{\sig'}$ is derivable for some signature $\sig'$.
For the sake of contradiction, suppose that $\theta$ were such a choreographing assignment.
Then, for the specification's latter two axioms to be choreographable, both $\theta(a) = \proc{a}$ and $\theta(b) = \proc{b}$ must hold.
In that case, however, the specification's first axiom cannot be choreographed properly because $\theta$ maps more than one of the axiom's symbols to recursively defined propositions.


\section{Encoding \aclp*{NFA}}

Recall from \cref{??} our string rewriting specification of how \iac{NFA} processes its input.
Given \iac{NFA} $\aut{A} = (Q, ?, F)$ over an input alphabet $\ialph$, the \ac{NFA}'s operational semantics are adequately captured by the folllwing string rewriting axioms:
\begin{equation*}
  \infer{a \oc q \reduces q'_a}{}
  \enspace\text{for each transition $q \nfareduces[a] q'_a$.}
\end{equation*}
\begin{equation*}
  \infer{\emp \oc q \reduces F(q)}{}
  \enspace\text{for each state $q$, where}\enspace
  F(q) = \begin{cases*}
           (\octxe) & if $q \in F$ \\
           \symrej & if $q \notin F$\,.
         \end{cases*}
\end{equation*}

\subsection{A functional choreography}

One possible choreography for this specification treats the input symbols $a \in \ialph$ as atomic propositions $\atmR{a}$; states $q \in Q$ as recursively defined propostions $\proc{q}$;and the end-of-word marker $\emp$ as an atomic proposition $\atmR{\emp}$.
In other words, the \ac{NFA}'s input is treated as a sequence of messages, $\atmR{\emp} \oc \atmR{a}_n \dotsm \atmR{a}_2 \oc \atmR{a}_1$, and the \ac{NFA}'s states are treated as [recursive] processes.

$a \mapsto \atmR{a}$ for all $a \in \ialph$; $q \mapsto \proc{q}$ for all $q \in Q$; and $\emp \mapsto \atmR{\emp}$.

Using this assignment, the choreography constructed from the specification consists of the following definition, one for each \ac{NFA} state $q \in Q$:
\begin{equation*}
  \proc{q} \defd \bigwith_{a \in \ialph} \bigwith_{q\smash{'_a}} (\atmR{a} \limp \proc{q}'_a) \with (\atmR{\emp} \limp \nfa{F}(q))
  \,.
\end{equation*}

\begin{corollary}
  If $q \nfareduces[a] q'_a$, then $\atmR{a} \oc \proc{q} \Reduces \proc{q}'_a$.
  If $\atmR{a} \oc \proc{q} \reduces \octx'$, then $\octx' \Reduces \proc{q}'_a$ for some $q'_a$ that $a$-succeeds $q$.
  If $q \nfareduces[a] q'_a$, then $\atmR{a} \oc \proc{q} \Reduces \proc{q}'_a$.
  If $\atmR{a} \oc \proc{q} \reduces \octx'$, then $\octx' \Reduces \proc{q}'_a$ for some $q'_a$ that $a$-succeeds $q$.
\end{corollary}

\begin{corollary}
  If $q \nfareduces[a] q'_a$, then $\atmR{a} \oc \proc{q} \Reduces \proc{q}'_a$.
  If $\atmR{a} \oc \proc{q} \reduces \octx'$, then $\octx' = \proc{q}'_a$ for some $q'_a$ that $a$-succeeds $q$.
\end{corollary}


\begin{theorem}
  Let $\aut{A} = (...)$ be \iac{DFA} over the input alphabet $\ialph$.
  $q \asim s$ if, and only if, $\proc{q} = \proc{s}$, for all states $q$ and $s$.
\end{theorem}
\begin{proof}
  ...
\end{proof}

But the same does not hold for \acp{NFA}.
\begin{proof}[Counterexample]
  ...
\end{proof}

\subsection{An object-oriented choreography}

\begin{equation*}
  \begin{lgathered}
    \proc{a} \defd \bigwith_{q \in Q} \bigwith_{q\smash{'_a}} (\atmL{q}'_a \pmir \atmL{q}) \\
    \proc{\emp} \defd \bigwith_{q \in Q} (\nfa{F}(q) \pmir \atmL{q})
    \,.
  \end{lgathered}
\end{equation*}

\begin{corollary}
  If $q \nfareduces[a] q'_a$, then $\proc{a} \oc \atmL{q} \Reduces \atmL{q}'_a$.
  If $\proc{a} \oc \atmL{q} \reduces \octx'$, then $\octx' \Reduces \atmL{q}'_a$ for some $q'_a$ that $a$-succeeds $q$.
\end{corollary}

Define a relation on input symbols $a$ and $b$ such that $\proc{a} = \proc{b}$.
Two symbols are then related exactly when they lead to the same successor states.

\section{Binary counters}

\subsection{An object-oriented choreography}

\begin{equation*}
  \begin{lgathered}
    \proc{e} \defd (\proc{e} \fuse \proc{b}_1 \pmir \atmL{i}) \with (\atmR{z} \pmir \atmL{d}) \\
    \proc{b}_0 \defd (\up \dn \proc{b}_1 \pmir \atmL{i}) \with (\atmL{d} \fuse \proc{b}'_0 \pmir \atmL{d}) \\
    \proc{b}_1 \defd (\atmL{i} \fuse \proc{b}_0 \pmir \atmL{i}) \with (\proc{b}_0 \fuse \atmR{s} \pmir \atmL{d}) \\
    \proc{b}'_0 \defd (\atmR{z} \limp \atmR{z}) \with (\atmR{s} \limp \proc{b}_1 \fuse \atmR{s})
  \end{lgathered}
\end{equation*}


Is the following result what we want?
\begin{corollary}
  If $\ainc{w}{n}$, then $\theta(w) \Reduces \theta(w')$ and $\aval{w'}{n}$, for some $w'$.
  If $\ainc{w}{n}$ and $\theta(w) \Reduces \octx'$, then $\octx' = \theta(w')$ and $\ainc{w'}{n}$, for some $w'$.
\end{corollary}


\begin{equation*}
  \begin{lgathered}
    \proc{e} \defd (\proc{e} \fuse \proc{b}_1 \pmir \atmL{i}) \with (\proc{z} \pmir \atmL{d}) \\
    \proc{b}_0 \defd (\up \dn \proc{b}_1 \pmir \atmL{i}) \with (\atmL{d} \fuse \atmL{b}'_0 \pmir \atmL{d}) \\
    \proc{b}_1 \defd (\atmL{i} \fuse \proc{b}_0 \pmir \atmL{i}) \with (\proc{b}_0 \fuse \proc{s} \pmir \atmL{d}) \\
    \proc{z} \defd \up \dn \proc{z} \pmir \atmL{b}'_0 \\
    \proc{s} \defd \proc{b}_1 \fuse \proc{s} \pmir \atmL{b}'_0
  \end{lgathered}
\end{equation*}

\begin{verbatim}
  bin = &{ i: bin, d: hdun }
  hdun = &{ x: +{ z: 1, s: bin } }

  1 |- e : bin
  bin |- b0 : bin
  bin |- b1 : bin
  1 |- z : hdun
  bin |- s : hdun
\end{verbatim}

\begin{equation*}
  \begin{lgathered}
    \proc{z} \defd (\up \dn \proc{z} \pmir \atmL{b}'_0) \with (\atmR{z} \pmir \atmL{x}) \\
    \proc{s} \defd (\proc{b}_1 \fuse \proc{s} \pmir \atmL{b}'_0) \with (\atmR{s} \pmir \atmL{x})
  \end{lgathered}
\end{equation*}


\subsection{A functional choreography}

\begin{equation*}
  \begin{lgathered}
    \proc{\imath} \defd (\atmR{e} \limp \atmR{e} \fuse \atmR{b}_1) \with (\atmR{b}_0 \limp \atmR{b}_1) \with (\atmR{b}_1 \limp \proc{\imath} \fuse \atmR{b}_0) \\
    \proc{d} \defd (\atmR{e} \limp \up \dn \proc{z}) \with (\atmR{b}_0 \limp \proc{d} \fuse \atmL{b}'_0) \with (\atmR{b}_1 \limp \atmR{b}_0 \fuse \proc{s}) \\
    \proc{z} \defd \up \dn \proc{z} \pmir \atmL{b}'_0 \\
    \proc{s} \defd \atmR{b}_1 \fuse \proc{s} \pmir \atmL{b}'_0
  \end{lgathered}
\end{equation*}


\begin{equation*}
  \begin{lgathered}
    \proc{\imath} \defd (\atmR{e} \limp \atmR{e} \fuse \atmR{b}_1) \with (\atmR{b}_0 \limp \atmR{b}_1) \with (\atmR{b}_1 \limp \proc{\imath} \fuse \atmR{b}_0) \\
    \proc{d} \defd (\atmR{e} \limp \atmR{z}) \with (\atmR{b}_0 \limp \proc{d} \fuse \proc{b}'_0) \with (\atmR{b}_1 \limp \atmR{b}_0 \fuse \atmR{s}) \\
    \proc{b}'_0 \defd (\atmR{z} \limp \atmR{z}) \with (\atmR{s} \limp \proc{b}_1 \fuse \atmR{s})
  \end{lgathered}
\end{equation*}

%%% Local Variables:
%%% mode: latex
%%% TeX-master: "thesis"
%%% End:
