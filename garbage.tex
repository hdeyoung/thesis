\chapter{GARBAGE?}

\section{}

In \cref{ch:string-rewriting}, we saw that string rewriting can be used to specify the dynamics of concurrent systems, but that those specifications are quite abstract.
Even the operational semantics is left completely abstract: permitted rewritings just \emph{happen}, as if a central, meta-level actor schedules and otherwise coordinates rewriting.

In the previous \lcnamecref{ch:ordered-rewriting}, we presented a different rewriting framework, this one derived from the ordered sequent calculus and closely related to the \citeauthor{Lambek:AMM58} calculus\autocite{Lambek:AMM58}.
Ordered rewriting, too, leaves the [...] completely abstract

At this high level of abstraction, string rewriting specifications are not amenable to 

In this and the previous \lcnamecrefs{ch:ordered-rewriting,ch:formula-as-process}, we have also seen that the formula-as-process ordered rewriting framework permits only those rewritings that have a sensible interpretation under local, message-passing communication.
Thus far, we have seen that the formula-as-process ordered rewriting framework precludes rewritings, such as $\atmL{a} \oc $, that are not sensible in [...].


Given a string rewriting alphabet $\sralph$, a mapping $\theta\colon \finwds{\sralph} \to {?}$ is a \emph{role assignment for $\sralph$} if it is a monoid homomorphism between finite strings and ordered contexts that uniquely casts each symbol $a \in \sralph$ in the role of either: an atom, $\atmL{a}$ or $\atmR{a}$; or of a recursively defined proposition, $\defp{a}$.

A pair $(\theta, \orsig)$ is a \emph{choreography} of the string rewriting specification $(\sralph, \srsig)$ if:
\begin{itemize}
\item $\theta \colon \finwds{\sralph} \to ?$ is a role assignment for $\sralph$;
\item $\orsig$ is a formula-as-process signature that provides definitions for each of the recursively defined propositions that appear in the image of $\theta$; and
\item 
  $\theta$ is a (strong) bisimulation between $\reduces_{\srsig}$ and $\reduces_{\orsig}$, the string rewriting and (formula-as-process) rewriting relations:
\begin{equation*}
  \begin{tikzcd}
    w \rar[reduces, subscript=\srsig] \dar[relation][swap]{\theta}
     & w\mathrlap{'} \dar[relation, exists]{\theta}
    \\
    \octx \rar[reduces, exists, subscript=\orsig]
     & \octx\mathrlap{'}
  \end{tikzcd}
  \hphantom{'}
  \quad\text{and}\quad
  \begin{tikzcd}
    w \rar[reduces, exists, subscript=\srsig] \dar[relation][swap]{\theta}
     & w\mathrlap{'} \dar[relation, exists]{\theta}
    \\
    \octx \rar[reduces, subscript=\orsig]
     & \octx\mathrlap{' \,.}
  \end{tikzcd}
  \hphantom{' \,.}
\end{equation*}
\end{itemize}


Using the formula-as-process ordered rewriting as a substrate, we would like to choreograph string rewriting specifications $(\sralph, \srsig)$ by mapping them to formula-as-process ordered rewriting.
Specifically, we would like to find a binary relation $\simu{R}$ between strings and ordered contexts and an ordered rewriting signature $\orsig$ such that $\simu{R}$ is a (strong) bisimulation between $\reduces_{\srsig}$ and $\reduces_{\orsig}$, the string rewriting and (formula-as-process) ordered rewriting relations.
\begin{equation*}
  \begin{tikzcd}
    w \rar[reduces, subscript=\srsig] \dar[relation][swap]{\simu{R}}
     & w\mathrlap{'} \dar[relation, exists]{\simu{R}}
    \\
    \octx \rar[reduces, exists, subscript=\orsig]
     & \octx\mathrlap{'}
  \end{tikzcd}
  \hphantom{'}
  \quad\text{and}\quad
  \begin{tikzcd}
    w \rar[reduces, exists, subscript=\srsig] \dar[relation][swap]{\simu{R}}
     & w\mathrlap{'} \dar[relation, exists]{\simu{R}}
    \\
    \octx \rar[reduces, subscript=\orsig]
     & \octx\mathrlap{' \,.}
  \end{tikzcd}
  \hphantom{' \,.}
\end{equation*}
\footnote{A relation $\simu{R}$ such that:
  $\octx \simu{R}^{-1} w \reduces_{\srsig} w'$ implies $\octx \reduces_{\orsig} \octx' \simu{R}^{-1} w'$ for some $\octx'$; and $w \simu{R} \octx \reduces_{\orsig} \octx'$ implies $w \reduces_{\srsig} w' \simu{R} \octx'$ for some $w'$.}



Because the formula-as-process ordered rewriting framework precludes rewritings that [...], this choreographing operationalizes string rewriting.

Given a string rewriting specification $\srsig$, we would like to find an ordered rewriting signature $\orsig$ that mimics 

In other words, More specifically, the relation $\simu{R}$ will be a monoid homomorphism so that 


Not all string rewriting specifications have a valid choreography.
For instance, the string rewriting specification $(\sralph, \srsig)$ where
\begin{equation*}
  \begin{lgathered}
    \sralph = \Set{a, b} \\
    \srsig = (a \wc b \reduces b) \,, (b \reduces \emp) \,, (a \reduces \emp)
  \end{lgathered}
\end{equation*}
has no valid choreography.
Suppose that $\theta$ were a role assignment that led to a valid choreography, $(\theta, \orsig)$.
For the constraints $\theta(b) \reduces_{\orsig} (\octxe)$ and $\theta(a) \reduces_{\orsig} (\octxe)$ to be satisfiable, $\theta$ would have to map $b \mapsto \defp{b}$ and $a \mapsto \defp{a}$.
However, the first axiom would then induce the constraint $\defp{a} \oc \defp{b} \reduces_{\orsig} \defp{b}$, which is not satisfiable -- there are no definitions for \fixnote{Except, there are: $\defp{a} \defd \up \one$.}

Our goal is not to synthesize a choreography from scratch for a given string rewriting specification, $(\sralph, \srsig)$.
Instead, our goal is to synthesize a (formula-as-process) ordered rewriting signature from \emph{a role assignment $\theta$} for a given string rewriting specification.


Given a string rewriting specification $(\sralph, \srsig)$ and a role assignment $\theta\colon \finwds{\sralph} \to \atmL{\sralph} \cup \atmR{\sralph} \cup \defp{\sralph}$\fixnote{fix}, we would like to determine whether $\theta$ gives rise to a meaningful choreography of $(\sralph, \srsig)$.
That is, we would to construct, if possible, an ordered rewriting signature $\orsig$ that makes $\theta$ a (strong) bisimulation between the string rewriting and formula-as-process ordered rewriting relations, $\reduces_{\srsig}$ and $\reduces_{\orsig}$, respectively.
% For this we will define a judgment $\chorsig{\theta}{\srsig}{\orsig}$ such that $\chorsig{\theta}{\srsig}{\orsig}$ implies
\begin{equation*}
  \begin{tikzcd}
    w \rar[reduces, subscript=\srsig] \dar[relation][swap]{\theta}
     & w\mathrlap{'} \dar[relation, exists]{\theta}
    \\
    \octx \rar[reduces, exists, subscript=\orsig]
     & \octx\mathrlap{'}
  \end{tikzcd}
  \hphantom{'}
  \quad\text{and}\quad
  \begin{tikzcd}
    w \rar[reduces, exists, subscript=\srsig] \dar[relation][swap]{\theta}
     & w\mathrlap{'} \dar[relation, exists]{\theta}
    \\
    \octx \rar[reduces, subscript=\orsig]
     & \octx\mathrlap{' \,.}
  \end{tikzcd}
  \hphantom{' \,.}
\end{equation*}
We will first explain by example how such a signature $\orsig$ is constructed, reserving a formal description of the choreographing procedure to \cref{??}.

\newthought{Recall} from \cref{ch:string-rewriting} the string rewriting specification of a system that can rewrite strings over $\sralph = \Set{a,b}$ into the empty string if the initial string ends in $b$;
that specification consists of the axioms
\begin{equation*}
  \srsig = (a \wc b \reduces b) , (b \reduces \emp)
  \,.
\end{equation*}
The monoid homomorphism $\theta$ such that $\theta(a) = \atmR{a}$ and $\theta(b) = \defp{b}$ is a role assignment for this specification.

We can apply the role assignment $\theta$ to the axioms $\srsig$ to see which ordered rewritings must hold of the relation $\reduces_{\orsig}$ if $(\theta, \orsig)$ is to be a meaningful choreography of the specification $(\sralph, \srsig)$.
In this example, the axioms $\srsig$ together with $\theta$ induce the rewritings
\begin{equation*}
  \theta(a \wc b) = \atmR{a} \oc \defp{b} \reduces_{\orsig} \defp{b} = \theta(b)
  \quad\text{and}\quad
  \theta(b) = \defp{b} \reduces_{\orsig} (\octxe) = \theta(\emp)
  \,.
\end{equation*}


\begin{equation*}
  \begin{tikzcd}
    a \wc b \rar[reduces, subscript=\srsig] \dar[relation][swap]{\theta}
     & b \dar[relation]{\theta}
    \\
    \atmR{a} \oc \defp{b} \rar[reduces, subscript=\orsig]
     & \defp{b}
  \end{tikzcd}
  \quad\text{and}\quad
  \begin{tikzcd}
    b \rar[reduces, subscript=\srsig] \dar[relation][swap]{\theta}
     & \emp \dar[relation]{\theta}
    \\
    \defp{b} \rar[reduces, subscript=\orsig]
     & (\octxe)
  \end{tikzcd}
\end{equation*}
So, to find a meaningful choreography $(\theta, \orsig)$ for the string rewriting specification $(\sralph, \srsig)$, it suffices to find a signature $\orsig$ for which the rewritings $\atmR{a} \oc \defp{b} \reduces_{\orsig} \defp{b}$ and $\defp{b} \reduces_{\orsig} (\octxe)$ -- and only those rewritings -- are derivable.
In other words, $\atmR{a} \oc \defp{b} \reduces_{\orsig} \defp{b}$ and $\defp{b} \reduces_{\orsig} (\octxe)$ serve as constraints on $\orsig$ that we must solve.

To solve these constraints, we must find a definition for $\defp{b}$ that makes those -- and only those -- rewritings derivable.
In this instance, such a solution is the definition $\defp{b} \defd (\atmR{a} \limp \up \dn \defp{b}) \with \up \one$ and the corresponding signature, $\orsig \defd \bigl(\defp{b} \defd (\atmR{a} \limp \up \dn \defp{b}) \with \up \one\bigr)$.
Here is how we arrive at that solution:
\begin{itemize}
\item Let's temporarily restrict our attention to the constraint $\atmR{a} \oc \defp{b} \reduces_{\orsig} \defp{b}$.
  Notice that $\atmR{a} \oc (\atmR{a} \limp \up \dn \defp{b}) \reduces \defp{b}$.
  By the universal property of left-handed implication, there must exist an open derivation of $\lfocus{\atmR{a}}{\defp{b}}{}{\p{C}_1}$ from [...].
\item 
  Turning our attention to the constraint $\defp{b} \reduces_{\orsig} (\octxe)$, notice that $\up \one \reduces (\octxe)$.
  By the universal properties of $\up \one$, there must exist an open derivation of $\lfocus{\atmR{\octx}_L}{\defp{b}}{\atmL{\octx}_R}{\p{C}}$ from $\lfocus{\atmR{\octx}_L}{\up \one}{\atmL{\octx}_R}{\p{C}}$.
\end{itemize}
The least proposition $\defp{b}$ that has both of these open derivations is $\defp{b} \defd (\atmR{a} \limp \up \dn \defp{b}) \with \up \one$.

More generally, suppose that we have a constraint $\atmR{\octx}_L \oc \defp{a} \oc \atmL{\octx}_R \reduces_{\orsig} \octx'$.
Then notice that (morally) $\atmR{\octx}_L \oc (\atmR{\octx}_L \limp (\up \bigfuse \octx') \pmir \atmL{\octx}_R) \oc \atmL{\octx}_R \reduces \octx'$.
By the universal propeties of $\atmR{\octx}_L \limp (\up \bigfuse \octx') \pmir \atmL{\octx}_R$, there must exist an open derivation of $\lfocus{\atmR{\octx}_L}{\defp{a}}{\atmL{\octx}_R}{\p{C}}$ from $\lfocus{\atmR{\octx}_L}{\atmR{\octx}_L \limp (\up \bigfuse \octx') \pmir \atmL{\octx}_R}{\atmL{\octx}_R}{\p{C}}$.

Returning to our running example, we need to find a definition for $\defp{b}$ such that both $\atmR{a} \oc \defp{b} \reduces \defp{b}$ and $\defp{b} \reduces (\octxe)$ will be derivable.
By inversion, these induced rewritings will be derivable exactly when both
\begin{gather*}
  \lfocus{\atmR{a}}{\defp{b}}{}{\p{C}_1} \text{ for some $\p{C}_1$ such that } \rfocus{\defp{b}}{\p{C}_1}
\shortintertext{and}
  \lfocus{}{\defp{b}}{}{\p{C}_2} \text{ for some $\p{C}_2$ such that } \rfocus{(\octxe)}{\p{C}_2}
  \,.
\end{gather*}
It is easy to check
\begin{enumerate*}[label=\emph{(\roman*)}]
\item that the first condition would be satisfied if $\defp{b}$ were $\atmR{a} \limp \up \dn \defp{b}$; and
\item that the second condition would be satisfied if $\defp{b}$ were $\up \one$.
\end{enumerate*}
If $\defp{b}$ were somehow simultaneously both $\atmR{a} \limp \up \dn \defp{b}$ and $\up \one$, then both conditions would be satisfied.
Fortunately, additive conjunction allows us to do just that: when $\defp{b} \defd (\atmR{a} \limp \up \dn \defp{b}) \with \up \one$, the induced rewritings, $\atmR{a} \oc \defp{b} \reduces \defp{b}$ and $\defp{b} \reduces (\octxe)$ -- and only those rewritings -- are derivable.
$\octx_L \oc \defp{b} \oc \octx_R \reduces \octx'$ only if either 
\begin{itemize}
\item $\octx_L \reduces \octx'_L$ and $\octx' = \octx'_L \oc \defp{b} \oc \octx_R$ for some $\octx'_L$;
\item $\octx' = \octx_L \oc \octx_R$;
\item $\octx_L = \octx'_L \oc \atmR{a}$ and $\octx' = \octx'_L \oc \defp{b} \oc \octx_R$; or 
\item $\octx_R \reduces \octx'_R$ and $\octx' = \octx_L \oc \defp{b} \oc \octx'_R$ for some $\octx'_R$.
\end{itemize}





To choreograph a string rewriting specification, we would like to assign one, and only one, role to each symbol $a \in \sralph$: in the choreography, each symbol $a$ becomes either a message, $\atmL{a}$ or $\atmR{a}$, or a recursively defined process, $\defp{a}$.
A monoid homomorphism\fixnote{isomorphism?} from strings to ordered contexts that satisfies this condition is called a \vocab{[...] assignment}.


When applied to the specification's axioms, the [...] assignment $\theta$ induces the rewriting steps
\begin{equation*}
  \atmR{a} \oc \defp{b} \reduces \defp{b}
  \quad\text{and}\quad
  \defp{b} \reduces (\octxe)
  \,,
\end{equation*}
which we denote by $\theta(\srsig)$.

For the [...] assignment $\theta$ to yield an actual choreography of the axioms $\srsig$, we must be able to solve these induced rewritings for $\defp{b}$, determining a definition for $\defp{b}$ that makes these -- and only these -- rewriting steps derivable.

More generally, a [...] assignment $\theta$ yields a well-specified choreography for a specification with axioms $\srsig$ if the induced ordered rewriting steps $\theta(\srsig)$ are solvable with definitions for all recursively defined processes that make the induced rewritings $\theta(\srsig)$ -- and only those rewritings -- derivable.
In other words, $\theta$

Returning to our running example, we need to find a definition for $\defp{b}$ such that both $\atmR{a} \oc \defp{b} \reduces \defp{b}$ and $\defp{b} \reduces (\octxe)$ will be derivable.
By inversion, these induced rewritings will be derivable exactly when both
\begin{gather*}
  \lfocus{\atmR{a}}{\defp{b}}{}{\p{C}_1} \text{ for some $\p{C}_1$ such that } \rfocus{\defp{b}}{\p{C}_1}
\shortintertext{and}
  \lfocus{}{\defp{b}}{}{\p{C}_2} \text{ for some $\p{C}_2$ such that } \rfocus{(\octxe)}{\p{C}_2}
  \,.
\end{gather*}
It is easy to check
\begin{enumerate*}[label=\emph{(\roman*)}]
\item that the first condition would be satisfied if $\defp{b}$ were $\atmR{a} \limp \up \dn \defp{b}$; and
\item that the second condition would be satisfied if $\defp{b}$ were $\up \one$.
\end{enumerate*}
If $\defp{b}$ were somehow simultaneously both $\atmR{a} \limp \up \dn \defp{b}$ and $\up \one$, then both conditions would be satisfied.
Fortunately, additive conjunction allows us to do just that: when $\defp{b} \defd (\atmR{a} \limp \up \dn \defp{b}) \with \up \one$, the induced rewritings, $\atmR{a} \oc \defp{b} \reduces \defp{b}$ and $\defp{b} \reduces (\octxe)$ -- and only those rewritings -- are derivable.
$\octx_L \oc \defp{b} \oc \octx_R \reduces \octx'$ only if either 
\begin{itemize}
\item $\octx_L \reduces \octx'_L$ and $\octx' = \octx'_L \oc \defp{b} \oc \octx_R$ for some $\octx'_L$;
\item $\octx' = \octx_L \oc \octx_R$;
\item $\octx_L = \octx'_L \oc \atmR{a}$ and $\octx' = \octx'_L \oc \defp{b} \oc \octx_R$; or 
\item $\octx_R \reduces \octx'_R$ and $\octx' = \octx_L \oc \defp{b} \oc \octx'_R$ for some $\octx'_R$.
\end{itemize}


Not all [...] assignments yield well-specified choreographies.
This happens when there is no solution for the recursively defined propositionsthat makes all of the induced rewritings derivable.
\begin{itemize}
\item
  \emph{Each induced rewriting must have at least one process in its premise\fixnote{wc}.}
  For example, the [...] assignments $\theta'$ such that either $\theta'(b) = \atmL{b}$ or $\theta'(b) = \atmR{b}$ holds do \emph{not} yield well-specified choreographies.
  From the string rewriting axiom $b \reduces \emp$, the [...] assignment $\theta'$ induces either $\atmL{b} \reduces (\octxe)$ or $\atmR{b} \reduces (\octxe)$, and there is no solution that makes either of these induced ordered rewritings derivable.

\item
  \emph{Each induced rewriting must have at most one process in its premise\fixnote{wc}.}
  For example, the [...] assignment $\theta'$ such that $\theta'(a) = \defp{a}$ and $\theta'(b) = \defp{b}$ hold does not yield a well-specified choreography because there is no solution that makes $\defp{a} \oc \defp{b} \reduces \defp{b}$ derivable.

\item
  \emph{Each message in the premises of induced rewritings must be flowing toward that premise's process\fixnote{wc}.}
  For example, the [...] assignment $\theta'$ such that $\theta'(a) = \atmL{a}$ and $\theta'(b) = \defp{b}$ hold does not yield a well-specified choreography because there is no solution that makes $\atmL{a} \oc \defp{b} \reduces \defp{b}$ derivable.
  In \ac{PFOR} there is no process $\defp{b}$ that can receive a message, like $\atmL{a}$, that is flowing away.
\end{itemize}


For the choreography to be well-specified, this [...] assignment must induce from the string rewriting specification's axioms a collection of locally achievable  ordered rewriting steps\fixnote{reductions}.
If the ordered rewriting steps induced by the [...] assignment cannot be achieved by local communication, then the choreography is not well-specified.

For example, recall from \cref{ch:string-rewriting} the string rewriting specification of a system that can rewrite strings over $\sralph = \Set{a,b}$ into the empty string if the initial string ends in $b$;
that specification used axioms
\begin{equation*}
  \srsig = (a \wc b \reduces b) , (b \reduces \emp)
  \,.
\end{equation*}

So, to choreograph this specification, we must choose an assignment of roles -- either message or process -- to symbols $a$ and $b$ --
let's choose $a \mapsto \atmR{a}$ and $b \mapsto \defp{b}$.
From the axioms $\srsig$, this assignment induces the rewritings
\begin{equation*}
  \atmR{a} \oc \defp{b} \reduces \defp{b}
  \quad\text{and}\quad
  \defp{b} \reduces (\octxe)
  \,.
\end{equation*}
Are these reductions achievable by purely local communication?
Because our formula-as-process interpretation of ordered rewriting ensures that all communication is local, we need only verify that there is a solution for $\defp{b}$ [...].

Any solution for $\defp{b}$ must be consistent with $\atmR{a} \limp \up \dn \defp{b}$ so that $\atmR{a} \oc \defp{b} \reduces \defp{b}$ is derivable.
Furthermore, any solution for $\defp{b}$ must be consistent with $\up \one$ so that $\defp{b} \reduces \octxe$ is derivable.
The least such solution is
\begin{equation*}
  \defp{b} \defd (\atmR{a} \limp \up \dn \defp{b}) \with \up \one
  \,,
\end{equation*}
It indeed validates the required reductions,
\begin{gather*}
  \atmR{a} \oc \defp{b} = \atmR{a} \oc \bigl((\atmR{a} \limp \up \dn \defp{b}) \with \up \one\bigr) \reduces \defp{b} \\
  \defp{b} = \atmR{a} \oc \bigl((\atmR{a} \limp \up \dn \defp{b}) \with \up \one\bigr) \reduces (\octxe)
  \,,
\end{gather*}
and only the required reductions:
\begin{quotation}
  If $\octx_L \oc \defp{b} \oc \octx_R \reduces \octx'$, then either:
  \begin{itemize}
  \item $\octx_L = \octx'_L \oc \atmR{a}$ and $\octx' = \octx'_L \oc \defp{b} \oc \octx_R$, for some $\octx'_L$;
  \item $\octx' = \octx_L \oc \octx_R$;
  \item $\octx_L \reduces \octx'_L$ and $\octx' = \octx'_L \oc \defp{b} \oc \octx_R$, for some $\octx'_L$; or
  \item $\octx_R \reduces \octx'_R$ and $\octx' = \octx_L \oc \defp{b} \oc \octx'_R$, for some $\octx'_R$.
  \end{itemize}
\end{quotation}

\begin{gather*}
  \atmL{a} \oc \defp{b} \reduces \defp{b}
  \quad\text{and}\quad
  \defp{b} \reduces (\octxe)
  \\
  \defp{a} \oc \defp{b} \reduces \defp{b}
  \quad\text{and}\quad
  \defp{b} \reduces (\octxe)
  \\
  \defp{a} \oc \atmL{b} \reduces \atmL{b}
  \quad\text{and}\quad
  \atmL{b} \reduces (\octxe)
\end{gather*}



To be well-specified, 

An \emph{[...] assignment} $\theta$ is a monoid homomorphism from strings to ordered contexts that injectively maps each symbol $a \in \sralph$ to either a message, $\atmL{a}$ or $\atmR{a}$, or a recursively defined process, $\defp{a}$.%
\footnote{Injectivity keeps $\theta$ from identifying distinct symbols.}

Given an [...] assignment $\theta$, a string rewriting specification's axioms induce rewriting steps that must hold if the specification is to have a choreography.
For each axiom $w \reduces w' \in \srsig$, the [...] assignment $\theta$ induces a requirement that a faithful choreography must satisfy the rewriting step $\theta(w) \reduces \theta(w')$.

\section{Constructing a choreography from a specification}

For an example of this procedure, let's construct a choreography for the string rewriting specification of the system from \cref{ch:string-rewriting} that can rewrite strings over $\ialph = \Set{a,b}$ into the empty string.
Recall that that specification consisted of the axioms
\begin{equation*}
  \srsig = (a \wc b \reduces b) , (b \reduces \emp)
  \,.
\end{equation*}


The first step in constructing a choreography is to choose a \emph{[...] assignment} that maps each symbol to either an atom or recursively defined proposition, [which represent a message or recursively defined process,respectively.]
For example, $\theta = \Set{a \mapsto \atmR{a} , b \mapsto \defp{b}}$ is an [...] assignment that maps $a$ to a right-directed message and $b$ to a process.
Like $\theta$, all [...] assignments must be injective, to keep distinct symbols from becoming identified in the choreography.

Next, we apply the [...] assignment to each of the string rewriting specification's axioms and simultaneously replace the empty string with the empty ordered context.\footnote{Strictly speaking, the monoid operations are also exchanged, but because both are indicated by juxtaposition, this happens silently.}
This results in a collection of ordered rewriting steps that the choreography must satisfy if it is to be a faithful reflection of the string rewriting specification.
Applying $\theta$ to the axioms of \cref{??} yields 
\begin{equation*}
  \atmR{a} \oc \defp{b} \reduces \defp{b}
  \quad\text{and}\quad
  \defp{b} \reduces \octxe
\end{equation*}
as rewritings required of the choreography.

Finally, we solve for the recursively defined propositions that appear in the required rewritings.
In this example, $\defp{b}$ must be consistent with $\atmR{a} \limp \up \dn \defp{b}$ if $\atmR{a} \oc \defp{b} \reduces \defp{b}$ is to be derivable;
$\defp{b}$ must also be consistent with $\up \one$ if $\defp{b} \reduces \octxe$ is to be derivable.
The least such solution is $\defp{b} \defd (\atmR{a} \limp \up \dn \defp{b}) \with \up \one$.
Indeed, under this definition, 
\begin{equation*}
  \atmR{a} \oc \defp{b} \reduces \defp{b}
  \quad\text{and}\quad
  \defp{b} \reduces \octxe
\end{equation*}
are both derivable.

\clearpage
\subsection{}

Not all [...] assignments yield choreographies.
For instance, suppose we had chosen $\theta' = \Set{a \mapsto \defp{a} , b \mapsto \atmR{b}}$ or any other assignment $\theta'$ that maps $b$ to an atom.
Applying $\theta'$ to the second axiom would yield either $\atmR{b} \reduces \octxe$ or $\atmL{b} \reduces \octxe$ as required rewriting steps.
Neither of these make for a valid choreography both of which require a message to be recognized and acted upon by the ether.


\subsection{}

To construct a choreography, we need to find a \emph{choreographing assignment} that consistently localizes each axiom.

An assignment that maps both $a$ and $b$ to messages, such as $\theta = \Set{a \mapsto \atmR{a} , b \mapsto \atmR{b}}$ which results in $\atmR{a} \oc \atmR{b} \reduces \atmR{b}$  and $\atmR{b} \reduces \octxe$,
 

  
As a string rewriting specification, the axioms are interpreted from a global perspective.
For instance, the first axiom states that when the symbols $a \oc b$ occur in that order, they may be rewritten to $b$.
But the axiom does not describe how that rewriting occurs.

With choreographies, we would like to work at a (slightly) lower level of abstraction to describe 


Suppose that we are given a string rewriting specification that consists of axioms $?$ over the rewriting alphabet $\sralph$.
A \vocab{choreographing assignment} is an injection in which each symbol $a \in \sralph$ is mapped to an ordered proposition: either an atomic proposition, $\atmL{a}$ or $\atmR{a}$, or a recursively defined proposition, $\defp{a}$.

Given a choregraphing assignment $\theta$, we may construct a choregraphy from the string rewriting specification.
Intuitively, each axiom is annotated according to $\theta$, and then the resulting [...] are used to construct a family of recursive definitions, one for each $\defp{a}$ in the image of $\theta$.

A choreography is an ordered rewriting specification that simulates the string rewriting specification [...].

Consider the recurring string rewriting specification with axioms
\begin{equation*}
  \infer{a \oc b \reduces b}{}
  \qquad\text{and}\qquad
  \infer{b \reduces \emp}{}
  \:.
\end{equation*}
We must consistently annotate each symbol as either a left-directed atom, right-directed atom, or recursively defined proposition in sucha way that each axiom's premise $w$ has the form $w_1 \wc a \wc w_2$ with 

\begin{equation*}
  \atmR{a} \oc \defp{b} \reduces \defp{b}
  \qquad\text{and}\qquad
  \defp{b} \reduces \octxe
\end{equation*}

Now we must solve for $\defp{b}$, determining a definition $\defp{b} \defd \n{B}$ such that these two rewriting steps are derivable.
For the first step to be derivable, $\defp{b}$ should have a definition that is consistent with $\atmR{a} \limp \up \dn \defp{b}$, for 
\begin{equation*}
  \atmR{a} \oc (\atmR{a} \limp \up \dn \defp{b}) \reduces \defp{b}
\end{equation*}

Consider the choreographing assignment $\theta$ that maps $a$ to the atom $\atmR{a}$ and $b$ to the recursively defined proposition $\defp{b}$.
Upon annotating the above string rewriting axioms according to $\theta$, we arrive at the [...]
\begin{equation*}
  \atmR{a} \oc \dprop{b} \reduces \dprop{b}
  \qquad\text{and}\qquad
  \dprop{b} \reduces \one
  \,.
\end{equation*}

\begin{equation*}
  \dprop{b} \defd \atmR{a} \limp \up \dn \dprop{b}
  \qquad\text{and}\qquad
  \dprop{b} \defd \up \one
  \,,
\end{equation*}
respectively.
Combining these into a single definition that allows a nondeterministic choice between the two, we have
\begin{equation*}
  \dprop{b} \defd (\atmR{a} \limp \up \dn \dprop{b}) \with \up \one
  \,,
\end{equation*}
or $\dprop{b} \defd (\atmR{a} \limp \up \dprop{b}) \with \one$ if the minimally necessary shifts are elided.

By construction, this choreography is adequate with respect to the specification, in the sense that it can simulate each of the specification's possible steps and vice versa.
\begin{itemize}
\item $w \reduces w'$ only if $\theta(w) \reduces \theta(w')$
  For example, just as the string rewriting specification admits $a \oc b \reduces b$, the ordered rewriting choreography admits
  \begin{equation*}
    \theta(a \oc b) = \atmR{a} \oc \defp{b} = \atmR{a} \oc \bigl((\atmR{a} \limp \up \dn \defp{b}) \with \one\bigr) \reduces \defp{b} = \theta(b)
    \,.
  \end{equation*}

\item $\theta(w) \reduces \octx'$ only if $w \reduces \theta^{-1}(\octx')$
  For example, just as the ordered rewriting choreography admits $\theta(b) = \defp{b} \reduces \octxe$, the string rewriting specification admits $b \reduces \emp = \theta^{-1}(\octxe)$.
\end{itemize}
% The choreography can simulate each of the specification's possible rewriting steps: for example, $\theta(a \oc b) = \atmR{a} \oc \dprop{b} \reduces \dprop{b} = \theta(b)$, just as $a \oc b \reduces b$.
% Conversely, each of the choreography's possible rewriting steps can be simulated by the specification: for example, $\theta^{-1}(\atmR{a} \oc \dprop{b}) = a \oc b \reduces b = \theta^{-1}(\dprop{b})$, just as $\atmR{a} \oc \dprop{b} \reduces \dprop{b}$.

\subsection{Formal description}

In this \lcnamecref{??}, we present a more formal description of the above procedure for choreographing string rewriting specifications.
We define a judgment $\chorsig{\theta}{\srsig}{\orsig}$ that, when given a string rewriting specification $(\sralph, \srsig)$ and [...] assignment $\theta$, yields a formula-as-process ordered rewriting signature $\orsig$ that makes $\theta$ a bisimulation [between $\reduces_{\srsig}$ and $\reduces_{\orsig}$] if such a signature exists:
\begin{equation*}
  \chorsig{\theta}{\srsig}{\orsig}
  \quad\text{implies}\quad
  \begin{tikzcd}
    w \rar[reduces, subscript=\srsig] \dar[relation][swap]{\theta}
     & w\mathrlap{'} \dar[relation, exists]{\theta}
    \\
    \octx \rar[reduces, exists, subscript=\orsig]
     & \octx\mathrlap{'}
  \end{tikzcd}
  \hphantom{'}
  \quad\text{and}\quad
  \begin{tikzcd}
    w \rar[reduces, exists, subscript=\srsig] \dar[relation][swap]{\theta}
     & w\mathrlap{'} \dar[relation, exists]{\theta}
    \\
    \octx \rar[reduces, subscript=\orsig]
     & \octx\mathrlap{' \,.}
  \end{tikzcd}
  \hphantom{' \,.}
\end{equation*}
This principal judgment
% is $\chorsig{\theta}{\srsig}{\orsig}$, and it
also relies on an auxiliary judgment, $\qimp{\atmR{\octx}_L}{\up \p{A}}{\atmL{\octx}_R}{\n{B}}$.
% Before giving the rules for the principal judgment, we will [...].

\newthought{The auxiliary} judgment $\qimp{\atmR{\octx}_L}{\up \p{A}}{\atmL{\octx}_R}{\n{B}}$ elaborates\fixnote{word choice?} the quasi-propo\-si\-tion $\atmR{\octx}_L \limp \up \p{A} \pmir \atmL{\octx}_R$ into a well-formed proposition $\n{B}$ by nondeterministically abstracting one-by-one from either the left or right contexts.%
\footnote{This procedure could be made deterministic by preferring one side over the other, but we refrain from doing so because the choice of side to prefer is completely arbitrary.}
This proposition $\n{B}$ is semantically equivalent to the quasi-proposition $\atmR{\octx}_L \limp \up \p{A} \pmir \atmL{\octx}_R$ in the sense that the two intuitively satisfy the same \enquote{left-focus judgments}:
% $\n{B}$ satisfies $\atmR{\octx}_L \oc \n{B} \oc \atmL{\octx}_R \reduces \octx'$ if, and only if, $\rfocus{\octx'}{\p{A}}$.
$\lfocus{\atmR{\lctx}_L}{\n{B}}{\atmL{\lctx}_R}{\p{C}}$ if, and only if, $\atmR{\lctx}_L = \atmR{\octx}_L$ and $\atmL{\lctx}_R = \atmL{\octx}_R$ and $\p{C} = \p{A}$.
(This is proved below as \cref{lem:qimp-correct}.)

The auxiliary elaboration judgment is defined inductively by the following rules.
\begin{inferences}
  \infer[\jrule{$\up$Q}]{\qimp{(\octxe)}{\up \p{A}}{(\octxe)}{\up \p{A}}}{}
  \\
  \infer[\jrule{$\limp$Q}]{\qimp{(\atmR{\octx}_L \oc \p{\atmR{a}})}{\up \p{A}}{\atmL{\octx}_R}{\p{\atmR{a}} \limp \n{B}}}{
    \qimp{\atmR{\octx}_L}{\up \p{A}}{\atmL{\octx}_R}{\n{B}}}
  \and
  \infer[\jrule{$\pmir$Q}]{\qimp{\atmR{\octx}_L}{\up \p{A}}{(\p{\atmL{a}} \oc \atmL{\octx}_R)}{\n{B} \pmir \p{\atmL{a}}}}{
    \qimp{\atmR{\octx}_L}{\up \p{A}}{\atmL{\octx}_R}{\n{B}}}
\end{inferences}
The $\jrule{$\limp$Q}$ rule states that the quasi-proposition $(\atmR{\octx}_L \oc \p{\atmR{a}}) \limp \up \p{A} \pmir \atmL{\octx}_R$ is equivalent to $\p{\atmR{a}} \limp \n{B}$ if $\atmR{\octx}_L \limp \up \p{A} \pmir \atmL{\octx}_R$ is equivalent to $\n{B}$.
Notice that the $\jrule{$\limp$Q}$ rule moves $\p{\atmR{a}}\!\!$ from the right of $\atmR{\octx}_L$ to the left of $\n{B}$;
this is admittedly counterintuitive, but it is closely related to the equally counterintuitive currying law for left-handed implication in ordered logic:
% Likewise, the quasi-proposition $\atmR{\octx}_L \limp \up \p{A} \pmir (\p{\atmL{a}} \oc \atmL{\octx}_R)$ is equivalent to $(\atmR{\octx}_L \limp \up \p{A} \pmir \atmL{\octx}_R) \pmir \p{\atmL{a}}$.
$(A \fuse B) \limp C \dashv\vdash B \limp (A \limp C)$.
Symmetrically, the $\jrule{$\pmir$Q}$ rule is closely related to the currying law for right-handed implication: $C \pmir (A \fuse B) \dashv\vdash (C \pmir B) \pmir A$.

This intuition is captured in the proof of the following \lcnamecref{lem:qimp-correct}.
\begin{lemma}\label{lem:qimp-correct}
  % If $\chorax{\theta}{w_1}{\p{C}}{w_2}{\n{B}}$, then $\lfocus{\theta(w_1)}{\n{B}}{\theta(w_2)}{\p{C}}$.
  If $\qimp{\atmR{\octx}_L}{\up \p{A}}{\atmL{\octx}_R}{\n{B}}$, then $\lfocus{\atmR{\lctx}_L}{\n{B}}{\atmL{\lctx}_R}{\p{C}}$ if, and only if, $\atmR{\lctx}_L = \atmR{\octx}_L$ and $\atmL{\lctx}_R = \atmL{\octx}_R$ and $\p{C} = \p{A}$.
  %
  % Moreover, if $\qimp{\atmR{\octx}_L}{\up \p{A}}{\atmL{\octx}_R}{\n{B}}$, then $\atmR{\lctx}_L \oc \n{B} \oc \atmL{\lctx}_R \reduces \lctx'$ if, and only if, there exist contexts $\atmR{\lctx}'_L$, $\lctx'_0$, and$\atmL{\lctx}'_R$ such that $\atmR{\lctx}'_L \oc \atmR{\octx}_L = \atmR{\lctx}_L$ and $\atmL{\octx}_R \oc \atmL{\lctx}'_R = \atmL{\lctx}_R$ and $\rfocus{\lctx'_0}{\p{A}}$ and $\lctx' = \atmR{\lctx}'_L \oc \lctx'_0 \oc \atmL{\lctx}'_R$.
\end{lemma}
\begin{proof}
  By induction over the structure of the given elaboration.

  As an example case, consider
  % \begin{itemize}
  % \item Consider the case in which
  %   \begin{equation*}
  %     \infer{\chorax{\theta}{\emp}{\p{A}}{\emp}{\up \p{A}}}{}
  %     \,.
  %   \end{equation*}
  %   We must show that $\lfocus{\atmR{\octx}_L}{\up \p{A}}{\atmL{\octx}_R}{\p{C}}$ if, and only if, $\atmR{\octx}_L = \atmL{\octx}_R = \theta(\emp)$ and $\p{A} = \p{C}$.
  %   Indeed, the $\lrule{\up}$ rule is the unique rule for left-focusing on $\up \p{A}$, and $\octxe = \theta(\emp)$ because $\theta$ is a monoid homomorphism.
  % 
  % \item Consider the case in which
    \begin{equation*}
      \infer{\qimp{(\atmR{\octx}_L \oc \p{\atmR{a}})}{\up \p{A}}{\atmL{\octx}_R}{\p{\atmR{a}} \limp \n{B}}}{
        \qimp{\atmR{\octx}_L}{\up \p{A}}{\atmL{\octx}_R}{\n{B}}}
      \,.
    \end{equation*}
    We must show that $\lfocus{\atmR{\lctx}_L}{\p{\atmR{a}} \limp \n{B}}{\atmL{\lctx}_R}{\p{C}}$ if, and only if, $\atmR{\lctx}_L = \atmR{\octx}_L \oc \p{\atmR{a}}$ and $\atmL{\lctx}_R = \atmL{\octx}_R$ and $\p{C} = \p{A}$.
    Indeed, the $\lrule{\limp}$ rule is the unique rule for left-focusing on the proposition $\p{\atmR{a}} \limp \n{B}$, so $\lfocus{\atmR{\lctx}_L}{\p{\atmR{a}} \limp \n{B}}{\atmL{\lctx}_R}{\p{C}}$ if, and only if, $\atmR{\lctx}_L = \atmR{\lctx}'_L \oc \p{\atmR{a}}\!\!$ and $\lfocus{\atmR{\lctx}'_L}{\n{B}}{\atmL{\lctx}_R}{\p{C}}$ for some $\atmR{\lctx}'_L$.
    By the inductive hypothesis, we have $\lfocus{\atmR{\lctx}'_L}{\n{B}}{\atmL{\lctx}_R}{\p{C}}$ if, and only if, $\atmR{\lctx}'_L = \atmR{\octx}_L$ and $\atmL{\lctx}_R = \atmL{\octx}_R$ and $\p{C} = \p{A}$.
    Putting everything together, $\lfocus{\atmR{\lctx}_L}{\p{\atmR{a}} \limp \n{B}}{\atmL{\lctx}_R}{\p{C}}$ if, and only if, $\atmR{\lctx}_L = \atmR{\octx}_L \oc \p{\atmR{a}}$ and $\atmL{\lctx}_R = \atmL{\octx}_R$ and $\p{C} = \p{A}$, as required.
  % 
  % \item
  %   The case in which
  % \begin{equation*}
  %   \infer{\chorax{\theta}{w_1}{\p{A}}{b \oc w_2}{\n{B} \pmir \atmL{b}}}{
  %     \chorax{\theta}{w_1}{\p{A}}{w_2}{\n{B}} &
  %     \text{($\theta(b) = \atmL{b}$)}}
  % \end{equation*}
  %   is symmtric to the previous one.
  % %
  % \qedhere
  % \end{itemize}
\end{proof}



\newthought{The principal} judgment is $\chorsig{\theta}{\srsig}{\orsig}$.
Given a [...] assignment $\theta$ and a string rewriting signature $\srsig$, this judgment produces a formula-as-process ordered rewriting signature $\orsig$ that, together with $\theta$, constitutes a [well-specified]\fixnote{?} choreography of the string rewriting specification $(\sralph, \srsig)$.

In other words, $\orsig$ is a solution to $\theta(\srsig)$, the rewritings induced by $\theta$ from the string rewriting axioms $\srsig$.
That is, if $\chorsig{\theta}{\srsig}{\orsig}$, then $\theta$ is a (strong) bisimulation between $\reduces_{\srsig}$ and $\reduces_{\orsig}$.
\marginnote{$\!
  \begin{tikzcd}[ampersand replacement=\&]
    w \rar[reduces, subscript=\srsig] \dar[relation][swap]{\theta}
     \& w\mathrlap{'} \dar[relation, exists]{\theta}
    \\
    \theta(w) \rar[reduces, exists, subscript=\orsig]
     \& \theta(w')
  \end{tikzcd}
  $ and $
  \begin{tikzcd}[ampersand replacement=\&]
    w \rar[reduces, exists, subscript=\srsig] \dar[relation][swap]{\theta}
     \& w\mathrlap{'} \dar[relation, exists]{\theta}
    \\
    \theta(w) \rar[reduces, subscript=\orsig]
     \& \octx \mathrlap{' = \theta(w')}
  \end{tikzcd}
  %\hphantom{' = \theta(w')}
  $}
\footnote{%
  Actually, we end up proving a stronger soundness result in \cref{??}.}
% The exact converse -- that $\theta(w) \reduces_{\orsig} \theta(w')$ implies $w \reduces_{\srsig} w'$ -- does hold, but we can prove an even stronger soundness result.
%
If $\chorsig{\theta}{\srsig}{\orsig}$ is not derivable for any $\orsig$, then the [...] assignment $\theta$ does not yield a [well-specified]\fixnote{?} choreography of $\srsig$.

% This judgment relies on an auxiliary judgment, $\qimp{\atmR{\octx}_L}{\up \p{A}}{\atmL{\octx}_R}{\n{B}}$, that transforms the quasi-proposition $\atmR{\octx}_L \limp \up \p{A} \pmir \atmL{\octx}_R$ into a well-formed proposition $\n{B}$ by nondeterministically abstracting atoms one-by-one from either the left or right contexts.
% The proposition $\n{B}$ is semantically equivalent to the quasi-proposition $\atmR{\octx}_L \limp \up \p{A} \pmir \atmL{\octx}_R$ in the sense that $\n{B}$ satisfies $\atmR{\octx}_L \oc \n{B} \oc \atmL{\octx}_R \reduces \octx'$ if, and only if, $\rfocus{\octx'}{\p{A}}$.

This choreographing judgment is defined by just two rules:
\begin{gather*}
  \infer{\chorsig{\theta}{\srsige}{\orsige}}{}
  \\
  \infer{\chorsig{\theta}{\srsig_0, \bigl(w^L_i \wc a \wc w^R_i \reduces w'_i\bigr)_{i \in \mathcal{I}}}{\orsig_0, \bigl(\defp{a} \defd \bigwith_{i \in \mathcal{I}} \n{A}_i\bigr)}}{
    \begin{array}[b]{@{}c@{}}
      \text{($\theta(a) = \defp{a}$)} \quad
      \chorsig{\theta}{\srsig_0}{\orsig_0} \quad
      \text{($\defp{a} \notin \dom{\orsig_0}$)}
      \\
      \multipremise{i \in \mathcal{I}}{
        \text{$\bigl(\theta(w^L_i) = \atmR{\octx}^L_i\bigr)$} \quad
        \text{$\bigl(\theta(w^R_i) = \atmL{\octx}^R_i\bigr)$} \quad
        \qimp{\atmR{\octx}^L_i}{\up \bigfuse \theta(w'_i)}{\atmL{\octx}^R_i}{\n{A}_i}}
    \end{array}}
\end{gather*}
The first of these rules is straightforward: an empty \acl{SR} signature choreographs as an empty ordered rewriting signature.
The second rule is quite a lot to parse and needs to be broken down step by step:
\begin{enumerate}
\item
  Choose a symbol $a$ that is mapped by $\theta$ to a recursively defined proposition, $\defp{a}$.
  Then reorganize the \ac{SR} signature $\srsig$, collecting together all axioms in $\srsig$ that have an $a$ in their premises.
  Let $\bigl(w^L_i \wc a \wc w^R_i \reduces w'_i\bigr)_{i \in \mathcal{I}}$ be those axioms, so that $\srsig = \srsig_0 , \bigl(w^L_i \wc a \wc w^R_i \reduces w'_i\bigr)_{i \in \mathcal{I}}$ for some $\srsig_0$.
\item
  Construct a well-specified choreography $\orsig_0$ from $\srsig_0$ and $\theta$, using the judgment $\chorsig{\theta}{\srsig_0}{\orsig_0}$.
  Check that $\orsig_0$ gives no definition for $\defp{a}$, otherwise there is some axiom in $\srsig_0$ that contains $a$ in its premise and $\bigl(w^L_i \wc a \wc w^R_i \reduces w'_i\bigr)_{i \in \mathcal{I}}$ does not correctly constitute all such axioms.
\item
  Check that each $w^L_i$ contains only those symbols that map to right-directed atoms, using the side condition $\theta(w^L_i) = \atmR{\octx}^L_i$.
  Symmetrically, check that each $w^R_i$ contains only symbols that map to left-directed atoms, using the side condition $\theta(w^R_i) = \atmL{\octx}^R_i$.
\item
  Elaborate each quasi-proposition $\atmR{\octx}^L_i \limp \up \bigfuse \theta(w'_i) \pmir \atmL{\octx}^R_i$ into a semantically equivalent proposition $\n{A}_i$.
  Based on \cref{??}, $\lfocus{\theta(w^L_i)}{\n{A}_i}{\theta(w^R_i)}{\bigfuse \theta(w'_i)}$, and so this proposition acts as the image of the axiom $w^L_i \wc a \wc w^R_i \reduces w'_i$ under $\theta$ -- that is, $\theta(w^L_i) \oc \n{A}_i \oc \theta(w^R_i) \reduces \theta(w'_i)$.
\item
  Collect the $\n{A}_i$s into a single definition, $\defp{a} \defd \bigwith_{i \in \mathcal{I}} \n{A}_i$, which, based on steps 2 and 4, describes all of the axioms from $\srsig$ that contain $a$ in their premises.
\end{enumerate}

If $\chorsig{\theta}{\srsig}{\orsig}$, then $\theta$ is a bisimulation.
That is, $\chorsig{\theta}{\srsig}{\orsig}$ implies
\begin{equation*}
  \begin{tikzcd}
    w \rar[reduces, subscript=\srsig] \dar[relation][swap]{\theta}
      & w\mathrlap{'} \dar[relation, exists]{\theta}
    \\
    \theta(w) \rar[reduces, exists, subscript=\orsig] & \theta(w')
  \end{tikzcd}
  \quad\text{and}\quad
  \begin{tikzcd}
    w \rar[reduces, exists, subscript=\srsig] \dar[relation][swap]{\theta}
      & w\mathrlap{'} \dar[relation, exists]{\theta}
    \\
    \theta(w) \rar[reduces, subscript=\orsig] & \octx \mathrlap{' = \theta(w') \,.}
  \end{tikzcd}
  \hphantom{' = \theta(w') \,.}
\end{equation*}



As stated earlier, when $\chorsig{\theta}{\srsig}{\orsig}$, the string rewriting step $w \reduces_{\srsig} w'$ holds if, and only if, the ordered rewriting step $\theta(w) \reduces_{\orsig} \theta(w')$ holds.
We prove the left-to-right direction as the following completeness \lcnamecref{thm:chor-complete} and then prove a stronger soundness \lcnamecref{thm:chor-sound} that implies the right-to-left direction.
% Each axiom $w \reduces w'$ in the string rewriting signature $\srsig$ is processed in turn.
% \begin{itemize}
% \item First, we verify that the premise $w$ may be expressed as $w = w_1 \wc a \wc w_2$, where:
% $\theta$ assigns a process role to $a$ (\ie, $\theta(a) = \defp{a}$ for some $\defp{a}$);
% right-directed message roles to all symbols in $w_1$ (\ie, $\theta(w_1) = \atmR{\octx}_L$ for some $\atmR{\octx}_L$);
% and left-directed messages roles to all symbols in $w_2$ (\ie, $\theta(w_2) = \atmL{\octx}_R$ for some $\atmL{\octx}_R$).

% \item
%   Next, we construct the quasi-proposition $\theta(w_1) \limp \up \bigfuse \theta(w') \pmir \theta(w_2)$.

% \item
%   Last, we inductively 
% \end{itemize}

% To define this choreographing judgment, we also use an auxiliary judgment that choreographs individual axioms.
% Given a [...] assignment $\theta$, strings $w_1$ and $w_2$, and a positive proposition $\p{A}$, the judgment $\chorax{\theta}{w_1}{\p{A}}{w_2}{\n{B}}$ checks that $\theta(w_1)$ and $\theta(w_2)$ consist of only right- and left-directed atoms, respectively, and then produces a negative proposition $\n{B}$ that is morally $\theta(w_1) \limp \up \p{A} \pmir \theta(w_2)$ -- that is, a proposition $\n{B}$ such that $\lfocus{\theta(w_1)}{\n{B}}{\theta(w_2)}{\p{A}}$.

% Each of the axioms is processed one-by-one.
% From the axiom $w \reduces w'$, a symbol $a$ is nondeterministically selected from the premise $w$;
% the selected symbol must be assigned a process role by $\theta$.

% For each axiom $w \reduces w'$, we verfify that the premise $w$ may be expressed as $w = w_1 \wc a \wc w_2$, where $\theta$ assigns a process role to $a$, right-directed message roles to all symbols in $w_1$, and left-directed messages roles to all symbols in $w_2$.
% Then, for each process $\defp{a}$, all axioms with premises $w_1 \wc a \wc w_2$
% \begin{inferences}
%   \infer{(\octxe) \limp \up \p{A} \pmir (\octxe) \rightsquigarrow \up \p{A}}{}
%   \\
%   \infer{(\atmR{\octx}_L \oc \atmR{a}) \limp \up \p{A} \pmir \atmL{\octx}_R \rightsquigarrow \atmR{a} \limp \n{B}}{
%     \atmR{\octx}_L \limp \up \p{A} \pmir \atmL{\octx}_R \rightsquigarrow \n{B}}
%   \and
%   \infer{\atmR{\octx}_L \limp \up \p{A} \pmir (\atmL{a} \oc \atmL{\octx}_R) \rightsquigarrow \n{B} \pmir \atmL{a}}{
%     \atmR{\octx}_L \limp \up \p{A} \pmir \atmL{\octx}_R \rightsquigarrow \n{B}}
% \end{inferences}
% When $\theta(b) = \atmR{b}$, the quasi-proposition $\theta(w_1 \wc b) \limp \up \p{A} \pmir \theta(w_2)$ is equivalent to $\atmR{b} \limp \bigl(\theta(w_1) \limp \up \p{A} \pmir \theta(w_2)\bigr)$.%
% \footnote{That $b$ moves from the right of $w_1$ to the left of $\theta(w_1)$ as $\atmR{b}$ is somewhat counterintuitive, but the proof of \cref{??} explains.}
% Likewise, when $\theta(b) = \atmL{b}$, the quasi-proposition $\theta(w_1) \limp \up \p{A} \pmir \theta(b \wc w_2)$ is equivalent to $\bigl(\theta(w_1) \limp \up \p{A} \pmir \theta(w_2)\bigr) \pmir \atmL{b}$.
 
% \begin{inferences}
%   \infer{(\octxe) \limp \up \p{A} \pmir (\octxe) \rightsquigarrow \up \p{A}}{}
%   \\
%   \infer{(\atmR{\octx}_L \oc \atmR{a}) \limp \up \p{A} \pmir \atmL{\octx}_R \rightsquigarrow \atmR{a} \limp \n{B}}{
%     \atmR{\octx}_L \limp \up \p{A} \pmir \atmL{\octx}_R \rightsquigarrow \n{B}}
%   \and
%   \infer{\atmR{\octx}_L \limp \up \p{A} \pmir (\atmL{a} \oc \atmL{\octx}_R) \rightsquigarrow \n{B} \pmir \atmL{a}}{
%     \atmR{\octx}_L \limp \up \p{A} \pmir \atmL{\octx}_R \rightsquigarrow \n{B}}
%   \\
%   \infer{\chorsig{\theta}{\srsig, (w_1 \wc a \wc w_2 \reduces w')}{\orsig, (\defp{a} \defd \n{A} \with \n{B})}}{
%     \begin{array}[b]{@{}c@{}}
%       \text{($\theta(w_1) = \atmR{\octx}_L$)} \quad
%       \text{($\theta(a) = \defp{a}$)} \quad
%       \text{($\theta(w_2) = \atmL{\octx}_R$)} \\
%       \atmR{\octx}_L \limp \up \bigfuse \theta(w') \pmir \atmL{\octx}_R \rightsquigarrow \n{B} \quad
%       \chorsig{\theta}{\srsig}{\orsig, (\defp{a} \defd \n{A})}
%     \end{array}}
%   \\
%   \infer{\chorsig{\theta}{\srsig, (w_1 \wc a \wc w_2 \reduces w')}{\orsig, (\defp{a} \defd \n{B})}}{
%     \begin{array}[b]{@{}c@{}}
%       \text{($\theta(w_1) = \atmR{\octx}_L$)} \quad
%       \text{($\theta(a) = \defp{a}$)} \quad
%       \text{($\theta(w_2) = \atmL{\octx}_R$)} \\
%       \atmR{\octx}_L \limp \up \bigfuse \theta(w') \pmir \atmL{\octx}_R \rightsquigarrow \n{B} \quad
%       \chorsig{\theta}{\srsig}{\orsig} \quad
%       \text{($\defp{a} \notin \dom{\orsig}$)}
%     \end{array}}
% \end{inferences}

% Judgments $\chorsig{\theta}{\srsig}{\orsig}$ and $\chorax{\theta}{w_1}{\n{A}}{w_2}{\n{B}}$.
% In both judgments, all terms before the $\chorarrow$ are inputs; all terms after the $\chorarrow$ are outputs.

% \begin{inferences}
%   \infer{\chorsig{\theta}{\srsig, (w_1 \wc a \wc w_2 \reduces w')}{\orsig, (\defp{a} \defd \n{A} \with \n{B})}}{
%     \begin{array}[b]{@{}c@{}}
%       \text{($\theta(w_1) = \atmR{\octx}_L$)} \quad
%       \text{($\theta(a) = \defp{a}$)} \quad
%       \text{($\theta(w_2) = \atmL{\octx}_R$)} \\
%       \atmR{\octx}_L \limp \up \bigfuse \theta(w') \pmir \atmL{\octx}_R \rightsquigarrow \n{B} \quad
%       \chorsig{\theta}{\srsig}{\orsig, (\defp{a} \defd \n{A})}
%     \end{array}}
%   \\
%   \infer{\chorsig{\theta}{\srsig, (w_1 \wc a \wc w_2 \reduces w')}{\orsig, (\defp{a} \defd \n{B})}}{
%     \begin{array}[b]{@{}c@{}}
%       \text{($\theta(w_1) = \atmR{\octx}_L$)} \quad
%       \text{($\theta(a) = \defp{a}$)} \quad
%       \text{($\theta(w_2) = \atmL{\octx}_R$)} \\
%       \atmR{\octx}_L \limp \up \bigfuse \theta(w') \pmir \atmL{\octx}_R \rightsquigarrow \n{B} \quad
%       \chorsig{\theta}{\srsig}{\orsig} \quad
%       \text{($\defp{a} \notin \dom{\orsig}$)}
%     \end{array}}
% \end{inferences}


% \begin{inferences}
%   \infer{\chorsig{\theta}{\sige}{\sige}}{}
%   \and
%   \infer{\chorsig{\theta}{\sig, w \reduces w'}{\sig', \proc{a} \defd \n{A}_1 \with \n{A}_2(\up \p{C})}}{
%     \chorsig{\theta}{\sig}{\sig'} &
%     \chorax{\theta}{w \reduces w'}{\proc{a}}{\n{A}_2(\Box)}{\p{C}} &
%     \text{($\sig'(\proc{a}) = \n{A}_1$)}}
%   \\
%   \infer{\chorsig{\theta}{\sig, w \reduces w'}{\sig', \proc{a} \defd \n{A}(\up \p{C})}}{
%     \chorsig{\theta}{\sig}{\sig'} &
%     \chorax{\thea}{w \reduces w'}{\proc{a}}{\n{A}(\Box)}{\p{C}} &
%     \text{($\proc{a} \notin \dom{\sig'}$)}}
%   \\
%   \infer{\chorax{\theta}{a \reduces w'}{\proc{a}}{\Box}{\bigfuse \octx'}}{
%     \text{($\theta(a) = \proc{a}$)} &
%     \text{($\theta(w') = \octx'$)}}
%   \and
%   \infer{\chorax{\theta}{b \oc w \reduces w'}{\proc{a}}{\n{A}(\atmR{b} \limp \Box)}{\p{C}}}{
%     \chorax{\theta}{w \reduces w'}{\proc{a}}{\n{A}(\Box)}{\p{C}} &
%     \text{($\theta(b) = \atmR{b}$)}}
%   \\
%   \infer{\chorax{\theta}{w \oc b \reduces w'}{\proc{a}}{\n{A}(\Box \pmir \atmL{b})}{C}}{
%     \chorax{\theta}{w \reduces w'}{\proc{a}}{\n{A}(\Box)}{\p{C}} &
%     \text{($\theta(b) = \atmL{b}$)}}
% \end{inferences}




\begin{lemma}[Weakening]
  If $\octx \reduces_{\orsig} \octx'$ and $\dom{\orsig} \cap \dom{\orsig'} = \emptyset$, then $\octx \reduces_{\orsig, \orsig'} \octx'$.
  % Similarly, if $\octx \reduces_{\orsig, (\defp{a} \defd \n{A})} \octx'$ or $\octx \reduces_{\orsig, (\defp{a} \defd \n{B})} \octx'$, then $\octx \reduces_{\orsig, (\defp{a} \defd \n{A} \with \n{B})} \octx'$.
\end{lemma}
\begin{proof}
  By induction over the structure of the given rewriting step.
\end{proof}

% \begin{lemma}
%   If $(w \reduces w') \in \srsig$ and $\chorsig{\theta}{\srsig}{\orsig}$, then $\theta(w) \reduces_{\orsig} \theta(w')$.
% \end{lemma}
% \begin{proof}
%   By induction over the structure of the given choreographing derivation, $\chorsig{\theta}{\srsig}{\orsig}$.
%   \begin{itemize}
%   \item Consider the case in which $w = w_1 \oc a \oc w_2 \reduces w'$ is the axiom in question and
%     \begin{equation*}
%       \infer{\chorsig{\theta}{\srsig, w_1 \oc a \oc w_2 \reduces w'}{\orsig, \defp{a} \defd \n{A} \with \n{B}}}{
%         \text{($\theta(a) = \defp{a}$)} &
%         \chorax{\theta}{w_1}{\bigfuse \theta(w')}{w_2}{\n{B}} &
%         \chorsig{\theta}{\srsig}{\orsig, \defp{a} = \n{A}}}
%     \end{equation*}
%     It follows from \cref{??} that $\lfocus{\theta(w_1)}{\n{B}}{\theta(w_2)}{\bigfuse \theta(w')}$, and hence $\lfocus{\theta(w_1)}{\n{A} \with \n{B}}{\theta(w_2)}{\bigfuse \theta(w')}$.
%     And because $\rfocus{\theta(w')}{\bigfuse \theta(w')}$ and $\defp{a} \defd \n{A} \with \n{B}$, we have $\theta(w) = \theta(w_1) \oc \defp{a} \oc \theta(w_2) \reduces \theta(w')$.
  
%   \item Consider the case in which
%     \begin{equation*}
%       \infer{\chorsig{\theta}{\sig, v_1 \oc a \oc v_2 \reduces v'}{\sig', \hat{a} \defd \n{A} \with \n{B}}}{
%         \text{($\theta(a) = \hat{a}$)} &
%         \chorax{\theta}{v_1}{\bigfuse \theta(v')}{v_2}{\n{B}} &
%         \chorsig{\theta}{\sig}{\sig'} &
%         \text{($\sig'(\hat{a}) = \n{A}$)}}
%     \end{equation*}
%     and the axiom $w \reduces w'$ comes from $\sig$.
%     By the inductive hypothesis, $\theta(w) \reduces_{\sig'} \theta(w')$.
%     By \cref{??}, $\theta(w) \reduces_{\sig', \hat{a} \defd \n{A} \with \n{B}} \theta(w')$.
  
%   \item Consider the case in which
%     \begin{equation*}
%       \infer{\chorsig{\theta}{\sig, v_1 \oc a \oc v_2 \reduces v'}{\sig', \hat{a} \defd \n{B}}}{
%         \text{($\theta(a) = \hat{a}$)} &
%         \chorax{\theta}{v_1}{\bigfuse \theta(v')}{v_2}{\n{B}} &
%         \chorsig{\theta}{\sig}{\sig'} &
%         \text{($\hat{a} \notin \dom{\sig'}$)}}
%     \end{equation*}
%     and the axiom $w \reduces w'$ comes from $\sig$.
%     By the inductive hypothesis, $\theta(w) \reduces_{\sig'} \theta(w')$.
%     By \cref{??}, $\theta(w) \reduces_{\sig', \hat{a} \defd \n{B}} \theta(w')$.
%   %
%   \qedhere
%   \end{itemize}
% \end{proof}

\begin{theorem}[Completeness]\leavevmode
  If $\chorsig{\theta}{\srsig}{\orsig}$, then $w \reduces_{\srsig} w'$ implies $\theta(w) \reduces_{\orsig} \theta(w')$.%
  \marginnote{If $\chorsig{\theta}{\srsig}{\orsig}$, then $
    \begin{tikzcd}[ampersand replacement=\&]
      w \rar[reduces, subscript=\srsig] \dar[relation][swap]{\theta}
       \& w\mathrlap{'} \dar[relation, exists]{\theta}
      \\
      \theta(w) \rar[reduces, exists, subscript=\orsig]
       \& \theta(w')
    \end{tikzcd}$}%
\end{theorem}
\begin{proof}
  By simultaneous structural induction on the given choreographing derivation, $\chorsig{\theta}{\srsig}{\orsig}$, and ordered rewriting step, $w \reduces_{\srsig} w'$.
  \begin{itemize}
  \item
    Consider the case in which
    \begin{equation*}
      \chorsig{\theta}{\srsig}{\orsig}
      \qquad\text{and}\qquad
      w =
      \infer[\jrule{$\reduces$C}]{w_1 \wc w_0 \wc w_2 \reduces_{\srsig} w_1 \wc w'_0 \wc w_2}{
        w_0 \reduces_{\srsig} w'_0}
      = w'
      \,.
    \end{equation*}
    By the inductive hypothesis, $\theta(w_0) \reduces_{\orsig} \theta(w'_0)$.
    It follows from ordered rewriting's $\jrule{$\reduces$C}$ rule that
    \begin{equation*}
      \theta(w) = \theta(w_1) \oc \theta(w_0) \oc \theta(w_2) \reduces_{\orsig} \theta(w_1) \oc \theta(w'_0) \oc \theta(w_2) = \theta(w')
      \,.
    \end{equation*}

  \item
    Consider the case in which
    \begin{gather*}
      \infer{\chorsig{\theta}{\srsig_0, \bigl(w^L_i \wc a \wc w^R_i \reduces w'_i\bigr)_{i \in \mathcal{I}}}{\orsig_0, \bigl(\defp{a} \defd \bigwith_{i \in \mathcal{I}} \n{A}_i\bigr)}}{
        \begin{array}[b]{@{}c@{}}
          \text{($\theta(a) = \defp{a}$)} \quad
          \chorsig{\theta}{\srsig_0}{\orsig_0} \quad
          \text{($\defp{a} \notin \dom{\orsig_0}$)}
          \\
          \multipremise{i \in \mathcal{I}}{
            \text{$\bigl(\theta(w^L_i) = \atmR{\octx}^L_i\bigr)$} \quad
            \text{$\bigl(\theta(w^R_i) = \atmL{\octx}^R_i\bigr)$} \quad
            \qimp{\atmR{\octx}^L_i}{\up \bigfuse \theta(w'_i)}{\atmL{\octx}^R_i}{\n{A}_i}}
        \end{array}}
    %
    \shortintertext{and}
    %
      w = \infer[\jrule{$\reduces$AX}]{w^L_k \wc a \wc w^R_k \reduces_{\srsig} w'_k}{(w^L_k \wc a \wc w^R_k \reduces w'_k) \in \srsig} = w'
    \end{gather*}
    for some $k \in \mathcal{I}$, where $\srsig = \srsig_0, (w^L_i \wc a \wc w^R_i \reduces w'_i)_{i \in \mathcal{I}}$ and $\orsig = \orsig_0 , (\bigwith_{i \in \mathcal{I}} \n{A}_i)$.

    By \cref{??}, $\lfocus{\theta(w^L_k)}{\n{A}_k}{\theta(w^R_k)}{\bigfuse \theta(w'_k)}$.
    $\lfocus{\theta(w^L_k)}{\bigwith_{i \in \mathcal{I}} \n{A}_i}{\theta(w^R_k)}{\bigfuse \theta(w'_k)}$.
    Because $\rfocus{\theta(w'_k)}{\bigfuse \theta(w'_k)}$~\parencref{??}, it follows by the $\jrule{$\reduces$I}$ rule that $\theta(w^L_k) \oc \bigl(\bigwith_{i \in \mathcal{I}} \n{A}_i\bigr) \oc \theta(w^R_k) \reduces_{\orsig} \theta(w'_k)$, and so $\theta(w) = \theta(w^L_k) \oc \defp{a} \oc \theta(w^R_k) \reduces_{\orsig} \theta(w'_k) = \theta(w')$.

  \item
    Consider the case in which
    \begin{gather*}
      \infer{\chorsig{\theta}{\srsig_0, \bigl(v^L_i \wc a \wc v^R_i \reduces v'_i\bigr)_{i \in \mathcal{I}}}{\orsig_0, \bigl(\defp{a} \defd \bigwith_{i \in \mathcal{I}} \n{A}_i\bigr)}}{
        \begin{array}[b]{@{}c@{}}
          \text{($\theta(a) = \defp{a}$)} \quad
          \chorsig{\theta}{\srsig_0}{\orsig_0} \quad
          \text{($\defp{a} \notin \dom{\orsig_0}$)}
          \\
          \multipremise{i \in \mathcal{I}}{
            \text{$\bigl(\theta(v^L_i) = \atmR{\octx}^L_i\bigr)$} \quad
            \text{$\bigl(\theta(v^R_i) = \atmL{\octx}^R_i\bigr)$} \quad
            \qimp{\atmR{\octx}^L_i}{\up \bigfuse \theta(v'_i)}{\atmL{\octx}^R_i}{\n{A}_i}}
        \end{array}}
    %
    \shortintertext{and}
    %
      \infer[\jrule{$\reduces$AX}]{w \reduces_{\srsig} w'}{
        (w \reduces w') \in \srsig_0}
    \end{gather*}
    where $(w \reduces w') \in \srsig_0$ and $\srsig = \srsig_0, (v^L_i \wc a \wc v^R_i \reduces v'_i)_{i \in \mathcal{I}}$ and $\orsig = \orsig_0 , (\bigwith_{i \in \mathcal{I}} \n{A}_i)$.

    By the inductive hypothesis, $\theta(w) \reduces_{\orsig_0} \theta(w')$.
    It follows from weakening~\parencref{??} that $\theta(w) \reduces_{\orsig} \theta(w')$.    

  \item 
    The case in which
    \begin{equation*}
      \infer{\chorsig{\theta}{\srsige}{\orsige}}{}
      \qquad\text{and}\qquad
      \infer[\jrule{$\reduces$AX}]{w \reduces_{\srsig} w'}{
        (w \reduces w') \in \srsig}
    \end{equation*}
    where $\srsig = \srsige$ and $\orsig = \orsige$ is vacuous.
  % \item
  %   Consider the case in which
  %   \begin{gather*}
  %     \infer{\chorsig{\theta}{\srsig_0, (w_1 \wc a \wc w_2 \reduces w')}{\orsig_0, (\defp{a} \defd \n{A} \with \n{B})}}{
  %       \begin{array}[b]{@{}c@{}}
  %         \text{($\theta(w_1) = \atmR{\octx}_L$)} \quad
  %         \text{($\theta(a) = \defp{a}$)} \quad
  %         \text{($\theta(w_2) = \atmL{\octx}_R$)} \\
  %         \atmR{\octx}_L \limp \up \bigfuse \theta(w') \pmir \atmL{\octx}_R \rightsquigarrow \n{B} \quad
  %         \chorsig{\theta}{\srsig_0}{\orsig_0, (\defp{a} \defd \n{A})}
  %       \end{array}}
  %   \shortintertext{and}
  %     w =
  %     \infer[\jrule{$\reduces$AX}]{w_1 \wc a \wc w_2 \reduces_{\srsig} w'}{}
  %   \end{gather*}
  %   where $\srsig = \srsig_0 , (w_1 \wc a \wc w_2 \reduces w')$ and $\orsig = \orsig_0, (\defp{a} \defd \n{A} \with \n{B})$.

  %   By \cref{lem:chorax-sound-complete}, $\lfocus{\theta(w_1)}{\n{B}}{\theta(w_2)}{\bigfuse \theta(w')}$.
  %   Upon adding the $\lrule{\with}_2$ rule, $\lfocus{\theta(w_1)}{\n{A} \with \n{B}}{\theta(w_2)}{\bigfuse \theta(w')}$.
  %   Because $\rfocus{\theta(w')}{\bigfuse \theta(w')}$~\parencref{??}, it follows by the $\jrule{$\reduces$I}$ rule that $\theta(w_1) \oc (\n{A} \with \n{B}) \oc \theta(w_2) \reduces_{\orsig} \theta(w')$, and so $\theta(w) = \theta(w_1) \oc \defp{a} \oc \theta(w_2) = \theta(w_1) \oc (\n{A} \with \n{B}) \oc \theta(w_2) \reduces_{\orsig} \theta(w')$


  % \item
  %   Consider the case in which
  %   \begin{gather*}
  %     \infer{\chorsig{\theta}{\srsig_0, (w_1 \wc a \wc w_2 \reduces w')}{\orsig_0, (\defp{a} \defd \n{B})}}{
  %       \begin{array}[b]{@{}c@{}}
  %         \text{($\theta(w_1) = \atmR{\octx}_L$)} \quad
  %         \text{($\theta(a) = \defp{a}$)} \quad
  %         \text{($\theta(w_2) = \atmL{\octx}_R$)} \\
  %         \atmR{\octx}_L \limp \up \bigfuse \theta(w') \pmir \atmL{\octx}_R \rightsquigarrow \n{B} \quad
  %         \chorsig{\theta}{\srsig_0}{\orsig_0} \quad
  %         \text{($\defp{a} \notin \dom{\orsig_0}$)}
  %       \end{array}}
  %   \shortintertext{and}
  %     w =
  %     \infer[\jrule{$\reduces$AX}]{w_1 \wc a \wc w_2 \reduces_{\srsig} w'}{}
  %   \end{gather*}
  %   where $\srsig = \srsig_0 , (w_1 \wc a \wc w_2 \reduces w')$ and $\orsig = \orsig_0 , (\defp{a} \defd \n{B})$.

  %   By \cref{lem:chorax-sound-complete}, $\lfocus{\theta(w_1)}{\n{B}}{\theta(w_2)}{\bigfuse \theta(w')}$.
  %   Because $\rfocus{\theta(w')}{\bigfuse \theta(w')}$, it follows by the $\jrule{$\reduces$I}$ rule that $\theta(w_1) \oc \n{B} \oc \theta(w_2) \reduces_{\orsig} \theta(w')$, and so $\theta(w) = \theta(w_1) \oc \defp{a} \oc \theta(w_2) = \theta(w_1) \oc \n{B} \oc \theta(w_2) \reduces_{\orsig} \theta(w')$.
    
  % \item
  %   Consider the case in which
  %   \begin{gather*}
  %     \infer{\chorsig{\theta}{\srsig_0, (v_1 \wc b \wc v_2 \reduces v')}{\orsig_0, (\defp{b} \defd \n{A} \with \n{B})}}{
  %       \begin{array}[b]{@{}c@{}}
  %         \text{($\theta(v_1) = \atmR{\octx}_L$)} \quad
  %         \text{($\theta(b) = \defp{b}$)} \quad
  %         \text{($\theta(v_2) = \atmL{\octx}_R$)} \\
  %         \atmR{\octx}_L \limp \up \bigfuse \theta(v') \pmir \atmL{\octx}_R \rightsquigarrow \n{B} \quad
  %         \chorsig{\theta}{\srsig_0}{\orsig_0, (\defp{b} \defd \n{A})}
  %       \end{array}}
  %   \shortintertext{and}
  %     \infer[\jrule{$\reduces$AX}]{w \reduces_{\srsig} w'}{
  %       w \reduces w' \in \srsig_0}
  %   \end{gather*}
  %   where $\srsig = \srsig_0, (v_1 \wc b \wc v_2 \reduces v')$ and $\orsig = \orsig_0 , (\defp{b} \defd \n{A} \with \n{B})$.

  %   By the inductive hypothesis, $\theta(w) \reduces_{\orsig_0, (\defp{b} \defd \n{A})} \theta(w')$.
  %   It follows from the weakening \lcnamecref{??}~\parencref{??} that $\theta(w) \reduces_{\orsig} \theta(w')$, as required.

  % \item
  %   The case in which
  %   \begin{gather*}
  %     \infer{\chorsig{\theta}{\srsig_0, (v_1 \wc b \wc v_2 \reduces v')}{\orsig_0, (\defp{b} \defd \n{B})}}{
  %       \begin{array}[b]{@{}c@{}}
  %         \text{($\theta(v_1) = \atmR{\octx}_L$)} \quad
  %         \text{($\theta(b) = \defp{b}$)} \quad
  %         \text{($\theta(v_2) = \atmL{\octx}_R$)} \\
  %         \atmR{\octx}_L \limp \up \bigfuse \theta(v') \pmir \atmL{\octx}_R \rightsquigarrow \n{B} \quad
  %         \chorsig{\theta}{\srsig_0}{\orsig_0} \quad
  %         \text{($\defp{b} \notin \dom{\orsig_0}$)}
  %       \end{array}}
  %   \shortintertext{and}
  %     \infer[\jrule{$\reduces$AX}]{w \reduces_{\srsig} w'}{
  %       w \reduces w' \in \srsig_0}
  %   \end{gather*}
  %   where $\srsig = \srsig_0, (v_1 \wc b \wc v_2 \reduces v')$ and $\orsig = \orsig_0 , (\defp{b} \defd \n{B})$ is similar to the previous one.
    %
  \qedhere
  \end{itemize}
\end{proof}




% \begin{lemma}
%   If $\chorax{\theta}{w_1}{\p{A}}{w_2}{\n{B}}$ and $\lfocus{\atmR{\octx}_L}{\n{B}}{\atmL{\octx}_R}{\p{C}}$, then $\atmR{\octx}_L = \theta(w_1)$ and $\atmL{\octx}_R = \theta(w_2)$ and $\p{A} = \p{C}$.
% \end{lemma}
% \begin{proof}
%   By induction over the structure of the given choreographing derivation, $\chorax{\theta}{w_1}{\p{A}}{w_2}{\n{B}}$.
%   \begin{itemize}
%   \item
%     Consider the case in which
%   \begin{equation*}
%     \infer{\chorax{\theta}{\emp}{\p{A}}{\emp}{\up \p{A}}}{}
%     \qquad\text{and}\qquad
%     \lfocus{\atmR{\octx}_1}{\up \p{A}}{\atmL{\octx}_2}{\p{C}}
%     \,.
%   \end{equation*}
%   By inversion on the left-focus derivation, $\atmR{\octx}_L = \octxe = \theta(\emp)$ and $\atmL{\octx}_R = \octxe = \theta(\emp)$, as well as $\p{A} = \p{C}$, as required.

%   \item
%     Consider the case in which
%   \begin{equation*}
%     \infer{\chorax{\theta}{w_1 \oc b}{\p{A}}{w_2}{\atmR{b} \limp \n{B}}}{
%       \text{($\theta(b) = \atmR{b}$)} &
%       \chorax{\theta}{w_1}{\p{A}}{w_2}{\n{B}}}
%     \qquad\text{and}\qquad
%     \lfocus{\atmR{\octx}_L}{\atmR{b} \limp \n{B}}{\atmL{\octx}_R}{\p{C}}
%     \,.
%   \end{equation*}
%   By inversion on the left-focus derivation for $\atmR{b} \limp \n{B}$, there exists $\atmR{\octx}'_1$ such that $\atmR{\octx}_L = \atmR{\octx}'_L \oc \atmR{b}$ and $\lfocus{\atmR{\octx}'_L}{\n{B}}{\atmL{\octx}_R}{\p{C}}$.
%   It follows from the inductive hypothesis that $\atmR{\octx}'_L = \theta(w_1)$ and $\atmL{\octx}_R = \theta(w_2)$ and $\p{A} = \p{C}$.
%   So $\atmR{\octx}_L = \theta(w_1) \oc \atmR{b} = \theta(w_1 \oc b)$.

%   \item
%     The case in which
% \begin{equation*}
%     \infer{\chorax{\theta}{w_1 \oc b}{\p{A}}{w_2}{\n{B} \pmir \atmL{b}}}{
%       \text{($\theta(b) = \atmL{b}$)} &
%       \chorax{\theta}{w_1}{\p{A}}{w_2}{\n{B}}}
%     \qquad\text{and}\qquad
%     \lfocus{\atmR{\octx}_L}{\n{B} \pmir \atmL{b}}{\atmL{\octx}_R}{\p{C}}
%   \end{equation*}
%   is symmetric.
%   \qedhere
%   \end{itemize}
% \end{proof}


\begin{lemma}
  If $\chorsig{\theta}{\srsig}{\orsig}$ and $\lfocus{\atmR{\octx}_L}{\defp{a}}{\atmL{\octx}_R}{_{\orsig} \p{C}}$, then there exists an axiom $(w_1 \oc a \oc w_2 \reduces w') \in \srsig$ such that $\atmR{\octx}_L = \theta(w_1)$, $\atmL{\octx}_R = \theta(w_2)$, and $\p{C} = \bigfuse \theta(w')$.
\end{lemma}
\begin{proof}
  By induction over the structure of the given choreographing derivation, $\chorsig{\theta}{\srsig}{\orsig}$.
  \begin{itemize}
  \item
    Consider the case in which
    \begin{gather*}
      \infer{\chorsig{\theta}{\srsig_0, \bigl(w^L_i \wc a \wc w^R_i \reduces w'_i\bigr)_{i \in \mathcal{I}}}{\orsig_0, \bigl(\defp{a} \defd \bigwith_{i \in \mathcal{I}} \n{A}_i\bigr)}}{
        \begin{array}[b]{@{}c@{}}
          \chorsig{\theta}{\srsig_0}{\orsig_0} \quad
          \text{($\theta(a) = \defp{a}$)} \quad
          \text{($\defp{a} \notin \dom{\orsig_0}$)}
          \\
          \multipremise{i \in \mathcal{I}}{
            \text{$\bigl(\theta(w^L_i) = \atmR{\lctx}^L_i\bigr)$} \quad
            \text{$\bigl(\theta(w^R_i) = \atmL{\lctx}^R_i\bigr)$} \quad
            \qimp{\atmR{\lctx}^L_i}{\up \bigfuse \theta(w'_i)}{\atmL{\lctx}^R_i}{\n{A}_i}}
        \end{array}}
    %
    \shortintertext{and}
    %
      \lfocus{\atmR{\octx}_L}{\defp{a} = \textstyle\bigwith_{i \in \mathcal{I}} \n{A}_i}{\atmL{\octx}_R}{_{\orsig} \p{C}}
    \end{gather*}
    where $\srsig = \srsig_0, \bigl(w^L_i \wc a \wc w^R_i \reduces w'_i\bigr)_{i \in \mathcal{I}}$ and $\orsig = \orsig_0, \bigl(\defp{a} \defd \bigwith_{i \in \mathcal{I}} \n{A}_i\bigr)$.

    By inversion on the left-focus derivation, either: $\lfocus{\atmR{\octx}_L}{\n{A}_k}{\atmL{\octx}_R}{\p{C}}$ for some $k \in \mathcal{I}$; or $\mathcal{I}$ is empty.
    \begin{itemize}
    \item
      If $\lfocus{\atmR{\octx}_L}{\n{A}_k}{\atmL{\octx}_R}{\p{C}}$ for some $k \in \mathcal{I}$, then \cref{??} allows us to conclude that $\atmR{\octx}_L = \atmR{\lctx}^L_k = \theta(w^L_k)$ and $\atmL{\octx}_R = \atmL{\lctx}^R_k = \theta(w^R_k)$ and $\p{C} = \bigfuse \theta(w'_k)$.
      Also, the axiom $w^L_k \wc a \wc w^R_k \reduces w'_k$ is contained in $\srsig$.
    \item
      Otherwise, if $\mathcal{I}$ is empty, then $\bigwith_{i \in \mathcal{I}} \n{A}_i = \top$.
      There is no $\lrule{\top}$ rule to derive $\lfocus{\atmR{\octx}_L}{\defp{a} = \top}{\atmL{\octx}_R}{_{\orsig} \p{C}}$, so this case is vacuous.
    \end{itemize}




  \item
    Consider the case in which
    \begin{gather*}
      \infer{\chorsig{\theta}{\srsig_0, \bigl(v^L_i \wc b \wc v^R_i \reduces v'_i\bigr)_{i \in \mathcal{I}}}{\orsig_0, \bigl(\defp{b} \defd \bigwith_{i \in \mathcal{I}} \n{B}_i\bigr)}}{
        \begin{array}[b]{@{}c@{}}
          \chorsig{\theta}{\srsig_0}{\orsig_0} \quad
          \text{($\theta(b) = \defp{b}$)} \quad
          \text{($\defp{b} \notin \dom{\orsig_0}$)}
          \\
          \multipremise{i \in \mathcal{I}}{
            \text{$\bigl(\theta(v^L_i) = \atmR{\lctx}^L_i\bigr)$} \quad
            \text{$\bigl(\theta(v^R_i) = \atmL{\lctx}^R_i\bigr)$} \quad
            \qimp{\atmR{\lctx}^L_i}{\up \bigfuse \theta(v'_i)}{\atmL{\lctx}^R_i}{\n{B}_i}}
        \end{array}}
    %
    \shortintertext{and}
    %
      \lfocus{\atmR{\octx}_L}{\defp{a}}{\atmL{\octx}_R}{_{\orsig} \p{C}}
    \end{gather*}
    where $a \neq b$ and $\srsig = \srsig_0, \bigl(v^L_i \wc b \wc v^R_i \reduces v'_i\bigr)_{i \in \mathcal{I}}$ and $\orsig = \orsig_0, \bigl(\defp{b} \defd \bigwith_{i \in \mathcal{I}} \n{B}_i\bigr)$.

    By the inductive hypthesis, there exists a string rewriting axiom $(w_1 \wc a \wc w_2 \reduces w') \in \srsig_0$ such that $\atmR{\octx}_L = \theta(w_1)$ and $\atmL{\octx}_R = \theta(w_2)$ and $\p{C} = \bigfuse \theta(w')$.
    The same axiom is contained in the signature $\srsig$.


  \item 
    The case in which
    \begin{equation*}
      \infer{\chorsig{\theta}{\srsige}{\orsige}}{}
      \qquad\text{and}\qquad
      \lfocus{\atmR{\octx}_L}{\defp{a}}{\atmL{\octx}_R}{_{\orsig} \p{C}}
    \end{equation*}
    where $\srsig = \srsige$ and $\orsig = \orsige$ is vacuous because there is no definition for $\defp{a}$ in the signature $\orsig$.


  % \item
  %   Consider the case in which
  %   \begin{gather*}
  %     \infer{\chorsig{\theta}{\srsig_0, (v_1 \wc b \wc v_2 \reduces v')}{\orsig_0, (\defp{b} \defd \n{A} \with \n{B})}}{
  %       \begin{array}[b]{@{}c@{}}
  %         \text{($\theta(v_1) = \atmR{\lctx}_L$)} \quad
  %         \text{($\theta(b) = \defp{b}$)} \quad
  %         \text{($\theta(v_2) = \atmL{\lctx}_R$)} \\
  %         \atmR{\lctx}_L \limp \up \bigfuse \theta(v') \pmir \atmL{\lctx}_R \rightsquigarrow \n{B} \quad
  %         \chorsig{\theta}{\srsig_0}{\orsig_0, (\defp{b} \defd \n{A})}
  %       \end{array}}
  %   \shortintertext{and}
  %     \lfocus{\atmR{\octx}_L}{\defp{a}}{\atmL{\octx}_R}{_{\orsig} \p{C}}
  %   \end{gather*}
  %   where $a \neq b$ and $\srsig = \srsig_0, (v_1 \wc b \wc v_2 \reduces v')$ and $\orsig = \orsig_0, (\defp{b} \defd \n{A} \with \n{B})$.

  %   By the inductive hypothesis, there exists a string rewriting axiom $(w_1 \wc a \wc w_2 \reduces w') \in \srsig_0$ such that $\atmR{\octx}_L = \theta(w_1)$, $\atmL{\octx}_R = \theta(w_2)$, and $\p{C} = \bigfuse \theta(w')$.
  % The same axiom is contained in the signature $\srsig$.

  % \item
  %   Consider the case in which
  %   \begin{gather*}
  %     \infer{\chorsig{\theta}{\srsig_0, (v_1 \wc b \wc v_2 \reduces v')}{\orsig_0, (\defp{b} \defd \n{B})}}{
  %       \begin{array}[b]{@{}c@{}}
  %         \text{($\theta(v_1) = \atmR{\lctx}_L$)} \quad
  %         \text{($\theta(b) = \defp{b}$)} \quad
  %         \text{($\theta(v_2) = \atmL{\lctx}_R$)} \\
  %         \atmR{\lctx}_L \limp \up \bigfuse \theta(v') \pmir \atmL{\lctx}_R \rightsquigarrow \n{B} \quad
  %         \chorsig{\theta}{\srsig_0}{\orsig_0} \quad
  %         \text{($\defp{b} \notin \dom{\orsig_0}$)}
  %       \end{array}}
  %   \shortintertext{and}
  %     \lfocus{\atmR{\octx}_L}{\defp{a}}{\atmL{\octx}_R}{_{\orsig} \p{C}}
  %   \end{gather*}
  %   where $a \neq b$ and $\srsig = \srsig_0, (v_1 \wc b \wc v_2 \reduces v')$ and $\orsig = \orsig_0, (\defp{b} \defd \n{B})$.

  %   By the inductive hypothesis, there exists a string rewriting axiom $(w_1 \wc a \wc w_2 \reduces w') \in \srsig_0$ such that $\atmR{\octx}_L = \theta(w_1)$, $\atmL{\octx}_R = \theta(w_2)$, and $\p{C} = \bigfuse \theta(w')$.
  % The same axiom is contained in the signature $\srsig$.

  % \item
  %   Consider the case in which
  %   \begin{gather*}
  %     \infer{\chorsig{\theta}{\srsig_0, (w_1 \wc a \wc w_2 \reduces w')}{\orsig_0, (\defp{a} \defd \n{A} \with \n{B})}}{
  %       \begin{array}[b]{@{}c@{}}
  %         \text{($\theta(w_1) = \atmR{\lctx}_L$)} \quad
  %         \text{($\theta(a) = \defp{a}$)} \quad
  %         \text{($\theta(w_2) = \atmL{\lctx}_R$)} \\
  %         \atmR{\lctx}_L \limp \up \bigfuse \theta(w') \pmir \atmL{\lctx}_R \rightsquigarrow \n{B} \quad
  %         \chorsig{\theta}{\srsig_0}{\orsig_0, (\defp{a} \defd \n{A})}
  %       \end{array}}
  %   \shortintertext{and}
  %     \infer[\lrule{\with}_2]{\lfocus{\atmR{\octx}_L}{\defp{a} = \n{A} \with \n{B}}{\atmL{\octx}_R}{_{\orsig} \p{C}}}{
  %       \lfocus{\atmR{\octx}_L}{\n{B}}{\atmL{\octx}_R}{_{\orsig} \p{C}}}
  %   \end{gather*}
  %   where $\srsig = \srsig_0, (w_1 \wc a \wc w_2 \reduces w')$ and $\orsig = \orsig_0, (\defp{a} \defd \n{A} \with \n{B})$.

  %   By \cref{??}, $\atmR{\octx}_L = \atmR{\lctx}_L = \theta(w_1)$ and $\atmL{\octx}_R = \atmL{\lctx}_R = \theta(w_2)$ and $\p{C} = \bigfuse \theta(w')$.
  %   And the axiom $w_1 \wc a \wc w_2 \reduces w'$ is contained in the signature $\orsig$.

  % \item
  %   Consider the case in which
  %   \begin{gather*}
  %     \infer{\chorsig{\theta}{\srsig_0, (v_1 \wc a \wc v_2 \reduces v')}{\orsig_0, (\defp{a} \defd \n{A} \with \n{B})}}{
  %       \begin{array}[b]{@{}c@{}}
  %         \text{($\theta(v_1) = \atmR{\lctx}_L$)} \quad
  %         \text{($\theta(a) = \defp{a}$)} \quad
  %         \text{($\theta(v_2) = \atmL{\lctx}_R$)} \\
  %         \atmR{\lctx}_L \limp \up \bigfuse \theta(v') \pmir \atmL{\lctx}_R \rightsquigarrow \n{B} \quad
  %         \chorsig{\theta}{\srsig_0}{\orsig_0, (\defp{a} \defd \n{A})}
  %       \end{array}}
  %   \shortintertext{and}
  %     \infer[\lrule{\with}_1]{\lfocus{\atmR{\octx}_L}{\defp{a} = \n{A} \with \n{B}}{\atmL{\octx}_R}{_{\orsig} \p{C}}}{
  %       \lfocus{\atmR{\octx}_L}{\n{A}}{\atmL{\octx}_R}{_{\orsig} \p{C}}}
  %   \end{gather*}
  %   where $\srsig = \srsig_0, (v_1 \wc a \wc v_2 \reduces v')$ and $\orsig = \orsig_0, (\defp{a} \defd \n{A} \with \n{B})$.

  %   Let $\orsig' = \orsig_0 , (\defp{a} \defd \n{A})$.
  %   Then $\lfocus{\atmR{\octx}_L}{\defp{a} = \n{A}}{\atmL{\octx}_R}{_{\orsig'} \p{C}}$.
  %   By inductive hypothesis, there exists an axiom $(w_1 \wc a \wc w_2 \reduces w') \in \srsig_0$ such that $\atmR{\octx}_L = \theta(w_1)$ and $\atmL{\octx}_R = \theta(w_2)$ and $\p{C} = \bigfuse \theta(w')$.
  %   That same axiom is also contained in the signature $\srsig$.

  % \item
  %   Consider the case in which
  %   \begin{gather*}
  %     \infer{\chorsig{\theta}{\srsig_0, (w_1 \wc a \wc w_2 \reduces w')}{\orsig_0, (\defp{a} \defd \n{B})}}{
  %       \begin{array}[b]{@{}c@{}}
  %         \text{($\theta(w_1) = \atmR{\lctx}_L$)} \quad
  %         \text{($\theta(a) = \defp{a}$)} \quad
  %         \text{($\theta(w_2) = \atmL{\lctx}_R$)} \\
  %         \atmR{\lctx}_L \limp \up \bigfuse \theta(w') \pmir \atmL{\lctx}_R \rightsquigarrow \n{B} \quad
  %         \chorsig{\theta}{\srsig_0}{\orsig_0} \quad
  %         \text{($\defp{a} \notin \dom{\orsig_0}$)}
  %       \end{array}}
  %   \shortintertext{and}
  %     \lfocus{\atmR{\octx}_L}{\defp{a} = \n{B}}{\atmL{\octx}_R}{_{\orsig} \p{C}}
  %   \end{gather*}
  %   where $\srsig = \srsig_0, (w_1 \wc a \wc w_2 \reduces w')$ and $\orsig = \orsig_0, (\defp{a} \defd \n{B})$.

  %   By \cref{??}, $\atmR{\octx}_L = \atmR{\lctx}_L = \theta(w_1)$ and $\atmL{\octx}_R = \atmL{\lctx}_R = \theta(w_2)$ and $\p{C} = \bigfuse \theta(w')$.
  %   And the axiom $w_1 \wc a \wc w_2 \reduces w'$ is contained in the signature $\orsig$.


  % % \item
  % %   The case in which
  % % \begin{gather*}
  % %   \infer{\chorsig{\theta}{\srsig, v_1 \oc b \oc v_2 \reduces v'}{\orsig, \defp{a} \defd \n{A}, \defp{b} \defd \n{B}}}{
  % %     \text{($\theta(b) = \defp{b}$)} &
  % %     \chorax{\theta}{v_1}{\bigfuse \theta(v')}{v_2}{\n{B}} &
  % %     \chorsig{\theta}{\srsig}{\orsig, \defp{a} \defd \n{A}} &
  % %     \text{($\defp{b} \notin \dom{\orsig}$)}}
  % %   \\\text{and}\\
  % %   \lfocus{\atmR{\octx}_L}{\defp{a}}{\atmL{\octx}_R}{\p{C}}
  % % \end{gather*}
  % % is similar.

  % % \item
  % % Consider the case in which
  % % \begin{gather*}
  % %   \infer{\chorsig{\theta}{\srsig, v_1 \oc a \oc v_2 \reduces v'}{\orsig, \defp{a} \defd \n{A}_1 \with \n{A}_2}}{
  % %     \text{($\theta(a) = \defp{a}$)} &
  % %     \chorax{\theta}{v_1}{\bigfuse \theta(v')}{v_2}{\n{A}_2} &
  % %     \chorsig{\theta}{\srsig}{\orsig, \defp{a} \defd \n{A}_1}}
  % %   \\\text{and}\\
  % %   \lfocus{\atmR{\octx}_L}{\defp{a}}{\atmL{\octx}_R}{\p{C}}
  % %   \,.
  % % \end{gather*}
  % % There are two cases, according to whether the $\lfocus{\atmR{\octx}_L}{\defp{a}}{\atmL{\octx}_R}{\p{C}}$ derivation ends with the $\lrule{\with}_1$ or $\lrule{\with}_2$ rule.
  % %   \begin{itemize}
  % %   \item If the left-focus derivation ends with the $\lrule{\with}_2$ rule, then $\lfocus{\atmR{\octx}_L}{\n{A}_2}{\atmL{\octx}_R}{\p{C}}$.
  % %     Because $\chorax{\theta}{v_1}{\bigfuse \theta(v')}{v_2}{\n{A}_2}$, it follows from \cref{??} that $\atmR{\octx}_L = \theta(v_1)$ and $\atmL{\octx}_R = \theta(v_2)$ and $\p{C} = \bigfuse \theta(v')$.
  % %     Choose the axiom $w_1 \oc a \oc w_2 \reduces w'$ to be $v_1 \oc a \oc v_2 \reduces v'$.

  % %   \item Otherwise, if the left-focus derivation instead ends with the $\lrule{\with}_1$ rule, then $\lfocus{\atmR{\octx}_L}{\n{A}_1}{\atmL{\octx}_R}{\p{C}}$.
  % %     By the inductive hypothesis, $\atmR{\octx}_L = \theta(w_1)$, $\atmL{\octx}_R = \theta(w_2)$, and $\p{C} = \bigfuse \theta(w')$ for some string rewriting axiom $(w_1 \oc a \oc w_2 \reduces w') \in \srsig$.
  % %     The same axiom is contained in the signuare $\srsig, v_1 \oc a \oc v_2 \reduces v'$.
  % %   \end{itemize}

  % % \item
  % % Consider the case in which 
  % % \begin{equation*}
  % %   \infer{\chorsig{\theta}{\srsig, w_1 \oc a \oc w_2 \reduces w'}{\orsig, \defp{a} \defd \n{A}}}{
  % %     \text{($\theta(a) = \defp{a}$)} &
  % %     \chorax{\theta}{w_1}{\bigfuse \theta(w')}{w_2}{\n{A}} &
  % %     \chorsig{\theta}{\srsig}{\orsig} &
  % %     \text{($\defp{a} \notin \dom{\orsig}$)}}
  % % \end{equation*}
  % % Because $\chorax{\theta}{w_1}{\bigfuse \theta(w')}{w_2}{\n{A}_2}$, it follows from \cref{??} that $\atmR{\octx}_L = \theta(w_1)$ and $\atmL{\octx}_R = \theta(w_2)$ and $\p{C} = \bigfuse \theta(w')$.
  %
  \qedhere
  \end{itemize}
\end{proof}

\begin{theorem}[Soundness]
  If $\chorsig{\theta}{\srsig}{\orsig}$ and $\theta(a) = \defp{a}$ and $\octx_L \oc \defp{a} \oc \octx_R \reduces_{\orsig} \octx'$, then either:
  \begin{itemize}
  \item $\octx_L = \octx'_L \oc \theta(w_1)$ and $\octx_R = \theta(w_2) \oc \octx'_R$ and $\octx' = \octx'_L \oc \theta(w') \oc \octx'_R$ for some contexts $\octx'_L$ and $\octx'_R$ and some strings $w_1$, $w_2$, and $w'$ such that $(w_1 \wc a \wc w_2 \reduces w') \in \srsig$ and $\lfocus{\theta(w_1)}{\defp{a}}{\theta(w_2)}{\bigfuse \theta(w')}$;
  \item $\octx_L \reduces_{\orsig} \octx'_L$ for some context $\octx'_L$ such that $\octx' = \octx'_L \oc \defp{a} \oc \octx_R$; or
  \item $\octx_R \reduces_{\orsig} \octx'_R$ for some context $\octx'_R$ such that $\octx' = \octx_L \oc \defp{a} \oc \octx'_R$.
  \end{itemize}
\end{theorem}
\begin{proof}
  As a negative proposition, $\defp{a}$ serves as a barrier for interactions between $\octx_L$ and $\octx_R$ -- in \ac{PFOR}, implications cannot consume negative propositions.
  Thus, any reduction on $\octx_L \oc \defp{a} \oc \octx_R$ must occur within either $\octx_L$ or $\octx_R$ alone or must arise from $\defp{a}$.

  If the reduction on $\octx_L \oc \defp{a} \oc \octx_R$ arises from $\defp{a}$, then it arises from a bipole that begins by focusing on $\defp{a}$.
  In other words, $\octx_L = \octx'_L \oc \atmR{\lctx}_L$ and $\octx_R = \atmL{\lctx}_R \oc \octx'_R$ and $\octx' = \octx'_L \oc \lctx' \oc \octx'_R$ for some contexts $\atmR{\lctx}_L$, $\atmL{\lctx}_R$, and $\lctx'$ and positive proposition $\p{C}$ such that $\lfocus{\atmR{\lctx}_L}{\defp{a}}{\atmL{\lctx}_R}{\p{C}}$ and $\rfocus{\lctx'}{\p{C}}$.
  By \cref{??}, there exists an axiom $(w_1 \wc a \wc w_2 \reduces w') \in \srsig$ such that $\atmR{\lctx}_L = \theta(w_1)$ and $\atmL{\lctx}_R = \theta(w_2)$ and $\p{C} = \bigfuse \theta(w')$.
  It follows that $\lctx' = \theta(w')$.
\end{proof}

\begin{corollary}[Soundness]
  If $\chorsig{\theta}{\srsig}{\orsig}$ and $\theta(w) \reduces_{\orsig} \octx'$, then $\octx' = \theta(w')$ for some $w'$ such that $w \reduces_{\srsig} w'$.
\end{corollary}

% \begin{theorem}[Soundness]
%   If $\chorsig{\theta}{\srsig}{\orsig}$ and $\theta(w) \reduces_{\orsig} \octx'$, then $\octx' = \theta(w')$ for some $w'$ such that $w \reduces_{\srsig} w'$.
% \end{theorem}
% \begin{proof}
%   By induction over the structure of the given ordered rewriting step, $\theta(w) \reduces_{\orsig} \octx'$.
%   \begin{itemize}
%   \item Consider the case in which
%     \begin{equation*}
%       \chorsig{\theta}{\srsig}{\orsig}
%       \qquad\text{and}\qquad
%       \theta(w) = 
%       \infer[\jrule{$\reduces$C}\smash{_{\jrule{L}}}]{\octx_1 \oc \octx_2 \reduces_{\orsig} \octx'_1 \oc \octx_2}{
%         \octx_1 \reduces_{\orsig} \octx'_1}
%       = \octx'
%       \,.
%     \end{equation*}
%     By inversion, $w = w_1 \oc w_2$ for some $w_1$ and $w_2$ such that $\octx_1 = \theta(w_1)$ and $\octx_2 = \theta(w_2)$.
%     From the inductive hypothesis, it follows that there exists a string $w'_1$ such that $w_1 \reduces_{\srsig} w'_1$ and $\octx'_1 = \theta(w'_1)$.
%     Let $w' = w'_1 \oc w_2$, and notice that $w = w_1 \oc w_2 \reduces_{\srsig} w'_1 \oc w_2 = w'$ and $\octx' = \theta(w'_1) \oc \theta(w_2) = \theta(w')$, as required.

%   \item
%     The case in which
%     \begin{equation*}
%       \chorsig{\theta}{\srsig}{\orsig}
%       \qquad\text{and}\qquad
%       \theta(w) = 
%       \infer[\jrule{$\reduces$C}\smash{_{\jrule{R}}}]{\octx_1 \oc \octx_2 \reduces_{\orsig} \octx_1 \oc \octx'_2}{
%         \octx_2 \reduces_{\orsig} \octx'_2}
%       = \octx'
%     \end{equation*}
%     is symmetric.

%   \item
%     Consider the case in which 
%     \begin{equation*}
%       \chorsig{\theta}{\srsig}{\orsig}
%       \qquad\text{and}\qquad
%       \theta(w) =
%       \infer{\atmR{\octx}_1 \oc \n{A} \oc \atmL{\octx}_2 \reduces_{\orsig} \octx'}{
%         \lfocus{\atmR{\octx}_1}{\n{A}}{\atmL{\octx}_2}{\p{C}} &
%         \rfocus{\octx'}{\p{C}}}
%       \,.
%     \end{equation*}
%     The image $\theta(w)$ can contain a negative proposition $\n{A}$ only if $w = w_1 \oc a \oc w_2$ for some $w_1$, $a$, and $w_2$ such that $\theta(a) = \defp{a}$ with $(\defp{a} \defd \n{A}) \in \orsig$.
%     By inversion, both $\theta(w_1) = \atmR{\octx}_1$ and $\theta(w_2) = \atmL{\octx}_2$ must hold.
%     It then follows from \cref{??} that $\p{C} = \bigfuse \theta(w')$ for some string $w'$ such that $(w_1 \oc a \oc w_2 \reduces w') \in \srsig$.
%     Because $\rfocus{\octx'}{\bigfuse \theta(w')}$ only if $\octx' = \theta(w')$~\parencref{??}, the string $w'$ is such that $w = w_1 \oc a \oc w_2 \reduces_{\srsig} w'$, with $\octx' = \theta(w')$.
%   %
%   \qedhere
%   \end{itemize}
% \end{proof}


\clearpage
\subsection{No choreography}

Not all string rewriting specifications admit a choreography.
For example, the specification
\begin{equation*}
  \infer{a \oc b \reduces b}{}
  \qquad
  \infer{a \reduces \emp}{}
  \qquad\text{and}\qquad
  \infer{b \reduces \emp}{}
\end{equation*}
cannot be given a choreography.
More precisely, there is no choreographing assignment $\theta$ such that $\chorsig{\theta}{\sig}{\sig'}$ is derivable for some signature $\sig'$.
For the sake of contradiction, suppose that $\theta$ were such a choreographing assignment.
Then, for the specification's latter two axioms to be choreographable, both $\theta(a) = \proc{a}$ and $\theta(b) = \proc{b}$ must hold.
In that case, however, the specification's first axiom cannot be choreographed properly because $\theta$ maps more than one of the axiom's symbols to recursively defined propositions.



