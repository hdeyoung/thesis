\chapter{Singleton logic}\label{ch:singleton-logic}



In this \lcnamecref{ch:singleton-logic}, we present one such logic, \emph{singleton logic}.
As its name suggests, singleton logic is the fragment of ordered logic obtained by requiring sequent antecedents to have exactly one hypothesis -- no more and no less.
That such a drastic restriction on the structure of contexts yields a well-defined, computationally useful logic is somewhat surprising.

A discussion of singleton logic's isomorphic\alertnote{Is it really isomorphic?} computational model is postponed to the following \lcnamecref{ch:singleton-processes}.
In this \lcnamecref{ch:singleton-logic}, we confine our attention to the logical aspects of singleton logic.

\section{Bringing symmetry to intuitionistic sequents}

\begin{equation*}
  \infer[\rrule{\limp}?]{\slseq{A |- B_1 \limp B_2}}{
    \slseq{B_1 \fuse A |- B_2}}
\end{equation*}

\begin{equation*}
  \infer[\lrule{\limp}?]{\slseq{A \fuse (B_1 \limp B_2) |- C}}{
    \slseq{A |- B_1} & \slseq{B_2 |- C}}
\end{equation*}

\begin{marginfigure}
  \begin{inferences}
    \infer[\rrule{\pmir}?]{\slseq{A |- B_2 \pmir B_1}}{
      \slseq{A \fuse B_1 |- B_2}}
    \and
    \infer[\lrule{\pmir}?]{\slseq{(B_2 \pmir B_1) \fuse A |- C}}{
      \slseq{A |- B_1} & \slseq{B_2 |- C}}
  \end{inferences}
  \caption{Equally problematic rules for right implication}
\end{marginfigure}

\begin{inferences}
  \infer[\rrule{\fuse}?]{\slseq{A_1 \fuse A_2 |- B_1 \fuse B_2}}{
    \slseq{A_1 |- B_1} & \slseq{A_2 |- B_2}}
  \and
  \infer[\lrule{\fuse}?]{\slseq{B_1 \fuse B_2 |- C}}{
    \slseq{B_1 \fuse B_2 |- C}}
\end{inferences}

\begin{inferences}
  \infer[\rrule{\plus}_1]{\slseq{A |- B_1 \plus B_2}}{
    \slseq{A |- B_1}}
  \and
  \infer[\rrule{\plus}_2]{\slseq{A |- B_1 \plus B_2}}{
    \slseq{A |- B_2}}
  \and
  \infer[\lrule{\plus}]{\slseq{B_1 \plus B_2 |- C}}{
    \slseq{B_1 |- C} & \slseq{B_2 |- C}}
\end{inferences}

\section{Singleton logic}

\begin{syntax*}
  Propositions &
    A & \alpha \mid A \plus B \mid \zero \mid A \with B \mid \top
\end{syntax*}

\begin{inferences}
  \infer[\jrule{CUT}^A]{\slseq{A |- C}}{
    \slseq{A |- B} & \slseq{B |- C}}
  \and
  \infer[\jrule{ID}^A]{\slseq{A |- A}}{}
  \\
  \infer[\rrule{\plus}_1]{\slseq{A |- B_1 \plus B_2}}{
    \slseq{A |- B_1}}
  \and
  \infer[\rrule{\plus}_2]{\slseq{A |- B_1 \plus B_2}}{
    \slseq{A |- B_2}}
  \and
  \infer[\lrule{\plus}]{\slseq{B_1 \plus B_2 |- C}}{
    \slseq{B_1 |- C} & \slseq{B_2 |- C}}
  \\
  \text{(no $\rrule{\zero}$ rule)}
  \and
  \infer[\lrule{\zero}]{\slseq{\zero |- C}}{}
  \\
  \infer[\rrule{\with}]{\slseq{A |- B_1 \with B_2}}{
    \slseq{A |- B_1} & \slseq{A |- B_2}}
  \and
  \infer[\lrule{\with}_1]{\slseq{B_1 \with B_2 |- C}}{
    \slseq{B_1 |- C}}
  \and
  \infer[\lrule{\with}_2]{\slseq{B_1 \with B_2 |- C}}{
    \slseq{B_2 |- C}}
  \\
  \infer[\rrule{\top}]{\slseq{A |- \top}}{}
  \and
  \text{(no $\lrule{\top}$ rule)}
\end{inferences}

This is not the additive fragment of ordered (or linear) logic.

\subsection{Cut reduction}

\begin{equation*}
  \infer[\jrule{CUT}^{B_1 \plus B_2}]{\slseq{A |- C}}{
    \infer[\rrule{\plus}_1]{\slseq{A |- B_1 \plus B_2}}{
      \slseq{A |- B_1}} &
    \infer[\lrule{\plus}]{\slseq{B_1 \plus B_2 |- C}}{
      \slseq{B_1 |- C} & \slseq{B_2 |- C}}}
  %
  \quad\cutreduces\quad
  %
  \infer[\jrule{CUT}^{B_1}]{\slseq{A |- C}}{
    \slseq{A |- B_1} & \slseq{B_1 |- C}}
\end{equation*}

\begin{equation*}
  \infer[\jrule{CUT}^B]{\slseq{A_1 \plus A_2 |- C}}{
    \infer[\lrule{\plus}]{\slseq{A_1 \plus A_2 |- B}}{
      \slseq{A_1 |- B} & \slseq{A_2 |- B}} &
    \slseq{B |- C}}
  %
  \quad\cutreduces
  %
  \infer[\lrule{\plus}]{\slseq{A_1 \plus A_2 |- C}}{
    \infer[\jrule{CUT}^B]{\slseq{A_1 |- C}}{
      \slseq{A_1 |- B} & \slseq{B |- C}} &
    \infer[\jrule{CUT}^B]{\slseq{A_2 |- C}}{
      \slseq{A_2 |- B} & \slseq{B |- C}}}
\end{equation*}

There are no right commutative cut reductions involving left rules -- more symmetric than ordered logic.

\begin{theorem}
  The $\jrule{CUT}^A$ and $\jrule{ID}^A$ rules are admissible in the system having
  \begin{equation*}
    \infer[\normalfont\jrule{ID}^{\alpha}]{\slseq{\alpha |- \alpha}}{}
  \end{equation*}
  as its only judgmental rule.
\end{theorem}

\subsection{Commuting conversions}

\begin{syntax*}
  Proof terms &
    M,N & \begin{array}[t]{@{}l@{}}
            \spawn{M}{N} \mid \fwd \\
              \mathllap{\mid {}} \selectR{\inl}[M] \mid \selectR{\inr}[M] \mid \caseL{\inl => N_1 | \inr => N_2} \\
              \mathllap{\mid {}} \caseL{} \\
              \mathllap{\mid {}} \caseR{\inl => M_1 | \inr => M_2} \mid \selectL{\inl}[N] \mid \selectL{\inr}[N] \\
              \mathllap{\mid {}} \caseR{}
          \end{array}
\end{syntax*}

\begin{inferences}
  \infer[\jrule{CUT}^A]{\slof{A |- \spawn{M}{N} : C}}{
    \slof{A |- M : B} & \slof{B |- N : C}}
  \and
  \infer[\jrule{ID}^A]{\slof{A |- \fwd : A}}{}
  \\
  \infer[\rrule{\plus}_1]{\slof{A |- \selectR{\inl}[M] : B_1 \plus B_2}}{
    \slof{A |- M : B_1}}
  \and
  \infer[\rrule{\plus}_2]{\slof{A |- \selectR{\inr}[M] : B_1 \plus B_2}}{
    \slof{A |- M : B_2}}
  \\
  \infer[\lrule{\plus}]{\slof{B_1 \plus B_2 |- \caseL{\inl => N_1 | \inr => N_2} : C}}{
    \slof{B_1 |- N_1 : C} & \slof{B_2 |- N_2 : C}}
  \\
  \text{(no $\rrule{\zero}$ rule)}
  \and
  \infer[\lrule{\zero}]{\slof{\zero |- \caseL{} : C}}{}
  \\
  \infer[\rrule{\with}]{\slof{A |- \caseR{\inl => M_1 | \inr => M_2} : B_1 \with B_2}}{
    \slof{A |- M_1 : B_1} & \slof{A |- M_2 : B_2}}
  \\
  \infer[\lrule{\with}_1]{\slof{B_1 \with B_2 |- \selectL{\inl}[N] : C}}{
    \slof{B_1 |- N : C}}
  \and
  \infer[\lrule{\with}_2]{\slof{B_1 \with B_2 |- \selectL{\inr}[N] : C}}{
    \slof{B_2 |- N : C}}
  \\
  \infer[\rrule{\top}]{\slof{A |- \caseR{} : \top}}{}
  \and
  \text{(no $\lrule{\top}$ rule)}
\end{inferences}

\begin{gather*}
  \selectR{\inl}[\caseL{\inl => M | \inr => N}] \equiv \caseL{\inl => \selectR{\inl}[M] | \inr => \selectR{\inl}[N]} \\
  % \selectR{\inr}[\caseL{\inl => M | \inr => N}] \equiv \caseL{\inl => \selectR{\inr}[M] | \inr => \selectR{\inr}[N]} \\
  \selectR{\inl}[\selectL{\inl}[M]] \equiv \selectL{\inl}[\selectR{\inl}[M]] \\
  \selectR{\inl}[\selectL{\inr}[M]] \equiv \selectL{\inr}[\selectR{\inl}[M]] \\
  \selectR{\inl}[\caseL{}] \equiv \caseL{} \\
  %
  \selectL{\inl}[\caseR{\inl => M | \inr => N}] \equiv \caseR{\inl => \selectL{\inl}[M] | \inr => \selectL{\inl}[N]} \\
  % \selectR{\inr}[\caseL{\inl => M | \inr => N}] \equiv \caseL{\inl => \selectR{\inr}[M] | \inr => \selectR{\inr}[N]} \\
  \selectL{\inl}[\selectR{\inl}[M]] \equiv \selectR{\inl}[\selectL{\inl}[M]] \\
  \selectL{\inl}[\selectR{\inr}[M]] \equiv \selectR{\inr}[\selectL{\inl}[M]] \\
  \selectL{\inl}[\caseR{}] \equiv \caseR{} \\
  %
  \begin{lgathered}
    \caseR{\inl => \caseL{\inl => M_1 | \inr => M_2}
         | \inr => \caseL{\inl => N_1 | \inr => N_2}} \\[-1ex]
      \quad \equiv \caseL{\inl => \caseR{\inl => M_1 | \inr => N_1}
                        | \inr => \caseR{\inl => M_2 | \inr => N_2}}
  \end{lgathered}
  \\
  \caseR{\inl => \caseL{} | \inr => \caseL{}} \equiv \caseL{} \\
  %
  \caseL{\inl => \caseR{} | \inr => \caseR{}} \equiv \caseR{} \\
  %
  \caseL{} \equiv \caseR{}
\end{gather*}

\subsection{Extensions of singleton logic}

\begin{itemize}
\item Multiplicative units (subsingletons)
\item Exponentials
\end{itemize}

\begin{syntax*}
  Propositions & A & \dotsb \mid \one \\
  Contexts & \sctx & A \mid \sctxe
\end{syntax*}

\begin{inferences}
  \infer[\rrule{\one}]{\slseq{\sctxe |- \one}}{}
  \and
  \infer[\lrule{\one}]{\slseq{\one |- C}}{
    \slseq{\sctxe |- C}}
\end{inferences}

\begin{syntax*}
  Propositions & A & \dotsb \mid \bot \\
  Conseq\relax uents & \cseq & C \mid \cseqe
\end{syntax*}

\begin{inferences}
  \infer[\rrule{\bot}]{\slseq{\sctx |- \bot}}{
    \slseq{\sctx |- \cseqe}}
  \and
  \infer[\lrule{\bot}]{\slseq{\bot |- \cseqe}}{}
\end{inferences}

\begin{syntax*}
  Propositions & A & \dotsb \mid \bang A \\
  Persistent contexts & \uctx & \uctxe \mid \uctx, A
\end{syntax*}

\begin{inferences}
  \infer[\jrule{CUT!}^A]{\slseq{\uctx, A ; \sctx |- \cseq}}{
    \slseq{\uctx ; \sctxe |- A} & \slseq{\uctx, A ; \sctx |- \cseq}}
  \and
  \infer[\jrule{COPY}]{\slseq{\uctx, A ; \sctxe |- \cseq}}{
    \slseq{\uctx, A ; A |- \cseq}}
  \\
  \infer[\rrule{\bang}]{\slseq{\uctx ; \sctxe |- \bang A}}{
    \slseq{\uctx ; \sctxe |- A}}
  \and
  \infer[\lrule{\bang}]{\slseq{\uctx ; \bang A |- \cseq}}{
    \slseq{\uctx, A ; \sctxe |- \cseq}}
\end{inferences}


\begin{inferences}
  \infer[\rrule{\bang\tensor}]{\slseq{\uctx ; \sctx |- A_1 \bang\tensor A_2}}{
    \slseq{\uctx ; \sctxe |- A_1} & \slseq{\uctx ; \sctx |- A_2}}
  \and
  \infer[\lrule{\bang\tensor}]{\slseq{\uctx ; A_1 \bang\tensor A_2 |- \cseq}}{
    \slseq{\uctx, A_1 ; A_2 |- \cseq}}
\end{inferences}


%%% Local Variables:
%%% mode: latex
%%% TeX-master: "thesis"
%%% End:
