\chapter{Singleton logic}\label{ch:singleton-logic}

Ordered logic is a substructural logic.
Although \cref{ch:ordered-logic} presented it from first principles, ordered logic can also be seen as 
%
Are there other logics that may be obtained from ordered logic by reconsidering the structural properties of sequents?

In this \lcnamecref{ch:singleton-logic}, we present one such logic, \emph{singleton logic}.
As its name suggests, singleton logic is the fragment of ordered logic obtained by requiring sequent antecedents to have exactly one hypothesis -- no more and no less.
That such a drastic restriction on the structure of contexts yields a well-defined, computationally useful logic is somewhat surprising.

\section{An algebraic view of structural properties}

Draws from \textcite{ReedPfenning:10}

From an algebraic perspective, persistent contexts form an idempotent, commutative monoid -- equivalently, a bounded semilattice.
\begin{center}
  \begin{tabular}{@{}>{\itshape}r>{$}c<{$}>{\itshape}l@{}}
    \normalfont\textsc{algebraic law} && \normalfont\pbox[b]{0.3\textswidth}{\scshape structural\\[-1ex]property}
    \\[1ex]
    Infimum &
      \begin{array}[b]{@{}c@{}}
        A \circledwedge B \preccurlyeq A \\[-.5ex]
        A \circledwedge B \preccurlyeq B \\[-.5ex]
        (C \preccurlyeq A) \land (C \preccurlyeq B) \Rightarrow (C \preccurlyeq A \circledwedge B)
      \end{array} &
      Weakening
    \\[1.5ex]
    Idempotence & A \preccurlyeq A \circledwedge A & Contraction
    \\[1.5ex]
    Commutativity & A \circledwedge B = B \circledwedge A & Exchange
    \\[1.5ex]
    Associativity & A \circledwedge (B \circledwedge C) = (A \circledwedge B) \circledwedge C & --
    \\[1.5ex]
    Identity & 1 \circledwedge A = A \circledwedge 1 = A & --
  \end{tabular}
\end{center}

\begin{gather*}
  (C = C \circledwedge A) \land (C = C \circledwedge B) \Rightarrow (C = C \circledwedge (A \circledwedge B)) \\
  A_1 \circledwedge A_2 = (A_1 \circledwedge A_2) \circledwedge A_i
\end{gather*}


Linear logic rejects weakening and contraction, so that linear contexts form a commutative monoid.
Ordered logic goes further and also rejects exchange, so that ordered contexts form a monoid.

We can also consider restricting other, more extreme restrictions that go beyond the usual structural properties of weakening, contraction, and exchange to the other algebraic laws.
less apparent structural properties.
Rejecting associativity, for example, leads to \emph{nonassociative logic}\autocite{}\alertnote{Where is this reference?}, in which contexts form a unital magma.
Carrying such restrictions even further, we might ask why contexts should form an algebraic structure at all.
Instead of restricting only the algebraic laws imposed on $\circledwedge$, why not eliminate the binary operation altogether?%
\footnote{There are other, nonalgebraic (\ie, computational) reasons to be interested in this restriction to singleton contexts, but we will postpone a discussion of them to \cref{ch:singleton-processes}.}

In doing so, the effect is to limit contexts to hold exactly one hypothesis.

\section{Singleton logic}

\begin{syntax*}
  Propositions &
    A & \alpha \mid A \plus B \mid \zero \mid A \with B \mid \top
  \\
  Contexts & \sctx & A
\end{syntax*}

\begin{inferences}
  \infer[\jrule{CUT}^A]{\slseq{\sctx |- C}}{
    \slseq{\sctx |- A} & \slseq{A |- C}}
  \and
  \infer[\jrule{ID}^A]{\slseq{A |- A}}{}
  \\
  \infer[\rrule{\plus}_1]{\slseq{\sctx |- A \plus B}}{
    \slseq{\sctx |- A}}
  \and
  \infer[\rrule{\plus}_2]{\slseq{\sctx |- A \plus B}}{
    \slseq{\sctx |- B}}
  \and
  \infer[\lrule{\plus}]{\slseq{A \plus B |- C}}{
    \slseq{A |- C} & \slseq{B |- C}}
  \\
  \text{(no $\rrule{\zero}$ rule)}
  \and
  \infer[\lrule{\zero}]{\slseq{\zero |- C}}{}
  \\
  \infer[\rrule{\with}]{\slseq{\sctx |- A \with B}}{
    \slseq{\sctx |- A} & \slseq{\sctx |- B}}
  \and
  \infer[\lrule{\with}_1]{\slseq{A \with B |- C}}{
    \slseq{A |- C}}
  \and
  \infer[\lrule{\with}_2]{\slseq{A \with B |- C}}{
    \slseq{B |- C}}
  \\
  \infer[\rrule{\top}]{\slseq{\sctx |- \top}}{}
  \and
  \text{(no $\lrule{\top}$ rule)}
\end{inferences}

This is not the additive fragment of linear logic.

\subsection{Cut reduction}

\begin{equation*}
  \infer[\jrule{CUT}^{A \plus B}]{\slseq{\sctx |- C}}{
    \infer[\rrule{\plus}_1]{\slseq{\sctx |- A \plus B}}{
      \slseq{\sctx |- A}} &
    \infer[\lrule{\plus}]{\slseq{A \plus B |- C}}{
      \slseq{A |- C} & \slseq{B |- C}}}
  %
  \quad\cutreduces\quad
  %
  \infer[\jrule{CUT}^A]{\slseq{\sctx |- C}}{
    \slseq{\sctx |- A} & \slseq{A |- C}}
\end{equation*}

There are no right commutative cut reductions involving left rules -- more symmetric than ordered logic.

\begin{theorem}
  The $\jrule{CUT}^A$ and $\jrule{ID}^A$ rules are admissible in the system having
  \begin{equation*}
    \infer[\normalfont\jrule{ID}^{\alpha}]{\slseq{\alpha |- \alpha}}{}
  \end{equation*}
  as its only judgmental rule.
\end{theorem}


\subsection{Extensions of singleton logic}

\begin{itemize}
\item Multiplicative units (subsingletons)
\item Exponentials
\end{itemize}

\begin{syntax*}
  Propositions & A & \dotsb \mid \one \\
  Contexts & \sctx & \dotsb \mid \sctxe
\end{syntax*}

\begin{inferences}
  \infer[\rrule{\one}]{\slseq{\sctxe |- \one}}{}
  \and
  \infer[\lrule{\one}]{\slseq{\one |- C}}{
    \slseq{\sctxe |- C}}
\end{inferences}

\begin{syntax*}
  Propositions & A & \dotsb \mid \bot \\
  Conseq\relax uents & \cseq & C \mid \cseqe
\end{syntax*}

\begin{inferences}
  \infer[\rrule{\bot}]{\slseq{\sctx |- \bot}}{
    \slseq{\sctx |- \cseqe}}
  \and
  \infer[\lrule{\bot}]{\slseq{\bot |- \cseqe}}{}
\end{inferences}

\begin{syntax*}
  Propositions & A & \dotsb \mid \bang A \\
  Persistent contexts & \uctx & \uctxe \mid \uctx, A
\end{syntax*}

\begin{inferences}
  \infer[\jrule{CUT!}^A]{\slseq{\uctx, A ; \sctx |- \cseq}}{
    \slseq{\uctx ; \sctxe |- A} & \slseq{\uctx, A ; \sctx |- \cseq}}
  \and
  \infer[\jrule{COPY}]{\slseq{\uctx, A ; \sctxe |- \cseq}}{
    \slseq{\uctx, A ; A |- \cseq}}
  \\
  \infer[\rrule{\bang}]{\slseq{\uctx ; \sctxe |- \bang A}}{
    \slseq{\uctx ; \sctxe |- A}}
  \and
  \infer[\lrule{\bang}]{\slseq{\uctx ; \bang A |- \cseq}}{
    \slseq{\uctx, A ; \sctxe |- \cseq}}
\end{inferences}


%%% Local Variables:
%%% mode: latex
%%% TeX-master: "thesis"
%%% End:
