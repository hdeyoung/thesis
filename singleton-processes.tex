\chapter{Session-typed processes}\label{ch:singleton-processes}

\begin{itemize}
\item Connections to SILL
\end{itemize}

\section{Session-typed processes: A Curry--Howard interpretation of singleton logic}

\begin{syntax*}
  Session types &
    A & \alpha \mid \plus*[sub=_{\ell \in L}]{\ell:A_{\ell}} \mid \with*[sub=_{\ell \in L}]{\ell:A_{\ell}}
  \\
  Process terms &
    P,Q & \spawn{P}{Q} \mid \fwd
            \begin{array}[t]{@{{} \mid {}}l@{}}
              \selectR{\kay} \mid \caseL[\ell \in L]{\ell => Q_{\ell}} \\
              \caseR[\ell \in L]{\ell => P_{\ell}} \mid \selectL{\kay}
            \end{array}
\end{syntax*}

\begin{inferences}
  \infer[\jrule{CUT}^A]{\slof{\sctx |- \spawn{P}{Q} : C}}{
    \slof{\sctx |- P : A} & \slof{A |- Q : C}}
  \and
  \infer[\jrule{ID}^A]{\slof{A |- \fwd : A}}{}
  \\
  \infer[\rrule{\plus}]{\slof{A_{\kay} |- \selectR{\kay} : \plus*[sub=_{\ell \in L}]{\ell:A_{\ell}}}}{
    \text{($\kay \in L$)}}
  \and
  \infer[\lrule{\plus}]{\slof{\plus*[sub=_{\ell \in L}]{\ell:A_{\ell}} |- \caseL[\ell \in L]{\ell => Q_{\ell}} : C}}{
    \multipremise{\ell \in L}{\slof{A_{\ell} |- Q_{\ell} : C}}}
  \\
  \infer[\rrule{\with}]{\slof{\sctx |- \caseR[\ell \in L]{\ell => P_{\ell}} : \with*[sub=_{\ell \in L}]{\ell:A_{\ell}}}}{
    \multipremise{\ell \in L}{\slof{\sctx |- P_{\ell} : A_{\ell}}}}
  \and
  \infer[\lrule{\with}]{\slof{\with*[sub=_{\ell \in L}]{\ell:A_{\ell}} |- \selectL{\kay} : A_{\kay}}}{
    \text{($\kay \in L$)}}
\end{inferences}

\subsection{Cut reduction}

\begin{gather*}
  \infer[\jrule{CUT}]{\slof{A_{\kay} |- \spawn{\selectR{\kay}}{\caseL[\ell \in L]{\ell => Q_{\ell}}} : C}}{
  \infer[\rrule{\plus}]{\slof{A_{\kay} |- \selectR{\kay} : \plus*[sub=_{\ell \in L}]{\ell:A_{\ell}}}}{
    \text{($\kay \in L$)}} &
  \infer[\lrule{\plus}]{\slof{\plus*[sub=_{\ell \in L}]{\ell:A_{\ell}} |- \caseL[\ell \in L]{\ell => Q_{\ell}} : C}}{
    \multipremise{\ell \in L}{\slof{A_{\ell} |- Q_{\ell} : C}}}}
  \\
  \cutreduces
  \\
  \slof{A_{\kay} |- Q_{\kay} : C}
\end{gather*}

Discussion of cut reduction vs.\ admissibility

\subsection{An operational semantics}

\begin{syntax*}
  Configurations &
    \cnf & \cnfe \mid (\cnf_1 \cc \cnf_2) \mid P
\end{syntax*}

\begin{inferences}
  \infer[\jrule{CUT}^{\sctx'}]{\slcof{\sctx |- \cnf_1 \cc \cnf_2 : \sctx''}}{
    \slcof{\sctx |- \cnf_1 : \sctx'} & \slcof{\sctx' |- \cnf_2 : \sctx''}}
  \and
  \infer[\jrule{ID}^{\sctx}]{\slcof{\sctx |- \cnfe : \sctx}}{}
  \and
  \infer[\jrule{PROC}]{\slcof{\sctx |- P : A}}{
    \slof{\sctx |- P : A}}
\end{inferences}

\begin{inferences}
  \infer{\spawn{P}{Q} \reduces P \cc Q}{}
  \and
  \infer{\fwd \reduces \cnfe}{}
  \\
  \infer{\selectR{\kay} \cc \caseL[\ell \in L]{\ell => Q_{\ell}} \reduces Q_{\kay}}{}
  \and
  \infer{\caseR[\ell \in L]{\ell => P_{\ell}} \cc \selectL{\kay} \reduces P_{\kay}}{}
\end{inferences}

\begin{itemize}
\item SSOS is an ordered rewriting specification.  (How does this work with definitions?)
\end{itemize}

\begin{equation*}
  \begin{lgathered}
    \proc{\spawn{P}{Q}} \defd \proc{P} \fuse \proc{Q} \\
    \proc{\fwd} \defd \one \\
    \proc{\caseL[\ell \in L]{\ell => Q_{\ell}}} \defd \bigwith_{\ell \in L}\bigl(\proc{\selectR{\ell}} \limp \proc{Q_{\ell}}\bigr) \\
    \proc{\caseR[\ell \in L]{\ell => P_{\ell}}} \defd \bigwith_{\ell \in L}\bigl(\proc{P_{\ell}} \pmir \proc{\selectL{\ell}}\bigr)
  \end{lgathered}
\end{equation*}

\paragraph{Hypersequent}



\subsection{Example: Binary counter}

\begin{equation*}
  \proc{b_1} \defd (\msgL{i} \fuse \proc{b_0} \pmir \msgL{i}) \with (\proc{b_0} \fuse \msgR{s} \pmir \msgL{d})
\end{equation*}

\subsection{Example: \Aclp*{DFA}}

Contrast with inability to express \acp{NFA} (languages vs.\ operational semantics)

%%% Local Variables:
%%% mode: latex
%%% TeX-master: "thesis"
%%% End:
