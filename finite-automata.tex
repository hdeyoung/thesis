\chapter{Binary relations and automata}\label{ch:automata}\label{ch:finite-automata}

In this \lcnamecref{ch:automata}, we review the definitions of alphabets, words, languages, and automata that we will use in the running examples throughout this document.
% The classic text by \textcite{Hopcroft+Ullman:??} is a excellent reference.
Our definitions, though equivalent to the classical ones found in \citeauthor{Hopcroft+:06}'s text\autocite{Hopcroft+:06}, differ slightly, having been tuned for the particular applications in this document.

First, however, we describe our notational conventions for binary relations.

\section{Binary relations}

In this and future \lcnamecrefs{ch:automata}, we make significant use of binary relations of various kinds.
These are often written in infix notation.

Given a binary relation $\simu{R}$, we write $\simu{R}^{-1}$ for its inverse.
For relations written as arrows, such as $\reduces$ and $\Reduces$, we often instead express their inverses by writing the arrow in the other direction.
For instance, $\secudeR$ would be the inverse of $\Reduces$, so that $y \secudeR x$ exactly when $x \Reduces y$.

We write the relational composition of $\simu{R}$ and $\simu{S}$ as juxtaposition, so that $x \simu{R}\simu{S} z$ holds exactly when there exists a $y$ such that $x \simu{R} y$ and $y \simu{S} z$.
\change[ap]{In particular, we often use relational composition in the various definitions of bisimilarity that are presented, so that one of the relations is some form of transition or reduction, which are often written as arrows.
For instance, $x \simu{R}\reduces z$ would hold exactly when there exists a $y$ such that $x \simu{R} y$ and $y \reduces z$.}

\section{Alphabets, words, and languages}

An alphabet $\ialph$ is simply a set of symbols, $a \in \ialph$.
A finite word $w$ over the alphabet $\ialph$ is then a (possibly empty) finite sequence of symbols drawn from $\ialph$;
we denote the empty word by $\emp$.
The finite words form a free monoid under concatenation, with $\emp$ being the unit.
We denote by $\finwds{\ialph}$ the set of all finite words over $\ialph$.

It is also possible to construct infinite words.
An infinite word over the alphabet $\ialph$ is a countably infinite sequence of letters drawn from $\ialph$;
we denote the set of all infinite words over $\ialph$ by $\infinwds{\ialph}$.
We also use $\wds{\ialph}$ to denote the set of all words -- finite or infinite -- over the alphabet $\ialph$; that is, $\wds{\ialph} = \finwds{\ialph} \union \infinwds{\ialph}$.

A language is a set of words.
Depending on the context, it will be a subset of either $\wds{\ialph}$, $\infinwds{\ialph}$, or $\finwds{\ialph}$.

% It will sometimes be useful to work with an augmented alphabet.
% Given a symbol $c \notin \ialph$, we may form the augmented alphabet $\augalph{\ialph}{c} = \ialph \union \{c\}$.
% The augmented alphabet $\augalph{\ialph}{\emp}$ is the most

% \subsection{Endmarked alphabets and words}

% An endmarked alphabet $\ealph{\ialph}$ is a pair $\ealph{\ialph} = (, )$ of finite alphabets $ $ and $ $.

\section{Nondeterministic and \aclp*{DFA}}

\begin{definition}\label{def:finite-automata:nfa}
  \emph{\Iac{NFA}}
   over a finite input alphabet $\ialph$ is a triple $\aut{A} = (Q, \nfapow, F)$ consisting of:
  \begin{itemize}
  \item a finite set of \vocab{states}, $Q$;
  \item a \vocab{transition function}, $\nfapow\colon Q \times \ialph \to \pow{Q}$ such that $\nfapow(q, a) \neq \emptyset$ for all states $q \in Q$ and input symbols $a \in \ialph$; and
  \item a subset of \vocab{final states}, $F \subseteq Q$,
  \end{itemize}
  If $q' \in \nfapow(q, a)$, then we say that $q'$ is an \emph{$a$-successor} of $q$ and write $q \nfareduces[a] q'$.%
  \footnote{The condition placed on $\nfapow$ thus serves to ensure that, for all input symbols $a$, each state $q$ has an $a$-successor -- that is, that the \ac{NFA} $\aut{A}$ cannot get stuck.}

  The transition function $\nfapow$ can be lifted to a relation involving finite input words: For each word $w = a_1 \wc a_2 \dotsm a_n \in \finwds{\ialph}$, define a relation $\mathord{\nfareduces[\smash{w}]} \subseteq Q \times Q$ such that $q \nfareduces[w] q'$ when $q = q_0 \nfareduces[a_1] q_1 \nfareduces[a_2] \dotsb \nfareduces[a_n] q_n = q'$ for some sequence of states $q_0, q_1, \dotsc, q_n \in Q$.

  The \ac{NFA} $\aut{A}$ accepts input word $w$ from state $q$ if there exists a state $q' \in Q$ such that $q \nfareduces[w] q' \in F$;
  otherwise, the automaton rejects word $w$ from state $q$.
  The language of all words accepted by automaton $\aut{A}$ from state $q$ is denoted by $\autlang{\aut{A}}{q}$.%
  \footnote{We sometimes omit the subscript if the automaton is clear from the context.}
\end{definition}

Notice that, unlike the classical definition of \acp{NFA}, this definition does not fix an initial state for the automaton.
This is because we will be primarily interested in the operational aspects of \iac{NFA}, rather than its linguistic aspects.

\begin{example}
  \begin{marginfigure}[4.5\baselineskip]
    \begin{equation*}
      \mathllap{\aut{A}_1 = {}}
      \begin{tikzpicture}[baseline=(q_0.base)]
        \graph [automaton] {
          q_0
           -> [loop above, "a,b"]
          q_0
           -> ["b"]
          q_1 [accepting]
           -> ["a,b"]
          q_2
           -> [loop above, "a,b"]
          q_2;
        };
      \end{tikzpicture}
    \end{equation*}
    \caption{\Iac*{NFA} that accepts, from state $q_0$, exactly those words that end with $b$.}\label{fig:nfa-example-ends-b}
  \end{marginfigure}
  %
  As a concrete example, consider the \ac{NFA} $\aut{A}_1$ over the input alphabet $\ialph = \set{a, b}$ that is depicted in \cref{fig:nfa-example-ends-b}.
  This \ac{NFA} accepts, from state $q_0$, exactly those words that end with $b$.
  For comparison, the only word accepted from state $q_1$ is $\emp$.
  This \ac{NFA} is indeed nondeterministic, as both $q_0 \nfareduces[\smash{b}] q_0$ and $q_0 \nfareduces[\smash{b}] q_1$ hold.
\end{example}

\begin{definition}
  \emph{\Iacf{DFA}} over a finite input alphabet $\ialph$ is \iac{NFA} $\aut{A} = (Q, \nfapow, F)$ over $\ialph$ in which $\nfapow(q, a)$ is a singleton set for all states $q$ and input symbols $a$.
  In this case, we write $\dfanext$ for the function from $Q \times \ialph$ to $Q$ that underlies $\nfapow$.
\end{definition}

\begin{example}
  \begin{marginfigure}[0.8\baselineskip]
    \begin{equation*}
      \mathllap{\aut{A}_2 = {}}
      \begin{tikzpicture}[baseline=(s_0.base)]
        \graph [automaton] {
          s_0
           -> [loop above, "a"]
          s_0
           -> [bend left, "b"]
          s_1 [accepting]
           -> [loop above, "b"]
          s_1
           -> [bend left, "a"]
          s_0;
        };
      \end{tikzpicture}
    \end{equation*}
    \caption{\Iac*{DFA} that accepts, from state $s_0$, exactly those words that end with $b$.}\label{fig:dfa-example-ends-b}
  \end{marginfigure}
  \Cref{fig:dfa-example-ends-b} depicts \iac{DFA} over the input alphabet $\ialph = \set{a, b}$ that accepts, from state $s_0$, exactly those words that end with $b$.
  For comparison, the empty word $\emp$, too, is accepted from the state $s_1$.
\end{example}

\subsection{\acs*{NFA} bisimilarity}\label{sec:finite-automata:bisimilarity}

In later \lcnamecrefs{ch:ordered-bisimilarity}, we will refer to a standard notion of bisimilarity for \acp{NFA}.

In general, two objects are bisimilar if they cannot be distinguished by an observer.
Here, \ac{NFA} bisimilarity is a relation on states, and the observer may provide an input word and observe whether the word is accepted or rejected by the given state.
% Two \ac{NFA} states are bisimilar if their successors are bisimilar and if they behave equivalently on the empty word, $\emp$.
%
\begin{definition}% [\ac*{NFA} bisimilarity]
  Let $\aut{A} = (Q, \nfapow, F)$ be \iac{NFA} over an input alphabet $\ialph$.
  An \emph{\acs{NFA} bisimulation} on $\aut{A}$ is a \emph{symmetric} binary relation on states, $\mathord{\simu{R}} \subseteq Q \times Q$, that satisfies the following conditions.
  \begin{thmdescription}[noitemsep]
  \item[Input bisimulation]
    If $q \simu{R}\nfareduces[a] s'$, then $q \nfareduces[a]\simu{R} s'$.%
    % (See the adjacent \lcnamecref{fig:nfa-bisim:diagrams}.)% 
    \begin{marginfigure}
      \begin{equation*}
        % \begin{tikzcd}
        %   q \rar[reduces]{a} \dar[relation][swap]{\simu{R}}
        %     & q\mathrlap{'} \dar[relation, exists]{\simu{R}}
        %   \\[.5ex]
        %   s \rar[reduces, exists]{a} & s\smash{\mathrlap{'}}
        % \end{tikzcd}
        % \qquad
        \begin{tikzcd}
          q \rar[reduces, exists]{a} \dar[relation][swap]{\simu{R}}
            & q\mathrlap{'} \dar[relation, exists]{\simu{R}}
          \\[.5ex]
          s \rar[reduces]{a} & s\smash{\mathrlap{'}}
        \end{tikzcd}
        \quad\text{\raisebox{-1ex}{and}}\quad
        \begin{tikzcd}
          q \dar[relation][swap]{\simu{R}}
            \drar[relation, exists]{\textstyle\in}
          \\[.5ex]
          s \rar[phantom]{\in} &[-2em] F
        \end{tikzcd}
      \end{equation*}
      \caption{\Acs*{NFA} bisimilarity, in diagrams}\label{fig:nfa-bisim:diagrams}
    \end{marginfigure}%
    \footnote[][\baselineskip]{\change[ic]{Recall that we use juxtaposition of relations to denote relational composition, so that, for example, $q \simu{R}\nfareduces[\smash{a}] s'$ holds exactly when there exists a state $s$ such that $q \simu{R} s$ and $s \nfareduces[\smash{a}] s'$.
    Similarly, $q \simu{R} s \in F$ holds if both $q \simu{R} s$ and $s \in F$ hold.}}

  \item[Finality bisimulation]
    If $q \simu{R} s \in F$, then $q \in F$.
  \end{thmdescription}
  \vocab{\ac{NFA} bisimilarity} on $\aut{A}$, written $\asim_{\aut{A}}$, is the largest bisimulation on $\aut{A}$.
  We will usually omit the subscript because the automaton is nearly always clear from context.
\end{definition}

As a matter of convenience, \ac{NFA} bisimilarity is defined on a single \ac{NFA}.
If we wish to discuss the bisimilarity of states from distinct \acp{NFA}, we can form the disjoint union of the two \acp{NFA} and work with its bisimilarity relation.

Bisimilarity is an equivalence relation.
% Here we choose to keep the two automata distinct and prove reflexivity, symmetry, and transitivity, not in their strictest form, but in forms that apply to bisimilarity as defined above.
\begin{theorem}\label{thm:nfa-bisim-equiv}
  Let $\aut{A} = (Q, \nfapow, F)$ be \iac{NFA} over an input alphabet $\ialph$.
  \Ac{NFA} bisimilarity on $\aut{A}$ is reflexive, symmetric, and transitive:
  \begin{thmdescription}[nosep]
  \item[Reflexivity] Equality on $Q$ is a bisimulation on $\aut{A}$.
  \item[Symmetry] If $\simu{R}$ is a bisimulation on $\aut{A}$, then so is $\simu{R}^{-1}$.
  \item[Transitivity] If $\simu{R}$ and $\simu{S}$ are bisimulations on $\aut{A}$, then so is $\simu{R}\simu{S}$.
  \end{thmdescription}
\end{theorem}
\begin{proof}
  By the definition of bisimulation.
\end{proof}

% \begin{theorem}\label{thm:nfa-bisim-refl}
%   Let $\aut{A} = (Q, \nfapow, F)$ be \iac{NFA} over an input alphabet $\ialph$.
%   Then equality on $Q$ is a bisimulation between $\aut{A}$ and itself.
%   Moreover, $q \asim q$ for all $q \in Q$.
% \end{theorem}
% \begin{proof}
%   By checking that equality on $Q$ satisfies the conditions of a bisimulation between $\aut{A}$ and itself.
% \end{proof}

% \begin{theorem}\label{thm:nfa-bisim-sym}
%   Let $\aut{A}_1 = (Q_1, \nfapow_1, F_1)$ and $\aut{A}_2 = (Q_2, \nfapow_2, F_2)$ be \acp{NFA} over an input alphabet $\ialph$, and let $\simu{R}$ be \iac{NFA} bisimulations between $\aut{A}_1$ and $\aut{A}_2$.
%   Then its relational inverse, $\simu{R}^{-1}$, is a bisimulation between $\aut{A}_2$ and $\aut{A}_1$.
%   Moreover, for all $q_1 \in Q_1$ and $q_2 \in Q_2$, if $q_1 \asim q_2$, then $q_2 \asim q_1$.
% \end{theorem}
% \begin{proof}
%   By checking that $\simu{R}^{-1}$ satisfies the conditions of a bisimulation between $\aut{A}_2$ and $\aut{A}_1$.
% \end{proof}

% \begin{theorem}\label{thm:nfa-bisim-trans}
%   Let $\aut{A}_1 = (Q_1, \nfapow_1, F_1)$, $\aut{A}_2 = (Q_2, \nfapow_2, F_2)$, and $\aut{A}_3 = (Q_3, \nfapow_3, F_3)$ be \acp{NFA} over an input alphabet $\ialph$, and let $\simu{R}$ and $\simu{S}$ be \ac{NFA} bisimulations between $\aut{A}_1$ and $\aut{A}_2$ and between $\aut{A}_2$ and $\aut{A}_3$, respectively.
%   Then their relational composition, $\simu{R}\simu{S}$, is a bisimulation between $\aut{A}_1$ and $\aut{A}_3$.
%   Moreover, for all $q_1 \in Q_1$ and $q_2 \in Q_2$ and $q_3 \in Q_3$, if $q_1 \asim q_2$ and $q_2 \asim q_3$, then $q_1 \asim q_3$.
% \end{theorem}
% \begin{proof}
%   By checking that $\simu{R}\simu{S}$ satisfies the conditions of a bisimulation between $\aut{A}_1$ and $\aut{A}_3$.
%   % By proving that $\Set{ (q, r) \in Q_1 \times Q_3 \given \exists s \in Q_2.\, (q \asim s) \land (s \asim r) }$ is \iac{NFA} bisimulation between $\aut{A}_1$ and $\aut{A}_3$.
% \end{proof}

% % Notice that these \lcnamecrefs{thm:nfa-bisim-sym} are symmetry- and transitivity-like properties, but because distinct automata are used, they cannot be symmetry and transitivity in a strict sense.
% When \ac{NFA} bisimilarity is employed as an endorelation on the states of a single \ac{NFA}, $\aut{A}$, bisimilarity is a bona fide equivalence relation.%
% %
% \begin{corollary}
%   Let $\aut{A}$ be \iac{NFA} over an input alphabet $\ialph$.
%   \Ac{NFA} bisimilarity on $\aut{A}$ is reflexive, symmetric, and transitive.
% \end{corollary}
% % %
% % \begin{proof}
% %   \Ac{NFA} bisimilarity can be proved to be reflexive by showing that the state equality relation is \iac{NFA} bisimulation.
% %   \Cref{thm:nfa-bisim-sym,thm:nfa-bisim-trans} prove that \ac{NFA} bisimilarity on $\aut{A}$ is symmetric and transitive.
% % \end{proof}

The input bisimulation condition satisfied by \iac{NFA} bisimulation can be lifted to a condition on words, not just input symbols.
%
\begin{theorem}\label{thm:nfa-bisim:words}
  Let $\aut{A} = (Q, \nfapow, F)$ be \iac{NFA} over an input alphabet $\ialph$, and let $\simu{R}$ be \iac{NFA} bisimulation on $\aut{A}$.
  % Then $s \simu{R}^{-1}\nfareduces[w] q'$ implies $s \nfareduces[w]\simu{R}^{-1} q'$; moreover, 
  Then $q \simu{R}\nfareduces[w] s'$ implies $q \nfareduces[w]\simu{R} s'$.
\end{theorem}
%
\begin{proof}
  By induction over the structure of word $w$.
\end{proof}


\Ac{NFA} bisimilarity implies language equivalence.
%
\begin{theorem}
  Let $\aut{A} = (Q, \nfapow, F)$ be \iac{NFA} over an input alphabet $\ialph$.
  Then $q \asim s$ implies $\autlang{\aut{A}}{q} = \autlang{\aut{A}}{s}$.
\end{theorem}
%
\begin{proof}
  Because bisimilarity is symmetric~\parencref{thm:nfa-bisim-equiv}, it suffices to show that $q \asim s$ implies $\autlang{\aut{A}}{q} \subseteq \autlang{\aut{A}}{s}$.
  Let $q \in Q$ and $s \in Q$ be bisimilar states, and choose an arbitrary word $w$ that is accepted from state $q$.
  By definition, $q \nfareduces[w] q'_w \in F$ for some state $q'_w$.
  It follows from \cref{thm:nfa-bisim:words} and the finality bisimulation condition that $s \nfareduces[w] s'_w \in F$, for some state $s'_w$, and so $w$ is also accepted from state $s$.
\end{proof}

But, because of nondeterminism, the converse does not hold.
\begin{falseclaim}
  Let $\aut{A} = (Q, \nfapow, F)$ be \iac{NFA} over an input alphabet $\ialph$.
  Then $\autlang{\aut{A}}{q} = \autlang{\aut{A}}{s}$ implies $q \asim s$.
\end{falseclaim}
%
\begin{proof}[Counterexample]
  Choose the \acp{NFA} $\aut{A}_1$ and $\aut{A}_2$ given in \cref{fig:nfa-example-ends-b,fig:dfa-example-ends-b}.
  Although the languages accepted by states $q_0$ and $s_0$ are the same, the two states are \emph{not} bisimilar.

  For the sake of deriving a contradiction, assume that $q_0 \asim s_0$.
  % and its symmetric reflection, $s_0 \asim q_0$.
  Because $q_0$ is one of the $b$-successors of $q_0$, it follows by the input bisimulation condition that $s_0 \nfareduces[\smash{b}]\asim q_0$.
  But $s_1$ is the unique $b$-successor of $s_0$, and so we may deduce that $s_1 \asim q_0$.
  Just as $s_1$ is a final state, the finality bisimulation condition demands that $q_0$ be final, which it is not.
  From this contradiction, we conclude that $q_0$ and $s_0$ are \emph{not} bisimilar. 
\end{proof}

However, if both automata are \acp{DFA}, then language equivalence does imply bisimilarity.
%
\begin{theorem}
  Let $\aut{A} = (Q, \dfanext, F)$ be \emph{\iac{DFA}} over an input alphabet $\ialph$.
  Then $\autlang{\aut{A}}{q} = \autlang{\aut{A}}{s}$ implies $q \asim s$.
\end{theorem}
%
\begin{proof}
  Let $\mathord{\simu{R}} = \Set{ (q, s) \given \autlang{\aut{A}}{q} = \autlang{\aut{A}}{s} }$; we will prove that $\simu{R}$ is a bisimulation on the \ac{DFA} $\aut{A}$.
  As the largest bisimulation, bisimilarity will then contain $\simu{R}$.
  %
  \begin{description}[parsep=0pt, listparindent=\parindent]
  \item[Input bisimulation]
    Assume that $q \simu{R} s \dfareduces[a] s'_a$ for some state $s$; we must show that $q \dfareduces[a]\simu{R} s'_a$.
    Because $\aut{A}$ is deterministic, it suffices to show that $\autlang{\aut{A}}{q'_a} = \autlang{\aut{A}}{s'_a}$, where $q'_a$ is the unique $a$-successor of $q$.

    Choose an arbitrary word $w$ from the language accepted from state $q'_a$.
    Then $a \wc w$ is in the language accepted from state $q$, and, because $q$ and $s$ are $\simu{R}$-related, also in the language accepted from $s$.
    Because $\aut{A}$ is deterministic, this can only be if $w$ is in the language accepted from state $s'_a$, the unique $a$-successor of $s$.
    Thus $\autlang{\aut{A}}{q'_a} \subseteq \autlang{\aut{A}}{s'_a}$.
    By symmetric reasoning, $\autlang{\aut{A}}{q'_a} \supseteq \autlang{\aut{A}}{s'_a}$.
    It follows that $q'_a \simu{R} s'_a$.

  \item[Finality bisimulation]
    Assume that $q \simu{R} s \in F$; we must show that $q \in F$.
    Because $s$ is a final state, it accepts the empty word, $\emp$.
    The state $q$ must also accept the empty word, because $\autlang{\aut{A}}{q} = \autlang{\aut{A}}{s}$.
    A state accepts the empty word only if it is a final state, so it follows that $q \in F$.
  %
  \qedhere
  \end{description}
\end{proof}


% \subsection{\Aclp*{NFA} with $\emp$-transitions}

% \begin{definition}
%   \Iac{NFA} with $\emp$-moves over a finite alphabet $\ialph$ is a triple $\aut{A} = (Q, \mathord{\nfareduces}, F)$ consisting of:
%   \begin{itemize}
%   \item a finite set of \vocab{states}, $Q$;
%   \item a \vocab{transition relation}, $\mathord{\nfareduces} \subseteq (Q \times \ialph) \times Q$; and
%   \item a subset of \vocab{final states}, $F \subseteq Q$.
%   \end{itemize}
%   We will write $q \nfaReduces[a] q'$ whenever $q \nfareduces[\emp] \dotsb \nfareduces[\emp] \nfareduces[a] \nfareduces[\emp] \dotsb \nfareduces[\emp] q'$.

%   The transition relation can be lifted to one involving finite input words: For each input word $w = a_1 \wc a_2 \dotsm a_n$, let $q \nfaReduces[w] q'$ if $q = q_0 \nfaReduces[a_1] q_1 \nfaReduces[a_2] \dotsb \nfaReduces[a_n] q_n = q'$ for some sequence of states $q_0, q_1, \dotsc, q_n \in Q$.

%   The automaton $\aut{A}$ accepts input word $w$ from state $q$ if there exists a state $q' \in Q$ such that $q \nfaReduces[w] q' \in F$;
%   otherwise, the automaton rejects word $w$ from state $q$.
% \end{definition}

% \begin{marginfigure}
%   \begin{tikzcd}
%     \graph {
      
%   \end{tikzcd}
% \end{marginfigure}


\section{Infinite-word sequential transducers}

Sequential and subsequential transducers\autocites{Ginsburg+Rose:CJM66}{Schuetzenberger:TCS77}  are usually described in terms of finite words.
In this document, we are interested in transducers for their computational behavior rather than for the functions they induce.
This, together with technical considerations that will become apparent in \cref{ch:process-chains}, means that we are only interested in \emph{infinite}-word sequential transducers.
% It is also possible to consider finite-word sequential (and subsequential) transducers, but in this document we are only interested in infinite-word sequential transducers, for reasons that will become clear in \cref{ch:process-chains}.

The following definition is adapted from \textfootcite{Beal+Carton:TCS02}.
\begin{definition}
  An \emph{infinite-word sequential transducer} over the finite input and output alphabets $\ialph$ and $\oalph$, respectively, is a triple $\aut{T} = (Q, \sftnext, \sftout)$ consisting of:
  \begin{itemize}
  \item a finite set of \vocab{states}, $Q$;
  \item a \vocab{transition function}, $\sftnext\colon Q \times \ialph \to Q$; and
  \item an \vocab{output function}, $\sftout\colon Q \times \ialph \to \finwds{\oalph}$.
  \end{itemize}
  We may define a function $\mathord{\Downarrow}\colon Q \times \infinwds{\ialph} \to \infinwds{\oalph}$ as the largest function such that $(q, a \wc w) \Downarrow \sftout(q, a) \wc v$ if $(\sftnext(q, a), w) \Downarrow v$.
  The transducer $\aut{T}$ maps, from state $q$, the infinite input word $w$ into the infinite output word $v$ if $(q, w) \Downarrow v$.
\end{definition}


\begin{example}
\begin{marginfigure}[5\baselineskip]
  \begin{equation*}
    \mathllap{\aut{T} = {}}
    \begin{tikzpicture}[baseline=(q_0.base)]
      \graph [automaton] {
        q_0
         -> [loop above, "$\tio{a | a}$"]
        q_0
         -> [bend left, "$\tio{b | b}$"]
        q_1 [right=-0.5em of q_0]
         -> [loop above, "$\tio{b | \emp}$"]
        q_1
         -> [bend left, "$\tio{a | a}$"]
        q_0 ;
        % e_0 [coordinate, below=-1.75em of q_0.south];
        % e_1 [coordinate, below=-5.25em of q_1.south];
        % (q_0.south) -> ["$\emp$", swap] e_0 ;
        % (q_1.south) -> ["$\vphantom{\emp}\smash{b}$"] e_1 ;
      };
    \end{tikzpicture}
  \end{equation*}
  \caption{An infinite-word sequential transducer that compresses runs of consecutive $b$s}\label{fig:finite-automata:sft-example}
\end{marginfigure}
  %
  As a concrete example consider the infinite-word transducer $\aut{T}$ over the input and output alphabets $\ialph = \oalph = \Set{ a, b }$ that is depicted in \cref{fig:finite-automata:sft-example}.
  This transducer maps, from state $q_0$, infinite input words into words in which each run of $b$s has been compressed into a single $b$.
  For instance, it maps $w = abbaabbba\dotsm$ to $v = abaaba\dotsm$ because $(q_0, w) \Downarrow v$.
\end{example}



% Notice that \acp{DFA} may be thought of as specialized \acp{SFT} that transduce each word to one of two output words to indicate acceptance.
% In other words, \iac{DFA} $\aut{A} = (Q, \dfanext, F)$ over an input alphabet $\ialph$ is isomorphic to the \ac{SFT} $\aut{T} = (Q, \dfanext, \sftout, \sftterm)$ over the in put aphabet $\ialph$ and an output alphabet $\oalph = \Set{ y, n }$ where:
% $\sftout(q, a) = \emp$ for all $(q, a) \in Q \times \ialph$; and
% $\sftterm(q) = y$ for all $q \in F$ and $\sftterm(q) = n$ otherwise.


% \section{\Aclp*{DPDA}}

% \begin{definition}
%   \Iac{DPDA} over a finite input alphabet $\ialph$ and a finite stack alphabet $\salph$ is a triple $\aut{A} = (Q, \mathord{\nfareduces}, F)$ consisting of:
%   \begin{itemize}
%   \item a finite set of \vocab{states}, $Q$;
%   \item a \vocab{transition relation} on state-stack pairs, $\mathord{\nfareduces} \subseteq (Q \times \finwds{\salph}) \times \augalph{\ialph}{\emp} \times (Q \times \finwds{\salph})$; and
%   \item a subset of \vocab{final states}, $F \subseteq Q$.
%   \end{itemize}

%   We will write $q \nfareduces[a] q'$ whenever $((q, a), q') \in \mathord{\nfareduces}$.

%   The transition relation can be lifted to one involving finite input words: For each word $w = \alpha_1 \wc \alpha_2 \dotsm \alpha_n \in \finwds{\ialph}$, let $q \nfareduces[w] q'$ if $q = q_0 \nfareduces[\alpha_1] q_1 \nfareduces[\alpha_2] \dotsb \nfareduces[\alpha_n] q_n = q'$ for some sequence of states $q_0, q_1, \dotsc, q_n \in Q$.

%   The \ac{DPDA} $\aut{A}$ accepts input word $w$ from state $q$ if there exists a state $q' \in Q$ such that $q \nfareduces[w] q' \in F$;
%   otherwise, the automaton rejects word $w$ from state $q$.
%   The language of all words accepted by automaton $\aut{A}$ from state $q$ is denoted $\autlang{\aut{A}}{q}$.%
%   \footnote{We sometimes omit the subscript if the automaton is clear from the context.}

%   $(q, s) = (q_0, s_0) \nfareduces[\alpha_1] (q_1, s_1) \nfareduces[\alpha_2] \dotsb \nfareduces[\alpha_n] (q', s')$ with $q' \in F$.
% \end{definition}

\section{Turing machines}

In \cref{sec:process-chains:turing-machines}, we will construct session-typed processes that represent Turing machines.
We are interested in how these machines compute, but not interested in what functions they compute.
In other words, our focus is on computation, not computability.

This means that our definitions of Turing machines differ from the definitions traditionally used, such as in \citeauthor{Hopcroft+:06}'s text\autocite{Hopcroft+:06}.
For example, we use infinite words to describe truly infinite tapes, rather than using finite words to describe the frontiers of unbounded tapes, and our machines also have no notion of acceptance.
The specific reasons for these differences will become clear in \cref{sec:process-chains:turing-machines}.

\begin{definition}
  A \vocab{two-way infinite tape Turing machine} over a finite alphabet $\ialph$ is a pair $\aut{M} = (Q, \delta)$ consisting of:
  \begin{itemize}
  \item a finite set of \emph{states}, $Q$; and
  \item a \emph{transition function} $\delta\colon Q \times \ialph \to Q \times \ialph \times \Set{\mathsf{L}, \mathsf{R}}$.
  \end{itemize}
  The two-way infinite tape is divided into two one-way halves, with the machine's \emph{head} placed between the two halves.
  A \emph{configuration} of $\aut{M}$ is thus either $w \itlhead{q} v$ or $w \itrhead{q} v$ for left-infinite word $w \in \ialph^{\bar{\omega}}$, right-infinite word $v \in \infinwds{\ialph}$, and state $q \in Q$.%
  \footnote{In other words, configurations are drawn from $\ialph^{\bar{\omega}} \times (\bigcup_{q \in Q} \Set{\!{} \itlhead{q} {}, {} \itrhead{q} {}\!}) \times \infinwds{\ialph}$.}
  The machine's head faces either the left- or right-hand half of the tape, as indicated by the notation surrounding the state $q$.

  To describe the machine's moves, we define a function $\treduces$ on configurations.
  For a left-facing head, this function is given by:
  \begin{equation*}
      w \wc a \itlhead{q} v \treduces
        \begin{cases*}
          w \itlhead{q'} b \wc v & if $\delta(q, a) = (q', b, \mathsf{L})$ \\
          w \wc b \itrhead{q'} v & if $\delta(q, a) = (q', b, \mathsf{R})$
        \end{cases*}
  \end{equation*}
  Symmetrically, for a right-facing head, this function is given by:
  \begin{equation*}
      w \itrhead{q} a \wc v \treduces
        \begin{cases*}
          w \itlhead{q'} b \wc v & if $\delta(q, a) = (q', b, \mathsf{L})$ \\
          w \wc b \itrhead{q'} v & if $\delta(q, a) = (q', b, \mathsf{R})$
        \end{cases*}
  \end{equation*}
  The direction that the head faces indicates the next symbol to be read from the tape.
  When a left-facing head is instructed to move right or a right-facing head is instructed to move left, the head's direction changes but its placement between the two tape halves does not.
\end{definition}


% \begin{definition}
%   A \vocab{two-way infinite tape Turing machine} over a finite alphabet $\ialph$ is a tuple $\aut{M} = (Q, \tblank, \delta)$ consisting of:
%   \begin{itemize}
%   \item a finite set of \emph{states}, $Q$;
%   \item a distinguished \emph{blank symbol}, $\tblank \in \ialph$; and
%   \item a \emph{transition function} $\delta\colon Q \times \ialph \to Q \times \ialph \times \Set{\mathsf{L}, \mathsf{R}}$.
%   \end{itemize}
%   The two-way infinite tape is divided into two one-way halves, with the machine's \emph{head} placed between the two halves.
%   A \emph{configuration} of $\aut{M}$ is thus either $w \itlhead{q} v$ or $w \itrhead{q} v$ for finite words $w, v \in \finwds{\ialph}$ and state $q \in Q$.%
%   \footnote{In other words, configurations are drawn from $\finwds{\ialph} \times (\bigcup_{q \in Q} \Set{\!{} \itlhead{q} {}, {} \itrhead{q} {}\!}) \times \finwds{\ialph}$.}
%   The machine's head faces either the left- or right-hand half of the tape, as indicated by the notation surrounding the state $q$.
% %  The words $w$ and $v$ in these configurations are finite

%   To describe the machine's moves, we define a function $\treduces$ on configurations.
%   For a left-facing head, this function is given by:
%   \begin{equation*}
%     \begin{lgathered}
%       w \wc a \itlhead{q} v \treduces
%         \begin{cases*}
%           w \itlhead{q'} b \wc v & if $\delta(q, a) = (q', b, \mathsf{L})$ \\
%           w \wc b \itrhead{q'} v & if $\delta(q, a) = (q', b, \mathsf{R})$
%         \end{cases*}
%       \\
%       \emp \itlhead{q} v \treduces \tblank \itlhead{q} v
%     \end{lgathered}
%   \end{equation*}
%   Symmetrically, for a right-facing head, this function is given by:
%   \begin{equation*}
%     \begin{lgathered}
%       w \itrhead{q} a \wc v \treduces
%         \begin{cases*}
%           w \itlhead{q'} b \wc v & if $\delta(q, a) = (q', b, \mathsf{L})$ \\
%           w \wc b \itrhead{q'} v & if $\delta(q, a) = (q', b, \mathsf{R})$
%         \end{cases*}
%       \\
%       w \itrhead{q} \emp \treduces w \itrhead{q} \tblank
%     \end{lgathered}
%   \end{equation*}
%   The direction that the head faces indicates the next symbol to be read from the tape.
%   In case a left-facing head is instructed to move right or a right-facing head is instructed to move left, the head's direction changes but its placement between the two tape halves does not.
%   If the head reaches the finite frontier, a blank is allocated.
% \end{definition}

\begin{definition}
  A \vocab{one-way infinite tape Turing machine} over a finite alphabet $\ialph$ is a tuple $\aut{M} = (Q, \delta, F)$ consisting of:
  \begin{itemize}
  \item a finite set of \emph{states}, $Q$;
  \item a \emph{transition function} $\delta\colon Q \times \ialph \to Q \times \ialph \times \Set{\mathsf{L}, \mathsf{R}}$; and
  \item a subset of \emph{final states}, $F \subseteq Q$.
  \end{itemize}
  Because the tape is only one-way infinite, a configuration of machine $\aut{M}$ is either $w \itlhead{q} v$ or $w \itrhead{q} v$ for left-infinite word $w \in \ialph^{\bar{\omega}}$, \emph{finite} word $v \in \finwds{\ialph}$, and state $q \in Q$; \ie, configurations are drawn from $\ialph^{\bar{\omega}} \times (\bigcup_{q \in Q} \Set{\!{} \itlhead{q} {}, {} \itrhead{q} {}\!}) \times \finwds{\ialph}$.
  % The machine's head faces either the left- or right-hand half of the tape, as indicated by the notation surrounding the state $q$.
%  The words $w$ and $v$ in these configurations are finite

  To describe the machine's moves, we define a function $\treduces$ on configurations.
  For a left-facing head, this function is given by:
  \begin{equation*}
      w \wc a \itlhead{q} v \treduces
        \begin{cases*}
          w \itlhead{q'} b \wc v & if $\delta(q, a) = (q', b, \mathsf{L})$ \\
          w \wc b \itrhead{q'} v & if $\delta(q, a) = (q', b, \mathsf{R})$
        \end{cases*}
  \end{equation*}
  This is exactly as it was for the two-way infinite tape Turing machines.
  Because the tape is only one-way infinite, the right-facing head's treatment differs:
  \begin{equation*}
    \!\begin{aligned}
      w \itrhead{q} a \wc v \treduces {} &
        \begin{cases*}
          w \itlhead{q'} b \wc v & if $\delta(q, a) = (q', b, \mathsf{L})$ \\
          w \wc b \itrhead{q'} v & if $\delta(q, a) = (q', b, \mathsf{R})$
        \end{cases*}
      \\
      w \itrhead{q} \emp \treduces {} &
        \begin{cases*}
          w & if $q \in F$ \\
          w \itlhead{q} \emp & if $q \notin F$
        \end{cases*}
    \end{aligned}
  \end{equation*}
  When a right-facing head reaches the finite end of the tape, its behavior depends on whether the state is final.
  If the state is final, the machine terminates; otherwise, the machine remains in the same state but effectively moves left one symbol (by turning to face left).
\end{definition}

Notice that the machine does not terminate as soon as it enters a final state -- it must also reach the finite end of the tape.
It is up to the machine's programmer, by appropriately crafting the transition function $\delta$, to ensure that final states eventually lead the machine to the tape's finite end.

% \begin{definition}
%   A \vocab{one-way infinite tape Turing machine} over a finite alphabet $\ialph$ is a tuple $\aut{M} = (Q, \tblank, \delta)$ consisting of:
%   \begin{itemize}
%   \item a finite set of states, $Q$;
%   % \item a finite input alphabet, $\ialph \subset \oalph$;
%   \item a distinguished blank symbol, $\tblank \in \ialph$; and
%   \item a \emph{transition function} $\delta\colon Q \times \ialph \to Q \times \ialph \times \Set{\mathsf{L}, \mathsf{R}}$.
%   \end{itemize}
%   A configuration of $\aut{M}$ is either $w \itlhead{q} v$ or $w \itrhead{q} v$ for words $w,v \in \finwds{\ialph}$ and states $q \in Q$.%
%   \footnote{In other words, configurations are drawn from $\finwds{\ialph} \times (\bigcup_{q \in Q} \Set{\itlhead{q}, \itrhead{q}}) \times \finwds{\ialph}$.}
%   The middle portion of a configuration, either $\itlhead{q}$ or $\itrhead{q}$, represents $\aut{M}$'s \emph{finite control}, which consists of the machine's current state together with the direction of the next symbol to be read.

%   We define a function $\treduces$ on these configurations.
%   If the machine's finite control is facing leftward and reads an $a$, it writes a symbol in place of the $a$, optionally changes the finite control's state, and then either advances or turns to face rightward, according to the transition function $\delta$.
%   \begin{equation*}
%     % \begin{lgathered}
%       w \wc a \itlhead{q} v \treduces
%         \begin{cases*}
%           w \itlhead{q'} b \wc v & if $\delta(q, a) = (q', b, \mathsf{L})$ \\
%           w \wc b \itrhead{q'} v & if $\delta(q, a) = (q', b, \mathsf{R})$
%         \end{cases*}
%     %   \\
%     %   \emp \itlhead{q} v \treduces \tblank \itlhead{q} v
%     % \end{lgathered}
%   \end{equation*}
%   Otherwise, if the configuration is $\emp \itlhead{q} v$, then the finite control has reached the left-hand frontier of the tape; because the tape is unbounded to the left, a new, blank cell is allocated in this case.
%   \begin{equation*}
%     \emp \itlhead{q} v \treduces \tblank \itlhead{q} v
%   \end{equation*}

%   A right-facing finite control is treated symmetrically, except that the right-hand end of the tape is fixed.
%   To ensure that the finite control does not run off the end of the tape, its direction
%   \begin{equation*}
%     \begin{lgathered}
%       w \itrhead{q} a \wc v \treduces
%         \begin{cases*}
%           w \itlhead{q'} b \wc v & if $\delta(q, a) = (q', b, \mathsf{L})$ \\
%           w \wc b \itrhead{q'} v & if $\delta(q, a) = (q', b, \mathsf{R})$
%         \end{cases*}
%       \\
%       w \itrhead{q} \emp \treduces w \itlhead{q} \emp
%     \end{lgathered}
%   \end{equation*}

% \end{definition}

% \begin{definition}
%   A \vocab{two-way infinite tape Turing machine} over a finite alphabet $\ialph$ is a tuple $(Q, \tblank, \delta)$ consisting of:
%   \begin{itemize}
%   \item a finite set of states, $Q$;
%   \item a distinguished blank symbol, $\tblank \in \ialph$; and
%   \item a transition function, $\delta\colon Q \times \ialph \to Q \times \ialph \times \Set{\mathsf{L}, \mathsf{R}}$.
%   \end{itemize}
%   A configuration of $\aut{M}$ is a triple $(w, q, v) \in \finwds{\ialph} \times \bigl((\Set{\mathord{\itlhead}} \times Q) \union (Q \times \Set{\mathord{\itrhead}})\bigr) \times \finwds{\ialph}$.
%   Depending on the direction that the machine's head is facing and whether it is at the frontier, there are several possible transitions:
%   \begin{equation*}
%     \begin{tabular}{@{}r@{\enspace}l@{\enspace}l@{}}
%       $w \wc a \itlhead q \oc v \treduces w \itlhead q'_a \oc b \wc v$ \\
%       and $w \oc q \itrhead a \wc v \treduces w \itlhead q'_a \oc b \wc v$
%         & if & $\delta(q, a) = (q'_a, b, \mathsf{L})$
%       \\[1ex]
%       $w \wc a \itlhead q \oc v \treduces w \wc b \oc q'_a \itrhead v$ \\
%       and $w \oc q \itrhead a \wc v \treduces w \wc b \oc q'_a \itrhead v$
%         & if & $\delta(q, a) = (q'_a, b, \mathsf{R})$
%       \\[1ex]
%       $\emp \itlhead q \oc v \treduces \tblank \itlhead q \oc v$ \\
%       and $w \oc q \itrhead \emp \treduces w \oc q \itrhead \tblank$
%     \end{tabular}
%   \end{equation*}
% \end{definition}


% \section{Chains of communicating automata}
% \fixnote{Remove these?}

% \begin{definition}
%   A \emph{chain of communicating automata} over a finite alphabet $\ialph$ is a tuple $\aut{C} = (Q, (\ialph_q)_{q \in Q}, (\oalph_q)_{q \in Q})$ consisting of:
%   \begin{itemize}
%   \item a finite set of \vocab{states}, $Q$, that is partitioned into: left- and right-reading states, $\rL{Q}$ and $\rR{Q}$; left- and right-writing states, $\wL{Q}$ and $\wR{Q}$; forking states, $\fS{Q}$; and halting states, $\hS{Q}$;
%   \item a state-indexed set of finite \vocab{left-hand alphabets}, $(\ialph_q)_{q \in Q}$, and a state-indexed set of finite \vocab{right-hand alphabets}, $(\oalph_q)_{q \in Q}$;
%   \item $\rL{\delta} \colon \Pi q{:}\rL{Q}. \ialph_q \to Q$, with the condition that $\ialph_q = \ialph_{q'}$ and $\oalph_q = \oalph_{q'}$ for all $q' \in \cod{\rL{\delta}_q}$, and $\rR{\delta} \colon (\exists q{:}\rR{Q}. \oalph_q) \to Q$;
%   \item $\wL{\delta} \colon \wL{Q} \to \exists q'{:}Q. \ialph_{q'}$ and $\wR{\delta} \colon \wR{Q} \to \exists q'{:}Q. \oalph_{q'}$; and
%   \item $\fS{\delta} \colon \fS{Q} \to Q \times Q$.
%   \end{itemize}

%   \vocab{Chain configurations}, $c$ and $d$, consist of a finite sequence of states $q_1, q_2, \dotsc, q_n \in Q$ with, for all $1 \leq i < n$, a finite word drawn from $\finwds{\ialph_{q_{i+1}}}$ between neighboring states $q_i$ and $q_{i+1}$.
%   In addition, a finite word drawn from $\finwds{\ialph_{q_1}}$ and a finite word drawn from $\finwds{\oalph_{q_n}}$ bracket the configuration.
%   Formally, then, a chain configuration is a string drawn from the set
%   \begin{equation*}
%     \finwds{\ialph_{q_1}} q_1 \finwds{\ialph_{q_2}} q_2 \dotsm \finwds{\ialph_{q_n}} q_n \finwds{\oalph_{q_n}}
%     \,,
%   \end{equation*}
%   for some finite sequence of states $q_1, q_2, \dotsc, q_n \in Q$.
%   The chain configuration is \vocab{well-formed} if $\oalph_{q_i} = \ialph_{q_{i+1}}$ for all $1 \leq i < n$.

%   \begin{itemize}
%   \item $c \wc a \,q\, d \reduces c \,q'\, d$ if $\rL{\delta}_q(a) = q'$, and $c \,q\, a \wc d \reduces c \,q'\, d$ if $\rR{\delta}_q(a) = q'$;
%   \item $c \,q\, d \reduces c \wc a \,q'\, d$ if $\wL{\delta}(q) = (q', a)$, and $c \,q\, d \reduces c \,q'\, a \wc d$ if $\wR{\delta}(q) = (q', a)$;

%   \item $c \,q\, d \reduces c \, q' \, q'' \, d$ if $\fS{\delta}(q) = (q', q'')$;
%   \item $c \,q\, d \reduces c \, d$ if $q \in \hS{Q}$ and $\ialph_q = \oalph_q$.
%   \end{itemize}
% \end{definition}


%%% Local Variables:
%%% mode: latex
%%% TeX-master: "thesis"
%%% End:
