\chapter{Preliminaries: Ordered logic}\label{ch:ordered-logic}

% In \citeyear{Lambek:??}, \citeauthor{Lambek:??} published a seminal paper developing a formal system for describing sentence structure.
% The Lambek calculus, when viewed from a logical perspective, forms the core of \emph{intuitionistic ordered logic}%
% \footnote{Also known as intuitionistic noncommutative linear logic.}%
% .


This \lcnamecref{ch:ordered-logic} serves to review a sequent calculus presentation of [the non-modal fragment of]
intuitionistic%
\footnote{Contrast with noncommutative logic}
ordered logic, also known as the (full) Lambek calculus\autocites{Lambek:AMM58}{Lambek:SLIM61}.
As a substructural logic, ordered logic eschews the usual structural properties of weakening, contraction, and exchange.
As in \citeauthor{Girard:TCS??}'s linear logic\autocite{Girard:TCS??}, the lack of weakening and contraction properties means that each antecedent must be used exactly once within a proof.
In ordered logic, the additional lack of an exchange property means that antecedents must also remain in order within a proof.

\Citeauthor{Lambek:AMM58} leveraged the noncommutativity of antecedents to give a formal description of sentence structure.
In this work, however, our interest is not mathematical linguistics but the logical foundations of concurrent computation, so the presentation of ordered logic has a more proof-theoretic slant and is derived from a presentation by ...


[Outline paragraph]


\section{A sequent calculus presentation of ordered logic}

Ordered logic eschews the usual structural properties of weakening, contraction, and exchange.
Like in linear logic, the absence of weakening and contraction means that antecedents may neither be discarded nor duplicated within a proof.
But ordered logic's rejection of the exchange property takes things one step further: antecedents may not even be permuted within a proof.

\subsection{Sequents and contexts}

In ordered logic's sequent calculus presentation, the basic judgment is a sequent
\begin{equation*}
  \oseq{A_1 \oc A_2 \dotsb A_n |- A} \,,
\end{equation*}
where the propositions $A_1, A_2, \dotsc, A_n$ are assumptions, or \emph{antecedents}, arranged into an ordered list%
\footnote{Antecedents may be freely reassociated, and so form a list, not a tree.}%
, not a multiset as in linear logic, nor a set as in persistent logic; the proposition $A$ is termed the \emph{consequent}.

To keep [the syntax of] sequents concise, the list of antecedents is usually packaged into an ordered context $\octx = A_1 \oc A_2 \dotsb A_n$, and the sequent written $\oseq{\octx |- A}$.
Algebraically, ordered contexts $\octx$ form a free (noncommutative) monoid:
\begin{equation*}
  \octx \Coloneqq \octx_1 \oc \octx_2 \mid \octxe \mid A \,,
\end{equation*}
where the monoid operation is concatenation, denoted by juxtaposition, and the unit element is the empty context, denoted by $(\octxe)$.
As a monoid, ordered contexts are equivalent up to associativity and unit laws (see adjacent \lcnamecref{fig:ordered-logic:monoid-laws}).%
\begin{marginfigure}
  \begin{gather*}
    (\octx_1 \oc \octx_2) \oc \octx_3 = \octx_1 \oc (\octx_2 \oc \octx_3) \\
    (\octxe) \oc \octx = \octx = \octx \oc (\octxe)
  \end{gather*}
  \caption{The monoid laws for ordered contexts}\label{fig:ordered-logic:monoid-laws}
\end{marginfigure}
However, 

\newthought{Like linear logic}\autocite{Wadler:??}\fixnote{Is this the right reference?}, ordered logic can be given a resource interpretation -- with a small twist.
% An ordered sequent $\oseq{A_1 \oc A_2 \dotsb A_n |- A}$ can be read as stating that resource $A$ can be produced from the resources $A_1, A_2, \dotsc, A_n$, and a proof of that sequent is a recipe for
A proof of an ordered sequent $\oseq{\octx |- A}$ can be interpreted as a recipe for producing resource $A$ from the resources $\octx$.
The twist is that these resources are inherently ordered and may not be permuted, exactly because ordered logic rejects the exchange property [that linear logic admits].


% Following the example of linear logic, ordered logic can be given a resource interpretation -- with a small twist.
% The absence of an exhange property means that resouces are now inherently ordered.
% The sequent $\oseq{\octx |- A}$ means that, by consuming the resources $\octx$, the resource $A$ can be produced.
% For ordered logic, however, 

\subsection{Judgmental principles}

Even without considering the specific structure of propositions, two judgmental principles must hold if sequents are to accurately describe the production of resources.

First, given a resource $A$, producing the same resource $A$ should be effortless -- it already exists!
This amounts to an identity principle for sequents:
  \begin{description}[labelindent=\parindent]
  \item[Identity principle] $\oseq{A |- A}$ for all propositions $A$.
  \end{description}
  This principle is adopted by the ordered sequent calculus as a primitive rule of inference:
  \begin{equation*}
    \infer[\jrule{ID}\smash{^A}]{\oseq{A |- A}}{}
    \,.
  \end{equation*}
Both the identity principle and its corresponding $\jrule{ID}$ rule capture the idea that resource production is a reflexive process.

Second, and dually, resource production should be transitive process.
If a resource $B$ can be produced from resource $A$ (\ie, $\oseq{A |- B}$), and if a resource $C$ can be produced from resource $B$ (\ie, $\oseq{B |- C}$), then, by chaining the productions, $C$ should be able to be produced from $A$ (\ie, $\oseq{A |- C}$).
For sequents, this amounts to a cut principle that is most useful in a generalized form:
\begin{description}[resume*]
\item[Cut principle]
  If $\oseq{\octx |- B}$ and $\oseq{\octx'_L \oc B \oc \octx'_R |- C}$, then $\oseq{\octx'_L \oc \octx \oc \octx'_R |- C}$.
\end{description}
As with the identity principle, this cut principle is adopted by the ordered sequent calculus as a primitive rule of inference:
\begin{equation*}
  \infer[\jrule{CUT}\smash{^B}]{\oseq{\octx'_L \oc \octx \oc \octx'_R |- C}}{
    \oseq{\octx |- B} & \oseq{\octx'_L \oc B \oc \octx'_R |- C}}
  \,.
\end{equation*}

The importance of these two judgmental principles goes beyond that of mere rules of inference.
As we will see in \cref{??}, the admissibility of these principles serves an important role in defining the meaning of the logical connectives.

% Even without considering the structure of propositions, two judgmental principles of the ordered sequent calculus are already apparent.
% \begin{itemize}
% \item Resource production is transitive: if resource $A$ can be produced from resources $\octx$, and if the resource $C$ can be produced from resources $\octx'_L \oc A \oc \octx'_R$, then $C$ can be produced from $\octx'_L \oc \octx \oc \octx'_R$ by first producing $A$ from the inner resources $\octx$ and then producing $C$ from the resulting resources, $\octx'_L \oc A \oc \octx'_R$.
% \end{itemize}

% \begin{inferences}
%   \infer[\jrule{CUT}\smash{^A}]{\oseq{\octx'_L \oc \octx \oc \octx'_R |- C}}{
%     \oseq{\octx |- A} & \oseq{\octx'_L \oc A \oc \octx'_R |- C}}
%   \and
%   \infer[\jrule{ID}\smash{^A}]{\oseq{A |- A}}{}
% \end{inferences}
% The $\jrule{CUT}$ rule shows that resource production is transitive: if resource $A$ can be produced from resources $\octx$, and if the resource $C$ can be produced from resources $\octx'_L \oc A \oc \octx'_R$, then $C$ can be produced from $\octx'_L \oc \octx \oc \octx'_R$ by first producing $A$ from the inner resources $\octx$ and then producing $C$ from the resulting resources, $\octx'_L \oc A \oc \octx'_R$.
% Dually, the $\jrule{ID}$ rule shows that resource production is also reflexive: given a resource $A$, the same resource $A$ can be produced effortlessly.

\subsection{The logical connectives}

The propositions of ordered logic are given by the following grammar.
\begin{syntax*}
  Propositions & A, B, C &
    \alpha
    \begin{array}[t]{@{{} \mid {}}l@{}}
      A \fuse B \mid \one \mid A \plus B \mid \zero \\
      A \with B \mid \top \mid A \limp B \mid B \pmir A
    \end{array}
\end{syntax*}
Among these are propositional atoms, $\alpha$, which stand in for arbitrary propositions.
The other propositions are built up from these atoms by using the logical connectives.

Under the resource interpretation of ordered logic, these logical connectives may be viewed as resource constructors.
A connective's right rule defines how to produce that kind of resource, while the corresponding left rules define how that kind of resource may be used.
% As a first example, consider ordered conjunction.

\paragraph*{Ordered conjunction and its unit}
Ordered conjunction\footnote{Also known as multiplicative conjunction.} is the proposition $A \fuse B$, read \enquote{$A$ fuse $B$}.
Under the resource interpretation, $A \fuse B$ is the side-by-side pair of resources $A$ and $B$, packaged as a single ordered resource.
Its sequent calculus inference rules are:
\begin{inferences}
  \infer[\rrule{\fuse}]{\oseq{\octx_1 \oc \octx_2 |- A \fuse B}}{
    \oseq{\octx_1 |- A} & \oseq{\octx_2 |- B}}
  \and
  \infer[\lrule{\fuse}]{\oseq{\octx'_L \oc (A \fuse B) \oc \octx'_R |- C}}{
    \oseq{\octx'_L \oc A \oc B \oc \octx'_R |- C}}
\end{inferences}
The right rule, $\rrule{\fuse}$, says that $A \fuse B$ may be produced by partitioning the available resources into $\octx_1 \oc \octx_2$ and then separately using the resources $\octx_1$ and $\octx_2$ to produce $A$ and $B$, respectively.
The left rule, $\lrule{\fuse}$, shows how to use resource $A \fuse B$: simply unwrap the package to leave the separate contents, resources $A$ and $B$, side by side.

Just as truth is the nullary analogue of conjunction in intuitionistic logic\fixnote{Is this the right name?}, multiplicative truth, $\one$, is the nullary analogue of ordered conjunction.
Under the resource interpretation, $\one$ is therefore an empty resource package that contains no resources.
\begin{inferences}
  \infer[\rrule{\one}]{\oseq{\octxe |- \one}}{}
  \and
  \infer[\lrule{\one}]{\oseq{\octx'_L \oc \one \oc \octx'_R |- C}}{
    \oseq{\octx'_L \oc \octx'_R |- C}}
\end{inferences}
The sequents $\oseq{\one \fuse A \dashv|- \oseq{A \dashv|- A \fuse \one}}$ are all derivable, so $\one$ is indeed the unit of $\fuse$.%
% \begin{marginfigure}
%   \centering
%   $\infer[\lrule{\fuse}]{\oseq{\one \fuse A |- A}}{
%      \infer[\lrule{\one}]{\oseq{\one \oc A |- A}}{
%        \infer[\jrule{ID}\smash{^A}]{\oseq{A |- A}}{}}}$%
%   \qquad
%   $\infer[\rrule{\fuse}]{\oseq{A |- \one \fuse A}}{
%      \infer[\rrule{\one}]{\oseq{\octxe |- \one}}{} &
%      \infer[\jrule{ID}\smash{^A}]{\oseq{A |- A}}{}}$%
%   \caption{Proofs of two of the unit laws for ordered conjunction}
% \end{marginfigure}

\paragraph{Disjunction and its unit}
Disjunction is the proposition $A \plus B$, read \enquote{$A$ plus $B$}.%
\footnote{This connective is also known as additive disjunction, in contrast with the multiplicative disjunction of classical linear logic; being intuitionistic, ordered logic does not have a purely multiplicative disjunction.
  See \textcite{Chang+:CMU03}.}
Under the resource interpretation, $A \plus B$ is a package that contains one of the resources $A$ or $B$
% either resource $A$ or resource $B$
(but not both).
\begin{inferences}
  \infer[\rrule{\plus}_1]{\oseq{\octx |- A \plus B}}{
    \oseq{\octx |- A}}
  \and
  \infer[\rrule{\plus}_2]{\oseq{\octx |- A \plus B}}{
    \oseq{\octx |- B}}
  \and
  \infer[\lrule{\plus}]{\oseq{\octx'_L \oc (A \plus B) \oc \octx'_R |- C}}{
    \oseq{\octx'_L \oc A \oc \octx'_R |- C} &
    \oseq{\octx'_L \oc B \oc \octx'_R |- C}}
\end{inferences}
The right rules, $\rrule{\plus}_1$ and $\rrule{\plus}_2$, say that a resource $A \plus B$ may be produced from the resources $\octx$ by producing either $A$ or $B$ and then wrapping that resource up as an $A \plus B$ package.
The left rule, $\lrule{\plus}$, shows how to use a resource $A \plus B$: unwrap the package and use whatever it contains -- whether an $A$ or a $B$.

Falsehood, $\zero$, can be viewed as the nullary analogue of disjunction:
\begin{inferences}
  \text{(no $\rrule{\zero}$ rule)}
  \and
  \infer[\lrule{\zero}]{\oseq{\octx'_L \oc \zero \oc \octx'_R |- C}}{}
\end{inferences}
The sequents $\oseq{\zero \plus A \dashv|- \oseq{A \dashv|- A \plus \zero}}$ are all derivable, so falsehood is indeed the unit of disjunction.

\paragraph{Alternative conjunction and its unit}
Alternative conjunction\footnote{Also known as additive conjunction.} is the proposition $A \with B$, read \enquote{$A$ with $B$};
it is dual to disjunction.
Under the resource interpretation, $A \with B$ is the resource that can be transformed -- irreversibly -- into either a resource $A$ or a resource $B$, whichever the user chooses.
\begin{inferences}
  \infer[\rrule{\with}]{\oseq{\octx |- A \with B}}{
    \oseq{\octx |- A} & \oseq{\octx |- B}}
  \and
  \infer[\lrule{\with}_1]{\oseq{\octx'_L \oc (A \with B) \oc \octx'_R |- C}}{
    \oseq{\octx'_L \oc A \oc \octx'_R |- C}}
  \and
  \infer[\lrule{\with}_2]{\oseq{\octx'_L \oc (A \with B) \oc \octx'_R |- C}}{
    \oseq{\octx'_L \oc B \oc \octx'_R |- C}}
\end{inferences}
The left rules, $\lrule{\with}_1$ and $\lrule{\with}_2$, show how to use a resource $A \with B$: transform it into either an $A$ or a $B$ and then use that resource.
The right rule, $\rrule{\with}$, says that to produce a resource $A \with B$ the producer must be prepared to produce either $A$ or $B$ -- whichever the user eventually chooses.
% it must be possible to produce $A$ from $\octx$ \emph{and} to produce $B$ from $\octx$ -- to be ready for either eventuality.

Additive truth, $\top$, can be viewed as the nullary analogue of alternative conjunction:
\begin{inferences}
  \infer[\rrule{\top}]{\oseq{\octx |- \top}}{}
  \and
  \text{(no $\lrule{\top}$ rule)}
\end{inferences}
Once again, the sequents $\oseq{\top \with A \dashv|- \oseq{A \dashv|- A \with \top}}$ are all derivable, so additive truth is indeed the unit of alternative conjunction.


\paragraph*{Left- and right-handed implications}
Left-handed implication is the proposition $A \limp B$, read \enquote{$A$ under $B$} or \enquote{$A$ left-implies $B$}.
% Under the resource interpretation, $A \limp B$ is the resource that, when placed with the resource $A$ to its immediate left, consumes the $A$ to produce resource $B$.
When interpreted as a resource, $A \limp B$ is the resource that can transform a left-adjacent resource~$A$ into the resource $B$.
% consume a resource $A$ from its immediate left and thereby produce the resource $B$.
\begin{inferences}
  \infer[\rrule{\limp}]{\oseq{\octx |- A \limp B}}{
    \oseq{A \oc \octx |- B}}
  \and
  \infer[\lrule{\limp}]{\oseq{\octx'_L \oc \octx \oc (A \limp B) \oc \octx'_R |- C}}{
    \oseq{\octx |- A} & \oseq{\octx'_L \oc B \oc \octx'_R |- C}}
\end{inferences}
The left rule, $\lrule{\limp}$, shows how to use a resource $A \limp B$: first produce $A$ from the left-adjacent resources $\octx$, then transform the left-adjacent $A$ into the resource $B$, and finally use that $B$.
The right rule, $\rrule{\limp}$, says that resources $\octx$ can produce $A \limp B$ if the same resources prefixed with $A$ -- that is, $A \oc \octx$ -- can produce $B$.

Right-handed implication, $B \pmir A$ (read \enquote{$B$ over $A$} or \enquote{$A$ right-implies $B$}), is symmetric to left-handed implication: $B \pmir A$ is the resource that can transform a \emph{right}-adjacent resource $A$ into the resource $B$.
\begin{inferences}
  \infer[\rrule{\pmir}]{\oseq{\octx |- B \pmir A}}{
    \oseq{\octx \oc A |- B}}
  \and
  \infer[\lrule{\pmir}]{\oseq{\octx'_L \oc (B \pmir A) \oc \octx \oc \octx'_R |- C}}{
    \oseq{\octx |- A} & \oseq{\octx'_L \oc B \oc \octx'_R |- C}}
\end{inferences}

The two forms of implication each enjoy their own currying laws: the sequents $\oseq{A \limp (B \limp C) \dashv|- (B \fuse A) \limp C}$ and $\oseq{(C \pmir B) \pmir A \dashv|- C \pmir (A \fuse B)}$ are derivable.


\paragraph*{Summary}
The sequent calculus presented above is summarized in \cref{fig:ordered-logic:sequent-calculus}.

\begin{figure}[tbp]
  \vspace*{\dimexpr-\abovedisplayskip-\abovecaptionskip\relax}
  \begin{syntax*}
    Propositions & A,B,C &
      \alpha \begin{array}[t]{@{{} \mid {}}l@{}}
               A \fuse B \mid \one \mid A \plus B \mid \zero \\
               A \with B \mid \top \mid A \limp B \mid B \pmir A
             \end{array}
             \\
    Contexts & \octx &
      \octx_1 \oc \octx_2 \mid \octxe \mid A
  \end{syntax*}
  \begin{inferences}
    \infer[\jrule{CUT}\smash{^A}]{\oseq{\octx'_L \oc \octx \oc \octx'_R |- C}}{
      \oseq{\octx |- A} & \oseq{\octx'_L \oc A \oc \octx'_R |- C}}
    \and
    \infer[\jrule{ID}\smash{^A}]{\oseq{A |- A}}{}
  \end{inferences}
  \begin{inferences}
    \infer[\rrule{\fuse}]{\oseq{\octx_1 \oc \octx_2 |- A \fuse B}}{
      \oseq{\octx_1 |- A} & \oseq{\octx_2 |- B}}
    \and
    \infer[\lrule{\fuse}]{\oseq{\octx'_L \oc (A \fuse B) \oc \octx'_R |- C}}{
      \oseq{\octx'_L \oc A \oc B \oc \octx'_R |- C}}
    % \\
    % \infer[\rrule{\esuf}]{\oseq{\octx_1 \oc \octx_2 |- B \esuf A}}{
    %   \oseq{\octx_1 |- A} & \oseq{\octx_2 |- B}}
    % \and
    % \infer[\lrule{\esuf}]{\oseq{\octx'_L \oc (B \esuf A) \oc \octx'_R |- C}}{
    %   \oseq{\octx'_L \oc A \oc B \oc \octx'_R |- C}}
    \\
    \infer[\rrule{\one}]{\oseq{\octxe |- \one}}{}
    \and
    \infer[\lrule{\one}]{\oseq{\octx'_L \oc \one \oc \octx'_R |- C}}{
      \oseq{\octx'_L \oc \octx'_R |- C}}
    \\
    \infer[\rrule{\plus}_1]{\oseq{\octx |- A \plus B}}{
      \oseq{\octx |- A}}
    \and
    \infer[\rrule{\plus}_2]{\oseq{\octx |- A \plus B}}{
      \oseq{\octx |- B}}
    \and
    \infer[\lrule{\plus}]{\oseq{\octx'_L \oc (A \plus B) \oc \octx'_R |- C}}{
      \oseq{\octx'_L \oc A \oc \octx'_R |- C} &
      \oseq{\octx'_L \oc B \oc \octx'_R |- C}}
    \\
    \text{(no $\rrule{\zero}$ rule)}
    \and
    \infer[\lrule{\zero}]{\oseq{\octx'_L \oc \zero \oc \octx'_R |- C}}{}
  \end{inferences}
  \begin{inferences}
    \infer[\rrule{\with}]{\oseq{\octx |- A \with B}}{
      \oseq{\octx |- A} & \oseq{\octx |- B}}
    \and
    \infer[\lrule{\with}_1]{\oseq{\octx'_L \oc (A \with B) \oc \octx'_R |- C}}{
      \oseq{\octx'_L \oc A \oc \octx'_R |- C}}
    \and
    \infer[\lrule{\with}_2]{\oseq{\octx'_L \oc (A \with B) \oc \octx'_R |- C}}{
      \oseq{\octx'_L \oc B \oc \octx'_R |- C}}
    \\
    \infer[\rrule{\top}]{\oseq{\octx |- \top}}{}
    \and
    \text{(no $\lrule{\top}$ rule)}
    \\
    \infer[\rrule{\limp}]{\oseq{\octx |- A \limp B}}{
      \oseq{A \oc \octx |- B}}
    \and
     \infer[\lrule{\limp}]{\oseq{\octx'_L \oc \octx \oc (A \limp B) \oc \octx'_R |- C}}{
      \oseq{\octx |- A} & \oseq{\octx'_L \oc B \oc \octx'_R |- C}}
    \\
    \infer[\rrule{\pmir}]{\oseq{\octx |- B \pmir A}}{
      \oseq{\octx \oc A |- B}}
    \and
    \infer[\lrule{\pmir}]{\oseq{\octx'_L \oc (B \pmir A) \oc \octx \oc \octx'_R |- C}}{
      \oseq{\octx |- A} & \oseq{\octx'_L \oc B \oc \octx'_R |- C}}
  \end{inferences}
  \caption{A sequent calculus presentation of ordered logic}\label{fig:ordered-logic:sequent-calculus}
\end{figure}

\section{A verificationist meaning-explanation of the sequent calculus}

The previous \lcnamecref{??} described a resource interpretation of a collection of inference rules, that we purported to be a sequent calculus presentation of ordered logic.
Although the rules have a seemingly sensible resource interpretation, do they constitute a well-defined \emph{logic}?

In the tradition of \citeauthor{Gentzen:MZ35} [, \citeauthor{Dummett:HUP91},] and \citeauthor{Martin-Lof:NJPL96}, a logic is well-defined if it rests on the solid foundation of a verificationist mean\-ing-\-ex\-pla\-na\-tion.
That is, the meaning of each logical connective [proposition?] must be given entirely by what counts as a verification of it.
And a verification is a proof that only decomposes that proposition into its subformulas, without dragging in other, unrelated propositions.
[In this way, the meaning of each logical connective is defined compositionally.]

For a sequent calculus, a verification is thus a proof that relies only on the right and left inference rules -- that is, a verification is a proof in which all occurrences of the cut and identity rules have been eliminated.
(Of course, propositional atoms do not have right and left rules, and so the identity rule is permitted for propositional atoms -- and propositional atoms alone.)

For this verificationist program to succeed, we need to be sure that every proof has a corresponding verification -- we need a proof normalization procedure.

We will say that a proof is \vocab{cut-free} if it contains no instances of the $\jrule{CUT}$ rule, and \vocab{long} if all instances of the $\jrule{ID}$ rule occur at propositional atoms, not compound propositions.
Verifications are thus exactly those proofs that are both cut-free and long.

This suggests a strategy for normalizing proofs to verifications: first eliminate all instances of $\jrule{CUT}$, and then eliminate any remaining non-atomic instances of $\jrule{ID}$.
Provided that the identity elimination step does not introduce any instances of $\jrule{CUT}$, the end result will be a cut-free, long proof -- that is, a verification.

\begin{theorem}[name=Cut elimination, restate=orderedcutelimination]\label{thm*:ordered-logic:cut-elimination}
  If a proof of\/ $\oseq{\octx |- A}$ exists, then a \emph{cut-free} proof of $\oseq{\octx |- A}$ exists.
\end{theorem}

\begin{restatable}[Identity elimination]
  If a proof of\/ $\oseq{\octx |- A}$, then a \emph{long} proof of\/ $\oseq{\octx |- A}$ exists.
  Moreover, cut-freeness is preserved.
\end{restatable}

\begin{corollary*}[Proof normalization]
  If\/ $\oseq{\octx |- A}$, then a verification of\/ $\oseq{\octx |- A}$ exists.
\end{corollary*}

As presented in the following \lcnamecrefs{sec:ordered-logic:cut-elimination}, the proofs of these \lcnamecrefs{thm*:ordered-logic:cut-elimination} are all constructive and amount to defining functions on proofs.

\subsection{Cut elimination}\label{sec:ordered-logic:cut-elimination}

To prove the cut elimination \lcnamecref{thm*:ordered-logic:cut-elimination} stated above, we will eventually use a straightforward induction on the structure of the given proof.
But first, we need to establish a cut principle for cut-free proofs:
%
\begin{lemma*}[Admissibility of cut]\label{lem*:ordered-logic:cut-admissibility}
  If cut-free proofs of\/ $\oseq{\octx |- A}$ and $\oseq{\octx'_L \oc A \oc \octx'_R |- C}$ exist, then a \emph{cut-free} proof of\/ $\oseq{\octx'_L \oc \octx \oc \octx'_R |- C}$.
\end{lemma*}

Before proceeding to this \lcnamecref{lem:ordered-logic:cut-admissibility}'s proof, it is worth emphasizing a subtle distinction between the sequent calculus's primitive $\jrule{CUT}$ rule and the admissible cut principle that this \lcnamecref{lem:ordered-logic:cut-admissibility} establishes.

To be completely formal, we ought to treat cut-freeness as an extrinsic, Curry-style property of proofs and indicate that property by decorating the turnstile: a proof of $\oseqcf{\octx |- A}$ is a cut-free proof of $\oseq{\octx |- A}$.
The admissible cut principle stated in \cref{lem:ordered-logic:cut-admissibility} could then be expressed as the rule
\begin{equation*}
  \infer-[\jrule{A-CUT}\smash{^A}]{\oseqcf{\octx'_L \oc \octx \oc \octx'_R |- C}}{
    \oseqcf{\octx |- A} & \oseqcf{\octx'_L \oc A \oc \octx'_R |- C}}
  \,
\end{equation*}
with the dotted line indicating that this is an admissible, not primitive, rule.
Writing it in this way emphasizes that the proof of \cref{lem:ordered-logic:cut-admissibility} will amount to defining a meta-level function that takes cut-free proofs of $\oseq{\octx |- A}$ and $\oseq{\octx'_L \oc A \oc \octx'_R |- C}$ and produces a \emph{cut-free} proof of $\oseq{\octx'_L \oc \octx \oc \octx'_R |- C}$.
Contrast this with the primitive $\jrule{CUT}$ rule of the ordered sequent calculus, which forms a (cut-full) proof of $\oseq{\octx'_L \oc \octx \oc \octx'_R |- C}$ from (potentially cut-full) proofs of $\oseq{\octx |- A}$ and $\oseq{\octx'_L \oc A \oc \octx'_R |- C}$.

From here on, we won't bother to be quite so pedantic, instead often omitting the turnstile decoration on cut-free proofs, with the understanding that any proofs to which the admissible $\jrule{A-CUT}$ rule is applied are necessarily cut-free.

\newthought{With that} clarification out of the way, we may proceed to proving the previously stated \lcnamecrefs{lem:ordered-logic:cut-admissibility} and \lcnamecref{thm:ordered-logic:normalization}.
%
\begin{lemma}[Admissibility of cut]\label{lem:ordered-logic:cut-admissibility}
  If cut-free proofs of $\oseq{\octx |- A}$ and $\oseq{\octx'_L \oc A \oc \octx'_R |- C}$ exist, then a \emph{cut-free} proof of $\oseq{\octx'_L \oc \octx \oc \octx'_R |- C}$ exists.
\end{lemma}
%
\begin{proof}
  We follow a standard technique for proving the admissibility of a cut principle\autocite{Pfenning:LICS95} -- a lexicographic structural induction, first on the structure of the cut formula, $A$, and then on the structures of the given proofs.

  As usual, the various cases can be sorted into three classes: identity cut reductions, principal cut reductions, and commutative cut reductions.
  %
  \begin{description}[parsep=0pt, listparindent=\parindent]
  \item[Identity cut reductions]
    In the cases where one of the two proofs is an instance of the $\jrule{ID}$ rule, the admissible cut can be reduced to the other proof alone.
    For example:
    \begin{equation*}
      \infer-[\jrule{A-CUT}\smash{^A}]{\oseq{\octx'_L \oc A \oc \octx'_R |- C}}{
        \infer[\jrule{ID}\smash{^A}]{\oseq{A |- A}}{} &
        \deduce{\oseq{\octx'_L \oc A \oc \octx'_R |- C}}{\EE}}
      =
      \deduce{\oseq{\octx'_L \oc A \oc \octx'_R |- C}}{\EE}
    \end{equation*}
    That the cut and identity principles are inverses here

  \item[Principal cut reductions]
    Another class of cases pair proofs that both end by introducing the cut formula -- on the right in the left-hand proof with a right rule, and on the left in the right-hand proof with a left rule.
    These cases are resolved by reducing the admissible cut to several instances of the admissible cut principle at proper subformulas of the cut formula.

    For example, the principal cut reduction for $A_1 \limp A_2$ yields cuts at the proper subformulas $A_1$ and $A_2$:
    \begin{gather*}
      \infer-[\jrule{A-CUT}\smash{^{A_1 \limp A_2}}]{\oseq{\octx'_L \oc \octx'_1 \oc \octx \oc \octx'_R |- C}}{
        \infer[\rrule{\limp}]{\oseq{\octx |- A_1 \limp A_2}}{
          \deduce{\oseq{A_1 \oc \octx |- A_2}}{\DD_1}} &
        \infer[\lrule{\limp}]{\oseq{\octx'_L \oc \octx'_1 \oc (A_1 \limp A_2) \oc \octx'_R |- C}}{
          \deduce{\oseq{\octx'_1 |- A_1}}{\EE_1} &
          \deduce{\oseq{\octx'_L \oc A_2 \oc \octx'_R |- C}}{\EE_2}}}
      \\=\\
      \infer-[\jrule{A-CUT}\smash{^{A_2}}]{\oseq{\octx'_L \oc \octx'_1 \oc \octx \oc \octx'_R |- C}}{
        \infer-[\jrule{A-CUT}\smash{^{A_1}}]{\oseq{\octx'_1 \oc \octx |- A_2}}{
          \deduce{\oseq{\octx'_1 |- A_1}}{\DD_1} &
          \deduce{\oseq{A_1 \oc \octx |- A_2}}{\EE_1}} &
        \deduce{\oseq{\octx'_L \oc A_2 \oc \octx'_R |- C}}{\EE_2}}
    \end{gather*}

  \item[Commutative cut reductions]
    In the remaining cases, one of the proofs ends with an inference on a side formula\footnote{A formula other than the cut formula.}.
    To reduce the admissible cut, it is permuted with the final inference in that proof;
    the reduced instance of the admissible cut is smaller because it occurs at the same cut formula but with smaller proofs.

    Commutative cut reductions are subcategorized as left- or right-\-com\-mu\-ta\-tive cut reductions according to the proof into which the admissible cut is permuted.
    For example, one left-commutative case involves a left branch that ends with the $\lrule{\limp}$ rule:
    \begin{gather*}
      \infer-[\jrule{A-CUT}\smash{^A}]{\oseq{\octx'_L \oc \octx_L \oc \octx_1 \oc (B_1 \limp B_2) \oc \octx_R \oc \octx'_R |- C}}{
        \infer[\lrule{\limp}]{\oseq{\octx_L \oc \octx_1 \oc (B_1 \limp B_2) \oc \octx_R |- A}}{
          \deduce{\oseq{\octx_1 |- B_1}}{\DD_1} &
          \deduce{\oseq{\octx_L \oc B_2 \oc \octx_R |- A}}{\DD_2}} &
        \deduce{\oseq{\octx'_L \oc A \oc \octx'_R |- C}}{\EE}}
      \\=\\
      \infer[\lrule{\limp}]{\oseq{\octx'_L \oc \octx_L \oc \octx_1 \oc (B_1 \limp B_2) \oc \octx_R \oc \octx'_R |- C}}{
        \deduce{\oseq{\octx_1 |- B_1}}{\DD_1} &
        \infer-[\jrule{A-CUT}\smash{^A}]{\oseq{\octx'_L \oc \octx_L \oc B_2 \oc \octx_R \oc \octx'_R |- C}}{
          \deduce{\oseq{\octx_L \oc B_2 \oc \octx_R |- A}}{\DD_2} &
          \deduce{\oseq{\octx'_L \oc A \oc \octx'_R |- C}}{\EE}}}
    \end{gather*}
    And one right-commutative case involves a right-hand proof that ends with the $\rrule{\limp}$ rule:
    \begin{equation*}
      \infer-[\jrule{A-CUT}\smash{^A}]{\oseq{\octx'_L \oc \octx \oc \octx'_R |- C_1 \limp C_2}}{
        \deduce{\oseq{\octx |- A}}{\DD} &
        \infer[\rrule{\limp}]{\oseq{\octx'_L \oc A \oc \octx'_R |- C_1 \limp C_2}}{
          \deduce{\oseq{C_1 \oc \octx'_L \oc A \oc \octx'_R |- C_2}}{\EE_1}}}
      =
      \infer[\rrule{\limp}]{\oseq{\octx'_L \oc \octx \oc \octx'_R |- C_1 \limp C_2}}{
        \infer-[\jrule{A-CUT}\smash{^A}]{\oseq{C_1 \oc \octx'_L \oc \octx \oc \octx'_R |- C_2}}{
          \deduce{\oseq{\octx |- A}}{\DD} &
          \deduce{\oseq{C_1 \oc \octx'_L \oc A \oc \octx'_R |- C_2}}{\EE_1}}}
    \end{equation*}
    And still other right-commutative cases involve a right-hand proof that ends with the $\lrule{\limp}$ rule.
    % \begin{gather*}
    %   \infer-[\jrule{A-CUT}\smash{^A}]{\oseq{\octx'_L \oc \octx \oc \octx'_M \oc \octx'_1 \oc (B_1 \limp B_2) \oc \octx'_R |- C}}{
    %     \deduce{\oseq{\octx |- A}}{\DD} &
    %     \infer[\lrule{\limp}]{\oseq{\octx'_L \oc A \oc \octx'_M \oc \octx'_1 \oc (B_1 \limp B_2) \oc \octx'_R |- C}}{
    %       \deduce{\oseq{\octx'_1 |- B_1}}{\EE_1} &
    %       \deduce{\oseq{\octx'_L \oc A \oc \octx'_M \oc B_2 \oc \octx'_R |- C}}{\EE_2}}}
    %   \\=\\
    %   \infer[\lrule{\limp}]{\oseq{\octx'_L \oc \octx \oc \octx'_M \oc \octx'_1 \oc (B_1 \limp B_2) \oc \octx'_R |- C}}{
    %     \deduce{\oseq{\octx'_1 |- B_1}}{\EE_1} &
    %     \infer-[\jrule{A-CUT}\smash{^A}]{\oseq{\octx'_L \oc \octx \oc \octx'_M \oc B_2 \oc \octx'_R |- C}}{
    %       \deduce{\oseq{\octx |- A}}{\DD} &
    %       \deduce{\oseq{\octx'_L \oc A \oc \octx'_M \oc B_2 \oc \octx'_R |- C}}{\EE_2}}}
    % \end{gather*}
  \qedhere
  \end{description}
\end{proof}

With the admissibility of a cut principle for cut-free proofs established, we may finally prove a cut elimination result.
%
\begin{theorem}[Cut elimination]\label{thm:ordered-logic:cut-elimination}
  If a proof of\/ $\oseq{\octx |- A}$ exists, then a \emph{cut-free} proof of $\oseq{\octx |- A}$ exists.
\end{theorem}
%
\begin{proof}
  By structural induction on the proof of $\oseq{\octx |- A}$, appealing to the admissibility of cut\parencref{lem:ordered-logic:cut-admissibility} when encountering a $\jrule{CUT}$ rule.
  This proof amounts to the definition of a meta-level function for normalizing proofs to cut-free form.

  If we display this \lcnamecref{thm:ordered-logic:cut-elimination} as an admissible rule, $\jrule{CE}$, then the crucial case in this proof is resolved as follows:
  \begin{equation*}
    \infer-[\jrule{CE}]{\oseqcf{\octx'_L \oc \octx \oc \octx'_R |- C}}{
      \infer[\jrule{CUT}\smash{^A}]{\oseq{\octx'_L \oc \octx \oc \octx'_R |- C}}{
        \oseq{\octx |- A} & \oseq{\octx'_L \oc A \oc \octx'_R |- C}}}
    =
    \infer-[\jrule{A-CUT}\smash{^A}]{\oseqcf{\octx'_L \oc \octx \oc \octx'_R |- C}}{
      \infer[\jrule{CE}]{\oseqcf{\octx |- A}}{
        \oseq{\octx |- A}} &
      \infer[\jrule{CE}]{\oseqcf{\octx'_L \oc A \oc \octx'_R |- C}}{
        \oseq{\octx'_L \oc A \oc \octx'_R |- C}}}
  \end{equation*}
  All other cases are resolved compositionally.
\end{proof}


\subsection{Identity elimination}

In addition to cut elimination, we would like to prove an identity elimination result --
First, however, we need to prove that the identity principle is admissible for 

\begin{lemma}[Admissibility of identity]\label{lem:ordered-logic:identity-admissibility}
  For all propositions $A$, a \emph{long} proof of $\oseq{A |- A}$ exists.
  Moreover, this proof is cut-free.
\end{lemma}
%
\begin{proof}
  As usual, by induction on the structure of the proposition $A$.

  In the base case of propositional atoms $\alpha$, the instance of the $\jrule{ID}$ rule at type $\alpha$ is itself already long:
  \begin{equation*}
    \infer-[\jrule{A-ID}\smash{^{\alpha}}]{\oseqel{\alpha |- \alpha}}{}
    =
    \infer[\jrule{ID}\smash{^{\alpha}}]{\oseqel{\alpha |- \alpha}}{}
  \end{equation*}

  For compound propositions, the long proof of $\oseq{A |- A}$ is constructed from right and left rules, together with calls to the admissible $\jrule{A-ID}$ rule at subformulas of $A$.
  For example, the identity expansion at $A_1 \limp A_2$ is:
  \begin{equation*}
    \infer-[\jrule{A-ID}\smash{^{A_1 \limp A_2}}]{\oseqel{A_1 \limp A_2 |- A_1 \limp A_2}}{}
    =
    \infer[\rrule{\limp}]{\oseqel{A_1 \limp A_2 |- A_1 \limp A_2}}{
      \infer[\lrule{\limp}]{\oseqel{A_1 \oc (A_1 \limp A_2) |- A_2}}{
        \infer-[\jrule{A-ID}\smash{^{A_1}}]{\oseqel{A_1 |- A_1}}{} &
        \infer-[\jrule{A-ID}\smash{^{A_2}}]{\oseqel{A_2 |- A_2}}{}}}
  \end{equation*}
  The remaining cases are similar.
\end{proof}

\begin{theorem}[Identity elimination]
  If a proof of\/ $\oseq{\octx |- A}$ exists, then a \emph{long} proof of\/ $\oseq{\octx |- A}$ exists.
  Moreover, cut-freeness is preserved.
\end{theorem}
%
\begin{proof}
  By structural induction on the proof of $\oseq{\octx |- A}$.

  If the given proof is instance of the identity rule, then an appeal to the admissible $\jrule{A-ID}$ rule\parencref{lem:ordered-logic:identity-admissibility} yields a long proof of $\oseq{\octx |- A}$:
  \begin{equation*}
    \infer-[\jrule{IE}]{\oseqel{A |- A}}{
      \infer[\jrule{ID}\smash{^A}]{\oseq{A |- A}}{}}
    =
    \infer-[\jrule{A-ID}\smash{^A}]{\oseqel{A |- A}}{}
  \end{equation*}
  As part of \cref{lem:ordered-logic:identity-admissibility}, we know that this proof is also cut-free.

  The remaining cases are resolved compositionally.
  For example:
  \begin{equation*}
    \infer-[\jrule{IE}]{\oseqel{\octx |- A_1 \limp A_2}}{
      \infer[\rrule{\limp}]{\oseq{\octx |- A_1 \limp A_2}}{
        \deduce{\oseq{A_1 \oc \octx |- A_2}}{\DD_1}}}
    =
    \infer[\rrule{\limp}]{\oseqel{\octx |- A_1 \limp A_2}}{
      \infer-[\jrule{IE}]{\oseqel{A_1 \oc \octx |- A_2}}{
        \deduce{\oseq{A_1 \oc \octx |- A_2}}{\DD_1}}}
  \end{equation*}
  Notice that no case introduces any instances of the $\jrule{CUT}$ beyond those that were already present.
  Thus, identity elimination preserves cut-freeness.
\end{proof}

\subsection{Proof normalization}

\begin{corollary}[Proof normalization]
  If a proof of\/ $\oseq{\octx |- A}$ exists, then a \emph{verification} (\ie, a cut-free, long proof) of\/ $\oseq{\octx |- A}$ exists.
\end{corollary}
%
\begin{proof}
  Given a proof of $\oseq{\octx |- A}$, first apply cut elimination\parencref{thm:ordered-logic:cut-elimination} to get a cut-free proof, $\oseqcf{\octx |- A}$.
  Then apply identity elimination\parencref{thm:ordered-logic:identity-elimination} to that cut-free proof to get a cut-free, long proof of the same sequent -- in other words, to get a verification $\oseqv{\octx |- A}$.
  \begin{equation*}
    \infer-[\jrule{NORM}]{\oseqv{\octx |- A}}{
      \deduce{\oseq{\octx |- A}}{\DD}}
    =
    \infer-[\jrule{IE}]{\oseqv{\octx |- A}}{
      \infer-[\jrule{CE}]{\oseqcf{\octx |- A}}{
        \deduce{\oseq{\octx |- A}}{\DD}}}
  \end{equation*}
\end{proof}


\section{Extensions}\label{sec:ordered-logic:extensions}

In this \lcnamecref{sec:ordered-logic:extensions}, we give a brief overview of several extensions to the preceding ordered sequent calculus
\begin{itemize*}[label=, before=\unskip{:}, itemjoin={,}, itemjoin*={, and}]
\item first-order universal and existential quantifiers
\item multiplicative falsehood
\item mobility and persistence modalities
\end{itemize*}.
These extensions are not crucial to the remainder of this dissertation, but are included for the sake of completeness.

\paragraph*{First-order quantification}

The manner in which first-order universal and existential quantifiers, $\forall x{:}\tau.A$ and $\exists x{:}\tau.A$, may be added to the ordered sequent calculus is completely standard.
Sequents are extended with a separate context, $\Sigma$, of well-sorted term variables, $x{:}\tau$; this new context is structural, admitting weakening, contraction, and exchange properties.

  \begin{inferences}
    \infer[\rrule{\forall}]{\oseq{\Sigma ; \octx |- \forall x{:}\tau.A}}{
      \oseq{\Sigma, a{:}\tau ; \octx |- [a/x]A}}
    \and
    \infer[\lrule{\forall}]{\oseq{\Sigma ; \octx'_L \oc (\forall x{:}\tau.A) \oc \octx_R |- C}}{
      \Sigma \vdash t : \tau &
      \oseq{\Sigma ; \octx_L \oc ([t/x]A) \oc \octx_R |- C}}
    \\
    \infer[\rrule{\exists}]{\oseq{\Sigma ; \octx |- \exists x{:}\tau.A}}{
      \Sigma \vdash t : \tau &
      \oseq{\Sigma ; \octx |- [t/x]A}}
    \and
    \infer[\lrule{\exists}]{\oseq{\Sigma ; \octx'_L \oc (\exists x{:}\tau.A) \oc \octx'_R |- C}}{
      \oseq{\Sigma, a{:}\tau ; \octx'_L \oc ([a/x]A) \oc \octx'_R |- C}}
  \end{inferences}

\paragraph*{Multiplicative falsehood}

\begin{inferences}
  \infer[\rrule{\bot}]{\oseq{\octx |- \bot}}{
    \oseq{\octx |- \cdot}}
  \and
  \infer[\lrule{\bot}]{\oseq{\bot |- \cdot}}{}
\end{inferences}


\paragraph*{Mobility and persistence modalities}


\clearpage

Ordered logic is a generalization of \citeauthor{Girard:TCS?}'s linear logic\autocite{Girard:TCS?} that further restricts the admitted structural properties.
In addition to ... the weakening and contraction

Like the intuitionistic linear sequent calculus, the ordered sequent calculus can be given a resource interpretation.
Because the ordered sequent calculus rejects the exchange property, the resouce interpretation differs slightly from that of linear logic.

\begin{itemize}
\item \citeauthor{Martin-Lof:NJPL96}: Distinguish propositions from judgments

\item Ordered sequents are hypothetical judgments
  \begin{equation*}
    \oseq{A_1 \oc A_2 \dotsb A_n |- A} \,,
  \end{equation*}
  meaning that using the (ordered) resources $A_1, A_2, \dotsc, A_n$, the resource $A$ can be produced.
  A proof of this sequent amounts to a recipe or set of instructions for producing resource $A$ from resources $A_1 \oc A_2 \dotsb A_n$.

\item Contexts as (noncommutative) free monoid over resources -- associative and unit laws

\item $A \fuse B$ as resources $A$ and $B$ side by side, as a single resource package.
  \begin{equation*}
    \infer[\rrule{\fuse}]{\oseq{\octx_1 \oc \octx_2 |- A \fuse B}}{
      \oseq{\octx_1 |- A} & \oseq{\octx_2 |- B}}
  \end{equation*}
  The sequence of resources $\octx$ can be used to produce the resource $A \fuse B$ if the resources $\octx$ can be partitioned into subsequences $\octx_1$ and $\octx_2$ and those subsequences can be used to produce the resources $A$ and $B$, respectively.
  \begin{equation*}
    \infer[\lrule{\fuse}]{\oseq{\octx'_L \oc (A \fuse B) \oc \octx'_R |- C}}{
      \oseq{\octx'_L \oc A \oc B \oc \octx'_R |- C}}
  \end{equation*}
  To use the resource package $A \fuse B$ to produce $C$, unwrap the package and use the side-by-side resources $A \oc B$ to produce $C$.

\item Harmony:
  \begin{gather*}
    \infer[\jrule{CUT}\smash{^{A \fuse B}}]{\oseq{\octx'_L \oc (\octx_1 \oc \octx_2) \oc \octx'_R |- C}}{
      \infer[\rrule{\fuse}]{\oseq{\octx_1 \oc \octx_2 |- A \fuse B}}{
        \oseq{\octx_1 |- A} & \oseq{\octx_2 |- B}} &
      \infer[\lrule{\fuse}]{\oseq{\octx'_L \oc (A \fuse B) \oc \octx'_R |- C}}{
        \oseq{\octx'_L \oc A \oc B \oc \octx'_R |- C}}}
    \\\rightsquigarrow\\
    \infer[\jrule{CUT}\smash{^B}]{\oseq{\octx'_L \oc \octx_1 \oc \octx_2 \oc \octx'_R |- C}}{
      \oseq{\octx_2 |- B} &
      \infer[\jrule{CUT}\smash{^A}]{\oseq{\octx'_L \oc \octx_1 \oc B \oc \octx'_R |- C}}{
        \oseq{\octx_1 |- A} & \oseq{\octx'_L \oc A \oc B \oc \octx'_R |- C}}}
  \end{gather*}

  \begin{equation*}
    \infer[\jrule{ID}\smash{^{A \fuse B}}]{\oseq{A \fuse B |- A \fuse B}}{}
    \leftrightsquigarrow
    \infer[\lrule{\fuse}]{\oseq{A \fuse B |- A \fuse B}}{
      \infer[\rrule{\fuse}]{\oseq{A \oc B |- A \fuse B}}{
        \infer[\jrule{ID}\smash{^A}]{\oseq{A |- A}}{} &
        \infer[\jrule{ID}\smash{^B}]{\oseq{B |- B}}{}}}
  \end{equation*}

\item How do we know that these rules constitute the beginnings of a logic?
  Either \citeauthor{Dummett:HUP91} (harmony of the logical rules) or \citeauthor{Martin-Lof:NJPL96} (verifications).
  \begin{itemize}
  \item These are not unrelated, but we will adopt verifications as our basis. 
  \item Verifications use only the logical rules, not the judgmental $\jrule{CUT}$ and $\jrule{ID}$.
    This way, the meaning of a proposition depends only on its internal structure, not on other propositions or connectives.
  \end{itemize}

\end{itemize}



\begin{align*}
  n(\slof{A |- \spawn{P_1}{^B P_2} : C}) &= \nspawn{n(\slof{A |- P_1 : B})}{n(\slof{B |- P_2 : C})} \\
  n(\slof{A |- \fwd : A}) &= \eta(A) \\
  n(\selectR{\kay}) &= \selectR{\kay} \\
  n(\slof{\plus*[sub=_{\ell \in L}]{\ell:A_{\ell}} |- \caseL[\ell \in L]{\ell => P_{\ell}} : C}) &= \caseL[\ell \in L]{\ell => n(\slof{A_{\ell} |- P_{\ell} : C})}
\end{align*}

\begin{align*}
  \nspawn{\fwd}{M} &= M \\
  \nspawn{(\spawn{N_0}{\selectR{\kay}})}{M} &= \nspawn{N_0}{(\nspawn{\selectR{\kay}}{M})} \\
  \nspawn{\selectR{\kay}}{\caseL[\ell \in L]{\ell => M_{\ell}}} &= M_{\kay} \\
  \nspawn{N}{\selectR{\kay}} &= \spawn{N}{\selectR{\kay}} \\
  \nspawn{N}{(\spawn{M_0}{\selectR{\kay}})} &= \spawn{(\nspawn{N}{M_0})}{\selectR{\kay}} \\
  \nspawn{\caseL[\ell \in L]{\ell => N_{\ell}}}{M} &= \caseL[\ell \in L]{\ell => \nspawn{N_{\ell}}{M}}
\end{align*}


\clearpage













\section{The non-modal fragment of intuitionistic ordered logic}

\subsection{Judgments, sequents, and contexts}

Following \citeauthor{Martin-Lof:NJPL96}\autocite{Martin-Lof:NJPL96}, we maintain a separation of ordered propositions, $A$, from judgments about those propositions.
  The categorical judgment $A \ord$ ... 

To allow hypothetical reasoning, a new form of categorical judgment, $A \ant$, for antecedents and generalize $A \ord$ to sequents
  \begin{equation*}
    \oseq{(A_1 \ant) \dotsm (A_i \ant) \dotsm (A_n \ant) |- A \ord}
    \,,
  \end{equation*}
  meaning the ordered sequence $A_1 \ant \dotsm A_n \ant$ can be transformed into $A \ord$.
  Because the judgment label can be inferred from a proposition's position in the sequent, we usually elide the labels and write $\oseq{A_1 \oc A_2 \dotsm A_n |- A}$.

To further streamline the syntax of sequents,
% keep sequents from becoming verbose and cumbersome,
antecedents are usually collected into an \vocab{ordered context}, $\octx = A_1 \oc A_2 \dotsm A_n$; sequents are then written $\oseq{\octx |- A}$.
% 
As strings of antecedents, ordered contexts form a free monoid.
The monoid operation is concatenation of contexts, written as juxtaposition; the unit element is the empty context, written as $\octxe$.
In other words, ordered contexts are generated by the grammar
\begin{syntax*}
  & \octx & \octx_1 \oc \octx_2 \mid \octxe \mid A \ant
  \,,
\end{syntax*}
and are subject to the usual associativity and unit laws.
\begin{marginfigure}
  % \vspace*{\dimexpr-\abovedisplayskip\relax}
  \begin{gather*}
    (\octx_1 \oc \octx_2) \oc \octx_3 = \octx_1 \oc (\octx_2 \oc \octx_3) \\
    (\octxe) \oc \octx = \octx = \octx \oc (\octxe)
  \end{gather*}
  \caption{Monoid laws for ordered contexts}
\end{marginfigure}%
% which may be silently applied as needed within proofs.
% These monoid laws are applied implicitly as needed within a proof.
Because contexts are ordered, the underlying monoid is not commutative.


\subsection{}

Following \textcite{Martin-Lof:NJPL96}, we maintain a separation of ordered propositions from judgements about those propositions.
The categorical judgement $A \ord$ holds if language $A$ is inhabited [if there exists a string of type $A$].

To allow hypothetical reasoning, the judgement $A \ord$ is generalized to sequents
\begin{equation*}
  \oseq{(A_1 \ant) \oc (A_2 \ant) \dotsb (A_n \ant) |- A \ord}
  \,,
\end{equation*}
meaning that any string drawn from the concatenation of languages $A_1 \oc A_2 \dotsb A_n$ is a member of the language $A$.
... if the concatenation of languages ... is inhabited, then any inhabitant also inhabits the language $A$.

\begin{itemize}
\item $\oseq{A_1 \oc A_2 \dotsb A_n |- A}$ means that a string of type $A$ may be obtained from the concatenation of strings of types $A_1, A_2, \dotsc, A_n$.
\item $\oseq{A_1 \oc A_2 \dotsb A_n |- A}$ means that any string expressible as the concatenation of strings of types $A_1, A_2, \dotsc, A_n$ is also a string of type $A$.
\end{itemize}

\begin{itemize}
\item $a$ -- trivial language containing only the single-letter string $a$
\item $A \fuse B$ -- concatenation of languages
\item $\one$ -- trivial language containing only the empty string
\item $A \with B$ -- intersection of languages
\item $\top$ -- universal language
\item $A \plus B$ -- union of languages
\item $\zero$ -- empty language
\item $A \limp B$ -- left quotient of languages
\item $B \pmir A$ -- right quotient of languages
\end{itemize}

$\vdash$ as language inclusion?
What about the language $a \with b$?  The intersection of languages $a$ and $b$ is empty, but $a \with b \nvdash \zero$.

The right rule says that any string that can be partitioned into consecutive pieces drawn from the languages $\octx_1$ and $\octx_2$ is a member of language $A \fuse B$ if $\octx_1$ is contained in $A$ and $\octx_2$ is contianed in $B$.
\begin{inferences}
  \infer[\rrule{\fuse}]{\oseq{\octx_1 \oc \octx_2 |- A \fuse B}}{
    \oseq{\octx_1 |- A} & \oseq{\octx_2 |- B}}
  \and
  \infer[\lrule{\fuse}]{\oseq{\octx'_L \oc (A \fuse B) \oc \octx'_R |- C}}{
    \oseq{\octx'_L \oc A \oc B \oc \octx'_R |- C}}
\end{inferences}

\subsection{Judgmental principles of sequents}

Even without knowing the specific structure of propositions, we can already enumerate several judgmental principles that must hold if sequents are to accurately reflect hypothetical reasoning.

First, if we assume that $A$ is ..., then 

If we are given a string that parses to $A$, then that string may be trivially transformed to a string that parses to $A$ -- simply use the same string.
Cast as a sequent:
\begin{description}
\item[Identity principle]
  $\oseq{A |- A}$ for all propositions $A$.
\end{description}

Dually, a string that parses to $A$ licenses a hypothesis of $A$:
\begin{description}
\item[Cut principle]
  If $\oseq{\octx |- A}$ and $\oseq{\octx'_L \oc A \oc \octx'_R |- C}$, then $\oseq{\octx'_L \oc \octx \oc \octx'_R |- C}$.
\end{description}

These two judgmental principles are adopted by the ordered sequent calculus as primitive rules of inference.
But their importance goes beyound that of mere rules of inference, with the two principles playing an important role in the meaning of the logical connectives.

\subsection{The logical connectives and their meanings}

\begin{inferences}
  \infer[\rrule{\fuse}]{\oseq{\octx_1 \oc \octx_2 |- A \fuse B}}{
    \oseq{\octx_1 |- A} & \oseq{\octx_2 |- B}}
  \and
  \infer[\lrule{\fuse}]{\oseq{\octx'_L \oc (A \fuse B) \oc \octx'_R |- C}}{
    \oseq{\octx'_L \oc A \oc B \oc \octx'_R |- C}}
\end{inferences}



\clearpage 

\newthought{Even without} knowing any specifics about propositions, two purely judgmental principles are already apparent.
%
First, there should be a trivial transformation of $A$ into $A$, for all propositions $A$.
In the sequent calculus, this idea is rendered as an \vocab{identity principle}:
\begin{quotation}
  $\oseq{A \ant |- A \ord}$ for all propositions $A$.
\end{quotation}
Stated differently, $\vdash$ is reflexive.%
\footnote{Actually, this is not precisely true because the two sides of the turnstile use different judgments.
The intuition is nevertheless useful.}

Second, and dually, a proof of $A \ord$ should license the use of $A \ant$ in another proof.
\begin{quotation}
  If $\oseq{\octx |- A \ord}$ and $\oseq{\octx'_L \oc (A \ant) \oc \octx'_R |- C \ord}$, then $\oseq{\octx'_L \oc \octx \oc \octx'_R |- C \ord}$.
\end{quotation}

These two judgmental principles are adopted as primitive rules of inference in the sequent calculus.
\begin{inferences}
  \infer[\jrule{CUT}\smash{^A}]{\oseq{\octx'_L \oc \octx \oc \octx'_R |- C}}{
    \oseq{\octx |- A} & \oseq{\octx'_L \oc A \oc \octx'_R |- C}}
  \and
  \infer[\jrule{ID}\smash{^A}]{\oseq{A |- A}}{}
\end{inferences}
The importance of the cut and identity principles goes beyond that of their corresponding inference rules, however.
Both principles are intimately linked to what counts as the meanings of the logical connectives.

\subsection{Propositions}

The propositional, purely ordered fragment of intuitionistic ordered logic has propositions given by the following grammar.
\begin{syntax*}
  Propositions &
    A,B,C & \begin{array}[t]{@{}l@{}}
              \alpha \mid A \limp B \mid B \pmir A
                \mid A \fuse B \mid \one \\
              \mathllap{\mid {}} A \with B \mid \top
                \mid A \plus B \mid \zero
            \end{array}
\end{syntax*}
The ordered propositions in this fragment are:
propositional variables, $\alpha$;
left- and right-handed implications, $A \limp B$ and $B \pmir A$, respectively;
ordered conjunction, $A \fuse B$, and its unit, $\one$;
alternative conjunction, $A \with B$, and its unit, $\top$;
and
additive disjunction, $A \plus B$, and its unit, $\zero$.

In the tradition of \citeauthor{Gentzen:MZ35} and \citeauthor{Martin-Lof:NJPL96}\autocites{Gentzen:MZ35}{Martin-Lof:NJPL96}, the meaning of a proposition $A$ is given by what counts as a verification of the judgment $A \ord$.

Left-handed implication has the following right and left rules.
\begin{inferences}
  \infer[\rrule{\limp}]{\oseq{\octx |- A \limp B}}{
    \oseq{A \oc \octx |- B}}
  \and
  \infer[\lrule{\limp}]{\oseq{\octx'_L \oc \octx \oc (A \limp B) \oc \octx'_R |- C}}{
    \oseq{\octx |- A} & \oseq{\octx'_L \oc B \oc \octx'_R |- C}}
\end{inferences}
According to the right rule, $\rrule{\limp}$, verifying the left-handed implication $A \limp B$ amounts to verifying $B$ under the left-extended context $A \oc \octx$ -- that is, a hypothetical proof of $\oseq{A \oc \octx |- B}$.
The left rule, $\lrule{\limp}$, shows how to use such a proof:
Prove $A$ and left-adjoin that proof to the verification of $A \limp B$, which is just a hypothetical proof of $\oseq{A \oc \octx |- B}$.
This yields 

Right-handed implication is symmetric to its left-handed counterpart:
\begin{inferences}
  \infer[\rrule{\pmir}]{\oseq{\octx |- B \pmir A}}{
    \oseq{\octx \oc A |- B}}
  \and
  \infer[\lrule{\pmir}]{\oseq{\octx'_L \oc (B \pmir A) \oc \octx \oc \octx'_R |- C}}{
    \oseq{\octx |- A} & \oseq{\octx'_L \oc B \oc \octx'_R |- C}}
\end{inferences}

Ordered conjunction is another multiplicative connective; its right and left rules are:
\begin{inferences}
  \infer[\rrule{\fuse}]{\oseq{\octx_1 \oc \octx_2 |- A \fuse B}}{
    \oseq{\octx_1 |- A} & \oseq{\octx_2 |- B}}
  \and
  \infer[\lrule{\fuse}]{\oseq{\octx'_L \oc (A \fuse B) \oc \octx'_R |- C}}{
    \oseq{\octx'_L \oc A \oc B \oc \octx'_R |- C}}
\end{inferences}
As its right rule makes clear, ordered conjunction internalizes concatenation of contexts as a logical connective.

$\one$ internalizes the empty context.
The logical constant $\one$ is the nullary analogue to binary ordered conjunction, as its right and left rules reflect:
\begin{inferences}
  \infer[\rrule{\one}]{\oseq{\octxe |- \one}}{}
  \and
  \infer[\lrule{\one}]{\oseq{\octx'_L \oc \one \oc \octx'_R |- C}}{
    \oseq{\octx'_L \oc \octx'_R |- C}}
\end{inferences}
Consequently, $\one$ is the unit of ordered conjunction: $\oseq{\one \fuse A \dashv|- \oseq{A \dashv|- A \fuse \one}}$, for all propositions $A$.


\begin{equation*}
  \infer[\rrule{\limp}]{\oseq{A \limp (B \limp C) |- (A \esuf B) \limp C}}{
    \infer[\lrule{\esuf}]{\oseq{(A \esuf B) \oc (A \limp (B \limp C)) |- C}}{
      \infer[\lrule{\limp}]{\oseq{B \oc A \oc (A \limp (B \limp C)) |- C}}{
        \infer[\jrule{ID}]{\oseq{A |- A}}{} &
        \infer[\lrule{\limp}]{\oseq{B \oc (B \limp C) |- C}}{
          \infer[\jrule{ID}]{\oseq{B |- B}}{} &
          \infer[\jrule{ID}]{\oseq{C |- C}}{}}}}}
\end{equation*}

\begin{equation*}
  \infer[\rrule{\limp}]{\oseq{(A \esuf B) \limp C |- A \limp (B \limp C)}}{
    \infer[\rrule{\limp}]{\oseq{A \oc ((A \esuf B) \limp C) |- B \limp C}}{
      \infer[\lrule{\limp}]{\oseq{B \oc A \oc ((A \esuf B) \limp C) |- C}}{
        \infer[\rrule{\esuf}]{\oseq{B \oc A |- A \esuf B}}{
          \infer[\jrule{ID}]{\oseq{B |- B}}{} &
          \infer[\jrule{ID}]{\oseq{A |- A}}{}} &
        \infer[\jrule{ID}]{\oseq{C |- C}}{}}}}
\end{equation*}

\begin{equation*}
  \infer[\rrule{\pmir}]{\oseq{(C \pmir B) \pmir A |- C \pmir (B \esuf A)}}{
    \infer[\lrule{\esuf}]{\oseq{((C \pmir B) \pmir A) \oc (B \esuf A) |- C}}{
      \infer[\lrule{\pmir}]{\oseq{((C \pmir B) \pmir A) \oc A \oc B |- C}}{
        \infer[\jrule{ID}]{\oseq{A |- A}}{} &
        \infer[\lrule{\pmir}]{\oseq{(C \pmir B) \oc B |- C}}{
          \infer[\jrule{ID}]{\oseq{B |- B}}{} &
          \infer[\jrule{ID}]{\oseq{C |- C}}{}}}}}
\end{equation*}

\begin{equation*}
  \infer[\rrule{\pmir}]{\oseq{C \pmir (B \esuf A) |- (C \pmir B) \pmir A}}{
    \infer[\rrule{\pmir}]{\oseq{(C \pmir (B \esuf A)) \oc A |- C \pmir B}}{
      \infer[\lrule{\pmir}]{\oseq{(C \pmir (B \esuf A)) \oc A \oc B |- C}}{
        \infer[\rrule{\esuf}]{\oseq{A \oc B |- B \esuf A}}{
          \infer[\jrule{ID}]{\oseq{A |- A}}{} &
          \infer[\jrule{ID}]{\oseq{B |- B}}{}} &
        \infer[\jrule{ID}]{\oseq{C |- C}}{}}}}
\end{equation*}

\begin{itemize}
\item Contexts form a monoid
\item None of the usual structural properties -- weakening, contraction, exchange -- but we still have (implicitly, judgmentally) associativity
\item Multiplicative falsehood\alertnote{Get rid of this?}
\end{itemize}

\subsection{Rules of logical inference}

According to its right rule, verifying $A \limp B$ amounts to verifying $B$ under the additional assumption that $A$ holds;
moreover, the assumption $A$ is prepended to the left end of context $\octx$, because $A \limp B$ is a left-handed implication.

To use a verification of $A \limp B$, we first verify $A$ and are thus justified in using $B$.

Using a verification of $A \limp B$ thus amounts to using the hypothetical verification of $B$ under $A$


The right-handed implication $B \pmir A$ is symmetric to left-handed implication.


\begin{figure}
  \begin{syntax*}
    Propositions & A, B, C &
      \begin{array}[t]{@{}l@{}}
        A \limp B \mid B \pmir A \mid A \fuse B \mid B \esuf A \mid \one \\
        \mathllap{\mid {}}
        A \plus B \mid \zero \mid A \with B \mid \top
      \end{array}
    \\
    Contexts & \octx &
      \octxe \mid \octx_1 \oc \octx_2 \mid A
  \end{syntax*}
  \begin{inferences}
    \infer[\jrule{CUT}\smash{^A}]{\oseq{\octx'_L \oc \octx \oc \octx'_R |- \cseq}}{
      \oseq{\octx |- A} & \oseq{\octx'_L \oc A \oc \octx'_R |- \cseq}}
    \and
    \infer[\jrule{ID}\smash{^A}]{\oseq{A |- A}}{}
    \\
    \infer[\rrule{\limp}]{\oseq{\octx |- A \limp B}}{
      \oseq{A \oc \octx |- B}}
    \and
     \infer[\lrule{\limp}]{\oseq{\octx'_L \oc \octx \oc (A \limp B) \oc \octx'_R |- C}}{
      \oseq{\octx |- A} & \oseq{\octx'_L \oc B \oc \octx'_R |- C}}
    \\
    \infer[\rrule{\pmir}]{\oseq{\octx |- B \pmir A}}{
      \oseq{\octx \oc A |- B}}
    \and
    \infer[\lrule{\pmir}]{\oseq{\octx'_L \oc (B \pmir A) \oc \octx \oc \octx'_R |- C}}{
      \oseq{\octx |- A} & \oseq{\octx'_L \oc B \oc \octx'_R |- C}}
    \\
    \infer[\rrule{\fuse}]{\oseq{\octx_A \oc \octx_B |- A \fuse B}}{
      \oseq{\octx_A |- A} & \oseq{\octx_B |- B}}
    \and
    \infer[\lrule{\fuse}]{\oseq{\octx'_L \oc (A \fuse B) \oc \octx'_R |- C}}{
      \oseq{\octx'_L \oc A \oc B \oc \octx'_R |- C}}
    \\
    \infer[\rrule{\esuf}]{\oseq{\octx_A \oc \octx_B |- B \esuf A}}{
      \oseq{\octx_A |- A} & \oseq{\octx_B |- B}}
    \and
    \infer[\lrule{\esuf}]{\oseq{\octx'_L \oc (B \esuf A) \oc \octx'_R |- C}}{
      \oseq{\octx'_L \oc A \oc B \oc \octx'_R |- C}}
    \\
    \infer[\rrule{\one}]{\oseq{\octxe |- \one}}{}
    \and
    \infer[\lrule{\one}]{\oseq{\octx'_L \oc \one \oc \octx'_R |- C}}{
      \oseq{\octx'_L \oc \octx'_R |- C}}
    \\
    \infer[\rrule{\with}]{\oseq{\octx |- A \with B}}{
      \oseq{\octx |- A} & \oseq{\octx |- B}}
    \and
    \infer[\lrule{\with}_1]{\oseq{\octx'_L \oc (A \with B) \oc \octx'_R |- C}}{
      \oseq{\octx'_L \oc A \oc \octx'_R |- C}}
    \and
    \infer[\lrule{\with}_2]{\oseq{\octx'_L \oc (A \with B) \oc \octx'_R |- C}}{
      \oseq{\octx'_L \oc B \oc \octx'_R |- C}}
    \\
    \infer[\rrule{\top}]{\oseq{\octx |- \top}}{}
    \and
    \text{(no $\lrule{\top}$ rule)}
    \\
    \infer[\rrule{\plus}_1]{\oseq{\octx |- A \plus B}}{
      \oseq{\octx |- A}}
    \and
    \infer[\rrule{\plus}_2]{\oseq{\octx |- A \plus B}}{
      \oseq{\octx |- B}}
    \and
    \infer[\lrule{\plus}]{\oseq{\octx'_L \oc (A \plus B) \oc \octx'_R |- C}}{
      \oseq{\octx'_L \oc A \oc \octx'_R |- C} &
      \oseq{\octx'_L \oc B \oc \octx'_R |- C}}
    \\
    \text{(no $\rrule{\zero}$ rule)}
    \and
    \infer[\lrule{\zero}]{\oseq{\octx'_L \oc \zero \oc \octx'_R |- C}}{}
  \end{inferences}
\end{figure}  

\begin{theorem}[Admissibility of cut]
  If $\oseq{\octx |- A}$ and $\oseq{\octx'_L \oc A \oc \octx'_R |- C}$, then $\oseq{\octx'_L \oc \octx \oc \octx'_R |- C}$.
\end{theorem}
\begin{proof}
  By lexicographic structural induction, first on the structure of the cut formula and then simultaneously on the structures of the given derivations.

  \begin{description}
  \item[Identity cases]
    \begin{equation*}
      \infer-[\jrule{CUT}^A]{\oseq{\octx'_L \oc A \oc \octx'_R |- C}}{
        \infer[\jrule{ID}^A]{\oseq{A |- A}}{} &
        \deduce{\oseq{\octx'_L \oc A \oc \octx'_R |- C}}{\EE}}
      =
      \deduce{\oseq{\octx'_L \oc A \oc \octx'_R |- C}}{\EE}
    \end{equation*}

  \item[Principal cases]
    \begin{equation*}
      \infer-[\jrule{CUT}^{A_1 \limp A_2}]{\oseq{\octx'_L \oc \octx'_{A_1} \oc \octx \oc \octx'_R |- C}}{
        \infer[\rrule{\limp}]{\oseq{\octx |- A_1 \limp A_2}}{
          \deduce{\oseq{A_1 \oc \octx |- A_2}}{\DD_1}} &
        \infer[\lrule{\limp}]{\oseq{\octx'_L \oc \octx'_{A_1} \oc (A_1 \limp A_2) \oc \octx'_R |- C}}{
          \deduce{\oseq{\octx'_{A_1} |- A_1}}{\EE_1} &
          \deduce{\oseq{\octx'_L \oc A_2 \oc \octx'_R |- C}}{\EE_2}}}
      =
      \infer-[\jrule{CUT}^{A_2}]{\oseq{\octx'_L \oc \octx'_{A_1} \oc \octx \oc \octx'_R |- C}}{
        \infer-[\jrule{CUT}^{A_1}]{\oseq{\octx'_{A_1} \oc \octx |- A_2}}{
          \deduce{\oseq{\octx'_{A_1} |- A_1}}{\EE_1} &
          \deduce{\oseq{A_1 \oc \octx |- A_2}}{\DD_1}} &
        \deduce{\oseq{\octx'_L \oc A_2 \oc \octx'_R |- C}}{\EE_2}}
    \end{equation*}

  \item[Commutative cases]
    \begin{equation*}
      \infer-[\jrule{CUT}^A]{\oseq{\octx'_L \oc \octx_L \oc (B_1 \with B_2) \oc \octx_R \oc \octx'_R |- C}}{
        \infer[\lrule{\with}_1]{\oseq{\octx_L \oc (B_1 \with B_2) \oc \octx_R |- A}}{
          \deduce{\oseq{\octx_L \oc B_1 \oc \octx_R |- A}}{\DD_1}} &
        \deduce{\oseq{\octx'_L \oc A \oc \octx'_R |- C}}{\EE}}
      =
      \infer[\lrule{\with}_1]{\oseq{\octx'_L \oc \octx_L \oc (B_1 \with B_2) \oc \octx_R \oc \octx'_R |- C}}{
        \infer-[\jrule{CUT}^A]{\oseq{\octx'_L \oc \octx_L \oc B_1 \oc \octx_R \oc \octx'_R |- C}}{
          \deduce{\oseq{\octx_L \oc B_1 \oc \octx_R |- A}}{\DD_1} &
          \deduce{\oseq{\octx'_L \oc A \oc \octx'_R |- C}}{\EE}}}
    \end{equation*}

    \begin{equation*}
      \infer-[\jrule{CUT}^A]{\oseq{\octx'_{L1} \oc \octx \oc \octx'_{L2} \oc (B_1 \with B_2) \oc \octx'_R |- C}}{
        \deduce{\oseq{\octx |- A}}{\DD} &
        \infer[\lrule{\with}_1]{\oseq{\octx'_{L1} \oc A \oc \octx'_{L2} \oc (B_1 \with B_2) \oc \octx'_R |- C}}{
          \deduce{\oseq{\octx'_{L1} \oc A \oc \octx'_{L2} \oc B_1 \oc \octx'_R |- C}}{\EE_1}}}
      =
    \end{equation*}
  \end{description}
\end{proof}

\begin{theorem}[Identity expansion]
  $\oseq{A |- A}$ for all propositions $A$.
\end{theorem}
\begin{proof}
  By induction on the structure of the proposition $A$.
\end{proof}

\section{Extensions}\label{sec:ordered-logic:extensions}

In this \lcnamecref{sec:ordered-logic:extensions}, we give a brief overview of several extensions to the preceding ordered sequent calculus
\begin{itemize*}[label=, before=\unskip{:}, itemjoin={,}, itemjoin*={, and}]
\item first-order universal and existential quantifiers
\item multiplicative falsehood
\item mobility and persistence modalities
\end{itemize*}.
These extensions are not crucial to the remainder of this dissertation, but are included for the sake of completeness.

\paragraph*{First-order quantification}

The manner in which first-order universal and existential quantifiers, $\forall x{:}\tau.A$ and $\exists x{:}\tau.A$, may be added to the ordered sequent calculus is completely standard.
Sequents are extended with a separate context, $\Sigma$, of well-sorted term variables, $x{:}\tau$; this new context is structural, admitting weakening, contraction, and exchange properties.

\begin{marginfigure}
  \begin{inferences}
    \infer[\rrule{\forall}]{\oseq{\Sigma ; \octx |- \forall x{:}\tau.A}}{
      \oseq{\Sigma, a{:}\tau ; \octx |- [a/x]A}}
    \\
    \infer[\lrule{\forall}]{\oseq{\Sigma ; \octx'_L \oc (\forall x{:}\tau.A) \oc \octx_R |- C}}{
      \Sigma \vdash t : \tau &
      \oseq{\Sigma ; \octx_L \oc ([t/x]A) \oc \octx_R |- C}}
    \\
    \infer[\rrule{\exists}]{\oseq{\Sigma ; \octx |- \exists x{:}\tau.A}}{
      \Sigma \vdash t : \tau &
      \oseq{\Sigma ; \octx |- [t/x]A}}
    \\
    \infer[\lrule{\exists}]{\oseq{\Sigma ; \octx'_L \oc (\exists x{:}\tau.A) \oc \octx'_R |- C}}{
      \oseq{\Sigma, a{:}\tau ; \octx'_L \oc ([a/x]A) \oc \octx'_R |- C}}
  \end{inferences}
\end{marginfigure}

\subsection{Multiplicative falsehood}

\subsection{Mobility and persistence modalities}



\section{Circular propositions and circular derivations}

\begin{itemize}
\item No exponentials; recursion/circularity instead (Milner)\alertnote{Should this go in ordered rewriting chapter instead?}
\item $\mu$MALL (Baelde) and circular proofs (Fortier and Santocanale)
\item Contractivity requirement
\item We will use only general recursion.
  Inductive and coinductive types are outside our scope.
\item Subset of infinite propositions/derivations
\end{itemize}


\section{Outline}

\subsection{Judgments, contexts, and sequents}

\begin{itemize}
\item Consequent and antecedent judgments 
  \begin{itemize}
  \item Hypothetical reasoning -- how to gloss an ordered sequent?
  \item Drop judgment labels because they are implied from position in sequents
  \end{itemize}
\item Ordered contexts: monoidal structure and structural properties as algebraic laws
\item Judgmental principles -- identity and cut 
  \begin{itemize}
  \item First rules of inference, but foreshadow connections to verifications 
  \end{itemize}
\end{itemize}

\subsection{Propositions and their meanings}

\begin{itemize}
\item right and left rules 
  \begin{itemize}
  \item right rules show how to verify propositions; left rules show how to use those verifications 
  \end{itemize}
\end{itemize}


%%% Local Variables:
%%% mode: latex
%%% TeX-master: "thesis"
%%% End:
