\chapter{Preliminaries: Intuitionistic ordered logic}\label{ch:ordered-logic}

% In \citeyear{Lambek:??}, \citeauthor{Lambek:??} published a seminal paper developing a formal system for describing sentence structure.
% The Lambek calculus, when viewed from a logical perspective, forms the core of \emph{intuitionistic ordered logic}%
% \footnote{Also known as intuitionistic noncommutative linear logic.}%
% .


This \lcnamecref{ch:ordered-logic} serves to review a sequent calculus presentation of
% the propositional fragment of
intuitionistic ordered logic.
As a substructural logic with the Lambek calculus\autocite{Lambek:??} at its core, intuitionistic ordered logic eschews the usual structural properties of weakening, contraction, and exchange.
As in \citeauthor{Girard:TCS??}'s linear logic\autocite{Girard:TCS??}, the lack of weakening and contraction properties means that each antecedent must be used exactly once within a proof.
In ordered logic, the additional lack of an exchange property means that antecedents must also remain in order within a proof.

\Citeauthor{Lambek:??} leveraged this noncommutativity to give a formal description of sentence structure.



[Outline paragraph]


\section{The propositional, purely ordered fragment of intuitionistic ordered logic}

\begin{itemize}
\item Lambek calculus, given a judgmental reconstruction by Polakow and Pfenning
\item Propositional fragment only, omitting first-order universal and existential quantifiers
\end{itemize}

\subsection{Judgments, sequents, and contexts}

\begin{itemize}
\item Following \citeauthor{Martin-Lof:??}\autocite{Martin-Lof:??}, we maintain a separation of [ordered] propositions, $A$, from judgments about those propositions.
  The categorical judgment $A \ord$ ... 

\item To allow hypothetical reasoning, we introduce a new form of categorical judgment, $A \ant$, for antecedents and generalize $A \ord$ to sequents
  \begin{equation*}
    \oseq{(A_1 \ant) \dotsm (A_i \ant) \dotsm (A_n \ant) |- A \ord}
  \end{equation*}
\end{itemize}

\subsection{Propositions and contexts}

The propositional, purely ordered fragment of intuitionistic ordered logic has propositions given by the following grammar.
\begin{syntax*}
  Propositions &
    A,B,C & \begin{array}[t]{@{}l@{}}
              \alpha \mid A \limp B \mid B \pmir A
                \mid A \fuse B \mid \one \\
              \mathllap{\mid {}} A \with B \mid \top
                \mid A \plus B \mid \zero
            \end{array}
  \\
  Ordered contexts &
    \octx & \octxe \mid \octx_1 \oc \octx_2 \mid A
  \\
  Conseq\relax uents &
    \cseq & A \mid \octxe
\end{syntax*}

\begin{itemize}
\item Contexts form a monoid
\item None of the usual structural properties -- weakening, contraction, exchange -- but we still have (implicitly, judgmentally) associativity
\item Multiplicative falsehood\alertnote{Get rid of this?}
\end{itemize}

\subsection{Rules of logical inference}

\begin{figure}
  \begin{inferences}
    \infer[\jrule{CUT}^A]{\oseq{\octx'_L \oc \octx \oc \octx'_R |- \cseq}}{
      \oseq{\octx |- A} & \oseq{\octx'_L \oc A \oc \octx'_R |- \cseq}}
    \and
    \infer[\jrule{ID}^A]{\oseq{A |- A}}{}
    \\
    \infer[\rrule{\limp}]{\oseq{\octx |- A \limp B}}{
      \oseq{A \oc \octx |- B}}
    \and
    \infer[\lrule{\limp}]{\oseq{\octx'_L \oc \octx \oc (A \limp B) \oc \octx'_R |- \cseq}}{
      \oseq{\octx |- A} & \oseq{\octx'_L \oc B \oc \octx'_R |- \cseq}}
    \\
    \infer[\rrule{\pmir}]{\oseq{\octx |- B \pmir A}}{
      \oseq{\octx \oc A |- B}}
    \and
    \infer[\lrule{\pmir}]{\oseq{\octx'_L \oc (B \pmir A) \oc \octx \oc \octx'_R |- \cseq}}{
      \oseq{\octx |- A} & \oseq{\octx'_L \oc B \oc \octx'_R |- \cseq}}
    \\
    \infer[\rrule{\fuse}]{\oseq{\octx_A \oc \octx_B |- A \fuse B}}{
      \oseq{\octx_A |- A} & \oseq{\octx_B |- B}}
    \and
    \infer[\lrule{\fuse}]{\oseq{\octx'_L \oc (A \fuse B) \oc \octx'_R |- \cseq}}{
      \oseq{\octx'_L \oc A \oc B \oc \octx'_R |- \cseq}}
    \\
    \infer[\rrule{\one}]{\oseq{\octxe |- \one}}{}
    \and
    \infer[\lrule{\one}]{\oseq{\octx'_L \oc \one \oc \octx'_R |- \cseq}}{
      \oseq{\octx'_L \oc \octx'_R |- \cseq}}
    \\
    \infer[\rrule{\with}]{\oseq{\octx |- A \with B}}{
      \oseq{\octx |- A} & \oseq{\octx |- B}}
    \and
    \infer[\lrule{\with}_1]{\oseq{\octx'_L \oc (A \with B) \oc \octx'_R |- \cseq}}{
      \oseq{\octx'_L \oc A \oc \octx'_R |- \cseq}}
    \and
    \infer[\lrule{\with}_2]{\oseq{\octx'_L \oc (A \with B) \oc \octx'_R |- \cseq}}{
      \oseq{\octx'_L \oc B \oc \octx'_R |- \cseq}}
    \\
    \infer[\rrule{\top}]{\oseq{\octx |- \top}}{}
    \and
    \text{(no $\lrule{\top}$ rule)}
    \\
    \infer[\rrule{\plus}_1]{\oseq{\octx |- A \plus B}}{
      \oseq{\octx |- A}}
    \and
    \infer[\rrule{\plus}_2]{\oseq{\octx |- A \plus B}}{
      \oseq{\octx |- B}}
    \and
    \infer[\lrule{\plus}]{\oseq{\octx'_L \oc (A \plus B) \oc \octx'_R |- \cseq}}{
      \oseq{\octx'_L \oc A \oc \octx'_R |- \cseq} &
      \oseq{\octx'_L \oc B \oc \octx'_R |- \cseq}}
    \\
    \text{(no $\rrule{\zero}$ rule)}
    \and
    \infer[\lrule{\zero}]{\oseq{\octx'_L \oc \zero \oc \octx'_R |- \cseq}}{}
    \\
    \infer[\rrule{\bot}]{\oseq{\octx |- \bot}}{
      \oseq{\octx |- \cseqe}}
    \and
    \infer[\lrule{\bot}]{\oseq{\bot |- \cseqe}}{}
  \end{inferences}
\end{figure}  

\begin{theorem}[Admissibility of cut]
  If $\oseq{\octx |- A}$ and $\oseq{\octx'_L \oc A \oc \octx'_R |- \cseq}$, then $\oseq{\octx'_L \oc \octx \oc \octx'_R |- \cseq}$.
\end{theorem}
\begin{proof}
  By lexicographic induction, first on the structure of the cut formula and then simultaneously on the structures of the given derivations.
\end{proof}

\begin{theorem}[Identity expansion]
  $\oseq{A |- A}$ for all propositions $A$.
\end{theorem}
\begin{proof}
  By induction on the structure of the proposition $A$.
\end{proof}

\subsection{Circular propositions and circular derivations}

\begin{itemize}
\item No exponentials; recursion/circularity instead (Milner)\alertnote{Should this go in ordered rewriting chapter instead?}
\item $\mu$MALL (Baelde) and circular proofs (Fortier and Santocanale)
\item Contractivity requirement
\item We will use only general recursion.
  Inductive and coinductive types are outside our scope.
\item Subset of infinite propositions/derivations
\end{itemize}

%%% Local Variables:
%%% mode: latex
%%% TeX-master: "thesis"
%%% End:
