\chapter{Conclusion}\label{ch:conclusion}

In this document, we have explored two proof-theoretic characterizations of concurrency -- proof construction and proof reduction -- in the context of concurrent systems that have chain topologies.

On the proof-construction side, we have shown a new way of systematically deriving a rewriting framework from the ordered sequent calculus~\parencref{ch:ordered-rewriting}, and identified a message-passing fragment of that framework~\parencref{ch:choreographies}.
We have shown how to take string rewriting specifications of concurrent systems~\parencref{ch:string-rewriting} and choreograph them into message-passing ordered rewriting~\parencref{ch:choreographies}.

On the proof-reduction side, we have uncovered a semi-axiomatic sequent calculus for singleton logic, a logic that restricts sequents to have exactly one antecedent~\parencref{ch:singleton-logic}.
We have demonstrated that the semi-axiomatic nature of this calculus gives rise to a clean correspondence between proof normalization and asynchronous message-passing communication~\parencref{ch:process-chains}.

Lastly, we have shown that the asynchronous processes that arise from the semi-axiomatic sequent calculus for singleton logic can be faithfully embedded within the message-passing ordered rewriting framework~\parencref{ch:correspond}.
This has provided a relationship between the proof-construction and proof-reduction characterizations of concurrency.

This document now closes by discussing a few avenues for future work.

\section{Potential avenues for future work}\label{sec:conclusion:nondeterminism}

\subsection{From ordered rewriting to multiset rewriting, singleton processes to linear processes}

One obvious avenue for future work is to extend the ideas in this document to linear logic.
We conjecture that the string rewriting specifications of \cref{ch:string-rewriting} would be replaced with multiset rewriting specifications\autocite{Meseguer:TCS92}; the formula-as-process ordered rewriting of \cref{ch:formula-as-process}, with formula-as-process linear rewriting based on the focused linear sequent calculus\autocites{Miller:ELP92}{Cervesato+Scedrov:IC09}; and the singleton processes of \cref{ch:process-chains}, with \acs{SILL} processes\autocite{Caires+:MSCS16}.

% For the proof construction and rewriting perspective on concurrency, this would involve a shift from ordered rewriting to multiset rewriting, as understood logically in terms of the focused linear sequent calculus.

For instance, in the formula-as-process ordered rewriting choreography of the binary counter~\parencref{sec:formula-as-process:counters-oo}, we used the coinductively defined proposition $\defp{b}_0$ given by 
\begin{equation*}
  \defp{b}_0 \defd (\up \dn \defp{b}_1 \pmir \atmL{i}) \with (\up (\atmL{d} \fuse \dn \defp{b}'_0) \pmir \atmL{d})
  \,,
\end{equation*}
where here all of the shifts have been made explicit.
With a move to rewriting based on linear logic, rather than ordered logic, we would likely use the coinductively defined proposition $\defp{b}_0(x, y)$ given by
\begin{equation*}
  \defp{b}_0(x, y) \defd
    \begin{lgathered}[t]
      \bigl(\forall y'.\, i(y, y') \lolli \up \dn \defp{b}_1(x, y')\bigr) \\
      {} \with \bigl(\forall y'.\, d(y, y') \lolli \up \exists x'.\, d(x, x') \tensor \dn \defp{b}'_0(x', y')\bigr)
    \,,
    \end{lgathered}
\end{equation*}
together with similar defined linear propositions $\defp{e}(x, y)$, $\defp{b}_1(x, y)$, and $\defp{b}'_0(x, y)$.
Here the first-order parameters $x$ and $y$ thread together propositions in the context in a way that maintains the binary counter's essential structure.
In line with work by \textcite{Simmons+Pfenning:HOSC11}, $\defp{b}_0(x, y)$ is the \emph{destination-passing} embedding of $\defp{b}_0$.
In a formula-as-process interpretation, the destinations $x$ and $y$ can be viewed as channels that connect processes.

This example, however, does not fully exploit the expressive power of first-order linearity because the binary counter still has a chain topology.
Because \ac{SILL} processes admit tree topologies, defined propositions that use destinations in a tree topology should be allowed.
But a judgment for checking that destinations do not form cycles would likely be needed, for otherwise a destinations-as-channels interpretation would lead to ill-formed \ac{SILL} processes.

We would also need to characterize a general procedure for choreographing multiset rewriting specifications, in the vein of what was presented in \cref{ch:choreographies} for string rewriting specifications.


\subsection{First-order extension}

Another avenue for future work would be to extend the results contained in this document to first-order, not propositional, ordered rewriting and first-order polymorphic session-typed processes.
For example, we might embed sending and receiving processes by 
\begin{equation*}
  \begin{lgathered}[t]
    \trproc{a \shortleftarrow \mathsf{recvR}; P} = \dn \forall a.\, (\up \trproc{P} \pmir \atmL{\mathsf{tm}}(a)) \\
    \trproc{\mathsf{sendL}\:t} = \atmL{\mathsf{tm}}(t)
  \,,
  \end{lgathered}
\end{equation*}
and similarly for $\mathsf{sendR}$ and $\mathsf{recvL}$.
It would be nice if the extralogical $\atmL{\mathsf{tm}}$ and $\atmR{\mathsf{tm}}$ predicates could be done away with, but that appears to be impossible.
It seems that, even in an asynchronous calculus, the $\atmL{\mathsf{tm}}$ and $\atmR{\mathsf{tm}}$ atoms provide the small but necessary amount of synchronization to ensure that the intended term $t$ is used to instantiate the receiving proposition's universal quantifier.
However, it does seem plausible that terms could be packaged with the transmission of other labels, such as $\atmL{\kay}(t)$, which would hide  $\atmL{\mathsf{tm}}$ and $\atmR{\mathsf{tm}}$.

It would also be interesting to consider how session-typed processes with second-order polymorphism\autocite{Caires+:ESOP13} would be embedded in the rewriting framework.
A conjecture is that second-order polymorphism would be needed in the rewriting framework so that the embedding could be something like 
\begin{equation*}
  \begin{lgathered}[t]
    \trproc{\alpha \shortleftarrow \mathsf{recvR}; P} = \dn \forall \alpha.\, (\up \trproc{P} \pmir \atmL{\mathsf{pr}}(\alpha)) \\
    \trproc{\mathsf{sendL}\:P} = \atmL{\mathsf{pr}}(\trproc{P})
  \,.
  \end{lgathered}
\end{equation*}


\subsection{Session-typed nondeterministic choice}

In \cref{sec:correspond:types}, we leveraged the bisimilarity of process expressions and their embedding within formula-as-process ordered rewriting to reverse-engineer a session type system for ordered rewriting.
In particular, from the session-typing rules for $\caseL[\ell \in L]{\ell => P_{\ell}}$ and $\caseR[\ell \in L]{\ell => P_{\ell}}$, we arrived at rules for typing deterministic choices:
\begin{gather*}
  \infer[\lrule{\plus}]{\slof{\plus*[sub=_{\ell \in L}]{\ell:B_{\ell}} |- \dn {\textstyle \bigwith_{\ell \in L}(\atmR{\ell} \limp \up \p{A}_{\ell})} : C}}{
    \multipremise{\ell \in L}{\slof{B_{\ell} |- \p{A}_{\ell} : C}}}
  \shortintertext{and}
  \infer[\rrule{\with}]{\slof{A |- \dn {\textstyle \bigwith_{\ell \in L}(\up \p{A}_{\ell} \pmir \atmL{\ell})} : \with*[sub=_{\ell \in L}]{\ell:B_{\ell}}}}{
    \multipremise{\ell \in L}{\slof{A |- \p{A}_{\ell} : B_{\ell}}}}
\end{gather*}
Focusing in combination with the left- or right-handed implication ensures that the choices embodied by the alternative conjunction here are deterministic, not nondeterministic, choices.

However, in terms of the ordered propositions alone, it would seem more natural to have a typing rule for $\n{A}_1 \with \n{A}_2$ -- that is, a rule something like
\begin{equation*}
  \infer{\slof{A |- \n{A}_1 \with \n{A}_2 : C}}{\vdots}
  \,.
\end{equation*}
Finding such a rule for the ordered proposition $\n{A}_1 \with \n{A}_2$ might then allow us, by leveraging the ideas behind the bisimilar embedding of \cref{ch:correspond}, to reverse-engineer a process expression for some form of well-behaved, well-typed nondeterministic choice.

\Textfootcite{Stock:JUB20} has begun to look into incorporating nondeterministic choice into session-typed processes, emphasizing the operational considerations.
It would be interesting to consider whether his ideas can be adapted to the formula-as-process ordered rewriting framework.


\subsection{Induction, coinduction, termination, and productivity}

\Textfootcite{Derakhshan+Pfenning:LMCS20} have developed an infinitary calculus in which inductive and coinductive session types can be used to guarantee the termination and productivity of well-typed processes in a Curry--Howard interpretation of singleton logic.
Leveraging types, their validity condition on circular proofs is locally and effectively decidable.
\Textfootcite{Somayyajula+Pfenning:20} have done something similar with sized types.
An interesting question is whether their ideas might be applied to (formula-as-process) ordered rewriting.

Productivity is about observable progress and even untyped ordered rewriting has a notion of observation, as embodied in ordered bisimilarity~\parencref{ch:ordered-bisimilarity}.
So it seems likely that productive ordered rewriting systems can be characterized.

What is unclear is whether there is an locally and effectively decidable condition on ordered propositions that can guarantee productivity.
The local decidability of condition on circular proofs introduced by \citeauthor{Derakhshan+Pfenning:LMCS20}
% Can their validity condition on circular proofs be mapped to a condition on ordered propositions?
very much relies on the unrolling of inductive and coinductive types.
But ordered rewriting is untyped, at least natively, so whether productivity can be characterized in a locally decidable way is unclear.
% Yet productivity
% So it does not seem unreasonable to hope for some kind of validity condition in spite of the absence of types.
Also, unlike proofs, rewriting traces are constructed from open-ended derivations.
Does that open-endedness in any way affect the existence or shape of the productivity condition?

\subsection{Generative invariants and session types}

\Textfootcite{Simmons:CMU12} describes \emph{generative invariants} as a way to express invariants of ordered logical specifications.
These generative invariants generalize context-free grammars, as well as regular worlds from \textsc{lf}.
A generative invariant for a binary counter specification in his framework has similarities to the session type for binary counters given in \cref{sec:process-chains:binary-counter}:
\begin{align*}
  \mathsf{ctr} &\defd e \with (\mathsf{ctr} \fuse b_0) \with (\mathsf{ctr} \fuse b_1) \with (\mathsf{ctr} \fuse i) \\
  \mathsf{ctr}' &\defd z \with (\mathsf{ctr} \fuse s) \with (\mathsf{ctr} \fuse d) \with (\mathsf{ctr}' \fuse b'_0) \\[1ex]
  \ctr &\defd \with*{ i: \ctr , d: \plus*{ z: \ctre , s: \ctr } }
\end{align*}
It would be interesting to investigate whether these similarities can be extrapolated to a correspondence between generative invariants and session types.
If so, generative invariants might serve as a form of session typing for ordered rewriting that is more native than the system presented in \cref{sec:correspond:types}.

% In addition, the generative invariant for the binary counter looks a lot like the object-oriented choreography, but in an abstracted form:
% \begin{align*}
%   \mathsf{ctr} &\defd (\mathsf{ctr} \fuse i) \with \bigl((z \with (\mathsf{ctr} \fuse s)) \fuse d\bigr)
%   \\
%   \defp{e} &\defd (\defp{e} \fuse \defp{b}_1 \pmir \atmL{i}) \with (\atmR{z} \pmir \atmL{d}) \\
%   \defp{b}_0 &\defd (\up \dn \defp{b}_1 \pmir \atmL{i}) \with (\atmR{z} \pmir \atmL{d}) \\
% \end{align*}



%%% Local Variables:
%%% mode: latex
%%% TeX-master: "thesis"
%%% End:
