\chapter{Conclusion}\label{ch:conclusion}

\section{Avenues for future work}

\subsection{From ordered rewriting to multiset rewriting, singleton processes to linear processes}

One obvious avenue for future work is to extend the ideas in this document to linear logic.
We conjecture that the string rewriting specifications of \cref{??} would be replaced with multiset rewriting specifications\autocite{??}; the formula-as-process ordered rewriting of \cref{??}, with formula-as-process linear rewriting based on the focused linear sequent calculus\autocite{??}; and the singleton processes of \cref{??}, with \textsc{sill} processes\autocite{??}.

% For the proof construction and rewriting perspective on concurrency, this would involve a shift from ordered rewriting to multiset rewriting, as understood logically in terms of the focused linear sequent calculus.

For instance, in the formula-as-process ordered rewriting choreography of the binary counter~\parencref{??}, we used the coinductively defined proposition $\defp{b}_0$ given by 
\begin{equation*}
  \defp{b}_0 \defd (\up \dn \defp{b}_1 \pmir \atmL{i}) \with (\up (\atmL{d} \fuse \dn \defp{b}'_0) \pmir \atmL{d})
  \,,
\end{equation*}
where here all of the shifts have been made explicit.
With a move to rewriting based on linear logic, rather than ordered logic, we would likely use the coinductively defined proposition $\defp{b}_0(x, y)$ given by
\begin{equation*}
  \defp{b}_0(x, y) \defd
    \begin{lgathered}[t]
      \bigl(\forall y'.\, i(y, y') \lolli \up \dn \defp{b}_1(x, y')\bigr) \\
      {} \with \bigl(\forall y'.\, d(y, y') \lolli \up \exists x'.\, d(x, x') \tensor \dn \defp{b}'_0(x', y')\bigr)
    \,,
    \end{lgathered}
\end{equation*}
together with similar defined linear propositions $\defp{e}(x, y)$, $\defp{b}_1(x, y)$, and $\defp{b}'_0(x, y)$.
Here the first-order parameters $x$ and $y$ thread together propositions in the context in a way that maintains the binary counter's essential structure.
In line with work by \textcite{Simmons+Pfenning:??}, $\defp{b}_0(x, y)$ is the \emph{destination-passing} embedding of $\defp{b}_0$.
In a formula-as-process interpretation, the destinations $x$ and $y$ can be viewed as channels that connect processes.

This example, however, does not fully exploit the expressive power of first-order linearity because the binary counter still has a linear topology.
Because \textsc{sill} processes admit tree topologies, defined propositions that use destinations in a tree topology should be allowed.
But a judgment for checking that destinations do not form cycles would likely be needed, for otherwise a destinations-as-channels interpretation would lead to ill-formed \textsc{sill} processes.

We would also need to characterize a general procedure for choreographing multiset rewriting specifications.%
\fixnote{??}


\subsection{First-order extension}

Another avenue for future work would be to extend the results contained in this document to first-order, not propositional, ordered rewriting and first-order polymorphic session-typed processes.
For example, we might embed sending and receiving processes by 
\begin{equation*}
  \begin{lgathered}[t]
    \trproc{a \shortleftarrow \mathsf{recvR}; P} = \dn \forall a.\, (\up \trproc{P} \pmir \atmL{\mathsf{tm}}(a)) \\
    \trproc{\mathsf{sendL}\:t} = \atmL{\mathsf{tm}}(t)
  \,,
  \end{lgathered}
\end{equation*}
and similarly for $\mathsf{sendR}$ and $\mathsf{recvL}$.
It would be nice if the extralogical $\atmL{\mathsf{tm}}$ and $\atmR{\mathsf{tm}}$ predicates could be done away with, but that appears to be impossible.
It seems that, even in an asynchronous calculus, the $\atmL{\mathsf{tm}}$ and $\atmR{\mathsf{tm}}$ atoms provide the small but necessary amount of synchronization to ensure that the intended term $t$ is used to instantiate the receiving proposition's universal quantifier.
However, it does seem plausible that terms could be packaged with the transmission of other labels, such as $\atmL{\kay}(t)$, which would hide  $\atmL{\mathsf{tm}}$ and $\atmR{\mathsf{tm}}$.

It would also be interesting to consider how session-typed processes with second-order polymorphism\autocite{??} would be embedded in the rewriting framework.
A conjecture is that second-order polymorphism would be needed in the rewriting framework so that the embedding could be something like 
\begin{equation*}
  \begin{lgathered}[t]
    \trproc{\alpha \shortleftarrow \mathsf{recvR}; P} = \dn \forall \alpha.\, (\up \trproc{P} \pmir \atmL{\mathsf{proc}}(\alpha)) \\
    \trproc{\mathsf{sendL}\:P} = \atmL{\mathsf{proc}}(\trproc{P})
  \,.
  \end{lgathered}
\end{equation*}


\subsection{Session-typed nondeterministic choice}

In \cref{sec:??}, we leveraged the bisimilarity of process expressions and their embedding within formula-as-process ordered rewriting to reverse-engineer a session type system for ordered rewriting.
In particular, from the session-typing rules for $\caseL[\ell \in L]{\ell => P_{\ell}}$ and $\caseR[\ell \in L]{\ell => P_{\ell}}$, we arrived at rules for typing deterministic choices:
\begin{inferences}
  \infer[\lrule{\plus}]{\slof{\plus*[sub=_{\ell \in L}]{\ell:B_{\ell}} |- \dn {\textstyle \bigwith_{\ell \in L}(\atmR{\ell} \limp \up \p{A}_{\ell})} : C}}{
    \multipremise{\ell \in L}{\slof{B_{\ell} |- \p{A}_{\ell} : C}}}
  \\\text{and}\\
  \infer[\rrule{\with}]{\slof{A |- \dn {\textstyle \bigwith_{\ell \in L}(\up \p{A}_{\ell} \pmir \atmL{\ell})} : \with*[sub=_{\ell \in L}]{\ell:B_{\ell}}}}{
    \multipremise{\ell \in L}{\slof{A |- \p{A}_{\ell} : B_{\ell}}}}
\end{inferences}
Focusing in combination with the left- or right-handed implicaiton ensures that the choices embodied by the alternative conjunction here are determinstic, not nondeterministic, choices.

However, in terms of the ordered propositions alone, it would seem more natural to have a typing rule for $\n{A}_1 \with \n{A}_2$.
A rule something like
\begin{equation*}
  \infer{\slof{A |- \n{A}_1 \with \n{A}_2 : C}}{\vdots}
  \,.
\end{equation*}
Finding such a rule for the ordered proposition $\n{A}_1 \with \n{A}_2$ might then allow us, by leveraging the ideas behind the bisimilar embedding of \cref{ch:??}, to reverse-engineer a process expression for some form of well-behaved, well-typed nondeterministic choice.

\Textcite{??} has begun to look into incorporating nondeterministic choice into session-typed processes.
It would be interesting to consider how his ideas might carry over to the formula-as-process ordered rewriting framework.


\subsection{Induction and coinduction, termination and productivity}

\Textcite{??} have developed an infinitary calculus in which inductive and coinductive session types can be used to guarantee the termination and productivity of well-typed processes in a Curry--Howard interpretation of singleton logic.
An interesting question is how similar ideas might be applied to (formula-as-process) ordered rewriting.

Can their validity condition on circular proofs be mapped to a condition on ordered propositions?
That condition very much relies on the unfolding of inductive and coinductive types, but ordered rewriting is untyped, at least natively.
Yet productivity is about observable progress and even untyped ordered rewriting has a notion of observation, as embodied in ordered bisimilarity~\parencref{??}.
So it does not seem unreasonable to hope for some kind of validity condition in spite of the absence of types.

Also, unlike proofs, rewriting traces are inherently open-ended.
Does that open-endedness in any way affect the shape of the validity condition?

\section{Conclusion}


%%% Local Variables:
%%% mode: latex
%%% TeX-master: "thesis"
%%% End:
