\chapter{From ordered rewriting to message-passing concurrency}\label{ch:ordered-bisimilarity}

The previous \lcnamecref{ch:ordered-rewriting} introduced a process-as-formula view of the Lambek calculus, used to provide local, message-passing choreographies of the global, string rewriting specifications seen in \cref{ch:string-rewriting}.

With the notion of process identity uninterpreted atomic propositions as messages and 

This \lcnamecref{ch:ordered-bisimilarity} explores the question of when two propositions have equivalent behavior under this process-as-formula view.
In keeping with the large body of work on bisimilarity for message-passing processes\autocite{??}, we develop a notion of bisimilarity for ordered propositions.
This \vocab{ordered rewriting bisimilarity} treats the uninterpreted atomic propositions as the sole observables, in keeping with their interpretation as messages.
Messages are observable, but processes are opaque.


\section{}





\section{Ordered rewriting bisimilarity}

\paragraph*{Atoms are observable}

An ordered context $\octx$ may be composed when surrounded by ordered contexts $\octx_L$ and $\octx_R$.

Thus, in $\octx_L \oc \octx \oc \octx_R$, we view $\octx$ as existing within the environment formed by its surrounding contexts, $\octx_L$ and $\octx_R$.
The context $\octx$ then interacts with that environment along two interfaces: the left end of $\octx$ may interact with the right end of $\octx_L$, and, symmetrically, the right end of $\octx$ may interact with the left end of $\octx_R$.

An atom's location and direction are crucial to its observability.
For an atom to be observable, it must be possible for an external observer to receive that atom as a message.
In $\atmL{a} \oc \octx$ and $\octx \oc \atmR{b}$, the atoms $\atmL{a}$ and $\atmR{b}$, respectively, are observable, because $\octx_O \oc (\p{A} \pmir \atmL{a}) \oc \atmL{a} \oc \octx \reduces \octx_O \oc \p{A} \oc \octx$

But those same atoms are not observable in $\octx \oc \atmL{a}$ and $\atmR{b} \oc \octx$.

\begin{theorem}
  If $\octx = \atmL{a} \oc \octx_0 \reduces \octx'$, then $\octx' = \atmL{a} \oc \octx'_0$ for some $\octx'_0$ such that $\octx_0 \reduces \octx'_0$.
  Symmetrically, if $\octx = \octx_0 \oc \atmR{a} \reduces \octx'$, then $\octx' = \octx'_0 \oc \atmR{a}$ for some $\octx'_0$ such that $\octx_0 \reduces \octx'_0$.
\end{theorem}
\begin{proof}
  By inversion on the given reduction, making use of the fact that $\atmL{a} \limp \n{B}$ and $\n{B} \pmir \atmR{a}$ are not well-formed propositions.
\end{proof}

\paragraph*{Ordered rewriting is asynchronous}

Notice that $\octx_L \oc (\up \p{A} \pmir \atmL{a}) \oc (\atmL{a} \fuse \p{B}) \oc \octx_R \Reduces \octx_L \oc \p{A} \oc \p{B} \oc \octx_R$ in two steps, first decomposing $\atmL{a} \fuse \p{B}$ into $\atmL{a}$ and $\p{B}$, and then using that $\atmL{a}$ to decompose $\up \p{A} \pmir \atmL{a}$ into $\p{A}$.
But the rewriting cannot occur in a single, synchronous step:
\begin{equation*}
  \octx_L \oc (\up \p{A} \pmir \atmL{a}) \oc (\atmL{a} \fuse \p{B}) \oc \octx_R
    \nreduces \octx_L \oc \p{A} \oc \p{B} \oc \octx_R
  \,.
\end{equation*}

For this reason, ordered rewriting is asynchronous, and we should expect the notion of bisimilarity that we develop to be similar to bisimilarity developed for the asynchronous $\pi$-calculus\autocite{Amadio+:TCS98}.

\subsection{}

Because outgoing atoms are observable at a context's edges, there is a built-in notion of (immediate) output transition: a context $\octx$ outputs $\atmL{a}$ to its left exactly when $\octx = \atmL{a} \oc \octx'$, for some $\octx'$.
Symmetrically, a context $\octx$ outputs $\atmR{b}$ to its right exactly when $\octx = \octx' \oc \atmR{b}$.
We could adopt a process-calculus--like labeled transition notation for these output transitions -- such as $\octx = \atmL{a} \oc \octx' \reduces[\atmL{a}] \octx'$ and $\octx = \octx' \oc \atmR{a} \reduces[\atmR{a}] \octx'$ -- but that 

A weak output transition would then be
So, in this setting, $\octx$ would have a weak output transition to $\octx'$ if there exists a context $\octx_0$ such that $\octx \Reduces \atmL{a} \oc \octx_0$ and $\octx_0 \Reduces \octx'$ -- or, more simply, if $\octx \Reduces \atmL{a} \oc \octx'$.


\section{}

This \lcnamecref{ch:ordered-bisimilarity} marks a change in our perspective on ordered rewriting.
In the previous \lcnamecref{ch:ordered-rewriting}, we viewed ordered rewriting as an abstract framework for global specifications of concurrent systems, in the vein of previous work on [...].
The emphasis was placed squarely on state transformation [...].

Although useful for reasoning about abstract properties of concurrent systems, these global specifications do not immediately suggest [...].
Therefore, in this [...], we instead refine ordered rewriting into a framework for message-passing concurrency among processes with independent threads of control.

This message-passing view is obtained through a \vocab{process-as-formula}\autocite{??} reading of ordered propositions and contexts.
The logical connectives are reinterpreted as process constructors, so that propositions are seen as processes; positive atomic propositions, as messages; and contexts, as process configurations.

In this \lcnamecref{ch:ordered-bisimilarity}, we would instead like to decrease the level of abstraction and view ordered rewriting as a framework for message-passing concurrency.

 -- specifically, message-passing among processes arranged in a chain\fixnote{linear?} topology.
With their independent threads of control, processes bring a more local character to ordered rewriting, bringing it closer to a process calculus such as the $\pi$-calculus.

This message-passing view is obtained through a \vocab{process-as-formula}\autocite{??} view of ordered propositions and contexts.
The logical connectives are reinterpreted as process constructors, so that propositions are seen as processes; positive atomic propositions, as messages; and contexts, as process configurations.



This \lcnamecref{ch:ordered-bisimilarity} marks a change in our perspective on ordered rewriting.
In the previous \lcnamecref{ch:ordered-rewriting}, we viewed ordered rewriting as an abstract framework for concurrent state transformation, in the vein of previous work on multiset rewriting\autocite{??} [or even Petri nets\autocite{??}].
With the emphasis on transformation of the entire state, our view of concurrent computation was inherently global.

In this \lcnamecref{ch:ordered-bisimilarity}, we would instead like to view ordered rewriting as a framework for message-passing concurrency -- specifically, message-passing among processes arranged in a chain\fixnote{linear?} topology.
With their independent threads of control, processes bring a more local character to ordered rewriting, bringing it closer to a process calculus such as the $\pi$-calculus.

This message-passing view is obtained through a \vocab{process-as-formula}\autocite{??} view of ordered propositions and contexts.
The logical connectives are reinterpreted as process constructors, so that propositions are seen as processes; positive atomic propositions, as messages; and contexts, as process configurations.


% negative propositions can be seen as processes, positive atoms can be seen as messages, and ordered contexts can be seen as process configurations.

% Accordingly, we take a \vocab{process-as-formula}\autocite{??} view, in which [...].
% Specifically, uninterpreted positive atoms will act like messages and negative propositions will act like processes.
% Owing to the conceptual introduction of processes with independent threads of control, this view of concurrency is more local in character and brings ordered rewriting closer to a process calculus, such as the $\pi$-calculus.

Interestingly, this change in perspective necessitates very few formal changes to the ordered rewriting framework.
The primary change is that left- and right-handed implications are restricted to positive atoms, corresponding to the common first-order restriction that input processes receive only messages.

Despite the few formal changes, the new local [, message-passing] perspective does raise a new, important question: when do two processes have equivalent behavior?
In keeping with the large body of work on bisimilarity\autocite{??}, we develop a notion of bisimilarity between ordered contexts.
Several examples [...]

Despite requiring only very few formal changes, the shift from global to local perspective does raise an important question: when do two processes have equivalent behavior?
We answer this question by developing a notion of bisimilarity for ordered contexts.
In keeping with the large body of work on bisimilarity\autocite{??}, we develop a notion of bisimilarity between ordered contexts.

\begin{itemize}
\item Assign direction to uninterpreted atoms so they act like messagesl
\item Defined atoms are like processes
\item Left and right implications are restricted to messages (compare with higher-order $\pi$ calculus)
\end{itemize}


\section{}

\begin{equation*}
  a \oc \dfa{q} \reduces \dfa{q}'_a
\end{equation*}
where $q \dfareduces[a] q'_a$, for each pair $(q, a) \in Q \times \ialph$; and 
\begin{equation*}
  \emp \oc \dfa{q} \reduces
    \begin{cases*}
      \one & if $q \in F$ \\
      \top & if $q \notin F$
    \end{cases*}
\end{equation*}
for each $q \in Q$.

As a specification of \acp{DFA}, this works well.
But as an implementation, it is significantly lacking.
The rewriting axioms $a \oc \dfa{q} \reduces \dfa{q}'_a$ presume that a conductor orchestrates the interactions between input symbols and \ac{DFA} states, but a local\fixnote{distributed?} implementation 

This specification could be choreographed in (at least) two ways.
One choreography treats the input symbols $a$ as messages that are received by the states $\dfa{q}$, acting as processes.
\begin{equation*}
  \dfa{q} \defd (\emp \limp \dfa{F}(q)) \with \bigwith_{a \in \ialph}(\atmR{a} \limp \dfa{q}'_a)
\end{equation*}
Because the input word is delivered like data to the state, this choreography has a functional flavor.

Another choreography of the same specification is dual, treating the input symbols as processes
\begin{equation*}
  a \defd \bigwith_{q \in Q}(\atmL{q}'_a \pmir \atmL{q})
  \quad\text{and}\quad
  \emp \defd \bigwith_{q \in Q}(\dfa{F}(q) \pmir \atmL{q})
\end{equation*}

\section{}

To interpret polarized ordered propositions as processes, we adapt the \vocab{process-as-formula} view of logical connectives initiated by \textcite{??}.
The logical connectives are read as process constructors, so that positive atomic propositions may be seen as messages; negative propositions, [may be seen] as processes; ordered contexts, [may be seen] as process configurations with a chain topology; and positive propositions, [may be seen] as processes that reify those configurations.

To keep the interpretation as simple as possible, we introduce three syntactic restrictions on the ordered propositions.
Each of these restrictions may be relaxed at the expense of some additional complexity, as we will discuss in \cref{??}.

First, each positive atom is consistently assigned a direction, either left-directed, $\atmL{a}$, or right-directed, $\atmR{a}$.
When positive atoms are viewed as messages, these directions indicate the message's sender and intended recipient.
For example, in the context $\dn \n{C} \oc \atmL{a} \oc \p{B}$, the right-to-left direction of $\atmL{a}$ incicates that $\p{B}$ was the sender and $\dn \n{C}$ is the intended recipient.



Second, recursively defined \emph{positive} propositions are disallowed.\fixnote{is this necessary?}

Third, the left- and right-handed implications are restricted to accept only atoms with an incoming direction: $\atmR{a} \limp \n{B}$ and $\n{B} \pmir \atmL{a}$.
[In conjunction with atoms' directions,] this acts as a mild form of typing -- an input process may receive only intended messages.
Something like $\atmL{a} \oc (a \limp \up \p{B}) \reduces \p{B}$ should \emph{not} be possible, because its process-as-formula reading


\begin{syntax*}
  Ordered contexts &
    \octx & \octx_1 \oc \octx_2 \mid \octxe \mid \p{A}
\end{syntax*}
Concatenation of contexts, $\octx_1 \oc \octx_2$, is viewed as end-to-end composition of process configurations;
the empty context, $\octxe$, is the empty process configuration;
and [...].

To keep the interpretation as simple as possible, we introduce three syntactic restrictions on propositions.
Each of these restrictions may be relaxed at the expense of some additional complexity, as we will discuss in \cref{??}.

First, each positive atom is consistently assigned a direction, either left-directed, $\atmL{a}$, or right-directed, $\atmR{a}$.
Because 
Second, recursively defined \emph{positive} propositions are disallowed.
Thus, the positive propositions are generated by the following grammar.
\begin{syntax*}
  Positive props. &
    \p{A} & \atmL{a} \mid \atmR{a} \mid \p{A} \fuse \p{B} \mid \one \mid \dn \n{A}
\end{syntax*}
Atoms $\atmL{a}$ and $\atmR{a}$ are viewed as left- and right-directed messages; ordered conjunction, $\p{A} \fuse \p{B}$, denotes end-to-end composition of processes $\p{A}$ and $\p{B}$; $\one$ denotes the terminating 

\begin{tabular}{@{}rl@{}}
  $\octx_1 \oc \octx_2$ & end-to-end composition of configurations \\
  $\octxe$ & empty configuration \\
  $\p{A}$ & 
\end{tabular}

\begin{tabular}{@{}rl@{}}
  $\atmR{a}$ & right-directed message \\
  $\atmL{a}$ & left-directed message \\
  $\p{A} \fuse \p{B}$ & process composition \\
  $\one$ & terminating process \\
  $\dn \n{A}$ & 
\end{tabular}

\begin{tabular}{@{}rl@{}}
  $\n{\alpha} \defd \n{A}$ & recursively defined process \\
  $\atmR{a} \limp \n{B}$ & receive $\atmR{a}$ from the left, then continue as $\n{B}$ \\
  $\n{B} \pmir \atmL{a}$ & receive $\atmL{a}$ from the right, then continue as $\n{B}$ \\
  $\n{A} \with \n{B}$ & nondeterministically choose to continue as $\n{A}$ or $\n{B}$ \\
  $\top$ & \\
  $\up \p{A}$ &
\end{tabular}

\begin{tabular}{@{}rl@{}}
  $\octx_1 \oc \octx_2$ & composition of configurations $\octx_1$ and $\octx_2$ \\
  $\octxe$ & empty configuration \\
  $\p{A}$ & single process configuration
\end{tabular}

\begin{syntax*}
  Positive props. &
    \p{A} & \atmR{a} \mid \atmL{a} \mid \p{A} \fuse \p{B} \mid \one \mid \dn \n{A}
\end{syntax*}

Atoms' directions act as a very mild form of typing.
The left- and right-handed implications are restricted to accept only atoms with an incoming direction: $\atmR{a} \limp \n{B}$ and $\n{B} \pmir \atmL{a}$.
The full syntax of negative propositions is thus:
\begin{syntax*}
  Negative props. &
    \n{A} & \n{\alpha} \mid \atmR{a} \limp \n{B} \mid \n{B} \pmir \atmL{a} \mid \n{A} \with \n{B} \mid \top \mid \up \p{A}
  \,,
\end{syntax*}
with equirecursively defined negative propositions $\n{\alpha} \defd \n{A}$.


 acting like recursively defined processes.

In addition to fully general ordered contexts of positive propositions, it will also be useful to characterize two refinements: contexts that contain only atoms of one direction or the other.
We use an arrow decoration to indicate the direction.
\begin{syntax*}
  Ordered contexts &
    \octx & \octx_1 \oc \octx_2 \mid \octxe \mid \p{A}
  \\[-2\jot]
  Right-directed &
    \atmR{\octx} & \atmR{\octx}_1 \oc \atmR{\octx}_2 \mid \octxe \mid \atmR{a}
  \\[-2\jot]
  Left-directed &
    \atmL{\octx} & \atmL{\octx}_1 \oc \atmL{\octx}_2 \mid \octxe \mid \atmL{a}
\end{syntax*}

Having restricted the premises of left- and right-handed implications to incoming atoms, $\atmR{a}$ and $\atmL{a}$, respectively, the left focus judgment and its rules may be refined.
The judgment is now $\lfocus{\atmR{\octx}_L}{\n{A}}{\atmL{\octx}_R}{\p{C}}$, because [...inputs can only be incoming messages...].
Other than this refinement, the inference rules remain essentially the same as in \cref{??}.
The revised 


\begin{figure}
  \begin{syntax*}
    Positive props. &
      \p{A} & \atmR{a} \mid \atmL{a} \mid \p{A} \fuse \p{B} \mid \one \mid \dn \n{A}
    \\
    Negative props. &
      \n{A} & \n{\alpha} \mid
                \atmR{a} \limp \n{B} \mid \n{B} \pmir \atmL{a} \mid
                \n{A} \with \n{B} \mid \top \mid \up \p{A}
    \\
    Ordered contexts &
        \octx & \octx_1 \oc \octx_2 \mid \octxe \mid \p{A} \\[-2\jot]
      & \atmR{\octx} & \atmR{\octx}_1 \oc \atmR{\octx}_2 \mid \octxe \mid \atmR{a} \\[-2\jot]
      & \atmL{\octx} & \atmL{\octx}_1 \oc \atmL{\octx}_2 \mid \octxe \mid \atmL{a}
  \end{syntax*}
  \begin{inferences}[Rewriting: $\octx \reduces \octx'$ and $\octx \Reduces \octx'$]
    \infer[\jrule{$\dn$D}]{\atmR{\octx}_L \oc \dn \n{A} \oc \atmL{\octx}_R \reduces \p{C}}{
      \lfocus{\atmR{\octx}_L}{\n{A}}{\atmL{\octx}_R}{\p{C}}}
    \and
    \infer[\jrule{$\fuse$D}]{\p{A} \fuse \p{B} \reduces \p{A} \oc \p{B}}{}
    \and
    \infer[\jrule{$\one$D}]{\one \reduces \octxe}{}
    \\
    \text{(no $\jrule{$\plus$D}$ and $\jrule{$\zero$D}$ rules)}
    \\
    \infer[\jrule{$\reduces$C}_{\jrule{L}}]{\octx_1 \oc \octx_2 \reduces \octx'_1 \oc \octx_2}{
      \octx_1 \reduces \octx'_1}
    \and
    \infer[\jrule{$\reduces$C}_{\jrule{R}}]{\octx_1 \oc \octx_2 \reduces \octx'_1 \oc \octx_2}{
      \octx_1 \reduces \octx'_1}
  \end{inferences}
  \begin{inferences}
    \infer[\jrule{$\Reduces$R}]{\octx \Reduces \octx}{}
    \and
    \infer[\jrule{$\Reduces$T}]{\octx \Reduces \octx''}{
      \octx \reduces \octx' & \octx' \Reduces \octx''}
  \end{inferences}

  \begin{inferences}[Left focus: $\lfocus{\atmR{\octx}_L}{\n{A}}{\atmL{\octx}_R}{\p{C}}$]
    \infer[\lrule{\limp}']{\lfocus{\atmR{\octx}_L}{\atmR{a} \limp \n{B}}{\atmL{\octx}_R}{\p{C}}}{
      \lfocus{\atmR{\octx}_L \oc \atmR{a}}{\n{B}}{\atmL{\octx}_R}{\p{C}}}
    \and
    \infer[\lrule{\pmir}']{\lfocus{\atmR{\octx}_L}{\n{B} \pmir \atmL{a}}{\atmL{\octx}_R}{\p{C}}}{
      \lfocus{\atmR{\octx}_L}{\n{B}}{\atmL{a} \oc \atmL{\octx}_R}{\p{C}}}
    \\
    \infer[\lrule{\with}_1]{\lfocus{\atmR{\octx}_L}{\n{A} \with \n{B}}{\atmL{\octx}_R}{\p{C}}}{
      \lfocus{\atmR{\octx}_L}{\n{A}}{\atmL{\octx}_R}{\p{C}}}
    \and
    \infer[\lrule{\with}_2]{\lfocus{\atmR{\octx}_L}{\n{A} \with \n{B}}{\atmL{\octx}_R}{\p{C}}}{
      \lfocus{\atmR{\octx}_L}{\n{B}}{\atmL{\octx}_R}{\p{C}}}
    \and
    \text{(no $\lrule{\top}$ rule)}
    \\
    \infer[\lrule{\up}]{\lfocus{}{\up \p{A}}{}{\p{A}}}{}
  \end{inferences}
  \caption{A weakly focused ordered rewriting framework}
\end{figure}

\begin{figure}
 \begin{inferences}[Input transition: $\ireduces{\atmR{\octx}_L \oc ##1 \oc \atmL{\octx}_R}{\octx}{\octx'}$]
    \infer{\ireduces{\atmR{\octx}_L \oc #1 \oc \atmL{\octx}_R}{\dn \n{A}}{\p{C}}}{
      \lfocus{\atmR{\octx}_L}{\n{A}}{\atmL{\octx}_R}{\p{C}}}
    \and
    \infer{\ireduces{\atmR{\octx}_L \oc #1 \oc \atmL{\octx}_R}{\atmR{a} \oc \octx}{\octx'}}{
      \ireduces{\atmR{\octx}_L \oc \atmR{a} \oc #1 \oc \atmL{\octx}_R}{\octx}{\octx'}}
    \and
    \infer{\ireduces{\atmR{\octx}_L \oc #1 \oc \atmL{\octx}_R}{\octx \oc \atmL{a}}{\octx'}}{
      \ireduces{\atmR{\octx}_L \oc #1 \oc \atmL{a} \oc \atmL{\octx}_R}{\octx}{\octx'}}
    \\
    \infer{\ireduces{#1 \oc \atmL{\octx}_R}{\p{A} \oc \octx}{\p{A} \oc \octx'}}{
      \ireduces{#1 \oc \atmL{\octx}_R}{\octx}{\octx'}}
    \and
    \infer{\ireduces{\atmR{\octx}_L \oc #1}{\octx \oc \p{A}}{\octx' \oc \p{A}}}{
      \ireduces{\atmR{\octx}_L \oc #1}{\octx}{\octx'}}
  \end{inferences}
  \caption{A weakly focused ordered rewriting framework}
\end{figure}


\section{Input transitions}

With the above restriction of left- and right-handed implications to atomic premises of hte 

\begin{inferences}
  \infer{\ireduces{\atmR{\octx}_L \oc #1 \oc \atmL{\octx}_R}{\dn \n{A}}{\p{C}}}{
    \lfocus{\atmR{\octx}_L}{\n{A}}{\atmL{\octx}_R}{\p{C}}}
  \\
  \infer{\ireduces{\atmR{\octx}_L \oc #1 \oc \atmL{\octx}_R}{\atmR{a} \oc \octx}{\octx'}}{
    \ireduces{\atmR{\octx}_L \oc \atmR{a} \oc #1 \oc \atmL{\octx}_R}{\octx}{\octx'}}
  \and
  \infer{\ireduces{\atmR{\octx}_L \oc #1 \oc \atmL{\octx}_R}{\octx \oc \atmL{a}}{\octx'}}{
    \ireduces{\atmR{\octx}_L \oc #1 \oc \atmL{a} \oc \atmL{\octx}_R}{\octx}{\octx'}}
  \\
  \infer{\ireduces{#1 \oc \atmL{\octx}_R}{\p{A} \oc \octx}{\p{A} \oc \octx'}}{
    \ireduces{#1 \oc \atmL{\octx}_R}{\octx}{\octx'}}
  \and
  \infer{\ireduces{\atmR{\octx}_L \oc #1}{\octx \oc \p{A}}{\octx' \oc \p{A}}}{
    \ireduces{\atmR{\octx}_L \oc #1}{\octx}{\octx'}}
\end{inferences}

In its most basic form, an input transition derives from the inputs required by [...].

The following \lcnamecref{thm:input-transition-reduction} relates input transitions to reductions.
\begin{theorem}\label{thm:input-transition-reduction}
  If $\ireduces{\atmR{\octx}_L \oc #1 \oc \atmL{\octx}_R}{\octx}{\octx'}$, then $\atmR{\octx}_L \oc \octx \oc \atmL{\octx}_R \reduces \octx'$.
  Conversely, if $\octx \reduces \octx'$, then there exist $\octx_L$ and $\octx_R$ such that either:
  \begin{itemize}
  \item $\octx = \octx_L \oc \atmR{\lctx}_L \oc \octx_0 \oc \atmL{\lctx}_R \oc \octx_R$ and $\ireduces{\atmR{\lctx}_L \oc #1 \oc \atmL{\lctx}_R}{\octx_0}{\octx'_0}$ and $\octx' = \octx_L \oc \octx'_0 \oc \octx_R$, for some $\atmR{\lctx}_L$, $\octx_0$, $\atmL{\lctx}_R$, and $\octx'_0$;
  \item $\octx = \octx_L \oc (\p{A} \fuse \p{B}) \oc \octx_R$ and $\octx' = \octx_L \oc \p{A} \oc \p{B} \oc \octx_R$, for some $\p{A}$ and $\p{B}$; or
  \item $\octx = \octx_L \oc \one \oc \octx_R$ and $\octx' = \octx_L \oc \octx_R$.
  \end{itemize}
\end{theorem}
\begin{proof}
  By structural induction on the given input transition or reduction, respectively.
\end{proof}

\begin{lemma}\label{lem:input-framing}
  If $\ireduces{\atmR{\lctx}_L \oc #1 \oc \atmL{\lctx}_R}{\atmR{a} \oc \octx}{\octx'}$, then either:
  \begin{itemize}[nosep]
  \item $\atmR{a}$ satisfies an input demand -- \ie, $\ireduces{\atmR{\lctx}_L \oc \atmR{a} \oc #1 \oc \atmL{\lctx}_R}{\octx}{\octx'}$; or
  \item $\atmR{a}$ does not participate in the input transition -- \ie, $\atmR{\lctx}_L = \octxe$ and $\octx' = \atmR{a} \oc \octx'_a$ for some $\octx'_a$ such that $\ireduces{#1 \oc \atmL{\lctx}_R}{\octx}{\octx'_a}$.
  \end{itemize}
  Symmetrically, if $\ireduces{\atmR{\lctx}_L \oc #1 \oc \atmL{\lctx}_R}{\octx \oc \atmL{a}}{\octx'}$, then either:
  \begin{itemize}[nosep]
  \item $\atmL{a}$ satisfies an input demand -- \ie, $\ireduces{\atmR{\lctx}_L \oc #1 \oc \atmL{a} \oc \atmL{\lctx}_R}{\octx}{\octx'}$; or
  \item $\atmL{a}$ does not participate in the input transition -- \ie, $\atmL{\lctx}_R = \octxe$ and $\octx' = \octx'_a \oc \atmL{a}$ for some $\octx'_a$ such that $\ireduces{\atmR{\lctx}_L \oc #1}{\octx}{\octx'_a}$.
  \end{itemize}
\end{lemma}
\begin{proof}
  By structural induction on the given input transition.
\end{proof}

\section{Rewriting bisimilarity}

With the shift from a global, state transformation view of ordered rewriting to a local, \enquote{formula-as-process} view, it is now possible to consider how the individual \enquote{formula-as-process} -- or, more generally, \enquote{context-as-configuration} -- components behave and how they interact with each other.
Because each component has its own, local thread of control, we can describe its behavior only to the extent that its behavior is observable.
To the extent that its behavior can be witnessed by an external observer

Now that we can decompose concurrent systems into individual \enquote{formula-as-process} -- or \enquote{context-as-configuration} -- components, 

Intuitively, for example, the contexts\fixnote{configurations?} $\atmR{a} \oc (\atmR{a} \limp \atmR{b})$ and $\atmR{b}$ should be behaviorally equivalent: an internal reduction transforms $\atmR{a} \oc (\atmR{a} \limp \atmR{b})$ into $\atmR{b}$, and no other interactions -- reductions or input or output transitions -- are possible from $\atmR{a} \oc (\atmR{a} \limp \atmR{b})$.
As another example, $\atmR{a} \limp (\atmR{c} \pmir \atmL{b})$ and $(\atmR{a} \limp \atmR{c}) \pmir \atmL{b}$ should also be behaviorally equivalent, intuitively because they are logically equivalent.

Following the vast literature on various forms of bisimilarity\footnote{See \textcite{??} for a survey.}, we will develop a notion of \vocab{rewriting bisimilarity} on ordered contexts.
First, we need a few auxiliary definitions.

related to Deng et al.




\begin{definition}[Framed binary relations]
  Let $\simu{R}$ be a binary relation over ordered contexts.
  Given ordered contexts $\lctx_L$ and $\lctx_R$, let $\lrframe{\lctx_L}{\simu{R}}{\lctx_R}$ be the least binary relation such that:
  \begin{inferences}
    \infer{\lctx_L \oc \octx \oc \lctx_R \lrframe{\lctx_L}{\simu{R}}{\lctx_R} \lctx_L \oc \octx' \oc \lctx_R}{
      \octx \simu{R} \octx'}
  \end{inferences}

  Furthermore, let $\ctxc{\simu{R}}$ be the input contextual closure of $\simu{R}$ -- that is, $\octx \ctxc{\simu{R}} \lctx$ if and only if $\octx \lrframe{\atmR{\lctx}_L}{\simu{R}}{\atmL{\lctx}_R} \lctx$ for some $\atmR{\lctx}_L$ and $\atmL{\lctx}_R$.
  Equivalently, $\ctxc{\simu{R}}$ is the least binary relation such that:
  \begin{inferences}
    \infer{\octx \ctxc{\simu{R}} \lctx}{
      \octx \simu{R} \lctx}
    \and
    \infer{\atmR{a} \oc \octx \ctxc{\simu{R}} \atmR{a} \oc \lctx}{
      \octx \ctxc{\simu{R}} \lctx}
    \and
    \infer{\octx \oc \atmL{a} \ctxc{\simu{R}} \lctx \oc \atmL{a}}{
      \octx \ctxc{\simu{R}} \lctx}
  \end{inferences}
\end{definition}

Processes should be equivalent only if they have the same input/output behavior.
In this setting, there is a single type of observable behavior: output of outward-directed messeges.
and input of inward-directed messeges.
 

\begin{definition}
  A \vocab{rewriting bisimulation}, $\simu{R}$, is a symmetric binary relation among contexts that satisfies the following conditions.
  \begin{thmdescription}
  \item[Output bisimulation]
    If $\octx \simu{R}\Reduces \atmL{\lctx}'_L \oc \lctx' \oc \atmR{\lctx}'_R$, then $\octx \Reduces\lrframe{\atmL{\lctx}'_L}{\simu{R}}{\atmR{\lctx}'_R} \atmL{\lctx}'_L \oc \lctx' \oc \atmR{\lctx}'_R$.
  \item[Input bisimulation]
    If $\atmR{\lctx}_L \oc \octx \oc \atmL{\lctx}_R \lrframe{\atmR{\lctx}_L}{\simu{R}}{\atmL{\lctx}_R}\Reduces \lctx'$, then $\atmR{\lctx}_L \oc \octx \oc \atmL{\lctx}_R \Reduces\simu{R} \lctx'$.
  \end{thmdescription}
  \begin{marginfigure}
    \begin{center}
      \begin{tabular}{@{}c@{}}
        \begin{tikzcd}[sep=large]
          \octx
            \rar[relation, "\simu{R}"]
            \dar[Reduces, exists]
          &
          \lctx
            \dar[Reduces]
          \\
          \atmL{\lctx}'_L \oc \octx' \oc \atmR{\lctx}'_R
            \rar[relation, exists, "\lrframe{\atmL{\lctx}'_L}{\simu{R}}{\atmR{\lctx}'_R}"]
          &
          \atmL{\lctx}'_L \oc \lctx' \oc \atmR{\lctx}'_R
        \end{tikzcd}
        \\
        \emph{Output bisimulation}
        \\
        \begin{tikzcd}[sep=large]
          \atmR{\lctx}_L \oc \octx \oc \atmL{\lctx}_R
            \rar[relation, "\lrframe{\atmR{\lctx}_L}{\simu{R}}{\atmL{\lctx}_R}"]
            \dar[Reduces, exists]
          &
          \atmR{\lctx}_L \oc \lctx \oc \atmL{\lctx}_R
            \dar[Reduces]
          \\
          \octx'
            \rar[relation, exists, "\simu{R}"]
          &
          \lctx'
        \end{tikzcd}
        \\
        \emph{Input bisimulation}
      \end{tabular}
    \end{center}
    \caption{Rewriting bisimulation conditions, in diagrams}
  \end{marginfigure}
  \vocab{Rewriting bisimilarity}, $\osim$, is the largest rewriting bisimulation.
\end{definition}

These are very strong conditions -- arbitrary traces and quantify over all output/input contexts.

Notice that a third, reduction bisimulation property is a trivial instance of the output and input bisimulation conditions -- namely when the output and input contexts, $\atmL{\lctx}'_L$ and $\atmR{\lctx}'_R$ and $\atmR{\lctx}_L$ and $\atmL{\lctx}_R$, respectively, are empty:
\begin{theorem}\label{thm:bisim-reduction-closure}
  If $\simu{R}$ is a rewriting bisimulation, then $\simu{R}$ satisifies:
  \begin{thmdescription}
  \item[Reduction bisimulation]
    If $\octx \simu{R}\Reduces \lctx'$, then $\octx \Reduces\simu{R} \lctx'$.
  \end{thmdescription}
\end{theorem}

Because rewriting bisimilarity is defined coinductively, with very strong conditions, the \lcnamecref{??} itself is [...].
\begin{itemize}
\item The contexts $\atmL{a} \pmir \atmL{a}$ and $\octxe$ are \emph{not} bisimilar.
  Suppose, for the sake of contradiction, that they are bisimilar and so, framing $\atmL{b}$ onto the right, we have $(\atmL{a} \pmir \atmL{a}) \oc \atmL{b} \rframe{\osim}{\atmL{b}} \atmL{b}$.
  Composing the input and ouput bisimulation conditions, $(\atmL{a} \pmir \atmL{a}) \oc \atmL{b} \Reduces\lframe{\atmL{b}}{\osim} \atmL{b}$ must follow.
  However, this is impossible: $(\atmL{a} \pmir \atmL{a}) \oc \atmL{b}$ is irreducible and does not expose $\atmL{b}$ at its left end.
  Therefore, $\atmL{a} \pmir \atmL{a}$ and $\octxe$ \emph{cannot} be bisimilar.

\item The contexts $\atmR{a}$ and $\atmR{a} \with \atmR{b}$ are not bisimilar.
  The context $\atmR{a} \with \atmR{b}$ can output $\atmR{b}$ at its right end: $\atmR{a} \with \atmR{b} \reduces \atmR{b}$.
  But $\atmR{a}$ cannot simulate that output: the output bisimulation condition demands $\atmR{a} \Reduces\rframe{\osim}{\atmR{b}} \atmR{b}$, which is impossible.
\end{itemize}

Now we would like to confirm our earlier intuition about the equivalence of $\atmR{a} \oc (\atmR{a} \pmir \atmR{b})$ and $\atmR{b}$ by proving that $\atmR{a} \oc (\atmR{a} \pmir \atmR{b}) \osim \atmR{b}$.
Unfortunately, the definition of rewriting bisimilarity is not immediately suitable for establishing that two contexts are bisimilar.
The output and input bisimulation conditions are so strong [...].

For instance,
\begin{description}
\item[Input bisimulation]
  $\atmR{\lctx}_L \oc \atmR{a} \oc (\atmR{a} \limp \atmR{b}) \oc \atmL{\lctx}_R \Reduces \lctx'$ implies $\atmR{\lctx}_L \oc \atmR{b} \oc \atmL{\lctx}_R \osim \lctx'$; and % , for all $\atmR{\lctx}_L$ and $\atmL{\lctx}_R$; and
  $\atmR{\lctx}_L \oc \atmR{b} \oc \atmL{\lctx}_R \Reduces \lctx'$ implies $\atmR{\lctx}_L \oc \atmR{a} \oc (\atmR{a} \limp \atmR{b}) \oc \atmL{\lctx}_R \osim \lctx'$; and % , for all $\atmR{\lctx}_L$ and $\atmL{\lctx}_R$; and

\item[Output bisimulation]
  $\atmR{a} \oc (\atmR{a} \limp \atmR{b}) \Reduces \atmL{\lctx}'_L \oc \lctx' \oc \atmR{\lctx}'_R$ implies $\atmR{b} \Reduces\lrframe{\atmL{\lctx}'_L}{\osim}{\atmR{\lctx}'_R} \atmL{\lctx}'_L \oc \lctx' \oc \atmR{\lctx}'_R$; and
  $\atmR{b} \Reduces \atmL{\lctx}'_L \oc \lctx' \oc \atmR{\lctx}'_R$ implies $\atmR{a} \oc (\atmR{a} \limp \atmR{b}) \Reduces\lrframe{\atmL{\lctx}'_L}{\osim}{\atmR{\lctx}'_R} \atmL{\lctx}'_L \oc \lctx' \oc \atmR{\lctx}'_R$.
\end{description}
In this small example, it is possible to imagine tediously proving these statements -- after all, there are not that many traces involving $\atmR{a} \oc (\atmR{a} \limp \atmR{b})$.
However, in general, a proof technique for rewriting bisimilarity is needed.


Simple examples of bisimilar (or non-bisimilar) contexts
\begin{itemize}
\item $\atmL{a} \pmir \atmL{a} \nosim \octxe$ because input bisimulation followed by output bisimulation demands that $\atmR{b} \oc (\atmL{a} \pmir \atmL{a}) \Reduces\lrframe{}{\osim}{\atmR{b}} \atmR{b}$, which is impossible because $\atmR{b} \oc (\atmL{a} \pmir \atmL{a})$ has no nontrivial reductions and does not expose $\atmR{b}$ at its right.
\item $\atmR{a} \with \atmR{b} \nosim \atmR{a}$ because $\atmR{a} \Reduces\lrframe{}{\osim}{\atmR{b}} \atmR{b}$ is impossible.
\item {[$\atmR{a} \with \top \osim \atmR{a}$, but only because rewriting is (weakly) focused.]}
\item $\atmR{a} \oc (\atmR{a} \limp \atmR{b}) \osim \atmR{b}$ intuitively because $\atmR{a} \oc (\atmR{a} \limp \atmR{b})$ has no input transitions and reduces to $\atmR{b}$.
  Need a proof technique to establish this.
\item $\atmR{a} \limp (\atmR{c} \pmir \atmL{b}) \osim (\atmR{a} \limp \atmR{c}) \pmir \atmL{b}$ intuitively because the two propositions are logically equivalent.
  Both have the same input transitions.
  Also, $\atmR{a} \limp \up \dn (\atmR{c} \pmir \atmL{b}) \osim \up \dn (\atmR{a} \limp \atmR{c}) \pmir \atmL{b}$.
\end{itemize}


\subsection{Labeled bisimilarity: A proof technique for rewriting bisimilarity}

In the $\pi$-calculus, bisimilarity is similarly too strong to be used directly in proving the equivalence of processes.
There, a sound proof technique for bisimilarity is built around a labeled transition system and a notion of labeled bisimulation.
Because the labeled transition system is image-finite, proving that two processes are labeled bisimilar is more tractable than directly proving them [to be] bisimilar.

In this \lcnamecref{??}, we follow that strategy and develop \vocab{labeled bisimilarity} as a sound and complete proof technique for rewriting bisimilariy.
Like the $\pi$-calculus analogues, labeled bisimilarity is more tractable than rewriting bisimilarity because it uses labeled input transitions in place of [full] rewriting sequences.

\begin{definition}
  A \vocab{labeled bisimulation}, $\simu{R}$, is a symmetric binary relation [among contexts] that satisfies the following conditions.
  \begin{thmdescription}
  \item[Immediate output bisimulation]
    If $\octx \simu{R} \lctx = \atmL{\lctx}'_L \oc \lctx' \oc \atmR{\lctx}'_R$, then $\octx \Reduces\lrframe{\atmL{\lctx}'_L}{\simu{R}}{\atmR{\lctx}'_R} \lctx$.
  \item[Immediate input bisimulation]
    If $\octx \simu{R} \lctx$ and $\ireduces{\atmR{\lctx}_L \oc #1 \oc \atmL{\lctx}_R}{\lctx}{\lctx'}$, then $\atmR{\lctx}_L \oc \octx \oc \atmL{\lctx}_R \Reduces\simu{R} \lctx'$.
  \item[Reduction bisimulation]
    If $\octx \simu{R}\reduces \lctx'$, then $\octx \Reduces\simu{S} \lctx'$.
  \item[Emptiness bisimulation]
    If $\octx \simu{R} \octxe$, then:
    \begin{itemize*}[label=, afterlabel=]
    \item $\atmR{\lctx} \oc \octx \Reduces\rframe{\simu{R}}{\atmR{\lctx}} \atmR{\lctx}$ for all $\atmR{\lctx}$; and
    \item $\octx \oc \atmL{\lctx} \Reduces\lframe{\atmL{\lctx}}{\simu{R}} \atmL{\lctx}$ for all $\atmL{\lctx}$.
    \end{itemize*}
  \end{thmdescription}
  \vocab{Labeled bisimilarity} is the largest labeled bisimulation.
  \begin{marginfigure}
    \begin{center}
      \begin{tabular}{@{}c@{}}
        \begin{tikzcd}[sep=large]
          \octx
            \rar[relation, "\simu{R}"]
            \dar[Reduces, exists]
          &
          \lctx \mathrlap{{} = \atmL{\lctx}'_L \oc \lctx' \oc \atmR{\lctx}'_R}
          \\
          \atmL{\lctx}'_L \oc \octx' \oc \atmR{\lctx}'_R
            \urar[relation, exists, "\lrframe{\atmL{\lctx}'_L}{\simu{S}}{\atmR{\lctx}'_R}" {sloped, below}]
        \end{tikzcd}%
        \phantom{${} = \atmL{\lctx}'_L \oc \lctx' \oc \atmR{\lctx}'_R$}
        \\
        \emph{Immediate output bisimulation}
        \\
        \begin{tikzcd}[sep=large]
          \atmR{\lctx}_L \oc \octx \oc \atmL{\lctx}_R
            \rar[relation, "\lrframe{\atmR{\lctx}_L}{\simu{R}}{\atmL{\lctx}_R}"]
            \dar[Reduces, exists]
          &
          \atmR{\lctx}_L \oc \lctx \oc \atmL{\lctx}_R
          &[-2.5em]
          \atmR{\lctx}_L \oc [\lctx] \oc \atmL{\lctx}_R
            \dar[reduces]
          \\
          \octx'
            \arrow[relation, exists, "\simu{S}"]{rr}
          &&
          \lctx'
        \end{tikzcd}
        \\
        \emph{Immediate input bisimulation}
        \\
        \begin{tikzcd}[sep=large]
          \octx
            \rar[relation, "\simu{R}"]
            \dar[Reduces, exists]
          &
          \lctx
            \dar[reduces]
          \\
          \octx'
            \rar[relation, exists, "\simu{S}"]
          &
          \lctx'
        \end{tikzcd}
        \\
        \emph{Reduction bisimulation}
        \\
        \begin{tikzcd}[sep=large]
          \atmR{\lctx} \oc \octx
            \rar[relation, "\lrframe{\atmR{\lctx}}{\simu{R}}{}"]
            \dar[Reduces, exists]
          &
          \atmR{\lctx}
          \\
          \octx' \oc \atmR{\lctx}
            \urar[relation, exists, "\lrframe{}{\simu{S}}{\atmR{\lctx}}" {sloped, below}]
        \end{tikzcd}
        \qquad
        \begin{tikzcd}[sep=large]
          \octx \oc \atmL{\lctx}
            \rar[relation, "\lrframe{}{\simu{R}}{\atmL{\lctx}}"]
            \dar[Reduces, exists]
          &
          \atmL{\lctx}
          \\
          \atmL{\lctx} \oc \octx'
            \urar[relation, exists, "\lrframe{\atmL{\lctx}}{\simu{S}}{}" {sloped, below}]
        \end{tikzcd}
        \\
        \emph{Emptiness bisimulation}
      \end{tabular}
    \end{center}
    \caption{Labeled bisimulation conditions, in diagrams}
  \end{marginfigure}
\end{definition}



The emptiness bisimulation condition is necessary to [...].
Notice that it is equivalent to $\octx \simu{R} \octxe$ implies $\octx \Reduces \octxe$.
A similar condition appears in \textcite{Deng+:LINEARITY12}.


\begin{theorem}[Completeness of labeled bisimilarity]
  Every rewriting bisimulation is also a labeled bisimulation, and labeled bisimilarity consequently contains rewriting bisimilarity.
\end{theorem}
\begin{proof}
  Let $\simu{R}$ be a rewriting bisimulation.
  The immediate output, immediate input, and reduction conditions are trivial instances of the output and input bisimulation conditions.
  For instance, to prove that $\simu{R}$ is an immediate input bisimulation, assume that $\octx \simu{R} \lctx$ and $\ireduces{\atmR{\lctx}_L \oc #1 \oc \atmL{\lctx}_R}{\lctx}{\lctx'}$; then $\atmR{\lctx}_L \oc \octx \oc \atmL{\lctx}_R \lrframe{\atmR{\lctx}_L}{\simu{R}}{\atmL{\lctx}_R} \reduces \lctx'$.
  Because $\simu{R}$ is a rewriting bisimulation, it follows from the input bisimulation property that $\atmR{\lctx}_L \oc \octx \oc \atmL{\lctx}_R \Reduces\simu{R} \lctx'$.

  The emptiness bisimulation condition follows from the composition of the input bisimulation property with the output bisimulation property.%
  \begin{marginfigure}
    \begin{center}
      \begin{tabular}{@{}c@{\quad}c@{}}
        \begin{tikzcd}
          \atmR{\lctx} \oc \octx
            \rar[relation, "\lrframe{\atmR{\lctx}}{\simu{R}}{}"]
            \dar[Reduces]
          &
          \atmR{\lctx}
            \arrow[Reduces, loop right]{}
          \\
          \octx'
            \urar[relation, "\simu{R}" sloped]
            \dar[Reduces]
          \\
          \octx'' \oc \atmR{\lctx}
            \arrow[relation, "\lrframe{}{\simu{R}}{\atmR{\lctx}}" {sloped, below}]{uur}
        \end{tikzcd}
        &
        \begin{tikzcd}
          \octx \oc \atmL{\lctx}
            \rar[relation, "\lrframe{}{\simu{R}}{\atmL{\lctx}}"]
            \dar[Reduces]
          &
          \atmL{\lctx}
            \arrow[Reduces, loop right]{}
          \\
          \octx'
            \urar[relation, "\simu{R}" sloped]
            \dar[Reduces]
          \\
          \atmL{\lctx} \oc \octx''
            \arrow[relation, "\lrframe{\atmL{\lctx}}{\simu{R}}{}" {sloped, below}]{uur}
        \end{tikzcd}
      \end{tabular}
    \end{center}
    \caption{Emptiness bisimulation property as a consequence of input and output bisimulation properties}
  \end{marginfigure}
\end{proof}



Unfortunately, the direct converse is not true: a labeled bisimulation is not necessarily itself a rewriting bisimulation.
For example, consider the least symmetric binary relation $\simu{R}$ such that $\atmR{a} \limp (\atmR{c} \pmir \atmL{b}) \simu{R} (\atmR{a} \limp \atmR{c}) \pmir \atmL{b}$ and $\atmR{c} \simu{R} \atmR{c}$ and $(\octxe) \simu{R} (\octxe)$.
The relation $\simu{R}$ is a labeled bisimulation, but it does not satisfy the input bisimulation condition, because $\atmR{a} \oc (\atmR{a} \limp (\atmR{c} \pmir \atmL{b})) \lframe{\atmR{a}}{\simu{R}} \atmR{a} \oc ((\atmR{a} \limp \atmR{c}) \pmir \atmL{b})$ does not imply $\atmR{a} \oc (\atmR{a} \limp (\atmR{c} \pmir \atmL{b})) \Reduces\simu{R} \atmR{a} \oc ((\atmR{a} \limp \atmR{c}) \pmir \atmR{b})$, and so is not a rewriting bisimulation.

However, a slightly weaker statement \emph{is} true: a labeled bisimulation is contained within \emph{some} rewriting bisimulation.
Specifically, if $\simu{R}$ is a labeled bisimulation, then its input contextual closure, $\ctxc{\simu{R}}$, is such a rewriting bisimulation.
Fortunately, this will be enough to prove that labeled bisimilarity is sound.



\begin{definition}
  A symmetric binary relation $\simu{R}$ \vocab{(labeled-)progresses} to binary relation $\simu{S}$ if the two relations satisfy the following conditions.
  \begin{thmdescription}
  \item[Immediate output bisimulation]
    If $\octx \simu{R} \lctx = \atmL{\lctx}'_L \oc \lctx' \oc \atmR{\lctx}'_R$, then $\octx \Reduces\lrframe{\atmL{\lctx}'_L}{\simu{S}}{\atmR{\lctx}'_R} \lctx$.
  \item[Immediate input bisimulation]
    If $\octx \simu{R} \lctx$ and $\ireduces{\atmR{\lctx}_L \oc #1 \oc \atmL{\lctx}_R}{\lctx}{\lctx'}$, then $\atmR{\lctx}_L \oc \octx \oc \atmL{\lctx}_R \Reduces\simu{S} \lctx'$.
  \item[Reduction bisimulation]
    If $\octx \simu{R}\reduces \lctx'$, then $\octx \Reduces\simu{S} \lctx'$.
  \item[Emptiness bisimulation]
    If $\octx \simu{R} \octxe$, then:
    \begin{itemize*}[label=, afterlabel=]
    \item $\atmR{\lctx} \oc \octx \Reduces\lrframe{}{\simu{S}}{\atmR{\lctx}} \atmR{\lctx}$ for all $\atmR{\lctx}$; and
    \item $\octx \oc \atmL{\lctx} \Reduces\lrframe{\atmL{\lctx}}{\simu{S}}{} \atmL{\lctx}$ for all $\atmL{\lctx}$.
    \end{itemize*}
  \end{thmdescription}
\end{definition}
%
\noindent
Notice that the labeled bisimulations are exactly those relations that progress to themselves.



% \begin{lemma}
%   If $\simu{R}$ is a labeled bisimulation, then $\ctxc{\simu{R}}$ satisfies the following properties.
%   \begin{thmdescription}
%   \item[Immediate output bisimulation]
%     If $\octx \ctxc{\simu{R}} \lctx = \atmL{\lctx}'_L \oc \lctx' \oc \atmR{\lctx}'_R$, then $\octx \Reduces\lrframe{\atmL{\lctx}'_L}{\ctxc{\simu{R}}}{\atmR{\lctx}'_R} \lctx$.
%   \item[Reduction bisimulation]
%     If $\octx \ctxc{\simu{R}}\reduces \lctx'$, then $\octx \Reduces\ctxc{\simu{R}} \lctx'$.
%   \end{thmdescription}
% \end{lemma}
% \begin{proof}
%   The two properties are established separately. 
%   \begin{description}
%   \item[Immediate output bisimulation]
%     Assume that $\octx \ctxc{\simu{R}} \lctx = \atmL{\lctx}'_L \oc \lctx' \oc \atmR{\lctx}'_R$.
%     There are four cases, according to whether $\atmL{\lctx}'_L$ and $\atmR{\lctx}'_R$ are empty.
%     \begin{itemize}[parsep=0pt, listparindent=\parindent]
%     \item Consider the case in which both $\atmL{\lctx}'_L$ and $\atmR{\lctx}'_R$ are empty; in this case, we must show that $\octx \Reduces\ctxc{\simu{R}} \lctx$.
%       Using a trivial trace, that follows directly from the assumption $\octx \ctxc{\simu{R}} \lctx$.

%     \item Consider the case in which both $\atmL{\lctx}'_L$ and $\atmR{\lctx}'_R$ are nonempty.
%       The context $\lctx$ therefore exposes output atoms\fixnote{messages?} at its left and right ends.
%       And so the $\ctxc{\simu{R}}$-related contexts $\octx$ and $\lctx$ must, in fact, be $\simu{R}$-related, for otherwise at least one end of $\lctx$ would expose an input, not output, atom.
%       In other words, $\octx \simu{R} \lctx = \atmL{\lctx}'_L \oc \lctx' \oc \atmR{\lctx}'_R$.
%       Because $\simu{R}$ is a labeled bisimulation and satisfies the immediate output bisimulation property, it follows that $\octx \Reduces\lrframe{\atmL{\lctx}'_L}{\simu{R}}{\atmR{\lctx}'_R} \lctx$.
%       Since $\ctxc{\simu{R}}$ trivially contains $\simu{R}$, we conclude that $\octx \Reduces\lrframe{\atmL{\lctx}'_L}{\ctxc{\simu{R}}}{\atmR{\lctx}'_R} \lctx$, as required.

%     \item Consider the case in which $\atmL{\lctx}'_L$ is nonempty and $\atmR{\lctx}'_R$ is empty; in this case, we must show that $\octx \Reduces\lrframe{\atmL{\lctx}'_L}{\ctxc{\simu{R}}}{} \lctx$.
%       Similar to the previous case, $\lctx$ exposes output atoms at its left end because $\atmL{\lctx}'_L$ is nonempty.
%       And so the $\ctxc{\simu{R}}$-related contexts $\octx$ and $\lctx$ must, in fact, be $\lrframe{}{\simu{R}}{\atmL{\octx}_R}$-related, for some $\atmL{\octx}_R$, for otherwise the left end of $\lctx$ would expose an input, not output, atom.
%       In other words, $\octx \lrframe{}{\simu{R}}{\atmL{\octx}_R} \lctx = \atmL{\lctx}'_L \oc \lctx'$.

%       There are two subcases, according to how far the left edge of $\atmL{\octx}_R$ extends into $\lctx$.
%       \begin{itemize}
%       \item Suppose that the left edge of $\atmL{\octx}_R$ does not extend into $\atmL{\lctx}'_L$.
%         Because $\simu{R}$ is a labeled bisimulation, we may appeal to the immediate output bisimulation property after framing off $\atmL{\octx}_R$ -- we deduce $\octx \mathrel{\bigl((\Reduces\lrframe{\atmL{\lctx}'_L}{\simu{R}}{})\atmL{\octx}_R\bigr)} \lctx$.
%         Reduction is closed under framing, so we conclude $\octx \Reduces\lrframe{\atmL{\lctx}'_L}{\ctxc{\simu{R}}}{} \lctx$.
%       \item Otherwise, suppose that the left edge of $\atmL{\octx}_R$ does indeed extend into $\atmL{\lctx}'_L$.
%         In this case, there exist contexts $\atmL{\lctx}''_L$ and $\atmL{\octx}'_L$ such that $\atmL{\lctx}'_L = \atmL{\lctx}''_L \oc \atmL{\octx}'_L$ and $\atmL{\octx}_R = \atmL{\octx}'_L \oc \lctx'$.
%         Because $\simu{R}$ is a labeled bisimulation, we will compose the immediate output and emptiness bisimulation properties to establish $\octx \Reduces\lrframe{\atmL{\lctx}'_L}{\ctxc{\simu{R}}}{} \lctx$.

%         After framing off $\atmL{\octx}_R$, we may appeal to the immediate output bisimulation property and deduce $\octx \mathrel{\bigl((\Reduces\lrframe{\atmL{\lctx}''_L}{\simu{R}}{})\atmL{\octx}_R\bigr)} \lctx = \atmL{\lctx}''_L \oc \atmL{\octx}_R$.
%         Reduction is closed under framing, so $\octx \Reduces\lrframe{\atmL{\lctx}''_L}{\simu{R}}{\atmL{\octx}_R} \lctx$.
%         Upon respelling $\atmL{\octx}_R$ as $\atmL{\octx}'_L \oc \lctx'$ and framing off $\atmL{\lctx}''_L$ and $\lctx'$, we may appeal to the emptiness bisimulation property and deduce $\octx \Reduces\mathrel{\bigl(\atmL{\lctx}''_L\mathord{(\Reduces\lrframe{\atmL{\octx}'_L}{\simu{R}}{})}\lctx'\bigr)} \lctx = \atmL{\lctx}''_L \oc \atmL{\octx}'_L \oc \lctx'$.
%         Once again, reduction is closed under framing and we may respell $\atmL{\lctx}''_L \oc \atmL{\octx}'_L$ as $\atmL{\lctx}'_L$, so $\octx \Reduces\lrframe{\atmL{\lctx}'_L}{\simu{R}}{\lctx'} \lctx$.
%         Finally, because $\lctx'$ is an input context, we may conclude that $\octx \Reduces\lrframe{\atmL{\lctx}'_L}{\ctxc{\simu{R}}}{} \lctx$.
%       \end{itemize}

%     \item The case in which $\atmR{\lctx}'_R$ is nonempty and $\atmL{\lctx}'_L$ is empty is symmetric to the previous case.
%     \end{itemize}

%   \item[Reduction bisimulation]
%     Assume that $\octx \ctxc{\simu{R}}\reduces \lctx'$.
%     There are two cases: either the reduction arises from the $\simu{R}$-related component alone, or it arises from an input transition of the $\simu{R}$-related component that has its input demands met by the framing environment.
%     \begin{itemize}
%     \item Consider the case in which the reduction arises from the $\simu{R}$-related component alone -- that is, the case in which $\octx \ctxc{\simu{R}\reduces} \lctx'$.
%       Because $\simu{R}$ is a labeled bisimulation and therefore satisfies reduction bisimulation, it follows that $\octx \ctxc{\Reduces\simu{R}} \lctx'$.
%       Reduction is closed under framing, so $\octx \Reduces\ctxc{\simu{R}} \lctx'$.
%     \item Consider the case in which the reduction arises from an input transition of the $\simu{R}$-related component that has its input demands met by the framing environment -- that is, the case in which $\octx \ctxc{\lrframe{\atmR{\lctx}_L}{\simu{R}}{\atmL{\lctx}_R}\mathrel{(\prescript{\atmR{\lctx}_L}{}{\simu{I}}^{\atmL{\lctx}_R})}} \lctx'$ for some $\atmR{\lctx}_L$ and $\atmL{\lctx}_R$, where $\atmR{\lctx}_L \oc \lctx_0 \oc \atmL{\lctx}_R \mathrel{(\prescript{\atmR{\lctx}_L}{}{\simu{I}}^{\atmL{\lctx}_R})} \lctx'_0$ if $\ireduces{\atmR{\lctx}_L \oc #1 \oc \atmL{\lctx}_R}{\lctx_0}{\lctx'_0}$.
%       Because $\simu{R}$ is a labeled bisimulation and therefore satisfies immediate input bisimulation, it follows that $\octx \ctxc{\Reduces\simu{R}} \lctx'$.
%       Once again, reduction is closed under framing, so $\octx \Reduces\ctxc{\simu{R}} \lctx'$.
%     \end{itemize}
%     Note that it is impossible for the reduction to arise from the framing environment alone, because $\ctxc{\simu{R}}$ surrounds the $\simu{R}$-related components with only input messages, which are passive.
%   \qedhere
%   \end{description}
% \end{proof}

% \begin{theorem}
%   If $\simu{R}$ is a labeled bisimulation, then $\ctxc{\simu{R}}$ is a rewriting bisimulation.
% \end{theorem}
% \begin{proof}
%   To prove that $\ctxc{\simu{R}}$ is a rewriting bisimulation whenever $\simu{R}$ is a labeled bisimulation, we shall establish output and input bisimulation properties for $\ctxc{\simu{R}}$.
%   \begin{description}
%   \item[Output bisimulation]
%     follows by composing the reduction bisimulation and immediate output bisimulation properties of $\ctxc{\simu{R}}$, as proved in \cref{??}.%
%     \fixnote{Diagram?}
%   \item[Input bisimulation]
%     for $\ctxc{\simu{R}}$ is simply an instance of its reduction bisimulation property, as proved in \cref{??}, because $\ctxc{\simu{R}}$ is closed under framing of input message contexts.%
%     \fixnote{Diagram?}
%   \qedhere
%   \end{description}
% \end{proof}

% \begin{corollary}
%   Rewriting bisimilarity contains labeled bisimilarity.
% \end{corollary}


% \begin{theorem}
%   Given a binary relation $\simu{R}$, let $\ctxc{\simu{R}}$ be the \emph{input context[ual] closure} of $\simu{R}$ -- the least relation such that $\lctx \ctxc{\simu{R}} \octx$ if $\lctx \lrframe{\atmR{\lctx}_L}{\simu{R}}{\atmL{\lctx}_R} \octx$ for some $\atmR{\lctx}_L$ and $\atmL{\lctx}_R$.
%   If $\simu{R}$ is a labeled bisimulation, then $\ctxc{\simu{R}}$ is a rewriting bisimulation.
% \end{theorem}
% \begin{proof}
%   \begin{itemize}
%   \item
%     Suppose that $\octx \simu{R}^{-1}\reduces\Reduces \atmL{\lctx}'_L \oc \lctx' \oc \atmR{\lctx}'_R$.
%     $\octx \Reduces\simu{R}^{-1}\Reduces \atmL{\lctx}'_L \oc \lctx' \oc \atmR{\lctx}'_R$
%     $\octx \Reduces\lrframe{\atmL{\lctx}'_L}{\simu{R}}{\atmR{\lctx}'_R}^{-1} \atmL{\lctx}'_L \oc \lctx' \oc \atmR{\lctx}'_R$
%   \item
%     Suppose that $\atmR{\lctx}_L \oc \octx \oc \atmL{\lctx}_R \lrframe{\atmR{\lctx}_L}{\simu{R}}{\atmL{\lctx}_R}^{-1} \lctx'$.
%     So $\octx \simu{R}^{-1} \lctx'_0$ and $\lctx' = \atmR{\lctx}_L \oc \lctx'_0 \oc \atmL{\lctx}_R$.
%   \end{itemize}
% \end{proof}


% \begin{theorem}
%   A symmetric relation $\simu{R}$ is contained in bisimilarity if it satisfies the following conditions.
%   \begin{thmdescription}
%   \item[Immediate output]
%     If $\octx \simu{R}^{-1} \lctx = \atmL{\lctx}'_L \oc \lctx' \oc \atmR{\lctx}'_R$, then $\octx \Reduces\lrframe{\atmL{\lctx}'_L}{\simu{R}}{\atmR{\lctx}'_R}^{-1} \lctx$.
%   \item[Immediate input]
%     If $\octx \simu{R}^{-1} \lctx$ and $\ireduces{\atmR{\lctx}_L \oc #1 \oc \atmL{\lctx}_R}{\lctx}{\lctx'}$, then $\atmR{\lctx}_L \oc \octx \oc \atmL{\lctx}_R \Reduces\simu{R}^{-1} \lctx'$.
%   \item[Reduction closure]
%     If $\octx \simu{R}^{-1}\reduces \lctx'$, then $\octx \Reduces\simu{R}^{-1} \lctx'$.
%   % \item[Emptiness]
%   %   If $\octx \simu{R}^{-1} \octxe$, then:
%   %   \begin{itemize*}[label=, afterlabel=]
%   %   \item $\atmR{\lctx} \oc \octx \Reduces\lrframe{}{\simu{R}}{\atmR{\lctx}}^{-1} \atmR{\lctx}$ for all $\atmR{\lctx}$; and
%   %   \item $\octx \oc \atmL{\lctx} \Reduces\lrframe{\atmL{\lctx}}{\simu{R}}{}^{-1} \atmL{\lctx}$ for all $\atmL{\lctx}$.
%   %   \end{itemize*}
%   \item[Input contextuality]
%     If $\lctx \simu{R} \octx$, then $\atmR{\lctx}_L \oc \lctx \oc \atmL{\lctx}_R \simu{R} \atmR{\lctx}_L \oc \octx \oc \atmL{\lctx}_R$ for all $\atmR{\lctx}_L$ and $\atmL{\lctx}_R$.
%   \end{thmdescription}
% \end{theorem}
% \begin{proof}
%   \begin{description}
%   \item[Output bisimulation]
%     Suppose that $\octx \simu{R}^{-1} \lctx \Reduces \atmL{\lctx}'_L \oc \lctx' \oc \atmR{\lctx}'_R$.
%     By induction on the structure of the given trace, we can show $\octx \Reduces\lrframe{\atmL{\lctx}'_L}{\simu{R}}{\atmR{\lctx}'_R}^{-1} \atmL{\lctx}'_L \oc \lctx' \oc \atmR{\lctx}'_R$.
%     The base case follows from the immediate output bisimulation property of $\simu{R}$;
%     the inductive case follows from reduction closure of $\simu{R}$ and the inductive hypothesis.
%   \item[Input bisimulation]
%     Suppose that $\atmR{\lctx}_L \oc \octx \oc \atmL{\lctx}_R \lrframe{\atmR{\lctx}_L}{\simu{R}}{\atmL{\lctx}_R}^{-1} \atmR{\lctx}_L \oc \lctx \oc \atmL{\lctx}_R \Reduces \lctx'$.
%     We can show, by induction on the structure of the given trace, that $\atmR{\lctx}_L \oc \octx \oc \atmL{\lctx}_R \Reduces\simu{R}^{-1} \lctx'$.
%     The base case follows from the input context closure property of $\simu{R}$.
%     \begin{itemize}
%     \item Suppose that $\atmR{\lctx}_L \oc \octx \oc \atmL{\lctx}_R \lrframe{\atmR{\lctx}_L}{\simu{R}}{\atmL{\lctx}_R}^{-1} \atmR{\lctx}_L \oc \lctx \oc \atmL{\lctx}_R \reduces \lctx_1 \Reduces \lctx'$.
%       By Lemma?, there are three possible cases:
%       \begin{itemize}
%       \item If $\lctx \reduces \lctx'_1$ and $\lctx_1 = \atmR{\lctx}_L \oc \lctx'_1 \oc \atmL{\lctx}_R$, then it follows from the reduction closure property of $\simu{R}$ that $\atmR{\lctx}_L \oc \octx \oc \atmL{\lctx}_R \Reduces\lrframe{\atmR{\lctx}_L}{\simu{R}}{\atmL{\lctx}_R}^{-1}\Reduces \lctx'$.
%       \end{itemize}
%       In any case, $\atmR{\lctx}_L \oc \octx \oc \atmL{\lctx}_R \Reduces\lrframe{\atmR{\lctx}^1_L}{\simu{R}}{\atmL{\lctx}^1_R}^{-1} \lctx_1 \Reduces \lctx'$.
%       It follows from the inductive hypothesis that $\atmR{\lctx}_L \oc \octx \oc \atmL{\lctx}_R \Reduces\simu{R}^{-1} \lctx'$.
%     \end{itemize}
%     the inductive case follows from reduction closure of $\simu{R}$ and the inductive hypothesis.
%   \end{description}
% \end{proof}


% \begin{theorem}
%   Let $\simu{R}$ be a symmetric relation that satisfies the following conditions;
%   then $\ctxc{\simu{R}}$ is a rewriting bisimulation.
%   \begin{thmdescription}
%   \item[Immediate output]
%     If $\octx \simu{R}^{-1} \lctx = \atmL{\lctx}'_L \oc \lctx' \oc \atmR{\lctx}'_R$, then $\octx \Reduces\lrframe{\atmL{\lctx}'_L}{\ctxc{\simu{R}}}{\atmR{\lctx}'_R}^{-1} \lctx$.
%   \item[Immediate input]
%     If $\octx \simu{R}^{-1} \lctx$ and $\ireduces{\atmR{\lctx}_L \oc #1 \oc \atmL{\lctx}_R}{\lctx}{\lctx'}$, then $\atmR{\lctx}_L \oc \octx \oc \atmL{\lctx}_R \Reduces\ctxc{\simu{R}}^{-1} \lctx'$.
%   \item[Reduction closure]
%     If $\octx \simu{R}^{-1}\reduces \lctx'$, then $\octx \Reduces\ctxc{\simu{R}}^{-1} \lctx'$.
%   \item[Emptiness]
%     If $\octx \simu{R}^{-1} \octxe$, then:
%     \begin{itemize*}[label=, afterlabel=]
%     \item $\atmR{\lctx} \oc \octx \Reduces\lrframe{}{\ctxc{\simu{R}}}{\atmR{\lctx}}^{-1} \atmR{\lctx}$ for all $\atmR{\lctx}$; and
%     \item $\octx \oc \atmL{\lctx} \Reduces\lrframe{\atmL{\lctx}}{\ctxc{\simu{R}}}{}^{-1} \atmL{\lctx}$ for all $\atmL{\lctx}$.
%     \end{itemize*}
%   % \item[Input contextuality]
%   %   If $\lctx \simu{R} \octx$, then $\atmR{\lctx}_L \oc \lctx \oc \atmL{\lctx}_R \simu{R} \atmR{\lctx}_L \oc \octx \oc \atmL{\lctx}_R$ for all $\atmR{\lctx}_L$ and $\atmL{\lctx}_R$.
%   \end{thmdescription}
% \end{theorem}
% \begin{proof}
%   \begin{description}
%   \item[Immediate output]
%     We must show that $\octx \ctxc{\simu{R}}^{-1} \lctx = \atmL{\lctx}'_L \oc \lctx' \oc \atmR{\lctx}'_R$ implies $\octx \Reduces\lrframe{\atmL{\lctx}'_L}{\ctxc{\simu{R}}}{\atmR{\lctx}'_R}^{-1} \lctx$.
%     \begin{itemize}
%     \item 
%     \end{itemize}
%   \end{description}
% \end{proof}

% \subsection{}

% \begin{definition}
%   A symmetric binary relation $\simu{R}$ is a labeled bisimulation up to reflexivity and context if it satisfies the following conditions.
%   \begin{thmdescription}
%   \item[Immediate output bisimulation]
%     If $\octx \simu{R} \lctx = \atmL{\lctx}'_L \oc \lctx' \oc \atmR{\lctx}'_R$, then $\octx \Reduces\lrframe{\atmL{\lctx}'_L}{(\osim\ctxc{\reflc{\simu{R}}})}{\atmR{\lctx}'_R} \lctx$.
%   \item[Immediate input bisimulation]
%     If $\octx \simu{R} \lctx$ and $\ireduces{\atmR{\lctx}_L \oc #1 \oc \atmL{\lctx}_R}{\lctx}{\lctx'}$, then $\atmR{\lctx}_L \oc \octx \oc \atmL{\lctx}_R \Reduces\osim\ctxc{\reflc{\simu{R}}} \lctx'$.
%   \item[Reduction bisimulation]
%     If $\octx \simu{R}\reduces \lctx'$, then $\octx \Reduces\osim\ctxc{\reflc{\simu{R}}} \lctx'$.
%   \item[Emptiness bisimulation]
%     If $\octx \simu{R} \octxe$, then:
%     \begin{itemize*}[label=, afterlabel=]
%     \item $\atmR{\lctx} \oc \octx \Reduces\lrframe{}{(\osim\ctxc{\reflc{\simu{R}}})}{\atmR{\lctx}} \atmR{\lctx}$ for all $\atmR{\lctx}$; and 
%     \item $\octx \oc \atmL{\lctx} \Reduces\lrframe{\atmL{\lctx}}{(\osim\ctxc{\reflc{\simu{R}}})}{} \atmL{\lctx}$ for all $\atmL{\lctx}$
%     \end{itemize*}%
%     .
%   \end{thmdescription}
% \end{definition}

% \begin{lemma}
%   If $\simu{R}$ is a labeled bisimulation up to reflexivity and context, then $\osim\ctxc{\reflc{\simu{R}}}$ satisfies the following conditions.
%   \begin{thmdescription}
%   \item[Immediate output bisimulation]
%     If $\octx \osim\ctxc{\reflc{\simu{R}}} \lctx = \atmL{\lctx}'_L \oc \lctx' \oc \atmR{\lctx}'_R$, then $\octx \Reduces\lrframe{\atmL{\lctx}'_L}{(\osim\ctxc{\reflc{\simu{R}}})}{\atmR{\lctx}'_R} \lctx$.
%   \item[Reduction bisimulation]
%     If $\octx \osim\ctxc{\reflc{\simu{R}}}\reduces \lctx'$, then $\octx \Reduces\osim\ctxc{\reflc{\simu{R}}} \lctx'$.
%   \end{thmdescription}
% \end{lemma}
% \begin{proof}
%   \begin{description}
%   \item[Immediate output bisimulation]
%     Assume that $\octx \ctxc{\reflc{\simu{R}}} \lctx = \atmL{\lctx}'_L \oc \lctx' \oc \atmR{\lctx}'_R$; we must show that $\octx \Reduces\lrframe{\atmL{\lctx}'_L}{\ctxc{\reflc{\simu{R}}}}{\atmR{\lctx}'_R} \lctx$.
%     There are two cases: either the $\reflc{\simu{R}}$-related components of $\octx$ and $\lctx$ are equal or merely $\simu{R}$-related.
%     \begin{itemize}
%     \item If the $\reflc{\simu{R}}$-related components of $\octx$ and $\lctx$ are in fact equal, then so are $\octx$ and $\lctx$.
%       $\octx \ctxc{\reflc{\simu{R}}} \lctx$
%     \item
%       Otherwise, the $\reflc{\simu{R}}$-related components of $\octx$ and $\lctx$ are merely $\simu{R}$-related, with $\octx \ctxc{\simu{R}} \lctx = \atmL{\lctx}'_L \oc \lctx' \oc \atmR{\lctx}'_R$.
%       There are four subcases according to whether $\atmL{\lctx}'_L$ and $\atmR{\lctx}'_R$ are empty.
%       \begin{itemize}
%       \item Consider the subcase in which both $\atmL{\lctx}'_L$ and $\atmR{\lctx}'_R$ are empty;
%         in this subcase, we must show that $\octx \Reduces\ctxc{\reflc{\simu{R}}} \lctx$.
%         Because $\ctxc{\reflc{\simu{R}}}$ includes $\ctxc{\simu{R}}$, that follows directly.

%       \item Consider the subcase in which both $\atmL{\lctx}'_L$ and $\atmR{\lctx}'_R$ are nonempty.
%         The context $\lctx$ therefore exposes output atoms at its left and right ends, and so the $\ctxc{\simu{R}}$-related contexts $\octx$ and $\lctx$ must, in fact, be $\simu{R}$-related, for otherwise at least one end of $\lctx$ would expose an input, not output, atom.
%         In other words, $\octx \simu{R} \lctx = \atmL{\lctx}'_L \oc \lctx' \oc \atmR{\lctx}'_R$.
%         Because $\simu{R}$ is a labeled bisimulation up to reflexivity and contect, $\octx \Reduces\lrframe{\atmL{\lctx}'_L}{\ctxc{\reflc{\simu{R}}}}{\atmR{\lctx}'_R} \lctx$ follows from the immediate output property.

%       \item Consider the subcase in which $\atmL{\lctx}'_L$ is nonempty and $\atmR{\lctx}'_R$ is empty;
%         in this subcase, we must show that $\octx \Reduces\lrframe{\atmL{\lctx}'_L}{\ctxc{\reflc{\simu{R}}}}{} \lctx$.
%         Similarly to the previous case, $\lctx$ exposes output atoms at its left end because $\atmL{\lctx}'_L$ is nonempty.
%         And so the $\ctxc{\simu{R}}$-related contexts $\octx$ and $\lctx$ must, in fact, be $\lrframe{}{\simu{R}}{\atmL{\octx}_R}$-related, for some input context $\atmL{\octx}_R$.
%         In other words, $\octx \lrframe{}{\simu{R}}{\atmL{\octx}_R} \lctx = \atmL{\lctx}'_L \oc \lctx'$.

%         \begin{itemize}
%         \item Suppose that the left edge of $\atmL{\octx}_R$ does not extend into $\atmL{\lctx}'_L$.
%           Because $\simu{R}$ is a labeled bisimulation up to reflexivity and context, we may appeal to the immediate output property after framing off $\atmL{\octx}_R$ -- we deduce $\octx \lrframe{}{(\Reduces\lrframe{\atmL{\lctx}'_L}{\ctxc{\reflc{\simu{R}}}}{})}{\atmL{\octx}_R} \lctx$.
%           Reduction is closed under framing, so $\octx \Reduces\lrframe{\atmL{\lctx}'_L}{\ctxc{\reflc{\simu{R}}}}{} \lctx$.

%         \item Otherwise, suppose that the left edge of $\atmL{\octx}_R$ does extend into $\atmL{\lctx}'_L$ -- in other words, that $\atmL{\lctx}'_L = \atmL{\lctx}''_L \oc \atmL{\octx}'_L$ and $\atmL{\octx}_R = \atmL{\octx}'_L \oc \lctx'$ for some $\atmL{\octx},_L$.
%           We therefore have $\octx \lrframe{}{\simu{R}}{\atmL{\octx}_R} \lctx = \atmL{\lctx}''_L \oc \atmL{\octx}_R$.
%           Because $\simu{R}$ is a labeled bisimulation up to reflexivity and context, we may appeal to the immediate output property after framing off $\atmL{\octx}_R$ -- we deduce $\octx \lrframe{}{(\Reduces\lrframe{\atmL{\lctx}''_L}{\ctxc{\reflc{\simu{R}}}}{})}{\atmL{\octx}_R} \lctx$.
%           Reduction is closed under framing, so $\octx \Reduces\lrframe{\atmL{\lctx}''_L}{\ctxc{\reflc{\simu{R}}}}{\atmL{\octx}_R} \lctx$.

%           Respelling $\atmL{\octx}_R$ as $\atmL{\octx}'_L \oc \lctx'$, we have $\octx \Reduces\lrframe{\atmL{\lctx}''_L}{\lrframe{}{\ctxc{\reflc{\simu{R}}}}{\atmL{\octx}'_L}}{\lctx'} \lctx = \atmL{\lctx}''_L \oc \atmL{\octx}'_L \oc \lctx'$.
%           Once again, because $\simu{R}$ is a labeled bisimulation up to reflexivity and context, we may appeal to the emptiness property after framing off $\atmL{\lctx}''_L$ and $\lctx'$ -- we deduce $\octx \Reduces\lrframe{\atmL{\lctx}''_L}{(\Reduces\lrframe{\atmL{\octx}'_L}{\ctxc{\reflc{\simu{R}}}}{})}{\lctx'} \atmL{\lctx}''_L \oc \atmL{\octx}'_L \oc \lctx' = \lctx$.
%           Again, reduction is closed under framing, so $\octx \Reduces\lrframe{\atmL{\lctx}'_L}{\ctxc{\reflc{\simu{R}}}}{} \lctx$.
%         \end{itemize}
%       \item The subcase in which $\atmL{\lctx}'_L$ is nonempty and $\atmR{\lctx}'_R$ is empty.
%       \end{itemize}
%     \end{itemize}

%   \item[Reduction bisimulation]
%     Assume that $\octx \ctxc{\reflc{\simu{R}}}\reduces \lctx'$.
%     The $\reflc{\simu{R}}$-related components are either equal or $\simu{R}$-related.
%     In the latter case, either the reduction arises from the $\simu{R}$-related component alone, or it arises from an input transition of the $\simu{R}$-component that has its input demands met by the framing environment. 
%     There are thus three cases:
%     \begin{itemize}
%     \item Consider the case in which the $\reflc{\simu{R}}$-related components are in fact equal -- that is, the case in which $\octx \ctxc{=}\reduces \lctx'$.
%       $\octx \reduces \lctx'$.
%       Because $\ctxc{\reflc{\simu{R}}}$ is reflexive, $\octx \Reduces\ctxc{\reflc{\simu{R}}} \lctx'$.
      
%     \item Consider the case in which the reduction arises from the $\simu{R}$-related component alone -- that is, the case in which $\octx \ctxc{\simu{R}\reduces} \lctx'$.
%       Because $\simu{R}$ satisfies reduction bisimulation up to reflexivity and context, $\octx \ctxc{\Reduces\ctxc{\reflc{\simu{R}}}} \lctx'$.
%       Reduction is closed under framing and $\ctxc{}$ is an idempotent operation, so $\octx \Reduces\ctxc{\reflc{\simu{R}}} \lctx'$.
  
%     \item Consider the case in which the reduction arises from an input transition of the $\simu{R}$-related component that has had its input demands met by the framing environment -- that is, the case in which [...].
%       Because $\simu{R}$ satisfies immediate input bisimulation up to reflexivity and context, it follows that $\octx \ctxc{\Reduces\ctxc{\reflc{\simu{R}}}} \lctx'$.
%       Once again, reduction is closed under framing and $\ctxc{}$ is an idempotent operation, so $\octx \Reduces\ctxc{\reflc{\simu{R}}} \lctx'$.
%     \end{itemize}
%   \end{description}
% \end{proof}


% \section{}

% Given a binary relation $\simu{R}$, let $\simu{R}^*$ be the least \emph{reflexive} relation containing $\lrframe{\atmR{a}}{\simu{R}}{}$ for all $\atmR{a}$.
% If $\simu{R}$ is a blabeled bisimulation, then so is $\simu{R}^*$.
% \begin{description}
% \item[Immediate output bisimulation]
%   Assume that $\octx \lrframe{\atmR{a}}{\simu{R}}{} \lctx = \atmL{\lctx}'_L \oc \lctx' \oc \atmR{\lctx}'_R$ for some $\atmR{a}$; we must show that $\octx \Reduces\lrframe{\atmL{\lctx}'_L}{\simu{R}^*}{\atmR{\lctx}'_R} \lctx$.
%   \begin{itemize}
%   \item $\lctx'$ is nonempty.
%   \item $\lctx'$ is empty.
%   \end{itemize}

% \item[Reduction bisimulation]
%   Assume that $\octx \lrframe{\atmR{a}}{\simu{R}}{}\reduces \lctx'$.
%   \begin{itemize}
%   \item $\octx \lrframe{\atmR{a}}{(\simu{R}\reduces)}{} \lctx'$
%   \item $\octx \lrframe{\atmR{a}}{\simu{R}}{} \atmR{a} \oc \lctx$ where $\ireduces{\atmR{a} \oc #1}{\lctx}{\lctx'}$
%   \end{itemize}

% \item[Immediate input bisimulation]
%   Assume that $\octx \lrframe{\atmR{a}}{\simu{R}}{} \lctx$ and $\ireduces{\atmR{\lctx}_L \oc #1 \oc \atmL{\lctx}_R}{\lctx}{\lctx'}$; we must show that $\atmR{\lctx}_L \oc \octx \oc \atmL{\lctx}_R \Reduces\simu{R}^* \lctx'$.
%   \begin{itemize}
%   \item $\lctx = \atmR{a} \oc \lctx_0$ and $\ireduces{\atmR{\lctx}_L \oc \atmR{a} \oc #1 \oc \atmL{\lctx}_R}{\lctx_0}{\lctx'}$
%   \item $\lctx = \atmR{a} \oc \lctx_0$ and $\ireduces{#1 \oc \atmL{\lctx}_R}{\lctx_0}{\lctx'_0}$
%   \end{itemize}
% \end{description}

% If $\simu{R}$ is a labeled bisimulation, then $\ctxc{\simu{R}}$ satisfies immediate output and reduction properties.
% \begin{description}
% \item[Immediate output]
%   Assume that $\octx \lrframe{\atmR{a}}{\ctxc{\simu{R}}}{} \lctx = \atmL{\lctx}'_L \oc \lctx' \oc \atmR{\lctx}'_R$.
%   Inductive hypothesis.
%   Lemma: $\octx \Reduces\reflc{\lrframe{\atmR{a}}{\ctxc{\simu{R}}}{}} \lctx$.
% \item[Reduction]
%   Assume that $\octx \lrframe{\atmR{a}}{\ctxc{\simu{R}}}{}\reduces \lctx'$.
%   By the inductive hypothesis, $\ctxc{\simu{R}}$ satisfies the reduction property.
%   So, the above lemma yields $\octx \Reduces\reflc{\lrframe{\atmR{a}}{\ctxc{\simu{R}}}{}} \lctx'$.
%   \begin{itemize}
%   \item $\octx \ctxc{\Reduces\simu{R}^*} \lctx'$
%   \end{itemize}
% \end{description}


% \section{}

\begin{lemma}\label{lem:labeled-bisim-union}
  Let $\simu{S}$ be a labeled bisimulation.
  If $\simu{R}$ progresses to $\simu{R} \union \simu{S}$, then $\simu{R} \union \simu{S}$ is also a labeled bisimulation.
\end{lemma}
\begin{proof}
  When $\simu{S}$ is a labeled bisimulation and $\simu{R}$ progresses to $\simu{R} \union \simu{S}$, then $\simu{R} \union \simu{S}$ satisfies the conditions required of a labeled bisimulation.
  If $\octx \mathrel{(\simu{R} \union \simu{S})} \lctx$ because $\octx$ and $\lctx$ are $\simu{R}$-related
\end{proof}

Next, we show that 
\begin{lemma}\label{lem:single-input-atom}
  If $\simu{R}$ is a labeled bisimulation, then so are $\lframe{\atmR{a}}{\simu{R}} \union \simu{R}$ and $\rframe{\simu{R}}{\atmL{a}} \union \simu{R}$, for all $\atmR{a}$.
\end{lemma}
\begin{proof}
  Let $\simu{R}$ be a labeled bisimulation.
  We shall prove that $\lframe{\atmR{a}}{\simu{R}} \union \simu{R}$ is a labeled bisimulation; the proof for $\rframe{\simu{R}}{\atmL{a}} \union \simu{R}$ is symmetric.

  According to \cref{lem:labeled-bisim-union}, because $\simu{R}$ is a labeled bisimulation, it suffices to show that $\lframe{\atmR{a}}{\simu{R}}$ progresses to $\lframe{\atmR{a}}{\simu{R}} \union \simu{R}$.
  We prove each property in turn.
  \begin{description}
  \item[Immediate output bisimulation]
    Assume that $\octx \lframe{\atmR{a}}{\simu{R}} \lctx = \atmL{\lctx}'_L \oc \lctx' \oc \atmR{\lctx}'_R$; we must show that $\octx \Reduces\lrframe[\big]{\atmL{\lctx}'_L}{(\lframe{\atmR{a}}{\simu{R}} \union \simu{R})}{\atmR{\lctx}'_R} \lctx$.
    Because the input atom $\atmR{a}$ cannot be unified with the output atoms $\atmL{\lctx}'_L$, the context $\atmL{\lctx}'_L$ must be empty.
    We distinguish cases on the size of $\lctx'$.
    \begin{itemize}
    \item
      Consider the case in which $\lctx'$ is nonempty.
      Because $\simu{R}$ is a labeled bisimulation, we may appeal to its immediate output bisimulation property after framing off $\atmR{a}$ and deduce that $\octx \lframe[\big]{\atmR{a}}{(\Reduces\rframe{\simu{R}}{\atmR{\lctx}'_R})} \lctx$.
      Reduction is closed under framing, so we conclude that $\octx \Reduces\rframe{\lframe{\atmR{a}}{\simu{R}}}{\atmR{\lctx}'_R} \lctx$, as required.
      \begin{marginfigure}
        $
        \phantom{\octx = {}}
        \begin{tikzcd}[%
          /tikz/column 1/.append style={anchor=base east},
          /tikz/column 2/.append style={anchor=base east}
        ]
          \mathllap{\octx = {}} \atmR{a} \oc \octx_0
            \rar[relation, "\lframe{\atmR{a}}{\simu{R}}"]
            \dar[Reduces]
          &
          \atmR{a} \oc \lctx'_0 \oc \atmR{\lctx}'_R \mathrlap{{} = \lctx}
          \\
          \atmR{a} \oc \octx'_0
            \urar[relation, "\rframe{\lframe{\atmR{a}}{\simu{R}}}{\atmR{\lctx}'_R}" {sloped, below}]
          \\
          \octx_0
            \rar[relation, "\simu{R}"]
            \dar[Reduces]
          &
          \lctx'_0 \oc \atmR{\lctx}'_R
          \\
          \octx'_0
            \urar[relation, "\rframe{\simu{R}}{\atmR{\lctx}'_R}" {sloped, below}]
        \end{tikzcd}
        \phantom{{} = \lctx}
        $
      \end{marginfigure}
    %
    \item
      Consider the case in which $\lctx'$ is empty -- that is, the case in which $\octx \lframe{\atmR{a}}{\simu{R}} \lctx = \atmR{\lctx}'_R = \atmR{a} \oc \atmR{\lctx}''_R$ for some $\atmR{\lctx}''_R$.
      Because $\simu{R}$ is a labeled bisimulation, we may appeal to its immediate output bisimulation property after framing off $\atmR{a}$ and deduce that $\octx \lframe{\atmR{a}}{(\Reduces\rframe{\simu{R}}{\atmR{\lctx}''_R})} \atmR{\lctx}'_R$.
      Reduction is closed under framing, so $\octx \Reduces\rframe{\lframe{\atmR{a}}{\simu{R}}}{\atmR{\lctx}''_R} \atmR{\lctx}'_R$.
      After framing off $\atmR{\lctx}''_R$, we may subsequently appeal to the emptiness bisimulation property of $\simu{R}$ and deduce that $\octx \Reduces\rframe{(\Reduces\rframe{\simu{R}}{\atmR{a}})}{\atmR{\lctx}''_R} \atmR{\lctx}'_R$.
      Once again, reduction is closed under framing, so we conclude that $\octx \Reduces\rframe{\simu{R}}{\atmR{\lctx}'_R} \lctx$, as required.
      \begin{marginfigure}
        $
        \phantom{\octx = {}}
        \begin{tikzcd}[%
          /tikz/column 1/.append style={anchor=base east},
          /tikz/column 2/.append style={anchor=base east}
        ]
          \mathllap{\octx = {}} \atmR{a} \oc \octx_0
            \rar[relation, "\lframe{\atmR{a}}{\simu{R}}"]
            \dar[Reduces]
          &
          \atmR{a} \oc \atmR{\lctx}''_R \mathrlap{{} = \atmR{\lctx}'_R}
          \\
          \atmR{a} \oc \octx'_0
            \urar[relation, "\rframe{\lframe{\atmR{a}}{\simu{R}}}{\atmR{\lctx}''_R}" {sloped, below}]
            \dar[Reduces]
          \\
          \octx''_0 \oc \atmR{a}
            \arrow[relation, "\rframe{\rframe{\simu{R}}{\atmR{a}}}{\atmR{\lctx}''_R}" {sloped, below}]{uur}
        \end{tikzcd}
        \phantom{{} = \lctx}
        $
      \end{marginfigure}
    \end{itemize}

  \item[Immediate input bisimulation]
    Assume that $\octx \lrframe{\atmR{a}}{\simu{R}}{} \lctx$ and $\ireduces{\atmR{\lctx}_L \oc #1 \oc \atmL{\lctx}_R}{\lctx}{\lctx'}$; we must show that $\atmR{\lctx}_L \oc \octx \oc \atmL{\lctx}_R \Reduces\mathrel{(\lframe{\atmR{a}}{\simu{R}} \union \simu{R})} \lctx'$.
    According to \cref{lem:input-framing}, there are two cases: either $\atmR{a}$ satisfies an input demand, or it does not participate in the given input transition.
    \begin{itemize}
    \item
      Consider the case in which $\atmR{a}$ does participate in the input transition -- that is, the case in which $\octx \lframe{\atmR{a}}{\simu{R}} \atmR{a} \oc \lctx_0 = \lctx$ and $\ireduces{\atmR{\lctx}_L \oc \atmR{a} \oc #1 \oc \atmL{\lctx}_R}{\lctx_0}{\lctx'}$, for some $\lctx_0$.
      Because $\simu{R}$ is a labeled bisimulation, we may appeal to its immediate input bisimulation property and deduce that $\atmR{\lctx}_L \oc \octx \oc \atmL{\lctx}_R \Reduces\simu{R} \lctx'$, as required.
    %
    \item
      Consider the case in which $\atmR{a}$ does not participate in the transition -- that is, the case in which $\atmR{\lctx}_L$ is empty and $\octx \lframe{\atmR{a}}{\simu{R}} \atmR{a} \oc \lctx_0 = \lctx$ and $\ireduces{#1 \oc \atmL{\lctx}_R}{\lctx_0}{\lctx'_0}$ and $\lctx' = \atmR{a} \oc \lctx'_0$, for some $\lctx_0$ and $\lctx'_0$.
      Because $\simu{R}$ is a labeled bisimulation, we may appeal to its immediate input bisimulation property after framing off $\atmR{a}$ and deduce that $\octx \oc \atmL{\lctx}_R \lframe{\atmR{a}}{(\Reduces\simu{R})} \lctx'$.
      Reduction is closed under framing, so we conclude that $\octx \oc \atmL{\lctx}_R \Reduces\lframe{\atmR{a}}{\simu{R}} \lctx'$, as required.
    \end{itemize}

  \item[Reduction bisimulation]
    Assume that $\octx \lframe{\atmR{a}}{\simu{R}}\reduces \lctx'$; we must show that $\octx \Reduces\mathrel{(\lframe{\atmR{a}}{\simu{R}} \union \simu{R})} \lctx'$.
    We distinguish cases on the origin of the given reduction.
    \begin{itemize}
    \item
      Consider the case in which the reduction arises from the $\simu{R}$-related component alone -- that is, the case in which $\octx \lframe{\atmR{a}}{(\simu{R}\reduces)} \lctx'$.
      Because $\simu{R}$ is a labeled bisimulation, we may appeal to its reduction bisimulation property after framing off $\atmR{a}$ and deduce that $\octx \lframe{\atmR{a}}{(\Reduces\simu{R})} \lctx'$.
      Reduction is closed under framing, so we conclude that $\octx \Reduces\lframe{\atmR{a}}{\simu{R}} \lctx'$, as required.
    \item
      Consider the case in which the reduction arises from an input transition on the $\simu{R}$-related component -- that is, the case in which $\octx \lframe{\atmR{a}}{\simu{R}} \atmR{a} \oc \lctx_0 = \lctx$ and $\ireduces{\atmR{a} \oc #1}{\lctx_0}{\lctx'}$, for some $\lctx_0$.
      Because $\simu{R}$ is a labeled bisimulation, we may appeal to its immediate input bisimulation property and deduce that $\octx \Reduces\simu{R} \lctx'$, as required.
    \end{itemize}

  \item[Emptiness bisimulation]
    Assume that $\octx \lframe{\atmR{a}}{\simu{R}} \octxe$.
    This is, in fact, impossible because the empty context does not contain $\atmR{a}$.
  %
  \qedhere
  \end{description}
\end{proof}

\begin{lemma}\label{lem:ctxc-labeled-bisim}
  If $\simu{R}$ is a labeled bisimulation, then so is $\ctxc{\simu{R}}$.
\end{lemma}
\begin{proof}
  Let $(\simu{S}_n)_{n \in \nats}$ be the indexed family of relations given by
  \begin{align*}
    \mathord{\simu{S}_0} &= \mathord{\simu{R}} \\
    \mathord{\simu{S}_{n+1}} &= \textstyle
                                  \parens[size=Big]{\bigunion_{\atmR{a}} \mathord{\lframe{\atmR{a}}{\simu{S}_n}}}
                                  \union \parens[size=Big]{\bigunion_{\atmL{a}} \mathord{\rframe{\simu{S}_n}{\atmL{a}}}}
                                  \union \mathord{S}_n
    \,.
  \end{align*}
  % We shall prove that $\mathord{\ctxc{\simu{R}}} = \bigunion_{n=0}^{\infty}{\mathord{\simu{S}_n}}$.
  %
  It is easy to prove by structural induction that each $\ctxc{\simu{R}}$-related pair of contexts is also $\simu{S}_n$-related for some natural number $n$; and so $\ctxc{\simu{R}}$ is contained within $\bigunion_{n=0}^{\infty}{\mathord{\simu{S}_n}}$.
  % 
  Conversely, using \cref{lem:single-input-atom}, it is equally easy to prove by induction on $n$ that each $\simu{S}_n$ is contained within $\ctxc{\simu{R}}$ and, moreover, that each $\simu{S}_n$ is a labeled bisimulation.

  Because each $\simu{S}_n$ is a labeled bisimulation, so is their least upper bound, namely $\bigunion_{n=0}^{\infty}{\mathord{\simu{S}_n}} = \mathord{\ctxc{\simu{R}}}$.
  %
  % For each $\ctxc{\simu{R}}$-related pair of contexts, $\octx$ and $\lctx$, there exists a least natural number $n$ for which those contexts are $\simu{S}_n$-related.
  % Specifically, if $\octx$ and $\lctx$ are $\ctxc{\simu{R}}$-related by virtue of $\octx \lrframe{\atmR{\lctx}_L}{\simu{R}}{\atmL{\lctx}_R} \lctx$, then $\card{\atmR{\lctx}_L} + \card{\atmL{\lctx}_R}$ is the least natural number $n$ for which $\octx$ and $\lctx$ are $\simu{S}_n$-related.
  %
  % Using \cref{lem:single-input-atom}, it is easy to prove by induction on $n$ that each $\simu{S}_n$ is contained within $\ctxc{\simu{R}}$ and, moreover, that each $\simu{S}_n$ is a labeled bisimulation.
  % Therefore, $\mathord{\ctxc{\simu{R}}} = \bigunion_{n=0}^{\infty}{\mathord{\simu{S}_n}}$.
  % Because each $\simu{S}_n$ is a labeled bisimulation, so is their least upper bound, $\ctxc{\simu{R}}$.
\end{proof}

\begin{theorem}\label{thm:labeled-proof-technique}
  If $\simu{R}$ is a labeled bisimulation, then rewriting bisimilarity contains $\simu{R}$.
\end{theorem}
\begin{proof}
  Let $\simu{R}$ be a labeled bisimulation.
  By \cref{lem:ctxc-labeled-bisim}, so is $\ctxc{\simu{R}}$.
  The relation $\ctxc{\simu{R}}$ is also a rewriting bisimulation, as we will show by proving each property in turn.
  (Notice, too, that $\ctxc{\simu{R}}$ is symmetric because $\simu{R}$ is.)
  \begin{description}[listparindent=\parindent]
  \item[Output bisimulation]
    Assume that $\octx \ctxc{\simu{R}}\Reduces \atmL{\lctx}'_L \oc \lctx' \oc \atmR{\lctx}'_R$; we must show that $\octx \Reduces\lrframe{\atmL{\lctx}'_L}{\ctxc{\simu{R}}}{\atmR{\lctx}'_R} \atmL{\lctx}'_L \oc \lctx' \oc \atmR{\lctx}'_R$.

    As a labeled bisimulation, $\ctxc{\simu{R}}$ satisfies the reduction bisimulation property, so we deduce that $\octx \Reduces\ctxc{\simu{R}} \atmL{\lctx}'_L \oc \lctx' \oc \atmR{\lctx}'_R$.
    The relation $\ctxc{\simu{R}}$ also satisfies the immediate output bisimulation property, so we conclude that $\octx \Reduces\lrframe{\atmL{\lctx}'_L}{\ctxc{\simu{R}}}{\atmR{\lctx}'_R} \atmL{\lctx}'_L \oc \lctx' \oc \atmR{\lctx}'_R$, as required.
  %
  \item[Input bisimulation]
    Assume that $\octx \ctxc{\simu{R}} \lctx$ and $\ireduces{\atmR{\lctx}_L \oc #1 \oc \atmL{\lctx}_R}{\lctx}{\lctx'}$; we must show that $\atmR{\lctx}_L \oc \octx \oc \atmL{\lctx}_R \Reduces\ctxc{\simu{R}} \lctx'$.

    By \cref{??}, the given input transition gives rise to a reduction: $\atmR{\lctx}_L \oc \octx \oc \atmL{\lctx}_R \lrframe{\atmR{\lctx}_L}{\ctxc{\simu{R}}}{\atmL{\lctx}_R}\reduces \lctx'$.
    Because $\ctxc{\simu{R}}$ is input contextual, we deduce that $\atmR{\lctx}_L \oc \octx \oc \atmL{\lctx}_R \ctxc{\simu{R}}\reduces \lctx'$.
    As a labeled bisimulation, $\ctxc{\simu{R}}$ satisfies the reduction bisimulation property, so we conclude that $\atmR{\lctx}_L \oc \octx \oc \atmL{\lctx}_R \Reduces\ctxc{\simu{R}} \lctx'$, as required.
  %
  \qedhere
  \end{description}
\end{proof}

\begin{corollary}
  Labeled bisimilarity is sound and complete with respect to rewriting bisimilarity.
\end{corollary}

\newthought{As a simple example} of this [labeled bisimilarity] proof technique, we shall now establish that $\atmR{a} \oc (\atmR{a} \limp \atmR{b})$ and $\atmR{b}$ are rewriting-bisimilar contexts.
Let $\simu{R}$ be the least symmetric binary relation for which $\atmR{a} \oc (\atmR{a} \limp \atmR{b}) \simu{R} \atmR{b}$ and $\atmR{b} \simu{R} \atmR{b}$ and $\octxe \simu{R} \octxe$ hold.
The relation $\simu{R}$ is a labeled bisimulation:
\begin{itemize}
\item The immediate output bisimulation condition holds because $\atmR{a} \oc (\atmR{a} \limp \atmR{b})$ can simulate $\atmR{b}$'s output of $\atmR{b}$ (with $\atmR{a} \oc (\atmR{a} \limp \atmR{b}) \reduces\rframe{\simu{R}}{\atmR{b}} \atmR{b}$) and the former makes no immediate outputs of its own.
\item The immediate input bisimulation condition holds vacuously for [the relation] $\simu{R}$ because neither $\atmR{a} \oc (\atmR{a} \limp \atmR{b})$ nor $\atmR{b}$ accept any inputs on either side.
\item The reduction bisumulation condition holds because $\atmR{b}$ can simulate the reduction $\atmR{a} \oc (\atmR{a} \limp \atmR{b}) \reduces \atmR{b}$ trivially (with $\atmR{b} \Reduces\simu{R} \atmR{b}$).
\item The emptiness bisimulation condition holds trivially: $\atmR{\lctx} \Reduces\rframe{\simu{R}}{\atmR{\lctx}} \atmR{\lctx}$ for all $\atmR{\lctx}$ because $\octxe \simu{R} \octxe$, and symmetrically for all $\atmL{\lctx}$.
\end{itemize}
We may conclude from the above proof technique~\parencref{thm:labeled-proof-technique} that $\simu{R}$ is contained within rewriting bisimilarity and that $\atmR{a} \oc (\atmR{a} \limp \atmR{b})$ and $\atmR{b}$ are indeed bisimilar.

We can similarly prove that $\atmR{a} \limp (\atmR{c} \pmir \atmL{b})$ and $(\atmR{a} \limp \atmR{c}) \pmir \atmL{b}$ are rewriting-bisimilar by showing that the least symmetric relation $\simu{R}$ such that $\atmR{a} \limp (\atmR{c} \pmir \atmL{b}) \simu{R} (\atmR{a} \limp \atmR{c}) \pmir \atmL{b}$ and $\atmR{c} \simu{R} \atmR{c}$ and $\octxe \simu{R} \octxe$ is a labeled bisimulation.

Somewhat surprisingly, even $\atmR{a} \limp \up \dn (\atmR{c} \pmir \atmL{b})$ and $\up \dn (\atmR{a} \limp \atmR{c}) \pmir \atmL{b}$ are bisimilar.
Once again, this can be proved by constructing an appropriate label bisimulation, namely the least symmetric relation $\simu{R}$ such that:
$\atmR{a} \limp \up \dn (\atmR{c} \pmir \atmL{b}) \simu{R} \up \dn (\atmR{a} \limp \atmR{c}) \pmir \atmL{b}$ and
$\atmR{c} \pmir \atmL{b} \simu{R} \atmR{a} \oc \bigl(\up \dn (\atmR{a} \limp \atmR{c}) \pmir \atmL{b}\bigr)$ and
$\bigl(\atmR{a} \limp \up \dn (\atmR{c} \pmir \atmL{b})\bigr) \oc \atmL{b} \simu{R} \atmR{a} \limp \atmR{c}$ and
$\atmR{c} \simu{R} \atmR{c}$ and
$\octxe \simu{R} \octxe$.



% $\atmR{a} \limp \up \dn (\atmR{c} \pmir \atmL{b}) \simu{R} \up \dn (\atmR{a} \limp \atmR{c}) \pmir \atmL{b}$
% and
% $\atmR{c} \pmir \atmL{b} \simu{R} \atmR{a} \oc \bigl(\up \dn (\atmR{a} \limp \atmR{c}) \pmir \atmL{b}\bigr)$
% and
% $\bigl(\atmR{a} \limp \up \dn (\atmR{c} \pmir \atmL{b})\bigr) \oc \atmL{b} \simu{R} \atmR{a} \limp \atmR{c}$
% and
% $(\atmR{c} \pmir \atmL{b}) \oc \atmL{b} \simu{R} \atmR{c}$
% and
% $\atmR{a} \oc (\atmR{a} \limp \atmR{c}) \simu{R} \atmR{c}$
% and
% $\atmR{c} \simu{R} \atmR{c}$
% and
% $\octxe \simu{R} \octxe$.
% The relation is a labeled bisimulation because:
% \begin{itemize}
% \item Notice that $\ireduces{#1 \oc \atmL{b}}{\up \dn (\atmR{a} \limp \atmR{c}) \pmir \atmL{b}}{\atmR{a} \limp \atmR{c}}$.
%   And indeed $\bigl(\atmR{a} \limp \up \dn (\atmR{c} \pmir \atmL{b})\bigr) \oc \atmL{b} \Reduces\simu{R} \atmR{a} \limp \atmR{c}$.
%   Similarly, notice that $\ireduces{\atmR{a} \oc #1}{\atmR{a} \limp \up \dn (\atmR{c} \pmir \atmL{b})}{\atmR{c} \pmir \atmL{b}}$.
%   And indeed $\atmR{a} \oc \bigl(\up \dn (\atmR{a} \limp \atmR{c}) \pmir \atmL{b}\bigr) \Reduces\simu{R} \atmR{c} \pmir \atmL{b}$.
% \item Notice that $\ireduces{\atmR{a} \oc #1}{\atmR{a} \limp \atmR{c}}{\atmR{c}}$.
%   And indeed $\atmR{a} \oc \bigl(\up \dn (\atmR{a} \limp \atmR{c}) \pmir \atmL{b}\bigr) \oc \atmL{b} \Reduces\simu{R} \atmR{c}$.
%   Notice that $\ireduces{\atmR{a} \oc #1}{\bigl(\atmR{a} \limp \up \dn (\atmR{c} \pmir \atmL{b})\bigr) \oc \atmL{b}}{(\atmR{c} \pmir \atmL{b}) \oc \atmL{b}}$.
%   And indeed $\atmR{a} \oc \bigl(\up \dn (\atmR{a} \limp \atmR{c}) \pmir \atmL{b}\bigr) \oc \atmL{b} \Reduces\simu{R} (\atmR{c} \pmir \atmL{b}) \oc \atmL{b}$.
% \end{itemize}


\subsection{A simple up-to proof technique: Reflexivity}

As a slight enhancement of the above proof technique, 

\begin{lemma}\label{lem:identity-labeled-bisim}
  The identity relation is a labeled bisimulation.
\end{lemma}
\begin{proof}
  Each of the labeled bisimulation conditions is trivially true of the identity relation.
\end{proof}

Let us call a relation $\simu{R}$ a labeled bisimulation \vocab{up to reflexivity} if $\simu{R}$ progresses to its reflexive closure, $\reflc{\simu{R}}$.
\begin{theorem}\label{thm:bisim-technique-up-to-refl}
  If $\simu{R}$ is a labeled bisimulation up to reflexivity, then rewriting bisimilarity contains $\simu{R}$.
\end{theorem}
\begin{proof}
  Let $\simu{R}$ be a labeled bisimulation up to reflexivity.
  notice that 
  Because the identity relation is a labeled bisimulation \parencref{??}, it follows from \cref{??} that $\reflc{\simu{R}}$, the reflexive closure of $\simu{R}$, is a labeled bisimulation.
  By \cref{??}, we may conclude that rewriting bisimilarity contains $\reflc{\simu{R}}$ and hence $\simu{R}$.
\end{proof}

% \begin{lemma}
%   If $\simu{R}$ is a labeled bisimulation up to reflexivity, then $\lframe{\atmR{a}}{\reflc{\simu{R}}} \union \reflc{\simu{R}}$ and $\rframe{\reflc{\simu{R}}}{\atmL{a}} \union \reflc{\simu{R}}$ are labeled bisimulations, for all $\atmR{a}$.
% \end{lemma}
% \begin{proof}
%   Let $\simu{R}$ be a labeled bisimulation up to reflexivity.
%   We shall prove that $\lframe{\atmR{a}}{\reflc{\simu{R}}} \union \reflc{\simu{R}}$ is a labeled bisimulation; the proof for $\rframe{\reflc{\simu{R}}}{\atmL{a}} \union \reflc{\simu{R}}$ is symmetric.

%   First, notice that $\mathord{\lframe{\atmR{a}}{\reflc{\simu{R}}}} \union \mathord{\reflc{\simu{R}}} = \mathord{\lframe{\atmR{a}}{\simu{R}}} \union \mathord{\reflc{\simu{R}}}$.
%   Because $\simu{R}$ is a labeled bisimulation up to reflexivity, $\reflc{\simu{R}}$ is a labeled bisimulation \parencref{??}.
%   According to \cref{??}, to show that $\lframe{\atmR{a}}{\reflc{\simu{R}}} \union \reflc{\simu{R}}$ is a labeled bisimulation, it therefore suffices to show that $\lframe{\atmR{a}}{\simu{R}}$ progresses to $\lframe{\atmR{a}}{\reflc{\simu{R}}} \union \reflc{\simu{R}}$.
%   We prove each property in turn.
%   \begin{description}
%   \item[Immediate output bisimulation]
%     Assume that $\octx \lframe{\atmR{a}}{\simu{R}} \lctx = \atmL{\lctx}'_L \oc \lctx' \oc \atmR{\lctx}'_R$; we must show that $\octx \Reduces\lrframe{\atmL{\lctx}'_L}{(\lframe{\atmR{a}}{\reflc{\simu{R}}} \union \reflc{\simu{R}})}{\atmR{\lctx}'_R} \lctx$.
%     Because the input atom $\atmR{a}$ cannot be unified with the output atoms $\atmL{\lctx}'_L$, the context $\atmL{\lctx}'_L$ must be empty.
%     We distinguish cases on the size of $\lctx'$.
%     \begin{itemize}
%     \item
%       Consider the case in which $\lctx'$ is nonempty.
%       Because $\simu{R}$ is a labeled bisimulation up to reflexivity, we may appeal to its immediate output bisimulation property after framing off $\atmR{a}$ and deduce that $\octx \lframe{\atmR{a}}{(\Reduces\rframe{\reflc{\simu{R}}}{\atmR{\lctx}'_R})} \lctx$.
%       Reduction is closed under framing, so we conclude that $\octx \Reduces\rframe{\lframe{\atmR{a}}{\reflc{\simu{R}}}}{\atmR{\lctx}'_R} \lctx$, as required.
%     %
%     \item
%       Consider the case in which $\lctx'$ is empty -- that is, the case in which $\octx \lframe{\atmR{a}}{\simu{R}} \lctx = \atmR{\lctx}'_R = \atmR{a} \oc \atmR{\lctx}''_R$ for some $\atmR{\lctx}''_R$.
%       Because $\simu{R}$ is a labeled bisimulation up to reflexivity, we may appeal to its immediate output bisimulation property after framing off $\atmR{a}$ and deduce that $\octx \lframe{\atmR{a}}{(\Reduces\rframe{\reflc{\simu{R}}}{\atmR{\lctx}''_R})} \atmR{\lctx}'_R$.
%       Reduction is closed under framing, so $\octx \Reduces\rframe{\lframe{\atmR{a}}{\reflc{\simu{R}}}}{\atmR{\lctx}''_R} \atmR{\lctx}'_R$.
%       After framing off $\atmR{\lctx}''_R$, we may subsequently appeal to the emptiness bisimulation property of $\reflc{\simu{R}}$ and deduce that $\octx \Reduces\rframe{(\Reduces\rframe{\reflc{\simu{R}}}{\atmR{a}})}{\atmR{\lctx}''_R} \atmR{\lctx}'_R$.
%       Once again, reduction is closed under framing, so we conclude that $\octx \Reduces\rframe{\reflc{\simu{R}}}{\atmR{\lctx}'_R} \lctx$, as required.
%     \end{itemize}

%   \item[Immediate input bisimulation]
%     Assume that $\octx \lrframe{\atmR{a}}{\simu{R}}{} \lctx$ and $\ireduces{\atmR{\lctx}_L \oc #1 \oc \atmL{\lctx}_R}{\lctx}{\lctx'}$; we must show that $\atmR{\lctx}_L \oc \octx \oc \atmL{\lctx}_R \Reduces\mathrel{(\lframe{\atmR{a}}{\reflc{\simu{R}}} \union \reflc{\simu{R}})} \lctx'$.
%     According to \cref{??}, there are two cases: either $\atmR{a}$ does not participate in the given input transition, or it is an input demand that is already present.
%     \begin{itemize}
%     \item
%       Consider the case in which $\atmR{a}$ does not participate in the transition -- that is, the case in which $\atmR{\lctx}_L$ is empty and $\octx \lframe{\atmR{a}}{\simu{R}} \atmR{a} \oc \lctx_0 = \lctx$ and $\ireduces{#1 \oc \atmL{\lctx}_R}{\lctx_0}{\lctx'_0}$ and $\lctx' = \atmR{a} \oc \lctx'_0$, for some $\lctx_0$ and $\lctx'_0$.
%       Because $\simu{R}$ is a labeled bisimulation up to reflexivity, we may appeal to its immediate input bisimulation property after framing off $\atmR{a}$ and deduce that $\octx \oc \atmL{\lctx}_R \lframe{\atmR{a}}{(\Reduces\reflc{\simu{R}})} \lctx'$.
%       Reduction is closed under framing, so we conclude that $\octx \oc \atmL{\lctx}_R \Reduces\lframe{\atmR{a}}{\reflc{\simu{R}}} \lctx'$, as required.
%     %
%     \item
%       Consider the case in which $\atmR{a}$ does participate in the input transition -- that is, the case in which $\octx \lframe{\atmR{a}}{\simu{R}} \atmR{a} \oc \lctx_0 = \lctx$ and $\ireduces{\atmR{\lctx}_L \oc \atmR{a} \oc #1 \oc \atmL{\lctx}_R}{\lctx_0}{\lctx'}$, for some $\lctx_0$.
%       Because $\simu{R}$ is a labeled bisimulation up to reflexivity, we may appeal to its immediate input bisimulation property and deduce that $\atmR{\lctx}_L \oc \octx \oc \atmL{\lctx}_R \Reduces\reflc{\simu{R}} \lctx'$, as required.
%     \end{itemize}

%   \item[Reduction bisimulation]
%     Assume that $\octx \lframe{\atmR{a}}{\simu{R}}\reduces \lctx'$; we must show that $\octx \Reduces\mathrel{(\lframe{\atmR{a}}{\reflc{\simu{R}}} \union \reflc{\simu{R}})} \lctx'$.
%     We distinguish cases on the origin of the given reduction.
%     \begin{itemize}
%     \item
%       Consider the case in which the reduction arises from the $\simu{R}$-related component alone -- that is, the case in which $\octx \lframe{\atmR{a}}{(\simu{R}\reduces)} \lctx'$.
%       Because $\simu{R}$ is a labeled bisimulation up to reflexivity, we may appeal to its reduction bisimulation property after framing off $\atmR{a}$ and deduce that $\octx \lframe{\atmR{a}}{(\Reduces\reflc{\simu{R}})} \lctx'$.
%       Reduction is closed under framing, so we conclude that $\octx \Reduces\lframe{\atmR{a}}{\reflc{\simu{R}}} \lctx'$, as required.
%     \item
%       Consider the case in which the reduction arises from an input transition on the $\simu{R}$-related component -- that is, the case in which $\octx \lframe{\atmR{a}}{\simu{R}} \atmR{a} \oc \lctx_0 = \lctx$ and $\ireduces{\atmR{a} \oc #1}{\lctx_0}{\lctx'}$, for some $\lctx_0$.
%       Because $\simu{R}$ is a labeled bisimulation up to reflexivity, we may appeal to its immediate input bisimulation property and deduce that $\octx \Reduces\reflc{\simu{R}} \lctx'$, as required.
%     \end{itemize}

%   \item[Emptiness bisimulation]
%     Assume that $\octx \lframe{\atmR{a}}{\simu{R}} \octxe$.
%     This is, in fact, impossible because the empty context does not contain $\atmR{a}$.
%   %
%   \qedhere
%   \end{description}
% \end{proof}


% \begin{lemma}
%   If $\simu{R}$ is a labeled bisimulation up to reflexivity and context, then $\lframe{\atmR{a}}{\ctxc{\reflc{\simu{R}}}} \union \ctxc{\reflc{\simu{R}}}$ is a labeled bisimulation.
% \end{lemma}
% \begin{proof}
%   \begin{description}
%   \item[Output] Assume that $\octx \lframe{\atmR{a}}{\ctxc{\simu{R}}} \lctx = \atmL{\lctx}'_L \oc \lctx' \oc \atmR{\lctx}'_R$.
%     $\octx_0 \ctxc{\simu{R}} \lctx'_0 \oc \atmR{\lctx}'_R$.
%     \begin{itemize}
%     \item $\octx \lframe{\atmR{a}}{(\Reduces\rframe{\reflc{\simu{R}}}{\atmR{\lctx}'_R})} \lctx$.
%     \item $\octx \lframe{\atmR{a}}{(\Reduces\rframe{\reflc{\simu{R}}}{\atmR{\lctx}''_R})} \lctx$.
%       $\octx \Reduces\rframe{\lframe{\atmR{a}}{\reflc{\simu{R}}}}{\atmR{\lctx}''_R} \lctx$.
%     \end{itemize}
%     Assume that $\octx \simu{R} \lctx = ...$.
%     $\octx \Reduces\lrframe{\atmL{\lctx}'_L}{\reflc{\simu{R}}}{\atmR{\lctx}'_R} \lctx$.
%   \end{description}
% \end{proof}

% \begin{theorem}
%   If $\simu{R}$ is a labeled bisimulation up to reflexivity, then $\ctxc{\reflc{\simu{R}}}$ is a labeled bisimulation.
% \end{theorem}
% \begin{proof}
%   \begin{align*}
%     \mathord{\simu{S}_0} &= \mathord{\reflc{\simu{R}}} \\
%     \mathord{\simu{S}_{n+1}} &= \mathord{\reflc{\simu{S}_n}} \union \parens[size=Big]{\bigunion_{\atmR{a}} \lframe{\atmR{a}}{\reflc{\simu{S}_n}}} \union \parens[size=Big]{\bigunion_{\atmL{a}} \rframe{\reflc{\simu{S}_n}}{\atmL{a}}} 
%   \end{align*}
% \end{proof}



\section{}

Rewriting bisimilarity satisfies the usual properties expected of a notion of bisimilarity: it is a congruence [...].
%
\begin{theorem}
  Rewriting bisimilarity is an equivalence relation.
\end{theorem}
\begin{proof}
  Equality of contexts can be shown to be a bisimulation, so rewriting bisimilarity is reflexive.
  Rewriting bisimilarity is symmetric, by definition.
  The relation $\osim^2$ can be shown to be a bisimulation, so rewriting bisimilarity is transitive.
\end{proof}



\begin{lemma}
  Rewriting bisimilarity contains $\lframe{\p{A}}{\osim}$ and $\rframe{\osim}{\p{A}}$, for all propositions $\p{A}$.
\end{lemma}
\begin{proof}
  By induction over the structure of $\p{A}$.
  We show only the proof for $\lframe{\p{A}}{\osim}$; the proof for $\rframe{\osim}{\p{A}}$ is symmetric.

  According to \cref{??}, it suffices to show that $\lframe{\p{A}}{\osim} \union \osim$ is a labeled bisimulation.
  By \cref{??}, we need only show that $\lframe{\p{A}}{\osim}$ progresses to $\lframe{\p{A}}{\osim} \union \osim$.
  Many of the cases follow the pattern laid out in the proof of \cref{??}, substituting $\p{A}$ for $\atmR{a}$; we show only the new cases.
  \begin{description}
  \item[Immediate output bisimulation]
    Assume that $\octx \lframe{\p{A}}{\osim} \lctx = \atmL{\lctx}'_L \oc \lctx' \oc \atmR{\lctx}'_R$;
    we must show that $\octx \Reduces\lrframe[\big]{\atmL{\lctx}'_L}{(\lframe{\p{A}}{\osim} \union \osim)}{\atmR{\lctx}'_R} \lctx$.

    Unlike in the proof of \cref{??}, here it is possible that $\p{A} = \atmL{a}$ with $\atmL{\lctx}'_L$ nonempty: $\atmL{\lctx}'_L = \atmL{a} \oc \atmL{\lctx}''_L$, for some $\atmL{\lctx}''_L$.
    Because $\osim$ is a labeled bisimulation~\parencref{??}, we may appeal to its immediate output bisimulation property after framing $\atmL{a}$ and deduce that $\octx \lframe[\big]{\atmL{a}}{(\Reduces\lrframe{\atmL{\lctx}''_L}{\osim}{\atmR{\lctx}'_R})} \lctx$.
    Reduction is closed under framing, so $\octx \Reduces\lrframe{\atmL{\lctx}'_L}{\osim}{\atmR{\lctx}'_R} \lctx$, as required.

    The other cases follow the pattern laid out in the proof of \cref{??}.

  \item[Immediate input bisimulation]
    Assume that $\octx \lframe{\p{A}}{\osim} \lctx$ and $\ireduces{\atmR{\lctx}_L \oc #1 \oc \atmL{\lctx}_R}{\lctx}{\lctx'}$;
    we must show that $\atmR{\lctx}_L \oc \octx \oc \atmL{\lctx}_R \Reduces\mathrel{\bigl(\lframe{\p{A}}{\osim} \union \osim\bigr)} \lctx'$.

    All cases here follow the pattern laid out in the proof of \cref{??}.
    
  \item[Reduction bisimulation]
    Assume that $\octx \lframe{\p{A}}{\osim}\reduces \lctx'$;
    we must show that $\octx \Reduces\mathrel{(\lframe{\p{A}}{\osim} \union \osim)} \lctx'$.

    As in the proof of \cref{??}, we distinguish cases based on the origin of the reduction.
    Here, there are three new cases because the reduction might originate from $\p{A}$; the other cases follow the pattern laid out in the proof of \cref{??}.
    \begin{itemize}
    \item
      Consider the case in which $\p{A} = \p{A}_1 \fuse \p{A}_2$ and $\octx \lframe{\p{A}}{\osim} (\p{A}_1 \fuse \p{A}_2) \oc \lctx_0 \reduces \p{A}_1 \oc \p{A}_2 \oc \lctx_0 = \lctx'$ for some $\lctx_0$.
      Notice that $\octx \reduces\lframe{\p{A}_1}{\lframe{\p{A}_2}{\osim}} \lctx'$.
      By appealing to the inductive hypothesis for $\p{A}_2$, we deduce that rewriting bisimilarity contains $\lframe{\p{A}_2}{\osim}$, and so $\octx \reduces\lframe{\p{A}_1}{\osim} \lctx'$.
      Similar reasoning for $\p{A}_1$ allows us to conclude that $\octx \reduces\osim \lctx'$, as required.

    \item
      Consider the case in which $\p{A} = \one$ and $\octx \lframe{\p{A}}{\osim} \one \oc \lctx' \reduces \lctx'$.
      Notice that $\octx \reduces\osim \lctx'$, as required.

    \item
      Consider the case in which $\p{A} = \n{A}_0$ and $\octx \lframe[\big]{\p{A}}{\osim} \dn \n{A}_0 \oc \atmL{\lctx}_L \oc \lctx'_0 \reduces \p{C} \oc \lctx'_0 = \lctx'$ because $\lfocus{}{\n{A}_0}{\atmL{\lctx}_L}{\p{C}}$.
      Because $\osim$ is a labeled bisimulation~\parencref{??}, we may appeal to immediate output bisimulation property after framing off $\dn \n{A}_0$ and deduce that $\octx \lframe[\big]{\p{A}}{(\Reduces\lframe{\atmL{\lctx}_L}{\osim})} \dn \n{A}_0 \oc \atmL{\lctx}_L \oc \lctx'_0$.
      Reduction is closed under framing, so $\octx \Reduces\lframe[\big]{(\dn \n{A} \oc \atmL{\lctx}_L)}{\osim} \dn \n{A} \oc \atmL{\lctx}_L \oc \lctx'_0$.
      We can insert the reduction $\dn \n{A} \oc \atmL{\lctx}_L \reduces \p{C}$ and arrive at $\octx \Reduces\lframe{\p{C}}{\osim} \p{C} \oc \lctx'_0 = \lctx'$.
      The proposition $\p{C}$ is a subformula of $\dn \n{A}$, so, by the inductive hypothesis, $\lframe{\p{C}}{\osim}$ is contained within $\osim$.
      It follows that $\octx \Reduces\osim \lctx'$, as required. 
    \end{itemize}

  \item[Emptiness bisimulation]
    Assume that $\octx \lframe{\p{A}}{\osim} \octxe$.
    As before, this is impossible.
  %
  \qedhere
  \end{description}
\end{proof}

% \begin{itemize}
% \item Assume that $\octx \lframe{\p{A}}{\osim} \lctx = \atmL{\lctx}'_L \oc \lctx' \oc \atmR{\lctx}'_R$;
%   we must show that $\octx \Reduces\lrframe[\big]{\atmL{\lctx}'_L}{(\lframe{\p{A}}{\osim} \union \osim)}{\atmR{\lctx}'_R} \lctx$.
% \item 
%   Assume that $\octx \lframe{(\p{A}_1 \fuse \p{A}_2)}{\osim} \lctx \reduces \p{A}_1 \oc \p{A}_2 \oc \lctx'_0 = \lctx'$;
%   we must show that $\octx \Reduces\mathrel{\bigl(\lframe{(\p{A}_1 \fuse \p{A}_2)}{\osim} \union \osim\bigr)} \lctx'$.
%   Notice that $\octx \reduces\lframe{\p{A}_1}{\lframe{\p{A}_2}{\osim}} \lctx'$.
%   By the inductive hypothesis on $\p{A}_1$ and $\p{A}_2$, respectively, $\octx \reduces\lframe{\p{A}_1}{\osim} \lctx'$, and then $\octx \reduces\osim \lctx'$.
% \item
%   Assume that $\octx \lframe{\one}{\osim} \lctx \reduces \lctx'$.
%   Notice that $\octx \reduces\osim \lctx'$.
% \item
%   Assume that $\octx \lframe{(\dn \n{A})}{\osim} (\dn \n{A}) \oc \atmL{\lctx}'_L \oc \lctx_0 = \lctx$ and $\lfocus{}{\n{A}}{\atmL{\lctx}'_L}{\p{C}}$ and $\lctx' = \p{C} \oc \lctx_9$.
%   Notice that $\octx \lframe[\big]{\dn \n{A}}{(\Reduces\lframe{\atmL{\lctx}'_L}{\osim})} \lctx$, and so $\octx \Reduces\reduces\lframe{\p{C}}{\osim} \lctx'$.
%   By the inductive hypothesis, $\octx \Reduces\reduces\osim \lctx'$.
% \end{itemize}


% \begin{theorem}
%   If $\simu{R}$ is a labeled bisimulation, then so is $\lframe{\lctx_L}{\simu{R}}$ for all $\lctx_L$.
% \end{theorem}
% %
% \begin{proof}
%   \begin{description}
%   \item[Immediate output bisimulation]
%     Assume that $\octx \lframe{\octx_L}{\simu{R}} \lctx = \atmL{\lctx}'_L \oc \lctx' \oc \atmR{\lctx}'_R$;
%     we must show that $\octx \Reduces\lrframe[\big]{\atmL{\lctx}'_L}{(\lframe{\p{A}}{\simu{R}} \union \simu{R})}{\atmR{\lctx}'_R} \lctx$.
%     \begin{itemize}
%     \item
%       Consider the case in which the context $\atmL{\lctx}'_L$ is nonempty -- that is, the case in which $\octx \lframe{\p{A}}{\simu{R}} \lctx = \p{A} \oc \atmL{\lctx}''_L \oc \lctx' \oc \atmR{\lctx}'_R$ and $\atmL{\lctx}'_L = \p{A} \oc \atmL{\lctx}''_L$, for some $\atmL{\lctx}''_L$.
%       Because $\simu{R}$ is a labeled bisimulation, we may appeal to its immediate output bisimulation property after framing off $\p{A}$ and deduce that $\octx \lframe[\big]{\p{A}}{(\Reduces\lrframe{\atmL{\lctx}''_L}{\simu{R}}{\atmR{\lctx}'_R})} \lctx$.
%       Reduction is closed under framing, 
      
%     \item
%       Consider the case in which the context $\atmL{\lctx}'_L$ is empty and $\lctx'$ is nonempty.
%       Because $\simu{R}$ is a labeled bisimulation, we may appeal to its immediate output bisimulation property after framing off $\p{A}$ and deduce that $\octx \lframe[\big]{\p{A}}{(\Reduces\rframe{\simu{R}}{\atmR{\lctx}'_R})} \lctx$.
%       Reduction is closed under framing, so we conclude that $\octx \Reduces\rframe{\lframe{\p{A}}{\simu{R}}}{\atmR{\lctx}'_R} \lctx$, as required.

%     \item
%       Consider the case in which the contexts $\atmL{\lctx}'_L$ and $\lctx'$ are empty. -- that is, the case in whihc $\p{A} = \atmR{a}$ and $\octx \lframe{\p{A}}{\simu{R}} \lctx = \atmR{\lctx}'_R = \atmR{a} \oc \atmR{\lctx}''_R$, for some $\atmR{\lctx}''_R$.
%       Because $\simu{R}$ is a labeled bisimulation, we may appeal to its immediate output bisimulation property after framing off $\p{A}$ and deduce that $\octx \lframe[\big]{\p{A}}{(\Reduces\rframe{\simu{R}}{\atmR{\lctx}''_R})} \lctx$.
%       Reduction is closed under framing, so $\octx \Reduces\rframe{\lframe{\p{A}}{\simu{R}}}{\atmR{\lctx}''_R} \lctx$.
%       After framing off $\atmR{\lctx}''_R$, we may appeal to the emptiness bisimulation property of $\simu{R}$ and deduce that $\octx \Reduces\rframe[\big]{(\Reduces\rframe{\simu{R}}{\atmR{a}})}{\atmR{\lctx}''_R} \lctx$.
%       Once again, reduction is closed under framing, so we conclude that $\octx \Reduces\rframe{\simu{R}}{\atmR{\lctx}'_R} \lctx$, as required.
%     \end{itemize}

%   \item[Immediate input bisimulation]
%     Assume that $\octx \lframe{\p{A}}{\simu{R}} \lctx$ and $\ireduces{\atmR{\lctx}_L \oc #1 \oc \atmL{\lctx}_R}{\lctx}{\lctx'}$;
%     we must show that $\atmR{\lctx}_L \oc \octx \oc \atmL{\lctx}_R \Reduces\mathrel{(\lframe{\p{A}}{\simu{R}} \union \simu{R})} \lctx'$.
%     \begin{itemize}
%     \item
%       Consider the case in which $\p{A} = \atmR{a}$ and does participate in the input transition -- that is, the case in which $\octx \lframe{\atmR{a}}{\simu{R}} \atmR{a} \oc \lctx_0 = \lctx$ and $\ireduces{\atmR{\lctx}_L \oc \atmR{a} \oc #1 \oc \atmL{\lctx}_R}{\lctx_0}{\lctx'}$, for some $\lctx_0$.
%       Because $\simu{R}$ is a labeled bisimulation, we may appeal to its immediate input bisimulation property and deduce that $\atmR{\lctx}_L \oc \octx \oc \atmL{\lctx}_R \Reduces\simu{R} \lctx'$, as required.
%     \item
%       Consider the case in which $\p{A}$ does not participate in the input transition -- that is, the case in which $\atmR{\lctx}_L$ is empty and $\octx \lframe{\p{A}}{\simu{R}} \p} \oc \lctx_0 = \lctx$ and $\ireduces{#1 \oc \atmL{\lctx}_R}{\lctx_0}{\lctx'_0}$ and $\lctx' = \p{A} \oc \lctx'_0$, for some $\lctx_0$ and $\lctx'_0$.
%       Because $\simu{R}$ is a labeled bisimulation, we may appeal to its immediate input bisimulation property after framing off $\p{A}$ and deduce that $\octx \oc \atmL{\lctx}_R \lframe{\p{A}}{(\Reduces\simu{R})} \lctx'$.
%       Reduction is closed under framing, so we conclude that $\octx \oc \atmL{\lctx}_R \Reduces\lframe{\p{A}}{\simu{R}} \lctx'$, as required.
%     \end{itemize}

%   \item[Reduction bisimulation]
%     Assume that $\octx \lframe{\p{A}}{\simu{R}}\reduces \lctx'$;
%     we must show that $\octx \Reduces\mathrel{(\lframe{\p{A}}{\simu{R}} \union \simu{R})} \lctx'$.
%     \begin{itemize}
%     \item
%       Consider the case in which the reduction arises from the $\simu{R}$-related component alone -- that is, the case in which $\octx \lframe{\p{A}}{(\simu{R}\reduces)} \lctx'$.
%       Because $\simu{R}$ is a labeled bisimulation, we may appeal to its reduction bisimulation property after framing off $\p{A}$ and deduce that $\octx \lframe{\p{A}}{(\Reduces\simu{R})} \lctx'$.
%       Reduction is closed under framing, so we conclude that $\octx \Reduces\lframe{\p{A}}{\simu{R}} \lctx'$.

%     \item
%       Consider the case in which the reduction arises from an input transition on the $\simu{R}$-related component -- that is, the case in which $\p{A} = \atmR{a}$ and $\octx \lframe{\atmR{a}}{\simu{R}} \atmR{a} \oc \lctx_0 = \lctx$ and $\ireduces{\atmR{a} \oc #1}{\lctx_0}{\lctx'}$, for some $\lctx_0$.
%       Because $\simu{R}$ is a labeled bisimulation, we may appeal to its immediate input bisimulation property and deduce that $\octx \Reduces\simu{R} \lctx'$, as required.

%     \item 
%       Consider the case in which the reduction arises from $\p{A}$ alone -- that is, the case in which $\octx = \p{A} \oc \octx_0$ and
%       \begin{itemize}
%       \item $\octx \lframe{(\p{A}_1 \fuse \p{A}_2)}{\simu{R}} \lctx = (\p{A}_1 \fuse \p{A}_2) \oc \lctx_0 \reduces \p{A}_1 \oc \p{A}_2 \oc \lctx_0 = \lctx'$, for some $\lctx_0$.
%         Then $\octx \reduces\lframe{\p{A}_1}{\lframe{\p{A}_2}{\simu{R}}} \lctx'$.
%       \end{itemize}
%     \end{itemize}

%   \item[Emptiness bisimulation]
%     Assume that $\octx \lframe{\p{A}}{\simu{R}} \octxe$.
%     This is, in fact, impossible because the empty context does not contain $\p{A}$.
%   \end{description}
% \end{proof}


\begin{theorem}
  If $\octx \osim \lctx$, then $\octx_L \oc \octx \oc \octx_R \osim \octx_L \oc \lctx \oc \octx_R$ for all $\octx_L$ and $\octx_R$.
\end{theorem}
\begin{proof}
  By induction on the structures of $\octx_L$ and $\octx_R$, appealing to \cref{??}.
\end{proof}


\begin{theorem}[Reduction closure]
  Let $\simu{R}$ be a rewriting bisimulation.
  Then $\simu{R}$ is reduction-closed: $\octx \simu{R}\Reduces \lctx'$ implies $\octx \Reduces\simu{R} \lctx'$.
\end{theorem}
%
\begin{proof}
  Reduction closure follows immediately as the trivial instance of either the output or input bisimulation properties.
\end{proof}




\subsection{Counterexample}


This definition is too fine, ruling out desirable equivalences.
For example, $e \oc b_0 \not\osim e$.
Suppose, for the sake of deriving a contradiction, that $e \oc b_0 \osim e$.
Because $e \oc b_0 \oc \atm{d} \Reduces \atm{z} \oc b'_0$, it follows from input bisimilarity that $e \oc \atm{d} \Reduces\miso \atm{z} \oc b'_0$.
So either $\atm{z} \oc b'_0 \osim e \oc \atm{d}$ or $\atm{z} \oc b'_0 \osim \atm{z}$.
The former is impossible because $\atm{z} \oc b'_0$ cannot produce $\atm{d}$ on the right\footnote{Nor, in fact, on the left.} and so violates output bisimilarity.

The latter is also impossible.
It has an output of $\atm{z}$ on the left of $\atm{z} \oc b'_0$, from which output bisimilarity yields $b'_0 \osim \octxe$.
From input bisimilarity, $b'_0 \oc \atm{a} \osim \atm{a}$ follows, for any $\atm{a}$.
And, that violates output bisimilarity because $b'_0 \oc \atm{a}$, which does not reduce, cannot match the left output that $\atm{a}$ makes.

The key feature of this counterexample is that atoms' lack of direction means that the output bisimilarity condition also applies to atoms intended to act as inputs ($\atm{d}$ and $\atm{a}$, for instance).


\section{Example of rewriting bisimilarity: \Aclp*{NFA}}

As a more elaborate example of rewriting bisimilarity, we can return to the encoding of \acp{NFA} proposed in \cref{??}.
Recall that 
\begin{equation*}
  \nfa{q} \defd
\end{equation*}
Like \ac{DFA} states, \ac{NFA} states that have equal encodings are bisimilar; unnlike \ac{DFA} states, bisimilar \ac{NFA} states do not have equal encodings. 
However, bisimilar \ac{NFA} states do have encodings that are rewriting-bisimilar.
In other words, the \ac{NFA} encoding preserves bisimilarity: $q \asim s$ if, and only if, $\nfa{q} \osim \nfa{s}$.

[...]

Before proving this statement, we need a few \lcnamecref{lem:a-succ-bisim,lem:final-bisim}.
%
\begin{lemma}\label{lem:nfa-reduces}
  For all states $q$:
  \begin{enumerate}[nosep]
  \item\label{enum:nfa-reduces-1} $\nfa{q} \nreduces$.
  \item\label{enum:nfa-reduces-2} If $\atmR{a} \oc \nfa{q} \Reduces \octx'$, then either $\atmR{a} \oc \nfa{q} = \octx'$ or $\atmR{a} \oc \nfa{q} \reduces \nfa{q}'_a = \octx'$ for some state $q'_a$ that $a$-succeeds $q$.
  \item\label{enum:nfa-reduces-3} If $\atmR{\emp} \oc \nfa{q} \Reduces \octx'$, then either: $\atmR{\emp} \oc \nfa{q} = \octx'$; $\atmR{\emp} \oc \nfa{q} \reduces \nfa{F}(q) = \octx'$; or $q$ is a final state and $\atmR{\emp} \oc \nfa{q} \reduces \nfa{F}(q) \reduces \octxe = \octx'$.
  \end{enumerate}
\end{lemma}
\begin{proof}
  Part~\ref{enum:nfa-reduces-1} is proved by inversion of a hypothetical rewriting of $\nfa{q}$.

  Part~\ref{enum:nfa-reduces-2} is proved by inversion of the given rewriting sequence:
  If the rewriting sequence is nontrivial, it must be $\atmR{a} \oc \nfa{q} \reduces \nfa{q}'_a \Reduces \octx'$ for some state $q'_a$ that $a$-succeeds $q$; by part~\ref{enum:nfa-reduces-1}, we deduce that $\octx' = \nfa{q}'_a$.
  Otherwise, if the rewriting sequence is trivial, the desired result is immediate.
\end{proof}
%
\noindent
These results hold only because ordered rewriting is weakly focused;
under an unfocused rewriting framework, $\nfa{q}$ would admit rewritings, such as $\nfa{q} \Reduces \atmR{\emp} \limp \nfa{F}(q)$, and $\atmR{a} \oc \nfa{q}$ would admit rewritings to contexts other than encodings of $a$-successors.


\begin{lemma}\label{lem:a-succ-bisim}
  If $\atmR{a} \oc \nfa{q} \Reduces\osim \nfa{q}'$, then $\nfa{q}'_a \osim \nfa{q}'$ for some state $q'_a$ that $a$-succeeds $q$.
\end{lemma}
\begin{proof}
  Assume that $\atmR{a} \oc \nfa{q} \Reduces\osim \nfa{q}'$.
  According to \cref{lem:nfa-reduces}, there are two cases: either
  \begin{enumerate*}[label=\emph{(\roman*)}]
  \item $\atmR{a} \oc \nfa{q} \osim \nfa{q}'$ or
  \item $\nfa{q}'_a \osim \nfa{q}'$ for some state $q'_a$ that $a$-succeeds $q$.
  \end{enumerate*}
  In the latter case, the desired result is immediate.

  In the former case, because the underlying \ac{NFA} is well-formed~\parencref{??}, $q$ has at least one $a$-successor;
  let $q'_a$ be one such successor.
  By definition of the encoding, $\atmR{a} \oc \nfa{q} \reduces \nfa{q}'_a$.
  Because rewriting bisimilarity is reduction-closed (\cref{thm:bisim-reduction-closure}), $\nfa{q}'_a \osim\secudeR \nfa{q}'$.
  States are encoded by latent\autocite{??} propositions (\cref{lem:nfa-reduces}), and so we may conclude that, in fact, $\nfa{q}'_a \osim \nfa{q}'$.
  %
  % \begin{itemize}
  % \item Consider the case in which the trace is trivial: $\atmR{a} \oc \nfa{q} \osim \nfa{q}'$.
  %   Because the underlying \ac{NFA} is well-formed~\parencref{??}, $q$ has at least one $a$-successor;
  %   let $q'_a$ be one such successor.
  %   By definition of the encoding, $\atmR{a} \oc \nfa{q} \reduces \nfa{q}'_a$.
  %   Because rewriting bisimilarity is reduction-closed (\cref{thm:bisim-reduction-closure}), $\nfa{q}'_a \osim\secudeR \nfa{q}'$.
  %   States are encoded by latent\autocite{??} propositions (\cref{??}), and so we may conclude that, in fact, $\nfa{q}'_a \osim \nfa{q}'$.
  %
  % \item Consider the case in which the trace contains at least one step.
  %   By inversion, that step corresponds to \iac{NFA} transition: $\atmR{a} \oc \nfa{q} \reduces \nfa{q}'_a \Reduces\osim \nfa{q}'$, for some state $q'_a$ that $a$-succeeds $q$.
  %   Once again, because states are encoded by latent propositions, the trace from $\nfa{q}'_a$ must be trivial.
  % \qedhere
  % \end{itemize}
\end{proof}
%
\begin{lemma}\label{lem:final-bisim}
  If $\atmR{\emp} \oc \nfa{q} \Reduces\osim \nfa{F}(s)$, then $q \in F$ if, and only if, $s \in F$.
\end{lemma}
%
\begin{proof}
  \begin{itemize}
  \item Consider the case in which the trace is trivial -- \ie, $\atmR{\emp} \oc \nfa{q} \osim \nfa{F}(s)$.
    By definition of the encoding, $\nfa{F}(q) \secuder \atmR{\emp} \oc \nfa{q} \osim \nfa{F}(s)$.
    $\nfa{F}(q) \osim\secudeR \nfa{F}(s)$
  \end{itemize}
\end{proof}


\begin{lemma}
  If $\nfa{F}(q) \Reduces\osim \nfa{F}(s)$, then $q \in F$ if, and only if, $s \in F$.
\end{lemma}
%
\begin{proof}
  Assume that $\nfa{F}(q) \Reduces\osim \nfa{F}(s)$ and $q \notin F$.
  By inversion, The trace can only be the trivial one, so $\nfa{F}(q) = \top$ and $\nfa{F}(s)$ are bisimilar.
  Suppose, for the sake of contradiction, that $s \in F$ and so $\nfa{F}(s) = \one$.
  Then $\top \osim \one$; hence, $\atmR{a} \oc \top \Reduces\osim \atmR{a}$ follows from the input bisimilarity property.
  But output bisimilarity implies $\atmR{a} \oc \top \Reduces\rframe{\osim}{\atmR{a}} \atmR{a}$, which is impossible because $\atmR{a} \oc \top$ cannot produce $\atmR{a}$ at its right end.
\end{proof}


\begin{theorem}[\Ac*{NFA} adequacy]
  Let $\aut{A} = (Q, ?, F)$ be \iac{NFA} over the input alphabet $\ialph$.
  Then, for all states $q$, $q'$, and $s$:
  \begin{enumerate}
  \item\label{enum:nfa-adequacy-1} $q \asim s$ if, and only if, $\nfa{q} \osim \nfa{s}$.
  \item\label{enum:nfa-adequacy-2} $q \asim\nfareduces[a]\asim q'$ if, and only if, $\atmR{a} \oc \nfa{q} \osim\reduces\osim \nfa{q}'$, for all input symbols $a \in \ialph$.
    Moreover, $q \asim\nfareduces[w]\asim q'$ if, and only if, $\atmR{w} \oc \nfa{q} \osim\Reduces\osim \nfa{q}'$, for all finite words $w \in \finwds{\ialph}$.
  \item\label{enum:nfa-adequacy-3} $q \in F$ if, and only if, $\atmR{\emp} \oc \nfa{q} \reduces \one$.
%   \item $q \asim\nfareduces[w]\asim q'$ if, and only if, $\atmR{w} \oc \nfa{q} \osim\Reduces\osim \nfa{q}'$, for all finite words $w$.
  \end{enumerate}
\end{theorem}
%
\begin{proof}
  Each part is proved in turn.
  The proof of part~\ref{enum:nfa-adequacy-2} depends on the proof of part~\ref{enum:nfa-adequacy-1}.
  \begin{enumerate}[parsep=0em, listparindent=\parindent]
  %% Part one
  \item
    We shall show that \ac{NFA} bisimilarity coincides with rewriting bisimilarity of encodings, proving each direction separately.
    \begin{itemize}[parsep=0em, listparindent=\parindent]
    \item
      To prove that bisimilar \ac{NFA} states have bisimilar encodings -- \ie, that $q \asim s$ implies $\nfa{q} \osim \nfa{s}$ -- we will show that the symmetric relation $\mathord{\simu{R}} = \Set{(\nfa{q}, \nfa{s}) \given q \asim s}$ is a labeled bisimulation up to reflexivity and, by \cref{thm:bisim-technique-up-to-refl}, is included in rewriting bisimilarity.
      \begin{description}
      \item[Immediate output bisimulation]
        Assume that $\nfa{q} \simu{R} \nfa{s} = \atmL{\lctx}'_L \oc \lctx' \oc \atmR{\lctx}'_R$; we must show that $\nfa{q} \Reduces\lrframe{\atmL{\lctx}'_L}{\reflc{\simu{R}}}{\atmR{\lctx}'_R} \nfa{s}$.
        By definition of the encoding, $\nfa{s}$ is a negative propostion and does not expose outputs.
        Therefore, $\atmL{\lctx}'_L$ and $\atmR{\lctx}'_R$ are empty and $\lctx'$ is $\nfa{s}$.
        The required $\nfa{q} \Reduces\lrframe{\atmL{\lctx}'_L}{\reflc{\simu{R}}}{\atmR{\lctx}'_R} \nfa{s}$ follows trivially.

      \item[Immediate input bisimulation]
        Assume that $\nfa{q} \simu{R} \nfa{s}$ and $\ireduces{\atmR{\lctx}_L \oc #1 \oc \atmL{\lctx}_R}{\nfa{s}}{\lctx'}$; we must show that $\atmR{\lctx}_L \oc \nfa{q} \oc \atmL{\lctx}_R \Reduces\reflc{\simu{R}} \lctx'$.
        Inversion of the input transition yields two cases.
        \begin{itemize}
        \item
          Consider the case in which the input transition is $\ireduces{\atmR{a} \oc #1}{\nfa{s}}{\nfa{s}'_a}$, where $s$ is $a$-succeeded by $s'_a$.
          Because $q$ and $s$ are bisimilar, there must exist an $a$-successor of $q$, say $q'_a$, that is bisimilar to $s'_a$.
          By definition of the encoding, we thus have $\atmR{a} \oc \nfa{q} \reduces \nfa{q}'_a$.
          So indeed, because $q'_a$ and $s'_a$ are bisimilar states, $\atmR{a} \oc \nfa{q} \Reduces\reflc{\simu{R}} \nfa{s}'_a$, as required.

        \item
          Consider the case in which the input transition is $\ireduces{\atmR{\emp} \oc #1}{\nfa{s}}{\nfa{F}(s)}$.
          Because $q$ and $s$ are bisimilar states, $\nfa{F}(q) = \nfa{F}(s)$~\parencref{??}.
          By definition of the encoding, $\atmR{\emp} \oc \nfa{q} \reduces \nfa{F}(q)$, and so, indeed, $\atmR{\emp} \oc \nfa{q} \Reduces\reflc{\simu{R}} \nfa{F}(s)$, as required.
        \end{itemize}

      \item[Reduction bisimulation]
        Assume that $\nfa{q} \simu{R} \nfa{s} \reduces \lctx'$.
        The reduction bisimulation property holds vacuously because states are encoded as latent propositions, and so $\nfa{s}$ does not reduce~\parencref{lem:nfa-latent}.

      \item[Emptiness bisimulation]
        Assume that $\nfa{q} \simu{R} \nfa{s} = \octxe$.
        The emptiness bisimulation property also holds vacuously because states are encoded as propositions, not empty contexts.
      \end{description}

    \item
      To prove the converse -- that states with bisimilar encodings are themselves bisimilar -- we will show that the relation $\mathord{\simu{R}} = \Set{(q,s) \given \nfa{q} \osim \nfa{s}}$, which relates states if they have rewriting-bisimilar encodings, is \iac{NFA} bisimulation and is therefore included in bisimilarity.

      Because rewriting bisimilarity is symmetric, so too is the relation $\simu{R}$.
      \begin{itemize}[parsep=0em, listparindent=\parindent]
      \item Let $q$ and $s$ be states with bisimilar encodings, and let $q'_a$ be an $a$-successor of $q$;
        we must exhibit a state $s'_a$ that $a$-succeeds $s$ and has an encoding that is bisimilar to that of $q'_a$.

        By definition of the encoding, $\atmR{a} \oc \nfa{q} \reduces \nfa{q}'_a$.
        Because $q$ and $s$ have bisimilar encodings, the input bisimulation property allows us to deduce that $\atmR{a} \oc \nfa{s} \Reduces\osim \nfa{q}'_a$.
        An appeal to \cref{lem:a-succ-bisim} provides exactly what is needed: a state $s'_a$ that $a$-succeeds $s$ and has an encoding bisimilar to that of $q'_a$.
        
      \item Let $q$ and $s$ be states with bisimilar encodings, and assume that $q$ is a final state;
        we must show that $s$ is also a final state.

        By definition of the encoding, $\atmR{\emp} \oc \nfa{q} \reduces \nfa{F}(q) = \one$.
        Because $q$ and $s$ have bisimilar encodings, it follows from input bisimilarity that $\atmR{\emp} \oc \nfa{s} \Reduces\osim \nfa{F}(q)$.
        An appeal to \cref{??} yields $\nfa{F}(q) = \nfa{F}(s)$, from which we deduce that $s$ is also a final state.
      \end{itemize}
      


      % Let $\simu{R}$ be the binary relation that relates two states if their encodings are rewriting-bisimilar -- that is, 
      % we shall show that $\simu{R}$ is \iac{NFA} bisimulation and therefore included in \ac{NFA} bisimilarity.
      % \begin{itemize}[listparindent=\parindent]
      % \item Let $q$ and $s$ be states with bisimilar encodings, and let $q'_a$ be an $a$-successor of $q$;
      %   we must exhibit a state $s'_a$ that $a$-succeeds $s$ and has an encoding that is bisimilar to that of $q'_a$.

      %   By definition of the encoding, $\atmR{a} \oc \nfa{q} \reduces \nfa{q}'_a$.
      %   Because $q$ and $s$ have bisimilar encodings, the input bisimilarity property allows us to deduce that $\atmR{a} \oc \nfa{s} \Reduces\miso \nfa{q}'_a$.
      %   An appeal to \cref{lem:??} provides exactly what is needed: a state $s'_a$ that $a$-succeeds $s$ and has an encoding bisimilar to that of $q'_a$.
        
      % \item Let $q$ and $s$ be states with bisimilar encodings, and assume that $q$ is a final state;
      %   we must show that $s$ is also a final state.

      %   By definition of the encoding, $\atmR{\emp} \oc \nfa{q} \reduces \nfa{F}(q) = \one$.
      %   Because $q$ and $s$ have bisimilar encodings, it follows from input bisimilarity that $\atmR{\emp} \oc \nfa{s} \Reduces\miso \nfa{F}(q)$.
      %   \begin{itemize}
      %   \item $\nfa{F}(s) \secuder \atmR{\emp} \oc \nfa{s} \miso \nfa{F}(q)$.
      %     Then $\nfa{F}(s) \miso\secudeR \nfa{F}(q)$.
      %   \item $\nfa{F}(s) \Reduces\miso \nfa{F}(q)$.
          
      %   \end{itemize}
      % \end{itemize}
      
    \end{itemize}

  % %% Part two
  % \item
  %   We would like to prove that $q \asim\nfareduces[a]\asim q'$ if, and only if, $\atmR{a} \oc \nfa{q} \osim\reduces\osim \nfa{q}'$.
  %   Because bisimilar states have bisimilar encodings (part~\ref{??}), rewriting bisimilarity is an atomic congruence (\cref{??}), and \ac{NFA} bisimilarity is reflexive (\cref{??}), it suffices to show:
  %   \begin{enumerate*}
  %   \item that $q \nfareduces[a] q'$ implies $\atmR{a} \oc \nfa{q} \reduces \nfa{q}'$; and
  %   \item that $\atmR{a} \oc \nfa{q} \osim\reduces\osim \nfa{q}'$ implies $q \nfareduces[a]\asim q'$.
  %   \end{enumerate*}
  %   %
  %   \begin{enumerate}
  %   \item
  %     Let $q'$ be an $a$-successor of state $q$.
  %     There exists, by definition of the encoding, a trace $\atmR{a} \oc \nfa{q} \reduces \nfa{q}'$.
  %   \item
  %     Assume that a trace $\atmR{a} \oc \nfa{q} \osim\reduces\osim \nfa{q}'$ exists.
  %     Because bisimlarity is, by definition, reduction-closed and transitive (\cref{??}), $\atmR{a} \oc \nfa{q} \Reduces\osim \nfa{q}'$.
      
  %   \end{enumerate}

  %% Part four
  \item
    We would like to prove that $q \asim\nfareduces[w]\asim q'$ if, and only if, $\rev{\atmR{w}} \oc \nfa{q} \osim\Reduces\osim \nfa{q}'$.
    Because bisimilar states have bisimilar encodings (part~\ref{enum:nfa-adequacy-1}), because rewriting bisimilarity is left-congruent (\cref{??}), reduction-closed (\cref{thm:bisim-reduction-closure}), and transitive (\cref{??}), and because \ac{NFA} bisimilarity is reflexive (\cref{??}), it suffices to show:
    \begin{enumerate*}
    \item that $q \nfareduces[w] q'$ implies $\rev{\atmR{w}} \oc \nfa{q} \Reduces \nfa{q}'$; and
    \item that $\rev{\atmR{w}} \oc \nfa{q} \Reduces\osim \nfa{q}'$ implies $q \nfareduces[w]\asim q'$.
    \end{enumerate*}
    %
    We prove these two statements in turn.
    %
    \begin{enumerate}
    \item
      That $q \nfareduces[w] q'$ implies $\rev{\atmR{w}} \oc \nfa{q} \Reduces \nfa{q}'$ can be proved by a straightforward induction over the structure of word $w$;
      we omit the details.

      % We shall prove that $q \nfareduces[w] q'$ implies $\rev{\atmR{w}} \oc \nfa{q} \Reduces \nfa{q}'$ by induction over the structure of word $w$.
      % \begin{itemize}
      % \item Consider the case of the empty word, $\emp$; we must show that $q = q'$ implies $\nfa{q} \Reduces \nfa{q}'$.
      %   Because the encoding is a function, this is immediate.
      % \item Consider the case of a nonempty word, $a \wc w$; we must show that $q \nfareduces[a]\nfareduces[w] q'$ implies $\rev{\atmR{w}} \oc \atmR{a} \oc \nfa{q} \Reduces \nfa{q}'$.
      %   Let $q'_a$ be a state that is an $a$-successor of state $q$ and is itself $w$-succeeded by state $q'$.
      %   There exists, by definition of the encoding, a trace
      %   \begin{equation*}
      %     \rev{\atmR{w}} \oc \atmR{a} \oc \nfa{q}
      %       \reduces \rev{\atmR{w}} \oc \nfa{q}'_a
      %       \Reduces \nfa{q}'
      %     \,,
      %   \end{equation*}
      %   with the trace's tail justified by an appeal to the inductive hypothesis.
      % \end{itemize}

    \item
      We shall prove that $\rev{\atmR{w}} \oc \nfa{q} \Reduces\osim \nfa{q}'$ implies $q \nfareduces[w]\asim q'$ by induction over the structure of word $w$.
      \begin{itemize}
      \item Consider the case of the empty word, $\emp$; we must show that $\nfa{q} \Reduces\osim \nfa{q}'$ implies $q \asim q'$.
        Because states are encoded by latent propositions (\cref{lem:nfa-reduces}), the given trace can only be the trivial one; thus, $\nfa{q} \osim \nfa{q}'$.
        An appeal to part~\ref{enum:nfa-adequacy-1} allows us to conclude that $q \asim q'$.
      \item Consider the case of a nonempty word, $a \wc w$; we must show that $\rev{\atmR{w}} \oc \atmR{a} \oc \nfa{q} \Reduces\osim \nfa{q}'$ implies $q \nfareduces[a]\nfareduces[w]\asim q'$.
        By inversion, the given trace must begin by inputting $a$:
        \begin{equation*}
          \rev{\atmR{w}} \oc \atmR{a} \oc \nfa{q} \reduces \rev{\atmR{w}} \oc \nfa{q}'_a \Reduces\osim \nfa{q}'
          \,,
        \end{equation*}
        where $q'_a$ is an $a$-successor of state $q$.
        An appeal to the inductive hypothesis on the trace's tail yields $q'_a \nfareduces[w]\asim q'$, and so the \ac{NFA} admits $q \nfareduces[a]\nfareduces[w]\asim q'$, as required.
      \end{itemize}
    \end{enumerate}

    \noindent
    Additionally, we would like to prove an apparently stronger statement: $q \asim\nfareduces[a]\asim q'$ if, and only if, $\atmR{a} \oc \nfa{q} \osim\reduces\osim \nfa{q}'$.
    From the above proof involving finite words, we know that $q \asim\nfareduces[a]\asim q'$ if, and only if, $\atmR{a} \oc \nfa{q} \osim\Reduces\osim \nfa{q}'$.
    Therefore, it suffices to show that the multi-step $\atmR{a} \oc \nfa{q} \osim\Reduces\osim \nfa{q}'$ is equivalent to the single-step $\atmR{a} \oc \nfa{q} \osim\reduces\osim \nfa{q}'$.
    \begin{itemize}
    \item 
      To prove the left-to-right direction, begin with the multi-step assumption that $\atmR{a} \oc \nfa{q} \osim\Reduces\osim \nfa{q}'$.
      Because rewriting bisimilarity is reduction-closed (\cref{thm:bisim-reduction-closure}) and transitive (\cref{??}), $\atmR{a} \oc \nfa{q} \Reduces\osim \nfa{q}'$.
      By \cref{lem:a-succ-bisim}, there exists a state $q'_a$ that $a$-succeeds $q$ such that
      % $q'_a$ is an $a$-successor of state $q$ and
      $\nfa{q}'_a \osim \nfa{q}'$.
      Then, by definition of the encoding (and reflexivity of rewriting bisimilarity (\cref{??})), it follows that $\atmR{a} \oc \nfa{q} \osim\reduces\osim \nfa{q}'$, as required.
      % States are encoded as latent propositions (\cref{??}), and so the trace from $\nfa{q}'_a$ can only be trivial: $\atmR{a} \oc \nfa{q}'_a \osim\reduces\osim \nfa{q}'$, as required.
    \item
      The right-to-left direction is trivial because a single step is a particular form of trace.
    \end{itemize}

  %% Part three
  \item
    We shall prove that the final states are exactly those states $q$ such that $\atmR{\emp} \oc \nfa{q} \reduces \one$.
    \begin{itemize}
    \item
      Let $q$ be a final state; accordingly, $\nfa{F}(q) = \one$.
      There exists, by definition of the encoding, a trace $\atmR{\emp} \oc \nfa{q} \reduces \nfa{F}(q) = \one$.
    \item
      Assume that $\atmR{\emp} \oc \nfa{q} \reduces \one$.
      The only step possible from $\atmR{\emp} \oc \nfa{q}$ is $\atmR{\emp} \oc \nfa{q} \reduces \nfa{F}(q)$, and so $\nfa{F}(q) = \one$.
      It follows that $q$ is a final state.
    %
    \qedhere
    \end{itemize}

  % %% Part four
  % \item
  %   We would like to prove that $q \asim\nfareduces[w]\asim q'$ if, and only if, $\rev{\atmR{w}} \oc \nfa{q} \osim\Reduces\osim \nfa{q}'$.
  %   Because bisimilar states have bisimilar encodings (part~\ref{??}), rewriting bisimilarity is left-congruent (\cref{??}), reduction-closed (\cref{??}), and transitive (\cref{??}), and \ac{NFA} bisimilarity is reflexive (\cref{??}), it suffices to show:
  %   \begin{enumerate*}
  %   \item that $q \nfareduces[w] q'$ implies $\rev{\atmR{w}} \oc \nfa{q} \Reduces \nfa{q}'$; and
  %   \item that $\rev{\atmR{w}} \oc \nfa{q} \Reduces\osim \nfa{q}'$ implies $q \nfareduces[w]\asim q'$.
  %   \end{enumerate*}

  %   Both statements can be established by induction over the structure of word $w$.
  %   The latter proof is slightly more involved and deserves a bit of explanation.
  %   \begin{itemize}
  %   \item Consider the case in which $w$ is the empty word; we must show that $\nfa{q} \Reduces\osim \nfa{q}'$ implies $q \asim q'$.
  %     With no input symbols available to $\nfa{q}$, the given trace can only be the trivial one, and so it suffices to show that $\nfa{q} \osim \nfa{q}'$ implies $q \asim q'$.
  %     This is exactly the content of the right-to-left direction of part~\ref{??}.

  %     Assume that $\nfa{q} \osim\Reduces\osim \nfa{q}'$.
  %     Because bisimilarity is reduction-closed (\cref{??}) and transitive (\cref{??}), $\nfa{q} \Reduces\osim \nfa{q}'$.
  %     With no input symbols available to $\nfa{q}$, this trace can only be the trivial one, and so $\nfa{q} \osim \nfa{q}'$.
  %     Because states with bisimilar encodings are themselves bisimilar (part~\ref{??}), it follows that $q \asim q'$.

  %   \item
  %     Consider the case of a nonempty word, $a \wc w$; we must show that $\rev{\atmR{w}} \oc \atmR{a} \oc \nfa{q} \Reduces\osim \nfa{q}'$ implies $q \nfareduces[a]\nfareduces[w]\asim q'$.
  %     %
  %     By inversion, the given trace must begin by inputting $\atmR{a}$: in other words, $\rev{\atmR{w}} \oc \atmR{a} \oc \nfa{q} \reduces \rev{\atmR{w}} \oc \nfa{q}'_a \Reduces\osim \nfa{q}'$, where $q'_a$ is an $a$-successor of $q$.
  %     Appealing to the inductive hypothesis on the trace's tail yields $q'_a \nfareduces[w]\asim q'$, and so $q \nfareduces[a]\nfareduces[w]\asim q'$, as required.

  %     Assume that $\atmR{w} \oc \atmR{a} \oc \nfa{q} \osim\Reduces\osim \nfa{q}'$.
  %     Because bisimilarity is reduction-closed (\cref{??}) and transitive (\cref{??}), $\atmR{w} \oc \atmR{a} \oc \nfa{q} \Reduces\osim \nfa{q}'$.
  %     By inversion, the trace must begin by inputting $\atmR{a}$: in other words, $\atmR{w} \oc \atmR{a} \oc \nfa{q} \reduces \atmR{w} \oc \nfa{q}'_a \Reduces\osim \nfa{q}'$, where $q'_a$ is an $a$-successor of $q$.
  %     Appealing to the inductive hypothesis on the trace's tail yields $q'_a \nfareduces[w]\asim q'$, and so $q \nfareduces[a]\nfareduces[w]\asim q'$, as required.
  %   \end{itemize}


  %   \begin{itemize}
  %   \item Assume that, up to bisimilarity, $q'$ is an $a$-successor of $q$ -- that is, that $q \misa s \nfareduces[a] s'_a \asim q'$ for some states $s$ and $s'_a$.
  %     By definition of the encoding, $\atmR{a} \oc \nfa{s} \reduces \nfa{s}'_a$.
  %     Because bisimilar states have bisimilar encodings (part~\ref{??}), $\nfa{q} \miso \nfa{s}$ and $\nfa{s}'_a \osim \nfa{q}'$.
  %     Moreover, because rewriting bisimilarity is an atomic congruence (\cref{??}), $\atmR{a} \oc \nffa{q} \miso \atmR{a} \oc \nfa{s}$.
  %     and putting everything together, we have $\atmR{a} \oc \nfa{q} \miso\reduces\osim \nfa{s}'_a$.

  %   \item Assume that, up to bisimilarity, $\atmR{a} \oc \nfa{q}$ rewrites to $\nfa{q}'$ -- that is, assume that $\atmR{a} \oc \nfa{q} \miso\reduces\osim \nfa{q}'$.
  %     An appeal to the reduction bisimilarity property yields $\atmR{a} \oc \nfa{q} \Reduces\miso\osim \nfa{q}'$.
  %     Because bisimilarity is a symmetric relation, it follows from \cref{lem:??} that there exists a state $q'_a$ that $a$-succeeds $q$ and has an encoding that is bisimilar to that of $q'$.
  %     Moreover, because states with bisimilar encodings are themselves bisimilar (part~\ref{??}) and because bisimilarity is reflexive, $q \misa\nfareduces[a]\asim q'$.
  %   \end{itemize}

  % \item
  % \item 
  %   \begin{itemize}
  %   \item We must show that $q \misa\asim q'$ if, and only if, $\nfa{q} \miso\Reduces\osim \nfa{q}'$.
  %   \item We must show that $q \misa\nfareduces[a]\nfareduces[w]\asim q'$ if, and only if, $\atmR{w} \oc \atmR{a} \oc \nfa{q} \miso\Reduces\osim \nfa{q}'$.
      
  %   \end{itemize}
  \end{enumerate}

  % Let $\simu{R}$ be the binary relation on ordered contexts such that $\octx$ and $\lctx$ are $\simu{R}$-related if they are equal to the encodings of a pair of bisimilar states -- that is, $\mathord{\simu{R}} = \set{(\octx, \lctx) \given \exists q,s \in Q.\, (\octx = \nfa{q}) \land (q \asim s) \land (\nfa{s} = \lctx)}$.
  % \begin{itemize}
  % \item Let $q$ and $s$ be bisimilar states, and assume that $\ireduces{\atmR{\lctx}_L \oc #1 \oc \atmL{\lctx}_R}{\nfa{q}}{\lctx'}$.
  %   By inversion, there are two cases; in either case, the context $\atmL{\lctx}_R$ must be empty.
  %   \begin{itemize}
  %   \item Consider the case in which the context $\atmR{\lctx}_L$ is a single atom $\atmR{\emp}$ and $\lctx' = \nfa{F}(q)$.
  %     By the encoding's construction, $\ireduces{\atmR{\emp} \oc #1}{\nfa{s}}{\nfa{F}(s)}$.
  %     And, because states $q$ and $s$ are bisimilar, the two are both final or both nonfinal states.
  %     $\nfa{F}(s)$
  %   \item  or $\atmR{a}$ for some input symbol $a$.
  %   \end{itemize}
  % \end{itemize}
  

  % Let $\simu{R}$ be the binary relation on states such that $q$ and $s$ are $\simu{R}$-related if their encodings are ordered bisimilar -- that is, $\mathord{\simu{R}} = \set{(q, s) \given \nfa{q} \osim \nfa{s}}$.
  % \begin{itemize}
  % \item Assume that $s \simu{R}^{-1} q \nfareduces[a] q'_a$.
  %   Because $q'_a$ is an $a$-successor of $q$, there exists a trace $\atmR{a} \oc \nfa{q} \Reduces \nfa{q}'_a$.
  %   Because $q$ and $s$ have bisimilar encodings, it then follows from the input bisimilarity property that $\atmR{a} \oc \nfa{s} \Reduces\miso \nfa{q}'_a$.
  %   By inversion, there are two cases
  %   \begin{itemize}
  %   \item $\nfa{s}'_a \secuder \atmR{a} \oc \nfa{s} \miso \nfa{q}'_a$, so $\nfa{q}'_a \Reduces\osim \nfa{s}'_a$
  %   \item $\atmR{a} \oc \nfa{s} \reduces \nfa{s}'_a \miso \nfa{q}'_a$
  %   \end{itemize}
  %   By inversion of this trace, there must exist a state $s'_a$ that is an $a$-successor of $s$ and has an encoding that is bisimilar to the encoding of $q'_a$ -- in other words, $s \nfareduces[a] s'_a \simu{R}^{-1} q'_a$.

  % \item Assume that $s \simu{R}^{-1} q \in F$.
  %   With $q$ being a final state, there exists a trace $\atmR{\emp} \oc \nfa{q} \Reduces \octxe$.
  %   Because $q$ and $s$ have bisimilar encodings, $\atmR{\emp} \oc \nfa{s} \Reduces\miso \octxe$.
  %   By inversion of this trace, $\nfa{F}(s)$ is bisimilar to the empty context.
  %   That is impossible if $s \notin F$, so $s$ must be a final state, like $q$.
  % \end{itemize}

  % To establish the completness of our \ac{NFA} encoding with respect to bisimularity, it then suffices to show that ordered bisimularity contains the relation $\simu{R}$.
  % Appealing to the preceding proof technique for ordered bisimilarity\parencref{thm:ord-bisim-technique}, we need only establish that $\simu{R}$ has immediate output bisimulation, immediate input bisimulation, reduction bisimulation, and emptiness bisimulation properties.

  % Only the immediate input bisimulation and reduction bisimulation conditions apply to the relation $\simu{R}$.
  % \begin{description}
  % \item[Immediate input bisimulation]
  %   Assume that $\lctx \simu{R} \octx$ and $\ireduces{\atmR{\lctx}_L \oc #1 \oc \atmL{\lctx}_R}{\lctx}{\lctx'}$;
  %   we must show that $\atmR{\lctx}_L \oc \octx \oc \atmL{\lctx}_R \Reduces\refl*{\simu{R}}^{-1} \lctx'$.

  %   Inversion allows us to deduce $\lctx = \nfa{q}$ and $\octx = \nfa{s}$ for some states $q$ and $s$ such that $q \asim s$.
  %   Examining the encoding, we see that there are two possible input transitions from $\nfa{q}$.
  %   \begin{itemize}
  %   \item Consider the input transition $\ireduces{\atmR{a} \oc #1}{\nfa{q}}{\nfa{q}'_a}$, with $a \in \ialph$ and $q \nfareduces[a] q'_a$ -- that is, $\atmR{\lctx}_L = \atmR{a}$; $\atmL{\lctx}_R = \octxe$; and $\lctx' = \nfa{q}'_a$.
  %     We must show that $\atmR{a} \oc \nfa{s} \Reduces\refl*{\simu{R}}^{-1} \nfa{q}'_a$.

  %     Because $q$ and $s$ are bisimilar states, $s \nfareduces[a] s'_a \misa q'_a$ for some state $s'_a$.
  %     Recall from \cref{thm:nfa-encoding-reduces} that the encoding of \acp{NFA} is complete with respect to input transitions; so, $\atmR{a} \oc \nfa{s} \reduces \nfa{s}'_a$.
  %     As $q'_a$ and $s'_a$ are bisimilar states, we conclude that $\atmR{a} \oc \nfa{s} \reduces\refl*{\simu{R}}^{-1} \nfa{q}'_a$, as required.

  %   \item Consider the input transition $\ireduces{\atmR{\emp} \oc #1}{\nfa{q}}{\octxe}$ when $q$ is a final state -- that is, $\atmR{\lctx}_L = \atmR{\emp}$ and $\atmL{\lctx}_R = \lctx' = \octxe$.
  %     We must show that $\atmR{\emp} \oc \nfa{s} \Reduces\refl*{\simu{R}}^{-1} \octxe$.

  %     Because $q$ and $s$ are bisimilar states, $s$ must also be a final state.
  %     Recall from \cref{thm:nfa-encoding-reduces} that the encoding of \acp{NFA} is complete with respect to input transitions; so, $\atmR{\emp} \oc \nfa{s} \reduces \octxe$.
  %     We conclude that $\atmR{\emp} \oc \nfa{s} \reduces\refl*{\simu{R}}^{-1} \octxe$, as required.
  %   \end{itemize}
  % %
  % \item[Reduction bisimulation]
  % \end{description}
\end{proof}


% \begin{proof}
%   The parts are proved in order, with parts [...] depending on part [...].
%   \begin{enumerate}
%   \item To prove that bisimilar states are exactly those states that have bisimilar encodings, we take each direction in turn.
%     \begin{itemize}
%     \item First, we will prove that bisimilar states have bisimilar encodings.
%       Let $\simu{R}$ be the binary relation that relates two states' encodings if their underlying states are \ac{NFA}-bisimilar -- that is, $\mathord{\simu{R}} = \set{(\nfa{q}, \nfa{s}) \given q \asim s}$; we shall show that $\simu{R}$ satisfies the conditions of \cref{thm:??} and is therefore included in rewriting bisimilarity.


%     \item Conversely, we will now prove that states that have bisimilar encodings are themselves bisimilar.
%       Let $\simu{R}$ be the binary relation that relates two states if their encodings are rewriting-bisimilar -- that is, $\mathord{\simu{R}} = \set{(q,s) \given \nfa{q} \osim \nfa{s}}$;
%       we shall show that $\simu{R}$ is \iac{NFA} bisimulation and therefore included in \ac{NFA} bisimilarity.
%       \begin{itemize}[listparindent=\parindent]
%       \item Let $q$ and $s$ be states with bisimilar encodings, and let $q'_a$ be an $a$-successor of $q$;
%         we must exhibit a state $s'_a$ that $a$-succeeds $s$ and has an encoding that is bisimilar to that of $q'_a$.

%         By definition of the encoding, $\atmR{a} \oc \nfa{q} \reduces \nfa{q}'_a$.
%         Because $q$ and $s$ have bisimilar encodings, the input bisimilarity property allows us to deduce that $\atmR{a} \oc \nfa{s} \Reduces\miso \nfa{q}'_a$.
%         An appeal to \cref{lem:??} provides exactly what is needed: a state $s'_a$ that $a$-succeeds $s$ and has an encoding bisimilar to that of $q'_a$.
        
%       \item Let $q$ and $s$ be states with bisimilar encodings, and assume that $q$ is a final state;
%         we must show that $s$ is also a final state.

%         By definition of the encoding, $\atmR{\emp} \oc \nfa{q} \reduces \nfa{F}(q) = \one$.
%         Because $q$ and $s$ have bisimilar encodings, it follows from input bisimilarity that $\atmR{\emp} \oc \nfa{s} \Reduces\miso \nfa{F}(q)$.
%         \begin{itemize}
%         \item $\nfa{F}(s) \secuder \atmR{\emp} \oc \nfa{s} \miso \nfa{F}(q)$.
%           Then $\nfa{F}(s) \miso\secudeR \nfa{F}(q)$.
%         \item $\nfa{F}(s) \Reduces\miso \nfa{F}(q)$.
          
%         \end{itemize}
%       \end{itemize}
      
%     \end{itemize}

%   \item 
%     \begin{itemize}
%     \item Assume that, up to bisimilarity, $q'$ is an $a$-successor of $q$ -- that is, that $q \misa s \nfareduces[a] s'_a \asim q'$ for some states $s$ and $s'_a$.
%       By definition of the encoding, $\atmR{a} \oc \nfa{s} \reduces \nfa{s}'_a$.
%       Because bisimilar states have bisimilar encodings (part~\ref{??}), $\nfa{q} \miso \nfa{s}$ and $\nfa{s}'_a \osim \nfa{q}'$.
%       Moreover, because rewriting bisimilarity is an atomic congruence (\cref{??}), $\atmR{a} \oc \nffa{q} \miso \atmR{a} \oc \nfa{s}$.
%       and putting everything together, we have $\atmR{a} \oc \nfa{q} \miso\reduces\osim \nfa{s}'_a$.

%     \item Assume that, up to bisimilarity, $\atmR{a} \oc \nfa{q}$ rewrites to $\nfa{q}'$ -- that is, assume that $\atmR{a} \oc \nfa{q} \miso\reduces\osim \nfa{q}'$.
%       An appeal to the reduction bisimilarity property yields $\atmR{a} \oc \nfa{q} \Reduces\miso\osim \nfa{q}'$.
%       Because bisimilarity is a symmetric relation, it follows from \cref{lem:??} that there exists a state $q'_a$ that $a$-succeeds $q$ and has an encoding that is bisimilar to that of $q'$.
%       Moreover, because states with bisimilar encodings are themselves bisimilar (part~\ref{??}) and because bisimilarity is reflexive, $q \misa\nfareduces[a]\asim q'$.
%     \end{itemize}

%   \item
%   \item 
%     \begin{itemize}
%     \item We must show that $q \misa\asim q'$ if, and only if, $\nfa{q} \miso\Reduces\osim \nfa{q}'$.
%     \item We must show that $q \misa\nfareduces[a]\nfareduces[w]\asim q'$ if, and only if, $\atmR{w} \oc \atmR{a} \oc \nfa{q} \miso\Reduces\osim \nfa{q}'$.
      
%     \end{itemize}
%   \end{enumerate}

%   Let $\simu{R}$ be the binary relation on ordered contexts such that $\octx$ and $\lctx$ are $\simu{R}$-related if they are equal to the encodings of a pair of bisimilar states -- that is, $\mathord{\simu{R}} = \set{(\octx, \lctx) \given \exists q,s \in Q.\, (\octx = \nfa{q}) \land (q \asim s) \land (\nfa{s} = \lctx)}$.
%   \begin{itemize}
%   \item Let $q$ and $s$ be bisimilar states, and assume that $\ireduces{\atmR{\lctx}_L \oc #1 \oc \atmL{\lctx}_R}{\nfa{q}}{\lctx'}$.
%     By inversion, there are two cases; in either case, the context $\atmL{\lctx}_R$ must be empty.
%     \begin{itemize}
%     \item Consider the case in which the context $\atmR{\lctx}_L$ is a single atom $\atmR{\emp}$ and $\lctx' = \nfa{F}(q)$.
%       By the encoding's construction, $\ireduces{\atmR{\emp} \oc #1}{\nfa{s}}{\nfa{F}(s)}$.
%       And, because states $q$ and $s$ are bisimilar, the two are both final or both nonfinal states.
%       $\nfa{F}(s)$
%     \item  or $\atmR{a}$ for some input symbol $a$.
%     \end{itemize}
%   \end{itemize}
  

%   Let $\simu{R}$ be the binary relation on states such that $q$ and $s$ are $\simu{R}$-related if their encodings are ordered bisimilar -- that is, $\mathord{\simu{R}} = \set{(q, s) \given \nfa{q} \osim \nfa{s}}$.
%   \begin{itemize}
%   \item Assume that $s \simu{R}^{-1} q \nfareduces[a] q'_a$.
%     Because $q'_a$ is an $a$-successor of $q$, there exists a trace $\atmR{a} \oc \nfa{q} \Reduces \nfa{q}'_a$.
%     Because $q$ and $s$ have bisimilar encodings, it then follows from the input bisimilarity property that $\atmR{a} \oc \nfa{s} \Reduces\miso \nfa{q}'_a$.
%     By inversion, there are two cases
%     \begin{itemize}
%     \item $\nfa{s}'_a \secuder \atmR{a} \oc \nfa{s} \miso \nfa{q}'_a$, so $\nfa{q}'_a \Reduces\osim \nfa{s}'_a$
%     \item $\atmR{a} \oc \nfa{s} \reduces \nfa{s}'_a \miso \nfa{q}'_a$
%     \end{itemize}
%     By inversion of this trace, there must exist a state $s'_a$ that is an $a$-successor of $s$ and has an encoding that is bisimilar to the encoding of $q'_a$ -- in other words, $s \nfareduces[a] s'_a \simu{R}^{-1} q'_a$.

%   \item Assume that $s \simu{R}^{-1} q \in F$.
%     With $q$ being a final state, there exists a trace $\atmR{\emp} \oc \nfa{q} \Reduces \octxe$.
%     Because $q$ and $s$ have bisimilar encodings, $\atmR{\emp} \oc \nfa{s} \Reduces\miso \octxe$.
%     By inversion of this trace, $\nfa{F}(s)$ is bisimilar to the empty context.
%     That is impossible if $s \notin F$, so $s$ must be a final state, like $q$.
%   \end{itemize}

%   To establish the completness of our \ac{NFA} encoding with respect to bisimularity, it then suffices to show that ordered bisimularity contains the relation $\simu{R}$.
%   Appealing to the preceding proof technique for ordered bisimilarity\parencref{thm:ord-bisim-technique}, we need only establish that $\simu{R}$ has immediate output bisimulation, immediate input bisimulation, reduction bisimulation, and emptiness bisimulation properties.

%   Only the immediate input bisimulation and reduction bisimulation conditions apply to the relation $\simu{R}$.
%   \begin{description}
%   \item[Immediate input bisimulation]
%     Assume that $\lctx \simu{R} \octx$ and $\ireduces{\atmR{\lctx}_L \oc #1 \oc \atmL{\lctx}_R}{\lctx}{\lctx'}$;
%     we must show that $\atmR{\lctx}_L \oc \octx \oc \atmL{\lctx}_R \Reduces\refl*{\simu{R}}^{-1} \lctx'$.

%     Inversion allows us to deduce $\lctx = \nfa{q}$ and $\octx = \nfa{s}$ for some states $q$ and $s$ such that $q \asim s$.
%     Examining the encoding, we see that there are two possible input transitions from $\nfa{q}$.
%     \begin{itemize}
%     \item Consider the input transition $\ireduces{\atmR{a} \oc #1}{\nfa{q}}{\nfa{q}'_a}$, with $a \in \ialph$ and $q \nfareduces[a] q'_a$ -- that is, $\atmR{\lctx}_L = \atmR{a}$; $\atmL{\lctx}_R = \octxe$; and $\lctx' = \nfa{q}'_a$.
%       We must show that $\atmR{a} \oc \nfa{s} \Reduces\refl*{\simu{R}}^{-1} \nfa{q}'_a$.

%       Because $q$ and $s$ are bisimilar states, $s \nfareduces[a] s'_a \misa q'_a$ for some state $s'_a$.
%       Recall from \cref{thm:nfa-encoding-reduces} that the encoding of \acp{NFA} is complete with respect to input transitions; so, $\atmR{a} \oc \nfa{s} \reduces \nfa{s}'_a$.
%       As $q'_a$ and $s'_a$ are bisimilar states, we conclude that $\atmR{a} \oc \nfa{s} \reduces\refl*{\simu{R}}^{-1} \nfa{q}'_a$, as required.

%     \item Consider the input transition $\ireduces{\atmR{\emp} \oc #1}{\nfa{q}}{\octxe}$ when $q$ is a final state -- that is, $\atmR{\lctx}_L = \atmR{\emp}$ and $\atmL{\lctx}_R = \lctx' = \octxe$.
%       We must show that $\atmR{\emp} \oc \nfa{s} \Reduces\refl*{\simu{R}}^{-1} \octxe$.

%       Because $q$ and $s$ are bisimilar states, $s$ must also be a final state.
%       Recall from \cref{thm:nfa-encoding-reduces} that the encoding of \acp{NFA} is complete with respect to input transitions; so, $\atmR{\emp} \oc \nfa{s} \reduces \octxe$.
%       We conclude that $\atmR{\emp} \oc \nfa{s} \reduces\refl*{\simu{R}}^{-1} \octxe$, as required.
%     \end{itemize}
%   %
%   \item[Reduction bisimulation]
%   \end{description}
% \end{proof}

% \begin{theorem}
%   If $\nfa{q} \osim \nfa{s}$, then $q \asim s$.
% \end{theorem}
% %
% \begin{proof}
%   Let $\simu{R}$ be the binary relation on states such that $q \simu{R} s$ exactly when $\nfa{q} \osim \nfa{s}$.
%   We will show that $\simu{R}$ is \iacs{NFA} bisimulation.

%   Among other properties, we must show that $\simu{R}$ simulates inputs.
%   Assume that $\nfa{s} \miso \nfa{q}$ and $q \nfareduces[a] q'$; we must show that $s \nfareduces[a] s'$ for some $s'$ such that $\nfa{s}' \miso \nfa{q}'$.
%   Because $\atmR{a} \oc \nfa{q} \Reduces \nfa{q}'$, it follows by input bisimilarity that $\atmR{a} \oc \nfa{s} \Reduces\miso \nfa{q}'$.
%   There are two cases, according to the structure of the reduction sequence from $\atmR{a} \oc \nfa{s}$.
%   \begin{itemize}
%   \item If the reduction sequence is trivial, then $\atmR{a} \oc \nfa{s} \miso \nfa{q}'$.
%     Because the transition relation is left-total, $s \nfareduces[a] s'$ for some state $s'$.
%     It follows that $\atmR{a} \oc \nfa{s} \Reduces \nfa{s}'$, and so, by input bisimilarity, $\nfa{q}' \Reduces\osim \nfa{s}'$.
%     However, $\nfa{q}' \longarrownot\reduces$, allowing us to conclude that $\nfa{q}' \osim \nfa{s}'$.
%   \item If the reduction sequence is nontrivial, then $\atmR{a} \oc \nfa{s} \reduces\Reduces\miso \nfa{q}'$.
%     Then
%     \begin{equation*}
%       \with_{s^* \mid s \nfareduces[a] s^*} \nfa{s}^* \Reduces\miso \nfa{q}'
%     \end{equation*}
%     It follows that $\with_{s^* \in S} \nfa{s}^* \miso \nfa{q}$ where $S$ is a subset of the $a$-successors of state $s$.
%     Because bisimilarity is reduction-closed, $\nfa{s}^* \miso \nfa{q}'$ for each $s^* \in S$.

%     How do we know that the subset $S$ is nonempty?
%     In other words, what happens if $\top \miso \nfa{q}'$?
%   \end{itemize}

%   Assume that $\nfa{s} \miso \nfa{q}$ and $q$ is a final state;
%   we must show that $s$ is also a final state.
%   Because $q$ is final, $\atmR{\emp} \oc \nfa{q} \Reduces \octxe$.
%   By input bisimilarity, $\atmR{\emp} \oc \nfa{s} \Reduces\miso \octxe$.
%   Choose a fresh atom $\atmR{x}$.
%   It follows by emptiness bisimilarity that $\atmR{x} \oc \atmR{\emp} \oc \nfa{s} \Reduces\rframe{\osim}{\atmR{x}}^{-1} \atmR{x}$.
%   However, $\atmR{x} \oc \atmR{\emp} \oc \nfa{s}$ exposes $\atmR{x}$ on the right only if $s$ is also a final state.
% \end{proof}


\section{Example of rewriting bisimilarity: Binary counter}

For a further application of rewriting bisimilarity, we can revisit the binary counter of \cref{??}.
Under a message-passing interpretation of ordered rewriting, the definitions of $e$, $b_0$, $b_1$, and $b'_0$ are the same as in \cref{??}, but with directions consistently assigned to the uninterpreted, positive atomic propositions:
\begin{equation*}
  \begin{lgathered}
    e \defd (e \fuse b_1 \pmir \atmL{i}) \with (\atmR{z} \pmir \atmL{d}) \\
    b_0 \defd (\up \dn b_1 \pmir \atmL{i}) \with (\atmL{d} \fuse b'_0 \pmir \atmL{d}) \\
    b'_0 \defd (\atmR{z} \limp \atmR{z}) \with (\atmR{s} \limp b_1 \fuse \atmR{s}) \\
    b_1 \defd (\atmL{i} \fuse b_0 \pmir \atmL{i}) \with (b_0 \fuse \atmR{s} \pmir \atmL{d})
  \end{lgathered}
\end{equation*}
The increment and decrement messages, $\atmL{i}$ and $\atmL{d}$, are left-directed, whereas the zero and successor messages, $\atmR{z}$ and $\atmR{s}$, are right-directed.


\begin{inferences}
  \infer{\aval{e}{0}}{}
  \and
  \infer{\aval{\octx \oc b_0}{2n}}{
    \aval{\octx}{n}}
  \and
  \infer{\aval{\octx \oc b_1}{2n+1}}{
    \aval{\octx}{n}}
  \\
  \infer{\ainc{e}{0}}{}
  \and
  \infer{\ainc{\octx \oc b_0}{2n}}{
    \ainc{\octx}{n}}
  \and
  \infer{\ainc{\octx \oc b_1}{2n+1}}{
    \ainc{\octx}{n}}
  \and
  \infer{\ainc{\octx \oc \atmL{i}}{n+1}}{
    \ainc{\octx}{n}}
  \\
  \infer{\ainc{e \fuse b_1}{1}}{}
  \and
  \infer{\ainc{\octx \oc (i \fuse b_0)}{2(n+1)}}{
    \ainc{\octx}{n}}
  \\
  \infer{\adec{\octx \oc \atmL{d}}{n}}{
    \ainc{\octx}{n}}
  \and
  \infer{\adec{\atmR{z}}{0}}{}
  \and
  \infer{\adec{\octx \oc \atmR{s}}{n+1}}{
    \ainc{\octx}{n}}
  \and
  \infer{\adec{\octx \oc b'_0}{2n}}{
    \adec{\octx}{n}}
  \\
  \infer{\adec{\octx \oc (\atmL{d} \fuse b'_0)}{2n}}{
    \ainc{\octx}{n}}
  \and
  \infer{\adec{\octx \oc (b_1 \fuse \atmR{s})}{2n+2}}{
    \ainc{\octx}{n}}
  \and
  \infer{\adec{\octx \oc (b_0 \fuse \atmR{s})}{2n+1}}{
    \ainc{\octx}{n}}
\end{inferences}

\begin{theorem}[Big-step adequacy of increments]\label{thm:msg-inc-big-adequacy}
  If $\ainc{\octx}{n}$, then:
  \begin{itemize}[nosep]
  \item $\octx \Reduces\aval{}{n'}$ if, and only if, $n = n'$; and
  \item $\octx \Reduces \atmL{\octx}'_L \oc \octx' \oc \atmR{\octx}'_R$ only if $\ainc{\octx'}{n}$ and both $\atmL{\octx}'_L$ and $\atmR{\octx}'_R$ are empty.
  \end{itemize}
\end{theorem}

\begin{theorem}[Big-step adequacy of decrements]\label{thm:msg-dec-big-adequacy}
  If $\adec{\octx}{n}$, then:
  \begin{itemize}[nosep]
  \item $\octx \Reduces \atmR{z}$ if $n = 0$;
  \item $\octx \Reduces \octx' \oc \atmR{s}$ for some $\ainc{\octx'}{n-1}$ if $n > 0$; and
  \item $\octx \Reduces \atmL{\octx}'_L \oc \octx' \oc \atmR{\octx}'_R$ only if $\atmL{\octx}'_L = \octxe$ and either:
    \begin{itemize}[nosep]
    \item $\adec{\octx'}{n}$ and $\atmR{\octx}'_R = \octxe$;
    \item $n = 0$ and $\octx' = \octxe$ and $\atmR{\octx}'_R = \atmR{z}$; or
    \item $n > 0$ and $\ainc{\octx'}{n-1}$ and $\atmR{\octx}'_R = \atmR{s}$.
    \end{itemize}
  \end{itemize}
\end{theorem}

Compare with:
\begin{theorem}[Big-step adequacy of decrements]
  If $\adec{\octx}{n}$, then:
  \begin{itemize}[nosep]
  \item $\octx \Reduces \atmR{z}$ if $n = 0$;
  \item $\octx \Reduces \octx' \oc \atmR{s}$ for some $\ainc{\octx'}{n-1}$ if $n > 0$;
  \item $\octx \Reduces \atmR{z}$ only if $n = 0$; and
  \item $\octx \Reduces \octx' \oc \atmR{s}$ only if $n > 0$ and $\ainc{\octx'}{n-1}$.
  \end{itemize}
\end{theorem}

\subsection{Counters with equal denotations are bisimilar}

States with equal denotations are bisimilar, and conversely, bimsimilar states have equal denotations.
\begin{theorem}\leavevmode
  If $\ainc{\octx}{n}$ and $\ainc{\lctx}{n'}$, then $\octx \osim \lctx$ if, and only if, $n = n'$.
  Similarly, if $\adec{\octx}{n}$ and $\adec{\lctx}{n'}$, then $\octx \osim \lctx$ if, and only if, $n = n'$.
\end{theorem}
\begin{proof}
  We prove each direction separately.
  \begin{itemize}
  \item
    To prove that states with equal denotations are bisimilar, consider the relation $\simu{R}$ given by 
    % Let $\simu{R}$ be a relation on states with equal denotations: 
  \begin{equation*}
    \mathord{\simu{R}}
    =
    \Set{(\octx, \lctx) \given \exists n \in \nats.\,(\ainc{\octx}{n}) \land (\ainc{\lctx}{n})}
    \union
    \Set{(\octx, \lctx) \given \exists n \in \nats.\,(\adec{\octx}{n}) \land (\adec{\lctx}{n})}
    \,.
  \end{equation*}
  We shall show that $\simu{R}$ progresses to its reflexive closure and then conclude, by \cref{??}, that $\simu{R}$ is contained within rewriting bisimilarity.
  \begin{description}
  \item[Immediate output bisimulation]
    Assume that $\octx \simu{R} \lctx = \atmL{\lctx}'_L \oc \lctx' \oc \atmR{\lctx}'_R$.
    According to big-step adequacy of decrements~\parencref{thm:msg-dec-big-adequacy}, there are three cases that derive from $\lctx$.
    \begin{itemize}
    \item 
      Consider the case in which $\adec{\octx}{0}$ and $\lctx = \adec{\atmR{z}}{0}$.
      According to big-step adequacy of decrements~\parencref{thm:msg-dec-big-adequacy}, $\octx \Reduces \atmR{z}$ and so, indeed, $\octx \Reduces\rframe{\reflc{\simu{R}}}{\atmR{z}} \atmR{z}$.

    \item 
      Consider the case in which $\adec{\octx}{n}$ and $n > 0$ and $\lctx = \lctx' \oc \atmR{s}$ and $\ainc{\lctx'}{n-1}$.
      According to big-step adequacy of decrements~\parencref{thm:msg-dec-big-adequacy}, $\octx \Reduces \octx' \oc \atmR{s}$ for some $\octx'$ such that $\ainc{\octx'}{n-1}$.
      It immediately follows that $\octx \Reduces\rframe{\simu{R}}{\atmR{s}} \lctx$.

    \item
      Consider the case in which $\adec{\octx}{n}$ and $\adec{\lctx}{n}$, with both $\atmL{\lctx}'_L$ and $\atmR{\lctx}'_R$ empty.
      The required $\octx \Reduces\lrframe{\atmL{\lctx}'_L}{\reflc{\simu{R}}}{\atmR{\lctx}'_R} \lctx$ is trivial.
    \end{itemize}


    % \begin{itemize}
    % \item 
    %   Consider the case in which $\octx \simu{R} \atmR{z}$ and $\adec{\octx}{0}$.
    %   According to \cref{??}, $\octx \Reduces \atmR{z}$ and so, indeed, $\octx \Reduces\rframe{\simu{R}}{\atmR{z}} \atmR{z}$.

    % \item 
    %   Consider the case in which $\octx \simu{R} \lctx = \lctx' \oc \atmR{s}$ and $\adec{\octx}{n'+1} = n$ and $\ainc{\lctx'}{n'}$.
    %   According to \cref{??}, $\octx \Reduces \octx' \oc \atmR{s}$ for some $\octx'$ such that $\ainc{\octx'}{n'}$.
    %   It immediately follows that $\octx \Reduces\rframe{\simu{R}}{\atmR{s}} \lctx$.

    % \item
    %   Consider the case in which $\octx \simu{R} \lctx = \lctx'$ and both $\atmL{\lctx}'_L$ and $\atmR{\lctx}'_R$ are empty.
    %   The required $\octx \Reduces\lrframe{\atmL{\lctx}'_L}{\simu{R}}{\atmR{\lctx}'_R} \lctx$ is trivial.
    % \end{itemize}

  \item[Immediate input bisimulation]
    Assume that $\octx \simu{R} \lctx$ and $\ireduces{\atmR{\lctx}_L \oc #1 \oc \atmL{\lctx}_R}{\lctx}{\lctx'}$.
    There are several cases.
    \begin{itemize}
    \item
      Consider the case in which $\ainc{\octx}{n}$ and $\ainc{\lctx}{n}$, for some $n$.
      According to \cref{??}, the input transition is either $\ireduces{#1 \oc \atmL{i}}{\lctx}{\lctx'}$ or $\ireduces{#1 \oc \atmL{d}}{\lctx}{\lctx'}$.

      For the former transition, we may apply the $\jrule{$\atmL{i}$-I}$ rule to deduce that $\ainc{\octx \oc \atmL{i}}{n+1}$ and $\ainc{\lctx \oc \atmL{i}}{n+1}$.
      Because $\lctx \oc \atmL{i} \reduces \lctx'$, it follows from preservation\parencref{??} that $\ainc{\lctx'}{n+1}$.
      We conclude that $\octx \oc \atmL{i} \Reduces\simu{R} \lctx'$, as required.

      For the latter input transition, a similar argument can be used.

    \item 
      Consider the case in which $\adec{\octx}{n}$ and $\adec{\lctx}{n}$, for some $n$.
      By \cref{??}, the input transition can only be $\ireduces{#1}{\lctx}{\lctx'}$, and so $\lctx \reduces \lctx'$.
      It follows from preservation\parencref{??} that $\adec{\lctx'}{n}$; we conclude that $\octx \Reduces\simu{R} \lctx'$, as required.
    \end{itemize}

  \item[Reduction bisimulation]
    Assume that $\octx \simu{R} \lctx \reduces \lctx'$.
    The states $\octx$ and $\lctx$ are either both increment- or both decrement-states with equal denotations.
    In either case, the relevant preservation result \parencref{??} allows us to deduce that $\lctx'$ has the same denotation and then conclude that $\octx \Reduces\simu{R} \lctx'$, as required.

  \item[Emptiness bisimulation]
    This is vacuously true.
  \end{description}

  \item
    To prove the converse, that bisimilar states have equal denotations, we shall use a lexicographic induction, first on the denotation $n$ and then on [..].
  \begin{enumerate}
  \item Assume that $\octx$ and $\lctx$ are bisimilar decrement states denoting $n$ and $n'$, respectively.
    We distinguish cases on $n$.
    \begin{itemize}
    \item
      Consider the case in which $n = 0$.
      By big-step adequacy of decrements~\parencref{thm:msg-dec-big-adequacy}, $\octx \Reduces \atmR{z}$.
      Because $\octx$ and $\lctx$ are bisimilar, $\lctx \Reduces\rframe{\osim}{\atmR{z}} \atmR{z}$.
      According to \cref{thm:msg-dec-big-adequacy} again, $\lctx$ eventually emits $\atmR{z}$ only if its denotation is $n' = 0$, and so $n = 0 = n'$.

    \item 
      Consider the case in which $n > 0$.
      By big-step adequacy of decrements~\parencref{thm:msg-dec-big-adequacy}, $\octx \Reduces \octx' \oc \atmR{s}$ for some $\octx'$ such that $\ainc{\octx'}{n-1}$.
      Because $\octx$ and $\lctx$ are bisimilar, $\lctx \Reduces\rframe{\osim}{\atmR{s}} \octx' \oc \atmR{s}$; in other words, $\lctx \Reduces \lctx' \oc \atmR{s}$ for some $\lctx'$ such that $\octx' \osim \lctx'$.
      According to \cref{thm:msg-dec-big-adequacy} again, $\lctx$ eventually emits $\atmR{s}$ only if $n' > 0$ and $\ainc{\lctx'}{n'-1}$.
      By the inductive hypothesis, it follows that $n-1 = n'-1$, and so $n = n'$ as required.
    \end{itemize}
    
  \item Assume that $\octx$ and $\lctx$ are bisimilar increment states denoting $n$ and $n'$, respectively.
    By applying the $\jrule{$\atmL{d}$-D}$ rule, we may deduce that $\adec{\octx \oc \atmL{d}}{n}$ and $\adec{\lctx \oc \atmL{d}}{n'}$.
    Moreover, because rewriting bisimilarity is a congruence~\parencref{??}, $\octx \oc \atmL{d} \osim \lctx \oc \atmL{d}$.
    By part~\ref{item:??} of the inductive hypothesis, we conclude that $n = n'$, as required.
  %
  \qedhere
  \end{enumerate}
  \end{itemize}
\end{proof}


% \begin{theorem}[Adequacy of binary counter decrements]
%   If $\adec{\octx}{n}$, then:
%   \begin{itemize}[nosep]
%   \item $\octx \Reduces \atmR{z}$ if, and only if, $n = 0$;
%   \item $\octx \Reduces \octx' \oc \atmR{s}$ for some $\octx'$ such that $\ainc{\octx'}{n-1}$, if $n > 0$; and
%   \item $\octx \Reduces \octx' \oc \atmR{s}$ only if $n > 0$ and $\ainc{\octx'}{n-1}$.
%   \end{itemize}
% \end{theorem}

% \begin{theorem}[Small-step adequacy of binary counter decrements]\leavevmode
%   \begin{thmdescription}
%   \item[Preservation]
%     If $\adec{\octx}{n}$ and $\octx \reduces \octx'$, then $\adec{\octx'}{n}$.
%   \item[Progress]
%     If $\adec{\octx}{n}$, then either:
%     \begin{itemize}[nosep]
%     \item $\octx \reduces \octx'$, for some $\octx'$;
%     \item $n = 0$ and $\octx = \atmR{z}$; or
%     \item $n > 0$ and $\octx = \octx' \oc \atmR{s}$, for some $\octx'$ such that $\ainc{\octx'}{n-1}$.\fixnote{Value instead of increment relation?}
%     \end{itemize}
%   \item[Termination]
%     If $\adec{\octx}{n}$, then every rewriting sequence from $\octx$ is finite.
%   \end{thmdescription}
% \end{theorem}


\subsection{An alternative specification of a binary counter}

The above description of a binary counter, repeated here%
\begin{marginfigure}
  \begin{equation*}
    \begin{lgathered}
      e \defd (e \fuse b_1 \pmir \atmL{i}) \with (\atmR{z} \pmir \atmL{d}) \\
      b_0 \defd (\up \dn b_1 \pmir \atmL{i}) \with (\atmL{d} \fuse b'_0 \pmir \atmL{d}) \\
      b_1 \defd (\atmL{i} \fuse b_0 \pmir \atmL{i}) \with (b_0 \fuse \atmL{s} \pmir \atmL{d}) \\
      b'_0 \defd (\atmR{z} \limp \atmR{z}) \with (\atmR{s} \limp b_1 \fuse \atmR{s})
    \end{lgathered}
  \end{equation*}
  \caption{An object-oriented specification of a binary counter}
\end{marginfigure}
for convenience, could be described as object-oriented.
Like objects, the processes $e$, $b_0$, and $b_1$ dispatch on incoming messages $\atmL{i}$ and $\atmL{d}$, and the process $b'_0$ dispatches on incoming messages $\atmR{z}$ and $\atmR{s}$.%
\footnote{For a study of the relationship between (session-typed) processes and objects, see \textcite{Balzer+Pfenning:AGERE15}.}

Alternatively, we could specify the binary counter in a dual way: like functions are applied to data, the processes $i$ and $d$ act on incoming messages $\atmR{e}$, $\atmR{b}_0$, and $\atmR{b}_1$, and the processes $z$ and $s$ act on incoming $\atmL{b}'_0$ messages.
\begin{equation*}
  \begin{lgathered}
    i \defd (\atmR{e} \limp \atmR{e} \fuse \atmR{b}_1) \with (\atmR{b}_0 \limp \atmR{b}_1) \with (\atmR{b}_1 \limp i \fuse \atmR{b}_0) \\
    d \defd (\atmR{e} \limp \up \dn z) \with (\atmR{b}_0 \limp d \fuse \atmL{b}'_0) \with (\atmR{b}_1 \limp \atmR{b}_0 \fuse s) \\
    z \defd \up \dn z \pmir \atmL{b}'_0 \\
    s \defd \atmR{b}_1 \fuse s \pmir \atmL{b}'_0
  \end{lgathered}
\end{equation*}
In contrast with the earlier object-oriented specification, this specification could be described as functional in style.
\begin{equation*}
  \atmR{e} \oc \atmR{b}_1 \oc i \Reduces \atmR{e} \oc i \oc \atmR{b}_0 \Reduces \atmR{e} \oc \atmR{b}_1 \oc \atmR{b}_1
\end{equation*}

Intuitively, we should expect these two specifications to be equivalent descriptions of a binary counter.
To make this equivalence concrete, we might imagine defining a binary relation $\simu{D}$ on binary counters that makes the duality precise;
for example, $e \oc b_1 \oc \atmL{i} \simu{D} \atmR{e} \oc \atmR{b}_1 \oc i$.

However, in defining the duality relation, we implicitly observe and compare the counters' internal structures.
Although certainly possible at the meta-level, this is somewhat unsatisfying because it doesn't compare the counters' \emph{behaviors}.
There ought to be a way to characterize the counters' equivalence using bisimilarity.

Doing so requires a few small changes to 



% \section{Unary counter}

% \begin{equation*}
%   \begin{lgathered}
%     z \defd (z \fuse s \pmir \atmL{i}) \with (\atmR{z} \pmir \atmL{d}) \\
%     s \defd (s \fuse s \pmir \atmL{i}) \with (\atmR{s} \pmir \atmL{d})
%   \end{lgathered}
% \end{equation*}

% \begin{inferences}
%   \infer{\ainc{z}{0}}{}
%   \and
%   \infer{\ainc{\lctx \oc s}{n+1}}{
%     \ainc{\lctx}{n}}
%   \and
%   \infer{\ainc{\lctx \oc \atmL{i}}{n+1}}{
%     \ainc{\lctx}{n}}
%   \\
%   \infer{\ainc{z \fuse s}{1}}{}
%   \and
%   \infer{\ainc{\lctx \oc (s \fuse s)}{n+2}}{
%     \ainc{\lctx}{n}}
%   \\
%   \infer{\adec{\lctx \oc \atmL{d}}{n}}{
%     \ainc{\lctx}{n}}
%   \and
%   \infer{\adec{\atmR{z}}{0}}{}
%   \and
%   \infer{\adec{\lctx \oc \atmR{s}}{n+1}}{
%     \ainc{\lctx}{n}}
% \end{inferences}

% \begin{theorem}[Adequacy of unary counter increments]
%   If $\ainc{\lctx}{n}$, then $\lctx \Reduces\aval{}{n'}$ if, and only if, $n' = n+1$.
% \end{theorem}

% \begin{theorem}[Small-step adequacy of unary counter increments]\leavevmode
%   \begin{thmdescription}
%   \item[Preservation]
%     If $\ainc{\lctx}{n}$ and $\lctx \reduces \lctx'$, then $\ainc{\lctx'}{n}$.
%   \item[Progress]
%     If $\ainc{\lctx}{n}$, then either:
%     \begin{itemize}[nosep]
%     \item $\lctx \reduces \lctx'$, for some $\lctx'$; or
%     \item $\lctx \nreduces$ and $\aval{\lctx}{n}$.
%     \end{itemize}
%   \item[Termination]
%     If $\ainc{\lctx}{n}$, then every rewriting sequence from $\lctx$ is finite.
%   \end{thmdescription}
% \end{theorem}

% \begin{theorem}[Adequacy of unary counter decrements]
%   If $\adec{\lctx}{n}$, then:
%   \begin{itemize}[nosep]
%   \item $\lctx \Reduces \atmR{z}$ if, and only if, $n = 0$;
%   \item $\lctx \Reduces \lctx' \oc \atmR{s}$ for some $\lctx'$ such that $\ainc{\lctx'}{n-1}$, if $n > 0$; and
%   \item $\lctx \Reduces \lctx' \oc \atmR{s}$ only if $n > 0$ and $\ainc{\lctx'}{n-1}$.
%   \end{itemize}
% \end{theorem}

% \begin{theorem}[Small-step adequacy of unary counter decrements]\leavevmode
%   \begin{thmdescription}
%   \item[Preservation]
%     If $\adec{\lctx}{n}$ and $\lctx \reduces \lctx'$, then $\adec{\lctx'}{n}$.
%   \item[Progress]
%     If $\adec{\lctx}{n}$, then either:
%     \begin{itemize}[nosep]
%     \item $\lctx \reduces \lctx'$, for some $\lctx'$;
%     \item $n = 0$ and $\lctx = \atmR{z}$; or
%     \item $n > 0$ and $\lctx = \lctx' \oc \atmR{s}$, for some $\lctx'$ such that $\ainc{\lctx'}{n-1}$.\fixnote{Value instead of increment relation?}
%     \end{itemize}
%   \item[Termination]
%     If $\adec{\lctx}{n}$, then every rewriting sequence from $\lctx$ is finite.
%   \end{thmdescription}
% \end{theorem}


% % \begin{lemma}\leavevmode
% %   \begin{itemize}[nosep]
% % %  \item If $\ainc{\lctx}{0}$, then $\lctx = z$.
% % %  \item If $\ainc{\lctx}{n+1}$, then $\lctx \Reduces \lctx' \oc s$ for some $\ainc{\lctx'}{n}$.
% %   \item If $\adec{\lctx}{0}$, then $\lctx \Reduces \atmR{z}$.
% %   \item If $\adec{\lctx}{n+1}$, then $\lctx \Reduces \lctx' \oc \atmR{s}$ for some $\ainc{\lctx'}{n}$.
% % %   \end{itemize}
% % % \end{lemma}

% % % \begin{lemma}\leavevmode
% % %   \begin{itemize}[nosep]
% % %  \item If $\ainc{\octx}{0}$, then either $\octx = e$ or $\octx = \octx' \oc b_0$ for some $\ainc{\octx'}{0}$.
% % %  \item If $\ainc{\octx}{2n} > 0$, then $\octx \Reduces \octx' \oc b_0$ for some $\ainc{\octx'}{n}$.
% % %  \item If $\ainc{\octx}{2n+1}$, then $\octx \Reduces \octx' \oc b_1$ for some $\ainc{\octx'}{n}$.
% %   % \item If $\adec{\octx}{0}$, then $\octx \Reduces \atmR{z}$.
% %   % \item If $\adec{\octx}{n+1}$, then $\octx \Reduces \octx' \oc \atmR{s}$ for some $\ainc{\octx'}{n}$.
% %   \end{itemize}
% % \end{lemma}

% \begin{theorem}[Bisimilarity of counters]\leavevmode
%   \begin{itemize}[nosep]
%   \item If $\ainc{\octx}{n}$ and $\ainc{\lctx}{n}$, then $\octx \osim \lctx$.
%   \item If $\adec{\octx}{n}$ and $\adec{\lctx}{n}$, then $\octx \osim \lctx$.
%   \end{itemize}
% \end{theorem}
% \begin{proof}
%   Let $\simu{R}$ be the symmetric closure of $\Set{(\octx, \lctx) \given \exists n.\, (\ainc{\octx}{n}) \land (\ainc{\lctx}{n})} \union \Set{(\octx, \lctx) \given \exists n.\, (\adec{\octx}{n}) \land (\adec{\lctx}{n})} \union \Set{(\octxe, \octxe)}$.
%   We will show that $\simu{R}$ is a labeled bisimulation.
%   \begin{description}
%   \item[Immediate output bisimulation]
%     \begin{itemize}
%     \item Consider the case in which $\adec{\octx}{0}$ and $\adec{\atmR{z}}{0}$.
%       By \cref{??}, $\octx \Reduces \atmR{z}$.
%       It follows that $\octx \Reduces\rframe{\simu{R}}{\atmR{z}} \atmR{z}$, as required.
%     \item Consider the case in which $\adec{\octx}{n+1}$ and $\adec{\lctx \oc \atmR{s}}{n+1}$ because $\ainc{\lctx}{n}$.
%       By \cref{??}, $\octx \Reduces \octx' \oc \atmR{s}$ for some $\ainc{\octx'}{n}$.
%       It follows that $\octx \Reduces\rframe{\simu{R}}{\atmR{s}} \lctx \oc \atmR{s}$, as required.
%     \item Consider the case in which $\adec{\atmR{z}}{0}$ and $\adec{\lctx}{0}$.
%       By \cref{??}, $\lctx \Reduces \atmR{z}$.
%       It follows that $\lctx \Reduces\rframe{\simu{R}}{\atmR{z}} \atmR{z}$, as required.
%     \item Consider the case in which $\adec{\octx \oc \atmR{s}}{n+1}$ and $\adec{\lctx}{n+1}$ because $\ainc{\octx}{n}$.
%       By \cref{??}, $\lctx \Reduces \lctx' \oc \atmR{s}$ for some $\ainc{\lctx'}{n}$.
%       It follows that $\lctx \Reduces\rframe{\simu{R}}{\atmR{s}} \octx \oc \atmR{s}$, as required.
%     \end{itemize}

%   \item[Immediate input bisimulation]
%     \begin{itemize}
%     \item Consider the case in which $\ainc{\octx}{n}$ and $\ainc{\lctx}{n}$ and $\ireduces{#1 \oc \atmL{i}}{\lctx}{\lctx'}$.
%       Notice that $\ainc{\lctx \oc \atmL{i}}{n+1}$ and $\lctx \oc \atmL{i} \reduces \lctx'$; by preservation \parencref{??}, $\ainc{\lctx'}{n+1}$.
%       It is then trivial that $\octx \oc \atmL{i} \Reduces\simu{R} \lctx'$, as required.
%     \item Consider the case in which $\ainc{\octx}{n}$ and $\ainc{\lctx}{n}$ and $\ireduces{#1 \oc \atmL{d}}{\lctx}{\lctx'}$.
%       Notice that $\adec{\lctx \oc \atmL{d}}{n}$ and $\lctx \oc \atmL{d} \reduces \lctx'$; by preservation \parencref{??}, $\adec{\lctx'}{n}$.
%       It is then trivial that $\octx \oc \atmL{d} \Reduces\simu{R} \lctx'$, as required.
%     \end{itemize}

%   \item[Reduction bisimulation]
%     \begin{itemize}
%     \item Consider the case in which $\ainc{\octx}{n}$ and $\ainc{\lctx}{n}$ and $\lctx \reduces \lctx'$.
%       By preservation \parencref{??}, $\ainc{\lctx'}{n}$.
%       It is then trivial that $\octx \Reduces\simu{R} \lctx'$, as required.
%     \item Consider the case in which $\adec{\octx}{n}$ and $\adec{\lctx}{n}$ and $\lctx \reduces \lctx'$.
%       In this case, the proof is similar to the above increment case.
%     \item Consider the cases in which a binary counter $\octx$ reduces; these are analogous to the previous cases involving a unary counter that reduces.
%     \end{itemize}

%   \item[Emptiness bisimulation]
%     The only case involving empty contexts is that of $\octxe \simu{R} \octxe$.
%     In this case, indeed $\atmR{\lctx} \Reduces\rframe{\simu{R}}{\atmR{\lctx}} \atmR{\lctx}$ for all $\atmR{\lctx}$, and, symmetrically, $\atmL{\lctx} \Reduces\lframe{\atmL{\lctx}}{\simu{R}} \atmL{\lctx}$ for all $\atmL{\lctx}$.
%   \end{description}

%   % \begin{description}
%   % \item[Immediate output]
%   %   Vacuous
%   % \item[Immediate input] 
%   %   \begin{itemize}
%   %   \item Suppose that $\ainc{e}{0}$ and $\ainc{z}{0}$.
%   %     \begin{itemize}
%   %     \item Suppose that $\ireduces{#1 \oc \atmL{i}}{e}{e \fuse b_1}$.
%   %       $z \oc \atmL{i} \Reduces \ainc{z \oc s}{1}$.
%   %       Symmetrically involving transition on $z$.
%   %     \item Suppose that $\ireduces{#1 \oc \atmL{d}}{e}{\atmR{z}}$.
%   %       $z \oc \atmL{d} \Reduces \adec{\atmR{z}}{0}$.
%   %       Symmetrically.
%   %     \end{itemize}
%   %   \item $\ainc{\octx \oc b_0}{0}$ and$\ainc{z}{0}$
%   %     \begin{itemize}
%   %     \item $\octx \oc b_1$ and $z \fuse s$
%   %     \item $\octx \oc (d \fuse b'_0)$ and $\atmR{z}$
%   %     \end{itemize}
%   %   \end{itemize}
%   % \item[Reduction] 
%   %   Preservation
%   % \item[Emptiness] 
%   %   Vacuous
%   % \end{description}

%   % \begin{description}
%   % \item[Immediate output]
%   %   \begin{itemize}
%   %   \item $\atmR{z}$ and $\atmR{z}$
%   %   \item $\octx \oc \atmR{s}$ and $\lctx \oc \atmR{s}$
%   %   \end{itemize}
%   % \item[Immediate input] 
%   %   Vacuous
%   % \item[Reduction] 
%   %   Preservation
%   % \item[Emptiness] 
%   %   Vacuous
%   % \end{description}
% \end{proof}


% \section{Without using the adequacy relations}

% \begin{inferences}
%   \infer{e \simu{R}_v z}{}
%   \and
%   \infer{\octx \oc b_0 \simu{R}_v \lctx}{
%     \octx \simu{R}_v\simu{D} \lctx}
%   \and
%   \infer{\octx \oc b_1 \simu{R}_v \lctx \oc s}{
%     \octx \simu{R}_v\simu{D} \lctx}
%   \\
%   \infer{z \simu{D} z}{}
%   \and
%   \infer{\lctx \oc s \simu{D} \lctx' \oc s \oc s}{
%     \lctx \simu{D} \lctx'}
% \end{inferences}

% \begin{inferences}
%   \infer{e \simu{R}_i z}{}
%   \and
%   \infer{\octx \oc b_0 \simu{R}_i \lctx}{
%     \octx \simu{R}_i\simu{D} \lctx}
%   \and
%   \infer{\octx \oc b_1 \simu{R}_i \lctx \oc s}{
%     \octx \simu{R}_i\simu{D} \lctx}
%   \and
%   \infer{\octx \oc \atmL{i} \simu{R}_i \lctx \oc s}{
%     \octx \simu{R}_i \lctx}
%   \\
%   \infer{e \fuse b_1 \simu{R}_i \lctx}{
%     e \oc b_1 \simu{R}_i \lctx}
%   \and
%   \infer{\octx \oc (\atmL{i} \fuse b_0) \simu{R}_i \lctx}{
%     \octx \oc \atmL{i} \oc b_0 \simu{R}_i \lctx}
%   \\
%   \infer{\octx \simu{R}_i z \fuse s}{
%     \octx \simu{R}_i z \oc s}
%   \and
%   \infer{\octx \simu{R}_i \lctx \oc (s \fuse s)}{
%     \octx \simu{R}_i \lctx \oc s \oc s}
% \end{inferences}

% \begin{inferences}
%   \infer{\octx \oc \atmL{d} \simu{R}_d \lctx \oc \atmL{d}}{
%     \octx \simu{R}_i \lctx}
%   \and
%   \infer{\atmR{z} \simu{R}_d \atmR{z}}{}
%   \and
%   \infer{\octx \oc \atmR{s} \simu{R}_d \lctx \oc \atmR{s}}{
%     \octx \simu{R}_i \lctx}
%   \and
%   \infer{\octx \oc b'_0 \simu{R}_d \lctx}{
%     \octx \simu{R}_d\simu{D}_d \lctx}
%   \\
%   \infer{\octx \oc (\atmL{d} \fuse b'_0) \simu{R}_d \lctx}{
%     \octx \oc \atmL{d} \oc b'_0 \simu{R}_d \lctx}
%   \and
%   \infer{\octx \oc (b_1 \fuse \atmR{s}) \simu{R}_d \lctx}{
%     \octx \oc b_1 \oc \atmR{s} \simu{R}_d \lctx}
%   \and
%   \infer{\octx \oc (b_0 \fuse \atmR{s}) \simu{R}_d \lctx}{
%     \octx \oc b_0 \oc \atmR{s} \simu{R}_d \lctx}
% \end{inferences}

% \begin{inferences}
%   \infer{\atmR{z} \simu{D}_d \atmR{z}}{}
%   \and
%   \infer{\lctx \oc \atmR{s} \simu{D}_d \lctx' \oc s \oc \atmR{s}}{
%     \lctx \simu{D} \lctx'}
% \end{inferences}

% \begin{description}
% \item[Immediate output bisimulation]
%   Consider the case in which 
% \end{description}


% \begin{description}
% \item[Immediate input]
%   \begin{itemize}
%   \item $z \oc \atmL{i} \Reduces\simu{R} e \fuse b_1$
%   \item $\lctx \oc \atmL{i} \Reduces\simu{R} \octx \oc b_1$ when $\octx \simu{R}\simu{D} \lctx$
%   \item $\lctx \oc s \oc \atmL{i} \Reduces\simu{R} \octx \oc (\atmL{i} \fuse b_0)$ when $\octx \simu{R}\simu{D} \lctx$
%   \item $e \oc \atmL{i} \Reduces\simu{R} z \fuse s$
%   \item $\octx \oc b_0 \oc \atmL{i} \Reduces\simu{R} z \fuse s$ when $\octx \simu{R} z$
%   \item $\octx \oc b_0 \oc \atmL{i} \Reduces\simu{R} \lctx \oc s \oc (s \fuse s)$ when $\octx \simu{R}\simu{D} \lctx \oc s \oc s$
%   \item $\octx \oc b_1 \oc \atmL{i} \Reduces\simu{R} \lctx \oc (s \fuse s)$ when $\octx \simu{R}\simu{D} \lctx$.
%     In the case, $\octx \oc \atmL{i} \oc b_0 \simu{R} \lctx \oc s \oc s$.
%   \end{itemize}
% \item[Reduction] 
%   \begin{itemize}
%   \item $\lctx \Reduces\simu{R} \octx' \oc b_0$ when $\lctx \simu{D}^{-1}\Reduces\simu{R} \octx'$
%   \item $\lctx \oc s \Reduces\simu{R} \octx' \oc b_1$ when $\lctx \simu{D}^{-1}\Reduces\simu{R} \octx'$
%   \item $\lctx \oc s \Reduces\simu{R} \octx' \oc \atmL{i}$ when $\lctx \Reduces\simu{R} \octx'$
%   \item $\lctx \Reduces\simu{R} e \oc b_1$ when $\lctx \simu{R} e \oc b_1$
%   \item $z \fuse s \Reduces\simu{R} \octx'$ when $z \oc s \Reduces\simu{R} \octx'$
%   \item $\octx \oc b_0 \Reduces\simu{R} \lctx'$ when $\octx \simu{R}\simu{D}\reduces \lctx'$
%   \item $\octx \oc b_1 \Reduces\simu{R} \lctx'$ when $\octx \simu{R}\simu{D}\reduces \lctx'$
%   \item $\octx \oc \atmL{i} \Reduces\simu{R} \lctx' \oc s$ when $\octx \Reduces\simu{R} \lctx'$
%   \item $e \fuse b_1 \Reduces\simu{R} \lctx'$ when $e \oc b_1 \Reduces\simu{R} \lctx'$
%   \item $\octx \oc (\atmL{i} \fuse b_0) \Reduces\simu{R} \lctx'$ when $\octx \oc \atmL{i} \oc b_0 \Reduces\simu{R} \lctx'$
%   \item $\octx \Reduces\simu{R} z \oc s$ when $\octx \simu{R} z \oc s$
%   \item $\octx \Reduces\simu{R} \lctx' \oc (s \fuse s)$ when $\octx \Reduces\simu{R} \lctx' \oc s \oc s$
%   \item $\octx \Reduces\simu{R} \lctx \oc s \oc s$ when $\octx \simu{R} \lctx \oc s \oc s$
%   \end{itemize}
% \end{description}


\begin{equation*}
  \begin{lgathered}
    c \defd (i \fuse c \pmir \atmL{i}) \with (d \fuse \atmL{u} \pmir \atmL{d}) \\
    z \defd (\up \dn z \pmir \atmL{b}'_0) \with (\atmR{z} \pmir \atmL{u}) \\
    s \defd (\atmR{b}_1 \fuse s \pmir \atmL{b}'_0) \with (c \fuse \atmR{s} \pmir \atmL{u})
  \end{lgathered}
\end{equation*}

\begin{inferences}
  \infer{\iainc{\atmR{e}}{0}}{}
  \and
  \infer{\iainc{\lctx \oc \atmR{b}_0}{2n}}{
    \iainc{\lctx}{n}}
  \and
  \infer{\iainc{\lctx \oc \atmR{b}_1}{2n+1}}{
    \iainc{\lctx}{n}}
  \and
  \infer{\iainc{\lctx \oc i}{n+1}}{
    \iainc{\lctx}{n}}
  \\
  \infer{\iainc{\atmR{e} \fuse \atmR{b}_1}{1}}{}
  \and
  \infer{\iainc{\lctx \oc (i \fuse \atmR{b}_0)}{2(n+1)}}{
    \iainc{\lctx}{n}}
\end{inferences}

\begin{inferences}
  \infer{\eainc{\lctx \oc c}{n}}{
    \iainc{\lctx}{n}}
  \and
  \infer{\eainc{\lctx \oc \atmL{i}}{n+1}}{
    \eainc{\lctx}{n}}
  \and
  \infer{\eainc{\lctx \oc (i \fuse c)}{n+1}}{
    \iainc{\lctx}{n}}
\end{inferences}

\begin{theorem}[small-step adequacy of increments]
  \leavevmode
  \begin{thmdescription}
  \item[Value soundness]

  \item[Preservation]
    If $\eainc{\lctx}{n}$ and $\lctx \reduces \lctx'$, then $\eainc{\lctx'}{n}$.

  \item[Progress]
    If $\eainc{\lctx}{n}$, then either:
    \begin{itemize*}
    \item $\lctx \reduces \lctx'$, for some $\lctx'$; or
    \item $\eaval{\lctx}{n}$.
    \end{itemize*}

  \item[Termination]
    If $\eainc{\lctx}{n}$, then every rewriting sequence from $\lctx$ is finite.
  \end{thmdescription}
\end{theorem}

\begin{inferences}
  \infer{\iadec{\lctx \oc d}{n}}{
    \iainc{\lctx}{n}}
  \and
  \infer{\iadec{\lctx \oc \atmL{b}'_0}{2n}}{
    \iadec{\lctx}{n}}
  \and
  \infer{\iadec{z}{0}}{}
  \and
  \infer{\iadec{\lctx \oc s}{n+1}}{
    \iainc{\lctx}{n}}
  \\
  \infer{\iadec{\lctx \oc (d \fuse \atmL{b}'_0)}{2n}}{
    \iainc{\lctx}{n}}
  \and
  \infer{\iadec{\lctx \oc (\atmR{b}_0 \fuse s)}{2n+1}}{
    \iainc{\lctx}{n}}
  \and
  \infer{\iadec{\lctx \oc (\atmR{b}_1 \fuse s)}{2n+2}}{
    \iainc{\lctx}{n}}
\end{inferences}

\begin{inferences}
  \infer{\eadec{\lctx \oc \atmL{d}}{n}}{
    \eainc{\lctx}{n}}
  \and
  \infer{\eadec{\lctx \oc \atmL{u}}{n}}{
    \iadec{\lctx}{n}}
  \and
  \infer{\eadec{\atmR{z}}{0}}{}
  \and
  \infer{\eadec{\lctx \oc c \oc \atmR{s}}{n+1}}{
    \iainc{\lctx}{n}}
  \\
  \infer{\eadec{\lctx \oc (d \fuse \atmL{u})}{n}}{
    \iainc{\lctx}{n}}
  \and
  \infer{\eadec{\lctx \oc (c \fuse \atmR{s})}{n+1}}{
    \iainc{\lctx}{n}}
\end{inferences}


\begin{theorem}[Small-step adequacy of decrements]
  \leavevmode
  \begin{thmdescription}
  \item[Preservation]
    If $\eadec{\lctx}{n}$ and $\lctx \reduces \lctx'$, then $\eadec{\lctx'}{n}$.

  \item[Progress]
    If $\eadec{\lctx}{n}$, then either:
    \begin{itemize}[nosep]
    \item $\lctx \reduces \lctx'$ for some $\lctx'$;
    \item $n = 0$ and $\lctx = \atmR{z}$;
    \item $n > 0$ and $\lctx = \lctx' \oc c \oc \atmR{s}$, for some $\lctx'$ such that $\iainc{\lctx'}{n-1}$.
    \end{itemize}

  \item[Termination]
    If $\eadec{\lctx}{n}$, then every rewriting sequence from $\lctx$ is finite.
  \end{thmdescription}
\end{theorem}

\begin{theorem}[Big-step adequacy of decrements]
  If $\eadec{\lctx}{n}$, then:
  \begin{itemize}[nosep]
  \item $\lctx \Reduces \atmL{\lctx}'_L \oc \lctx' \oc \atmR{\lctx}'_R$ only if either: $\atmL{\lctx}'_L = \atmR{\lctx}'_R = \octxe$; or $n = 0$ and $\atmL{\lctx}'_L = \octxe$ and $\atmR{\lctx}'_R = \atmR{z}$; or $n > 0$ and $\atmL{\lctx}'_L = \octxe$ and $\atmR{\lctx}'_R = \atmR{s}$;
  \item $\lctx \Reduces \atmR{z}$ if $n = 0$;
  \item $\lctx \Reduces \lctx' \oc c \oc \atmR{s}$ for some $\lctx'$ such that $\iainc{\lctx'}{n-1}$, if $n > 0$; and 
  \item $\lctx \Reduces \lctx' \oc \atmR{s}$ only if $n > 0$ and $\lctx' = \lctx' \oc c$ for some $\lctx'$ such that $\iainc{\lctx'}{n-1}$.
  \end{itemize}
\end{theorem}

\begin{theorem}[Big-step adequacy of decrements]
  If $\eadec{\lctx}{n}$, then:
  \begin{itemize}[nosep]
  \item $\lctx \Reduces \atmR{z}$ if, and only if, $n = 0$;
  \item $\lctx \Reduces \lctx' \oc c \oc \atmR{s}$ for some $\lctx'$ such that $\iainc{\lctx'}{n-1}$, if $n > 0$; and 
  \item $\lctx \Reduces \lctx' \oc \atmR{s}$ only if $n > 0$ and $\lctx' = \lctx' \oc c$ for some $\lctx'$ such that $\iainc{\lctx'}{n-1}$.
  \end{itemize}
\end{theorem}

\begin{theorem}
  If $\ainc{\octx}{n}$ and $\eainc{\lctx}{n'}$, then $\octx \osim \lctx$ if, and only if, $n = n'$.
  Similarly, if $\adec{\octx}{n}$ and $\eadec{\lctx}{n'}$, then $\octx \osim \lctx$ if, and only if, $n = n'$.
\end{theorem}

\begin{proof}
  \begin{itemize}
  \item
    $\adec{\octx}{0}$ and $\eadec{\atmR{z}}{0}$.
    $\octx \Reduces \atmR{z}$
  \item 
    $\adec{\octx}{n+1}$ and $\eadec{\lctx \oc c \oc \atmR{s}}{n+1}$.
    $\octx \Reduces \octx' \oc \atmR{s}$ and $\ainc{\octx'}{n}$.
    $\octx' \simu{R} \lctx \oc c$
  \end{itemize}
\end{proof}


% \begin{theorem}\leavevmode
% \begin{itemize}[nosep]
% \item If $\ainc{\octx}{n}$ and $\eainc{\lctx}{n}$, then $\octx \osim \lctx$.
% \item If $\adec{\octx}{n}$ and $\eadec{\lctx}{n}$, then $\octx \osim \lctx$.
% \end{itemize}
% \end{theorem}
% %
% \begin{proof}
%   \begin{description}
%   \item[Immediate output bisimulation]
%     \begin{itemize}
%     \item Consider the case in which $\adec{\octx}{0}$ and $\eadec{\atmR{z}}{0}$.
%       By decrement adequacy, $\octx \Reduces \atmR{z}$.
%       It is trivial that $\octx \Reduces\rframe{\simu{R}}{\atmR{z}} \atmR{z}$.

%     \item Consider the case in which $\adec{\octx}{n+1}$ and $\eadec{\lctx \oc p \oc \atmR{s}}{n+1}$ where $\iainc{\lctx}{n}$.
%       By decrement adequacy, $\octx \Reduces \octx' \oc \atmR{s}$ for some $\ainc{\octx'}{n}$.
%       Notice that $\eainc{\lctx \oc p}{n}$.
%       It follows that $\octx \Reduces\rframe{\simu{R}}{\atmR{s}} \lctx \oc p \oc \atmR{s}$.
      
%     \item Consider the case in which $\adec{\atmR{z}}{0}$ and $\eadec{\lctx}{0}$.
%       By decrement adequacy, $\lctx \Reduces \atmR{z}$.
%       It is trivial that $\lctx \Reduces\rframe{\simu{R}}{\atmR{z}} \atmR{z}$.

%     \item
%       Consider the case in which $\adec{\octx' \oc \atmR{s}}{n+1}$ because $\ainc{\octx'}{n}$ and $\eadec{\lctx}{n+1}$.
%       By decrement adequacy, $\lctx \Reduces \lctx' \oc p \oc \atmR{s}$ for some $\iainc{\lctx'}{n}$.
%       Notice that $\eainc{\lctx' \oc p}{n}$.
%       It follows that $\lctx \Reduces\rframe{\simu{R}}{\atmR{s}} \octx' \oc \atmR{s}$.
%     \end{itemize}
%   \item[Immediate input bisimulation]
%     \begin{itemize}
%     \item Consider the case in which $\ainc{\octx}{n}$ and $\eainc{\lctx}{n}$ and $\ireduces{#1 \oc \atmL{i}}{\lctx}{\lctx'}$.
%       Notice that $\eainc{\lctx \oc \atmL{i}}{n+1}$ and $\lctx \oc \atmL{i} \reduces \lctx'$;
%       by preservation\parencref{??}, $\eainc{\lctx'}{n+1}$.
%       Also, notice that $\ainc{\octx \oc \atmL{i}}{n+1}$, so it is trivial that $\octx \oc \atmL{i} \Reduces\simu{R} \lctx'$, as required.
%       %
%     \item Consider the case in which $\ainc{\octx}{n}$ and $\eainc{\lctx}{n}$ and $\ireduces{#1 \oc \atmL{i}}{\octx}{\octx'}$.
%       Notice that $\ainc{\octx \oc \atmL{i}}{n+1}$ and $\octx \oc \atmL{i} \reduces \octx'$;
%       by preservation\parencref{??}, $\ainc{\octx'}{n+1}$.
%       Also, notice that $\ainc{\lctx \oc \atmL{i}}{n+1}$, so it is trivial that $\lctx \oc \atmL{i} \Reduces\simu{R} \octx'$, as required.
%     \end{itemize}

%   \item[Reduction bisimulation]
%     \begin{itemize}
%     \item Consider the case in which $\ainc{\octx}{n}$ and $\eainc{\lctx}{n}$ and $\lctx \reduces \lctx'$.
%       By preservation\parencref{??}, $\eainc{\lctx'}{n}$.
%       So it is trivial that $\octx \Reduces\simu{R} \lctx'$, as required.
%       %
%     \item Consider the case in which $\ainc{\octx}{n}$ and $\eainc{\lctx}{n}$ and $\octx \reduces \octx'$.
%       By preservation\parencref{??}, $\eainc{\octx'}{n}$.
%       So it is trivial that $\lctx \Reduces\simu{R} \octx'$, as required.
%       %
%     \end{itemize}
%   \end{description}
% \end{proof}


%%% Local Variables:
%%% mode: latex
%%% TeX-master: "thesis"
%%% End:
