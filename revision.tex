\chapter{}

\section{Ordered rewriting for specifications}

\subsection{\Aclp*{DFA}}

\begin{equation*}
  \infer[]{a \oc q \reduces q'_a}{}
\end{equation*}
for each \ac{DFA} transition $q \dfareduces[a] q'_a$, and 
\begin{equation*}
  \infer[]{\emp \oc q \reduces F(q)}{}
\end{equation*}
for each state $q$, where $F(q) = \one$ if $q$ is a final state and $F(q) = \top$ otherwise.

\begin{itemize}
\item $q \dfareduces[a] q'_a$ if, and only if, $a \oc q \reduces q'_a$; and 
\item $q \in F$ if, and only if, $\emp \oc q \reduces \one$.
\end{itemize}

\subsection{\Aclp*{NFA}}

Equally straightforward

\subsection{Binary counters}

Values

\paragraph*{An increment operation}
To use ordered rewriting to specify [...]
\begin{equation*}
  \infer[]{e \oc i \reduces e \oc b_1}{}
  \qquad
  \infer[]{b_0 \oc i \reduces b_1}{}
  \qquad
  \infer[]{b_1 \oc i \reduces i \oc b_0}{}
\end{equation*}

Small- and big-step adequacy theorems for increments
\begin{itemize}
\item Slightly simplified because there is no $\fuse$
\end{itemize}


\paragraph*{A decrement operation}
\begin{equation*}
  \infer[]{e \oc d \reduces z}{}
  \qquad
  \infer[]{b_0 \oc d \reduces d \oc b'_0}{}
  \qquad
  \infer[]{b_1 \oc d \reduces b_0 \oc s}{}
  \qquad
  \infer[]{z \oc b'_0 \reduces z}{}
  \qquad
  \infer[]{s \oc b'_0 \reduces b_1 \oc s}{}
\end{equation*}

\begin{itemize}
\item Significantly simpler because there is no $\with$, so we don't need (weak) focusing
\end{itemize}


\section{Choreography}

\subsection{Definitions and messages}

Unfocused

\subsection{Revisiting the binary counter}

Two choreographies: object-oriented and functional.
Compare
\begin{equation*}
  \infer[]{e \oc \atmL{i} \reduces e \oc b_1}{}
  \qquad\text{with}\qquad
  \infer[]{\atmR{e} \oc i \reduces \atmR{e} \oc \atmR{b}_1}{}
\end{equation*}
Instead of adding these as rewriting axioms, we can give recursive definitions for the process atoms:
\begin{equation*}
  \begin{lgathered}[t]
    e \defd (e \fuse b_1 \pmir \atmL{i}) \with (\atmR{z} \pmir \atmL{d}) \\
    b_0 \defd (b_1 \pmir \atmL{i}) \with (\atmL{d} \fuse b'_0 \pmir \atmL{d}) \\
    b_1 \defd (\atmL{i} \fuse b_0 \pmir \atmL{i}) \with (b_0 \fuse \atmR{s} \pmir \atmL{d}) \\
    b'_0 \defd (\atmR{z} \limp \atmR{z}) \with (\atmR{s} \limp b_1 \fuse \atmR{s})
  \end{lgathered}
  \qquad
  \begin{lgathered}[t]
    i \defd (\atmR{e} \limp \atmR{e} \fuse \atmR{b}_1) \with (\atmR{b}_0 \limp \atmR{b}_1) \with (\atmR{b}_1 \limp i \fuse \atmR{b}_0) \\
    d \defd (\atmR{e} \limp z) \with (\atmR{b}_0 \limp d \fuse \atmL{b}'_0) \with (\atmR{b}_1 \limp \atmR{b}_0 \fuse s) \\
    z \defd (z \pmir \atmL{b}'_0) \\
    s \defd (\atmR{b}_1 \fuse s \pmir \atmL{b}'_0)
  \end{lgathered}
\end{equation*}

How to define adequacy with respect to the specification?
\begin{itemize}
\item $e \oc \atmL{i} \reduces \octx'$ if, and only if, $\octx' = e \oc b_1$ does not hold with unfocused or weakly focused rewriting.
\item $e \oc \atmL{i} \reduces \octx'$ if, and only if, $\octx' \Reduces e \oc b_1$ does not hold.
  $\octx' = e \oc \atmL{i} \Reduces e \oc b_1$ but $e \oc \atmL{i} \nreduces e \oc \atmL{i}$.
\item How about $e \oc \atmL{i} \Reduces \octx'$ if, and only if, $\octx' \Reduces e \oc b_1$?
\end{itemize}


Two problems arise: values, such as $e$, can rewrite, and stuck states can occur.
$e \oc \atmL{i} \reduces (\atmR{z} \pmir \atmL{d}) \oc \atmL{i} \nreduces$.

\subsection{Weak focusing}

\subsection{Binary counter, again}

With axioms, $e \oc \atmL{i} \reduces e \oc b_1$; with definitions, $e \oc \atmL{i} \reduces e \fuse b_1 \reduces e \oc b_1$.

If $\ainc{\octx}{n}$ and $\octx \reduces \octx'$, then 


\subsection{\Acp*{DFA} and \acp*{NFA}, revisited}

The specification is 
\begin{equation*}
  \infer[]{a \oc q \reduces q'_a}{}
\end{equation*}
for each \ac{NFA} transition $q \nfareduces[a] q'_a$.

The functional choreography yields $\atmR{a} \oc \nfa{q} \reduces \nfa{q}'_a$ for each specification axiom $a \oc q \reduces q'_a$.
The converse does not hold: $\atmR{a} \oc \nfa{q} \reduces \nfa{q}'$ does not imply $a \oc q \reduces q'$ in the specification.

Working up to \ac{DFA} bisimilarity, we can prove adequacy of the \ac{DFA} encoding, as before.
Adequacy of the \ac{NFA} encoding requires rewriting bisimilarity, leading into the next chapter.


\begin{equation*}
  \infer[]{\hat{a} \oc \atmL{q} \reduces \atmL{q}'_a}{}
\end{equation*}
with
\begin{equation*}
  \hat{a} \defd \bigwith_{q \in Q}(\atmL{q}'_a \pmir \atmL{q})
\end{equation*}


\section{Rewriting bisimilarity}

As before, but now in a separate chapter from the message-passing choreographies.


\section{Singleton logic}

\section{Process interpretation}

\section{From choreographies to processes}



%%% Local Variables:
%%% mode: latex
%%% TeX-master: "thesis"
%%% End:
