% arara: pdflatex
% arara: biber
% arara: pdflatex
% arara: pdflatex
\documentclass[
  tufte-book,
  notoc,
  symmetric,
  biblatex={
    citestyle=authoryear-comp,
    % citestyle=authortitle-icomp,
    autocite=footnote,
    maxcitenames=2,
    bibstyle=authoryear,
    dashed=false,
    mergedate=basic,
    maxbibnames=99,
    backref=true,
    doi=false,
    url=false,
    isbn=false,
  }
]{tufte-thesis}

\usepackage{xpatch}
% \xapptobibmacro{cite}{\setunit{\nametitledelim}\printfield{year}}{}{}

\usepackage{tikz}
\usetikzlibrary { graphs , automata , quotes , cd }

\usepackage{logic}
\RenewDocumentCommand \limp { } { \mathbin { \setminus } }
\RenewDocumentCommand \pmir { } { \mathbin { / } }

\usepackage{proof}
% \usepackage{mathpartir}
% \ExplSyntaxOn
%   \cs_set_eq:NN \__infer:nn \infer
%   \RenewDocumentCommand \infer { m m }
%     { \__infer:nn {#2} {#1} }
% \ExplSyntaxOff

\usepackage{singleton}

\usepackage{thesis-macros}

\usepackage{pbox}

\newcommand* \ie {\textit{i.e.}}
\newcommand* \alertnote [1]{\footnote{\color{red} #1}}

%%% Local Variables:
%%% mode: latex
%%% TeX-master: "thesis"
%%% End:


\geometry{letterpaper,left=1.25in,top=1in,headsep=2\baselineskip,textwidth=26pc,marginparsep=2pc,marginparwidth=12pc,textheight=44\baselineskip,headheight=\baselineskip}


\usepackage{cmu-titlepage2}


\addbibresource{thesis.bib}

%% Workaround to \NoCaseChange and \tl_map_tokens:nn issue:
% \ExplSyntaxOn
% \cs_gset:Npn \NoCaseChange #1 { \use:n {#1} }
% \ExplSyntaxOff

\begin{document}

\frontmatter
% \pagenumbering{roman}
% \pagestyle{empty}

\title{Session-Typed Ordered Logical Specifications}

\author{Henry DeYoung}
\date{December 2020}
\Year{2020}
\trnumber{CMU-CS-20-133}

\committee{%
  Frank Pfenning (Chair)\\
  Iliano Cervesato\\
  Robert Harper\\
  Andr\'{e} Platzer\\
  Simon Gay (University of Glasgow)\\
  Carsten Sch\"{u}rmann (IT University of Copenhagen)%
}

\support{This research was sponsored by National Science Foundation award number: 1718276; by a National Science Foundation Graduate Research Fellowship award; and by a Google Lime Scholarship award.}
% This material is based upon work supported by

\disclaimer{The views and conclusions contained in this document are those of the author and should not be interpreted as representing the official policies, either expressed or implied, of any sponsoring institution, the U.S.\ government or any other entity.}

\keywords{concurrency, bisimilarity, session types, proof construction, proof reduction, ordered logic, singleton logic}

\maketitle

\cleardoublepage
\begin{center}
\vspace*{2in}
\textit{For my parents and brother}
\end{center}

\makeatletter
\clearpage
\thispagestyle{empty}
\@ifundefined{@keywords}{}{\vspace*{\fill} \noindent{\bf Keywords:} \@keywords}
\cleardoublepage
\makeatother

\begin{abstract}
\normalsize
Concurrent systems are ubiquitous, but notoriously difficult to get right: subtle races and deadlocks can lurk even in the most extensively tested of systems.
In a quest to tame concurrency, researchers have successfully applied the principle of \emph{computation as deduction} to concurrency in two distinct ways: \emph{concurrency as proof reduction} and \emph{concurrency as proof construction}.
These two approaches to concurrency have complementary advantages, with the proof-construction approach excelling at global specification of a system's dynamics, while the proof-reduction approach is best suited to implementation of the processes that comprise the system.

This document explores the relationship between these two different proof-theoretic characterizations of concurrency.
To focus on the essential aspects of their relationship, the exploration is carried out in the context of concurrent systems that have chain topologies.
From a proof-construction perspective, chain topologies arise from ordered logic; from a proof-reduction perspective, they arise from \emph{singleton logic}, a variant of ordered logic that restricts sequents to have exactly one antecedent.

In this context, a rewriting framework is systematically derived from the ordered sequent calculus, and a \emph{message-passing} fragment of that rewriting framework is identified.
String rewriting specifications of concurrent systems can be \emph{choreographed} into this fragment, and the fragment supports a notion of bisimilarity.
Along the way, we also uncover a \emph{semi-axiomatic sequent calculus} for singleton logic, which blends a standard sequent calculus with axiomatic aspects of Hilbert systems, and we then establish a correspondence between semi-axiomatic proof normalization and asynchronous message-passing communication.
Ultimately, the message-passing processes can be faithfully embedded within the message-pass\-ing ordered rewriting framework in a bisimilar way.
Perhaps surprising is that, because the embedding is left-invertible, we can also identify fairly broad conditions under which local, process implementations can be \emph{extracted} from global, message-passing ordered rewriting specifications.
\end{abstract}

\begin{acknowledgments}
  \normalsize
  First, and foremost, the debt of gratitude that I owe Frank Pfenning cannot be overstated.
  Frank has been such a wonderful advisor and mentor to me throughout my time here at CMU, always patient and kind.
  Thank you, Frank, for inviting me on many a hike through the woods of proof theory and programming languages with you.

  I also wish to thank the members of my thesis committee -- Iliano Cervesato, Bob Harper, Andr\'{e} Platzer, Simon Gay, and Carsten Sch\"{u}rmann -- for being so generous with their time in talking with me and for their detailed feedback on my thesis document.  

  I am also grateful to my co-authors and the other people with whom I've had invaluable discussions over the years:
  Stephanie Balzer, Lu\'{\i}s Caires, Ankush Das, Anupam Datta, Farzaneh Derakhshan, Deepak Garg, Dilsun K\i rl\i\ Kaynar, Limin Jia, Andreia Mordido, Klaas Pruiksma, Rob Simmons, and Bernardo Toninho.

  Also, I would like to thank to Mark Stehlik for guiding me toward logic and programming languages.
  What still impresses me even now, all these years out of undergrad, is how he was able to recognize that it was just the field for me -- before I had any inkling of that myself.

  I am also thankful for everyone over the years who has helped me with scribing and typing my school work.

  Last, but most importantly, I want to thank my parents and my brother for all their love, care, and support through all these nearly 36 years -- not through only the good times, but most especially through the hard times.
  I love you.
\end{acknowledgments}

\oddsidemargin=\saveoddsidemargin
\evensidemargin=\saveevensidemargin

\tableofcontents
% \listoffigures
% \listoftables

\mainmatter
% \pagenumbering{arabic}

% \pagestyle{fancy}

\chapter{Introduction}\label{ch:introduction}

\begin{itemize}
\item Computation as deduction: clear, expressive, and provably correct programs
  \begin{itemize}
  \item Examples of sucess stories
  \item Can it be applied to concurrency?
  \end{itemize}
\item Proof constuction and proof reduction views of con currency
  \begin{itemize}
  \item Proof construction: good for specifications 
  \item Proof reduction: good for implementions 
  \end{itemize}
\item Thesis statement: Session types bridge these two views
\item Ordered logic as a proving ground 
  \begin{itemize}
  \item Ordered rewriting for proof construction
  \item Singleton logic (purely additive fragment of ordered logic) for proof reduction
  \end{itemize}
\item Ordered rewriting (chapter 4) for specifications
  \begin{itemize}
  \item DFAs and NFAs 
  \item Binary counters
  \end{itemize}
\item Refinement of ordered rewriting for choreographies (chapter 5)
  \begin{itemize}
  \item Recursive definitions as processes; atoms as messages 
  \item Untyped (mostly, except for directions)
  \item Rewriting bisimilarity for observational equivalence
    \begin{itemize}
    \item Examples
    \end{itemize}
  \end{itemize}
\item Singleton logic and its semi-axiomatic calculus (chapter 6)
\item 
\end{itemize}


Concurrent systems are notoriously difficult to get right.

Beginning with Curry's observation that Hilbert [...] corresponds to a form of computation based on combinatory reduction\autocite{??}, and continuing with Howard's discovery of an isomorphism between [Gentzen's] intuitionistic natural deduction and Church's simply-typed $\lambda$-calculus, computation-as-deduction has been the gold standard for clear, expressive, and provably correct programs.

Computation-as-deduction can be divided into two classes: proof-search-as-computation and proof-reduction-as-computation.
The former provides a logically grounded basis for the backward- and forward-chaining logic programming paradigms, whereas the latter is the foundation for the functional programming paradigm.

Logically grounded concurrent computation 

More recently, a proof-reduction description of concurrency has been discovered by \textcite{??} with \textcite{??}.
In this isomorphism, linear propositions correspond to session types; sequent proofs, to session-typed processes; and cut reduction, to synchronous message-passing communication.

This thesis seeks to bring these two apparently divergent views of concurrency together.
Is there a class of specifications for which well-typed implementations can automatically be extracted?

Thesis statement: Session types form the bridge. 

%%% Local Variables:
%%% mode: latex
%%% TeX-master: "thesis"
%%% End:


\part{Preliminaries}\label{part:preliminaries}

\chapter{Binary relations and automata}\label{ch:automata}\label{ch:finite-automata}

In this \lcnamecref{ch:automata}, we review the definitions of alphabets, words, languages, and automata that we will use in the running examples throughout this document.
% The classic text by \textcite{Hopcroft+Ullman:??} is a excellent reference.
Our definitions, though equivalent to the classical ones found in \citeauthor{Hopcroft+:06}'s text\autocite{Hopcroft+:06}, differ slightly, having been tuned for the particular applications in this document.

First, however, we describe our notational conventions for binary relations.

\section{Binary relations}

In this and future \lcnamecrefs{ch:automata}, we make significant use of binary relations of various kinds.
These are often written in infix notation.

Given a binary relation $\simu{R}$, we write $\simu{R}^{-1}$ for its inverse.
For relations written as arrows, such as $\reduces$ and $\Reduces$, we often instead express their inverses by writing the arrow in the other direction.
For instance, $\secudeR$ would be the inverse of $\Reduces$, so that $y \secudeR x$ exactly when $x \Reduces y$.

We write the relational composition of $\simu{R}$ and $\simu{S}$ as juxtaposition, so that $x \simu{R}\simu{S} z$ holds exactly when there exists a $y$ such that $x \simu{R} y$ and $y \simu{S} z$.

\section{Alphabets, words, and languages}

An alphabet $\ialph$ is simply a set of symbols, $a \in \ialph$.
A finite word $w$ over the alphabet $\ialph$ is then a (possibly empty) finite sequence of symbols drawn from $\ialph$;
we denote the empty word by $\emp$.
The finite words form a free monoid under concatenation, with $\emp$ being the unit.
We denote by $\finwds{\ialph}$ the set of all finite words over $\ialph$.

It is also possible to construct infinite words.
An infinite word over the alphabet $\ialph$ is a countably infinite sequence of letters drawn from $\ialph$;
we denote the set of all infinite words over $\ialph$ by $\infinwds{\ialph}$.
We also use $\wds{\ialph}$ to denote the set of all words -- finite or infinite -- over the alphabet $\ialph$; that is, $\wds{\ialph} = \finwds{\ialph} \union \infinwds{\ialph}$.

A language is a set of words.
Depending on the context, it will be a subset of either $\wds{\ialph}$, $\infinwds{\ialph}$, or $\finwds{\ialph}$.

% It will sometimes be useful to work with an augmented alphabet.
% Given a symbol $c \notin \ialph$, we may form the augmented alphabet $\augalph{\ialph}{c} = \ialph \union \{c\}$.
% The augmented alphabet $\augalph{\ialph}{\emp}$ is the most

% \subsection{Endmarked alphabets and words}

% An endmarked alphabet $\ealph{\ialph}$ is a pair $\ealph{\ialph} = (, )$ of finite alphabets $ $ and $ $.

\section{Nondeterministic and \aclp*{DFA}}

\begin{definition}\label{def:finite-automata:nfa}
  \emph{\Iac{NFA}}
   over a finite input alphabet $\ialph$ is a triple $\aut{A} = (Q, \nfapow, F)$ consisting of:
  \begin{itemize}
  \item a finite set of \vocab{states}, $Q$;
  \item a \vocab{transition function}, $\nfapow\colon Q \times \ialph \to \pow{Q}$ such that $\nfapow(q, a) \neq \emptyset$ for all states $q \in Q$ and input symbols $a \in \ialph$; and
  \item a subset of \vocab{final states}, $F \subseteq Q$,
  \end{itemize}
  If $q' \in \nfapow(q, a)$, then we say that $q'$ is an \emph{$a$-successor} of $q$ and write $q \nfareduces[a] q'$.%
  \footnote{The condition placed on $\nfapow$ thus serves to ensure that, for all input symbols $a$, each state $q$ has an $a$-successor -- that is, that the \ac{NFA} $\aut{A}$ cannot get stuck.}

  The transition function $\nfapow$ can be lifted to a relation involving finite input words: For each word $w = a_1 \wc a_2 \dotsm a_n \in \finwds{\ialph}$, define a relation $\mathord{\nfareduces[\smash{w}]} \subseteq Q \times Q$ such that $q \nfareduces[w] q'$ when $q = q_0 \nfareduces[a_1] q_1 \nfareduces[a_2] \dotsb \nfareduces[a_n] q_n = q'$ for some sequence of states $q_0, q_1, \dotsc, q_n \in Q$.

  The \ac{NFA} $\aut{A}$ accepts input word $w$ from state $q$ if there exists a state $q' \in Q$ such that $q \nfareduces[w] q' \in F$;
  otherwise, the automaton rejects word $w$ from state $q$.
  The language of all words accepted by automaton $\aut{A}$ from state $q$ is denoted by $\autlang{\aut{A}}{q}$.%
  \footnote{We sometimes omit the subscript if the automaton is clear from the context.}
\end{definition}

Notice that, unlike the classical definition of \acp{NFA}, this definition does not fix an initial state for the automaton.
This is because we will be primarily interested in the operational aspects of \iac{NFA}, rather than its linguistic aspects.

\begin{example}
  \begin{marginfigure}[4.5\baselineskip]
    \begin{equation*}
      \mathllap{\aut{A}_1 = {}}
      \begin{tikzpicture}[baseline=(q_0.base)]
        \graph [automaton] {
          q_0
           -> [loop above, "a,b"]
          q_0
           -> ["b"]
          q_1 [accepting]
           -> ["a,b"]
          q_2
           -> [loop above, "a,b"]
          q_2;
        };
      \end{tikzpicture}
    \end{equation*}
    \caption{\Iac*{NFA} that accepts, from state $q_0$, exactly those words that end with $b$.}\label{fig:nfa-example-ends-b}
  \end{marginfigure}
  %
  As a concrete example, consider the \ac{NFA} $\aut{A}_1$ over the input alphabet $\ialph = \set{a, b}$ that is depicted in \cref{fig:nfa-example-ends-b}.
  This \ac{NFA} accepts, from state $q_0$, exactly those words that end with $b$.
  For comparison, the only word accepted from state $q_1$ is $\emp$.
  This \ac{NFA} is indeed nondeterministic, as both $q_0 \nfareduces[\smash{b}] q_0$ and $q_0 \nfareduces[\smash{b}] q_1$ hold.
\end{example}

\begin{definition}
  \emph{\Iacf{DFA}} over a finite input alphabet $\ialph$ is \iac{NFA} $\aut{A} = (Q, \nfapow, F)$ over $\ialph$ in which $\nfapow(q, a)$ is a singleton set for all states $q$ and input symbols $a$.
  In this case, we write $\dfanext$ for the function from $Q \times \ialph$ to $Q$ that underlies $\nfapow$.
\end{definition}

\begin{example}
  \begin{marginfigure}[0.8\baselineskip]
    \begin{equation*}
      \mathllap{\aut{A}_2 = {}}
      \begin{tikzpicture}[baseline=(s_0.base)]
        \graph [automaton] {
          s_0
           -> [loop above, "a"]
          s_0
           -> [bend left, "b"]
          s_1 [accepting]
           -> [loop above, "b"]
          s_1
           -> [bend left, "a"]
          s_0;
        };
      \end{tikzpicture}
    \end{equation*}
    \caption{\Iac*{DFA} that accepts, from state $s_0$, exactly those words that end with $b$.}\label{fig:dfa-example-ends-b}
  \end{marginfigure}
  \Cref{fig:dfa-example-ends-b} depicts \iac{DFA} over the input alphabet $\ialph = \set{a, b}$ that accepts, from state $s_0$, exactly those words that end with $b$.
  For comparison, the empty word $\emp$, too, is accepted from the state $s_1$.
\end{example}

\subsection{\acs*{NFA} bisimilarity}

In later \lcnamecrefs{ch:ordered-bisimilarity}, we will refer to a standard notion of bisimilarity for \acp{NFA}.

In general, two objects are bisimilar if they cannot be distinguished by an observer.
Here, \ac{NFA} bisimilarity is a relation on states, and the observer may provide an input word and observe whether the word is accepted or rejected by the given state.
% Two \ac{NFA} states are bisimilar if their successors are bisimilar and if they behave equivalently on the empty word, $\emp$.
%
\begin{definition}% [\ac*{NFA} bisimilarity]
  Let $\aut{A} = (Q, \nfapow, F)$ be \iac{NFA} over an input alphabet $\ialph$.
  An \emph{\acs{NFA} bisimulation} on $\aut{A}$ is a \emph{symmetric} binary relation on states, $\mathord{\simu{R}} \subseteq Q \times Q$, that satisfies the following conditions.
  \begin{thmdescription}[noitemsep]
  \item[Input bisimulation]
    If $q \simu{R}\nfareduces[a] s'$, then $q \nfareduces[a]\simu{R} s'$.
    % (See the adjacent \lcnamecref{fig:nfa-bisim:diagrams}.)% 
    \begin{marginfigure}
      \begin{equation*}
        % \begin{tikzcd}
        %   q \rar[reduces]{a} \dar[relation][swap]{\simu{R}}
        %     & q\mathrlap{'} \dar[relation, exists]{\simu{R}}
        %   \\[.5ex]
        %   s \rar[reduces, exists]{a} & s\smash{\mathrlap{'}}
        % \end{tikzcd}
        % \qquad
        \begin{tikzcd}
          q \rar[reduces, exists]{a} \dar[relation][swap]{\simu{R}}
            & q\mathrlap{'} \dar[relation, exists]{\simu{R}}
          \\[.5ex]
          s \rar[reduces]{a} & s\smash{\mathrlap{'}}
        \end{tikzcd}
        \quad\text{\raisebox{-1ex}{and}}\quad
        \begin{tikzcd}
          q \dar[relation][swap]{\simu{R}}
            \drar[relation, exists]{\textstyle\in}
          \\[.5ex]
          s \rar[phantom]{\in} &[-2em] F
        \end{tikzcd}
      \end{equation*}
      \caption{\Acs*{NFA} bisimilarity, in diagrams}\label{fig:nfa-bisim:diagrams}
    \end{marginfigure}%

  \item[Finality bisimulation]
    If $q \simu{R} s \in F$, then $q \in F$.
  \end{thmdescription}
  \vocab{\ac{NFA} bisimilarity} on $\aut{A}$, written $\asim_{\aut{A}}$, is the largest bisimulation on $\aut{A}$.
  We will usually omit the subscript because the automaton is nearly always clear from context.
\end{definition}

As a matter of convenience, \ac{NFA} bisimilarity is defined on a single \ac{NFA}.
If we wish to discuss the bisimilarity of states from distinct \acp{NFA}, we can form the disjoint union of the two \acp{NFA} and work with its bisimilarity relation.

Bisimilarity is an equivalence relation.
% Here we choose to keep the two automata distinct and prove reflexivity, symmetry, and transitivity, not in their strictest form, but in forms that apply to bisimilarity as defined above.
\begin{theorem}\label{thm:nfa-bisim-equiv}
  Let $\aut{A} = (Q, \nfapow, F)$ be \iac{NFA} over an input alphabet $\ialph$.
  \Ac{NFA} bisimilarity on $\aut{A}$ is reflexive, symmetric, and transitive:
  \begin{thmdescription}[nosep]
  \item[Reflexivity] Equality on $Q$ is a bisimulation on $\aut{A}$.
  \item[Symmetry] If $\simu{R}$ is a bisimulation on $\aut{A}$, then so is $\simu{R}^{-1}$.
  \item[Transitivity] If $\simu{R}$ and $\simu{S}$ are bisimulations on $\aut{A}$, then so is $\simu{R}\simu{S}$.
  \end{thmdescription}
\end{theorem}
\begin{proof}
  By the definition of bisimulation.
\end{proof}

% \begin{theorem}\label{thm:nfa-bisim-refl}
%   Let $\aut{A} = (Q, \nfapow, F)$ be \iac{NFA} over an input alphabet $\ialph$.
%   Then equality on $Q$ is a bisimulation between $\aut{A}$ and itself.
%   Moreover, $q \asim q$ for all $q \in Q$.
% \end{theorem}
% \begin{proof}
%   By checking that equality on $Q$ satisfies the conditions of a bisimulation between $\aut{A}$ and itself.
% \end{proof}

% \begin{theorem}\label{thm:nfa-bisim-sym}
%   Let $\aut{A}_1 = (Q_1, \nfapow_1, F_1)$ and $\aut{A}_2 = (Q_2, \nfapow_2, F_2)$ be \acp{NFA} over an input alphabet $\ialph$, and let $\simu{R}$ be \iac{NFA} bisimulations between $\aut{A}_1$ and $\aut{A}_2$.
%   Then its relational inverse, $\simu{R}^{-1}$, is a bisimulation between $\aut{A}_2$ and $\aut{A}_1$.
%   Moreover, for all $q_1 \in Q_1$ and $q_2 \in Q_2$, if $q_1 \asim q_2$, then $q_2 \asim q_1$.
% \end{theorem}
% \begin{proof}
%   By checking that $\simu{R}^{-1}$ satisfies the conditions of a bisimulation between $\aut{A}_2$ and $\aut{A}_1$.
% \end{proof}

% \begin{theorem}\label{thm:nfa-bisim-trans}
%   Let $\aut{A}_1 = (Q_1, \nfapow_1, F_1)$, $\aut{A}_2 = (Q_2, \nfapow_2, F_2)$, and $\aut{A}_3 = (Q_3, \nfapow_3, F_3)$ be \acp{NFA} over an input alphabet $\ialph$, and let $\simu{R}$ and $\simu{S}$ be \ac{NFA} bisimulations between $\aut{A}_1$ and $\aut{A}_2$ and between $\aut{A}_2$ and $\aut{A}_3$, respectively.
%   Then their relational composition, $\simu{R}\simu{S}$, is a bisimulation between $\aut{A}_1$ and $\aut{A}_3$.
%   Moreover, for all $q_1 \in Q_1$ and $q_2 \in Q_2$ and $q_3 \in Q_3$, if $q_1 \asim q_2$ and $q_2 \asim q_3$, then $q_1 \asim q_3$.
% \end{theorem}
% \begin{proof}
%   By checking that $\simu{R}\simu{S}$ satisfies the conditions of a bisimulation between $\aut{A}_1$ and $\aut{A}_3$.
%   % By proving that $\Set{ (q, r) \in Q_1 \times Q_3 \given \exists s \in Q_2.\, (q \asim s) \land (s \asim r) }$ is \iac{NFA} bisimulation between $\aut{A}_1$ and $\aut{A}_3$.
% \end{proof}

% % Notice that these \lcnamecrefs{thm:nfa-bisim-sym} are symmetry- and transitivity-like properties, but because distinct automata are used, they cannot be symmetry and transitivity in a strict sense.
% When \ac{NFA} bisimilarity is employed as an endorelation on the states of a single \ac{NFA}, $\aut{A}$, bisimilarity is a bona fide equivalence relation.%
% %
% \begin{corollary}
%   Let $\aut{A}$ be \iac{NFA} over an input alphabet $\ialph$.
%   \Ac{NFA} bisimilarity on $\aut{A}$ is reflexive, symmetric, and transitive.
% \end{corollary}
% % %
% % \begin{proof}
% %   \Ac{NFA} bisimilarity can be proved to be reflexive by showing that the state equality relation is \iac{NFA} bisimulation.
% %   \Cref{thm:nfa-bisim-sym,thm:nfa-bisim-trans} prove that \ac{NFA} bisimilarity on $\aut{A}$ is symmetric and transitive.
% % \end{proof}

The input bisimulation condition satisfied by \iac{NFA} bisimulation can be lifted to a condition on words, not just input symbols.
%
\begin{theorem}\label{thm:nfa-bisim:words}
  Let $\aut{A} = (Q, \nfapow, F)$ be \iac{NFA} over an input alphabet $\ialph$, and let $\simu{R}$ be \iac{NFA} bisimulation on $\aut{A}$.
  % Then $s \simu{R}^{-1}\nfareduces[w] q'$ implies $s \nfareduces[w]\simu{R}^{-1} q'$; moreover, 
  Then $q \simu{R}\nfareduces[w] s'$ implies $q \nfareduces[w]\simu{R} s'$.
\end{theorem}
%
\begin{proof}
  By induction over the structure of word $w$.
\end{proof}


\Ac{NFA} bisimilarity implies language equivalence.
%
\begin{theorem}
  Let $\aut{A} = (Q, \nfapow, F)$ be \iac{NFA} over an input alphabet $\ialph$.
  Then $q \asim s$ implies $\autlang{\aut{A}}{q} = \autlang{\aut{A}}{s}$.
\end{theorem}
%
\begin{proof}
  Because bisimilarity is symmetric~\parencref{thm:nfa-bisim-equiv}, it suffices to show that $q \asim s$ implies $\autlang{\aut{A}}{q} \subseteq \autlang{\aut{A}}{s}$.
  Let $q \in Q$ and $s \in Q$ be bisimilar states, and choose an arbitrary word $w$ that is accepted from state $q$.
  By definition, $q \nfareduces[w] q'_w \in F$ for some state $q'_w$.
  It follows from \cref{thm:nfa-bisim:words} and the finality bisimulation condition that $s \nfareduces[w] s'_w \in F$, for some state $s'_w$, and so $w$ is also accepted from state $s$.
\end{proof}

But, because of nondeterminism, the converse does not hold.
\begin{falseclaim}
  Let $\aut{A} = (Q, \nfapow, F)$ be \iac{NFA} over an input alphabet $\ialph$.
  Then $\autlang{\aut{A}}{q} = \autlang{\aut{A}}{s}$ implies $q \asim s$.
\end{falseclaim}
%
\begin{proof}[Counterexample]
  Choose the \acp{NFA} $\aut{A}_1$ and $\aut{A}_2$ given in \cref{fig:nfa-example-ends-b,fig:dfa-example-ends-b}.
  Although the languages accepted by states $q_0$ and $s_0$ are the same, the two states are \emph{not} bisimilar.

  For the sake of deriving a contradiction, assume that $q_0 \asim s_0$.
  % and its symmetric reflection, $s_0 \asim q_0$.
  Because $q_0$ is one of the $b$-successors of $q_0$, it follows by the input bisimulation condition that $s_0 \nfareduces[\smash{b}]\asim q_0$.
  But $s_1$ is the unique $b$-successor of $s_0$, and so we may deduce that $s_1 \asim q_0$.
  Just as $s_1$ is a final state, the finality bisimulation condition demands that $q_0$ be final, which it is not.
  From this contradiction, we conclude that $q_0$ and $s_0$ are \emph{not} bisimilar. 
\end{proof}

However, if both automata are \acp{DFA}, then language equivalence does imply bisimilarity.
%
\begin{theorem}
  Let $\aut{A} = (Q, \dfanext, F)$ be \emph{\iac{DFA}} over an input alphabet $\ialph$.
  Then $\autlang{\aut{A}}{q} = \autlang{\aut{A}}{s}$ implies $q \asim s$.
\end{theorem}
%
\begin{proof}
  Let $\mathord{\simu{R}} = \Set{ (q, s) \given \autlang{\aut{A}}{q} = \autlang{\aut{A}}{s} }$; we will prove that $\simu{R}$ is a bisimulation on the \ac{DFA} $\aut{A}$.
  As the largest bisimulation, bisimilarity will then contain $\simu{R}$.
  %
  \begin{description}[parsep=0pt, listparindent=\parindent]
  \item[Input bisimulation]
    Assume that $q \simu{R} s \dfareduces[a] s'_a$ for some state $s$; we must show that $q \dfareduces[a]\simu{R} s'_a$.
    Because $\aut{A}$ is deterministic, it suffices to show that $\autlang{\aut{A}}{q'_a} = \autlang{\aut{A}}{s'_a}$, where $q'_a$ is the unique $a$-successor of $q$.

    Choose an arbitrary word $w$ from the language accepted from state $q'_a$.
    Then $a \wc w$ is in the language accepted from state $q$, and, because $q$ and $s$ are $\simu{R}$-related, also in the language accepted from $s$.
    Because $\aut{A}$ is deterministic, this can only be if $w$ is in the language accepted from state $s'_a$, the unique $a$-successor of $s$.
    Thus $\autlang{\aut{A}}{q'_a} \subseteq \autlang{\aut{A}}{s'_a}$.
    By symmetric reasoning, $\autlang{\aut{A}}{q'_a} \supseteq \autlang{\aut{A}}{s'_a}$.
    It follows that $q'_a \simu{R} s'_a$.

  \item[Finality bisimulation]
    Assume that $q \simu{R} s \in F$; we must show that $q \in F$.
    Because $s$ is a final state, it accepts the empty word, $\emp$.
    The state $q$ must also accept the empty word, because $\autlang{\aut{A}}{q} = \autlang{\aut{A}}{s}$.
    A state accepts the empty word only if it is a final state, so it follows that $q \in F$.
  %
  \qedhere
  \end{description}
\end{proof}


% \subsection{\Aclp*{NFA} with $\emp$-transitions}

% \begin{definition}
%   \Iac{NFA} with $\emp$-moves over a finite alphabet $\ialph$ is a triple $\aut{A} = (Q, \mathord{\nfareduces}, F)$ consisting of:
%   \begin{itemize}
%   \item a finite set of \vocab{states}, $Q$;
%   \item a \vocab{transition relation}, $\mathord{\nfareduces} \subseteq (Q \times \ialph) \times Q$; and
%   \item a subset of \vocab{final states}, $F \subseteq Q$.
%   \end{itemize}
%   We will write $q \nfaReduces[a] q'$ whenever $q \nfareduces[\emp] \dotsb \nfareduces[\emp] \nfareduces[a] \nfareduces[\emp] \dotsb \nfareduces[\emp] q'$.

%   The transition relation can be lifted to one involving finite input words: For each input word $w = a_1 \wc a_2 \dotsm a_n$, let $q \nfaReduces[w] q'$ if $q = q_0 \nfaReduces[a_1] q_1 \nfaReduces[a_2] \dotsb \nfaReduces[a_n] q_n = q'$ for some sequence of states $q_0, q_1, \dotsc, q_n \in Q$.

%   The automaton $\aut{A}$ accepts input word $w$ from state $q$ if there exists a state $q' \in Q$ such that $q \nfaReduces[w] q' \in F$;
%   otherwise, the automaton rejects word $w$ from state $q$.
% \end{definition}

% \begin{marginfigure}
%   \begin{tikzcd}
%     \graph {
      
%   \end{tikzcd}
% \end{marginfigure}


\section{Infinite-word sequential transducers}

Sequential and subsequential transducers\autocites{Ginsburg+Rose:CJM66}{Schuetzenberger:TCS77}  are usually described in terms of finite words.
In this document, we are interested in transducers for their computational behavior rather than for the functions they induce.
This, together with technical considerations that will become apparent in \cref{ch:process-chains}, means that we are only interested in \emph{infinite}-word sequential transducers.
% It is also possible to consider finite-word sequential (and subsequential) transducers, but in this document we are only interested in infinite-word sequential transducers, for reasons that will become clear in \cref{ch:process-chains}.

The following definition is adapted from \textfootcite{Beal+Carton:TCS02}.
\begin{definition}
  An \emph{infinite-word sequential transducer} over the finite input and output alphabets $\ialph$ and $\oalph$, respectively, is a triple $\aut{T} = (Q, \sftnext, \sftout)$ consisting of:
  \begin{itemize}
  \item a finite set of \vocab{states}, $Q$;
  \item a \vocab{transition function}, $\sftnext\colon Q \times \ialph \to Q$; and
  \item an \vocab{output function}, $\sftout\colon Q \times \ialph \to \finwds{\oalph}$.
  \end{itemize}
  We may define a function $\mathord{\Downarrow}\colon Q \times \infinwds{\ialph} \to \infinwds{\oalph}$ as the largest function such that $(q, a \wc w) \Downarrow \sftout(q, a) \wc v$ if $(\sftnext(q, a), w) \Downarrow v$.
  The transducer $\aut{T}$ maps, from state $q$, the infinite input word $w$ into the infinite output word $v$ if $(q, w) \Downarrow v$.
\end{definition}


\begin{example}
\begin{marginfigure}[5\baselineskip]
  \begin{equation*}
    \mathllap{\aut{T} = {}}
    \begin{tikzpicture}[baseline=(q_0.base)]
      \graph [automaton] {
        q_0
         -> [loop above, "$\tio{a | a}$"]
        q_0
         -> [bend left, "$\tio{b | b}$"]
        q_1 [right=-0.5em of q_0]
         -> [loop above, "$\tio{b | \emp}$"]
        q_1
         -> [bend left, "$\tio{a | a}$"]
        q_0 ;
        % e_0 [coordinate, below=-1.75em of q_0.south];
        % e_1 [coordinate, below=-5.25em of q_1.south];
        % (q_0.south) -> ["$\emp$", swap] e_0 ;
        % (q_1.south) -> ["$\vphantom{\emp}\smash{b}$"] e_1 ;
      };
    \end{tikzpicture}
  \end{equation*}
  \caption{An infinite-word sequential transducer that compresses runs of consecutive $b$s}\label{fig:finite-automata:sft-example}
\end{marginfigure}
  %
  As a concrete example consider the infinite-word transducer $\aut{T}$ over the input and output alphabets $\ialph = \oalph = \Set{ a, b }$ that is depicted in \cref{fig:finite-automata:sft-example}.
  This transducer maps, from state $q_0$, infinite input words into words in which each run of $b$s has been compressed into a single $b$.
  For instance, it maps $w = abbaabbba\dotsm$ to $v = abaaba\dotsm$ because $(q_0, w) \Downarrow v$.
\end{example}



% Notice that \acp{DFA} may be thought of as specialized \acp{SFT} that transduce each word to one of two output words to indicate acceptance.
% In other words, \iac{DFA} $\aut{A} = (Q, \dfanext, F)$ over an input alphabet $\ialph$ is isomorphic to the \ac{SFT} $\aut{T} = (Q, \dfanext, \sftout, \sftterm)$ over the in put aphabet $\ialph$ and an output alphabet $\oalph = \Set{ y, n }$ where:
% $\sftout(q, a) = \emp$ for all $(q, a) \in Q \times \ialph$; and
% $\sftterm(q) = y$ for all $q \in F$ and $\sftterm(q) = n$ otherwise.


% \section{\Aclp*{DPDA}}

% \begin{definition}
%   \Iac{DPDA} over a finite input alphabet $\ialph$ and a finite stack alphabet $\salph$ is a triple $\aut{A} = (Q, \mathord{\nfareduces}, F)$ consisting of:
%   \begin{itemize}
%   \item a finite set of \vocab{states}, $Q$;
%   \item a \vocab{transition relation} on state-stack pairs, $\mathord{\nfareduces} \subseteq (Q \times \finwds{\salph}) \times \augalph{\ialph}{\emp} \times (Q \times \finwds{\salph})$; and
%   \item a subset of \vocab{final states}, $F \subseteq Q$.
%   \end{itemize}

%   We will write $q \nfareduces[a] q'$ whenever $((q, a), q') \in \mathord{\nfareduces}$.

%   The transition relation can be lifted to one involving finite input words: For each word $w = \alpha_1 \wc \alpha_2 \dotsm \alpha_n \in \finwds{\ialph}$, let $q \nfareduces[w] q'$ if $q = q_0 \nfareduces[\alpha_1] q_1 \nfareduces[\alpha_2] \dotsb \nfareduces[\alpha_n] q_n = q'$ for some sequence of states $q_0, q_1, \dotsc, q_n \in Q$.

%   The \ac{DPDA} $\aut{A}$ accepts input word $w$ from state $q$ if there exists a state $q' \in Q$ such that $q \nfareduces[w] q' \in F$;
%   otherwise, the automaton rejects word $w$ from state $q$.
%   The language of all words accepted by automaton $\aut{A}$ from state $q$ is denoted $\autlang{\aut{A}}{q}$.%
%   \footnote{We sometimes omit the subscript if the automaton is clear from the context.}

%   $(q, s) = (q_0, s_0) \nfareduces[\alpha_1] (q_1, s_1) \nfareduces[\alpha_2] \dotsb \nfareduces[\alpha_n] (q', s')$ with $q' \in F$.
% \end{definition}

\section{Turing machines}

In \cref{sec:process-chains:turing-machines}, we will construct session-typed processes that represent Turing machines.
We are interested in how these machines compute, but not interested in what functions they compute.
In other words, our focus is on computation, not computability.

This means that our definitions of Turing machines differ from the definitions traditionally used, such as in \citeauthor{Hopcroft+:06}'s text\autocite{Hopcroft+:06}.
For example, we use infinite words to describe truly infinite tapes, rather than using finite words to describe the frontiers of unbounded tapes, and our machines also have no notion of acceptance.
The specific reasons for these differences will become clear in \cref{sec:process-chains:turing-machines}.

\begin{definition}
  A \vocab{two-way infinite tape Turing machine} over a finite alphabet $\ialph$ is a pair $\aut{M} = (Q, \delta)$ consisting of:
  \begin{itemize}
  \item a finite set of \emph{states}, $Q$; and
  \item a \emph{transition function} $\delta\colon Q \times \ialph \to Q \times \ialph \times \Set{\mathsf{L}, \mathsf{R}}$.
  \end{itemize}
  The two-way infinite tape is divided into two one-way halves, with the machine's \emph{head} placed between the two halves.
  A \emph{configuration} of $\aut{M}$ is thus either $w \itlhead{q} v$ or $w \itrhead{q} v$ for left-infinite word $w \in \ialph^{\bar{\omega}}$, right-infinite word $v \in \infinwds{\ialph}$, and state $q \in Q$.%
  \footnote{In other words, configurations are drawn from $\ialph^{\bar{\omega}} \times (\bigcup_{q \in Q} \Set{\!{} \itlhead{q} {}, {} \itrhead{q} {}\!}) \times \infinwds{\ialph}$.}
  The machine's head faces either the left- or right-hand half of the tape, as indicated by the notation surrounding the state $q$.

  To describe the machine's moves, we define a function $\treduces$ on configurations.
  For a left-facing head, this function is given by:
  \begin{equation*}
      w \wc a \itlhead{q} v \treduces
        \begin{cases*}
          w \itlhead{q'} b \wc v & if $\delta(q, a) = (q', b, \mathsf{L})$ \\
          w \wc b \itrhead{q'} v & if $\delta(q, a) = (q', b, \mathsf{R})$
        \end{cases*}
  \end{equation*}
  Symmetrically, for a right-facing head, this function is given by:
  \begin{equation*}
      w \itrhead{q} a \wc v \treduces
        \begin{cases*}
          w \itlhead{q'} b \wc v & if $\delta(q, a) = (q', b, \mathsf{L})$ \\
          w \wc b \itrhead{q'} v & if $\delta(q, a) = (q', b, \mathsf{R})$
        \end{cases*}
  \end{equation*}
  The direction that the head faces indicates the next symbol to be read from the tape.
  When a left-facing head is instructed to move right or a right-facing head is instructed to move left, the head's direction changes but its placement between the two tape halves does not.
\end{definition}


% \begin{definition}
%   A \vocab{two-way infinite tape Turing machine} over a finite alphabet $\ialph$ is a tuple $\aut{M} = (Q, \tblank, \delta)$ consisting of:
%   \begin{itemize}
%   \item a finite set of \emph{states}, $Q$;
%   \item a distinguished \emph{blank symbol}, $\tblank \in \ialph$; and
%   \item a \emph{transition function} $\delta\colon Q \times \ialph \to Q \times \ialph \times \Set{\mathsf{L}, \mathsf{R}}$.
%   \end{itemize}
%   The two-way infinite tape is divided into two one-way halves, with the machine's \emph{head} placed between the two halves.
%   A \emph{configuration} of $\aut{M}$ is thus either $w \itlhead{q} v$ or $w \itrhead{q} v$ for finite words $w, v \in \finwds{\ialph}$ and state $q \in Q$.%
%   \footnote{In other words, configurations are drawn from $\finwds{\ialph} \times (\bigcup_{q \in Q} \Set{\!{} \itlhead{q} {}, {} \itrhead{q} {}\!}) \times \finwds{\ialph}$.}
%   The machine's head faces either the left- or right-hand half of the tape, as indicated by the notation surrounding the state $q$.
% %  The words $w$ and $v$ in these configurations are finite

%   To describe the machine's moves, we define a function $\treduces$ on configurations.
%   For a left-facing head, this function is given by:
%   \begin{equation*}
%     \begin{lgathered}
%       w \wc a \itlhead{q} v \treduces
%         \begin{cases*}
%           w \itlhead{q'} b \wc v & if $\delta(q, a) = (q', b, \mathsf{L})$ \\
%           w \wc b \itrhead{q'} v & if $\delta(q, a) = (q', b, \mathsf{R})$
%         \end{cases*}
%       \\
%       \emp \itlhead{q} v \treduces \tblank \itlhead{q} v
%     \end{lgathered}
%   \end{equation*}
%   Symmetrically, for a right-facing head, this function is given by:
%   \begin{equation*}
%     \begin{lgathered}
%       w \itrhead{q} a \wc v \treduces
%         \begin{cases*}
%           w \itlhead{q'} b \wc v & if $\delta(q, a) = (q', b, \mathsf{L})$ \\
%           w \wc b \itrhead{q'} v & if $\delta(q, a) = (q', b, \mathsf{R})$
%         \end{cases*}
%       \\
%       w \itrhead{q} \emp \treduces w \itrhead{q} \tblank
%     \end{lgathered}
%   \end{equation*}
%   The direction that the head faces indicates the next symbol to be read from the tape.
%   In case a left-facing head is instructed to move right or a right-facing head is instructed to move left, the head's direction changes but its placement between the two tape halves does not.
%   If the head reaches the finite frontier, a blank is allocated.
% \end{definition}

\begin{definition}
  A \vocab{one-way infinite tape Turing machine} over a finite alphabet $\ialph$ is a tuple $\aut{M} = (Q, \delta, F)$ consisting of:
  \begin{itemize}
  \item a finite set of \emph{states}, $Q$;
  \item a \emph{transition function} $\delta\colon Q \times \ialph \to Q \times \ialph \times \Set{\mathsf{L}, \mathsf{R}}$; and
  \item a subset of \emph{final states}, $F \subseteq Q$.
  \end{itemize}
  Because the tape is only one-way infinite, a configuration of machine $\aut{M}$ is either $w \itlhead{q} v$ or $w \itrhead{q} v$ for left-infinite word $w \in \ialph^{\bar{\omega}}$, \emph{finite} word $v \in \finwds{\ialph}$, and state $q \in Q$; \ie, configurations are drawn from $\ialph^{\bar{\omega}} \times (\bigcup_{q \in Q} \Set{\!{} \itlhead{q} {}, {} \itrhead{q} {}\!}) \times \finwds{\ialph}$.
  % The machine's head faces either the left- or right-hand half of the tape, as indicated by the notation surrounding the state $q$.
%  The words $w$ and $v$ in these configurations are finite

  To describe the machine's moves, we define a function $\treduces$ on configurations.
  For a left-facing head, this function is given by:
  \begin{equation*}
      w \wc a \itlhead{q} v \treduces
        \begin{cases*}
          w \itlhead{q'} b \wc v & if $\delta(q, a) = (q', b, \mathsf{L})$ \\
          w \wc b \itrhead{q'} v & if $\delta(q, a) = (q', b, \mathsf{R})$
        \end{cases*}
  \end{equation*}
  This is exactly as it was for the two-way infinite tape Turing machines.
  Because the tape is only one-way infinite, the right-facing head's treatment differs:
  \begin{equation*}
    \!\begin{aligned}
      w \itrhead{q} a \wc v \treduces {} &
        \begin{cases*}
          w \itlhead{q'} b \wc v & if $\delta(q, a) = (q', b, \mathsf{L})$ \\
          w \wc b \itrhead{q'} v & if $\delta(q, a) = (q', b, \mathsf{R})$
        \end{cases*}
      \\
      w \itrhead{q} \emp \treduces {} &
        \begin{cases*}
          w & if $q \in F$ \\
          w \itlhead{q} \emp & if $q \notin F$
        \end{cases*}
    \end{aligned}
  \end{equation*}
  When a right-facing head reaches the finite end of the tape, its behavior depends on whether the state is final.
  If the state is final, the machine terminates; otherwise, the machine remains in the same state but effectively moves left one symbol (by turning to face left).
\end{definition}

Notice that the machine does not terminate as soon as it enters a final state -- it must also reach the finite end of the tape.
It is up to the machine's programmer, by appropriately crafting the transition function $\delta$, to ensure that final states eventually lead the machine to the tape's finite end.

% \begin{definition}
%   A \vocab{one-way infinite tape Turing machine} over a finite alphabet $\ialph$ is a tuple $\aut{M} = (Q, \tblank, \delta)$ consisting of:
%   \begin{itemize}
%   \item a finite set of states, $Q$;
%   % \item a finite input alphabet, $\ialph \subset \oalph$;
%   \item a distinguished blank symbol, $\tblank \in \ialph$; and
%   \item a \emph{transition function} $\delta\colon Q \times \ialph \to Q \times \ialph \times \Set{\mathsf{L}, \mathsf{R}}$.
%   \end{itemize}
%   A configuration of $\aut{M}$ is either $w \itlhead{q} v$ or $w \itrhead{q} v$ for words $w,v \in \finwds{\ialph}$ and states $q \in Q$.%
%   \footnote{In other words, configurations are drawn from $\finwds{\ialph} \times (\bigcup_{q \in Q} \Set{\itlhead{q}, \itrhead{q}}) \times \finwds{\ialph}$.}
%   The middle portion of a configuration, either $\itlhead{q}$ or $\itrhead{q}$, represents $\aut{M}$'s \emph{finite control}, which consists of the machine's current state together with the direction of the next symbol to be read.

%   We define a function $\treduces$ on these configurations.
%   If the machine's finite control is facing leftward and reads an $a$, it writes a symbol in place of the $a$, optionally changes the finite control's state, and then either advances or turns to face rightward, according to the transition function $\delta$.
%   \begin{equation*}
%     % \begin{lgathered}
%       w \wc a \itlhead{q} v \treduces
%         \begin{cases*}
%           w \itlhead{q'} b \wc v & if $\delta(q, a) = (q', b, \mathsf{L})$ \\
%           w \wc b \itrhead{q'} v & if $\delta(q, a) = (q', b, \mathsf{R})$
%         \end{cases*}
%     %   \\
%     %   \emp \itlhead{q} v \treduces \tblank \itlhead{q} v
%     % \end{lgathered}
%   \end{equation*}
%   Otherwise, if the configuration is $\emp \itlhead{q} v$, then the finite control has reached the left-hand frontier of the tape; because the tape is unbounded to the left, a new, blank cell is allocated in this case.
%   \begin{equation*}
%     \emp \itlhead{q} v \treduces \tblank \itlhead{q} v
%   \end{equation*}

%   A right-facing finite control is treated symmetrically, except that the right-hand end of the tape is fixed.
%   To ensure that the finite control does not run off the end of the tape, its direction
%   \begin{equation*}
%     \begin{lgathered}
%       w \itrhead{q} a \wc v \treduces
%         \begin{cases*}
%           w \itlhead{q'} b \wc v & if $\delta(q, a) = (q', b, \mathsf{L})$ \\
%           w \wc b \itrhead{q'} v & if $\delta(q, a) = (q', b, \mathsf{R})$
%         \end{cases*}
%       \\
%       w \itrhead{q} \emp \treduces w \itlhead{q} \emp
%     \end{lgathered}
%   \end{equation*}

% \end{definition}

% \begin{definition}
%   A \vocab{two-way infinite tape Turing machine} over a finite alphabet $\ialph$ is a tuple $(Q, \tblank, \delta)$ consisting of:
%   \begin{itemize}
%   \item a finite set of states, $Q$;
%   \item a distinguished blank symbol, $\tblank \in \ialph$; and
%   \item a transition function, $\delta\colon Q \times \ialph \to Q \times \ialph \times \Set{\mathsf{L}, \mathsf{R}}$.
%   \end{itemize}
%   A configuration of $\aut{M}$ is a triple $(w, q, v) \in \finwds{\ialph} \times \bigl((\Set{\mathord{\itlhead}} \times Q) \union (Q \times \Set{\mathord{\itrhead}})\bigr) \times \finwds{\ialph}$.
%   Depending on the direction that the machine's head is facing and whether it is at the frontier, there are several possible transitions:
%   \begin{equation*}
%     \begin{tabular}{@{}r@{\enspace}l@{\enspace}l@{}}
%       $w \wc a \itlhead q \oc v \treduces w \itlhead q'_a \oc b \wc v$ \\
%       and $w \oc q \itrhead a \wc v \treduces w \itlhead q'_a \oc b \wc v$
%         & if & $\delta(q, a) = (q'_a, b, \mathsf{L})$
%       \\[1ex]
%       $w \wc a \itlhead q \oc v \treduces w \wc b \oc q'_a \itrhead v$ \\
%       and $w \oc q \itrhead a \wc v \treduces w \wc b \oc q'_a \itrhead v$
%         & if & $\delta(q, a) = (q'_a, b, \mathsf{R})$
%       \\[1ex]
%       $\emp \itlhead q \oc v \treduces \tblank \itlhead q \oc v$ \\
%       and $w \oc q \itrhead \emp \treduces w \oc q \itrhead \tblank$
%     \end{tabular}
%   \end{equation*}
% \end{definition}


% \section{Chains of communicating automata}
% \fixnote{Remove these?}

% \begin{definition}
%   A \emph{chain of communicating automata} over a finite alphabet $\ialph$ is a tuple $\aut{C} = (Q, (\ialph_q)_{q \in Q}, (\oalph_q)_{q \in Q})$ consisting of:
%   \begin{itemize}
%   \item a finite set of \vocab{states}, $Q$, that is partitioned into: left- and right-reading states, $\rL{Q}$ and $\rR{Q}$; left- and right-writing states, $\wL{Q}$ and $\wR{Q}$; forking states, $\fS{Q}$; and halting states, $\hS{Q}$;
%   \item a state-indexed set of finite \vocab{left-hand alphabets}, $(\ialph_q)_{q \in Q}$, and a state-indexed set of finite \vocab{right-hand alphabets}, $(\oalph_q)_{q \in Q}$;
%   \item $\rL{\delta} \colon \Pi q{:}\rL{Q}. \ialph_q \to Q$, with the condition that $\ialph_q = \ialph_{q'}$ and $\oalph_q = \oalph_{q'}$ for all $q' \in \cod{\rL{\delta}_q}$, and $\rR{\delta} \colon (\exists q{:}\rR{Q}. \oalph_q) \to Q$;
%   \item $\wL{\delta} \colon \wL{Q} \to \exists q'{:}Q. \ialph_{q'}$ and $\wR{\delta} \colon \wR{Q} \to \exists q'{:}Q. \oalph_{q'}$; and
%   \item $\fS{\delta} \colon \fS{Q} \to Q \times Q$.
%   \end{itemize}

%   \vocab{Chain configurations}, $c$ and $d$, consist of a finite sequence of states $q_1, q_2, \dotsc, q_n \in Q$ with, for all $1 \leq i < n$, a finite word drawn from $\finwds{\ialph_{q_{i+1}}}$ between neighboring states $q_i$ and $q_{i+1}$.
%   In addition, a finite word drawn from $\finwds{\ialph_{q_1}}$ and a finite word drawn from $\finwds{\oalph_{q_n}}$ bracket the configuration.
%   Formally, then, a chain configuration is a string drawn from the set
%   \begin{equation*}
%     \finwds{\ialph_{q_1}} q_1 \finwds{\ialph_{q_2}} q_2 \dotsm \finwds{\ialph_{q_n}} q_n \finwds{\oalph_{q_n}}
%     \,,
%   \end{equation*}
%   for some finite sequence of states $q_1, q_2, \dotsc, q_n \in Q$.
%   The chain configuration is \vocab{well-formed} if $\oalph_{q_i} = \ialph_{q_{i+1}}$ for all $1 \leq i < n$.

%   \begin{itemize}
%   \item $c \wc a \,q\, d \reduces c \,q'\, d$ if $\rL{\delta}_q(a) = q'$, and $c \,q\, a \wc d \reduces c \,q'\, d$ if $\rR{\delta}_q(a) = q'$;
%   \item $c \,q\, d \reduces c \wc a \,q'\, d$ if $\wL{\delta}(q) = (q', a)$, and $c \,q\, d \reduces c \,q'\, a \wc d$ if $\wR{\delta}(q) = (q', a)$;

%   \item $c \,q\, d \reduces c \, q' \, q'' \, d$ if $\fS{\delta}(q) = (q', q'')$;
%   \item $c \,q\, d \reduces c \, d$ if $q \in \hS{Q}$ and $\ialph_q = \oalph_q$.
%   \end{itemize}
% \end{definition}


%%% Local Variables:
%%% mode: latex
%%% TeX-master: "thesis"
%%% End:

\chapter{Ordered logic}\label{ch:ordered-logic}

% In \citeyear{Lambek:??}, \citeauthor{Lambek:??} published a seminal paper developing a formal system for describing sentence structure.
% The Lambek calculus, when viewed from a logical perspective, forms the core of \emph{intuitionistic ordered logic}%
% \footnote{Also known as intuitionistic noncommutative linear logic.}%
% .


% [This \lcnamecref{ch:ordered-logic} serves to review a sequent calculus presentation of ordered logic, also known as the (full) Lambek calculus\autocite{Lambek:AMM58}.]

In its traditional form, intuitionistic logic\footnote{And classical logic, too.} presumes that hypotheses admit three structural properties
\begin{description*}[
  mode=unboxed,
  before=\unskip:\space,
  font=\normalfont\itshape, afterlabel={,\space},
  itemjoin=;\space, itemjoin*=; and\space%
]
\item[weakening] that hypotheses need not be used
\item[contraction] that hypotheses may be reused indefinitely
\item[exchange] that hypotheses may be freely permuted
\end{description*}.

Substructural logics are so named because they reject some or all of these structural properties.
Most famously, linear logic\autocite{Girard:TCS87} is substructural because it rejects both weakening and contraction.
The result is a system in which each hypothesis must be used exactly once; accordingly, linear hypotheses may be viewed as consumable resources\autocite{Girard:TCS87}.

Ordered logic, also known as the (full) Lambek calculus,\autocites{Lambek:AMM58}{Lambek:SLIM61}{Abrusci:MLQ90}{Kanazawa:LLI92} goes a substructural step further.
Like its linear cousin, ordered logic rejects weakening and contraction, making ordered hypotheses resources, too.
But ordered logic additionally eschews exchange; 
ordered hypotheses are resources that must remain in order, with no reshuffling.


This \lcnamecref{ch:ordered-logic}
serves to review a sequent calculus presentation of ordered logic.
% \Citeauthor{Lambek:AMM58} originally developed the calculus to give a formal description of sentence structure.
% In this work, however, our interest is not mathematical linguistics but the logical foundations of concurrent computation.
%
%
%
%
% As a substructural logic, ordered logic eschews the usual structural properties of antecedents -- weakening, contraction, and exchange.
% As in \citeauthor{Girard:TCS87}'s linear logic, the lack of weakening and contraction properties means that each antecedent must be used exactly once within a proof.
% Ordered logic's additional lack of an exchange property means that antecedents must also remain in order within a proof.
% 
\Citeauthor{Lambek:AMM58} leveraged the noncommutativity of antecedents to give a formal description of sentence structure.
In this work, however, our interest is not mathematical linguistics but the logical foundations of concurrent computation.
Accordingly, the description of ordered logic in this \lcnamecref{ch:ordered-logic} has a proof-theoretic emphasis and is derived from presentations by \citeauthor{Polakow+Pfenning:MFPS99}\autocites{Polakow+Pfenning:MFPS99}{Pfenning:CMU16}.

\newthought{\Cref{sec:ordered-logic:sequent-calculus} introduces} ordered logic's sequent calculus as a collection of inference rules, informally justifying them with a resource interpretation similar to that of linear logic.

For this collection of rules to constitute a well-defined logic, it must have a verificationist meaning-theory in the tradition of \citeauthor{Gentzen:MZ35}, \citeauthor{Dummett:WJ76}, and \citeauthor{Martin-Lof:Siena83}\autocites{Gentzen:MZ35}{Dummett:WJ76}{Martin-Lof:Siena83}.
Together, the cut elimination and identity elimination \lcnamecrefs{thm:ordered-logic:cut-elimination}~\parencref[\cref{thm:ordered-logic:cut-elimination,thm:ordered-logic:id-elimination},]{sec:ordered-logic:verifications} serve to establish a proof normalization result: every proof has a corresponding verification.

\Cref{sec:ordered-logic:circular} sketches an extension of the ordered sequent calculus with circular propositions and proofs, and \cref{sec:ordered-logic:extensions} briefly describes several other extensions that are possible.

The reader who is familiar with ordered logic's sequent calculus and its basic metatheory -- particularly the cut elimination result -- should feel free to skip this \lcnamecref{ch:ordered-logic}.


\section{A sequent calculus presentation of ordered logic}\label{sec:ordered-logic:sequent-calculus}

The full sequent calculus for ordered logic will be summarized in \cref{fig:ordered-logic:sequent-calculus}, but first we will discuss the calculus's judgmental principles and logical connectives one by one.

\subsection{Sequents and contexts}

\paragraph{Seq\-uents}
In ordered logic's sequent calculus presentation, the basic judgment is a sequent
\begin{equation*}
  \oseq{A_1 \oc A_2 \dotsb A_n |- A} \,,
\end{equation*}
where the propositions $A_1, A_2, \dotsc, A_n$ are assumptions, or \emph{antecedents}, that are arranged into an ordered list%
% \footnote{Antecedents may be freely reassociated, and so form a list, not a tree.}%
% , not a multiset as in linear logic, nor a set as in intuitionistic logic
; the proposition $A$ is termed the \emph{consequent}.

Ordered logic eschews the usual structural properties of antecedents -- namely weakening, contraction, and exchange.
As in linear logic, the absence of weakening and contraction means that
% the
antecedents
% , $A_1 \oc A_2 \dotsb A_n$,
may neither be discarded nor duplicated within a proof.
Neither a proof of $\oseq{A_2 \dotsb A_n |- A}$ nor of $\oseq{A_1 \oc A_1 \oc A_2 \dotsb A_n |- A}$ implies a proof of $\oseq{A_1 \oc A_2 \dotsb A_n |- A}$, for example.
But ordered logic's rejection of the exchange property takes things one step further: antecedents may not even be permuted within a proof.
For example, $\oseq{A_2 \oc A_1 \dotsb A_n |- A}$ does not imply $\oseq{A_1 \oc A_2 \dotsb A_n |- A}$.

Like linear sequents\autocite{Girard:TCS87}, ordered sequents can be given a resource interpretation -- but with a slight twist.
% An ordered sequent $\oseq{A_1 \oc A_2 \dotsb A_n |- A}$ can be read as stating that resource $A$ can be produced from the resources $A_1, A_2, \dotsc, A_n$, and a proof of that sequent is a recipe for
A proof of an ordered sequent
$\oseq{A_1 \oc A_2 \dotsb A_n |- A}$
can be interpreted as a recipe for producing resource $A$ from the resources $A_1 \oc A_2 \dotsb A_n$.
The small twist is that these resources are inherently ordered and may not be permuted, exactly because ordered logic rejects the exchange property that linear logic admits.

\paragraph{Contexts}
To keep the notation for sequents concise, the list of antecedents is usually packaged into an ordered context $\octx = A_1 \oc A_2 \dotsb A_n$, with the sequent then written $\oseq{\octx |- A}$.
Algebraically, ordered contexts $\octx$ form a free noncommutative monoid:
\begin{equation*}
  \octx , \lctx \Coloneqq \octx_1 \oc \octx_2 \mid \octxe \mid A \,,
\end{equation*}
where the monoid operation is concatenation, denoted by juxtaposition, and the unit element is the empty context, denoted by $(\octxe)$.
(We will also sometimes use the metavariable $\lctx$ for \emph{ordered} contexts.)
As a monoid, ordered contexts are equivalent up to associativity and unit laws (see adjacent \lcnamecref{fig:ordered-logic:monoid-laws}).%
\begin{marginfigure}[-5\baselineskip]
  \begin{gather*}
    (\octx_1 \oc \octx_2) \oc \octx_3 = \octx_1 \oc (\octx_2 \oc \octx_3) \\
    (\octxe) \oc \octx = \octx = \octx \oc (\octxe)
  \end{gather*}
  \caption{Monoid laws for ordered contexts}\label{fig:ordered-logic:monoid-laws}
\end{marginfigure}
We choose to keep this equivalence implicit, however, treating equivalent contexts as syntactically indistinguishable.%
\footnote{Throughout this document, we will encoun\-ter free noncommutative monoids in various guises.
Each time, we will choose to keep the equivalence induced by the monoid laws implicit, as we do here.}
%
Associativity means that contexts are indeed lists, not trees; and noncommutativity means that those lists are ordered, not multisets as in linear logic.



% The full sequent calculus for ordered logic will be summarized in \cref{fig:ordered-logic}, but first we will discuss the calculus's judgmental principles and logical connectives one by one.

% Following the example of linear logic, ordered logic can be given a resource interpretation -- with a small twist.
% The absence of an exhange property means that resouces are now inherently ordered.
% The sequent $\oseq{\octx |- A}$ means that, by consuming the resources $\octx$, the resource $A$ can be produced.
% For ordered logic, however, 

\subsection{Judgmental principles}

Even without considering the specific structure of propositions, two judgmental principles must hold if sequents are to accurately describe the production of resources.

First, given a resource $A$, producing the same resource $A$ should be effortless -- it already exists!
This amounts to an identity principle for sequents:
  \begin{description}[labelindent=\parindent]
  \item[Identity principle] $\oseq{A |- A}$ for all propositions $A$.
  \end{description}
  This principle is adopted by the ordered sequent calculus as a primitive rule of inference:
  \begin{equation*}
    \infer[\jrule{ID}\smash{^A}]{\oseq{A |- A}}{}
    \,.
  \end{equation*}
Both the identity principle and its corresponding $\jrule{ID}$ rule capture the idea that resource production is a reflexive process.

Second, and dually, resource production should be transitive process.
If a resource $B$ can be produced from resource $A$ (\ie, $\oseq{A |- B}$), and if a resource $C$ can be produced from resource $B$ (\ie, $\oseq{B |- C}$), then, by chaining the productions, it ought to be possible to produce $C$ from $A$ (\ie, $\oseq{A |- C}$).
For sequents, this amounts to a cut principle that is most useful in a generalized form:
\begin{description}[resume*]
\item[Cut principle]
  If $\oseq{\octx |- B}$ and $\oseq{\octx'_L \oc B \oc \octx'_R |- C}$, then $\oseq{\octx'_L \oc \octx \oc \octx'_R |- C}$.
\end{description}
As with the identity principle, this cut principle is adopted by the ordered sequent calculus as a primitive rule of inference:
\begin{equation*}
  \infer[\jrule{CUT}\smash{^B}]{\oseq{\octx'_L \oc \octx \oc \octx'_R |- C}}{
    \oseq{\octx |- B} & \oseq{\octx'_L \oc B \oc \octx'_R |- C}}
  \,.
\end{equation*}

The importance of these two judgmental principles goes beyond that of mere rules of inference.
As we will see in \cref{sec:ordered-logic:meanings}, the admissibility of these principles serves an important role in defining the meaning of the logical connectives.

% Even without considering the structure of propositions, two judgmental principles of the ordered sequent calculus are already apparent.
% \begin{itemize}
% \item Resource production is transitive: if resource $A$ can be produced from resources $\octx$, and if the resource $C$ can be produced from resources $\octx'_L \oc A \oc \octx'_R$, then $C$ can be produced from $\octx'_L \oc \octx \oc \octx'_R$ by first producing $A$ from the inner resources $\octx$ and then producing $C$ from the resulting resources, $\octx'_L \oc A \oc \octx'_R$.
% \end{itemize}

% \begin{inferences}
%   \infer[\jrule{CUT}\smash{^A}]{\oseq{\octx'_L \oc \octx \oc \octx'_R |- C}}{
%     \oseq{\octx |- A} & \oseq{\octx'_L \oc A \oc \octx'_R |- C}}
%   \and
%   \infer[\jrule{ID}\smash{^A}]{\oseq{A |- A}}{}
% \end{inferences}
% The $\jrule{CUT}$ rule shows that resource production is transitive: if resource $A$ can be produced from resources $\octx$, and if the resource $C$ can be produced from resources $\octx'_L \oc A \oc \octx'_R$, then $C$ can be produced from $\octx'_L \oc \octx \oc \octx'_R$ by first producing $A$ from the inner resources $\octx$ and then producing $C$ from the resulting resources, $\octx'_L \oc A \oc \octx'_R$.
% Dually, the $\jrule{ID}$ rule shows that resource production is also reflexive: given a resource $A$, the same resource $A$ can be produced effortlessly.

\subsection{The ordered logical connectives}

The propositions of ordered logic are given by the following grammar.
\begin{syntax*}
  Propositions & A, B, C &
    a
    \begin{array}[t]{@{{} \mid {}}l@{}}
      A \fuse B \mid \one \mid A \esuf B \mid A \plus B \mid \zero \\
      A \with B \mid \top \mid A \limp B \mid B \pmir A
    \end{array}
\end{syntax*}
Among these are propositional atoms, $a$, which stand in for arbitrary propositions.
The other propositions are built up from these atoms by using the logical connectives.

Under the resource interpretation of ordered logic, these logical connectives may be viewed as resource constructors.
A connective's right rule defines how to produce that kind of resource, while the corresponding left rules define how that kind of resource may be used.
% As a first example, consider ordered conjunction.

\paragraph*{Ordered conjunction and its unit}\label{p:ordered-logic:ordered-conjunction}
Ordered conjunction\footnote{Also known as multiplicative conjunction.} is the proposition $A \fuse B$, read \enquote{$A$ fuse $B$}.
Under the resource interpretation, $A \fuse B$ is the side-by-side pair of resources $A$ and $B$, packaged as a single ordered resource.
Its sequent calculus inference rules are:
\begin{inferences}
  \infer[\rrule{\fuse}]{\oseq{\octx_1 \oc \octx_2 |- A \fuse B}}{
    \oseq{\octx_1 |- A} & \oseq{\octx_2 |- B}}
  \and
  \infer[\lrule{\fuse}]{\oseq{\octx'_L \oc (A \fuse B) \oc \octx'_R |- C}}{
    \oseq{\octx'_L \oc A \oc B \oc \octx'_R |- C}}
\end{inferences}
The right rule, $\rrule{\fuse}$, says that $A \fuse B$ may be produced by partitioning the available resources into $\octx_1 \oc \octx_2$ and then separately using the resources $\octx_1$ and $\octx_2$ to produce $A$ and $B$, respectively.
The left rule, $\lrule{\fuse}$, shows how to use resource $A \fuse B$: simply unwrap the package to leave the separate contents, resources $A$ and $B$, side by side.

Just as truth is the nullary analogue of conjunction in intuitionistic logic, multiplicative truth, $\one$, is the nullary analogue of ordered conjunction.
Under the resource interpretation, $\one$ is therefore an empty resource package that contains no resources.
\begin{inferences}
  \infer[\rrule{\one}]{\oseq{\octxe |- \one}}{}
  \and
  \infer[\lrule{\one}]{\oseq{\octx'_L \oc \one \oc \octx'_R |- C}}{
    \oseq{\octx'_L \oc \octx'_R |- C}}
\end{inferences}
The sequents $\oseq{\one \fuse A \dashv|- \oseq{A \dashv|- A \fuse \one}}$ are all derivable%
\footnote{$\oseq{A \dashv|- B}$ stands for the sequents $\oseq{A |- B}$ and $\oseq{B |- A}$.}%
, so $\one$ is indeed $\fuse$'s unit.

In addition to $A \fuse B$, the proposition $A \esuf B$, read \enquote{$A$ twist $B$}, is included.
Under the resource interpretation, $A \esuf B$ is the side-by-side pair of resources $B$ and $A$, packaged as a single ordered resource.
If we gave sequent calculus inference rules for $A \esuf B$, the sequents $\oseq{B \fuse A \dashv|- A \esuf B}$ and $\oseq{\one \esuf A \dashv|- \oseq{A \dashv|- A \esuf \one}}$ would all be derivable.
Therefore, instead of taking $A \esuf B$ as a primitive and explicitly giving it inference rules, we choose to treat it as merely a notational definition for the ordered conjunction $B \fuse A$.


\paragraph{Disjunction and its unit}
Disjunction is the proposition $A \plus B$, read \enquote{$A$ plus $B$}.%
\footnote{This connective is also known as additive disjunction, in contrast with the multiplicative disjunction of classical linear logic; being intuitionistic, ordered logic does not have a purely multiplicative disjunction.
  See \textcite{Chang+:CMU03}.}
Under the resource interpretation, $A \plus B$ is a package that contains one of the resources $A$ or $B$
% either resource $A$ or resource $B$
(but not both).
\begin{inferences}
  \infer[\rrule{\plus}_1]{\oseq{\octx |- A \plus B}}{
    \oseq{\octx |- A}}
  \and
  \infer[\rrule{\plus}_2]{\oseq{\octx |- A \plus B}}{
    \oseq{\octx |- B}}
  \and
  \infer[\lrule{\plus}]{\oseq{\octx'_L \oc (A \plus B) \oc \octx'_R |- C}}{
    \oseq{\octx'_L \oc A \oc \octx'_R |- C} &
    \oseq{\octx'_L \oc B \oc \octx'_R |- C}}
\end{inferences}
The right rules, $\rrule{\plus}_1$ and $\rrule{\plus}_2$, say that a resource $A \plus B$ may be produced from the resources $\octx$ by producing either $A$ or $B$ and then wrapping that resource up as an $A \plus B$ package.
The left rule, $\lrule{\plus}$, shows how to use a resource $A \plus B$: unwrap the package and use whatever it contains -- whether an $A$ or a $B$.

Falsehood, $\zero$, can be viewed as the nullary analogue of disjunction:
\begin{inferences}
  \text{(no $\rrule{\zero}$ rule)}
  \and
  \infer[\lrule{\zero}]{\oseq{\octx'_L \oc \zero \oc \octx'_R |- C}}{}
\end{inferences}
The sequents $\oseq{\zero \plus A \dashv|- \oseq{A \dashv|- A \plus \zero}}$ are all derivable, so $\zero$ is indeed $\plus$'s unit.

\paragraph{Alternative conjunction and its unit}
Alternative conjunction\footnote{Also known as additive conjunction.} is the proposition $A \with B$, read \enquote{$A$ with $B$};
it is dual to disjunction.
Under the resource interpretation, $A \with B$ is the resource that can be transformed -- irreversibly -- into either a resource $A$ or a resource $B$, whichever the user chooses.
\begin{inferences}
  \infer[\rrule{\with}]{\oseq{\octx |- A \with B}}{
    \oseq{\octx |- A} & \oseq{\octx |- B}}
  \and
  \infer[\lrule{\with}_1]{\oseq{\octx'_L \oc (A \with B) \oc \octx'_R |- C}}{
    \oseq{\octx'_L \oc A \oc \octx'_R |- C}}
  \and
  \infer[\lrule{\with}_2]{\oseq{\octx'_L \oc (A \with B) \oc \octx'_R |- C}}{
    \oseq{\octx'_L \oc B \oc \octx'_R |- C}}
\end{inferences}
The left rules, $\lrule{\with}_1$ and $\lrule{\with}_2$, show how to use a resource $A \with B$: transform it into either an $A$ or a $B$ and then use that resource.
The right rule, $\rrule{\with}$, says that to produce a resource $A \with B$ the producer must be prepared to produce either $A$ or $B$ -- whichever the user eventually chooses.
% it must be possible to produce $A$ from $\octx$ \emph{and} to produce $B$ from $\octx$ -- to be ready for either eventuality.

Additive truth, $\top$, can be viewed as the nullary analogue of alternative conjunction:
\begin{inferences}
  \infer[\rrule{\top}]{\oseq{\octx |- \top}}{}
  \and
  \text{(no $\lrule{\top}$ rule)}
\end{inferences}
Once again, the sequents $\oseq{\top \with A \dashv|- \oseq{A \dashv|- A \with \top}}$ are all derivable, so $\top$ is indeed the unit of $\with$.


\paragraph*{Left- and right-handed implications}
Left-handed implication is the proposition $A \limp B$, read \enquote{$A$ under $B$} or \enquote{$A$ left-implies $B$}.
% Under the resource interpretation, $A \limp B$ is the resource that, when placed with the resource $A$ to its immediate left, consumes the $A$ to produce resource $B$.
When interpreted as a resource, $A \limp B$ is the resource that can transform a left-adjacent resource~$A$ into the resource $B$.
% consume a resource $A$ from its immediate left and thereby produce the resource $B$.
\begin{inferences}
  \infer[\rrule{\limp}]{\oseq{\octx |- A \limp B}}{
    \oseq{A \oc \octx |- B}}
  \and
  \infer[\lrule{\limp}]{\oseq{\octx'_L \oc \octx \oc (A \limp B) \oc \octx'_R |- C}}{
    \oseq{\octx |- A} & \oseq{\octx'_L \oc B \oc \octx'_R |- C}}
\end{inferences}
The left rule, $\lrule{\limp}$, shows how to use a resource $A \limp B$: first produce $A$ from the left-adjacent resources $\octx$, then transform the left-adjacent $A$ into the resource $B$, and finally use that $B$.
The right rule, $\rrule{\limp}$, says that resources $\octx$ can produce $A \limp B$ if the same resources prefixed with $A$ -- that is, $A \oc \octx$ -- can produce $B$.

Right-handed implication, $B \pmir A$ (read \enquote{$B$ over $A$} or \enquote{$A$ right-implies $B$}), is symmetric to left-handed implication: $B \pmir A$ is the resource that can transform a \emph{right}-adjacent resource $A$ into the resource $B$.
\begin{inferences}
  \infer[\rrule{\pmir}]{\oseq{\octx |- B \pmir A}}{
    \oseq{\octx \oc A |- B}}
  \and
  \infer[\lrule{\pmir}]{\oseq{\octx'_L \oc (B \pmir A) \oc \octx \oc \octx'_R |- C}}{
    \oseq{\octx |- A} & \oseq{\octx'_L \oc B \oc \octx'_R |- C}}
\end{inferences}

The two forms of implication each enjoy their own currying laws: the sequents $\oseq{A \limp (B \limp C) \dashv|- (A \esuf B) \limp C}$ and $\oseq{(C \pmir B) \pmir A \dashv|- C \pmir (A \fuse B)}$ are derivable.


% \paragraph*{Summary}
% The sequent calculus presented above is summarized in \cref{fig:ordered-logic:sequent-calculus}.


\begin{table}[tbp]
  \centering
  \begin{tabular}{@{}rll@{}}
    \toprule
    \multicolumn{2}{r}{\emph{Ordered logical connective}} % & \emph{Notation}
      & \emph{Resource interpretation}
    \\ \midrule
    Ordered conjunction & $A \fuse B$ & A side-by-side pair of resources $A$ and $B$ % , packaged as one % a single resource
    \\
    Multiplicative truth & $\one$ & The unit of ordered conjunction 
    \\
    Twisted conjunction & $A \esuf B$ & A side-by-side pair of resources $B$ and $A$ % , packaged as one % a single resource
    \\
    (Additive) disjunction & $A \plus B$ & % A resource package that contains
      A package containing  $A$ or $B$ (not both)
    \\
    (Additive) falsehood & $\zero$ & A package containing no resources 
    \\
    Alternative conjunction & $A \with B$ &% A resource that can be transformed into either $A$ or $B$
      A resource that transforms into $A$ or $B$
    \\
    Additive truth & $\top$ & An immutable resource
    \\
    Left-handed implication & $A \limp B$ & % A resource that
      Transforms a left-adjacent % resource
      $A$ into % resource
      $B$ \\
    Right-handed implication & $B \pmir A$ & % A resource that
      Transforms a right-adjacent % resource
      $A$ into % resource
      $B$ \\
    \bottomrule
  \end{tabular}
  \caption[][-0.25\baselineskip]{A resource interpretation of the ordered logical connectives}\label{tbl:ordered-logic:resources}\forceversofloat
\end{table}

\begin{figure}[tbp]
  \vspace*{\dimexpr-\abovedisplayskip-\abovecaptionskip\relax}
  \begin{syntax*}
    Propositions & A,B,C &
      a \begin{array}[t]{@{{} \mid {}}l@{}}
          A \fuse B \mid \one \mid A \esuf B \mid A \plus B \mid \zero \\
          A \with B \mid \top \mid A \limp B \mid B \pmir A
        \end{array}
    \\
    Contexts & \octx, \lctx &
      \octx_1 \oc \octx_2 \mid \octxe \mid A
  \end{syntax*}
  \begin{inferences}
    \infer[\jrule{CUT}\smash{^A}]{\oseq{\octx'_L \oc \octx \oc \octx'_R |- C}}{
      \oseq{\octx |- A} & \oseq{\octx'_L \oc A \oc \octx'_R |- C}}
    \and
    \infer[\jrule{ID}\smash{^A}]{\oseq{A |- A}}{}
  \end{inferences}
  \begin{inferences}
    \infer[\rrule{\fuse}]{\oseq{\octx_1 \oc \octx_2 |- A \fuse B}}{
      \oseq{\octx_1 |- A} & \oseq{\octx_2 |- B}}
    \and
    \infer[\lrule{\fuse}]{\oseq{\octx'_L \oc (A \fuse B) \oc \octx'_R |- C}}{
      \oseq{\octx'_L \oc A \oc B \oc \octx'_R |- C}}
    % \\
    % \infer[\rrule{\esuf}]{\oseq{\octx_1 \oc \octx_2 |- B \esuf A}}{
    %   \oseq{\octx_1 |- A} & \oseq{\octx_2 |- B}}
    % \and
    % \infer[\lrule{\esuf}]{\oseq{\octx'_L \oc (B \esuf A) \oc \octx'_R |- C}}{
    %   \oseq{\octx'_L \oc A \oc B \oc \octx'_R |- C}}
    \\
    \infer[\rrule{\one}]{\oseq{\octxe |- \one}}{}
    \and
    \infer[\lrule{\one}]{\oseq{\octx'_L \oc \one \oc \octx'_R |- C}}{
      \oseq{\octx'_L \oc \octx'_R |- C}}
    \\
    A \esuf B \overset{\text{def}}{=} B \fuse A
    \\
    \infer[\rrule{\plus}_1]{\oseq{\octx |- A \plus B}}{
      \oseq{\octx |- A}}
    \and
    \infer[\rrule{\plus}_2]{\oseq{\octx |- A \plus B}}{
      \oseq{\octx |- B}}
    \and
    \infer[\lrule{\plus}]{\oseq{\octx'_L \oc (A \plus B) \oc \octx'_R |- C}}{
      \oseq{\octx'_L \oc A \oc \octx'_R |- C} &
      \oseq{\octx'_L \oc B \oc \octx'_R |- C}}
    \\
    \text{(no $\rrule{\zero}$ rule)}
    \and
    \infer[\lrule{\zero}]{\oseq{\octx'_L \oc \zero \oc \octx'_R |- C}}{}
  \end{inferences}
  \begin{inferences}
    \infer[\rrule{\with}]{\oseq{\octx |- A \with B}}{
      \oseq{\octx |- A} & \oseq{\octx |- B}}
    \and
    \infer[\lrule{\with}_1]{\oseq{\octx'_L \oc (A \with B) \oc \octx'_R |- C}}{
      \oseq{\octx'_L \oc A \oc \octx'_R |- C}}
    \and
    \infer[\lrule{\with}_2]{\oseq{\octx'_L \oc (A \with B) \oc \octx'_R |- C}}{
      \oseq{\octx'_L \oc B \oc \octx'_R |- C}}
    \\
    \infer[\rrule{\top}]{\oseq{\octx |- \top}}{}
    \and
    \text{(no $\lrule{\top}$ rule)}
    \\
    \infer[\rrule{\limp}]{\oseq{\octx |- A \limp B}}{
      \oseq{A \oc \octx |- B}}
    \and
     \infer[\lrule{\limp}]{\oseq{\octx'_L \oc \octx \oc (A \limp B) \oc \octx'_R |- C}}{
      \oseq{\octx |- A} & \oseq{\octx'_L \oc B \oc \octx'_R |- C}}
    \\
    \infer[\rrule{\pmir}]{\oseq{\octx |- B \pmir A}}{
      \oseq{\octx \oc A |- B}}
    \and
    \infer[\lrule{\pmir}]{\oseq{\octx'_L \oc (B \pmir A) \oc \octx \oc \octx'_R |- C}}{
      \oseq{\octx |- A} & \oseq{\octx'_L \oc B \oc \octx'_R |- C}}
  \end{inferences}
  \caption{A summary of  ordered logic's sequent calculus, as presented in \cref{sec:ordered-logic:sequent-calculus}}\label{fig:ordered-logic:sequent-calculus}\forceversofloat
\end{figure}

\section{A verificationist meaning-theory of the ordered sequent calculus}\label{sec:ordered-logic:verifications}\label{sec:ordered-logic:meanings}

The previous \lcnamecref{sec:ordered-logic:sequent-calculus} presented a collection of inference rules that have an apparently sensible resource interpretation.
But how can we be sure that the rules constitute a well-defined \emph{logic}?

In the tradition of \citeauthor{Gentzen:MZ35}, \citeauthor{Dummett:WJ76}, and \citeauthor{Martin-Lof:Siena83}, a logic is well-defined if it rests on the solid foundation of a verificationist meaning-\-theory.\autocites{Gentzen:MZ35}{Dummett:WJ76}{Martin-Lof:Siena83}
In \citeauthor{Martin-Lof:Siena83}'s words, \enquote{The meaning of a proposition is determined by \textelp{} what counts as a verification of it.}
And a verification is a proof that decomposes that proposition into its subformulas, without dragging in other, unrelated propositions.
In this way, the meaning of a proposition is compositional.

For the ordered sequent calculus, a verification is thus a proof that relies only on the right and left inference rules (and the $\jrule{ID}^{a}$ rule for propositional atoms) -- the $\jrule{CUT}$ rule drags in an unrelated proposition as its cut formula; and, when $A$ is a compound proposition, the $\jrule{ID}^A$ rule fails to decompose $A$ to its subformulas.
A proof is \vocab{cut-free} if it does not contain any instances of the $\jrule{CUT}$ rule; similarly, a proof is \vocab{identity-long} if all instances of the $\jrule{ID}$ rule occur at propositional atoms.
Verifications are thus exactly those proofs that are both cut-free and identity-long.

Because meaning is based on verifications, every proof must have a corresponding verification if proofs are to be meaningful.
That is, we need to describe a procedure for normalizing arbitrary proofs to verifications.
The characterization of verifications as cut-free, identity-long proofs suggests a two-step strategy for proof normalization:
\begin{enumerate}[topsep=.33\baselineskip, noitemsep]
  \label{list:ordered-logic:normalization}
\item Eliminate all instances of $\jrule{CUT}$.
\item Without introducing new instances of $\jrule{CUT}$, eliminate all remaining instances of $\jrule{ID}$ that occur at non-atomic propositions.
\end{enumerate}
% Provided that the identity elimination step does not introduce any instances of $\jrule{CUT}$,
The end result will be a cut-free, identity-long proof -- a verification.

This normalization procedure is described by the constructive content of the following \lcnamecrefs{thm:ordered-logic:cut-elimination}; their proofs amount to defining functions on proofs.%
% As presented in the following \lcnamecrefs{sec:ordered-logic:cut-elimination}, the proofs of these \lcnamecrefs{thm:ordered-logic:cut-elimination} are all constructive and amount to defining functions on proofs.
%
\begin{restatable*}[
      name=Cut elimination,
      label=thm:ordered-logic:cut-elimination
    ]{theorem}{orderedcutelimination}
      If a proof of\/ $\oseq{\octx |- A}$ exists, then there exists a \emph{cut-free} proof of\/ $\oseq{\octx |- A}$.
    \end{restatable*}
%
\begin{restatable*}[
      name=Identity elimination,
      label=thm:ordered-logic:id-elimination
    ]{theorem}{orderedidelimination}
      If a proof of\/ $\oseq{\octx |- A}$ exists, then an \emph{identity-long} proof of\/ $\oseq{\octx |- A}$ exists.
      % Moreover, cut-freeness is preserved.
      Moreover, if the given proof is cut-free, so is the identity-long proof.
    \end{restatable*}
%
\begin{restatable*}[
  name=Proof normalization,
  label=cor:ordered-logic:proof-normalization
]{corollary}{orderedproofnormalization}
  If a proof of\/ $\oseq{\octx |- A}$ exists, then a verification (\ie, a cut-free, identity-long proof) of\/ $\oseq{\octx |- A}$ exists.
\end{restatable*}

We will now prove these \lcnamecrefs{thm:ordered-logic:cut-elimination}.


% For a sequent calculus, a verification is thus a proof that relies only on the right and left inference rules -- that is, a verification is a proof in which all occurrences of the cut and identity rules have been eliminated.
% (Of course, propositional atoms do not have right and left rules, and so the identity rule is permitted for propositional atoms -- and propositional atoms alone.)

% For this verificationist program to succeed, we need to be sure that every proof has a corresponding verification -- we need a proof normalization procedure.

% We will say that a proof is \vocab{cut-free} if it contains no instances of the $\jrule{CUT}$ rule, and \vocab{long} if all instances of the $\jrule{ID}$ rule occur at propositional atoms, not compound propositions.
% Verifications are thus exactly those proofs that are both cut-free and long.

% This suggests a strategy for normalizing proofs to verifications: first eliminate all instances of $\jrule{CUT}$, and then eliminate any remaining non-atomic instances of $\jrule{ID}$.
% Provided that the identity elimination step does not introduce any instances of $\jrule{CUT}$, the end result will be a cut-free, long proof -- that is, a verification.






\subsection{Cut elimination}\label{sec:ordered-logic:cut-elimination}

To prove the cut elimination \lcnamecref{thm:ordered-logic:cut-elimination} stated above, we will eventually use a straightforward induction on the structure of the given proof.
But first, we need to establish a cut principle for cut-free proofs:
%
\begin{restatable*}[
  name=Admissibility of cut,
  label=lem:ordered-logic:cut-admissibility
]{lemma}{orderedcutadmissibility}
  If cut-free proofs of\/ $\oseq{\octx |- A}$ and $\oseq{\octx'_L \oc A \oc \octx'_R |- C}$ exist, then there exists a \emph{cut-free} proof of\/ $\oseq{\octx'_L \oc \octx \oc \octx'_R |- C}$.
\end{restatable*}

Before proceeding to this \lcnamecref{lem:ordered-logic:cut-admissibility}'s proof, it is worth emphasizing a subtle distinction between the sequent calculus's primitive $\jrule{CUT}$ rule and the admissible cut principle that this \lcnamecref{lem:ordered-logic:cut-admissibility} establishes.
To be completely formal, we ought to treat cut-freeness as an extrinsic, Curry-style property of proofs and indicate that property by decorating the turnstile: a proof of $\oseqcf{\octx |- \mkern-2mu A}$ is a cut-free proof of $\oseq{\octx |- A}$.
The admissible cut principle stated in \cref{lem:ordered-logic:cut-admissibility} could then be expressed as the rule
\begin{equation*}
  \infer-[\jrule{A-CUT}\smash{^A}]{\oseqcf{\octx'_L \oc \octx \oc \octx'_R |- C}}{
    \oseqcf{\octx |- A} & \oseqcf{\octx'_L \oc A \oc \octx'_R |- C}}
  \,
\end{equation*}
with the dotted line indicating that this is an admissible, not primitive, rule.
Writing it in this way emphasizes that the proof of \cref{lem:ordered-logic:cut-admissibility} will amount to defining a meta-level function that takes cut-free proofs of $\oseq{\octx |- A}$ and $\oseq{\octx'_L \oc A \oc \octx'_R |- C}$ and produces a \emph{cut-free} proof of $\oseq{\octx'_L \oc \octx \oc \octx'_R |- C}$.
Contrast this with the primitive $\jrule{CUT}$ rule of the ordered sequent calculus, which forms a (cut-full) proof of $\oseq{\octx'_L \oc \octx \oc \octx'_R |- C}$ from (potentially cut-full) proofs of $\oseq{\octx |- A}$ and $\oseq{\octx'_L \oc A \oc \octx'_R |- C}$.

From here on, we won't bother to be quite so pedantic, instead often omitting the turnstile decoration on cut-free proofs, with the understanding that any proofs to which the admissible $\jrule{A-CUT}$ rule is applied are necessarily cut-free.%
\footnote{The distinction will become somewhat more important in \cref{ch:singleton-logic} when we introduce a \enquote{semi-axiomatic sequent calculus} for singleton logic.}

\newthought{With that} clarification out of the way, we may proceed to proving the previously stated \lcnamecref{lem:ordered-logic:cut-admissibility} and \lcnamecref{thm:ordered-logic:cut-elimination}.
%
\orderedcutadmissibility
%
\begin{proof}
  This \lcnamecref{lem:ordered-logic:cut-admissibility} was proved in a similar setting by \textfootcite{Polakow+Pfenning:MFPS99} using a standard technique for proving the admissibility of a cut principle\autocite{Pfenning:LICS95} --
  % We follow a standard technique for proving the admissibility of a cut principle\autocite{Pfenning:LICS95} --
  a lexicographic structural induction, first on the structure of the cut formula, $A$, and then on the structures of the given proofs.
  We review their proof here.

  As usual, the various cases can be classified into three categories: identity cases, principal cases, and commutative cases.
  %
  \begin{description}[parsep=0pt, listparindent=\parindent]
  \item[Identity cases]
    In the cases where one of the two proofs is an instance of the $\jrule{ID}$ rule, the admissible cut can be reduced to the other proof alone.
    For example:
    \begin{equation*}
      \infer-[\jrule{A-CUT}\smash{^A}]{\oseq{\octx'_L \oc A \oc \octx'_R |- C}}{
        \infer[\jrule{ID}\smash{^A}]{\oseq{A |- A}}{} &
        \deduce{\oseq{\octx'_L \oc A \oc \octx'_R |- C}}{\EE}}
      =
      \deduce{\oseq{\octx'_L \oc A \oc \octx'_R |- C}}{\EE}
    \end{equation*}
    That the cut and identity principles are inverses is reflected in these identity cases.

  \item[Principal cases]
    In another class of cases, both proofs end by introducing the cut formula -- on the right in the left-hand proof with a right rule, and on the left in the right-hand proof with a left rule.
    These cases are resolved by reducing the admissible cut to several instances of the admissible cut principle at proper subformulas of the cut formula.

    For example, the principal cut reduction for $A_1 \limp A_2$ yields cuts at the proper subformulas $A_1$ and $A_2$:
    \begin{gather*}
      \infer-[\jrule{A-CUT}\smash{^{A_1 \limp A_2}}]{\oseq{\octx'_L \oc \octx'_1 \oc \octx \oc \octx'_R |- C}}{
        \infer[\rrule{\limp}]{\oseq{\octx |- A_1 \limp A_2}}{
          \deduce{\oseq{A_1 \oc \octx |- A_2}}{\DD_1}} &
        \infer[\lrule{\limp}]{\oseq{\octx'_L \oc \octx'_1 \oc (A_1 \limp A_2) \oc \octx'_R |- C}}{
          \deduce{\oseq{\octx'_1 |- A_1}}{\EE_1} &
          \deduce{\oseq{\octx'_L \oc A_2 \oc \octx'_R |- C}}{\EE_2}}}
      \\=\\
      \infer-[\jrule{A-CUT}\smash{^{A_2}}]{\oseq{\octx'_L \oc \octx'_1 \oc \octx \oc \octx'_R |- C}}{
        \infer-[\jrule{A-CUT}\smash{^{A_1}}]{\oseq{\octx'_1 \oc \octx |- A_2}}{
          \deduce{\oseq{\octx'_1 |- A_1}}{\DD_1} &
          \deduce{\oseq{A_1 \oc \octx |- A_2}}{\EE_1}} &
        \deduce{\oseq{\octx'_L \oc A_2 \oc \octx'_R |- C}}{\EE_2}}
    \end{gather*}

  \item[Commutative cases]
    In the remaining cases, at least one of the two proofs ends by introducing a side formula, \ie, a formula other than the cut formula.
    To reduce the admissible cut, it is permuted with the final inference in that proof;
    the reduced instance of the admissible cut is smaller because it occurs with the same cut formula but smaller proofs.

    Commutative cases are subcategorized as left- or right-\-com\-mu\-ta\-tive cut reductions according to the branch into which the admissible cut is permuted.
    % For example, one left-commutative case involves a left branch that ends with the $\lrule{\limp}$ rule:
    % \begin{gather*}
    %   \infer-[\jrule{A-CUT}\smash{^A}]{\oseq{\octx'_L \oc \octx_L \oc \octx_1 \oc (B_1 \limp B_2) \oc \octx_R \oc \octx'_R |- C}}{
    %     \infer[\lrule{\limp}]{\oseq{\octx_L \oc \octx_1 \oc (B_1 \limp B_2) \oc \octx_R |- A}}{
    %       \deduce{\oseq{\octx_1 |- B_1}}{\DD_1} &
    %       \deduce{\oseq{\octx_L \oc B_2 \oc \octx_R |- A}}{\DD_2}} &
    %     \deduce{\oseq{\octx'_L \oc A \oc \octx'_R |- C}}{\EE}}
    %   \\=\\
    %   \infer[\lrule{\limp}]{\oseq{\octx'_L \oc \octx_L \oc \octx_1 \oc (B_1 \limp B_2) \oc \octx_R \oc \octx'_R |- C}}{
    %     \deduce{\oseq{\octx_1 |- B_1}}{\DD_1} &
    %     \infer-[\jrule{A-CUT}\smash{^A}]{\oseq{\octx'_L \oc \octx_L \oc B_2 \oc \octx_R \oc \octx'_R |- C}}{
    %       \deduce{\oseq{\octx_L \oc B_2 \oc \octx_R |- A}}{\DD_2} &
    %       \deduce{\oseq{\octx'_L \oc A \oc \octx'_R |- C}}{\EE}}}
    % \end{gather*}
    % And
    For example, one right-commutative case involves a right-hand proof that ends by introducing the consequent with the $\rrule{\limp}$ rule:
    \begin{equation*}
      \infer-[\jrule{A-CUT}\smash{^A}]{\oseq{\octx'_L \oc \octx \oc \octx'_R |- C_1 \limp C_2}}{
        \deduce{\oseq{\octx |- A}}{\DD} &
        \infer[\rrule{\limp}]{\oseq{\octx'_L \oc A \oc \octx'_R |- C_1 \limp C_2}}{
          \deduce{\oseq{C_1 \oc \octx'_L \oc A \oc \octx'_R |- C_2}}{\EE_1}}}
      =
      \infer[\rrule{\limp}]{\oseq{\octx'_L \oc \octx \oc \octx'_R |- C_1 \limp C_2}}{
        \infer-[\jrule{A-CUT}\smash{^A}]{\oseq{C_1 \oc \octx'_L \oc \octx \oc \octx'_R |- C_2}}{
          \deduce{\oseq{\octx |- A}}{\DD} &
          \deduce{\oseq{C_1 \oc \octx'_L \oc A \oc \octx'_R |- C_2}}{\EE_1}}}
    \end{equation*}
    % And still
    Among the other right-commutative cases are several involving a right-hand proof that ends by using a left rule, such as the $\lrule{\limp}$ rule, to introduce a side formula.
    This contrasts with the left-commutative cases: the left-hand proof can never use a right rule to introduce a side formula because its only consequent is the cut formula.
    %
%     Other right-commutative cases involve a right-hand proof that ends with the $\lrule{\limp}$ rule.
    % \begin{gather*}
    %   \infer-[\jrule{A-CUT}\smash{^A}]{\oseq{\octx'_L \oc \octx \oc \octx'_M \oc \octx'_1 \oc (B_1 \limp B_2) \oc \octx'_R |- C}}{
    %     \deduce{\oseq{\octx |- A}}{\DD} &
    %     \infer[\lrule{\limp}]{\oseq{\octx'_L \oc A \oc \octx'_M \oc \octx'_1 \oc (B_1 \limp B_2) \oc \octx'_R |- C}}{
    %       \deduce{\oseq{\octx'_1 |- B_1}}{\EE_1} &
    %       \deduce{\oseq{\octx'_L \oc A \oc \octx'_M \oc B_2 \oc \octx'_R |- C}}{\EE_2}}}
    %   \\=\\
    %   \infer[\lrule{\limp}]{\oseq{\octx'_L \oc \octx \oc \octx'_M \oc \octx'_1 \oc (B_1 \limp B_2) \oc \octx'_R |- C}}{
    %     \deduce{\oseq{\octx'_1 |- B_1}}{\EE_1} &
    %     \infer-[\jrule{A-CUT}\smash{^A}]{\oseq{\octx'_L \oc \octx \oc \octx'_M \oc B_2 \oc \octx'_R |- C}}{
    %       \deduce{\oseq{\octx |- A}}{\DD} &
    %       \deduce{\oseq{\octx'_L \oc A \oc \octx'_M \oc B_2 \oc \octx'_R |- C}}{\EE_2}}}
    % \end{gather*}
  \qedhere
  \end{description}
\end{proof}

With the admissibility of a cut principle for cut-free proofs established, we may finally prove a cut elimination result.
%
\orderedcutelimination
%
\begin{proof}
  We follow the proof sketched by \textfootcite{Polakow+Pfenning:MFPS99}.
  The proof is by structural induction on the proof of $\oseq{\octx |- A}$, with appeals to the admissibility of cut~\parencref{lem:ordered-logic:cut-admissibility} whenever a $\jrule{CUT}$ rule is encountered.

  Like the admissibility of cut \lcnamecref{lem:ordered-logic:cut-admissibility}, this \lcnamecref{thm:ordered-logic:cut-elimination} may be rendered as an admissible rule:
  \begin{equation*}
    \infer-[\jrule{CE}]{\oseqcf{\octx |- A}}{
      \oseq{\octx |- A}}
  \end{equation*}
  Writing the \lcnamecref{thm:ordered-logic:cut-elimination} in this way serves to emphasize that its proof amounts to the definition of a meta-level function for normalizing proofs to cut-free form.

  The crucial case is then resolved as follows:
  \begin{equation*}
    \infer-[\jrule{CE}]{\oseqcf{\octx'_L \oc \octx \oc \octx'_R |- C}}{
      \infer[\jrule{CUT}\smash{^A}]{\oseq{\octx'_L \oc \octx \oc \octx'_R |- C}}{
        \oseq{\octx |- A} & \oseq{\octx'_L \oc A \oc \octx'_R |- C}}}
    =
    \infer-[\jrule{A-CUT}\smash{^A}]{\oseqcf{\octx'_L \oc \octx \oc \octx'_R |- C}}{
      \infer-[\jrule{CE}]{\oseqcf{\octx |- A}}{
        \oseq{\octx |- A}} &
      \infer-[\jrule{CE}]{\oseqcf{\octx'_L \oc A \oc \octx'_R |- C}}{
        \oseq{\octx'_L \oc A \oc \octx'_R |- C}}}
  \end{equation*}
  All other cases are resolved compositionally.
\end{proof}


\subsection{Identity elimination}

By this cut elimination \lcnamecref{thm:ordered-logic:cut-elimination}, an arbitrary proof may be put into cut-free form.
Recall from earlier in this \lcnamecref{sec:ordered-logic:verifications} that the next step toward proof normalization is to eliminate all remaining instances of the $\jrule{ID}$ rule that occur at non-atomic propositions $A$.
Before proving identity elimination, we need to prove that an identity principle is admissible for identity-long proofs.
%
\begin{lemma}[
  name=Admissibility of identity,
  label=lem:ordered-logic:identity-admissibility
]
  For all propositions $A$, an \emph{identity-long} proof of $\oseq{A |- A}$ exists.
  Moreover, this proof is cut-free.
\end{lemma}
%
\begin{proof}
  As usual, by induction on the structure of the proposition $A$.
  As before, we may represent this \lcnamecref{lem:ordered-logic:identity-admissibility} as an admissible rule:
  \begin{equation*}
    \infer-[\jrule{A-ID}\smash{^A}]{\oseql{A |- A}}{}
  \end{equation*}
  to suggest that this proof amounts to defining a meta-level function on propositions $A$.

  In the base case of propositional atoms $a$, the instance of the $\jrule{ID}$ rule at $a$ is itself already identity-long:
  \begin{equation*}
    \infer-[\jrule{A-ID}\smash{^{a}}]{\oseql{a |- a}}{}
    =
    \infer[\jrule{ID}\smash{^{a}}]{\oseql{a |- a}}{}
  \end{equation*}

  For non-atomic propositions, the identity-long proof of $\oseq{A |- A}$ is constructed from right and left rules, together with calls to the admissible $\jrule{A-ID}$ rule at subformulas of $A$.
  For example, the identity expansion at $A_1 \limp A_2$ is:
  \begin{equation*}
    \infer-[\jrule{A-ID}\smash{^{A_1 \limp A_2}}]{\oseql{A_1 \limp A_2 |- A_1 \limp A_2}}{}
    =
    \infer[\rrule{\limp}]{\oseql{A_1 \limp A_2 |- A_1 \limp A_2}}{
      \infer[\lrule{\limp}]{\oseql{A_1 \oc (A_1 \limp A_2) |- A_2}}{
        \infer-[\jrule{A-ID}\smash{^{A_1}}]{\oseql{A_1 |- A_1}}{} &
        \infer-[\jrule{A-ID}\smash{^{A_2}}]{\oseql{A_2 |- A_2}}{}}}
  \end{equation*}
  The remaining cases are similarly compositional.
\end{proof}

\orderedidelimination
%
\begin{proof}
  As usual, by structural induction on the proof of $\oseq{\octx |- A}$.
  Once again, we may represent this \lcnamecref{thm:ordered-logic:id-elimination} as an admissible rule:
  \begin{equation*}
    \infer-[\jrule{IE}]{\oseql{\octx |- A}}{
      \oseq{\octx |- A}}
  \end{equation*}

  The crucial case in the definition of this admissible rule comes when  the given proof is instance of the $\jrule{ID}$ rule.
  An appeal to the admissible $\jrule{A-ID}$ rule~\parencref{lem:ordered-logic:identity-admissibility} then yields an identity-long proof of $\oseq{\octx |- A}$:
  \begin{equation*}
    \infer-[\jrule{IE}]{\oseql{A |- A}}{
      \infer[\jrule{ID}\smash{^A}]{\oseq{A |- A}}{}}
    =
    \infer-[\jrule{A-ID}\smash{^A}]{\oseql{A |- A}}{}
  \end{equation*}
  As part of \cref{lem:ordered-logic:identity-admissibility}, we know that this proof is also cut-free.

  The remaining cases are resolved compositionally.
  For example:
  \begin{equation*}
    \infer-[\jrule{IE}]{\oseql{\octx |- A_1 \limp A_2}}{
      \infer[\rrule{\limp}]{\oseq{\octx |- A_1 \limp A_2}}{
        \deduce{\oseq{A_1 \oc \octx |- A_2}}{\DD_1}}}
    =
    \infer[\rrule{\limp}]{\oseql{\octx |- A_1 \limp A_2}}{
      \infer-[\jrule{IE}]{\oseql{A_1 \oc \octx |- A_2}}{
        \deduce{\oseq{A_1 \oc \octx |- A_2}}{\DD_1}}}
  \end{equation*}
  Notice that no case introduces any instances of the $\jrule{CUT}$ beyond those that were already present in the given proof.
  Thus, identity elimination preserves cut-freeness.
\end{proof}

\subsection{Proof normalization}

With the cut and identity elimination results~\parencref{thm:ordered-logic:cut-elimination,thm:ordered-logic:id-elimination} in hand, normalization of proofs to verification is a straightforward \lcnamecref{cor:ordered-logic:proof-normalization}:

\orderedproofnormalization
%
\begin{proof}
  Given a proof of $\oseq{\octx |- A}$, applying cut elimination~\parencref{thm:ordered-logic:cut-elimination} and identity elimination~\parencref{thm:ordered-logic:id-elimination} in sequence yields a proof that is both cut-free and long -- in other words, a verification $\oseqv{\octx |- A}$.
  Using an admissible rule, this \lcnamecref{cor:ordered-logic:proof-normalization} may be represented as
  \begin{equation*}
    \infer-[\jrule{NORM}]{\oseqv{\octx |- A}}{
      \deduce{\oseq{\octx |- A}}{\DD}}
    =
    \infer-[\jrule{IE}]{\oseqv{\octx |- A}}{
      \infer-[\jrule{CE}]{\oseqcf{\octx |- A}}{
        \deduce{\oseq{\octx |- A}}{\DD}}}
  \qedhere
  \end{equation*}
\end{proof}

By establishing that every proof has a corresponding verification, we are now assured that the ordered sequent calculus presented in \cref{fig:ordered-logic:sequent-calculus} indeed constitutes a well-defined logic with a verificationist meaning-theory.



\section{Circular propositions and proofs}\label{sec:ordered-logic:circular}

By rejecting weakening and especially contraction, ordered logic as formulated above is bounded: there is exactly one of each antecedent, with no means of producing unbounded resources.
The antecedent $A \limp A \fuse A$ will indeed transform a left-adjacent resource $A$ into a pair of resources, $A \oc A$, effectively copying the initial $A$.
But because the antecedent $A \limp A \fuse A$ is itself a use-once resource, that is not enough to produce unbounded copies of $A$.

Linear logic traditionally introduces unboundedness with its \enquote*{of course} exponential, $\bang A$.
The proposition $\bang A$ is viewed as an unbounded number of copies of resource $A$ -- as many, or as few, copies of $A$ as desired.\autocite{Girard:TCS87}
Ordered logic can be extended with a related persistence modality\autocites{Abrusci:MLQ90}{Polakow+Pfenning:TLCA99}{Polakow+Pfenning:MFPS99}, so that $\bang A$ is an unbounded number of resources $A$, and similarly $\bang (A \limp B)$ is a means for transforming, an unbounded number of times, a left-adjacent $A$ into a $B$.
%
As the notation suggests, the modality $\bang A$ is not unrelated to process replication, $!P$, in the $\pi$-calculus.\autocite{Milner:CUP99}
A replication $!P$ represents an unbounded number of processes $P$ in parallel composition, $P \mid P \mid \dotsb$, in much the same way as $\bang A$ represents an unbounded number of resources $A$.

But the modality $\bang A$ is not the only way to introduce unbounded behavior in ordered logic.
As in the $\pi$-calculus\autocites{Milner+:IC92a}{Milner+:IC92b}, recursive definitions are another path to unboundedness.
Recursive definitions have been studied extensively\autocites{Hallnas:TCS91}{Eriksson:ELP91}{Schroeder-Heister:LICS93}{McDowell+Miller:TCS00}{Tiu+Momigliano:JAL12}, but have not, to the best of our knowledge, been previously used in the context of ordered logic.
We conjecture that persistence is strictly more expressive than recursive definitions

In ordered logic, the difference between the exponential and recursive definitions is essentially one of mobility.
% The proposition $\bang (p \limp A)$ makes unbounded number of resources $p \limp A$ possible by moving $p \limp A$ into a persistent context, from which $p \limp A$ can be copied \emph{anywhere} into the ordered context, as often as desired.
% the proposition $\bang (p \limp A)$ and the definition $\defp{p} \defd A$ is one of mobility.
The persistent resource $p \limp A$ that derives from $\bang (p \limp A)$ can be copied \emph{anywhere} into the ordered context.
By making the exponential a first-class logical connective, this mobility can be exploited.
For example, the proposition $\bang (p \limp a \fuse \bang p)$ would allow $p$ to move anywhere within the ordered context.
\Textfootcite{Kanovich+:MSCS19} show that cut elimination fails for classical non-commutative linear logic when contractions is strictly local.

In contrast, recursive definitions decouple unbounded behavior from mobility and are not first-class.
The definition $\defp{p} \defd a \fuse \defp{p}$ is the nearest we can get to $\bang (p \limp a \fuse \bang p)$, and this definition does not imply mobility.
As we will see in \cref{ch:process-chains}, recursive definitions will therefore prove to be a good match for recursive processes, which ought not to be able to move around within a configuration just because they have unbounded behavior.


\Textfootcite{Fortier+Santocanale:CSL13} have used recursive definitions together with circular derivations\footnote{\change[ic]{A circular derivation is a partial proof in which the unfilled premises at the leaves point recursively back to the root of the proof or other circular derivations.}} in a fragment of the linear sequent calculus, establishing a sound condition under which these definitions constitute least and greatest fixed points and these derivations constitute valid inductive and coinductive proofs.
Extending their work, \textfootcite{Derakhshan+Pfenning:LMCS20} have presented a related, locally decidable condition on circular derivations in first-order intuitionistic multiplicative and additive linear logic that ensures cut elimination is productive.
We know of no work on applying these ideas to ordered logic, but we expect that similar results ought to hold.
In any case, in the remainder of this document we are concerned only with general recursive definitions, not inductive or coinductive definitions.

\section{Other extensions}\label{sec:ordered-logic:extensions}

Ordered logic can also be extended in other directions that we briefly describe here:
% In this \lcnamecref{sec:ordered-logic:extensions}, we give a brief overview of several extensions to the preceding ordered sequent calculus
% \begin{itemize*}[label=, before=\unskip{:}, itemjoin={,}, itemjoin*={, and}]
first-order universal and existential quantifiers,
 multiplicative falsehood and disjunction, %\autocite{Chang+:CMU03},
 a mobility modality, %\autocite{Polakow+Pfenning:MFPS99},
 the aforementioned persistence modality, and
 subexponentials, %\autocites{Nigam+Miller:PPDP09}{Kanovich+:MSCS19}
 and adjunctions from adjoint logic. %\autocites{Benton:CSL94}{Pruiksma+:18}.
These extensions are not crucial to the remainder of this dissertation, but are mentioned for the sake of completeness.

% \paragraph*{First-order quantification}

Adding first-order universal and existential quantifiers, $\forall x{:}\tau.A$ and $\exists x{:}\tau.A$, to the ordered sequent calculus is completely standard.
Sequents are extended with a separate context of well-sorted term variables, $x{:}\tau$; this new context is structural, admitting weakening, contraction, and exchange properties.

  % \begin{inferences}
  %   \infer[\rrule{\forall}]{\oseq{\Sigma ; \octx |- \forall x{:}\tau.A}}{
  %     \oseq{\Sigma, a{:}\tau ; \octx |- [a/x]A}}
  %   \and
  %   \infer[\lrule{\forall}]{\oseq{\Sigma ; \octx'_L \oc (\forall x{:}\tau.A) \oc \octx'_R |- C}}{
  %     \Sigma \vdash t : \tau &
  %     \oseq{\Sigma ; \octx'_L \oc ([t/x]A) \oc \octx'_R |- C}}
  %   \\
  %   \infer[\rrule{\exists}]{\oseq{\Sigma ; \octx |- \exists x{:}\tau.A}}{
  %     \Sigma \vdash t : \tau &
  %     \oseq{\Sigma ; \octx |- [t/x]A}}
  %   \and
  %   \infer[\lrule{\exists}]{\oseq{\Sigma ; \octx'_L \oc (\exists x{:}\tau.A) \oc \octx'_R |- C}}{
  %     \oseq{\Sigma, a{:}\tau ; \octx'_L \oc ([a/x]A) \oc \octx'_R |- C}}
  % \end{inferences}

% \paragraph*{Multiplicative falsehood}

Multiplicative falsehood can be introduced into
% along a different dimension, 
the ordered sequent calculus, as in the intuitionistic linear sequent calculus:
by generalizing % can be generalized to allow
sequents to % carry
allow
an empty consequent, $\oseq{\octx |- (\cdot)}$.\autocite{Chang+:CMU03}
With this new judgment form, the cut principle and left rules must be revised to allow the empty consequent.
% For example, the $\jrule{CUT}$ and $\lrule{\fuse}$ rules are revised to:
% \begin{inferences}
%   \infer[\jrule{CUT}\smash{^A}]{\oseq{\octx'_L \oc \octx \oc \octx'_R |- \gamma}}{
%     \oseq{\octx |- A} & \oseq{\octx'_L \oc A \oc \octx'_R |- \gamma}}
%   \and
%   \infer[\lrule{\fuse}]{\oseq{\octx'_L \oc (A \fuse B) \oc \octx'_R |- \gamma}}{
%     \oseq{\octx'_L \oc A \oc B \oc \octx'_R |- \gamma}}
% \end{inferences}
% where $\gamma$ is a metavariable%
% \marginnote{%
%   $
%     \gamma \Coloneqq \cdot \mid C
%   $
% }
% standing for a consequent, either empty, $\cdot$, or a single proposition, $C$.
% 
% This new judgment makes it possible to introduce \emph{multiplicative falsehood}, $\bot$, as a logical constant.
Multiplicative falsehood, $\bot$, internalizes this judgment as a proposition and is, as its name suggests, dual to multiplicative truth, $\one$.
% Its right and left rules are:
% \begin{inferences}
%   \infer[\rrule{\bot}]{\oseq{\octx |- \bot}}{
%     \oseq{\octx |- \cdot}}
%   \and
%   \infer[\lrule{\bot}]{\oseq{\bot |- \cdot}}{}
% \end{inferences}
Multiplicative disjunction, $A \boxplus B$, can also be introduced like in the intuitionistic linear sequent calculus; it requires multiple-conclusion sequents.\autocite{Chang+:CMU03}

% \paragraph*{Mobility and persistence modalities}
% Linear\fixnote{Circular proofs, too!}

In addition to the aforementioned persistence modality, $\bang A$, it is possible to introduce a mobility modality, $\gnab A$.\autocite{Polakow+Pfenning:MFPS99}
Just as persistence is subject to all of the structural properties, $\gnab A$ is subject to exchange (but neither weakening nor contraction).
In this way, $\gnab A$ represents a mobile resource that may permute with other resources.
Related to these modalities are subexponentials\autocites{Nigam+Miller:PPDP09}{Kanovich+:MSCS19} and adjunctions from adjoint logic\autocites{Benton:CSL94}{Pruiksma+:18}.
Both subexponentials and adjunctions allow ordered logic to include multiple distinct layers, each with its own set of structural properties (that must, however, meet certain conditions for cut elimination).





% \clearpage

% Ordered logic is a generalization of \citeauthor{Girard:TCS87}'s linear logic\autocite{Girard:TCS87} that further restricts the admitted structural properties.
% In addition to ... the weakening and contraction

% Like the intuitionistic linear sequent calculus, the ordered sequent calculus can be given a resource interpretation.
% Because the ordered sequent calculus rejects the exchange property, the resouce interpretation differs slightly from that of linear logic.

% \begin{itemize}
% \item \citeauthor{Martin-Lof:NJPL96}: Distinguish propositions from judgments

% \item Ordered sequents are hypothetical judgments
%   \begin{equation*}
%     \oseq{A_1 \oc A_2 \dotsb A_n |- A} \,,
%   \end{equation*}
%   meaning that using the (ordered) resources $A_1, A_2, \dotsc, A_n$, the resource $A$ can be produced.
%   A proof of this sequent amounts to a recipe or set of instructions for producing resource $A$ from resources $A_1 \oc A_2 \dotsb A_n$.

% \item Contexts as (noncommutative) free monoid over resources -- associative and unit laws

% \item $A \fuse B$ as resources $A$ and $B$ side by side, as a single resource package.
%   \begin{equation*}
%     \infer[\rrule{\fuse}]{\oseq{\octx_1 \oc \octx_2 |- A \fuse B}}{
%       \oseq{\octx_1 |- A} & \oseq{\octx_2 |- B}}
%   \end{equation*}
%   The sequence of resources $\octx$ can be used to produce the resource $A \fuse B$ if the resources $\octx$ can be partitioned into subsequences $\octx_1$ and $\octx_2$ and those subsequences can be used to produce the resources $A$ and $B$, respectively.
%   \begin{equation*}
%     \infer[\lrule{\fuse}]{\oseq{\octx'_L \oc (A \fuse B) \oc \octx'_R |- C}}{
%       \oseq{\octx'_L \oc A \oc B \oc \octx'_R |- C}}
%   \end{equation*}
%   To use the resource package $A \fuse B$ to produce $C$, unwrap the package and use the side-by-side resources $A \oc B$ to produce $C$.

% \item Harmony:
%   \begin{gather*}
%     \infer[\jrule{CUT}\smash{^{A \fuse B}}]{\oseq{\octx'_L \oc (\octx_1 \oc \octx_2) \oc \octx'_R |- C}}{
%       \infer[\rrule{\fuse}]{\oseq{\octx_1 \oc \octx_2 |- A \fuse B}}{
%         \oseq{\octx_1 |- A} & \oseq{\octx_2 |- B}} &
%       \infer[\lrule{\fuse}]{\oseq{\octx'_L \oc (A \fuse B) \oc \octx'_R |- C}}{
%         \oseq{\octx'_L \oc A \oc B \oc \octx'_R |- C}}}
%     \\\rightsquigarrow\\
%     \infer[\jrule{CUT}\smash{^B}]{\oseq{\octx'_L \oc \octx_1 \oc \octx_2 \oc \octx'_R |- C}}{
%       \oseq{\octx_2 |- B} &
%       \infer[\jrule{CUT}\smash{^A}]{\oseq{\octx'_L \oc \octx_1 \oc B \oc \octx'_R |- C}}{
%         \oseq{\octx_1 |- A} & \oseq{\octx'_L \oc A \oc B \oc \octx'_R |- C}}}
%   \end{gather*}

%   \begin{equation*}
%     \infer[\jrule{ID}\smash{^{A \fuse B}}]{\oseq{A \fuse B |- A \fuse B}}{}
%     \leftrightsquigarrow
%     \infer[\lrule{\fuse}]{\oseq{A \fuse B |- A \fuse B}}{
%       \infer[\rrule{\fuse}]{\oseq{A \oc B |- A \fuse B}}{
%         \infer[\jrule{ID}\smash{^A}]{\oseq{A |- A}}{} &
%         \infer[\jrule{ID}\smash{^B}]{\oseq{B |- B}}{}}}
%   \end{equation*}

% \item How do we know that these rules constitute the beginnings of a logic?
%   Either \citeauthor{Dummett:HUP91} (harmony of the logical rules) or \citeauthor{Martin-Lof:NJPL96} (verifications).
%   \begin{itemize}
%   \item These are not unrelated, but we will adopt verifications as our basis. 
%   \item Verifications use only the logical rules, not the judgmental $\jrule{CUT}$ and $\jrule{ID}$.
%     This way, the meaning of a proposition depends only on its internal structure, not on other propositions or connectives.
%   \end{itemize}

% \end{itemize}



% \begin{align*}
%   n(\slof{A |- \spawn{P_1}{^B P_2} : C}) &= \nspawn{n(\slof{A |- P_1 : B})}{n(\slof{B |- P_2 : C})} \\
%   n(\slof{A |- \fwd : A}) &= \eta(A) \\
%   n(\selectR{\kay}) &= \selectR{\kay} \\
%   n(\slof{\plus*[sub=_{\ell \in L}]{\ell:A_{\ell}} |- \caseL[\ell \in L]{\ell => P_{\ell}} : C}) &= \caseL[\ell \in L]{\ell => n(\slof{A_{\ell} |- P_{\ell} : C})}
% \end{align*}

% \begin{align*}
%   \nspawn{\fwd}{M} &= M \\
%   \nspawn{(\spawn{N_0}{\selectR{\kay}})}{M} &= \nspawn{N_0}{(\nspawn{\selectR{\kay}}{M})} \\
%   \nspawn{\selectR{\kay}}{\caseL[\ell \in L]{\ell => M_{\ell}}} &= M_{\kay} \\
%   \nspawn{N}{\selectR{\kay}} &= \spawn{N}{\selectR{\kay}} \\
%   \nspawn{N}{(\spawn{M_0}{\selectR{\kay}})} &= \spawn{(\nspawn{N}{M_0})}{\selectR{\kay}} \\
%   \nspawn{\caseL[\ell \in L]{\ell => N_{\ell}}}{M} &= \caseL[\ell \in L]{\ell => \nspawn{N_{\ell}}{M}}
% \end{align*}


% \clearpage













% \section{The non-modal fragment of intuitionistic ordered logic}

% \subsection{Judgments, sequents, and contexts}

% Following \citeauthor{Martin-Lof:NJPL96}\autocite{Martin-Lof:NJPL96}, we maintain a separation of ordered propositions, $A$, from judgments about those propositions.
%   The categorical judgment $A \ord$ ... 

% To allow hypothetical reasoning, a new form of categorical judgment, $A \ant$, for antecedents and generalize $A \ord$ to sequents
%   \begin{equation*}
%     \oseq{(A_1 \ant) \dotsm (A_i \ant) \dotsm (A_n \ant) |- A \ord}
%     \,,
%   \end{equation*}
%   meaning the ordered sequence $A_1 \ant \dotsm A_n \ant$ can be transformed into $A \ord$.
%   Because the judgment label can be inferred from a proposition's position in the sequent, we usually elide the labels and write $\oseq{A_1 \oc A_2 \dotsm A_n |- A}$.

% To further streamline the syntax of sequents,
% % keep sequents from becoming verbose and cumbersome,
% antecedents are usually collected into an \vocab{ordered context}, $\octx = A_1 \oc A_2 \dotsm A_n$; sequents are then written $\oseq{\octx |- A}$.
% % 
% As strings of antecedents, ordered contexts form a free monoid.
% The monoid operation is concatenation of contexts, written as juxtaposition; the unit element is the empty context, written as $\octxe$.
% In other words, ordered contexts are generated by the grammar
% \begin{syntax*}
%   & \octx & \octx_1 \oc \octx_2 \mid \octxe \mid A \ant
%   \,,
% \end{syntax*}
% and are subject to the usual associativity and unit laws.
% \begin{marginfigure}
%   % \vspace*{\dimexpr-\abovedisplayskip\relax}
%   \begin{gather*}
%     (\octx_1 \oc \octx_2) \oc \octx_3 = \octx_1 \oc (\octx_2 \oc \octx_3) \\
%     (\octxe) \oc \octx = \octx = \octx \oc (\octxe)
%   \end{gather*}
%   \caption{Monoid laws for ordered contexts}
% \end{marginfigure}%
% % which may be silently applied as needed within proofs.
% % These monoid laws are applied implicitly as needed within a proof.
% Because contexts are ordered, the underlying monoid is not commutative.


% \subsection{}

% Following \textcite{Martin-Lof:NJPL96}, we maintain a separation of ordered propositions from judgements about those propositions.
% The categorical judgement $A \ord$ holds if language $A$ is inhabited [if there exists a string of type $A$].

% To allow hypothetical reasoning, the judgement $A \ord$ is generalized to sequents
% \begin{equation*}
%   \oseq{(A_1 \ant) \oc (A_2 \ant) \dotsb (A_n \ant) |- A \ord}
%   \,,
% \end{equation*}
% meaning that any string drawn from the concatenation of languages $A_1 \oc A_2 \dotsb A_n$ is a member of the language $A$.
% ... if the concatenation of languages ... is inhabited, then any inhabitant also inhabits the language $A$.

% \begin{itemize}
% \item $\oseq{A_1 \oc A_2 \dotsb A_n |- A}$ means that a string of type $A$ may be obtained from the concatenation of strings of types $A_1, A_2, \dotsc, A_n$.
% \item $\oseq{A_1 \oc A_2 \dotsb A_n |- A}$ means that any string expressible as the concatenation of strings of types $A_1, A_2, \dotsc, A_n$ is also a string of type $A$.
% \end{itemize}

% \begin{itemize}
% \item $a$ -- trivial language containing only the single-letter string $a$
% \item $A \fuse B$ -- concatenation of languages
% \item $\one$ -- trivial language containing only the empty string
% \item $A \with B$ -- intersection of languages
% \item $\top$ -- universal language
% \item $A \plus B$ -- union of languages
% \item $\zero$ -- empty language
% \item $A \limp B$ -- left quotient of languages
% \item $B \pmir A$ -- right quotient of languages
% \end{itemize}

% $\vdash$ as language inclusion?
% What about the language $a \with b$?  The intersection of languages $a$ and $b$ is empty, but $a \with b \nvdash \zero$.

% The right rule says that any string that can be partitioned into consecutive pieces drawn from the languages $\octx_1$ and $\octx_2$ is a member of language $A \fuse B$ if $\octx_1$ is contained in $A$ and $\octx_2$ is contianed in $B$.
% \begin{inferences}
%   \infer[\rrule{\fuse}]{\oseq{\octx_1 \oc \octx_2 |- A \fuse B}}{
%     \oseq{\octx_1 |- A} & \oseq{\octx_2 |- B}}
%   \and
%   \infer[\lrule{\fuse}]{\oseq{\octx'_L \oc (A \fuse B) \oc \octx'_R |- C}}{
%     \oseq{\octx'_L \oc A \oc B \oc \octx'_R |- C}}
% \end{inferences}

% \subsection{Judgmental principles of sequents}

% Even without knowing the specific structure of propositions, we can already enumerate several judgmental principles that must hold if sequents are to accurately reflect hypothetical reasoning.

% First, if we assume that $A$ is ..., then 

% If we are given a string that parses to $A$, then that string may be trivially transformed to a string that parses to $A$ -- simply use the same string.
% Cast as a sequent:
% \begin{description}
% \item[Identity principle]
%   $\oseq{A |- A}$ for all propositions $A$.
% \end{description}

% Dually, a string that parses to $A$ licenses a hypothesis of $A$:
% \begin{description}
% \item[Cut principle]
%   If $\oseq{\octx |- A}$ and $\oseq{\octx'_L \oc A \oc \octx'_R |- C}$, then $\oseq{\octx'_L \oc \octx \oc \octx'_R |- C}$.
% \end{description}

% These two judgmental principles are adopted by the ordered sequent calculus as primitive rules of inference.
% But their importance goes beyound that of mere rules of inference, with the two principles playing an important role in the meaning of the logical connectives.

% \subsection{The logical connectives and their meanings}

% \begin{inferences}
%   \infer[\rrule{\fuse}]{\oseq{\octx_1 \oc \octx_2 |- A \fuse B}}{
%     \oseq{\octx_1 |- A} & \oseq{\octx_2 |- B}}
%   \and
%   \infer[\lrule{\fuse}]{\oseq{\octx'_L \oc (A \fuse B) \oc \octx'_R |- C}}{
%     \oseq{\octx'_L \oc A \oc B \oc \octx'_R |- C}}
% \end{inferences}



% \clearpage 

% \newthought{Even without} knowing any specifics about propositions, two purely judgmental principles are already apparent.
% %
% First, there should be a trivial transformation of $A$ into $A$, for all propositions $A$.
% In the sequent calculus, this idea is rendered as an \vocab{identity principle}:
% \begin{quotation}
%   $\oseq{A \ant |- A \ord}$ for all propositions $A$.
% \end{quotation}
% Stated differently, $\vdash$ is reflexive.%
% \footnote{Actually, this is not precisely true because the two sides of the turnstile use different judgments.
% The intuition is nevertheless useful.}

% Second, and dually, a proof of $A \ord$ should license the use of $A \ant$ in another proof.
% \begin{quotation}
%   If $\oseq{\octx |- A \ord}$ and $\oseq{\octx'_L \oc (A \ant) \oc \octx'_R |- C \ord}$, then $\oseq{\octx'_L \oc \octx \oc \octx'_R |- C \ord}$.
% \end{quotation}

% These two judgmental principles are adopted as primitive rules of inference in the sequent calculus.
% \begin{inferences}
%   \infer[\jrule{CUT}\smash{^A}]{\oseq{\octx'_L \oc \octx \oc \octx'_R |- C}}{
%     \oseq{\octx |- A} & \oseq{\octx'_L \oc A \oc \octx'_R |- C}}
%   \and
%   \infer[\jrule{ID}\smash{^A}]{\oseq{A |- A}}{}
% \end{inferences}
% The importance of the cut and identity principles goes beyond that of their corresponding inference rules, however.
% Both principles are intimately linked to what counts as the meanings of the logical connectives.

% \subsection{Propositions}

% The propositional, purely ordered fragment of intuitionistic ordered logic has propositions given by the following grammar.
% \begin{syntax*}
%   Propositions &
%     A,B,C & \begin{array}[t]{@{}l@{}}
%               \alpha \mid A \limp B \mid B \pmir A
%                 \mid A \fuse B \mid \one \\
%               \mathllap{\mid {}} A \with B \mid \top
%                 \mid A \plus B \mid \zero
%             \end{array}
% \end{syntax*}
% The ordered propositions in this fragment are:
% propositional variables, $\alpha$;
% left- and right-handed implications, $A \limp B$ and $B \pmir A$, respectively;
% ordered conjunction, $A \fuse B$, and its unit, $\one$;
% alternative conjunction, $A \with B$, and its unit, $\top$;
% and
% additive disjunction, $A \plus B$, and its unit, $\zero$.

% In the tradition of \citeauthor{Gentzen:MZ35} and \citeauthor{Martin-Lof:NJPL96}\autocites{Gentzen:MZ35}{Martin-Lof:NJPL96}, the meaning of a proposition $A$ is given by what counts as a verification of the judgment $A \ord$.

% Left-handed implication has the following right and left rules.
% \begin{inferences}
%   \infer[\rrule{\limp}]{\oseq{\octx |- A \limp B}}{
%     \oseq{A \oc \octx |- B}}
%   \and
%   \infer[\lrule{\limp}]{\oseq{\octx'_L \oc \octx \oc (A \limp B) \oc \octx'_R |- C}}{
%     \oseq{\octx |- A} & \oseq{\octx'_L \oc B \oc \octx'_R |- C}}
% \end{inferences}
% According to the right rule, $\rrule{\limp}$, verifying the left-handed implication $A \limp B$ amounts to verifying $B$ under the left-extended context $A \oc \octx$ -- that is, a hypothetical proof of $\oseq{A \oc \octx |- B}$.
% The left rule, $\lrule{\limp}$, shows how to use such a proof:
% Prove $A$ and left-adjoin that proof to the verification of $A \limp B$, which is just a hypothetical proof of $\oseq{A \oc \octx |- B}$.
% This yields 

% Right-handed implication is symmetric to its left-handed counterpart:
% \begin{inferences}
%   \infer[\rrule{\pmir}]{\oseq{\octx |- B \pmir A}}{
%     \oseq{\octx \oc A |- B}}
%   \and
%   \infer[\lrule{\pmir}]{\oseq{\octx'_L \oc (B \pmir A) \oc \octx \oc \octx'_R |- C}}{
%     \oseq{\octx |- A} & \oseq{\octx'_L \oc B \oc \octx'_R |- C}}
% \end{inferences}

% Ordered conjunction is another multiplicative connective; its right and left rules are:
% \begin{inferences}
%   \infer[\rrule{\fuse}]{\oseq{\octx_1 \oc \octx_2 |- A \fuse B}}{
%     \oseq{\octx_1 |- A} & \oseq{\octx_2 |- B}}
%   \and
%   \infer[\lrule{\fuse}]{\oseq{\octx'_L \oc (A \fuse B) \oc \octx'_R |- C}}{
%     \oseq{\octx'_L \oc A \oc B \oc \octx'_R |- C}}
% \end{inferences}
% As its right rule makes clear, ordered conjunction internalizes concatenation of contexts as a logical connective.

% $\one$ internalizes the empty context.
% The logical constant $\one$ is the nullary analogue to binary ordered conjunction, as its right and left rules reflect:
% \begin{inferences}
%   \infer[\rrule{\one}]{\oseq{\octxe |- \one}}{}
%   \and
%   \infer[\lrule{\one}]{\oseq{\octx'_L \oc \one \oc \octx'_R |- C}}{
%     \oseq{\octx'_L \oc \octx'_R |- C}}
% \end{inferences}
% Consequently, $\one$ is the unit of ordered conjunction: $\oseq{\one \fuse A \dashv|- \oseq{A \dashv|- A \fuse \one}}$, for all propositions $A$.


% \begin{equation*}
%   \infer[\rrule{\limp}]{\oseq{A \limp (B \limp C) |- (A \esuf B) \limp C}}{
%     \infer[\lrule{\esuf}]{\oseq{(A \esuf B) \oc (A \limp (B \limp C)) |- C}}{
%       \infer[\lrule{\limp}]{\oseq{B \oc A \oc (A \limp (B \limp C)) |- C}}{
%         \infer[\jrule{ID}]{\oseq{A |- A}}{} &
%         \infer[\lrule{\limp}]{\oseq{B \oc (B \limp C) |- C}}{
%           \infer[\jrule{ID}]{\oseq{B |- B}}{} &
%           \infer[\jrule{ID}]{\oseq{C |- C}}{}}}}}
% \end{equation*}

% \begin{equation*}
%   \infer[\rrule{\limp}]{\oseq{(A \esuf B) \limp C |- A \limp (B \limp C)}}{
%     \infer[\rrule{\limp}]{\oseq{A \oc ((A \esuf B) \limp C) |- B \limp C}}{
%       \infer[\lrule{\limp}]{\oseq{B \oc A \oc ((A \esuf B) \limp C) |- C}}{
%         \infer[\rrule{\esuf}]{\oseq{B \oc A |- A \esuf B}}{
%           \infer[\jrule{ID}]{\oseq{B |- B}}{} &
%           \infer[\jrule{ID}]{\oseq{A |- A}}{}} &
%         \infer[\jrule{ID}]{\oseq{C |- C}}{}}}}
% \end{equation*}

% \begin{equation*}
%   \infer[\rrule{\pmir}]{\oseq{(C \pmir B) \pmir A |- C \pmir (B \esuf A)}}{
%     \infer[\lrule{\esuf}]{\oseq{((C \pmir B) \pmir A) \oc (B \esuf A) |- C}}{
%       \infer[\lrule{\pmir}]{\oseq{((C \pmir B) \pmir A) \oc A \oc B |- C}}{
%         \infer[\jrule{ID}]{\oseq{A |- A}}{} &
%         \infer[\lrule{\pmir}]{\oseq{(C \pmir B) \oc B |- C}}{
%           \infer[\jrule{ID}]{\oseq{B |- B}}{} &
%           \infer[\jrule{ID}]{\oseq{C |- C}}{}}}}}
% \end{equation*}

% \begin{equation*}
%   \infer[\rrule{\pmir}]{\oseq{C \pmir (B \esuf A) |- (C \pmir B) \pmir A}}{
%     \infer[\rrule{\pmir}]{\oseq{(C \pmir (B \esuf A)) \oc A |- C \pmir B}}{
%       \infer[\lrule{\pmir}]{\oseq{(C \pmir (B \esuf A)) \oc A \oc B |- C}}{
%         \infer[\rrule{\esuf}]{\oseq{A \oc B |- B \esuf A}}{
%           \infer[\jrule{ID}]{\oseq{A |- A}}{} &
%           \infer[\jrule{ID}]{\oseq{B |- B}}{}} &
%         \infer[\jrule{ID}]{\oseq{C |- C}}{}}}}
% \end{equation*}

% \begin{itemize}
% \item Contexts form a monoid
% \item None of the usual structural properties -- weakening, contraction, exchange -- but we still have (implicitly, judgmentally) associativity
% \item Multiplicative falsehood\alertnote{Get rid of this?}
% \end{itemize}

% \subsection{Rules of logical inference}

% According to its right rule, verifying $A \limp B$ amounts to verifying $B$ under the additional assumption that $A$ holds;
% moreover, the assumption $A$ is prepended to the left end of context $\octx$, because $A \limp B$ is a left-handed implication.

% To use a verification of $A \limp B$, we first verify $A$ and are thus justified in using $B$.

% Using a verification of $A \limp B$ thus amounts to using the hypothetical verification of $B$ under $A$


% The right-handed implication $B \pmir A$ is symmetric to left-handed implication.


% \begin{figure}
%   \begin{syntax*}
%     Propositions & A, B, C &
%       \begin{array}[t]{@{}l@{}}
%         A \limp B \mid B \pmir A \mid A \fuse B \mid B \esuf A \mid \one \\
%         \mathllap{\mid {}}
%         A \plus B \mid \zero \mid A \with B \mid \top
%       \end{array}
%     \\
%     Contexts & \octx &
%       \octxe \mid \octx_1 \oc \octx_2 \mid A
%   \end{syntax*}
%   \begin{inferences}
%     \infer[\jrule{CUT}\smash{^A}]{\oseq{\octx'_L \oc \octx \oc \octx'_R |- \cseq}}{
%       \oseq{\octx |- A} & \oseq{\octx'_L \oc A \oc \octx'_R |- \cseq}}
%     \and
%     \infer[\jrule{ID}\smash{^A}]{\oseq{A |- A}}{}
%     \\
%     \infer[\rrule{\limp}]{\oseq{\octx |- A \limp B}}{
%       \oseq{A \oc \octx |- B}}
%     \and
%      \infer[\lrule{\limp}]{\oseq{\octx'_L \oc \octx \oc (A \limp B) \oc \octx'_R |- C}}{
%       \oseq{\octx |- A} & \oseq{\octx'_L \oc B \oc \octx'_R |- C}}
%     \\
%     \infer[\rrule{\pmir}]{\oseq{\octx |- B \pmir A}}{
%       \oseq{\octx \oc A |- B}}
%     \and
%     \infer[\lrule{\pmir}]{\oseq{\octx'_L \oc (B \pmir A) \oc \octx \oc \octx'_R |- C}}{
%       \oseq{\octx |- A} & \oseq{\octx'_L \oc B \oc \octx'_R |- C}}
%     \\
%     \infer[\rrule{\fuse}]{\oseq{\octx_A \oc \octx_B |- A \fuse B}}{
%       \oseq{\octx_A |- A} & \oseq{\octx_B |- B}}
%     \and
%     \infer[\lrule{\fuse}]{\oseq{\octx'_L \oc (A \fuse B) \oc \octx'_R |- C}}{
%       \oseq{\octx'_L \oc A \oc B \oc \octx'_R |- C}}
%     \\
%     \infer[\rrule{\esuf}]{\oseq{\octx_A \oc \octx_B |- B \esuf A}}{
%       \oseq{\octx_A |- A} & \oseq{\octx_B |- B}}
%     \and
%     \infer[\lrule{\esuf}]{\oseq{\octx'_L \oc (B \esuf A) \oc \octx'_R |- C}}{
%       \oseq{\octx'_L \oc A \oc B \oc \octx'_R |- C}}
%     \\
%     \infer[\rrule{\one}]{\oseq{\octxe |- \one}}{}
%     \and
%     \infer[\lrule{\one}]{\oseq{\octx'_L \oc \one \oc \octx'_R |- C}}{
%       \oseq{\octx'_L \oc \octx'_R |- C}}
%     \\
%     \infer[\rrule{\with}]{\oseq{\octx |- A \with B}}{
%       \oseq{\octx |- A} & \oseq{\octx |- B}}
%     \and
%     \infer[\lrule{\with}_1]{\oseq{\octx'_L \oc (A \with B) \oc \octx'_R |- C}}{
%       \oseq{\octx'_L \oc A \oc \octx'_R |- C}}
%     \and
%     \infer[\lrule{\with}_2]{\oseq{\octx'_L \oc (A \with B) \oc \octx'_R |- C}}{
%       \oseq{\octx'_L \oc B \oc \octx'_R |- C}}
%     \\
%     \infer[\rrule{\top}]{\oseq{\octx |- \top}}{}
%     \and
%     \text{(no $\lrule{\top}$ rule)}
%     \\
%     \infer[\rrule{\plus}_1]{\oseq{\octx |- A \plus B}}{
%       \oseq{\octx |- A}}
%     \and
%     \infer[\rrule{\plus}_2]{\oseq{\octx |- A \plus B}}{
%       \oseq{\octx |- B}}
%     \and
%     \infer[\lrule{\plus}]{\oseq{\octx'_L \oc (A \plus B) \oc \octx'_R |- C}}{
%       \oseq{\octx'_L \oc A \oc \octx'_R |- C} &
%       \oseq{\octx'_L \oc B \oc \octx'_R |- C}}
%     \\
%     \text{(no $\rrule{\zero}$ rule)}
%     \and
%     \infer[\lrule{\zero}]{\oseq{\octx'_L \oc \zero \oc \octx'_R |- C}}{}
%   \end{inferences}
% \end{figure}  

% \begin{theorem}[Admissibility of cut]
%   If $\oseq{\octx |- A}$ and $\oseq{\octx'_L \oc A \oc \octx'_R |- C}$, then $\oseq{\octx'_L \oc \octx \oc \octx'_R |- C}$.
% \end{theorem}
% \begin{proof}
%   By lexicographic structural induction, first on the structure of the cut formula and then simultaneously on the structures of the given derivations.

%   \begin{description}
%   \item[Identity cases]
%     \begin{equation*}
%       \infer-[\jrule{CUT}^A]{\oseq{\octx'_L \oc A \oc \octx'_R |- C}}{
%         \infer[\jrule{ID}^A]{\oseq{A |- A}}{} &
%         \deduce{\oseq{\octx'_L \oc A \oc \octx'_R |- C}}{\EE}}
%       =
%       \deduce{\oseq{\octx'_L \oc A \oc \octx'_R |- C}}{\EE}
%     \end{equation*}

%   \item[Principal cases]
%     \begin{equation*}
%       \infer-[\jrule{CUT}^{A_1 \limp A_2}]{\oseq{\octx'_L \oc \octx'_{A_1} \oc \octx \oc \octx'_R |- C}}{
%         \infer[\rrule{\limp}]{\oseq{\octx |- A_1 \limp A_2}}{
%           \deduce{\oseq{A_1 \oc \octx |- A_2}}{\DD_1}} &
%         \infer[\lrule{\limp}]{\oseq{\octx'_L \oc \octx'_{A_1} \oc (A_1 \limp A_2) \oc \octx'_R |- C}}{
%           \deduce{\oseq{\octx'_{A_1} |- A_1}}{\EE_1} &
%           \deduce{\oseq{\octx'_L \oc A_2 \oc \octx'_R |- C}}{\EE_2}}}
%       =
%       \infer-[\jrule{CUT}^{A_2}]{\oseq{\octx'_L \oc \octx'_{A_1} \oc \octx \oc \octx'_R |- C}}{
%         \infer-[\jrule{CUT}^{A_1}]{\oseq{\octx'_{A_1} \oc \octx |- A_2}}{
%           \deduce{\oseq{\octx'_{A_1} |- A_1}}{\EE_1} &
%           \deduce{\oseq{A_1 \oc \octx |- A_2}}{\DD_1}} &
%         \deduce{\oseq{\octx'_L \oc A_2 \oc \octx'_R |- C}}{\EE_2}}
%     \end{equation*}

%   \item[Commutative cases]
%     \begin{equation*}
%       \infer-[\jrule{CUT}^A]{\oseq{\octx'_L \oc \octx_L \oc (B_1 \with B_2) \oc \octx_R \oc \octx'_R |- C}}{
%         \infer[\lrule{\with}_1]{\oseq{\octx_L \oc (B_1 \with B_2) \oc \octx_R |- A}}{
%           \deduce{\oseq{\octx_L \oc B_1 \oc \octx_R |- A}}{\DD_1}} &
%         \deduce{\oseq{\octx'_L \oc A \oc \octx'_R |- C}}{\EE}}
%       =
%       \infer[\lrule{\with}_1]{\oseq{\octx'_L \oc \octx_L \oc (B_1 \with B_2) \oc \octx_R \oc \octx'_R |- C}}{
%         \infer-[\jrule{CUT}^A]{\oseq{\octx'_L \oc \octx_L \oc B_1 \oc \octx_R \oc \octx'_R |- C}}{
%           \deduce{\oseq{\octx_L \oc B_1 \oc \octx_R |- A}}{\DD_1} &
%           \deduce{\oseq{\octx'_L \oc A \oc \octx'_R |- C}}{\EE}}}
%     \end{equation*}

%     \begin{equation*}
%       \infer-[\jrule{CUT}^A]{\oseq{\octx'_{L1} \oc \octx \oc \octx'_{L2} \oc (B_1 \with B_2) \oc \octx'_R |- C}}{
%         \deduce{\oseq{\octx |- A}}{\DD} &
%         \infer[\lrule{\with}_1]{\oseq{\octx'_{L1} \oc A \oc \octx'_{L2} \oc (B_1 \with B_2) \oc \octx'_R |- C}}{
%           \deduce{\oseq{\octx'_{L1} \oc A \oc \octx'_{L2} \oc B_1 \oc \octx'_R |- C}}{\EE_1}}}
%       =
%     \end{equation*}
%   \end{description}
% \end{proof}

% \begin{theorem}[Identity expansion]
%   $\oseq{A |- A}$ for all propositions $A$.
% \end{theorem}
% \begin{proof}
%   By induction on the structure of the proposition $A$.
% \end{proof}

% \section{Extensions}\label{sec:ordered-logic:extensions}

% In this \lcnamecref{sec:ordered-logic:extensions}, we give a brief overview of several extensions to the preceding ordered sequent calculus
% \begin{itemize*}[label=, before=\unskip{:}, itemjoin={,}, itemjoin*={, and}]
% \item first-order universal and existential quantifiers
% \item multiplicative falsehood
% \item mobility and persistence modalities
% \end{itemize*}.
% These extensions are not crucial to the remainder of this dissertation, but are included for the sake of completeness.

% \paragraph*{First-order quantification}

% The manner in which first-order universal and existential quantifiers, $\forall x{:}\tau.A$ and $\exists x{:}\tau.A$, may be added to the ordered sequent calculus is completely standard.
% Sequents are extended with a separate context, $\Sigma$, of well-sorted term variables, $x{:}\tau$; this new context is structural, admitting weakening, contraction, and exchange properties.

% \begin{marginfigure}
%   \begin{inferences}
%     \infer[\rrule{\forall}]{\oseq{\Sigma ; \octx |- \forall x{:}\tau.A}}{
%       \oseq{\Sigma, a{:}\tau ; \octx |- [a/x]A}}
%     \\
%     \infer[\lrule{\forall}]{\oseq{\Sigma ; \octx'_L \oc (\forall x{:}\tau.A) \oc \octx_R |- C}}{
%       \Sigma \vdash t : \tau &
%       \oseq{\Sigma ; \octx_L \oc ([t/x]A) \oc \octx_R |- C}}
%     \\
%     \infer[\rrule{\exists}]{\oseq{\Sigma ; \octx |- \exists x{:}\tau.A}}{
%       \Sigma \vdash t : \tau &
%       \oseq{\Sigma ; \octx |- [t/x]A}}
%     \\
%     \infer[\lrule{\exists}]{\oseq{\Sigma ; \octx'_L \oc (\exists x{:}\tau.A) \oc \octx'_R |- C}}{
%       \oseq{\Sigma, a{:}\tau ; \octx'_L \oc ([a/x]A) \oc \octx'_R |- C}}
%   \end{inferences}
% \end{marginfigure}

% \subsection{Multiplicative falsehood}

% \subsection{Mobility and persistence modalities}



% \section{Circular propositions and circular derivations}

% \begin{itemize}
% \item No exponentials; recursion/circularity instead (Milner)\alertnote{Should this go in ordered rewriting chapter instead?}
% \item $\mu$MALL (Baelde) and circular proofs (Fortier and Santocanale)
% \item Contractivity requirement
% \item We will use only general recursion.
%   Inductive and coinductive types are outside our scope.
% \item Subset of infinite propositions/derivations
% \end{itemize}

% \section{Outline}

% \subsection{Judgments, contexts, and sequents}

% \begin{itemize}
% \item Consequent and antecedent judgments 
%   \begin{itemize}
%   \item Hypothetical reasoning -- how to gloss an ordered sequent?
%   \item Drop judgment labels because they are implied from position in sequents
%   \end{itemize}
% \item Ordered contexts: monoidal structure and structural properties as algebraic laws
% \item Judgmental principles -- identity and cut 
%   \begin{itemize}
%   \item First rules of inference, but foreshadow connections to verifications 
%   \end{itemize}
% \end{itemize}

% \subsection{Propositions and their meanings}

% \begin{itemize}
% \item right and left rules 
%   \begin{itemize}
%   \item right rules show how to verify propositions; left rules show how to use those verifications 
%   \end{itemize}
% \end{itemize}


%%% Local Variables:
%%% mode: latex
%%% TeX-master: "thesis"
%%% End:


 \part[Concurrency as proof construction]{Concurrency as\\proof construction}\label{part:proof-construction}

\chapter{String rewriting for concurrent specifications}\label{ch:string-rewriting}

In this \lcnamecref{ch:string-rewriting}, we consider abstract rewriting as a framework for specifying the dynamics of concurrent systems.
This is not, of course, a new idea.
Multiset rewriting\autocites{Meseguer:TCS92}{Cervesato+Scedrov:IC09} has previously been put forward as a state-transformation model of concurrency, and has been used to describe security protocols\autocites{Cervesato+:CSFW99}{Durgin+:JCS04}, for example.
Unlike in multiset rewriting, we are particularly interested in concurrent systems whose components are arranged in a linear topology and have a monoidal structure.
Given that finite strings over an alphabet $\ialph$ form a free monoid, string rewriting, rather than multiset rewriting, is a good match for the structure we are interested in.

For a broad sketch of string rewriting, consider the finite strings over the alphabet $\Set{a,b}$, and let $\reduces$ be the least compatible binary relation over those strings that satisfies the axioms
\begin{equation}\label{eq:string-rewriting:ends-b-nondeterministic}
  \infer{a \wc b \reduces b}{}
  \qquad\text{and}\qquad
  \infer{b \reduces \emp}{}
  \:.
\end{equation}
This relation can be seen as a rewriting relation on strings.
For instance, because $a \wc b \wc b \reduces b \wc b$, we would say that $a \wc b \wc b$ may be rewritten to $b \wc b$.

More generally, under the rewriting axioms of \cref{eq:string-rewriting:ends-b-nondeterministic}, a string $w$ can be rewritten to the empty string -- that is, $w \reduces \dotsb \reduces \emp$ -- if, and only if, that string ends with $b$.
For example, the string $abb$ ends with $b$, and $abb$ can indeed be rewritten to the empty string:
\begin{equation*}
  a \wc b \wc b
    \reduces b \wc b
    \reduces b
    \reduces \emp
  \,.
\end{equation*}
In this way, the rewriting axioms of \cref{eq:string-rewriting:ends-b-nondeterministic} constitute a specification of a system that identifies those strings over the alphabet $\Set{a,b}$ that end with $b$.

The usual operational semantics for string rewriting employs committed\-choice nondeterminism, which
% this specification is nondeterministic
% and that that nondeterminism
can lead to stuck, or otherwise undesirable, states.
For example, although $abb$ certainly ends with $b$, the string $abb$ can be rewritten to $a$, a stuck state, if incorrect choices about which axioms to apply are made:
\begin{equation*}
  a \wc b \wc b \reduces a \wc b \reduces a \nreduces
  \,\!.
\end{equation*}
% even though $abb$ certainly ends with $b$.
No backtracking is performed to reconsider these choices.

Disjoint segments of a string may be rewritten independently.
For example, the substring $a \wc b$ can be rewritten to $b$, and the final $b$ of $a \wc b \wc b$ can be rewritten to the empty string.
Being independent, these rewritings can occur in either order, as shown in the adjacent \lcnamecref{fig:string-rewriting:independent1}.
\begin{marginfigure}
  \begin{equation*}
    \hphantom{a \wc b \wc {}}
    \begin{tikzcd}[ampersand replacement=\&, sep=small]
      \& b \wc b \drar[reduces]
      \\
      \mathllap{a \wc b \wc {}} b
        \urar[reduces]
        \drar[reduces]
      \&\& b
      \\
      \& \smash{a \wc b}\vphantom{a} \urar[reduces]
    \end{tikzcd}
  \end{equation*}
  \caption{The interleavings of two independent rewritings}\label{fig:string-rewriting:independent1}
\end{marginfigure}%
Concurrency arises when the various interleavings of independent rewritings are treated indistinguishably.

\newthought{The remainder of this \lcnamecref{ch:string-rewriting}} describes a string rewriting framework in more detail~\parencref{sec:string-rewriting:framework} and examines its properties, most importantly concurrent rewritings.
Then we present two extended examples of how string rewriting may be used to specify concurrent systems: \aclp*{NFA}~\parencref{sec:string-rewriting:nfa} and binary representations of natural numbers~\parencref{sec:string-rewriting:binary-counter}.
These will serve as recurring examples throughout the remainder of this document.

\section{A string rewriting framework}\label{sec:string-rewriting:framework}

In this \lcnamecref{sec:string-rewriting:framework}, we present a string rewriting framework and examine some of its basic properties.

\subsection{Symbols and strings}

String rewriting presupposes an alphabet, $\sralph$, of symbols $a$ from which finite strings are constructed.
This alphabet is usually, but need not be, finite.

Strings, $w$, are then finite lists of symbols: $w = a_1 \wc a_2 \dotsb a_n$.
Algebraically, strings form a free (noncommutative) monoid over symbols $a \in \sralph$ and may be described syntactically by the grammar
\begin{equation*}
  w \Coloneqq w_1 \wc w_2 \mid \emp \mid a
  \,,
\end{equation*}
where the monoid operation is string concatenation, denoted by $w_1 \wc w_2$, and the unit element is the empty string, denoted by $\emp$.%
% and $a$ is a symbol drawn from the alphabet $\ialph$.%
\footnote{Strings are isomorphic to the finite words used by automata~\parencref{ch:automata}, but the two serve different conceptual roles in this dissertation.}

As a monoid, strings are equivalent up to associativity and unit laws (see adjacent \lcnamecref{fig:string-rewriting:monoid-laws}).%
\begin{marginfigure}
  \begin{gather*}
    (w_1 \wc w_2) \wc w_3 = w_1 \wc (w_2 \wc w_3) \\
    \emp \wc w = w = w \wc \emp
  \end{gather*}
  \caption{Monoid laws for strings}\label{fig:string-rewriting:monoid-laws}
\end{marginfigure}
We choose to keep this equivalence implicit, however, treating equivalent strings as syntactically indistinguishable.
As usual for a free monoid, the alternative grammar
  % \begin{equation*}
    $w \Coloneqq \emp \mid a \wc w$
  % \end{equation*}
can be used to   describe the same strings.


% The reason for the change of notation -- from $w$ and $\emp$ at the outset to $\octx$ and $(\emp$ here -- will become clear in the subsequent \lcnamecref{ch:ordered-rewriting}.

\subsection{A rewriting relation}

At the heart of string rewriting is a binary relation, $\reduces$, over strings.
When $w \reduces w'$, we say that $w$ can be rewritten to $w'$.
This relation is defined as the least compatible relation satisfying a collection of rewriting axioms, chosen on a per-application basis%
, such as the axioms
% For example, in the new notation, the axioms of \cref{eq:string-rewriting:ends-b-nondeterministic} are
\begin{equation*}% \label{eq:string-rewriting:ends-b-simpler}
  \infer{a \wc b \reduces b}{}
  \qquad\text{and}\qquad
  \infer{b \reduces \emp}{}
\end{equation*}
shown earlier.
More generally, an axiom is any pair of concrete, finite strings, $w \reduces w'$, although axioms of the form $\emp \reduces w'$ are expressly forbidden.
% Axioms $w \reduces w'$ may only rewrite a \emph{nonempty} concrete string $w$ into another, possibly empty concrete string $w'$.

To be more formal, these axioms are collected into a signature, $\srsig$, that indexes the rewriting relation:
\begin{equation*}
  \srsig \Coloneqq \srsige \mid \srsig, w \reduces w' \quad\text{($w \neq \emp$)}
\end{equation*}
The axioms of this signature may then be used via a $\jrule{$\reduces$AX}$ rule,
\begin{equation*}
  \infer[\mathrlap{\jrule{$\reduces$AX}}]{w \reduces_{\srsig} w'}{
    w \reduces w' \in \srsig}
  \mathrlap{\phantom{\jrule{$\reduces$AX}}\,.}
\end{equation*}
Aside from this rule, all of the other rules for the rewriting relation simply pass on the signature $\srsig$ untouched; for this reason, we nearly always elide the signature index on the rewriting relation, writing $\reduces$ instead of $\reduces_{\srsig}$.
As an example signature, the axioms of our running example can be packaged as
\begin{equation}\label{eq:string-rewriting:example-signature}
  \begin{lgathered}
    \sralph = \Set{ a, b } \\
    \srsig = (a \wc b \reduces b) \,, (b \reduces \emp)
  \,.
  \end{lgathered}
\end{equation}



In addition to the application-specific axioms contained within a signature, rewriting is always permitted within substrings,
so we adopt the rule
\begin{equation*}
  \infer[\jrule{$\reduces$C}]{w_1 \wc w_0 \wc w_2 \reduces w_1 \wc w'_0 \wc w_2}{
    w_0 \reduces w'_0}
\end{equation*}
to ensure that the rewriting relation is compatible with the monoidal structure of strings.

% With these compatibility rules, the axioms of \cref{eq:string-rewriting:ends-b-simpler} can be streamlined: they are derivable from the simpler rules
% \begin{equation}\label{eq:string-rewriting:ends-b-simplest}
%   \infer{a \oc b \reduces b}{}
%   \qquad\text{and}\qquad
%   \infer{b \reduces \octxe}{}
%   \:.
% \end{equation}

The $\reduces$ relation thus describes the rewritings that are possible in a single step: exactly one axiom, perhaps embellished by the compatibility rules.
In addition to these single-step rewritings, it will frequently be useful to describe the rewritings that are possible in some finite number of steps.
For this, we construct a multi-step rewriting relation, $\Reduces$, from the reflexive, transitive closure of $\reduces$.%
\footnote{Usually written as $\reduces^*$, we instead choose $\Reduces$ for the reflexive, transitive closure because of its similarity with standard process calculus notation for weak transitions, $\Reduces[\smash{\raisebox{-0.25ex}{$\scriptstyle\alpha$}}]$.
  Our reasons for this choice of notation will become clearer in subsequent \lcnamecrefs{ch:ordered-rewriting}.}

Consistent with its monoidal structure, there are two equivalent formulations of this reflexive, transitive closure: each rewriting sequence $w \Reduces w'$ can be viewed as either a list or tree of individual rewriting steps.
We prefer the list-based formulation,
\begin{inferences}
  \infer[\jrule{$\Reduces$R}]{w \Reduces w}{}
  \and\text{and}\and
  \infer[\jrule{$\Reduces$T}]{w \Reduces w''}{
    w \reduces w' & w' \Reduces w''}
  \,,
\end{inferences}
because it tends to streamline proofs by structural induction.
However, on the basis of the following \lcnamecref{lem:string-rewriting:trace-transitivity}, we allow ourselves to freely switch between the two formulations as needed.
\begin{lemma}[Transitivity of $\Reduces$]\label{lem:string-rewriting:trace-transitivity}
  If $w \Reduces w'$ and $w' \Reduces w''$, then $w \Reduces w''$.
\end{lemma}
\begin{proof}
  By structural induction over the first of the given rewriting sequences, $w \Reduces w'$.
\end{proof}

% This closure, denoted by $\Reduces$, is the least relation satisfying 
% \footnote{The $\jrule{$\Reduces$T}$ rule could be replaced by
%   \begin{equation*}
%     \infer{\octx \Reduces \octx'}{
%       \octx \reduces \octx'}
%     \qquad\text{and}\qquad
%     \infer{\octx \Reduces \octx''}{
%       \octx \Reduces \octx' & \octx' \Reduces \octx''}
%     \:,
%   \end{equation*}
%   but the $\jrule{$\Reduces$T}$ rule is slightly more convenient to work with in inductive proofs.}

A summary of string rewriting is shown in \cref{fig:string-rewriting:summary}.
\begin{figure}[tbp]
  \vspace*{\dimexpr-\abovedisplayskip-\abovecaptionskip\relax}
  \begin{syntax*}
    Strings &
      w & w_1 \wc w_2 \mid \emp \mid a
    \\
    Signatures &
      \srsig & \srsige \mid \srsig, w \reduces w'
      \mathrlap{\quad\text{($w \neq \emp$)}}
  \end{syntax*}
  \begin{gather*}
    (w_1 \wc w_2) \wc w_3 = w_1 \wc (w_2 \wc w_3) \\
    \emp \wc w = w = w \wc \emp
  \end{gather*}
  \begin{inferences}
    \infer[\jrule{$\reduces$AX}]{w \reduces_{\srsig} w'}{
      w \reduces w' \in \srsig}
    \and
    \infer[\jrule{$\reduces$C}]{w_1 \wc w_0 \wc w_2 \reduces w_1 \wc w'_0 \wc w_2}{
      w_0 \reduces w'_0}
    \\
    \infer[\jrule{$\Reduces$R}]{w \Reduces w}{}
    \and
    \infer[\jrule{$\Reduces$T}]{w \Reduces w''}{
      w \reduces w' & w' \Reduces w''}
  \end{inferences}
  \caption{A string rewriting framework}\label{fig:string-rewriting:summary}
\end{figure}

\subsection{Properties of the string rewriting framework}\label{sec:string-rewriting:concurrency}

As an abstract rewriting system, the above string rewriting framework can be evaluated for several properties: confluence, termination, and, of particular interest to us, concurrency.

\paragraph*{Concurrency}
As an example multi-step rewriting sequence, observe that $a \wc b \wc b \Reduces \emp$, under the axioms of our running example~(\cref{eq:string-rewriting:example-signature}).
In fact, as shown in the adjacent \lcnamecref{fig:string-rewriting:abb}%
\begin{marginfigure}
  \begin{equation*}
    \hphantom{a \wc b \wc {}}
    \begin{tikzcd}%[/tikz/baseline=]
      &
      b \wc b
        \drar[reduces]
      \\
      \mathllap{a \wc b \wc {}} b
        \urar[reduces]
        \drar[reduces]
        \arrow[Reduces]{rr}
      &&
      b \mathrlap{{} \reduces \emp}
      \\
      &
      \smash{a \wc b}\vphantom{a}
        \urar[reduces]
      % \rar[reduces]
      % &
      % a \mathrlap{{} \nreduces}
    \end{tikzcd}
    \hphantom{{} \reduces \emp}
  \end{equation*}
  \caption{An example of concurrent string rewriting}\label{fig:string-rewriting:abb}
\end{marginfigure}%
, multiple sequences witness this rewriting.
The initial $a \wc b$ can first be rewritten to $b$ and then the terminal $b$ can be rewritten to $\emp$ (upper half of \lcnamecref{fig:string-rewriting:abb}); or vice versa: the terminal $b$ can first be rewritten to $\emp$ and then the initial $a \wc b$ can be rewritten to $b$ (lower half of \lcnamecref{fig:string-rewriting:abb}).
In either case, the remaining $b$ (which is the leftmost of the original $b$s) can finally be rewritten to $\emp$.

Notice that these two sequences differ only in how non-overlapping, and therefore independent, rewritings of the string's two segments are interleaved.
Consequently, the two sequences can be -- and indeed should be -- considered essentially equivalent.
The details of how the individual, small steps are interleaved are irrelevant, so that -- conceptually at least -- only the big-step sequence from $a \wc b \wc b$ to $b$ (and ultimately $\emp$) remains (middle of \lcnamecref{fig:string-rewriting:abb}).

In contrast, a third rewriting sequence does not admit this reordering:
the leftmost $b$ is rewritten first to $\emp$ and then the resulting $a \wc b$ is rewritten to $b$ (and ultimately $\emp$).
This sequence's two rewriting steps are not independent because the $b$ that participates in the rewriting of $a \wc b$ is not adjacent to the $a$ until after the first rewriting step occurs.
This is captured in the adjacent \lcnamecref{fig:string-rewriting:abb-complete} by distinguishing the two $b$s with subscripts.
% This sequence's rewriting of $a \wc b$ into $b$ is not independent of the rewriting of the leftmost $b$ because the 
%
\begin{marginfigure}
  \begin{equation*}
    \begin{tikzcd}%[/tikz/baseline=]
      &
      b_1 \wc b_2
        \drar[reduces]
        \rar[reduces]
      &
      b_2
        \drar[reduces]
      &
      \\
      a \wc b_1 \wc b_2
        \urar[reduces]
        \drar[reduces]
        \arrow[Reduces]{rr}
        \arrow[reduces]{ddr}
      &&
      b_1
        \rar[reduces]
      &
      \emp
      \\
      &
      a \wc b_1
        \urar[reduces]
        \rar[reduces]
      &
      a \mathrlap{{} \nreduces}
      \\
      &
      a \wc b_2
        \urar[reduces]
        \arrow[reduces]{uuur}
    \end{tikzcd}
  \end{equation*}
  \caption{When multiple occurences of $b$ are properly distinguished, a complete trace diagram can be given.}\label{fig:string-rewriting:abb-complete}
\end{marginfigure}%

More generally, this idea that the interleaving of independent actions is irrelevant is known as \vocab{concurrent equality}\autocite{Watkins+:CMU02}, and it forms the basis of concurrency\autocite{??}.
With the partial commutativity endowed by concurrent equality, the free monoid formed by rewriting sequences is, more specifically, a trace monoid.
As such, we will frequently refer to rewriting sequences as \vocab{traces}.

\paragraph*{Non-confluence}
We may also evaluate string rewriting for confluence.
Confluence requires that all strings with a common ancestor be joinable, \ie, that $w'_1 \secudeR\Reduces w'_2$ implies $w'_1 \Reduces\secudeR w'_2$, for all strings $w'_1$ and $w'_2$.

Because string rewriting is an asymmetric, committed-choice relation, some nondeterministic choices are irreversible.
For example, under the axioms of our running example (\cref{eq:string-rewriting:example-signature}), $a \wc b$ can be nondeterministically rewritten into either $a$ or $\emp$, as shown in \cref{fig:string-rewriting:abb-complete}%
% \begin{marginfigure}
%   \begin{equation*}
%     \begin{tikzcd}
%       & b \mathrlap{{} \reduces \emp}
%       \\
%       a \wc b
%         \urar[reduces]
%         \drar[reduces]
%       \\
%       & a \mathrlap{{} \nreduces}
%     \end{tikzcd}
%     \hphantom{{} \reduces \emp}
%   \end{equation*}
%   \caption{String rewriting is not confluent.}\label{fig:string-rewriting:ab-not-confluent}
% \end{marginfigure}%
.
However, neither $a$ nor $\emp$ can be rewritten, so confluence fails to hold for string rewriting in general.

\paragraph*{Non-termination}
In our running example, rewriting always terminates: each possible rewriting step removes exactly one symbol, and each string contains only finitely many symbols.

In general, however, string rewriting does not terminate even though strings are finite.
For a simple example, consider rewriting of strings over the alphabet $\Set{a,b}$ with axioms
$a \reduces b$ and $b \reduces a$.
Every finite trace from a nonempty string can always be extended by applying one of these axioms, so string rewriting in this example never terminates.

\section{Extended example: \Aclp*{NFA}}\label{sec:string-rewriting:nfa}

As an extended example of string rewriting, we will specify how \iac{NFA} processes its input.
Beginning with this specification, \acp{NFA} will serve as a recurring example throughout the remainder of this dissertation.
 
Given \iac{NFA} $\aut{A} = (Q, \nfapow, F)$ over an input alphabet $\ialph$, the idea is to introduce a string rewriting axiom for each transition that the \ac{NFA} can make:
\begin{equation*}
  \infer{a \wc q \reduces q'_a}{}
  \enspace\text{for each transition $q \nfareduces[a] q'_a$.}
\end{equation*}
In addition, the \ac{NFA}'s acceptance criteria is captured by introducing a distinguished symbol $\eow$ to act as an end-of-word marker, along with axioms\fixnote{Check these with choreography}
\begin{equation*}
  \infer{\eow \wc q \reduces F(q)}{}
  \enspace\text{for each state $q$, where }
  F(q) = \begin{cases*}
           \symacc & if $q \in F$ \\
           \symrej & if $q \notin F$\,.
         \end{cases*}
\end{equation*}
These axioms imply that rewriting occurs over the finite strings from $\Set{\eow} \times \finwds{\ialph} \times Q \union \Set{ \symacc , \symrej }$.
Expressed as a string rewriting signature, the \ac{NFA} $\aut{A}$ is
\begin{equation*}
  \srsig = \Set{ a \wc q \reduces q'_a \given q \nfareduces[a] q'_a }
             \union \Set{ \eow \wc q \reduces F(q) \given q \in Q }
  \,,
\end{equation*}
where $F(q)$ is defined as above.

For a concrete instance of this encoding, recall from \cref{ch:automata} the \ac{NFA} (repeated in the adjacent \lcnamecref{fig:string-rewriting:nfa-example-ends-b})%
%
\begin{marginfigure}
  \begin{equation*}
    \mathllap{\aut{A}_1 = {}}
    \begin{tikzpicture}[baseline=(q_0.base)]
        \graph [automaton] {
          q_0
           -> [loop above, "a,b"]
          q_0
           -> ["b"]
          q_1 [accepting]
           -> ["a,b"]
          q_2
           -> [loop above, "a,b"]
          q_2;
        };
      \end{tikzpicture}
  \end{equation*}
  \caption{\Iac*{NFA} that accepts, from state $q_0$, exactly those words that end with $b$. (Repeated from \cref{fig:nfa-example-ends-b}.)}\label{fig:string-rewriting:nfa-example-ends-b}
\end{marginfigure}
%
that accepts exactly those words, over the alphabet $\ialph = \set{a,b}$, that end with $b$; that \ac{NFA} is specified by the following string rewriting axioms:
\begin{equation*}
  \srsig_{\aut{A}_1}\!
    = \begin{array}[t]{@{}l@{\,,{}}c@{\,,{}}l@{}}
        (a \wc q_0 \reduces q_0) & (b \wc q_0 \reduces q_0) \,, (b \wc q_0 \reduces q_1) & (\eow \wc q_0 \reduces \symrej) \,, \\
        (a \wc q_1 \reduces q_2) & (b \wc q_1 \reduces q_2) & (\eow \wc q_1 \reduces \symacc) \,, \\
        (a \wc q_2 \reduces q_2) & (b \wc q_2 \reduces q_2) & (\eow \wc q_2 \reduces \symrej) \,.
      \end{array}
\end{equation*}
Indeed, just as the \ac{NFA} $\aut{A}_1$ accepts the input word $abb$, its rewriting specification admits a trace
\begin{equation*}
  \eow \wc b \wc b \wc a \wc q_0
    \reduces \eow \wc b \wc b \wc q_0
    \reduces \eow \wc b \wc q_0
    \reduces \eow \wc q_1
    \reduces \symacc
  \,.
\end{equation*}

More generally, this string rewriting specification of \acp{NFA} adequately describes their operational semantics, in the sense that it simulates all \ac{NFA} transitions.
Given the reversal (anti-)\-homo\-morph\-ism for finite words defined in the adjacent \lcnamecref{fig:string-rewriting:reversal}%
\begin{marginfigure}
  \begin{align*}
    \rev*{w_1 \wc w_2} &= \rev{w_2} \oc \rev{w_1} \\
    \rev{\emp} &= \emp \\
    \rev{a} &= a
  \end{align*}
  \caption{An (anti-)\-homo\-morph\-ism for reversal of finite words}\label{fig:string-rewriting:reversal}
\end{marginfigure}%
, we can prove the following adequacy result.
\begin{restatable}[
  name=Adequacy of \ac*{NFA} specification,
  label=thm:nfa-adequacy-string-rewriting
]{theorem}{nfaadequacystringrewriting}
  Let $\aut{A} = (Q, \nfapow, F)$ be \iac{NFA} over the input alphabet $\ialph$.
  \begin{itemize}[nosep]
  \item
    $q \nfareduces[a] q'_a$ if, and only if, $a \oc q \reduces q'_a$, for all input symbols $a \in \ialph$.
  \item
    $q \in F$ if, and only if, $\eow \oc q \reduces \symacc$.%, where the atom $\emp$ functions as an end-of-word marker.
  \item
    $q \nfareduces[w] q'$ if, and only if, $\rev{w} \oc q \Reduces q'$, for all finite words $w \in \finwds{\ialph}$.
  \end{itemize}  
\end{restatable}
\begin{proof}
  The first two parts follow immediately from the \ac{NFA}'s string rewriting specification; the third part follows by induction over the structure of the input word $w$.
\end{proof}

This adequacy \lcnamecref{thm:nfa-adequacy-string-rewriting} is relatively straightforward to state and prove because string rewriting is a good match for labeled transition systems, like the one that defines \iac{NFA}'s operational semantics.
On the other hand, when a system is not so clearly based on a labeled transition system, stating and proving the adequacy of its string rewriting specification becomes a bit more involved.
This is the case for the next example, binary representations of natural numbers.

\section{Extended example: Binary representations of natural numbers}\label{sec:string-rewriting:binary-counter}

For a second recurring example, we will use
binary representations of natural numbers equipped with increment and decrement operations. % , or \vocab{binary counters}.
% , will serve as a second recurring example throughout the remainder of this document.
Here we present a string rewriting specification of these \emph{binary counters}.
% further example of string rewriting, consider a specification of binary counters: binary representations of natural numbers equipped with increment and decrement operations.

\subsection{Binary representations}

In this setting, we represent a natural number in binary by a string that consists of a big-endian sequence of symbols $b_0$ and $b_1$, prefixed by the symbol $e$; leading $b_0$s are permitted.
For example, both $\octx = e \oc b_1$ and $\octx' = e \oc b_0 \oc b_1$ are valid binary representations of the natural number $1$.%
\footnote{Replace $\octx$ with $w$ or $c$?}

To be more precise, we inductively define a relation, $\aval{}{}$, that assigns to each binary representation a unique natural number denotation.
If $\aval{\octx}{n}$, we say that $\octx$ denotes, or represents, natural number $n$ in binary.
\begin{inferences}
  \infer[\jrule{$e$-V}]{\aval{e}{0}}{}
  \and
  \infer[\jrule{$b_0$-V}]{\aval{\octx \oc b_0}{2n}}{
    \aval{\octx}{n}}
  \and
  \infer[\jrule{$b_1$-V}]{\aval{\octx \oc b_1}{2n+1}}{
    \aval{\octx}{n}}
\end{inferences}
Besides providing a denotational semantics of binary numbers, the $\aval{}{}$ relation also serves to implicitly characterize the well-formed binary numbers as those strings $\octx$ that form the relation's domain of definition.%
\footnote{Alternatively, the well-formed binary numbers could be described more explicitly by the grammar
\begin{equation*}
  \octx \Coloneqq e \mid \octx \oc b_0 \mid \octx \oc b_1
  \,,
\end{equation*}
and then their denotations could be expressed in a more functional manner:
\begin{equation*}
  \begin{lgathered}
    \deno[V]{e} = 0 \\
    \deno[V]{\octx \oc b_0} = 2 \deno[V]{\octx} \\
    \deno[V]{\octx \oc b_1} = 2 \deno[V]{\octx} + 1
    \,.
  \end{lgathered}
\end{equation*}
We prefer the purely relational formulation, however.%
}


The adequacy of the $\aval{}{}$ relation is proved as the following \lcnamecref{thm:ordered-rewriting:binary-adequacy}.
%
\begin{theorem}[Adequacy of binary representations]\label{thm:ordered-rewriting:binary-adequacy}
  Binary representations and their $\aval{}{}$ relation are:
  \begin{thmdescription}
  \item[\emph{Functional}]
    For each binary number $\octx$, there exists a unique natural number $n$ such that $\aval{\octx}{n}$.
  \item[\emph{Surjective}]
    For each natural number $n$, there exists a binary number $\octx$ such that $\aval{\octx}{n}$.
  \item[\emph{Latent}]
    If $\aval{\octx}{n}$, then $\octx \nreduces$.
  \end{thmdescription}
\end{theorem}
\begin{proof}
  The three claims may be proved by induction over the structure of $\octx$, and by induction on $n$, respectively.
\end{proof}

Notice that the above $\jrule{$e$-V}$ and $\jrule{$b_0$-V}$ rules overlap when the denotation is $0$, giving rise to the leading $b_0$s that make the $\aval{}{}$ relation non-injective:
for example, both $\aval{e \oc b_1}{1}$ and $\aval{e \oc b_0 \oc b_1}{1}$ hold.
However, if the $\jrule{$b_0$-V}$ is restricted to \emph{nonzero} even numbers, then each natural number has a unique, canonical representation that is free of leading $b_0$s.%
\footnote{%
  A restriction of the $b_0$ rule to nonzero even numbers is:
  \begin{equation*}
    \infer{\aval{\octx \oc b_0}{2n}}{
      \aval{\octx}{n} & \text{($n > 0$)}}
  \,.
  \end{equation*}
  The leading-$b_0$-free representations could alternatively be seen as the canonical representatives of the equivalence classes induced by the relation among binary numbers that have the same denotation: $\octx \equiv \octx'$ if $\aval{\octx}{n}$ and $\aval{\octx'}{n}$ for some $n$.}


\subsection{An increment operation}

To use string rewriting to describe an increment operation on binary representations, we introduce a new symbol, $i$, that will serve as an increment instruction.

Given a binary number $\octx$ that represents $n$, we may append $i$ to form an active%
\footnote{The \enquote*{active}, \enquote*{latent}, and \enquote*{passive} terminology is borrowed from \textcite{Pfenning+Simmons:LICS09}.
  Active strings are immediately rewritable, but latent strings are rewritable only when combined with other, passive strings.
  The blurry line between latent and passive strings is exploited in \cref{ch:formula-as-process} when we discuss choreographies.}%
, computational string, $\octx \oc i$.
For $i$ to adequately represent the increment operation, the string $\octx \oc i$ must meet two conditions, captured by the following global desiderata:
% \begin{theorem}\label{thm:increment-structural-adequacy}
%   Let $\octx$ be a binary representation of $n$.
%   Then:
  \begin{itemize}% [nosep]
  \item
    % \emph{some} computation from $\octx \oc i$ results in a binary representation of $n+1$ -- that is, $\octx \oc i \Reduces\aval{}{n+1}$; and
    $\octx \oc i \Reduces\aval{}{n+1}$ -- that is, \emph{some} rewriting sequence results in a binary representation of $n+1$; and
  \item
    % \emph{any} computation from $\octx \oc i$ results in a binary representation of $n+1$ -- that is, $\octx \oc i \Reduces\aval{}{n'}$ only if $n' = n+1$.%
    $\octx \oc i \Reduces \octx'$ implies $\octx' \Reduces\aval{}{n+1}$ -- that is, \emph{any} rewriting sequence from $\octx \oc i$ can\fixnote{??} result in a binary representation of $n+1$.
  \end{itemize}
% \end{theorem}
% \noindent
For example, because $e \oc b_1$ denotes $1$, a computation $e \oc b_1 \oc i \Reduces\aval{}{2}$ must exist; moreover, every computation $e \oc b_1 \oc i \Reduces\aval{}{n'}$ must satisfy $n' = 2$.

\newthought{To achieve these} global desiderata, we introduce three string rewriting axioms that describe how the symbols $e$, $b_0$, and $b_1$ may be rewritten when they encounter $i$, the increment instruction:
\begin{inferences}
  \infer{e \oc i \reduces e \oc b_1}{}
  \and
  \infer{b_0 \oc i \reduces b_1}{}
  \and\text{and}\and
  \infer{b_1 \oc i \reduces i \oc b_0}{}
  \,.
\end{inferences}
These three axioms can be read as follows:
\begin{itemize}
\item
  To increment $e$, replace $e$ (and $i$) with $e \oc b_1$.
  % append $b_1$ as a new most\fixnote{or least?} significant bit, resulting in $e \oc b_1$.
\item
  To increment a binary number ending in $b_0$, flip that bit to $b_1$.
\item
  To increment a binary number ending in $b_1$, flip that bit to $b_0$ and carry the increment over to the more significant bits.
\end{itemize}
Comfortingly, $1+1 = 2$: a trace $e \oc b_1 \oc i \reduces e \oc i \oc b_0 \reduces e \oc b_1 \oc b_0$ indeed exists.

Owing to the notion of concurrent equality that string rewriting admits, increments may even be performed concurrently.
For example, there are two rewriting sequences that witness $e \oc b_1 \oc i \oc i \Reduces e \oc b_1 \oc b_1$:
\begin{equation*}
  \hphantom{e \oc b_1 \oc i \oc i \reduces e \oc i \oc b_0 \oc {}}%
  \begin{tikzcd}[
    cells={inner xsep=0.65ex,
           inner ysep=0.4ex},
    row sep=0.2em,
    column sep=scriptsize
  ]
    &[-0.2em]
    e \oc b_1 \oc b_0 \oc i
      \drar[reduces, start anchor=base east]%,
                    % end anchor=north west]
    &[-0.2em]
    \\
    \mathllap{e \oc b_1 \oc i \oc i \reduces e \oc i \oc b_0 \oc {}} i
      \urar[reduces,% start anchor=north east,
                     end anchor=base west]
      \ar[Reduces, gray, dashed, shorten <= .8ex, shorten >= .8ex]{rr}
      \drar[reduces,% start anchor=base east,
                     end anchor=west]
    &&
    e \mathrlap{{} \oc b_1 \oc b_1}
    \\
    &
    e \oc i \oc b_1
      \urar[reduces, start anchor=east]%,
                    % end anchor=base west]
    &
  \end{tikzcd}%
  \hphantom{{} \oc b_1 \oc b_1}
\end{equation*}
In other words, once the left most increment is carried past the least significant bit, the two increments can be interleaved, with no observable difference in the outcome.


\newthought{These increment axioms} introduce strings that occur as intermediate computational states within traces, such as $e \oc i \oc b_0 \oc i$ and $e \oc i \oc b_1$ in the above diagram.
To characterize the valid intermediate strings, we define a binary relation, $\ainc{}{}$, that assigns a natural number denotation to each such intermediate string, not only to the terminal values, as $\aval{}{}$ did.%
\footnote{Like the $\aval{}{}$ relation does for values, the $\ainc{}{}$ relation also serves to implicitly characterize the valid intermediate states as those contexts that form the relation's domain of definition.
As with values, the valid intermediate states could also be enumerated more explicitly and syntactically with a grammar and denotation function:
\begin{equation*}
  \octx \Coloneqq e \mid \octx \oc b_0 \mid \octx \oc b_1 \mid \octx \oc i
\end{equation*}
\begin{equation*}
  \begin{lgathered}
    \deno[I]{e} = 0 \\
    \deno[I]{\octx \oc b_0} = 2 \deno[I]{\octx} \\
    \deno[I]{\octx \oc b_1} = 2 \deno[I]{\octx} + 1 \\
    \deno[I]{\octx \oc i} = \deno[I]{\octx} + 1
  \end{lgathered}
\end{equation*}
However, we once again prefer the purely relational form.}%
%
\begin{inferences}
  \infer[\jrule{$e$-I}]{\ainc{e}{0}}{}
  \and
  \infer[\jrule{$b_0$-I}]{\ainc{\octx \oc b_0}{2n}}{
    \ainc{\octx}{n}}
  \and
  \infer[\jrule{$b_1$-I}]{\ainc{\octx \oc b_1}{2n+1}}{
    \ainc{\octx}{n}}
  \and
  \infer[\jrule{$i$-I}]{\ainc{\octx \oc i}{n+1}}{
    \ainc{\octx}{n}}
\end{inferences}
Binary values should themselves be valid, terminal computational states, so the first three rules are carried over from the $\aval{}{}$ relation.
The $\jrule{$i$-I}$ rule allows multiple increment instructions to be interspersed throughout the state.

With this $\ainc{}{}$ relation in hand, we can now prove a stronger, small-step adequacy theorem.
This small-step \lcnamecref{thm:string-rewriting:inc-small-step-adequacy} then implies the big-step desiderata from above.
%
\begin{theorem}[Small-step adequacy of increments]\label{thm:string-rewriting:inc-small-step-adequacy}
  \leavevmode
  \begin{thinthmdescription}
  \item[Value soundness]
    If $\aval{\octx}{n}$, then $\ainc{\octx}{n}$ and $\octx \nreduces$.
  \item[Preservation]
    If $\ainc{\octx}{n}$ and $\octx \reduces \octx'$, then $\ainc{\octx'}{n}$.
  \item[Progress]
    If $\ainc{\octx}{n}$, then either
    \begin{itemize*}[
      % mode=unboxed,
      label=, afterlabel=,
      before=\unskip:\space,
      itemjoin=;\space, itemjoin*=; or\space%
    ]
    \item $\octx \reduces \octx'$ for some $\octx'$
    \item $\aval{\octx}{n}$% \fixnote{Compare with \enquote{If $\ainc{\octx}{n}$, then $\aval{\octx}{n}$ if, and only if, $\octx \nreduces$.}}
    \end{itemize*}.
  \item[Termination]
    If $\ainc{\octx}{n}$, then every rewriting sequence from $\octx$ is finite.
  \end{thinthmdescription}
\end{theorem}
%
\begin{proof}
  Each part is proved separately.
  \begin{description}[
    parsep=0pt, listparindent=\parindent,
    labelsep=0.35em
  ]
  \item[Value soundness]
    can be proved by structural induction on the derivation of $\aval{\octx}{n}$.
  \item[Preservation and progress]
    can likewise be proved by structural induction on the derivation of $\ainc{\octx}{n}$.
  \item[Termination]
    can be proved using an explicit termination measure, $\card[i]{}$, that is strictly decreasing across each rewriting, $\octx \reduces \octx'$.
    Specifically, we use a measure (see the adjacent \lcnamecref{fig:string-rewriting:binary-counter:measure}),
    % For valid states $\octx$, we define a measure $\card{\octx}$ that is strictly decreasing across each rewriting $\octx \reduces \octx'$ (see the adjacent \lcnamecref{fig:ordered-rewriting:binary-counter:measure}).
    \begin{marginfigure}
      \begin{equation*}
        \begin{lgathered}
          \card[i]{e} = 0 \\
          \card[i]{\octx \oc b_0} = \card[i]{\octx} \\
          \card[i]{\octx \oc b_1} = \card[i]{\octx} + 1 \\
          \card[i]{\octx \oc i} = \card[i]{\octx} + 2
          % e + i > e + b1  =>  e >= 0
          % b0 + i > b1
          % b1 + i > i + b0 > b1  =>  i > 1
        \end{lgathered}
      \end{equation*}
      \caption{A termination measure, adapted from the standard amortized work analysis of increment for binary counters}\label{fig:string-rewriting:binary-counter:measure}
    \end{marginfigure}%
    adapted from the standard amortized constant work analysis of increment for binary counters\autocite{??}.
    The measure $\card[i]{}$ is such that $\octx \reduces \octx'$ implies $\card[i]{\octx} > \card[i]{\octx'}$;
    % That is, if $\octx$ is a valid state and $\octx \reduces \octx'$, then $\card{\octx} > \card{\octx'}$.
    because the measure is always nonnegative, only finitely many such rewritings can occur.

    As an example case, consider the intermediate state $\octx \oc b_1 \oc i$ and its rewriting $\octx \oc b_1 \oc i \reduces \octx \oc i \oc b_0$.
    Indeed, $\card[i]{\octx \oc b_1 \oc i} = \card[i]{\octx} + 3 > \card[i]{\octx} + 2 = \card[i]{\octx \oc i \oc b_0}$.
  \qedhere
  \end{description}
\end{proof}

\begin{corollary}[Big-step adequacy of increments]
  \leavevmode
  \begin{thinthmdescription}
  \item[Evaluation]
    If $\ainc{\octx}{n}$, then $\octx \Reduces\aval{}{n}$.
    In particular, if $\aval{\octx}{n}$, then $\octx \oc i \Reduces\aval{}{n+1}$.
  \item[Preservation]
    If $\ainc{\octx}{n}$ and $\octx \Reduces \octx'$, then $\ainc{\octx'}{n}$.
    In particular, if $\aval{\octx}{n}$ and $\octx \oc i \Reduces \octx'$, then $\octx' \Reduces\aval{}{n+1}$.
  \end{thinthmdescription}
\end{corollary}
\begin{proof}
  The two parts are proved separately.
  \begin{description}[labelsep=0.35em]
  \item[Evaluation] can be proved by repeatedly appealing to the progress and preservation results~\parencref{thm:string-rewriting:inc-small-step-adequacy}.
    By the accompanying termination result, a binary value must eventually be reached.
  \item[Preservation] can be proved by structural induction on the given trace.
  %
  \qedhere
  \end{description}
\end{proof}


% \clearpage
\subsection{A decrement operation}

Binary counters may also be equipped with a decrement operation.
Instead of examining decrements \emph{per se}, we will describe a very closely related operation: the normalization of binary representations to what might be called \vocab{head-unary form}.
(We will frequently abuse terminology, using \enquote*{head-unary normalization} and \enquote*{decrement operation} interchangably.)
A string $\octx$ will be said to be in head-unary form if it has one of two forms: $\octx = z$; or $\octx = \octx' \oc s$, for some binary number $\octx'$.

Just as appending the symbol $i$ to a counter $\octx$ initiates an increment, appending a symbol $d$ will cause the counter to begin normalizing to head-unary form.
For $d$ to adequately represent this operation, the string $\octx \oc d$ must satisfy the following global desiderata when $\adec{\octx}{n}$:
% The following \lcnamecref{thm:decrement-adequacy} serves as a specification of head-unary normalization, relating a value's head-unary form to its denotation.
%
% \begin{theorem}[Structural adequacy of decrements]
%   If $\aval{\octx}{n}$, then:
  \begin{itemize}% [nosep]
  \item $\octx \oc d \Reduces z$ if, and only if, $n=0$;
  \item $\octx \oc d \Reduces \octx' \oc s$ for some $\octx'$ such that $\aval{\octx'}{n-1}$, if $n > 0$; and
  \item $\octx \oc d \Reduces \octx' \oc s$ only if $n > 0$ and $\aval{\octx'}{n-1}$.
  \end{itemize}
% \end{theorem}
%
% \noindent
For example, because $e \oc b_1$ denotes $1$, a trace $e \oc b_1 \oc d \Reduces \octx' \oc s$ must exist, for some $\aval{\octx'}{0}$.

\newthought{To achieve these} global desiderata, we introduce three additional axioms that describe how the symbols $e$, $b_0$, and $b_1$ may be rewritten when they encounter $d$, the decrement instruction;
also, an intermediate symbol $b'_0$ and two more axioms are introduced:
\begin{inferences}
  \infer{e \oc d \reduces z}{}
  \and
  \infer{b_1 \oc d \reduces b_0 \oc s}{}
  \and
  \infer{b_0 \oc d \reduces d \oc b'_0}{}
  \\
  \infer{z \oc b'_0 \reduces z}{}
  \and\text{and}\and
  \infer{s \oc b'_0 \reduces b_1 \oc s}{}
  \,.
\end{inferences}
These five axioms can be read as follows:
\begin{itemize}
\item
  Because $e$ denotes $0$, its head-unary form is simply $z$.
\item
  Because $\octx \oc b_1$ denotes $2n+1$ if $\octx$ denotes $n$, its head-unary form, $\octx \oc b_0 \oc s$, can be constructed by flipping the least significant bit to $b_0$ and appending $s$.
\item
  Because $\octx \oc b_0$ denotes $2n$ if $\octx$ denotes $n$, its head-unary form can be contructed by recursively putting the more significant bits, $\octx$, into head-unary form and appending $b'_0$ to process that result.
  \begin{itemize}
  \item
    If $\octx$ has head-unary form $z$ and therefore denotes $0$, then $\octx \oc b_0$ also denotes $0$ and has head-unary form $z$.
  \item
    Otherwise, if $\octx$ has head-unary form $\octx' \oc s$ and thus denotes $n > 0$, then $\octx \oc b_0$ denotes $2n > 0$ and has head-unary form $\octx' \oc b_1 \oc s$, which can be constructed by replacing $s$ with $b_1 \oc s$.
  \end{itemize}
\end{itemize}
%
Comfortingly, $(1+1)-1 = 1$: the head-unary form of $e \oc b_1 \oc i$ is $e \oc b_0 \oc b_1 \oc s$:
\begin{equation*}
  \hphantom{e \oc b_1 \oc i \oc d \reduces e \oc i \oc b_0 \oc {}}%
  \begin{tikzcd}[
    cells={inner xsep=0.65ex,
           inner ysep=0.4ex},
    row sep=0em,
    column sep=scriptsize
  ]
    &[-0.2em]
    e \oc b_1 \oc b_0 \oc d
      \drar[reduces, start anchor=base east]%,
                    % end anchor=north west]
    &[-0.2em]
    \\
    \mathllap{e \oc b_1 \oc i \oc d \reduces e \oc i \oc b_0 \oc {}} d
      \urar[reduces,% start anchor=north east,
                     end anchor=base west]
      \ar[Reduces, gray, dashed, shorten <=.8ex, shorten >=.8ex]{rr}
      \drar[reduces,% start anchor=base east,
                     end anchor=west]
    &&
    e \mathrlap{{} \oc b_1 \oc d \oc b'_0
      \reduces e \oc b_0 \oc s \oc b'_0
      \reduces e \oc b_0 \oc b_1 \oc s}
    \\
    &
    e \oc i \oc d \oc b'_0
      \urar[reduces, start anchor=east]%,
                     % end anchor=base west]
    &
  \end{tikzcd}
 \hphantom{{} \oc b_1 \oc d \oc b'_0
      \reduces e \oc b_0 \oc s \oc b'_0
      \reduces e \oc b_0 \oc b_1 \oc s}
  .
\end{equation*}
Note the concurrency that derives from the independence of the increment and decrement after the initial step of rewriting.


\newthought{These decrement axioms} introduce more strings that may occur as intermediate computational states.
As before, we define a new binary relation, $\adec{}{}$, that assigns a natural number denotation to each string that may appear as an intermediate state during a decrement.
\begin{inferences}
  \infer[\jrule{$d$-D}]{\adec{\octx \oc d}{n}}{
    \ainc{\octx}{n}}
  \and
  \infer[\jrule{$b'_0$-D}]{\adec{\octx \oc b'_0}{2n}}{
    \adec{\octx}{n}}
  \and
  \infer[\jrule{$z$-D}]{\adec{z}{0}}{}
  \and
  \infer[\jrule{$s$-D}]{\adec{\octx \oc s}{n+1}}{
    \ainc{\octx}{n}}
\end{inferences}
At first glance, the $\jrule{$d$-D}$ rule may look a bit odd:
Why is the denotation unchanged by a decrement, $\octx \oc d$?
Because the operation is more accurately characterized as head-unary normalization, it makes sense that the denotation remains unchanged.
The operation described by $d$ does not change the binary counter's value -- it only expresses that same value in a different form.%
\footnote{Once again, the valid intermediate states could also be enumerated more explicitly and syntactically with a grammar and denotation function:%
\begin{align*}
  \octx &\Coloneqq e \mid \octx \oc b_0 \mid \octx \oc b_1 \mid \octx \oc i \\
  \lctx &\Coloneqq \octx \oc d \mid \lctx \oc b'_0 \mid z \mid \octx \oc s
\end{align*}
\begin{equation*}
  \begin{lgathered}
    \deno[D]{\octx \oc d} = \deno[I]{\octx} \\
    \deno[D]{\lctx \oc b'_0} = 2 \deno[D]{\lctx} \\
    \deno[D]{z} = 0 \\
    \deno[D]{\octx \oc s} = \deno[I]{\octx} + 1
  \end{lgathered}
\end{equation*}}%

Also, notice that the premises of the $\jrule{$d$-D}$ and $\jrule{$s$-D}$ rules use the increment-only denotation relation, $\ainc{}{}$, not the decrement relation, $\adec{}{}$.
These choices ensure that each counter has at most one $d$ and may not have any $i$ or $s$ symbols to the right of that $d$.
But the premise of the $\jrule{$b'_0$-D}$ does use the $\adec{}{}$ relation, so $d$ may have $b'_0$ symbols to its right.


% e + i > e + b1
% e + d > z
% b0 + i > b1
% b0 + d > d + b0'
% b1 + i > i + b0
% b1 + d > b0 + s
% z + b0' > z
% s + b0' > b1 + s

% |O d| = |O| + 1
% |O b0'| = |O| + 1
% |z| = 0
% |O s| = 0
% |e| = 0
% |O b0| = |O| + 2
% |O b1| = |O| + 3
% |O i| = |O| + 4

% |O d|d = |O|i + 1
% |O b0'|d = |O|d + 1
% |z|d = 0
% |O s|d = 0
%
% |O|i = 2L(O) + |O|

% |O d|d = |O| + 3L(O)
% |O b0'|d = |O|d + 2
% |z|d = 0
% |O s|d = |O|

% 3 > 0
% |O| + 3L(O) + 3 > |O| + 3L(O) + 2
% |O| + 1 + 3L(O) + 3 > |O|
% 2 > 0
% |O| + 2 > |O| + 1


% 4 > 3
% 1 > 0
% 6 > 3
% 3 > 2
% 7 > 6
% 5 > 0
% 1 > 0
% 1 > 0

% e + d > z
% b0 + d > d + b0'
% b1 + d > s
% b0' > 0
% b0' > 0

% b0' = 1

With this $\adec{}{}$ relation in hand, we can now prove a small-step adequacy \lcnamecref{thm:string-rewriting:dec-small-step-adequacy}.
This small-step \lcnamecref{thm:string-rewriting:dec-small-step-adequacy} then implies the big-step desiderata from above.
%
\begin{restatable}[
  name=Small-step adequacy of decrements,
  label=thm:string-rewriting:dec-small-step-adequacy
]{theorem}{thmadequacysmalldecstring}
  \leavevmode
  \begin{thinthmdescription}[leftmargin=0em]
  \item[Preservation]
    If $\adec{\octx}{n}$ and $\octx \reduces \octx'$, then $\adec{\octx'}{n}$.
  \item[Progress]
    If $\adec{\octx}{n}$, then either:
    \begin{itemize}[nosep]
    \item $\octx \reduces \octx'$, for some $\octx'$;
    \item $n = 0$ and $\octx = z$; or
    \item $n > 0$ and $\octx = \octx' \oc s$, for some $\octx'$ such that $\ainc{\octx'}{n-1}$.
    \end{itemize}
  \item[Termination]
    If $\adec{\octx}{n}$, then every rewriting sequence from $\octx$ is finite.
  \end{thinthmdescription}
\end{restatable}
\begin{proof}
  Each part is proved separately.
  \begin{description}[
    parsep=0pt, listparindent=\parindent,
    labelsep=0.35em
  ]
  \item[Preservation and progress]
    are proved, as before, by structural induction on the given derivation of $\adec{\octx}{n}$.
  \item[Termination] is proved by exhibiting a measure, $\card[d]{}$, given in the adjacent \lcnamecref{fig:string-rewriting:dec-measure}%
    \begin{marginfigure}
      \begin{equation*}
        \begin{lgathered}
          \card[d]{\octx \oc d} = \card[i]{\octx} + 3 \card{\octx} \\
          \card[d]{\octx \oc b'_0} = \card[d]{\octx} + 2 \\
          \card[d]{z} = 0 \\
          \card[d]{\octx \oc s} = \card[i]{\octx}
        \end{lgathered}
      \end{equation*}
      \caption{A termination measure for decrements, where $\card{\octx}$ denotes the length of string $\octx$}\label{fig:string-rewriting:dec-measure}
    \end{marginfigure}%
, that is strictly decreasing across each rewriting.
    Unlike the amortized constant work increments~\parencref[see proof of]{thm:string-rewriting:inc-small-step-adequacy}, this measure assigns a linear amount of potential to the decrement instruction.%
    \footnote{Actually, because the increment and decrement operations are defined only for binary representations, not head-unary forms, there can be at most one $d$.
      Therefore, it is actually possible to assign a constant amount of potential to each $d$.
      However, doing so would rely on a somewhat involved lexicographic measure that isn't particularly relevant to our aims in this dissertation, so we use the simpler linear potential.}%

    This measure is strictly decreasing across each rewriting: $\octx \reduces \octx'$ only if $\card[d]{\octx} > \card[d]{\octx'}$.
    As an example case, consider the intermediate state $\octx \oc b_0 \oc d$ and its rewriting $\octx \oc b_0 \oc d \reduces \octx \oc d \oc b'_0$.
    Indeed,
    \begin{equation*}
      \card[d]{\octx \oc b_0 \oc d}
        % = \card[i]{\octx \oc b_0} + 3\card{\octx \oc b_0}
        = \card[i]{\octx} + 3\card{\octx} + 3
        > \card[i]{\octx} + 3\card{\octx} + 2
        % = \card[d]{\octx \oc d} + 2
        = \card[d]{\octx \oc d \oc b'_0}
      \,.
    \qedhere
    \end{equation*}
  \end{description}
\end{proof}

\begin{restatable}[
  name=Big-step adequacy of decrements,
  label=cor:string-rewriting:dec-big-step-adequacy
]{corollary}{coradequacydecstring}
  If $\adec{\octx}{n}$, then:
  \begin{itemize}[nosep]
  \item $\octx \Reduces z$ if, and only if, $n = 0$;
  \item $\octx \Reduces \octx' \oc s$ for some $\octx'$ such that $\ainc{\octx'}{n-1}$, if $n > 0$; and
  \item $\octx \Reduces \octx' \oc s$ only if $n > 0$ and $\ainc{\octx'}{n-1}$.
  \end{itemize}
\end{restatable}
\begin{proof}
  From the small-step preservation result of \cref{thm:string-rewriting:dec-small-step-adequacy}, it is possible to prove, using a structural induction on the given trace, a big-step preservation result: namely, that $\adec{\octx}{n}$ and $\octx \Reduces \octx'$ only if $\adec{\octx'}{n}$.
  Each of the above claims then follows from either progress and termination~\parencref{thm:string-rewriting:dec-small-step-adequacy} or big-step preservation together with inversion.
\end{proof}

% $\adec{z}{n}$
% $\adec{\octx' \oc s}{n}$  ==>  $\ainc{\octx'}{n'}$ and $n = n'+1$
% 

%%% Local Variables:
%%% mode: latex
%%% TeX-master: "thesis"
%%% End:

\chapter{Ordered rewriting}\label{ch:ordered-rewriting}

In this \lcnamecref{ch:ordered-rewriting}, we develop a rewriting interpretation of the ordered sequent calculus from the previous \lcnamecref{ch:ordered-logic}.

In \citeyear{Lambek:AMM58}, \citeauthor{Lambek:AMM58} developed a syntactic calculus, now known as the Lambek calculus, for formally describing the structure of sentences.\autocite{Lambek:AMM58}
Words are assigned syntactic types, which roughly correspond to grammatical parts of speech.
From a logical perspective, the Lambek calculus can [also] be viewed as a precursor to (and generalization of) \citeauthor{Girard:TCS87}'s linear logic\autocite{Girard:TCS87}\relax.\autocites{Polakow+Pfenning:MFPS99}{Polakow+Pfenning:TLCA99}
Implicit in \citeauthor{Lambek:AMM58}'s original article is a third perspective of the calculus: string rewriting.

In this \lcnamecref{ch:ordered-rewriting}, we review the Lambek calculus from a [string] rewriting perspective.

\newthought{The previous \lcnamecref{ch:string-rewriting}} showed how to use string rewriting to specify, on a global level, the [...] of concurrent systems that have a linear topology.
Although useful for [...], these string rewriting specifications lack a clear notion of local, decentralized execution -- for each step of rewriting, the entire string is rewritten as a monolithic whole by a central conductor.

Keeping in mind our ultimate goal of decentralized\fixnote{distributed?} implementations of concurrent systems, these string rewriting specifications are too abstract.
Instead, we need to expose local interactions that are left implicit in the string rewriting specifications.

As an example, recall from \cref{ch:string-rewriting} the string rewriting specification of a system that may transform strings that end with $b$ into the empty string:
\begin{equation}
  \infer{a \oc b \reduces b}{}
  \qquad
  \infer{b \reduces \octxe}{}
  \:.
\end{equation}
This specification is non-local in two ways:
the central conductor must identify those substrings that can be rewritten according to one of the axioms.
In the [...] axiom, for example, there is no description of how the symbols $a$ and $b$ would identify each other and coordinate to effect a rewriting to $b$.

To [...], we introduce \vocab{choreographies}, which refine string rewriting specifications by consistently assigning each symbol one of two roles: message or process.
\begin{equation*}
  \infer{\atmR{a} \oc \proc{b} \reduces \proc{b}}{}
  \qquad\text{and}\qquad
  \infer{\proc{b} \reduces \octxe}{}
\end{equation*}

a recursively defined ordered proposition, such as
\begin{equation*}
  \proc{b} \defd (\atmR{a} \limp \up \dn \proc{b}) \with \one
\end{equation*}
for the process $\proc{b}$.

The remainder of this \lcnamecref{ch:ordered-rewriting} presents a formulation of the Lambek calculus from the ordered sequent calculus of \cref{ch:ordered-logic}.

Then, in 


One valid choreography for this specification views each symbol $b$ as a process that nondeterministically receives some number of messages $a$ before terminating.

If we annotate messages with an underbar and processes with a circumflex, then $\atm{a} \oc \hat{b} \reduces \hat{b}$ and $\hat{b} \reduces \octxe$.


\section{Ordered resource decomposition as rewriting}

\subsection{Most left rules decompose ordered resources}

Recall two of the ordered sequent calculus's left rules:
\begin{inferences}
  \infer[\lrule{\fuse}]{\oseq{\octx'_L \oc (A \fuse B) \oc \octx'_R |- C}}{
    \oseq{\octx'_L \oc A \oc B \oc \octx'_R |- C}}
  \and\text{and}\and
  \infer[\lrule{\with}_1]{\oseq{\octx'_L \oc (A \with B) \oc \octx'_R |- C}}{
    \oseq{\octx'_L \oc A \oc \octx'_R |- C}}
  \,.
\end{inferences}
Both rules decompose the principal resource: in the $\lrule{\fuse}$ rule, $A \fuse B$ into the separate resources $A \oc B$; and, in the $\lrule{\with}_1$ rule, $A \with B$ into $A$.
However, in both cases, the resource decomposition is somewhat obscured by boilerplate.
The framed contexts $\octx'_L$ and $\octx'_R$ and goal $C$ serve to enable the rules to be applied anywhere in the list of resources, without restriction;
these concerns are not specific to the $\lrule{\fuse}$ and $\lrule{\with}_1$ rules, but are general boilerplate that arguably should be factored out.

To decouple the resource decomposition from the surrounding boilerplate, we will introduce a new judgment, $\octx \reduces \octx'$, meaning \enquote{Resources $\octx$ may be decomposed into resources $\octx'$.}
The choice of notation for this judgment is not coincidental:
resource decomposition is a generalization of the string rewriting shown in \cref{ch:string-rewriting}.

% With this judgment in hand, the boilerplate can be factored into a uniform left rule, $\lrule{\star}$:
With this new decomposition judgment comes a cut principle, $\jrule{CUT}^{\reduces}$, into which all of the boilerplate is factored:
\begin{equation*}
  \infer[\jrule{CUT}\smash{^{\reduces}}]{\oseq{\octx'_L \oc \octx \oc \octx'_R |- C}}{
    \octx \reduces \octx' &
    \oseq{\octx'_L \oc \octx' \oc \octx'_R |- C}}
  .
\end{equation*}
The standard left rules can then be recovered from resource decomposition rules using this cut principle.
For example, the decomposition of $A \fuse B$ into $A \oc B$ is captured by
\begin{equation*}
  \infer[\jrule{$\fuse$D}]{A \fuse B \reduces A \oc B}{}
  \,,
\end{equation*}
and the standard $\lrule{\fuse}$ rule can then be recovered as shown in the adjacent \lcnamecref{fig:ordered-rewriting:fuse-refactoring}.%
\begin{marginfigure}
  \begin{equation*}
    \begin{gathered}[t]
      \infer[\mathrlap{\lrule{\fuse}}]{\oseq{\octx'_L \oc (A \fuse B) \oc \octx'_R |- C}}{
        \oseq{\octx'_L \oc A \oc B \oc \octx'_R |- C}}
      % 
      \\\leftrightsquigarrow\\
      % 
      \infer[\mathrlap{\jrule{CUT}\smash{^{\reduces}}}]{\oseq{\octx'_L \oc (A \fuse B) \oc \octx'_R |- C}}{
        \infer[\jrule{$\fuse$D}]{A \fuse B \reduces A \oc B}{} &
        \oseq{\octx'_L \oc A \oc B \oc \octx'_R |- C}}
    \end{gathered}
    \phantom{\jrule{CUT}\smash{^{\reduces}}}
  \end{equation*}
  \caption{Refactoring the $\lrule{\fuse}$ rule in terms of resource decomposition}\label{fig:ordered-rewriting:fuse-refactoring}
\end{marginfigure}
The left rules for $\one$ and $A \with B$ can be similarly refactored into the resource decomposition rules
\begin{inferences}
  \infer[\jrule{$\one$D}]{\one \reduces \octxe}{}
  \and
  \infer[\jrule{$\with$D}_1]{A \with B \reduces A}{}
  \and\text{and}\and
  \infer[\jrule{$\with$D}_2]{A \with B \reduces B}{}
  \,.
\end{inferences}

Even the left rules for left- and right-handed implications can be refactored in this way, despite the additional, minor premises that those rules carry.
To keep the correspondence between resource decomposition rules and left rules as close as possible, we could introduce the decomposition rules
\begin{equation}\label{eq:ordered-rewriting:limp-pmir-decomposition}
  \infer[\jrule{$\limp$D}']{\octx \oc (A \limp B) \reduces B}{
    \oseq{\octx |- A}}
  \qquad\text{and}\qquad
  \infer[\jrule{$\pmir$D}']{(B \pmir A) \oc \octx \reduces B}{
    \oseq{\octx |- A}}
  \,.
\end{equation}
Just as for ordered conjunction, the left rules for left- and right-handed implication would then be recoverable via the $\jrule{CUT}^{\reduces}$ rule~(see adjacent \lcnamecref{fig:ordered-rewriting:limp-refactoring-1}).%
\begin{marginfigure}
  \begin{equation*}
    \begin{gathered}
      \infer[\mathrlap{\lrule{\limp}}]{\oseq{\octx'_L \oc \octx \oc (A \limp B) \oc \octx'_R |- C}}{
        \oseq{\octx |- A} &
        \oseq{\octx'_L \oc B \oc \octx'_R |- C}}
      % 
      \\\leftrightsquigarrow\\
      % 
      \infer[\mathrlap{\jrule{CUT}\smash{^{\reduces}}}]{\oseq{\octx'_L \oc \octx \oc (A \limp B) \oc \octx'_R |- C}}{
        \infer[\jrule{$\limp$D}']{\octx \oc (A \limp B) \reduces B}{
          \oseq{\octx |- A}} &
        \oseq{\octx'_L \oc B \oc \octx'_R |- C}}
    \end{gathered}
    \phantom{\jrule{CUT}\smash{^{\reduces}}}
  \end{equation*}
  \caption{A possible refactoring of the $\lrule{\limp}$ rule in terms of resource decomposition}\label{fig:ordered-rewriting:limp-refactoring-1}
\end{marginfigure}

Although these
% $\jrule{$\limp$D}'$ and $\jrule{$\pmir$D}'$
rules keep the correspondence between resource decomposition rules and left rules close, they differ from the other decomposition rules in two significant ways.
First, the above $\jrule{$\limp$D}'$ and $\jrule{$\pmir$D}'$ rules have premises, and those premises create a dependence of the decomposition judgment upon general provability.
Second, the above $\jrule{$\limp$D}'$ and $\jrule{$\pmir$D}'$ rules do not decompose the principal proposition into \emph{immediate} subformulas since $\octx$ is involved.
This contrasts with, for example, the $\jrule{$\fuse$D}$ rule that decomposes $A \fuse B$ into the immediate subformulas $A \oc B$.

For these reasons, the above $\jrule{$\limp$D}'$ and $\jrule{$\pmir$D}'$ rules are somewhat undesirable.
Fortunately, there is an alternative.
Filling in the $\oseq{\octx |- A}$ premises with the $\jrule{ID}^A$ rule, we arrive at the derivable rules
\begin{equation}\label{eq:ordered-rewriting:limp-pmir-decomposition-nullary}
  \infer[\jrule{$\limp$D}]{A \oc (A \limp B) \reduces B}{}
  \qquad\text{and}\qquad
  \infer[\jrule{$\pmir$D}]{(B \pmir A) \oc A \reduces B}{}
  \,,
\end{equation}
which we adopt as decomposition rules in place of those in \cref{eq:ordered-rewriting:limp-pmir-decomposition}.
The standard $\lrule{\limp}$ and $\lrule{\pmir}$ rules can still be recovered from these more specific decomposition rules, thanks to $\jrule{CUT}$ (see adjacent \lcnamecref{fig:ordered-rewriting:limp-refactoring-2}).%
\begin{marginfigure}[-10\baselineskip]
  \begin{gather*}
    \infer[\lrule{\limp}]{\oseq{\octx'_L \oc \octx \oc (A \limp B) \oc \octx'_R |- C}}{
      \oseq{\octx |- A} &
      \oseq{\octx'_L \oc B \oc \octx'_R |- C}}
    %
    \\\leftrightsquigarrow\\
    %
    \infer[\jrule{CUT}\smash{^A}]{\oseq{\octx'_L \oc \octx \oc (A \limp B) \oc \octx'_R |- C}}{
      \oseq{\octx |- A} &
      \infer[\jrule{CUT}\smash{^{\reduces}}]{\oseq{\octx'_L \oc A \oc (A \limp B) \oc \octx'_R |- C}}{
        \infer[\jrule{$\limp$D}]{A \oc (A \limp B) \reduces B}{} &
        \oseq{\octx'_L \oc B \oc \octx'_R |- C}}}
  \end{gather*}
  \caption{Refactoring the $\lrule{\limp}$ rule in terms of resource decomposition, via $\jrule{$\limp$D}$ and $\jrule{CUT}\smash{^{\reduces}}$}\label{fig:ordered-rewriting:limp-refactoring-2}
\end{marginfigure}
These revised, nullary decomposition rules correct the earlier drawbacks: like the other decomposition rules, they now have no premises and only refer to immediate subformulas.
Moreover, these rules have the advantage of matching two of the axioms from \citeauthor{Lambek:AMM58}'s original article.\autocite{Lambek:AMM58}

\newthought{%
For most
ordered logical connectives}, this approach works perfectly.
Unfortunately, the left rules for additive disjunction, $A \plus B$, and its unit, $\zero$, are resistant to this kind of refactoring.
The difficulty with additive disjunction isn't that its left rule, $\lrule{\plus}$,%
\marginnote{%
  \begin{equation*}
    \infer[\lrule{\plus}]{\oseq{\octx'_L \oc (A \plus B) \oc \octx'_R |- C}}{
      \oseq{\octx'_L \oc A \oc \octx'_R |- C} &
      \oseq{\octx'_L \oc B \oc \octx'_R |- C}}
  \end{equation*} 
}
doesn't decompose the resource $A \plus B$.
The $\lrule{\plus}$ rule certainly does decompose $A \plus B$, but it does so [...].\fixnote{fix}
$A \plus B \reduces A \mid B$
[...] retain the standard $\lrule{\plus}$ and $\lrule{\zero}$ rules.

\begin{figure}[tbp]
  \vspace*{\dimexpr-\abovedisplayskip-\abovecaptionskip\relax}
  \begin{inferences}
    \infer[\jrule{CUT}\smash{^A}]{\oseq{\octx'_L \oc \octx \oc \octx'_R |- C}}{
      \oseq{\octx |- A} & \oseq{\octx'_L \oc A \oc \octx'_R |- C}}
    \and 
    \infer[\jrule{ID}\smash{^A}]{\oseq{A |- A}}{}
    \\
    \infer[\jrule{CUT}\smash{^{\reduces}}]{\oseq{\octx'_L \oc \octx \oc \octx'_R |- C}}{
      \octx \reduces \octx' & \oseq{\octx'_L \oc \octx' \oc \octx'_R |- C}}
    \\
    \infer[\rrule{\fuse}]{\oseq{\octx_1 \oc \octx_2 |- A \fuse B}}{
      \oseq{\octx_1 |- A} & \oseq{\octx_2 |- B}}
    \and
    \infer[\jrule{$\fuse$D}]{A \fuse B \reduces A \oc B}{}
    \\
    \infer[\rrule{\one}]{\oseq{\octxe |- \one}}{}
    \and
    \infer[\jrule{$\one$D}]{\one \reduces \octxe}{}
    \\
    \infer[\rrule{\with}]{\oseq{\octx |- A \with B}}{
      \oseq{\octx |- A} & \oseq{\octx |- B}}
    \and
    \infer[\jrule{$\with$D}_1]{A \with B \reduces A}{}
    \and
    \infer[\jrule{$\with$D}_2]{A \with B \reduces B}{}
    \\
    \infer[\rrule{\top}]{\oseq{\octx |- \top}}{}
    \and
    \text{(no $\jrule{$\top$D}$ rule)}
    \\
    \infer[\rrule{\limp}]{\oseq{\octx |- A \limp B}}{
      \oseq{A \oc \octx |- B}}
    \and
    \infer[\jrule{$\limp$D}]{A \oc (A \limp B) \reduces B}{}
    \\
    \infer[\rrule{\pmir}]{\oseq{\octx |- B \pmir A}}{
      \oseq{\octx \oc A |- B}}
    \and
    \infer[\jrule{$\pmir$D}]{(B \pmir A) \oc A \reduces B}{}
    \\
    \infer[\rrule{\plus}_1]{\oseq{\octx |- A \plus B}}{
      \oseq{\octx |- A}}
    \and
    \infer[\rrule{\plus}_2]{\oseq{\octx |- A \plus B}}{
      \oseq{\octx |- B}}
    \and
    \infer[\lrule{\plus}]{\oseq{\octx'_L \oc (A \plus B) \oc \octx'_R |- C}}{
      \oseq{\octx'_L \oc A \oc \octx'_R |- C} &
      \oseq{\octx'_L \oc B \oc \octx'_R |- C}}
    \\
    \text{(no $\rrule{\zero}$ rule)}
    \and
    \infer[\lrule{\zero}]{\oseq{\octx'_L \oc \zero \oc \octx'_R |- C}}{}
  \end{inferences}
  \caption{A refactoring of the ordered sequent calculus to emphasize that most left rules amount to resource decomposition}\label{fig:ordered-rewriting:decompose-seq-calc}
\end{figure}

\newthought{\Cref{fig:ordered-rewriting:decompose-seq-calc} presents} the refactored sequent calculus for ordered logic in its entirety.
This calculus is sound and complete with respect to the ordered sequent calculus~\parencref{fig:ordered-logic:sequent-calculus}.
%
\begin{theorem}[Soundness and completeness]
  $\oseq{\octx |- A}$ is derivable in the refactored calculus of \cref{fig:ordered-rewriting:decompose-seq-calc} if, and only if $\oseq{\octx |- A}$ is derivable in the usual ordered sequent calculus~\parencref{fig:ordered-logic:sequent-calculus}.
\end{theorem}
%
\begin{proof}
  Soundness, the right-to-left direction, can be proved by structural induction on the given derivation.
  The key lemma is the admissibility of $\jrule{CUT}^{\reduces}$ in the usual ordered sequent calculus:
  \begin{quotation}
    \normalsize If $\octx \reduces \octx'$ and $\oseq{\octx'_L \oc \octx' \oc \octx'_R |- C}$, then $\oseq{\octx'_L \oc \octx \oc \octx'_R |- C}$.
  \end{quotation}
  This lemma can be proved by case analysis of the decomposition $\octx \reduces \octx'$, reconstituting the corresponding left rule along the lines of the sketches from \cref{fig:ordered-rewriting:fuse-refactoring,fig:ordered-rewriting:limp-refactoring-2}.

% \end{proof}
%
% \begin{theorem}[Completeness]
%   If\/ $\oseq{\octx |- A}$ is derivable in the usual ordered sequent calculus~\parencref{fig:ordered-logic:sequent-calculus}, then $\oseq{\octx |- A}$ is derivable in the refactored calculus of \cref{fig:ordered-rewriting:decompose-seq-calc}.
% \end{theorem}
%
% \begin{proof}
  Completeness, the left-to-right direction, can be proved by structural induction on the given derivation.
  The critical cases are the left rules; they are resolved along the lines of the sketches shown in \cref{fig:ordered-rewriting:fuse-refactoring,fig:ordered-rewriting:limp-refactoring-2}.
\end{proof}

\subsection{Ordered resource decomposition as rewriting}

Thus far, we have used the decomposition judgment, $\octx \reduces \octx'$, and its rules as the basis for a reconfigured sequent-like calculus for ordered logic.
% But this refactoring also leads naturally to a rewriting system grounded in ordered logic.
% 
% Instead,
Additionally, 
% of taking the resource decomposition rules as a basis for a reconfigured sequent calculus,
we can instead view decomposition as the foundation of a rewriting system grounded in ordered logic.
For example, the decomposition of resource $A \fuse B$ into $A \oc B$ by the $\jrule{$\fuse$D}$ rule
% \marginnote{%
%   \begin{equation*}
%     \infer[\jrule{$\fuse$D}]{A \fuse B \reduces A \oc B}{}
%   \end{equation*}
% }%
can also be seen as \emph{rewriting} $A \fuse B$ into $A \oc B$.
More generally, the decomposition judgment $\octx \reduces \octx'$ can be read as \enquote{$\octx$ rewrites to $\octx'$.}

\Cref{fig:ordered-rewriting:rewriting} summarizes the rewriting system that we obtain from the refactored sequent-like calculus of \cref{fig:ordered-rewriting:decompose-seq-calc}%
%
\begin{figure}[tbp]
  \vspace{\dimexpr-\abovedisplayskip-\abovecaptionskip\relax}
  \begin{inferences}
    \infer[\jrule{$\fuse$D}]{A \fuse B \reduces A \oc B}{}
    \and
    \infer[\jrule{$\one$D}]{\one \reduces \octxe}{}
    \\
    \infer[\jrule{$\with$D}_1]{A \with B \reduces A}{}
    \and
    \infer[\jrule{$\with$D}_2]{A \with B \reduces B}{}
    \and
    \text{(no $\jrule{$\top$D}$ rule)}
    \\
    \infer[\jrule{$\limp$D}]{A \oc (A \limp B) \reduces B}{}
    \and
    \infer[\jrule{$\pmir$D}]{(B \pmir A) \oc A \reduces B}{}
    \and
    \text{(no $\jrule{$\plus$D}$ and $\jrule{$\zero$D}$ rules)}
    \\
    \infer[\jrule{$\reduces$C}]{\octx_L \oc \octx \oc \octx_R \reduces \octx_L \oc \octx' \oc \octx_R}{
      \octx \reduces \octx'}
  \end{inferences}
  \begin{inferences}
    \infer[\jrule{$\Reduces$R}]{\octx \Reduces \octx}{}
    \and
    \infer[\jrule{$\Reduces$T}]{\octx \Reduces \octx''}{
      \octx \reduces \octx' & \octx' \Reduces \octx''}
  \end{inferences}
  \caption{The \acs*{OR} rewriting fragment of ordered logic, based on resource decomposition}\label{fig:ordered-rewriting:rewriting}
\end{figure}%
%
; we dub this ordered rewriting system \acs{OR}.
Essentially, \ac{OR} is obtained by discarding all rules except for the decomposition rules.
However, if only the decomposition rules are used, rewritings cannot occur within a larger context.
For example, the $\jrule{$\limp$D}$ rule derives $A \oc (A \limp B) \reduces B$, but $\octx'_L \oc A \oc (A \limp B) \oc \octx'_R \reduces \octx'_L \oc B \oc \octx'_R$ would not be derivable in general.
In the refactored calculus of \cref{fig:ordered-rewriting:decompose-seq-calc}, this kind of framing is taken care of by the cut principle for decomposition, $\jrule{CUT}^{\reduces}$.
To express framing at the level of the $\octx \reduces \octx'$ judgment itself, we ensure that rewriting is compatible with concatenation of ordered contexts:
\begin{equation*}
  \infer[\jrule{$\reduces$C}]{\octx_L \oc \octx \oc \octx_R \reduces \octx_L \oc \octx' \oc \octx_R}{
    \octx \reduces \octx'}
  \,.
\end{equation*}

By forming the reflexive, transitive closure of $\reduces$, we may construct a multi-step rewriting relation, which we choose to write as $\Reduces$.%
\footnote{% [][0.5\baselineskip]{%
  Usually written as $\reduces^*$, we instead chose $\Reduces$ for the reflexive, transitive closure because of its similarity with process calculus notation for weak transitions, $\Reduces[\smash{\alpha}]$.
  Our reasons will become clearer in subsequent \lcnamecrefs{ch:ordered-bisimilarity}.%
}
Consistent with its monoidal structure, there are two equivalent formulations of this reflexive, transitive closure: each rewriting sequence $\octx \Reduces \octx'$ can be viewed as either a list or tree of individual rewriting steps.\fixnote{rewrite with reference to string rewriting}
We prefer the list-based formulation shown in \cref{fig:ordered-rewriting:rewriting} because it tends to streamline proofs by structural induction, but, on the basis of the following \lcnamecref{fact:ordered-rewriting:transitivity}, we allow ourselves to freely switch between the two formulations as needed.
%
\begin{fact}[Transitivity of $\Reduces$]\label{fact:ordered-rewriting:transitivity}
  If \kern0.15em$\octx \Reduces \octx'$ and\/ $\octx' \Reduces \octx''$, then\/ $\octx \Reduces \octx''$.
\end{fact}
%
\begin{proof}
  By induction on the structure of the first trace, $\octx \Reduces \octx'$.
\end{proof}

\newthought{A few remarks} about these rewriting relations are in order.
%
First, interpreting the resource decomposition rules as rewriting only confirms our preference for the nullary $\jrule{$\limp$D}$ and $\jrule{$\pmir$D}$ rules~(eq.~\ref{eq:ordered-rewriting:limp-pmir-decomposition-nullary}).
% [over the $\jrule{$\limp$D}'$ and $\jrule{$\pmir$D}'$ rules.]
The $\jrule{$\limp$D}'$ and $\jrule{$\pmir$D}'$ rules~(eq.~\ref{eq:ordered-rewriting:limp-pmir-decomposition}), with their $\oseq{\octx |- A}$ premises, would be problematic as rewriting rules because they would introduce a dependence of rewriting upon general provability
% By instead using the $\jrule{$\limp$D}$ and $\jrule{$\pmir$D}$ rules, we ensures that ordered rewriting is a syntactic procedure that
% Instead, we want ordered rewriting to be a syntactic procedure, withou 
 and the accompanying proof search would take \ac{OR} too far afield from traditional, syntactic notions of string and multiset rewriting.

Second, multi-step rewriting, $\Reduces$, is incomplete with respect to the usual ordered sequent calculus~\parencref{fig:ordered-logic:sequent-calculus} because all right rules have been discarded.%
%
 \begin{falseclaim}[Completeness]
  If \kern0.15em$\oseq{\octx |- A}$, then\/ $\octx \Reduces A$.
\end{falseclaim}
%
\begin{proof}[Counterexample]
  The sequent $\oseq{A \limp (C \pmir B) |- (A \limp C) \pmir B}$ is provable, and yet $A \limp (C \pmir B) \Longarrownot\Reduces (A \limp C) \pmir B$ (even though $A \oc (A \limp (C \pmir B)) \oc B \Reduces C$ does hold).
\end{proof}
\noindent
As expected from the way in which it was developed, ordered rewriting in \acs{OR} is, however, sound.
To state and prove soundness, we must first define an operation $\bigfuse \octx$ that reifies an ordered context as a single proposition (see adjacent \lcnamecref{fig:ordered-rewriting:bigfuse}).\fixnote{fix}%
%
\begin{marginfigure}
  \begin{align*}
    (\octx_1 \oc \octx_2) &= (\octx_1) \fuse (\octx_2) \\
    \mathord{\text{$\fuse$}} (\octxe) &= \one \\
    A &= A
  \end{align*}
  \begin{align*}
    \bigfuse (\octx_1 \oc \octx_2) &= (\bigfuse \octx_1) \fuse (\bigfuse \octx_2) \\
    \bigfuse (\octxe) &= \one \\
    \bigfuse A &= A
  \end{align*}
  \caption{From ordered contexts to propositions}\label{fig:ordered-rewriting:bigfuse}
\end{marginfigure}%
%
\begin{lemma}
  For all $\octx$ and $C$, we have $\oseq{\octx |- C}$ implies $\oseq{\bigfuse \octx |- C}$, as well as $\oseq{\octx |- \bigfuse \octx}$.
\end{lemma}
\begin{proof}
  By induction on the structure of the given context, $\octx$.
\end{proof}
%
\begin{theorem}[Soundness]
  If \kern0.15em$\octx \reduces \octx'$, then\/ $\oseq{\octx |- \bigfuse \octx'}$.
  Also, if \kern0.15em$\octx \Reduces \octx'$, then\/ $\oseq{\octx |- \bigfuse \octx'}$.
\end{theorem}
%
\begin{proof}
  By induction on the structure of the given step or trace.
\end{proof}

Last, notice that every rewriting step, $\octx \reduces \octx'$, strictly decreases the number of logical connectives that occur in the ordered context.
More formally, let $\card{\octx}_{\star}$ be a measure of the number of logical connectives that occur in $\octx$, as defined in the adjacent \lcnamecref{fig:ordered-rewriting:measure}.
%
\begin{marginfigure}
  \begin{align*}
    \card{\octx_1 \oc \octx_2}_{\star} &= \card{\octx_1}_{\star} + \card{\octx_2}_{\star} \\
    \card{\octxe}_{\star} &= 0 \\
    \card{A \star B}_{\star} &= \begin{tabular}[t]{@{}l@{}}
                          $1 + \card{A}_{\star} + \card{B}_{\star}$ \\
                          \quad if $\mathord{\star} = \mathord{\fuse}$, $\mathord{\with}$, $\mathord{\limp}$, $\mathord{\pmir}$, or $\mathord{\plus}$
                         \end{tabular} \\
    \card{A}_{\star} &= \mathrlap{1}
                    \quad \text{if $A = a$, $\one$, $\top$, or $\zero$}
  \end{align*}
  \caption{A measure of the number of logical connectives within an ordered context}\label{fig:ordered-rewriting:measure}
\end{marginfigure}%
%
We may then prove the following \lcnamecref{lem:ordered-rewriting:reduction}.%
%
\begin{lemma}\label{lem:ordered-rewriting:reduction}
  If \kern0.15em$\octx \reduces \octx'$, then $\card{\octx}_{\star} > \card{\octx'}_{\star}$.
  If \kern0.15em$\octx \Reduces \octx'$, then $\card{\octx}_{\star} \geq \card{\octx'}_{\star}$.
\end{lemma}
%
\begin{proof}
  By induction on the structure of the rewriting step.
\end{proof}
%
\noindent
On the basis of this \lcnamecref{lem:ordered-rewriting:reduction}, we will frequently refer to the rewriting relation, $\reduces$, as the \vocab{reduction relation}.
We may use this \lcnamecref{lem:ordered-rewriting:reduction} to prove that ordered rewriting is terminating.
% 
% Because each rewriting step reduces the number of logical connectives present in the state~\parencref{lem:ordered-rewriting:reduction}, ordered rewriting is terminating.
%
\begin{theorem}[Termination]\label{thm:ordered-rewriting:termination}
  For all ordered contexts $\octx$, every rewriting sequence from $\octx$ is finite.
  % $\octx_0 \reduces \octx_1 \reduces \octx_2 \reduces \dotsb$ exists.
\end{theorem}
%
\begin{proof}
  Let $\octx$ be an arbitrary ordered context.
  Beginning from state $\octx_0 = \octx$, some state $\octx_i$ will eventually be reached such that either: $\octx_i \nreduces$; or $\card{\octx_i}_{\star} = 0$ and $\octx_i \reduces \octx_{i+1}$.
  In the latter case, \cref{lem:ordered-rewriting:reduction} establishes $\card{\octx_{i+1}}_{\star} < 0$, which is impossible because $\card{}_{\star}$ is a measure.
\end{proof}


\subsection{Recursively defined propositions and unbounded ordered rewriting}

Although a seemingly pleasant property, termination~\parencref{thm:ordered-rewriting:termination} significantly limits the expressiveness of ordered rewriting.
For example, without unbounded rewriting, we cannot even use ordered rewriting to describe producer-consumer systems or finite automata.

As the proof of termination shows, rewriting is bounded
% $\card{\octx_0}$ is an upper bound on the length of any trace from state $\octx_0$,
precisely because
% $\octx_0$
states
consist of finitely many finite propositions.
One way to admit unbounded rewriting is therefore to permit circular propositions in the form of mutually recursive definitions, $\defp{p} \defd A$, where the grammar of ordered propositions now includes these recursively defined propositions $\defp{p}$:
\begin{equation*}
  A,B \Coloneqq a \mid A \fuse B \mid \one \mid A \limp B \mid B \pmir A \mid A \with B \mid \top \mid \defp{p}
  \,.
\end{equation*}
Sequent calculi with definitions of this kind have previously been studied by \textcites{Hallnas:??}{Erikkson:??}{Schroeder-Heister:??}{McDowell+Miller:??}{Tiu+Momigliano:??}, among others.

To rule out definitions like $\defp{p} \defd \defp{p}$ that do not correspond to sensible infinite propositions, we require that definitions be \vocab{contractive}\autocite{Gay+Hole:AI05} -- \ie, that the body of each recursive definition begin with a logical connective (or constant or atom) at the top level.

% could either permit states consisting of infinitely many finite propositions or states consisting of finitely many infinite propositions.
% We choose the latter route [...].
%%
%%
% Infinite propositions are described by mutually recursive definitions $\alpha \defd A$.
The recursive definitions are collected into a signature, $\orsig$, which indexes the rewriting relations: $\reduces_{\orsig}$ and $\Reduces_{\orsig}$.%
\footnote{We nearly always elide the index, as it is usually clear from context.} 
Syntactically, these signatures are given by
\begin{equation*}
  \orsig \Coloneqq \orsige \mid \orsig , (\defp{p} \defd A)
  \,.
\end{equation*}


\newthought{By analogy with} recursive types from functional programming\autocite{??}, we must now decide whether to treat definitions \emph{iso}\-re\-cur\-sively or \emph{equi}\-re\-cur\-sively.
Under an equirecursive interpretation, definitions $\defp{p} \defd A$ may be silently unrolled or rolled at will;
in other words, $\defp{p}$ is literally \emph{equal} to its unrolling -- $\defp{p} = A$.
In contrast, under an isorecursive interpretation, unrolling a recursively defined proposition would count as an explicit step of rewriting -- $\defp{p} \neq A$ but $\defp{p} \reduces A$, for example.

% Under the isorecursive interpretation, unrolling a recursively defined prop\-o\-sition counts as an explicit step of rewriting.
% We introduce the $\jrule{$\defd$D}$ rule to account for this unrolling:
% \begin{equation*}
%   \infer[\jrule{$\defd$D}]{\alpha \reduces_{\sig} A}{
%     \text{$(\alpha \defd A) \in \sig$}}
% \end{equation*}
% Because $A$ is seen as a proper subformula of [the recursively defined] $\alpha$, this unrolling rule aligns well with the rewriting-as-decomposition philosophy.%
% \footnote{In fact, we could have chosen to include recursive definitions in the sequent calculus, following \textcites{SchroederHeister:LICS93}{Tiu+Momigliano:JAL12} and others.
%   Had we done so, the $\jrule{$\defd$D}$ rule would be seen as the decomposition counterpart to the left rule
%   \begin{equation*}
%     \infer[\lrule{\defd}]{\oseq{\octx'_L \oc \alpha \oc \octx'_R |-_{\sig} C}}{
%       \bigl((\alpha \defd A) \in \sig\bigr) &
%       \oseq{\octx'_L \oc A \oc \octx'_R |-_{\sig} C}}
%   \end{equation*}
% }
% Conversely, there is no rule that permits the rolling of $A$ into $\alpha$, because such a rule would not be a decomposition.

We choose to interpret definitions equirecursively
because the equirecursive treatment, with its generous notion of equality, helps to minimize the overhead of recursively defined propositions.
As a simple example, under the equirecursive definition $\defp{p} \defd a \limp \defp{p}$, we have the trace
\begin{equation*}
  a \oc a \oc \defp{p} = a \oc a \oc (a \limp \defp{p}) \reduces a \oc \defp{p} = a \oc (a \limp \defp{p}) \reduces \defp{p}
\end{equation*}
or, more concisely, $a \oc a \oc \defp{p} \reduces a \oc \defp{p} \reduces \defp{p}$.
Had we chosen
% With
 an isorecursive treatment of the same definition, we would have only the more laborious
\begin{equation*}
  a \oc a \oc \defp{p} \reduces a \oc a \oc (a \limp \defp{p}) \reduces a \oc \defp{p} \reduces a \oc (a \limp \defp{p}) \reduces \defp{p}
  \,.
\end{equation*}
This choice differs from the aforementioned works on definitions, which use an isorecursive treatment with explicit right and left rules for recursively defined propositions.


\paragraph*{Replication}

In Milner's development of the $\pi$-calculus, there are two avenues to unbounded process behavior: recursive process definitions and replication.

\autocite{Aranda+:FMCO06}


\subsection{Properties of the \acs*{OR} \acl*{OR} framework}

\paragraph*{Concurency}

As an example of multi-step rewriting, observe that
\begin{equation*}
  % \octx = 
  a \oc (a \limp b) \oc (c \pmir a) \oc a \Reduces b \oc c
  % = \octx''
  .
\end{equation*}
In fact, as shown in the adjacent \lcnamecref{fig:ordered-rewriting:concurrent-example},%
%
\begin{marginfigure}
  \begin{equation*}
  \begin{tikzcd}[row sep=large, column sep=tiny]
    &
    \makebox[1em][c]{$a \oc (a \limp b) \oc (c \pmir a) \oc a$}
      \dlar \drar \arrow[Reduces]{dd}
    &
    \\
    b \oc (c \pmir a) \oc a
      \drar
    &&
    a \oc (a \limp b) \oc c
      \dlar
    \\
    &
    b \oc c
    &
  \end{tikzcd}
\end{equation*}
  \caption{An example of concurrency in ordered rewriting}\label{fig:ordered-rewriting:concurrent-example}
\end{marginfigure}
%
two sequences witness this rewriting: either
\begin{itemize*}[
  mode=unboxed,
  label=, afterlabel=
]
\item the initial state's left half, $a \oc (a \limp b)$, is first rewritten to $b$ and then its right half, $(c \pmir a) \oc a$, is rewritten to $c$; or
\item \textit{vice versa}, the right half is first rewritten to $c$ and then the left half is rewritten to $b$
\end{itemize*}.

Notice that these two sequences differ only in how non-overlapping, and therefore independent, rewritings of the initial state's two halves are interleaved.
Consequently, the two sequences can be -- and indeed should be -- considered essentially equivalent.
% In differing only by the order in which the non-overlapping left and right halves are rewritten, these two rewriting sequences are essentially equivalent.
The details of how the small-step rewrites are interleaved are irrelevant, so that
conceptually, at least, only the big-step trace from $a \oc (a \limp b) \oc (c \pmir a) \oc a$ to $b \oc c$ remains.
% The details of how the small-step rewrites are interleaved are -- and indeed should be -- swept away, so that conceptually only the big-step trace from $\alpha_1 \oc (\alpha_1 \limp \alpha_2) \oc (\beta_2 \pmir \beta_1) \oc \beta_1$ to $\alpha_2 \oc \beta_2$ remains.

More generally, this idea that the interleaving of independent actions is irrelevant is known as \vocab{concurrent equality}\autocite{Watkins+:CMU02}, and it forms the basis of concurrency.\autocite{??}
Concurrent equality also endows traces $\octx \Reduces \octx'$ with a free partially commutative monoid structure, \ie, traces form a trace monoid.


% Because the two indivisual rewriting steps are independent, 
% Nothing about the final result, $\alpha_2 \oc \beta_2$, suggests which rewriting sequence 


% The rewritings of the left and right halves are not overlapping and therefore independent.
% Their independence means that we may view the two rewriting sequences as equivalent -- the two rewriting steps

% More generally, any non-overlapping rewritings are independent and may occur in any order.
% Rewriting sequences that differ only by the order in which independent rewritings occur may be seen as equivalent sequences.
% This equivalence relation, \vocab{concurrent equality}\autocite{Watkins+:CMU02}

% because the left half of $\octx$ may be rewritten by the $\jrule{$\limp$D}$ rule to $\alpha_2$, and then the right half may be rewritten to $\beta_2$:


\clearpage

\paragraph*{Non-confluence}

As the relation $\Reduces$ forms a rewriting system, we may evaluate it along several standard dimensions: termination, confluence.


Although terminating, ordered rewriting is not confluent.
Confluence requires that all states with a common ancestor, \ie, states $\octx'_1$ and $\octx'_2$ such that $\octx'_1 \secudeR\Reduces \octx'_2$, be joinable, \ie, $\octx'_1 \Reduces\secudeR \octx'_2$.
Because ordered rewriting is directional\fixnote{Is this phrasing correct?} and the relation $\Reduces$ is not symmetric, some nondeterministic choices are irreversible.%
%
\begin{falseclaim}[Confluence]
  If\/ $\octx'_1 \secudeR\Reduces \octx'_2$, then $\octx'_1 \Reduces\secudeR \octx'_2$.
\end{falseclaim}
%
\begin{proof}[Counterexamples]
  Consider the state $a \with b$.
  By the rewriting rules for additive conjunction, $a \secuder a \with b \reduces b$, and hence $a \secudeR a \with b \Reduces b$.
  However, being atoms, neither $a$ nor $b$ reduces.
  And $a \neq b$, so $a \Reduces\secudeR b$ does \emph{not} hold.

  Even in the $\with$-free fragment, ordered rewriting is not confluent:
  for example,
  % consider the state $(\beta_1 \pmir \alpha) \oc \alpha \oc (\alpha \limp \beta_2)$.
  % By the rewriting rules for right- and left-handed implications,
  \begin{equation*}
    \nsecuder c \oc (a \limp b) \secudeR (c \pmir a) \oc a \oc (a \limp b) \Reduces (c \pmir a) \oc b \nreduces
    .
    \qedhere
  \end{equation*}
\end{proof}


% Viewing the resource decomposition rules for left- and right-handed implications as rewriting rules is slightly problematic, however.%
% Notice that the premises of these rules both require proofs of $\oseq{\octx |-  A}$.
% In the refactored sequent calculus of \cref{fig:ordered-rewriting:decompose-seq-calc}, that dependence of judgments is fine.
% But for a rewriting system, including arbitrary[/general] proofs would be odd -- rewriting should be a syntax-directed process and should not depend on provability.



% We write the reflexive, transitive closure of $\reduces$ as $\Reduces$.%
% \footnote{This notation is adopted for its similarity with the standard $\pi$-calculus notation for weak transitions, $\cramped{\Reduces[\alpha]}$.}

% This rewriting system is a proper fragment of ordered logic.
% \begin{equation*}
%   \oseq{A \limp (C \pmir B) \dashv|- (A \limp C) \pmir B}
%   \enspace\text{but}\enspace
%   A \limp (C \pmir B) \Longarrownot\Reduces (A \limp C) \pmir B
% \end{equation*}




\section{The \acs*{FOR} focused ordered rewriting framework}

The above ordered rewriting framework is based upon decomposition rules that are very fine-grained.
Each step of rewriting decomposes a proposition into only its immediate subformulas, and no further, such as in the very fine-grained step $a \oc \bigl((a \limp c \fuse a) \with (b \limp \one)\bigr) \reduces a \oc (a \limp c \fuse a)$.
% \begin{marginfigure}
%   \begin{equation*}
%     \begin{tikzcd}[sep=small]
%       & a \mathrlap{\oc (a \limp b) \reduces b}
%       \\
%       \mathllap{a \oc \bigl((a \limp b) \with (c \pmir a)\bigr)}
%         \urar[reduces] \drar[reduces]
%       \\
%       & a \mathrlap{\oc (c \pmir a) \nreduces}
%     \end{tikzcd}
%     \hphantom{\oc (a \limp b) \reduces b}
%   \end{equation*}
%   \caption{}
% \end{marginfigure}%
It is not possible to rewrite the context $a \oc \bigl((a \limp c \fuse a) \with (b \limp \one)\bigr)$ into $c \oc a$ (or even $c \fuse a$) in a single step, although it is possible in several steps: $a \oc \bigl((a \limp c \fuse a) \with (b \limp \one)\bigr) \Reduces c \oc a$, because
\begin{equation*}
  a \oc \bigl((a \limp c \fuse a) \with (b \limp \one)\bigr) \reduces a \oc (a \limp c \fuse a) \reduces c \fuse a \reduces c \oc a
  \,.
\end{equation*}

The decomposition rules are so fine-grained that rewriting may even get stuck in undesirable and unintended ways.
For instance, in the previous example, we might have instead nondeterministically committed to rewriting $a \oc \bigl((a \limp c \fuse a) \with (b \limp \one)\bigr)$ into $a \oc (b \limp \one)$ as the first step, and then $a \oc (b \limp \one)$ is stuck, with no further rewritings possible:
\begin{equation*}
  a \oc \bigl((a \limp c \fuse a) \with (b \limp \one)\bigr)
    \reduces a \oc (b \limp \one)
    \nreduces
    \!\,.
\end{equation*}
Instead, we would rather have a coarser notion of decomposition so that $a \oc \bigl((a \limp c \fuse a) \with (b \limp \one)\bigr) \reduces c \oc a$ is a single step%
\footnote{Or at least so that $a \oc \bigl((a \limp c \fuse a) \with (b \limp \one)\bigr) \reduces c \fuse a$ is a single step.}
and, conversely, so that $a \oc \bigl((a \limp c \fuse a) \with (b \limp \one)\bigr) \reduces \octx'$ only if $\octx' = c \oc a$.

\newthought{Focusing, as developed by} \textcite{Andreoli:??}, provides just the right coarse-grained decomposition through its complementary inversion and chaining strategies for proof search.
An inversion phase groups together successive invertible rules, and a chaining phase groups together successive noninvertible rules that are applied to a single \emph{in-focus} proposition;
together, a chaining phase followed by an inversion phase constitutes a \emph{bipole}.
Rather than having each of these rules give rise to a separate step, we can treat the entire bipole as an atomic step of rewriting.

This idea of using focusing to increase the granularity of rewriting steps dates back to, at least, the \acf{CLF}\autocites{??}{??} and was later streamlined by \textcite{Cervesato+Scedrov:IC09}.
\Textcite{Simmons:CMU12} has studied a focused ordered rewriting framework, though in a somewhat different formulation than the one we present here.

\newthought{The ordered propositions are polarized} into positive and negative classes, or \vocab{polarities}\autocite{??}, according to the invertibility of their sequent calculus rules; two \enquote*{shift} connectives, $\dn$ and $\up$, mediate between the two classes.
\begin{align*}
  % Positive props. &
    \p{A} &\Coloneqq \p{a} \mid \p{A} \fuse \p{B} \mid \one \mid \dn \n{A}
  \\
  % Negative props. &
    \n{A} &\Coloneqq \p{A} \limp \n{B} \mid \n{B} \pmir \p{A} \mid \n{A} \with \n{B} \mid \top \mid \up \p{A}
\end{align*}
The positive propositions, $\p{A}$, are those propositions that have invertible left rules, such as ordered conjunction; the negative propositions, $\n{A}$, are those that have invertible right rules, such as the left- and right-handed implications.
For reasons that will become clear in \cref{ch:formula-as-process}, we choose to assign a positive polarity to all atomic propositions, $\p{a}$.

Ordered contexts are then formed as the free monoid over negative propositions and positive atoms:
\begin{equation*}
  \np{\octx} \Coloneqq \np{\octx}_1 \oc \np{\octx}_2 \mid \octxe \mid \n{A} \mid \p{a}
  \,.
\end{equation*}
As usual, we do not distinguish those ordered contexts that are equivalent up to the monoid laws.
We may also reify an ordered context $\np{\octx}$ as a positive proposition, $\bigfuse \np{\octx}$, using the operation defined in the neighboring \lcnamecref{fig:ordered-rewriting:bigfuse}%
%
\begin{marginfigure}
  \begin{align*}
    \bigfuse (\np{\octx}_1 \oc \np{\octx}_2) &= (\bigfuse \np{\octx}_1) \fuse (\bigfuse \np{\octx}_2) \\
    \bigfuse (\octxe) &= \one \\
    \bigfuse \n{A} &= \dn \n{A} \\
    \bigfuse \p{a} &= \p{a}
  \end{align*}
  \caption{Reifying an ordered context as a positive proposition}
\end{marginfigure}%
%
.

\newthought{Each class of propositions} is then equipped with its own focusing judgment: a \vocab{left-focus judgment}, $\lfocus{\np{\octx}_L}{\n{A}}{\np{\octx}_R}{\p{C}}$, that focuses on a negative proposition, $\n{A}$, that occurs to the left of the turnstile; and a \vocab{right-focus judgment}, $\rfocus{\np{\octx}}{\p{A}}$, that focuses on a positive proposition, $\p{A}$, that occurs to the right of the turnstile.%
\footnote{We choose a left-facing turnstile for the right-focus judgment to emphasize its input/output mode; see the next paragraph.}

Following \textcite{Zeilberger:??}, each of these judgments can be read as a function that provides a form of extended decomposition -- the in-focus proposition is decomposed beyond its immediate subformulas, until subformulas of the opposite polarity are reached.
% The judgment $\rfocus{\octx}{\p{A}}$ decomposes the in-focus proposition $\p{A}$ into the ordered context $\octx$ of negative subformulas,
% and $\lfocus{\octx_L}{\n{A}}{\octx_R}{\p{C}}$ decomposes the in-focus proposition $\n{A}$ into the ordered contexts $\octx_L$ and $\octx_R$ and the positive subformula $\p{C}$.
% 
The two focusing judgments are defined inductively on the structure of the in-focus proposition, with the left-focus judgment depending on the right-focus judgment (though not vice versa).

The right-focus judgment, $\rfocus{\np{\octx}}{\p{A}}$, decomposes $\p{A}$ into the ordered context $\np{\octx}$ of its nearest negative subformulas, treating $\p{A}$ as input and $\np{\octx}$ as output.%
% \footnote{This input/output behavior explains why we choose to write the \emph{right}-focus judgment as $\rfocus{\np{\octx}}{\p{A}}$ with a \emph{left}-facing turnstile:
  % when written this way, the judgment's input is followed by its output.}
% The right-focus judgment, $\rfocus{\np{\octx}}{\p{A}}$, decomposes $\p{A}$ into the ordered context $\np{\octx}$ of its nearest negative subformulas;
% this judgment therefore takes an in-focus proposition, $\p{A}$, as an input and produces an ordered context, $\np{\octx}$, as output.
The judgment is given by the following rules.
\begin{inferences}
  \infer[\rrule{\fuse}]{\rfocus{\np{\octx}_1 \oc \np{\octx}_2}{\p{A} \fuse \p{B}}}{
    \rfocus{\np{\octx}_1}{\p{A}} & \rfocus{\np{\octx}_2}{\p{B}}}
  \and
  \infer[\rrule{\one}]{\rfocus{\octxe}{\one}}{}
  \\
  \infer[\jrule{ID}\smash{^{\p{a}}}]{\rfocus{\p{a}}{\p{a}}}{}
  \and
  \infer[\rrule{\dn}]{\rfocus{\n{A}}{\dn \n{A}}}{}
\end{inferences}
% This right-focus judgment decomposes a positive proposition until its [largest] negative subformulas are reached.
Ordered conjunctions $\p{A} \fuse \p{B}$ are decomposed into $\np{\octx}_1 \oc \np{\octx}_2$ by inductively decomposing $\p{A}$ and $\p{B}$ into $\np{\octx}_1$ and $\np{\octx}_2$, respectively, and $\one$ is decomposed into the empty context.
Atoms $\p{a}$ are not decomposed further%
\footnote{Alternatively, following \textcite{Simmons:CMU12}, atoms $\p{a}$ could be decomposed to suspensions $\langle\p{a}\rangle$, but we choose not to do that here.}%
, and $\dn \n{A}$ is decomposed into its immediate subformula of negative polarity, $\n{A}$.

This right-focus judgment is a left inverse of the $\bigfuse (-)$ operation:
\begin{lemma}
  $\rfocus{\np{\octx'{}}}{\bigfuse \np{\octx}}$ if, and only if, $\np{\octx} = \np{\octx'{}}$.
\end{lemma}
\begin{proof}
  Each direction is separately proved by structural induction on the context $\np{\octx}_2$.
\end{proof}

The left-focus judgment, $\lfocus{\np{\octx}_L}{\n{A}}{\np{\octx}_R}{\p{C}}$, decomposes $\n{A}$ into the ordered contexts $\np{\octx}_L$ and $\np{\octx}_R$ and positive subformula $\p{C}$, treating $\n{A}$ as input and $\np{\octx}_L$, $\np{\octx}_R$, and $\p{C}$ as outputs.
The judgment is given by the following rules.
\begin{inferences}
  \infer[\lrule{\limp}]{\lfocus{\np{\octx}_L \oc \np{\octx}_A}{\p{A} \limp \n{B}}{\np{\octx}_R}{\p{C}}}{
    \rfocus{\np{\octx}_A}{\p{A}} &
    \lfocus{\np{\octx}_L}{\n{B}}{\np{\octx}_R}{\p{C}}}
  \and
  \infer[\lrule{\pmir}]{\lfocus{\np{\octx}_L}{\n{B} \pmir \p{A}}{\np{\octx}_A \oc \np{\octx}_R}{\p{C}}}{
    \rfocus{\np{\octx}_A}{\p{A}} &
    \lfocus{\np{\octx}_L}{\n{B}}{\np{\octx}_R}{\p{C}}}
  \\
  \infer[\lrule{\with}_1]{\lfocus{\np{\octx}_L}{\n{A} \with \n{B}}{\np{\octx}_R}{\p{C}}}{
    \lfocus{\np{\octx}_L}{\n{A}}{\np{\octx}_R}{\p{C}}}
  \quad\enspace%\and\!
  \infer[\lrule{\with}_2]{\lfocus{\np{\octx}_L}{\n{A} \with \n{B}}{\np{\octx}_R}{\p{C}}}{
    \lfocus{\np{\octx}_L}{\n{B}}{\np{\octx}_R}{\p{C}}}
  \quad\enspace%\and\!
  \text{(no $\lrule{\top}$ rule)}
  \\
  \infer[\lrule{\up}]{\lfocus{}{\up \p{A}}{}{\p{A}}}{}
\end{inferences}
The left-focus judgment's rules parallel the usual sequent calculus rules, but maintaining focus on the immediate subformulas -- left focus for subformulas of negative polarity and right focus for subformulas of positive polarity.
The $\lrule{\up}$ rule ends left focus by decomposing an $\up \p{A}$ antecedent into an $\p{A}$ consequent.

Unlike the right-focus judgment, the left-focus judgment describes a relation (or nondeterministic function), owing to the two rules, $\lrule{\with}_1$ and $\lrule{\with}_2$, that may apply to alternative conjunctions.
For example, both 
\begin{gather*}
  \lfocus{\p{a}}{(\p{a} \limp \up (\p{c} \fuse \p{a})) \with (\p{b} \limp \up \one)}{}{\p{c} \fuse \p{a}}
\shortintertext{and}
  \lfocus{\p{b}}{(\p{a} \limp \up (\p{c} \fuse \p{a})) \with (\p{b} \limp \up \one)}{}{\one}
\end{gather*}
are both derivable when the negative proposition $(\p{a} \limp \up (\p{c} \fuse \p{a})) \with (\p{a} \limp \up \one)$ is in focus.

\newthought{A focused rewriting step arises} when a negative proposition, $\n{A}$, is put into focus and the resulting consequent, $\p{C}$, is subsequently decomposed into the ordered context.\fixnote{name for this rule?}%
\footnote{Writing the right-focus judgment as $\rfocus{\np{\octx'{}}}{\p{B}}$ gives this rule the flavor of a cut principle.}
In addition, the compatibility rule $\jrule{$\reduces$C}$ is retained.
\begin{inferences}
  \infer[\jrule{$\reduces$I}]{\np{\octx}_L \oc \n{A} \oc \np{\octx}_R \reduces \np{\octx'{}}}{
    \lfocus{\np{\octx}_L}{\n{A}}{\np{\octx}_R}{\p{C}} &
    \rfocus{\np{\octx'{}}}{\p{C}}}
  \and
  \infer[\jrule{$\reduces$C}]{\np{\octx}_L \oc \np{\octx} \oc \np{\octx}_R \reduces \np{\octx}_L \oc \np{\octx'{}} \oc \np{\octx}_R}{
    \np{\octx} \reduces \np{\octx'{}}}
\end{inferences}

With this $\jrule{$\reduces$I}$ rule, it is indeed possible to rewrite\fixnote{fix}
\begin{equation*}
  \p{a} \oc \bigl((\p{a} \limp \up (\p{c} \fuse \p{a})) \with (\p{b} \limp \up \one)\bigr)
    \reduces \p{c} \oc \p{a}
\end{equation*}
in a single, atomic step because $\lfocus{\p{a}}{(\p{a} \limp \up (\p{c} \fuse \p{a})) \with (\p{b} \limp \up \one)}{}{\p{c} \fuse \p{a}}$ and $\rfocus{\p{c} \oc \p{a}}{\p{c} \fuse \p{a}}$ hold.
Moreover, the larger granularity afforded by the left- and right-focus judgments precludes the small steps that led to unintended stuck states.
For example:
\begin{equation*}
  \p{a} \oc \bigl((\p{a} \limp \up (\p{c} \fuse \p{a})) \with (\p{b} \limp \up \one)\bigr)
    \reduces \np{\octx'{}}
  \enspace\text{only if}\enspace
  \np{\octx'{}} = \p{c} \oc \p{a}
  \,.
\end{equation*}

\subsection{Recursively defined propositions and focused ordered rewriting}\label{sec:ordered-rewriting:focused:recursive}

With the revisions to the granularity of rewriting steps that the focused rewriting framework brings, we should pause to consider how recursively defined propositions interact with focused rewriting.

Previously, in the unfocused rewriting framework, recursively defined propostions such as $\defp{p} \defd (a \limp \defp{p} \fuse a) \with (b \limp \one)$ were permitted.
With the fine granularity of rewriting imposed in that framework, it took three steps to rewrite $a \oc \defp{p}$ into $\defp{p} \oc a$:
\begin{equation*}
  a \oc \defp{p} = a \oc \bigl((a \limp \defp{p} \fuse a) \with (b \limp \one)\bigr) \reduces a \oc (a \limp \defp{p} \fuse a) \reduces \defp{p} \fuse a \reduces \defp{p} \oc a
  \,.
\end{equation*}
% Consider the recursively defined proposition $p \defd (a \limp p \fuse a) \with (b \limp \one)$.
% Previously, in the unfocused rewriting framework, it took three steps to rewrite $a \oc p$ into $p \oc a$:
% \begin{equation*}
%   a \oc p = a \oc \bigl((a \limp p \fuse a) \with (b \limp \one)\bigr) \reduces a \oc (a \limp p \fuse a) \reduces p \fuse a \reduces p \oc a
%   \,.
% \end{equation*}

In the polarized, focused rewriting framework, the analogous definition with only the minimally necessary shifts is $\n{\defp{p}} \defd (\p{a} \limp \up (\dn \n{\defp{p}} \fuse \p{a})) \with (\p{b} \limp \up \one)$.
With the coarser granularity of rewriting now afforded by focusing, it takes only one focused step to rewrite $\p{a} \oc \n{\defp{p}}$ into $\n{\defp{p}} \oc \p{a}$:
\begin{gather*}
  \p{a} \oc \n{\defp{p}} = \p{a} \oc \bigl((\p{a} \limp \up (\dn \n{\defp{p}} \fuse \p{a})) \with (\p{b} \limp \up \one)\bigr) \reduces \n{\defp{p}} \oc \p{a}
\shortintertext{because}
  \lfocus{\p{a}}{\n{\defp{p}}}{}{\dn \n{\defp{p}} \fuse \p{a}}
  \qquad\text{and}\qquad
  \rfocus{\n{\defp{p}} \oc \p{a}}{\dn \n{\defp{p}} \fuse \p{a}}
  \,.
\end{gather*}

\newthought{Because the left-focus judgment} is defined inductively, not coinductively, there are some recursively defined negative propositions that cannot successfully be put into focus.
Under the definition $\n{\defp{p}} \defd \p{a} \limp \n{\defp{p}}$, for example, there are no contexts $\np{\octx}_L$ and $\np{\octx}_R$ and positive consequent $\p{C}$ for which $\lfocus{\np{\octx}_L}{\n{\defp{p}}}{\np{\octx}_R}{\p{C}}$ is derivable.
To derive a left-focus judgment on $\n{\defp{p}}$, the finite context $\octx_L$ would need to hold an infinite stream of $\p{a}$ atoms, which is impossible in an inductively defined context.

However, by inserting a double shift, $\up \dn$, which allows focus to be blurred at the $\up$, the definition can be revised to one that admits left-focusing: when $\n{\defp{p}}$ is defined by $\n{\defp{p}} \defd \p{a} \limp \up \dn \n{\defp{p}}$, the judgment $\lfocus{\p{a}}{\n{\defp{p}}}{}{\dn \n{\defp{p}}}$ is derivable.
It follows that $\p{a} \oc \n{\defp{p}} \reduces \n{\defp{p}}$.

More generally, any recursively defined proposition that does not pass through an $\up$ shift along a main branch cannot be successfully put into focus.




\begin{figure}
  \vspace*{\dimexpr-\abovedisplayskip-\abovecaptionskip\relax}
  \begin{syntax*}
    Positive props. &
      \p{A} & \p{A} \fuse \p{B} \mid \one \mid \p{a} \mid \dn \n{A}
    \\
    Negative props. &
      \n{A} & \p{A} \limp \n{B} \mid \n{B} \pmir \p{A} \mid \n{A} \with \n{B} \mid \top \mid \n{\defp{p}} \mid \up \p{A}
    \\
    Contexts &
      \np{\octx} & \np{\octx}_1 \oc \np{\octx}_2 \mid \octxe \mid \n{A} \mid \p{a}
    \\
    Signatures &
      \orsig & \orsige \mid \orsig, \n{\defp{p}} \defd \n{A}
  \end{syntax*}
  \begin{inferences}
    \infer[\rrule{\fuse}]{\rfocus{\np{\octx}_1 \oc \np{\octx}_2}{\p{A} \fuse \p{B}}}{
      \rfocus{\np{\octx}_1}{\p{A}} & \rfocus{\np{\octx}_2}{\p{B}}}
    \and
    \infer[\rrule{\one}]{\rfocus{\octxe}{\one}}{}
    \\
    \infer[\jrule{ID}\smash{^{\p{a}}}]{\rfocus{\p{a}}{\p{a}}}{}
    \and
    \infer[\rrule{\dn}]{\rfocus{\n{A}}{\dn \n{A}}}{}
  \end{inferences}
  \begin{inferences}
    \infer[\lrule{\limp}]{\lfocus{\np{\octx}_L \oc \np{\octx}_A}{\p{A} \limp \n{B}}{\np{\octx}_R}{\p{C}}}{
      \rfocus{\np{\octx}_A}{\p{A}} &
      \lfocus{\np{\octx}_L}{\n{B}}{\np{\octx}_R}{\p{C}}}
    \and
    \infer[\lrule{\pmir}]{\lfocus{\np{\octx}_L}{\n{B} \pmir \p{A}}{\np{\octx}_A \oc \np{\octx}_R}{\p{C}}}{
      \rfocus{\np{\octx}_A}{\p{A}} &
      \lfocus{\np{\octx}_L}{\n{B}}{\np{\octx}_R}{\p{C}}}
    \\
    \infer[\lrule{\with}_1]{\lfocus{\np{\octx}_L}{\n{A} \with \n{B}}{\np{\octx}_R}{\p{C}}}{
      \lfocus{\np{\octx}_L}{\n{A}}{\np{\octx}_R}{\p{C}}}
    \and
    \infer[\lrule{\with}_2]{\lfocus{\np{\octx}_L}{\n{A} \with \n{B}}{\np{\octx}_R}{\p{C}}}{
      \lfocus{\np{\octx}_L}{\n{B}}{\np{\octx}_R}{\p{C}}}
    \and
    \text{(no $\lrule{\top}$ rule)}
    \\
    \infer[\lrule{\up}]{\lfocus{}{\up \p{A}}{}{\p{A}}}{}
  \end{inferences}
  \begin{inferences}
    \infer[\jrule{$\reduces$I}]{\octx_L \oc \n{A} \oc \octx_R \reduces \octx'}{
      \lfocus{\octx_L}{\n{A}}{\octx_R}{\p{C}} &
      \rfocus{\octx'}{\p{C}}}
    \and
    \infer[\jrule{$\reduces$C}]{\octx_L \oc \octx \oc \octx_R \reduces \octx_L \oc \octx' \oc \octx_R}{
      \octx \reduces \octx'}
  \end{inferences}
  \begin{inferences}
    \infer[\jrule{$\Reduces$R}]{\octx \Reduces \octx}{}
    \and
    \infer[\jrule{$\Reduces$T}]{\octx \Reduces \octx''}{
      \octx \reduces \octx' & \octx' \Reduces \octx''}
  \end{inferences}
  \caption{The \acs*{FOR} framework for focused ordered rewriting}
\end{figure}

\clearpage

\section{Using shifts to control focusing}

With careful placement of shifts, it is possible to control the behavior of \acl{FOR} in \acs{FOR}.
It is even possible to embed unfocused ordered rewriting and weakly focused ordered rewriting within \ac{FOR} in an operationally faithful way, as we show in \cref{sec:ordered-rewriting:embed-unfocused,sec:ordered-rewriting:embed-weakly-focused}.
But first, we discuss a minimal polarization strategy for propositions.

\subsection{A minimal polarization strategy}

Because the unpolarized and polarized propositions share the same logical connectives and constants, there is an obvious polarization strategy:
Given an unpolarized proposition, insert an $\up$ in front of each positive proposition that occurs where a negative subformula is required; symmetrically, insert a $\dn$ in front of each negative proposition that occurs where a positive subformula is required.
%
For example, the unpolarized proposition $a \fuse \bigl((a \limp c \fuse a) \with (b \limp \one)\bigr)$ becomes $\p{a} \fuse \dn \bigl((\p{a} \limp \up (\p{c} \fuse \p{a})) \with (\p{b} \limp \up \one)\bigr)$ under the minimal polarization strategy.

In other words, the minimal polarization is one that adds $\up$ and $\dn$ shifts only as required.
We will frequently elide these shifts because they can be easily inferred.

% Focused ordered rewriting is sound with respect to the unfocused rewriting framework of \cref{??}, in the sense that every focused rewriting step possible
% The neighboring \lcnamecref{fig:ordered-rewriting:minimal-polarization}%
% %
% \begin{marginfigure}
%   \begin{equation*}
%     \begin{aligned}
%       \embedp{a} &= \p{a} \\
%       \embedp*{A \fuse B}
%         &= \embedp{A} \fuse \embedp{B} \\
%       \embedp{\one} &= \one \\
%       \embedp{A} &= \dn \embedn{A} \quad\text{otherwise}
%     \\[2\jot]
%       \embedn*{A \limp B}
%         &= \embedp{A} \limp \embedn{B} \\
%       \embedn*{B \pmir A}
%         &= \embedn{B} \pmir \embedp{A} \\
%       \embedn*{A \with B}
%         &= \embedn{A} \with \embedn{B} \\
%       \embedn{\top} &= \top \\
%       \embedn{A} &= \up \embedp{A} \quad\text{otherwise}
%     \\[2\jot]
%       \embed*{\octx_1 \oc \octx_2}
%         &= \embed{\octx_1} \oc \embed{\octx_2} \\
%       \embed*{\octxe} &= \octxe \\
%       \embed{A} &=
%         \begin{cases*}
%           \p{a} & if $A = a$ \\
%           \embedn{A} & otherwise
%         \end{cases*}
%     \end{aligned}
%   \end{equation*}
%   \caption{A minimal polarization strategy}\label{fig:ordered-rewriting:minimal-polarization}
% \end{marginfigure}%
% %



% \begin{equation*}
%   \begin{aligned}
%     \embedp*{A} &=
%       \begin{cases*}
%         \p{a} & if $A = a$ \\
%         \embedp{A_1} \fuse \embedp{A_2} & if $A = A_1 \fuse A_2$ \\
%         \one & if $A = \one$ \\
%         \dn \embedn{A} & otherwise
%       \end{cases*}
%     \\
%     \embedn*{A} &=
%       \begin{cases*}
%         \embedp{A_1} \limp \embedn{A_2} & if $A = A_1 \limp A_2$ \\
%         \embedn{A_2} \pmir \embedp{A_1} & if $A = A_2 \pmir A_1$ \\
%         \embedn{A_1} \with \embedn{A_2} & if $A = A_1 \with A_2$ \\
%         \top & if $A = \top$ \\
%         \up \embedp{A} & otherwise
%       \end{cases*}
%     \\
%     \embed{\octx} &=
%       \begin{cases*}
%         \embed{\octx_1} \oc \embed{\octx_2} & if $\octx = \octx_1 \oc \octx_2$ \\
%         \octxe & if $\octx = \octxe$ \\
%         \embedn{A} & if $\octx = A \neq a$ \\
%         \p{a} & if $\octx = a$
%       \end{cases*}
%   \end{aligned}
% \end{equation*}

% \begin{theorem}
%   If $\embed{\octx_1} \reduces \np{\octx}_2$, then $\octx_1 \Reduces \octx_2$ for some $\octx_2$ such that $\embed{\octx_2} = \np{\octx}_2$.
% \end{theorem}

\subsection{Embedding unfocused ordered rewriting}

With careful placement of additional, non-minimal shifts, it is possible to embed unfocused ordered rewriting within the focused ordered rewriting framework in a operationally faithful way.
Specifically, we can define a mapping, $\embed*{}$, from contexts of unpolarized propositions to contexts of negative propositions and positive atoms in a way that strongly respects the operational behavior of unfocused ordered rewriting:
\begin{itemize}[noitemsep]
\item $\octx \reduces \octx'$ implies $\embed{\octx} \reduces \embed{\octx'{}}$; and
\item $\embed{\octx} \reduces \np{\lctx'{}}$ implies $\octx \reduces \octx'$, for some $\octx'$ such that $\np{\lctx'{}} = \embed{\octx'{}}$.
\end{itemize}
That is, $\embed*{}$ will be a \emph{strong reduction bisimulation}\autocite{Sangiorgi+Walker:CUP03}.
% By appropriately placing double shifts, $\dn \up$ or $\up \dn$, between each pair of 

Essentially, this embedding inserts a double shift, $\dn \up$, in front of each proper, nonatomic subformula.
These double shifts cause chaining and inversion to be interrupted after each step, forcing the focused rewriting to mimic the small-step behavior of unfocused rewriting.

The mapping $\embed*{}$ relies on two auxiliary mappings:
% $\embed*{}$, from unpolarized propositions to negative propositions and positive atoms;
$\embedn*{}$ and $\embedp*{}$, from unpolarized propositions to negative and positive propositions, respectively.
$\embedn{A}$ and $\embedp{A}$ produce negative and positive polarizations of $A$ that insert a $\dn \up$ shift in front of every proper, nonatomic subformula of $A$.
In addition, $\embedn{A}$ prepends an $\up$ shift whenever the top-level connective of $A$ has positive polarity, whereas $\embedn{A}$ prepends a $\dn$ shift whenever $A$ is not atomic.
% $\embed{A}$ produces either a positive atom or a negative polarization of $A$, according to wheter $A$ is atomic.
%
% \begin{marginfigure}
% \begin{equation*}
%   \begin{aligned}
%     \embedp{A} &= \begin{cases*}
%                     \p{a} & if $A = a$ \\
%                     \dn \embedn{A} & otherwise
%                \end{cases*}
%     \\
%     \embedn{A} &= \begin{cases*}
%                     \up (\embedp{A_1} \fuse \embedp{A_2}) & if $A = A_1 \fuse A_2$ \\
%                     \up \one & if $A = \one$ \\
%                     \embedp{A_1} \limp \up \embedp{A_2} & if $A = A_1 \limp A_2$ \\
%                     \up \embedp{A_2} \pmir \embedp{A_1} & if $A = A_2 \pmir A_1$ \\
%                     \up \embedp{A_1} \with \up \embedp{A_2} & if $A = A_1 \with A_2$ \\
%                     \top & if $A = \top$
%                   \end{cases*}
%     \\
%     \embed{A} &= \begin{cases*}
%                    \p{a} & if $A = a$ \\
%                    \embedn{A} & otherwise
%                  \end{cases*}
%     \\
%     \embed{\octx} &= \begin{cases*}
%                        \embed{\octx_1} \oc \embed{\octx_2} & if $\octx = \octx_1 \oc \octx_2$ \\
%                        (\octxe) & if $\octx = \octxe$ \\
%                        \embed{A} & if $\octx = A$
%                      \end{cases*}
%   \end{aligned}
% \end{equation*}
% % \caption{An embedding of unfocused ordered rewriting within the focused ordered rewriting framework}
% \end{marginfigure}
\begin{marginfigure}
  \begin{equation*}
    \begin{aligned}
      \embedn*{A \fuse B} &= \up (\embedp{A} \fuse \embedp{B}) \\
      \embedn{\one} &= \up \one \\
      \embedn*{A \limp B} &= \embedp{A} \limp \up \embedp{B} \\
      \embedn*{B \pmir A} &= \up \embedp{B} \pmir \embedp{A} \\
      \embedn*{A \with B} &= \up \embedp{A} \with \up \embedp{B} \\
      \embedn{\top} &= \top
      \\
      \embedp{A} &=
        \begin{cases*}
          \p{a} & if $A = a$ \\
          \dn \embedn{A} & otherwise
        \end{cases*}
      \\
      \embed*{\octx_1 \oc \octx_2} &= \embed{\octx_1} \oc \embed{\octx_2} \\
      \embed*{\octxe} &= \octxe \\
      \embed{A} &=
        \begin{cases*}
          \p{a} & if $A = a$ \\
          \embedn{A} & otherwise
        \end{cases*}
    \end{aligned}
  \end{equation*}
  \caption{An embedding of unfocused ordered rewriting (\ie, \acs*{OR}) within \acs*{FOR}}
\end{marginfigure}

\begin{theorem}
  The embedding $\embed{(-)}$ satisfies the following properties.
  \begin{itemize}[nosep]
  \item If $\octx \reduces \octx'$, then $\embed{\octx} \reduces \embed{\octx'{}}$.
  \item If $\embed{\octx} = \np{\lctx} \reduces \np{\lctx'{}}$, then $\octx \reduces \octx'$ for some $\octx'$ such that $\np{\lctx'{}} = \embed{\octx'{}}$.
  \end{itemize}
\end{theorem}
\begin{proof}
  The proofs of these properties require a straightforward lemma:
  for all unpolarized propositions $A$, 
  \begin{equation*}
    \rfocus{\np{\lctx}}{\embedp{A}} \text{\ if, and only if, } \np{\lctx} = \embed{A}
    \,.
  \end{equation*}

  The first property is then proved by induction over the structure of the given rewriting step, $\octx \reduces \octx'$.
  As an example, consider the case in which $\octx = A \oc (A \limp B) \reduces B = \octx'$.
  By definition, $\embed{\octx} = \embed{A} \oc (\embedp{A} \limp \up \embedp{B})$ and $\embed{\octx'{}} = \embed{B}$, and we can indeed derive $\lfocus{\embed{A}}{\embedp{A} \limp \up \embedp{B}}{}{\embedp{B}}$ and $\rfocus{\embed{B}}{\embedp{B}}$.
  So, as required, $\embed{\octx} = \embed{A} \oc (\embedp{A} \limp \up \embedp{B}) \reduces \embed{B} = \embed{\octx'{}}$.

  The second property is also proved by induction over the structure of the given rewriting step, this time $\embed{\octx} = \np{\lctx} \reduces \np{\lctx'{}}$.
  As an example, consider the case in which $\lfocus{\embed{\octx_L}}{\embedp{A} \limp \up \embedp{B}}{\embed{\octx_R}}{\p{C}}$ and $\rfocus{\np{\lctx'{}}}{\p{C}}$, for some $\octx_L$, $A$, $B$, $\octx_R$, and $\p{C}$ such that $\octx = \octx_L \oc (A \limp B) \oc \octx_R$.
  By inversion and the aforementioned lemma, we have $\octx_L = A$, $\octx_R = \octxe$, $\p{C} = \embedp{B}$, and $\np{\lctx'{}} = \embed{B}$.
  Indeed, as required, $\octx = A \oc (A \limp B) \reduces B = \octx'$ and $\np{\lctx'{}} = \embed{\octx'{}}$.
\end{proof}

\subsection{Embedding weakly focused ordered rewriting}

It is similarly possible to embed weakly focused ordered rewriting, a rewriting discipline based on weak focusing\autocite{??} in which the granularity of steps lies between that of the unfocused and fully focused ordered rewriting frameworks.
More specifically, weak focusing differs from full focusing in that it retains chaining but abandons eager inversion.
For example, with weakly focused rewriting,
\begin{equation*}
  \p{a} \oc \dn \bigl((\p{a} \limp \up (\p{c} \fuse \p{a})) \with (\p{b} \limp \up \one)\bigr)
    \reduces \p{c} \fuse \p{a}
    \reduces \p{c} \oc \p{a}
  \,,
\end{equation*}
where the inversion of $\p{c} \fuse \p{a}$ is now an atomic step of its own.

This weakly focused rewriting discipline could be achieved as an independent system with the rules shown in \cref{??}.
Notice that weakly focused rewriting restricts the left- and right-handed implications to have only atomic premises.
Although weak focusing is well-defined for arbitrary implications\autocite{??}, it is not clear how to give a rewriting interpretation of weak focusing unless this restriction is made.

\begin{figure}
  \vspace*{\dimexpr-\abovedisplayskip-\abovecaptionskip\relax}
\begin{inferences}
  \infer[\jrule{$\dn$D}]{\p{\octx}_L \oc \dn \n{A} \oc \p{\octx}_R \reduces \p{C}}{
    \lfocus{\p{\octx}_L}{\n{A}}{\p{\octx}_R}{\p{C}}}
  \and
  \infer[\jrule{$\fuse$D}]{\p{A} \fuse \p{B} \reduces \p{A} \oc \p{B}}{}
  \and
  \infer[\jrule{$\one$D}]{\one \reduces \octxe}{}
  \\
  \infer[\jrule{$\reduces$C}]{\p{\octx}_L \oc \p{\octx} \oc \p{\octx}_R \reduces \p{\octx}_L \oc \p{\octx'{}} \oc \p{\octx}_R}{
    \p{\octx} \reduces \p{\octx'{}}}
\end{inferences}
\begin{inferences}
  \infer[\lrule{\limp}]{\lfocus{\p{\octx}_L \oc \p{a}}{\p{a} \limp \n{B}}{\p{\octx}_R}{\p{C}}}{
    \lfocus{\p{\octx}_L}{\n{B}}{\p{\octx}_R}{\p{C}}}
  \and
  \infer[\lrule{\pmir}]{\lfocus{\p{\octx}_L}{\n{B} \pmir \p{a}}{\p{a} \oc \p{\octx}_R}{\p{C}}}{
    \lfocus{\p{\octx}_L}{\n{B}}{\p{\octx}_R}{\p{C}}} 
  \\
  \infer[\lrule{\with}_1]{\lfocus{\p{\octx}_L}{\n{A} \with \n{B}}{\p{\octx}_R}{\p{C}}}{
    \lfocus{\p{\octx}_L}{\n{A}}{\p{\octx}_R}{\p{C}}}
  \and
  \infer[\lrule{\with}_2]{\lfocus{\p{\octx}_L}{\n{A} \with \n{B}}{\p{\octx}_R}{\p{C}}}{
    \lfocus{\p{\octx}_L}{\n{B}}{\p{\octx}_R}{\p{C}}}
  \and
  \text{(no $\lrule{\top}$ rule)}
  \\
  \infer[\lrule{\up}]{\lfocus{}{\up \p{A}}{}{\p{A}}}{}
\end{inferences}
\caption{A framework for \emph{weakly} focused ordered rewriting}
\end{figure}

In fact, there is a better approach than using weakly focused ordered rewriting as yet another independent rewriting system.
Instead of using weakly focused rewriting directly, we can embed it within \ac{FOR} by inserting shifts at specific locations and then use the embedding.
From here on, we will exclusively use this embedding when weakly focused ordered rewriting is needed.%
\begin{marginfigure}
\begin{equation*}
  \begin{aligned}
    \embedp*{\p{A}} &=
      \begin{cases*}
        \p{a} & if $\p{A} = \p{a}$ \\
        \dn \embed*{\p{A}} & otherwise
      \end{cases*}
  \\
    \embedn*{\p{a} \limp \n{B}}
      &= \p{a} \limp \embedn*{\n{B}} \\
    \embedn*{\n{B} \pmir \p{a}}
      &= \embedn*{\n{B}} \pmir \p{a} \\
    \embedn*{\n{A} \with \n{B}}
      &= \embedn*{\n{A}} \with \embedn*{\n{B}} \\
    \embedn{\top} &= \top \\
    \embedn*{\up \p{A}} &= \up \embedp*{\p{A}}
  \\
    \embed*{\p{\octx}_1 \oc \p{\octx}_2}
      &= \embed*{\p{\octx}_1} \oc \embed*{\p{\octx}_2} \\
    \embed*{\octxe} &= \octxe \\
    \embed*{\p{a}} &= \p{a} \\
    \embed*{\p{A} \fuse \p{B}} &= \up (\embedp*{\p{A}} \fuse \embedp*{\p{B}}) \\
    \embed{\one} &= \up \one \\
    \embed*{\dn \n{A}} &= \embedn*{\n{A}}
  \end{aligned}
\end{equation*}
  \caption{An embedding of weakly focused ordered rewriting (\ie, \acs*{OR}) within \acs*{FOR}}
\end{marginfigure}

% \begin{equation*}
%   \begin{aligned}
%     \embedp*{\p{A}} &= \begin{cases*}
%                     \p{a} & if $\p{A} = \p{a}$ \\
%                     \dn \up (\embedp*{\p{A}_1} \fuse \embedp*{\p{A}_2}) & if $\p{A} = \p{A}_1 \fuse \p{A}_2$ \\
%                     \dn \up \one & if $\p{A} = \one$ \\
%                     \dn \embedn*{\n{A}_0} & if $\p{A} = \dn \n{A}_0$
%                \end{cases*}
%     \\
%     \embedn*{\n{A}} &= \begin{cases*}
%                          \embedp*{\p{A}_1} \limp \embedn*{\n{A}_2} & if $\n{A} = \p{A}_1 \limp \n{A}_2$ \\
%                     \embedn*{\n{A}_2} \pmir \embedp*{\p{A}_1} & if $\n{A} = \n{A}_2 \pmir \p{A}_1$ \\
%                     \embedn*{\n{A}_1} \with \embedn*{\n{A}_2} & if $\n{A} = \n{A}_1 \with \n{A}_2$ \\
%                     \top & if $\n{A} = \top$ \\
%                     \up \embedp*{\p{A}_0} & if $\n{A} = \up \p{A}_0$
%                   \end{cases*}
%     \\
%     \embed*{\p{A}} &=
%       \begin{cases*}
%         \p{a} & if $\p{A} = \p{a}$ \\
%         \up (\embedp*{\p{A}_1} \fuse \embedp*{\p{A}_2}) & if $\p{A} = \p{A}_1 \fuse \p{A}_2$ \\
%         \up \one & if $\p{A} = \one$ \\
%         \embedn*{\n{A}_0} & if $\p{A} = \dn \n{A}_0$
%       \end{cases*}
%     \\
%     \embed*{\p{\octx}} &=
%       \begin{cases*}
%         \embed*{\p{\octx}_1} \oc \embed*{\p{\octx}_2} & if $\p{\octx} = \p{\octx}_1 \oc \p{\octx}_2$ \\
%         (\octxe) & if $\octx = \octxe$ \\
%         \embed*{\p{A}} & if $\octx = \p{A}$
%       \end{cases*}
%   \end{aligned}
% \end{equation*}

\begin{theorem}
  The embedding $\embed{(-)}$ satisfies the following properties.
  \begin{itemize}[nosep]
  \item If $\p{\octx} \reduces \p{\octx'{}}$, then $\embed*{\p{\octx}} \reduces \embed*{\p{\octx'{}}}$.
  \item If $\embed*{\p{\octx}} \reduces \np{\lctx'{}}$, then $\p{\octx} \reduces \p{\octx'{}}$ for some $\p{\octx'{}}$ such that $\np{\lctx'{}} = \embed*{\p{\octx'{}}}$.
  \end{itemize}
\end{theorem}
\begin{proof}
  The proofs of these properties require two relatively straightforward lemmas:
  for all polarized propositions $\p{A}$ and $\n{A}$,
  \begin{itemize}
  \item $\rfocus{\np{\lctx}}{\embedp*{\p{A}}}$ if, and only if, $\np{\lctx} = \embed*{\p{A}}$; and
  \item $\lfocus{\np{\lctx}_L}{\embedn*{\n{A}}}{\np{\lctx}_R}{\p{B}}$ if, and only if,
    $\lfocus{\p{\octx}_L}{\n{A}}{\p{\octx}_R}{\p{C}}$ and
    $\np{\lctx}_L = \embed*{\p{\octx}_L}$,
    $\np{\lctx}_R = \embed*{\p{\octx}_R}$, and
    $\p{B} = \embedp*{\p{C}}$.
  \end{itemize}
  Both lemmas are proved by structural induction on the polarized proposition, $\p{A}$ and $\n{A}$, respectively.

  The first of the above properties is then proved by induction over the structure of the given weakly focused rewriting step, $\p{\octx} \reduces \p{\octx'}$.
  As an example, consider the case in which $\p{\octx}_L \oc \dn \n{A} \oc \p{\octx}_R \reduces \p{C}$ because $\lfocus{\p{\octx}_L}{\n{A}}{\p{\octx}_R}{\p{C}}$.
  By the above lemma, $\lfocus{\embed*{\p{\octx}_L}}{\embedn*{\n{A}}}{\embed*{\p{\octx}_R}}{\embedp*{\p{C}}}$ holds in the fully focused calculus.
  And so, as required, $\embed*{\p{\octx}_L} \oc \embed*{\dn \n{A}} \oc \embed*{\p{\octx}_R} \reduces \embed*{\p{C}}$.

  The second property is also proved by induction over the structure of the given rewriting step, this time the fully focused $\embed*{\p{\octx}} \reduces \np{\lctx'{}}$.
  As an example, consider the case in which $\lfocus{\np{\lctx}_L}{\p{a}_1 \limp \embedn*{\n{A}_2}}{\np{\lctx}_R}{\p{B}}$ and $\rfocus{\np{\lctx'{}}}{\p{B}}$.
  Inversion yields $\lfocus{\np{\lctx'_L{}}}{\embedn*{\n{A}_2}}{\np{\lctx}_R}{\p{B}}$ for some $\np{\lctx'_L{}}$ such that $\np{\lctx}_L = \np{\lctx'_L{}} \oc \p{a}_1$.
  Then, by the above lemma, $\lfocus{\p{\octx}_L}{\n{A}_2}{\p{\octx}_R}{\p{C}}$ holds in the weakly focused calculus, with $\np{\lctx'_L{}} = \embed*{\p{\octx}_L}$, $\np{\lctx}_R = \embed*{\p{\octx}_R}$, and $\p{B} = \embedp*{\p{C}}$.
  It follows that $\lfocus{\p{\octx}_L \oc \p{a}_1}{\p{a}_1 \limp \n{A}_2}{\p{\octx}_R}{\p{C}}$, and so $\p{\octx}_L \oc \p{a}_1 \oc \dn (\p{a}_1 \limp \n{A}_2) \oc \p{\octx}_R \reduces \p{C}$.
  Also notice that $\np{\lctx}_L = \embed*{\p{\octx}_L \oc \p{a}_1}$ and $\np{\lctx'{}} = \embed*{\p{C}}$, as required.
\end{proof}


\clearpage
\clearpage

\chapter{MOVE THESE}

\section{Choreographies}

Recall the string rewriting specification
\begin{equation*}
  \infer{a \oc b \reduces b}{}
  \qquad\text{and}\qquad
  \infer{b \reduces \emp}{}
  \:.
\end{equation*}

A choreography is a refinement of this specification in which each symbol $a$ of the rewriting alphabet is mapped to an ordered proposition: either an atomic proposition, $\atmL{a}$ or $\atmR{a}$, or a recursively defined proposition, $\proc{a}$.
In other words, a choreography is an injection from symbols to propositions.
\begin{equation*}
  \begin{tikzcd}
    w \rar[reduces] \dar[dash] & w' \dar[dash, exists]
    \\
    \mathllap{\theta}(w) \rar[Reduces, exists] & \theta(w')
  \end{tikzcd}
  \begin{tikzcd}
    \mathllap{\theta}(w) \rar[reduces] \dar[dash] & \octx' \rar[Reduces, exists] & \theta(w') \dar[dash, exists]
    \\
    w \arrow[reduces, exists]{rr} && w'
  \end{tikzcd}
\end{equation*}

$\atmR{a} \oc \proc{b}$

Suppose that $\theta$ is the mapping $a \mapsto \atmR{a}$ and $b \mapsto \proc{b}$.
and the choreography
\begin{equation*}
  \proc{b} \defd (\atmR{a} \limp \proc{b}) \with \one
  \,.
\end{equation*}
Notice that 
\begin{alignat*}{2}
  &a \oc b \reduces b
  &&\quad\text{and}\quad
  b \reduces \emp
\shortintertext{as well as}
  &\atmR{a} \oc \proc{b} \reduces \atmR{a} \oc (\atmR{a} \limp \proc{b}) \reduces \proc{b}
  &&\quad\text{and}\quad
  \proc{b} \reduces \one \reduces \octxe
  \:.
\end{alignat*}


\begin{equation*}
  \infer{a \oc b \reduces b}{}
  \qquad\text{and}\qquad
  \infer{c \oc b \reduces b}{}
\end{equation*}

\begin{equation*}
  \proc{b} \defd (\atmR{a} \limp \proc{b}) \with (\atmR{c} \limp \proc{b})
\end{equation*}

\begin{equation*}
  a \oc b \reduces w' \text{ implies $w' = b$}
  \quad\text{but}\quad
  \atmR{a} \oc \proc{b} \reduces \atmR{a} \oc (\atmR{c} \limp \proc{b}) \nreduces
\end{equation*}

\autocite{McDowell+:TCS03}


Judgments $\chorsig{\theta}{\sig}{\sig'}$ and $\chorax{\theta}{w \reduces w'}{\proc{a}}{A}$.
In both judgments, all terms before the $\chorarrow$ are inputs; all terms after the $\chorarrow$ are outputs.


\begin{inferences}
  \infer{\chorsig{\theta}{\sige}{\sige}}{}
  \and
  \infer{\chorsig{\theta}{\sig, w \reduces w'}{\sig', \proc{a} \defd A_1 \with A_2}}{
    \chorsig{\theta}{\sig}{\sig'} &
    \chorax{\theta}{w \reduces w'}{\proc{a}}{A_2} &
    \text{($\sig'(\proc{a}) = A_1$)}}
  \\
  \infer{\chorsig{\theta}{\sig, w \reduces w'}{\sig', \proc{a} \defd A}}{
    \chorsig{\theta}{\sig}{\sig'} &
    \chorax{\theta}{w \reduces w'}{\proc{a}}{A} &
    \text{($\proc{a} \notin \dom{\sig'}$)}}
  \\
  \infer{\chorax{\theta}{a \reduces w'}{\proc{a}}{\up (\bigfuse \octx')}}{
    \text{($\theta(a) = \proc{a}$)} &
    \text{($\theta(w') = \octx'$)}}
  \\
  \infer{\chorax{\theta}{b \oc w \reduces w'}{\proc{a}}{\atmR{b} \limp A}}{
    \chorax{\theta}{w \reduces w'}{\proc{a}}{A} &
    \text{($\theta(b) = \atmR{b}$)}}
  \and
  \infer{\chorax{\theta}{w \oc b \reduces w'}{\proc{a}}{A \pmir \atmL{b}}}{
    \chorax{\theta}{w \reduces w'}{\proc{a}}{A} &
    \text{($\theta(b) = \atmL{b}$)}}
\end{inferences}

\begin{theorem}
  \begin{itemize}
  \item If $\chorsig{\theta}{\sig}{\sig'}$ and $w \reduces_{\sig} w'$, then $\theta(w) \reduces_{\sig'} \theta(w')$.
    If $\chorsig{\theta}{\sig}{\sig'}$ and $\octx \reduces_{\sig'} \octx'$, then $\theta^{-1}(\octx) \reduces_{\sig} \theta^{-1}(\octx')$.
  \item If $\chorax{\theta}{w \reduces w'}{\proc{a}}{A}$, then $\theta(w) \reduces_{\proc{a} \defd A} \theta(w')$.
    If $\chorax{\theta}{w \reduces w'}{\proc{a}}{A}$ and $\octx \reduces_{\proc{a} \defd A} \octx'$, then $\theta^{-1}(\octx) \reduces \theta^{-1}(\octx')$.
  \end{itemize}
\end{theorem}
\begin{proof}
  $\proc{a} \reduces \bigfuse \theta(w')$

  $\atmR{b} \oc \theta(w) \reduces_{\proc{a} \defd \atmR{b} \limp A} \theta(w')$ if $\theta(w) \reduces_{\proc{a} \defd A} \theta(w')$

  $\theta(w) \oc \atmL{b} \reduces_{\proc{a} \defd A \pmir \atmL{b}} \theta(w')$ if $\theta(w) \reduces_{\proc{a} \defd A} \theta(w')$


  
\end{proof}

When $\theta = \Set{(a, \atmR{a}), (b, \proc{b})}$, the judgment $\chorsig{\theta}{\sig}{\proc{b} \defd (\atmR{a} \limp \proc{b}) \with \one}$ holds.
However, $b \reduces \emp$ but $\proc{b} \nreduces \octxe$.



\begin{equation*}
  \begin{lgathered}
    \proc{e} \defd \up (\dn \proc{e} \fuse \dn \proc{b}_1) \pmir \atmL{i} \\
    \proc{b}_0 \defd \up \dn \proc{b}_1 \pmir \atmL{i} \\
    \proc{b}_1 \defd \up (\atmL{i} \fuse \dn \proc{b}_0) \pmir \atmL{i}
  \end{lgathered}
\end{equation*}

\begin{equation*}
  \begin{lgathered}
    \proc{e} \defd \bigl(\up (\dn \proc{e} \fuse \dn \proc{b}_1) \pmir \atmL{i}\bigr) \with (\up \atmR{z} \pmir \atmL{d})
  \end{lgathered}
\end{equation*}

\begin{equation*}
  \begin{lgathered}
    \proc{\imath} \defd (\atmR{e} \limp \up (\atmR{e} \fuse \atmR{b}_1)) \with (\atmR{b}_0 \limp \up \atmR{b}_1) \with (\atmR{b}_1 \limp \up (\dn \proc{\imath} \fuse \atmR{b}_0))
  \end{lgathered}
\end{equation*}



\subsection{}



\section{}

Atomic ordered propositions are viewed as messages; compound ordered propositions, as processes; and ordered contexts, as configurations of processes.

The ordered contexts form a monoid over the positive propositions and are given by
\begin{equation*}
  \octx \Coloneqq \octx_1 \oc \octx_2 \mid \octxe \mid \p{A}
  \,.
\end{equation*}
In keeping with the monoid laws, we treat $(\octx_1 \oc \octx_2) \oc \octx_3$ and $\octx_1 \oc (\octx_2 \oc \octx_3)$ as syntactically indistinguishable, as we also do for $\octx \oc (\octxe)$ and $\octx$ and $(\octxe) \oc \octx$.

Each atom is consistently assigned a direction, either left-directed, $\atmL{a}$, or right-directed, $\atmR{a}$.

An atom's direction and position within the larger context together indicate whether, when viewed as a message, it is being sent or received.
In the context $\octx_1 \oc \atmR{a} \oc \octx_2$, the atom $\atmR{a}$ is a message being sent from $\octx_1$ to $\octx_2$.
Symmetrically, in the context $\octx_1 \oc \atmL{a} \oc \octx_2$, the atom $\atmL{a}$ is a message being sent from $\octx_1$ to $\octx_2$.

The context $\octx = \octx' \oc \atmR{a}$ is a process configuration that sends $\atmR{a}$ to its right and continues as $\octx'$.
Conversely, $\atmR{a} \oc \octx$ is a process configuration in which $\octx$ is the intended recipient of the message $\atmR{a}$.


\begin{align*}
  \p{A} &\Coloneqq \atmL{a} \mid \atmR{a} \mid \p{\hat{p}} \mid \p{A} \fuse \p{B} \mid \one \mid \dn \n{A} \\
  \n{A} &\Coloneqq \n{\hat{p}} \mid \atmR{a} \limp \n{B} \mid \n{B} \pmir \atmL{a} \mid \n{A} \with \n{B} \mid \top \mid \up \p{A}
\end{align*}

\section{Choreographing specifications}

\begin{equation*}
  \infer{a \oc b \reduces b}{}
  \qquad\text{and}\qquad
  \infer{b \reduces \octxe}{}
\end{equation*}

As a specification, these string rewriting axioms are quite reasonable.
However, as a [...], [...].

Toward our ultimate goal of relating the proof-construction and proof-reduction appraches to concurrency, we would like a description of this concurrent system that is slightly more concrete.

\begin{align*}
  \atmR{a} \oc \hat{b} &\reduces \hat{b} \\
  \hat{b} &\reduces \octxe
\end{align*}

\begin{equation*}
  \hat{b} \defd (\atmR{a} \limp \up \dn \hat{b}) \with \one
\end{equation*}


\begin{inferences}
  \infer{e \oc i \reduces e \oc b_1}{}
  \and
  \infer{b_0 \oc i \reduces b_1}{}
  \and\text{and}\and
  \infer{b_1 \oc i \reduces i \oc b_0}{}
\end{inferences}

\begin{equation*}
  \begin{lgathered}
    \bin{e} \defd \bin{e} \fuse \bin{b}_1 \pmir \atmL{i} \\
    \bin{b}_0 \defd \bin{b}_1 \pmir \atmL{i} \\
    \bin{b}_1 \defd \atmL{i} \fuse \bin{b}_0 \pmir \atmL{i}
  \end{lgathered}
\end{equation*}

\begin{equation*}
  \begin{lgathered}
    e \simu{R} \bin{e} \\
    b_0 \simu{R} \bin{b}_0 \\
    b_1 \simu{R} \bin{b}_1 \\
    i \simu{R} \atmL{i} \\
    e \oc b_1 \simu{R} \bin{e} \fuse \bin{b}_1 \\
    i \oc b_0 \simu{R} \atmL{i} \fuse \bin{b}_0
  \end{lgathered}
\end{equation*}
$\simu{R}$ is a reduction bisimulation.

$\bin{e} \oc \atmL{i} \Reduces \bin{e} \oc \bin{b}_1$
and $e \oc i \Reduces e \oc b_1$
and $e \oc b_1 \Reduces e \oc b_1$

\begin{equation*}
  \begin{lgathered}
    \bin{\imath} \defd (\atmR{e} \limp \atmR{e} \fuse \atmR{b}_1)
               \with (\atmR{b}_0 \limp \atmR{b}_1)
               \with (\atmR{b}_1 \limp \bin{\imath} \fuse \atmR{b}_0)
  \end{lgathered}
\end{equation*}

$\atmR{e} \oc \bin{\imath} \Reduces \atmR{e} \oc \atmR{b}_1$
and $e \oc i \Reduces e \oc b_1$
and $e \oc b_1 \Reduces e \oc b_1$


\begin{equation*}
  \dfa{q} \defd \bigwith_{a \in \ialph}(\atmR{a} \limp \dfa{q}'_a)
\end{equation*}

Compare:
\begin{itemize}
\item $q \dfareduces[a] q'_a$ if, and only if, $\atmR{a} \oc \dfa{q} \reduces \dfa{q}'_a$.
\item $q \dfareduces[a] q'_a$ if, and only if, $\atmR{a} \oc \dfa{q} \reduces \octx'$ for some $\octx' = \dfa{q}'_a$.
\item $q \dfareduces[a] q'_a$ and $\dfa{q}'_a = \octx'$ for some $q'_a$ if, and only if, $\atmR{a} \oc \dfa{q} \reduces \octx'$.
\end{itemize}
These differ in the placement of the existential quantifier.
The first pair are, in fact, false.

\emph{For the former:}
Assume that $q \dfareduces[a] q''_a$ and $\dfa{q}''_a = \octx' = \dfa{q}'_a$.
It might be that the states $q''_a$ and $q'_a$ are only bisimilar, not equal.
In that case, $q \dfareduces[a]\asim q'_a$ but, in general, not $q \dfareduces[a] q'_a$ directly.

\emph{For the latter:}
Assume that $q \dfareduces[a] q''_a$ and $\dfa{q}''_a = \octx'$.
Choosing $q'_a \coloneqq q''_a$, we indeed have $q \dfareduces[a] q'_a$ and $\dfa{q}'_a = \octx'$. 

\begin{equation*}
\begin{tikzcd}
  q \rar[reduces, "a", exists] \dar[dash, "\simu{R}"'] & q'_a \dar[dash, "\simu{R}"]
  \\
  \atmR{a} \oc \dfa{q} \rar[reduces] & \octx' \mathrlap{{} = \dfa{q}'_a}
\end{tikzcd}
\qquad\qquad
\begin{tikzcd}
  q \rar[reduces, "a", exists] \dar[dash, "\simu{R}"'] & q'_a \dar[dash, "\simu{R}", exists]
  \\
  \atmR{a} \oc \dfa{q} \rar[reduces] & \octx' \mathrlap{{} = \dfa{q}'_a}
\end{tikzcd}
\end{equation*}



\section{Encoding \aclp*{DFA}}

Recall from \cref{??} our string rewriting specification of how \iac{NFA} processes its input.
Given \iac{DFA} $\aut{A} = (Q, ?, F)$ over an input alphabet $\ialph$, the \ac{NFA}'s operational semantics are adequately captured by the folllwing string rewriting axioms:
\begin{equation*}
  \infer{a \oc q \reduces q'_a}{}
  \enspace\text{for each transition $q \nfareduces[a] q'_a$.}
\end{equation*}
\begin{equation*}
  \infer{\emp \oc q \reduces F(q)}{}
  \enspace\text{for each state $q$, where}\enspace
  F(q) = \begin{cases*}
           (\octxe) & if $q \in F$ \\
           \symrej & if $q \notin F$\,.
         \end{cases*}
\end{equation*}

\clearpage
\subsection{A functional choreography}

One possible choreography for this specification treats the input symbols $a \in \ialph$ as atomic propositions $\atmR{a}$; states $q \in Q$ as recursively defined propostions $\proc{q}$;and the end-of-word marker $\emp$ as an atomic proposition $\atmR{\emp}$.
In other words, the \ac{NFA}'s input is treated as a sequence of messages, $\atmR{\emp} \oc \atmR{a}_n \dotsm \atmR{a}_2 \oc \atmR{a}_1$, and the \ac{NFA}'s states are treated as [recursive] processes.

$a \mapsto \atmR{a}$ for all $a \in \ialph$; $q \mapsto \proc{q}$ for all $q \in Q$; and $\emp \mapsto \atmR{\emp}$.

Using this assignment, the choreography constructed from the specification consists of the following definition, one for each \ac{NFA} state $q \in Q$:
\begin{equation*}
  \proc{q} \defd \bigwith_{a \in \ialph} \bigwith_{q\smash{'_a}} (\atmR{a} \limp \proc{q}'_a) \with (\atmR{\emp} \limp \nfa{F}(q))
  \,.
\end{equation*}

\begin{corollary}
  If $a \oc q \reduces q'_a$, then $\atmR{a} \oc \proc{q} \Reduces \proc{q}'_a$.
  If $\atmR{a} \oc \proc{q} \reduces \octx'$, then $a \oc q \reduces w'$ and $\octx' \Reduces \theta(w')$.
\end{corollary}

\begin{corollary}
  If $q \nfareduces[a] q'_a$, then $\atmR{a} \oc \proc{q} \Reduces \proc{q}'_a$.
  If $\atmR{a} \oc \proc{q} \reduces \proc{q}'_a$, then $q \nfareduces[a] q''_a$ for some $q''_a$ such that $\proc{q}'_a = \proc{q}''_a$.
\end{corollary}

As an extended example, we will use ordered rewriting to specify how \iac{DFA} processes its input.
%
% \Acp{DFA} serve as an example of ordered rewriting,  can be used to specify how \iac{DFA} processes its input.
%
Given \iac{DFA} $\aut{A} = (Q, ?, F)$ over an input alphabet $\ialph$, the idea is to encode each state, $q \in Q$, as an ordered proposition, $\dfa{q}$, in such a way that the \ac{DFA}'s operational semantics are adequately captured by [ordered] rewriting.
%
% The basic idea is to define an encoding, $\dfa{q}$, of \ac{DFA} states as ordered propositions;
% this encoding should adequately reflect the \ac{DFA}'s operational semantics with ordered rewriting traces.
\fixnote{[In general, the behavior of \iac{DFA} state is recursive, so the proposition $\dfa{q}$ will be recursively defined.]}
%
% finite input words, $w \in \finwds{\ialph}$, are encoded as ordered contexts by $\emp \oc \rev{w}$

% \NewDocumentCommand \rev { s m } {
%   \IfBooleanTF {#1}
%     { (#2)^{\mathsf{R}} }
%     { #2^{\mathsf{R}} }
% }

% \begin{align*}
%   \rev{a} &= a \\
%   \rev*{w_1 \wc w_2} &= \rev{w_2} \oc \rev{w_1} \\
%   \rev{\emp} &= \octxe
% \end{align*}

Ideally, \ac{DFA} transitions $q \dfareduces[a] q'_a$ would be in bijective correspondence with rewriting steps $a \oc \dfa{q} \reduces \dfa{q}'_a$, where each input symbol $a$ is encoded by a matching [propositional] atom.
%
We will return to the possibility of this kind of tight correspondence in \cref{??}, but,
%
for now, we will content ourselves with a correspondence with traces rather than individual steps, adopting the following desiderata:
% Unfortunately, ordered rewriting's small step size turns out to be a poor match for [...], so in both cases we will instead content ourselves with corrspondances with \emph{traces}:
% a bijection between transitions $q \dfareduces[a] q'_a$ and \emph{traces} $a \oc \dfa{q} \Reduces \dfa{q}'_a$.
% Similarly, [...] a bijection between accepting states $q \in F$ and traces $\emp \oc \dfa{q} \Reduces \octxe$.
%
% This leads us to adopt the following as desiderata:
\begin{itemize}
\item
  $q \dfareduces[a] q'_a$ if, and only if, $a \oc \dfa{q} \Reduces \dfa{q}'_a$, for all input symbols $a \in \ialph$.
\item
  $q \in F$ if, and only if, $\emp \oc \dfa{q} \Reduces \one$, where the atom $\emp$ functions as an end-of-word marker.
% \item
%   $q \dfareduces[w] q'_w \in F$ if, and only if, $\emp \oc \rev{w} \oc \dfa{q} \Reduces \octxe$.
%   Also, $q \dfareduces[w] q'_w \notin F$ if, and only if, $\emp \oc \rev{w} \oc \dfa{q} \Reduces \top$.
\end{itemize}
Given the reversal (anti-)\-homo\-morph\-ism from finite words to ordered contexts defined in the adjacent \lcnamecref{fig:ordered-rewriting:reversal}%
\begin{marginfigure}
  \begin{align*}
    \rev*{w_1 \wc w_2} &= \rev{w_2} \oc \rev{w_1} \\
    \rev{\emp} &= \octxe \\
    \rev{a} &= a
  \end{align*}
  \caption{An (anti-)\-homo\-morph\-ism for reversal of finite words to ordered contexts}\label{fig:ordered-rewriting:reversal}
\end{marginfigure}%
, the first desideratum is subsumed by a third:
% property that covers finite words:
\begin{itemize}[resume*]
\item $q \dfareduces[w] q'$ if, and only if, $\rev{w} \oc \dfa{q} \Reduces \dfa{q}'$, for all finite words $w \in \finwds{\ialph}$.
\end{itemize}

From these desiderata [and the observation that \acp{DFA}' graphs frequently%
\fixnote{Actually, there is always at least one cycle in a well-formed \ac{DFA}.}
contain cycles], we arrive at the following encoding, in which each state is encoded by one of a collection of mutually recursive definitions:%
\fixnote{$q'_a$, using function or relation?}
\begin{gather*}
  \dfa{q} \defd
    \parens[size=big]{
      \bigwith_{a \in \ialph}(a \limp \dfa{q}'_a)}
    \with
    \parens[size=big]{\emp \limp \dfa{F}(q)}
  % \text{where
  %   $q \dfareduces[a] q'_a$ for all $a \in \ialph$
  %   and
  %   $\dfa{F}(q) = 
  %     \begin{cases*}
  %       \one & if $q \in F$ \\
  %       \top & if $q \notin F$
  %     \end{cases*}$%
  % }
  %
\shortintertext{where}
  %
  q \dfareduces[a] q'_a
  \text{, for all input symbols $a \in \ialph$,\quad and\quad}
  \dfa{F}(q) = 
    \begin{cases*}
      \one & if $q \in F$ \\
      \top & if $q \notin F$%
    \,.
    \end{cases*}
\end{gather*}
Just as each state $q$ has exactly one successor for each input symbol $a$, its encoding, $\dfa{q}$, has exactly one input clause, $(a \limp \dotsb)$, for each symbol $a$.



% The traces $a \oc \dfa{q} \Reduces \dfa{q}'_a$
% % for input symbols $a \in \ialph$
% suggest that $\dfa{q}$ should be a collection of clauses that input atoms $a$ from the left.
% And the traces $\emp \oc \dfa{q} \Reduces \octxe$ or $\emp \oc \dfa{q} \Reduces \top$ suggest that $\dfa{q}$ also contain a clause that inputs atom $\emp$ from the left.
% Thus, we arrive at the encoding


\newthought{For a concrete instance} of this encoding, recall from \cref{ch:automata} the \ac{DFA} (repeated in the adjacent \lcnamecref{fig:ordered-rewriting:dfa-example-ends-b})%
%
\begin{marginfigure}
  \begin{equation*}
    \mathllap{\aut{A}_2 = {}}
    \begin{tikzpicture}[baseline=(q_0.base)]
      \graph [automaton] {
        q_0
         -> [loop above, "a"]
        q_0
         -> ["b", bend left]
        q_1 [accepting]
         -> [loop above, "b"]
        q_1
         -> ["a", bend left]
        q_0;
      };
    \end{tikzpicture}
  \end{equation*}
  \caption{\Iac*{DFA} that accepts, from state $q_0$, exactly those words that end with $b$. (Repeated from \cref{fig:dfa-example-ends-b}.)}\label{fig:ordered-rewriting:dfa-example-ends-b}
\end{marginfigure}
%
that accepts exactly those words, over the alphabet $\ialph = \set{a,b}$, that end with $b$; that \ac{DFA} is encoded by the following definitions:
\begin{equation*}
  \begin{lgathered}
    \dfa{q}_0 \defd (a \limp \dfa{q}_0) \with (b \limp \dfa{q}_1) \with (\emp \limp \top) \\
    \dfa{q}_1 \defd (a \limp \dfa{q}_0) \with (b \limp \dfa{q}_1) \with (\emp \limp \one)
  \end{lgathered}
\end{equation*}
Indeed, just as the \ac{DFA} has a transition $q_0 \dfareduces[b] q_1$, its encoding admits a trace
\begin{align*}
  &b \oc \dfa{q}_0
     = b \oc \bigl((a \limp \dfa{q}_0) \with (b \limp \dfa{q}_1) \with (\emp \limp \top)\bigr)
     \Reduces b \oc (b \limp \dfa{q}_1)
     \reduces \dfa{q}_1
  \,.
\intertext{And, just as $q_1$ is an accepting state, its encoding also admits a trace}
  &\emp \oc \dfa{q}_1 = \emp \oc \bigl((a \limp \dfa{q}_0) \with (b \limp \dfa{q}_1) \with (\emp \limp \one)\bigr) \Reduces \emp \oc (\emp \limp \one) \reduces \one
  \,.
\end{align*}

\newthought{More generally}, this encoding is complete, in the sense that it simulates all \ac{DFA} transitions: $q \dfareduces[a] q'$ implies $a \oc \dfa{q} \Reduces \dfa{q}'$, for all states $q$ and $q'$ and input symbols $a$.

However, the converse does not hold -- the encoding is unsound because there are rewritings that cannot be simulated by \iac{DFA} transition.
% That is, $a \oc \dfa{q} \Reduces \dfa{q}'$ does \emph{not} imply $q \dfareduces[a] q'$.
% 
\begin{falseclaim}
  Let $\aut{A} = (Q, \mathord{\dfareduces}, F)$ be \iac{DFA} over the input alphabet $\ialph$.
  Then $a \oc \dfa{q} \Reduces \dfa{q}'$ implies $q \dfareduces[a] q'$, for all input symbols $a \in \ialph$.
\end{falseclaim}
%
\begin{marginfigure}
    \centering
    % \subfloat[][]{\label{fig:ordered-rewriting:dfa-counterexample:dfa}%
      \begin{equation*}
        \aut{A}'_2 = 
      \begin{tikzpicture}[baseline=(q_0.base)]
        \graph [automaton] {
          q_0
           -> [loop above, "a"]
          q_0
           -> ["b", bend left]
          q_1 [accepting]
           -> [loop above, "b"]
          q_1
           -> ["a", bend left]
          q_0;
          %
%          { [chain shift={(2,0)}]
            s_1 [accepting, below=1.5em of q_1.south]
             -> [loop right, "b"]
            s_1
             -> ["a", bend left]
            q_0;
%          };
        };
      \end{tikzpicture}
    \end{equation*}
    % }
    % \subfloat[][]{\label{fig:ordered-rewriting:dfa-counterexample:encoding}%
      $\!\begin{aligned}
        \dfa{q}_0 &\defd (a \limp \dfa{q}_0) \with (b \limp \dfa{q}_1) \with (\emp \limp \top) \\
        \dfa{q}_1 &\defd (a \limp \dfa{q}_0) \with (b \limp \dfa{q}_1) \with (\emp \limp \one) \\
        \dfa{s}_1 &\defd (a \limp \dfa{q}_0) \with (b \limp \dfa{s}_1) \with (\emp \limp \one)
      \end{aligned}$%
    % }
    \caption{{fig:ordered-rewriting:dfa-counterexample:dfa}~A slightly modified version of the \ac*{DFA} from \cref{fig:ordered-rewriting:dfa-example-ends-b}; and {fig:ordered-rewriting:dfa-counterexample:encoding}~its encoding}\label{fig:ordered-rewriting:dfa-counterexample}
  \end{marginfigure}%
\begin{proof}[Counterexample]
  Consider the \ac{DFA} and encoding shown in the adjacent \lcnamecref{fig:ordered-rewriting:dfa-counterexample}; it is the same \ac{DFA} as shown in \cref{fig:ordered-rewriting:dfa-example-ends-b}, but with one added state, $s_1$, that is unreachable from $q_0$ and $q_1$.
    %
  % When encoded as an ordered rewriting specification, it corresponds to the following definitions:
  % \begin{equation*}
  %   \begin{lgathered}
  %     \dfa{q}_0 \defd (a \limp \dfa{q}_0) \with (b \limp \dfa{q}_1) \with (\emp \limp \top) \\
  %     \dfa{q}_1 \defd (a \limp \dfa{q}_0) \with (b \limp \dfa{q}_1) \with (\emp \limp \one) \\
  %     \dfa{s}_1 \defd (a \limp \dfa{q}_0) \with (b \limp \dfa{s}_1) \with (\emp \limp \one)
  %   \end{lgathered}
  % \end{equation*}
  Notice that, as a coinductive consequence of the equirecursive treatment of definitions, $\dfa{q}_1 = \dfa{s}_1$.
  Previously, we saw that $b \oc \dfa{q}_0 \Reduces \dfa{q}_1$; hence $b \oc \dfa{q}_0 \Reduces \dfa{s}_1$.
  However, the \ac{DFA} has no $q_0 \dfareduces[b] s_1$ transition (because $q_1 \neq s_1$), and so this encoding is unsound with respect to the operational semantics of \acp{DFA}.
\end{proof}

As this counterexample shows, the lack of adequacy stems from attempting to use an encoding that is not injective -- here, $q_1 \neq s_1$ even though $\dfa{q}_1 = \dfa{s}_1$.
In other words, eqality of state encodings is a coarser eqvivalence than equality of the states themselves.

One possible remedy for this lack of adequacy might be to revise the encoding to have a stronger nominal character.
By tagging each state's encoding with an atom that is unique to that state, we can make the encoding manifestly injective.
For instance, given the pairwise distinct atoms $\Set{q \given q \in F}$ and $\Set{\bar{q} \given q \in Q - F}$ to tag final and non-final states, respectively, we could define an alternative encoding, $\check{q}$:
%
\begin{gather*}
  \check{q} \defd
    \parens[size=big]{
      \bigwith_{a \in \ialph}(a \limp \check{q}'_a)}
    \with
    \parens[size=big]{\emp \limp \check{F}(q)}
  %
  \shortintertext{where}
  %
  q \dfareduces[a] q'_a
  \text{, for all input symbols $a \in \ialph$,\quad and\quad}
  \check{F}(q) =
    \begin{cases*}
      q & if $q \in F$ \\
      \bar{q} & if $q \notin F$%
    \,.
    \end{cases*}
\end{gather*}
%
Under this alternative encoding, the states $q_1$ and $s_1$ of \cref{fig:ordered-rewriting:dfa-counterexample} are no longer a counterexample to injectivity:
Because $q_1$ and $s_1$ are distinct states, they correspond to distinct tags, and so $\check{q}_1 \neq \check{s}_1$.

% One possible remedy
% % for this apparent lack of adequacy
% might be to revise the encoding to have a stronger nominal character % .
% by tagging each state's encoding with an atom that is unique to that state.
% For instance, given the pairwise distinct atoms $\set{q \given q \in F}$ and $\set{\bar{q} \given q \in Q - F}$ to tag final and non-final states, respectively, we could define an alternative encoding, $\check{q}$, that is manifestly injective:
% %
% % \begin{marginfigure}
% \begin{gather*}
%   \check{q} \defd
%     \parens[size=big]{
%       \bigwith_{a \in \ialph}(a \limp \check{q}'_a)}
%     \with
%     \parens[size=big]{\emp \limp \check{F}(q)}
%   %
%   \shortintertext{where}
%   %
%   q \dfareduces[a] q'_a
%   \text{, for all input symbols $a \in \ialph$,\quad and\quad}
%   \check{F}(q) =
%     \begin{cases*}
%       q & if $q \in F$ \\
%       \bar{q} & if $q \notin F$%
%     \,.
%     \end{cases*}
% \end{gather*}
% % \end{marginfigure}%
% % , the encoding can be made to be injective.
% % With this change, the alternative encoding is now injective: $\check{q} = \check{s}$ implies $q = s$.

Although such a solution is certainly possible, it seems unsatisfyingly ad~hoc.
A closer examination of the preceding counterexample reveals that the states $q_1$ and $s_1$, while not equal, are in fact bisimilar~\parencref{??}.
In other words, although the encoding is not, strictly speaking, injective, it is injective \emph{up to bisimilarity}: $\dfa{q} = \dfa{s}$ implies $q \asim s$.
This suggests a more elegant solution to the apparent lack of adequacy: the encoding's adequacy should be judged up to \ac{DFA} bisimilarity.
%
\newcommand{\dfaadequacybisimbody}{%
  Let $\aut{A} = (Q, ?, F)$ be \iac{DFA} over the input alphabet $\ialph$.
  Then, for all states $q$, $q'$, and $s$:
  \begin{enumerate}
  \item\label{enum:ordered-rewriting:dfa-adequacy:1}
    $q \asim s$ if, and only if, $\dfa{q} = \dfa{s}$.
  \item\label{enum:ordered-rewriting:dfa-adequacy:2}
    $q \asim\dfareduces[a]\asim q'$ if, and only if, $a \oc \dfa{q} \Reduces \dfa{q}'$, for all input symbols $a \in \ialph$.    
    More generally, $q \asim\dfareduces[w]\asim q'$ if, and only if, $\rev{w} \oc \dfa{q} \Reduces \dfa{q}'$, for all finite words $w \in \finwds{\ialph}$.
  \item\label{enum:ordered-rewriting:dfa-adequacy:3}
    $q \in F$ if, and only if, $\emp \oc \dfa{q} \Reduces \one$.
  \end{enumerate}%
}%
%  
\begin{restatable*}[
  name=\ac*{DFA} adequacy up to bisimilarity,
  label=thm:ordered-rewriting:dfa-adequacy-bisim
]{theorem}{dfaadequacybisim}
  \dfaadequacybisimbody
% Let $\aut{A} = (Q, \mathord{\dfareduces}, F)$ be \iac{DFA} over the input alphabet $\ialph$.
%   Then, for all states $q$, $q'$, and $s$:
%   \begin{enumerate}
%   \item\label{enum:ordered-rewriting:dfa-adequacy:1}
%     $q \asim s$ if, and only if, $\dfa{q} = \dfa{s}$.
%   \item\label{enum:ordered-rewriting:dfa-adequacy:2}
%     $q \asim\dfareduces[a]\asim q'$ if, and only if, $a \oc \dfa{q} \Reduces \dfa{q}'$, for all input symbols $a \in \ialph$.    
%     More generally, $q \asim\dfareduces[w]\asim q'$ if, and only if, $\rev{w} \oc \dfa{q} \Reduces \dfa{q}'$, for all finite words $w \in \finwds{\ialph}$.
%   \item\label{enum:ordered-rewriting:dfa-adequacy:3}
%     $q \in F$ if, and only if, $\emp \oc \dfa{q} \Reduces \one$.
%   \end{enumerate}
\end{restatable*}

Before proving this \lcnamecref{thm:ordered-rewriting:dfa-adequacy-bisim}, we must first prove a \lcnamecref{lem:ordered-rewriting:dfa-traces}: the only traces from one state's encoding to another's are the trivial traces.
%
\begin{lemma}\label{lem:ordered-rewriting:dfa-traces}
  Let $\aut{A} = (Q, ?, F)$ be \iac{DFA} over the input alphabet $\ialph$.
  For all states $q$ and $s$, if $\dfa{q} \Reduces \dfa{s}$, then $\dfa{q} = \dfa{s}$.
\end{lemma}
%
\begin{proof}
  Assume that a trace $\dfa{q} \Reduces \dfa{s}$ exists.
  If the trace is trivial, then $\dfa{q} = \dfa{s}$ is immediate.
  Otherwise, the trace is nontrivial and consists of a strictly positive number of rewriting steps.
  By inversion, those rewriting steps drop one or more conjuncts from $\dfa{q}$ to form $\dfa{s}$.
  Every \ac{DFA} state's encoding contains exactly $\card{\ialph} + 1$ conjuncts -- one for each input symbol $a$ and one for the end-of-word marker, $\emp$.
  % Being the encoding of \iac{DFA} state, $\dfa{q}$ contains one $(\emp \limp \dotsb)$ conjunct and exactly one $(a \limp \dotsb)$ conjunct for each input symbol $a$.
  % Similarly, $\dfa{s}$ must contain the same.
  If even one conjunct is dropped from $\dfa{q}$, not enough conjuncts will remain to form $\dfa{s}$.
  Thus, a nontrivial trace $\dfa{q} \Reduces \dfa{s}$ cannot exist.
\end{proof}
%
\noindent
It is important to differentiate this \lcnamecref{lem:ordered-rewriting:dfa-traces} from the false claim that a state's encoding can take no rewriting steps.
There certainly exist nontrivial traces from $\dfa{q}$, but they do not arrive at the encoding of any state.

With this \lcnamecref{lem:ordered-rewriting:dfa-traces} now in hand, we can proceed to proving adequacy up to bisimilarity.
%
\dfaadequacybisim
%
\begin{proof}
  Each part is proved in turn.
  The proof of part~\ref{enum:ordered-rewriting:dfa-adequacy:2} % and~\ref{enum:ordered-rewriting:dfa-adequacy:4}
  depends on the proof of part~\ref{enum:ordered-rewriting:dfa-adequacy:1}.
  \begin{enumerate}[parsep=0em, listparindent=\parindent]
  %% Part one
  \item
    We shall show that bisimilarity coincides with equality of encodings, proving each direction separately.
    \begin{itemize}[parsep=0em, listparindent=\parindent]
    \item
      To prove that bisimilar \ac{DFA} states have equal encodings -- \ie, that $q \asim s$ implies $\dfa{q} = \dfa{s}$ -- a fairly straightforward proof by coinduction suffices.

      Let $q$ and $s$ be bisimilar states.
      By the definition of bisimilarity~\parencref{??}, two properties hold:
      \begin{itemize}
      \item For all input symbols $a$, the unique $a$-successors of $q$ and $s$ are also bisimilar.
      \item States $q$ and $s$ have matching finalities -- \ie, $q \in F$ if and only if $s \in F$.
      \end{itemize}
      Applying the coinductive hypothesis to the former property, we may deduce that, for all symbols $a$, the $a$-successors of $q$ and $s$ also have equal encodings.
      From the latter property, it follows that $\dfa{F}(q) = \dfa{F}(s)$.
      Because definitions are interpreted equirecursively, these equalities together imply that $q$ and $s$ themselves have equal encodings.

    \item
      To prove the converse -- that states with equal encodings are bisimilar -- we will show that the relation $\mathord{\simu{R}} = \Set{(q, s) \given \dfa{q} = \dfa{s}}$, which relates states if they have equal encodings, is a bisimulation and is therefore included in bisimilarity.
      \begin{itemize}
      \item
        The relation $\simu{R}$ is symmetric.
      \item
        We must show that $\simu{R}$-related states have $\simu{R}$-related $a$-successors, for all input symbols $a$.

        Let $q$ and $s$ be $\simu{R}$-related states.
        Being $\simu{R}$-related, $q$ and $s$ have equal encodings;
        because definitions are interpreted equirecursively, the unrollings of those encodings are also equal.
        By definition of the encoding, it follows that, for each input symbol $a$, the unique $a$-successors of $q$ and $s$ have equal encodings.
        Therefore, for each $a$, the $a$-successors of $q$ and $s$ are themselves $\simu{R}$-related.

      \item
        We must show that $\simu{R}$-related states have matching finalities.

        Let $q$ and $s$ be $\simu{R}$-related states, with $q$ a final state.
        Being $\simu{R}$-related, $q$ and $s$ have equal encodings;
        because definitions are interpreted equirecursively, the unrollings of those encodings are also equal.
        It follows that $\dfa{F}(q) = \dfa{F}(s)$, and so $s$ is also a final state.
      \end{itemize}
    \end{itemize}

  %% Part two
  \item
    We would like to prove that $q \asim\dfareduces[a]\asim q'$ if, and only if, $a \oc \dfa{q} \Reduces \dfa{q}'$, or, more generally, that $q \asim\dfareduces[w]\asim q'$ if, and only if, $\rev{w} \oc \dfa{q} \Reduces \dfa{q}'$.
    Because bisimilar states have equal encodings (part~\ref{enum:ordered-rewriting:dfa-adequacy:1}) and bisimilarity is reflexive (\cref{??}), it suffices to show two stronger statements:
    \begin{enumerate*}
    \item that $q \dfareduces[w] q'$ implies $\rev{w} \oc \dfa{q} \Reduces \dfa{q}'$; and
    \item that $\rev{w} \oc \dfa{q} \Reduces \dfa{q}'$ implies $q \dfareduces[w]\asim q'$.
    \end{enumerate*}
    %
    We prove these in turn.
    %
    \begin{enumerate}
    %% Subpart (a)
    \item
      We shall prove that $q \dfareduces[w] q'$ implies $\rev{w} \oc \dfa{q} \Reduces \dfa{q}'$ by induction over the structure of word $w$.
      \begin{itemize}
      \item
        Consider the case of the empty word, $\emp$; we must show that $q = q'$ implies $\dfa{q} \Reduces \dfa{q}'$.
        Because the encoding is a function, this is immediate.
      \item
        Consider the case of a nonempty word, $a \wc w$; we must show that $q \dfareduces[a]\dfareduces[w] q'$ implies $\rev{w} \oc a \oc \dfa{q} \Reduces \dfa{q}'$.
        Let $q'_a$ be an $a$-successor of state $q$ that is itself $w$-succeeded by state $q'$.
        There exists, by definition of the encoding, a trace
        \begin{equation*}
          \rev{w} \oc a \oc \dfa{q}
            \Reduces \rev{w} \oc a \oc (a \limp \dfa{q}'_a)
            \reduces \rev{w} \oc \dfa{q}'_a
            \Reduces \dfa{q}'
          \,,
        \end{equation*}
        with the trace's tail justified by an appeal to the inductive hypothesis.
        % Because $q'$ is a $w$-successor of $q'_a$, an appeal to the inductive hypothesis yields a trace $\rev{w} \oc \dfa{q}'_a \Reduces \dfa{q}'$.
      \end{itemize}

      % Let $q'$ be an $a$-successor of state $q$.
      % There exists, by definition of the encoding, a trace
      % \begin{equation*}
      %   a \oc \dfa{q} \Reduces a \oc (a \limp \dfa{q}') \reduces \dfa{q}'
      % \,.
      % \end{equation*}

    %% Subpart (b)
    \item
      We shall prove that $\rev{w} \oc \dfa{q} \Reduces \dfa{q}'$ implies $q \dfareduces[w]\asim q'$ by induction over the structure of word $w$.
      \begin{itemize}
      \item
        Consider the case of the empty word, $\emp$;
        we must show that $\dfa{q} \Reduces \dfa{q}'$ implies $q \asim q'$.
        By \cref{lem:ordered-rewriting:dfa-traces}, $\dfa{q} \Reduces \dfa{q}'$ implies that $q$ and $q'$ have equal encodings.
        Part~\ref{enum:ordered-rewriting:dfa-adequacy:1} can then be used to establish that $q$ and $q'$ are bisimilar.
      \item
        Consider the case of a nonempty word, $a \wc w$;
        we must show that $\rev{w} \oc a \oc \dfa{q} \Reduces \dfa{q}'$ implies $q \dfareduces[a]\dfareduces[w]\asim q'$.
        By inversion\fixnote{Is this enough justification?}, the given trace can only begin by inputting $a$:
        \begin{equation*}
          \rev{w} \oc a \oc \dfa{q}
            \Reduces \rev{w} \oc a \oc (a \limp \dfa{q}'_a)
            \reduces \rev{w} \oc \dfa{q}'_a
            \Reduces \dfa{q}'
          \,,
        \end{equation*}
        where $q'_a$ is an $a$-successor of state $q$.
        An appeal to the inductive hypothesis on the trace's tail yields $q'_a \dfareduces[w]\asim q'$, and so the \ac{DFA} admits $q \dfareduces[a]\dfareduces[w]\asim q'$, as required.
      \end{itemize}
      % Assume that a trace $a \oc \dfa{q} \Reduces \dfa{q}'$ exists.
      % By the input lemma, $\dfa{q} \Reduces (a \limp A) \oc \octx'$ for some proposition $A$ and context $\octx'$ such that $A \oc \octx' \Reduces \dfa{q}'$.
      % Upon inversion of the trace from $\dfa{q}$, we conclude that $A = \dfa{q}'_a$, where $q'_a$ is an $a$-successor of $q$, and that $\octx'$ is empty -- in other words, we have a trace $\dfa{q}'_a \Reduces \dfa{q}'$.
      % Such a trace exists only if $\dfa{q}'_a = \dfa{q}'$.
      % By part~\ref{enum:ordered-rewriting:dfa-adequacy:1} of this \lcnamecref{thm:ordered-rewriting:dfa-adequacy-bisim}, it follows that $q'_a$ and $q'$ are bisimilar.
    \end{enumerate}

  %% Part three
  \item
    We shall prove that the final states are exactly those states $q$ such that $\emp \oc \dfa{q} \Reduces \one$.
    \begin{itemize}
    \item
      Let $q$ be a final state; accordingly, $\dfa{F}(q) = \one$.
      There exists, by definition of the encoding, a trace
      \begin{equation*}
        \emp \oc \dfa{q} \Reduces \emp \oc (\emp \limp \dfa{F}(q)) \reduces \dfa{F}(q) = \one
      \,.
      \end{equation*}

    \item
      Assume that a trace $\emp \oc \dfa{q} \Reduces \one$ exists.
      By inversion\fixnote{Is this enough justification?}, this trace can only begin by inputting $\emp$:
      \begin{equation*}
        \emp \oc \dfa{q} \Reduces \emp \oc (\emp \limp \dfa{F}(q)) \reduces \dfa{F}(q) \Reduces \one
      \,.
      \end{equation*}
      The tail of this trace, $\dfa{F}(q) \Reduces \one$, can exist only if $q$ is a final state.
    %
    \qedhere
    \end{itemize}

  % %% Part four
  % \item 
  %   We would like to prove that $q \asim\dfareduces[w]\asim q'$ if, and only if, $\rev{w} \oc \dfa{q} \Reduces \dfa{q}'$.
  %   Because bisimilar states have equal encodings (part~\ref{enum:ordered-rewriting:dfa-adequacy:1}) and bisimilarity is reflexive (\cref{??}), it suffices to show:
  %   \begin{enumerate*}
  %   \item that $q \dfareduces[w] q'$ implies $\rev{w} \oc \dfa{q} \Reduces \dfa{q}'$; and
  %   \item that $\rev{w} \oc \dfa{q} \Reduces \dfa{q}'$ implies $q \dfareduces[w]\asim q'$.
  %   \end{enumerate*}

  %   Both statements can be established by induction over the structure of word $w$.
  %   The latter proof is slightly more involved and deserves a bit of explanation.
  %   \begin{itemize}
  %   \item Consider the case in which $w$ is the empty word; we must show that $\dfa{q} \Reduces \dfa{q}'$ implies $q \asim q'$.
  %     By \cref{lem:ordered-rewriting:dfa-traces}, $\dfa{q} \Reduces \dfa{q}'$ implies that $\dfa{q} = \dfa{q}'$.
  %     Part~\ref{enum:ordered-rewriting:dfa-adequacy:1} can then be used to establish $q$ and $q'$ as bisimilar.

  %   \item Consider the case of a nonempty word, $a \wc w$.
  %     We must show that $\rev{w} \oc a \oc \dfa{q} \Reduces \dfa{q}'$ implies $q \dfareduces[a]\dfareduces[w]\asim q'$.
  %     By inversion, the given trace must begin by inputting $a$:
  %     \begin{equation*}
  %       \rev{w} \oc a \oc \dfa{q} \Reduces \rev{w} \oc a \oc (a \limp \dfa{q}'_a) \reduces \rev{w} \oc \dfa{q}'_a \Reduces \dfa{q}'
  %       \,,
  %     \end{equation*}
  %     where $q'_a$ is an $a$-successor of state $q$.
  %     Appealing to the inductive hypothesis on the trace's tail yields $q'_a \dfareduces[w]\asim q'$, and so $q \dfareduces[a]\dfareduces[w]\asim q'$, as required.
  %   %
  %   \qedhere
  %   \end{itemize}
  \end{enumerate}
\end{proof}


\subsection{Encoding \aclp*{NFA}?}

We would certainly be remiss if we did not attempt to generalize the rewriting specification of \acp{DFA} to one for their nondeterministic cousins.

Differently from \ac{DFA} states, \iac{NFA} state $q$ may have several nondeterministic successors for each input symbol $a$.
To encode the \ac{NFA} state $q$, all of its $a$-successors are collected in an alternative conjunction underneath the left-handed input of $a$.
Thus, the encoding of \iac{NFA} state $q$ becomes
\begin{equation*}
  \nfa{q} \defd
    \parens[size=auto]{\displaystyle
      \bigwith_{a \in \ialph}
        \parens[size=big]{a \limp \parens{\bigwith_{q'_a} \nfa{q}'_a}}
    }
    \with
    \parens[size=big]{\emp \limp \nfa{F}(q)}
  \,,
\end{equation*}
where $\nfa{F}(q)$ is defined as for \acp{DFA}.

The adjacent \lcnamecref{fig:ordered-rewriting:nfa-example}
\begin{marginfigure}
  \centering
  % \subfloat[][]{\label{fig:ordered-rewriting:nfa-example:nfa}%
    \begin{tikzpicture}
      \graph [automaton] {
        q_0
         -> ["a,b", loop above]
        q_0
         -> ["b"]
        q_1 [accepting]
         -> ["a,b"]
        q_2
         -> ["a,b", loop above]
        q_2;
      };
    \end{tikzpicture}
  % }

%   \subfloat[][]{\label{fig:ordered-rewriting:nfa-example:encoding}%
      $\!\begin{aligned}
        \nfa{q}_0 &\defd (a \limp \nfa{q}_0) \with \bigl(b \limp (\nfa{q}_0 \with \nfa{q}_1)\bigr) \with (\emp \limp \top) \\
        \nfa{q}_1 &\defd (a \limp \nfa{q}_2) \with (b \limp \nfa{q}_2) \with (\emp \limp \one) \\
        \nfa{q}_2 &\defd (a \limp \nfa{q}_2) \with (b \limp \nfa{q}_2) \with (\emp \limp \top)
      \end{aligned}$
%     }

  \caption{{fig:ordered-rewriting:nfa-example:nfa}~\Iac*{NFA} that accepts exactly those words, over the alphabet $\ialph = \set{a,b}$, that end with $b$; and {fig:ordered-rewriting:nfa-example:encoding}~its encoding}\label{fig:ordered-rewriting:nfa-example}
\end{marginfigure}%
recalls from \cref{ch:automata} \iac{NFA} that accepts exactly those words, over the alphabet $\ialph = \set{a,b}$, that end with $b$.
Using the above encoding of \acp{NFA}, ordered rewriting does indeed simulate this \ac{NFA}.
For example, just as there are transitions $q_0 \nfareduces[b] q_0$ and $q_0 \nfareduces[b] q_1$, there are traces
\begin{equation*}
  \begin{tikzcd}[
    cells={inner xsep=0.65ex,
           inner ysep=0.4ex},
         % nodes={draw},
    row sep=0em,
    column sep=scriptsize
  ]
    &[-0.2em] \nfa{q}_0
    \\
    b \oc \nfa{q}_0 \Reduces b \oc \bigl(b \limp (\nfa{q}_0 \with \nfa{q}_1)\bigr) \reduces \nfa{q}_0 \with \nfa{q}_1
      \urar[reduces, start anchor=east]
      \drar[reduces, start anchor=base east]
    \\
    & \nfa{q}_1
  \end{tikzcd}
\end{equation*}

Unfortunately, while it does simulate \ac{NFA} behavior, this encoding is not adequate.
Like \ac{DFA} states, \ac{NFA} states that have equal encodings are bisimilar.
% \begin{proof}
%   Define a relation $\mathord{\simu{R}} = \set{(q, s) \given \nfa{q} = \nfa{s}}$; we will show that $\simu{R}$ is a bisimulation.
%   \begin{itemize}
%   \item Assume that $s \simu{R}^{-1} q \nfareduces[a] q'_a$.
%     By definition, $a \oc \nfa{q} \Reduces \nfa{q}'_a$.
%     Because $\nfa{q} = \nfa{s}$, it follows that $s \nfareduces[a] s'_a$ for some state $s'_a$ such that $\nfa{q}'_a = \nfa{s}'_a$ -- that is, $q'_a \simu{R} s'_a$.
%     Thus, $s \nfareduces[a]\simu{R}^{-1} q'_a$.
%   \item Assume that $q \simu{R} s$.
%     It follows that $\nfa{F}(q) = \nfa{F}(s)$.
%     Thus, $q$ is an accepting state if and only if $s$ is.
%   \end{itemize}
% \end{proof}
However, for \acp{NFA}, the converse does not hold: bisimilar states do not necessarily have equal encodings.
%
\begin{falseclaim}
  Let $\aut{A} = (Q, ?, F)$ be \iac{NFA} over input alphabet $\ialph$.
  Then $q \asim s$ implies $\nfa{q} = \nfa{s}$, for all states $q$ and $s$.
\end{falseclaim}
%
\begin{proof}[Counterexample]
  Consider the \ac{NFA} and encoding depicted in the adjacent \lcnamecref{fig:ordered-rewriting:nfa-counterexample}.
  \begin{marginfigure}
    \begin{alignat*}{2}
      \begin{tikzpicture}
        \graph [automaton] {
          q_0 [accepting]
           -> ["a", loop above]
          q_0
           -> ["a", overlay]
          q_1 [accepting, overlay]
           -> ["a", loop above, overlay]
          q_1;
        };
      \end{tikzpicture}
      &\quad&&
      \\
      &\quad& \nfa{q}_0 &\defd \bigl(a \limp (\nfa{q}_0 \with \nfa{q}_1)\bigr) \with (\emp \limp \one) \\
      &\quad& \nfa{q}_1 &\defd (a \limp \nfa{q}_1) \with (\emp \limp \one)
    \end{alignat*}
    \caption{\Iac*{NFA} that accepts all finite words over the alphabet $\ialph = \set{a}$}\label{fig:ordered-rewriting:nfa-counterexample}
  \end{marginfigure}
  It is easy to verify that the relation $\set{q_1} \times \set{q_0,q_1}$ is a bisimulation; in particular, $q_1$ simulates the $q_0 \nfareduces[a] q_1$ transition by its self-loop, $q_1 \nfareduces[a] q_1$.
  Hence, $q_0$ and $s_0$ are bisimilar.
  %
  % These same \acp{NFA} are encoded by the following definitions.
  % \begin{align*}
  %   \nfa{q}_0 &\defd (a \limp \nfa{q}_0) \with (\emp \limp \one)
  % \shortintertext{and}
  %   \nfa{s}_0 &\defd \bigl(a \limp (\nfa{s}_0 \with \nfa{s}_1)\bigr) \with (\emp \limp \one) \\
  %   \nfa{s}_1 &\defd (a \limp \nfa{s}_1) \with (\emp \limp \one)
  % \end{align*}
  It is equally easy to verify, by unrolling the definitions used in the encoding, that $\nfa{q}_0 \neq \nfa{s}_0$.
\end{proof}

For \acp{DFA}, bisimilar states do have equal encodings because the inherent determinism \ac{DFA} bisimilarity is a rather fine-grained equivalence.
Because each \ac{DFA} state has exactly one successor for each input symbol
The additional flexibility entailed by nondeterminism

Once again, it would be possible to construct an adequate encoding, by tagging each state with a unique atom.
% with a stronger nominal character

For the moment, we will put aside the question of an adequate encoding of \acp{NFA}.



\section{Introduction}

In the previous \lcnamecref{ch:ordered-logic}, we saw that the ordered sequent calculus can be given a resource interpretation in which sequents $\oseq{\octx |- A}$ may be read as \enquote{From resources $\octx$, resource goal $A$ is achievable.}
For instance, the left rule for ordered conjunction ($\lrule{\fuse}$, see adjacent display)%
\marginnote{%
  $\infer[\lrule{\fuse}]{\oseq{\octx'_L \oc (A \fuse B) \oc \octx'_R |- C}}{
     \oseq{\octx'_L \oc A \oc B \oc \octx'_R |- C}}$%
}
was read \enquote{Goal $C$ is achievable from resource $A \fuse B$ if it is achievable from the separate resources $A \oc B$.}

As alluded in the previous \lcnamecref{ch:ordered-logic}'s discussion of ordered conjunction\footnote{See \cpageref{p:ordered-logic:ordered-conjunction}.}, this $\lrule{\fuse}$ rule is essentially a rule of resource decomposition: it decomposes [the resource] $A \fuse B$ into the separate resources $A \oc B$ and relegates the unchanged goal $C$ to a secondary role.

\newthought{%
This \lcnamecref{ch:ordered-rewriting}%
}
begins by exploring a refactoring of the ordered sequent calculus's left rules around this idea of resource decomposition~\parencref{sec:ordered-rewriting:??}.
Most of the left rules can be easily refactored in this way, although a few will prove resistant to the change.

Emphasizing resource decomposition naturally leads us to a rewriting interpretation of (a fragment of) ordered logic~\parencref{sec:ordered-rewriting:??}.
This rewriting system is closely related to traditional notions of string rewriting\autocite{??}, but simultaneously restricts and generalizes [...] along distinct axes.

The connection of ordered logic and the Lambek calculus to rewriting is certainly not new.
\Citeauthor{Lambek:AMM58}'s original article\autocite{Lambek:AMM58}

This development borrows from \citeauthor{Cervesato+Scedrov:IC09}'s work on intuitionistic linear logic as an expressive rewriting framework that generalizes traditional notions of multiset rewriting.\autocite{Cervesato+Scedrov:IC09}



\newthought{Most} of the left rules could be seen as decomposing resources.
The left rules were seen as decomposing resources, such as the $\lrule{\fuse}$~rule%
\marginnote{%
  $\infer[\lrule{\fuse}]{\oseq{\octx'_L \oc (A \fuse B) \oc \octx'_R |- C}}{
     \oseq{\octx'_L \oc A \oc B \oc \octx'_R |- C}}$%
}
decomposing $A \fuse B$ into the resources $A \oc B$.
The right rules, on the other hand, were seen as ...

Replacing the left rules with a single, common rule ... and a new judgment, $\octx \reduces \octx'$, that exposes [makes [more] explicit] the decomposition of resources/state transformation aspect.


\section{Most left rules decompose ordered resources}

Recall two of the ordered sequent calculus's left rules: $\lrule{\fuse}$ and $\lrule{\with}_1$.
\begin{inferences}
  \infer[\lrule{\fuse}]{\oseq{\octx'_L \oc (A \fuse B) \oc \octx'_R |- C}}{
    \oseq{\octx'_L \oc A \oc B \oc \octx'_R |- C}}
  \and
  \infer[\lrule{\with}_1]{\oseq{\octx'_L \oc (A \with B) \oc \octx'_R |- C}}{
    \oseq{\octx'_L \oc A \oc \octx'_R |- C}}
\end{inferences}
Both rules decompose the principal resource: in the $\lrule{\fuse}$ rule, $A \fuse B$ into the separate resources $A \oc B$; and, in the $\lrule{\with}_1$ rule, $A \with B$ into $A$.
However, in both cases, the resource decomposition is somewhat obscured by boilerplate.
The framed contexts $\octx'_L$ and $\octx'_R$ and goal $C$ serve to enable the rules to be applied anywhere [in the string of resources], without restriction;
these concerns are not specific to the $\lrule{\fuse}$ and $\lrule{\with}_1$ rules, but are general boilerplate that arguably should be factored out.

To decouple the resource decomposition from the surrounding boilerplate, we will introduce a new judgment, $\octx \reduces \octx'$, meaning \enquote{Resources $\octx$ may be decomposed into [resources] $\octx'$.}
% With this judgment in hand, the boilerplate can be factored into a uniform left rule, $\lrule{\star}$:
With this new judgment comes a cut principle, $\jrule{CUT}^{\reduces}$, into which all of the boilerplate is factored:
\begin{equation*}
  \infer[\jrule{CUT}\smash{^{\reduces}}]{\oseq{\octx'_L \oc \octx \oc \octx'_R |- C}}{
    \octx \reduces \octx' &
    \oseq{\octx'_L \oc \octx' \oc \octx'_R |- C}}
  .
\end{equation*}

The standard left rules can be recovered from resource decomposition rules using this cut principle.
For example, the decomposition of $A \fuse B$ into $A \oc B$ is captured by
\begin{equation*}
  \infer[\jrule{$\fuse$D}]{A \fuse B \reduces A \oc B}{}
  ,
\end{equation*}
and the standard $\lrule{\fuse}$ rule can then be recovered as shown in the neighboring \lcnamecref{fig:ordered-rewriting:fuse-refactoring}.%
\begin{marginfigure}[-8\baselineskip]
  \begin{gather*}
    \infer[\lrule{\fuse}]{\oseq{\octx'_L \oc (A \fuse B) \oc \octx'_R |- C}}{
      \oseq{\octx'_L \oc A \oc B \oc \octx'_R |- C}}
    %
    \\\leftrightsquigarrow\\
    %
    \infer[\jrule{CUT}\smash{^{\reduces}}]{\oseq{\octx'_L \oc (A \fuse B) \oc \octx'_R |- C}}{
      \infer[\jrule{$\fuse$D}]{A \fuse B \reduces A \oc B}{} &
      \oseq{\octx'_L \oc A \oc B \oc \octx'_R |- C}}
    .
  \end{gather*}
  \caption{A refactoring of the $\lrule{\fuse}$ rule as resource decomposition}\label{fig:ordered-rewriting:fuse-refactoring}
\end{marginfigure}
The left rules for $\one$ and $A \with B$ can be similarly refactored into resource decomposition rules.

Even the left rules for left- and right-handed implications can be refactored in this way, despite the additional, minor premises that those rules carry.
To keep the correspondence between resource decomposition rules and left rules close, we could introduce the decomposition rules
\begin{inferences}
  \infer[\jrule{$\limp$D}']{\octx \oc (A \limp B) \reduces B}{
    \oseq{\octx |- A}}
  \and\text{and}\and
  \infer[\jrule{$\pmir$D}']{(B \pmir A) \oc \octx \reduces B}{
    \oseq{\octx |- A}}
  .
\end{inferences}
Just as for ordered conjunction, the left rules for left- and right-handed implication are then recovered by combining a decomposition rule with the $\jrule{CUT}^{\reduces}$ rule~(see adjacent \lcnamecref{fig:ordered-rewriting:limp-refactoring-1}).%
\begin{marginfigure}[-8\baselineskip]
  \begin{gather*}
    \infer[\lrule{\limp}]{\oseq{\octx'_L \oc \octx \oc (A \limp B) \oc \octx'_R |- C}}{
      \oseq{\octx |- A} &
      \oseq{\octx'_L \oc B \oc \octx'_R |- C}}
    %
    \\\leftrightsquigarrow\\
    %
    \infer[\jrule{CUT}\smash{^{\reduces}}]{\oseq{\octx'_L \oc \octx \oc (A \limp B) \oc \octx'_R |- C}}{
      \infer[\jrule{$\limp$D}']{\octx \oc (A \limp B) \reduces B}{
        \oseq{\octx |- A}} &
      \oseq{\octx'_L \oc B \oc \octx'_R |- C}}
  \end{gather*}
  \caption{A refactoring of the $\lrule{\limp}$ rule using a resource decomposition rule}\label{fig:ordered-rewriting:limp-refactoring-1}
\end{marginfigure}

Although these $\jrule{$\limp$D}'$ and $\jrule{$\pmir$D}'$ rules keep the correspondence between resource decomposition rules and left rules close, they differ from the other decomposition rules in two significant ways.
First, the above $\jrule{$\limp$D}'$ and $\jrule{$\pmir$D}'$ rules have premises, and those premises create a dependence of the decomposition judgment upon general provability.
Second, the above $\jrule{$\limp$D}'$ and $\jrule{$\pmir$D}'$ rules do not decompose the principal proposition into immediate subformulas.
This contrasts with, for example, the $\jrule{$\fuse$D}$ rule that decomposes $A \fuse B$ into the immediate subformulas $A \oc B$.

For these reasons, the above $\jrule{$\limp$D}'$ and $\jrule{$\pmir$D}'$ rules are somewhat undesirable.
Fortunately, there is an alternative.
Filling in the $\oseq{\octx |- A}$ premises with the $\jrule{ID}^A$ rule, we arrive at the derivable rules
\begin{inferences}
  \infer[\jrule{$\limp$D}]{A \oc (A \limp B) \reduces B}{}
  \and\text{and}\and
  \infer[\jrule{$\pmir$D}]{(B \pmir A) \oc A \reduces B}{}
  .
\end{inferences}
The standard $\lrule{\limp}$ and $\lrule{\pmir}$ rules can still be recovered from these more specific decomposition rules, thanks to $\jrule{CUT}$ (see adjacent \lcnamecref{fig:ordered-rewriting:limp-refactoring-2}).%
\begin{marginfigure}[-10\baselineskip]
  \begin{gather*}
    \infer[\lrule{\limp}]{\oseq{\octx'_L \oc \octx \oc (A \limp B) \oc \octx'_R |- C}}{
      \oseq{\octx |- A} &
      \oseq{\octx'_L \oc B \oc \octx'_R |- C}}
    %
    \\\leftrightsquigarrow\\
    %
    \infer[\jrule{CUT}\smash{^A}]{\oseq{\octx'_L \oc \octx \oc (A \limp B) \oc \octx'_R |- C}}{
      \oseq{\octx |- A} &
      \infer[\jrule{CUT}\smash{^{\reduces}}]{\oseq{\octx'_L \oc A \oc (A \limp B) \oc \octx'_R |- C}}{
        \infer[\jrule{$\limp$D}]{A \oc (A \limp B) \reduces B}{} &
        \oseq{\octx'_L \oc B \oc \octx'_R |- C}}}
  \end{gather*}
  \caption{A refactoring of the $\lrule{\limp}$ rule using an alternative resource decomposition rule}\label{fig:ordered-rewriting:limp-refactoring-2}
\end{marginfigure}
These revised, nullary decomposition rules correct the earlier drawbacks: like the other decomposition rules, they now have no premises and only refer to immediate subformulas.
Moreover, these rules have the advantage of matching two of the axioms from \citeauthor{Lambek:AMM58}'s original article.\autocite{Lambek:AMM58}


% For many of the ordered logical connectives, this approach  works perfectly.
% The decomposition of $A \fuse B$ into $A \oc B$ is, for example, captured by
% \begin{equation*}
%   \infer[\lrule{\fuse}']{A \fuse B \reduces A \oc B}{}
%   ,
% \end{equation*}
% so that the ordered sequent calculus's standard $\lrule{\fuse}$ rule
% % left rule for multiplicative conjunction
% is then derivable from the uniform left rule:
% \begin{equation*}
%   \infer[\lrule{\fuse}]{\oseq{\octx'_L \oc (A \fuse B) \oc \octx'_R |- C}}{
%     \oseq{\octx'_L \oc A \oc B \oc \octx'_R |- C}}
%   %
%   \enspace\leftrightsquigarrow\enspace
%   %
%   \infer[\lrule{\star}]{\oseq{\octx'_L \oc (A \fuse B) \oc \octx'_R |- C}}{
%     \infer[\lrule{\fuse}']{A \fuse B \reduces A \oc B}{} &
%     \oseq{\octx'_L \oc A \oc B \oc \octx'_R |- C}}
%   .
% \end{equation*}
% The left rules for $\one$ and $A \with B$ can be refactored in a similar way.
% Despite their additional, minor premises, even the left rules for left- and right-handed implications can be refactored in this way.
% \begin{inferences}
%   \infer[\lrule{\limp}']{\octx \oc (A \limp B) \reduces B}{
%     \oseq{\octx |- A}}
%   \and
%   \infer[\lrule{\pmir}']{(B \pmir A) \oc \octx \reduces B}{
%     \oseq{\octx |- A}}
% \end{inferences}

% \begin{equation*}
%   \infer[\lrule{\limp}]{\oseq{\octx'_L \oc \octx \oc (A \limp B) \oc \octx'_R |- C}}{
%     \oseq{\octx |- A} &
%     \oseq{\octx'_L \oc B \oc \octx'_R |- C}}
%   %
%   \enspace\leftrightsquigarrow\enspace
%   %
%   \infer[\lrule{\star}]{\oseq{\octx'_L \oc \octx \oc (A \limp B) \oc \octx'_R |- C}}{
%     \infer[\lrule{\limp}']{\octx \oc (A \limp B) \reduces B}{
%       \oseq{\octx |- A}} &
%     \oseq{\octx'_L \oc B \oc \octx'_R |- C}}
% \end{equation*}


\newthought{%
So, for most%
}
ordered logical connectives, this approach works perfectly.
Unfortunately, the left rules for additive disjunction, $A \plus B$, and its unit, $\zero$, are resistant to this kind of refactoring.
The difficulty with additive disjunction isn't that its left rule, $\lrule{\plus}$,%
\marginnote{%
  \begin{equation*}
    \infer[\lrule{\plus}]{\oseq{\octx'_L \oc (A \plus B) \oc \octx'_R |-  C}}{
      \oseq{\octx'_L \oc A \oc \octx'_R |-  C} &
      \oseq{\octx'_L \oc B \oc \octx'_R |-  C}}
  \end{equation*} 
}
doesn't decompose the resource $A \plus B$.
The $\lrule{\plus}$ rule certainly does decompose $A \plus B$, but it does so [...].
$A \plus B \reduces A \mid B$
[...] retain the standard $\lrule{\plus}$ and $\lrule{\zero}$ rules.

\begin{figure}[tbp]
  \begin{inferences}
    \infer[\jrule{CUT}\smash{^A}]{\oseq{\octx'_L \oc \octx \oc \octx'_R |- C}}{
      \oseq{\octx |- A} & \oseq{\octx'_L \oc A \oc \octx'_R |- C}}
    \and 
    \infer[\jrule{ID}\smash{^A}]{\oseq{A |- A}}{}
    \\
    \infer[\jrule{CUT}\smash{^{\reduces}}]{\oseq{\octx'_L \oc \octx \oc \octx'_R |- C}}{
      \octx \reduces \octx' & \oseq{\octx'_L \oc \octx' \oc \octx'_R |- C}}
    \\
    \infer[\rrule{\fuse}]{\oseq{\octx_1 \oc \octx_2 |- A \fuse B}}{
      \oseq{\octx_1 |- A} & \oseq{\octx_2 |- B}}
    \and
    \infer[\jrule{$\fuse$D}]{A \fuse B \reduces A \oc B}{}
    \\
    \infer[\rrule{\one}]{\oseq{\octxe |- \one}}{}
    \and
    \infer[\jrule{$\one$D}]{\one \reduces \octxe}{}
    \\
    \infer[\rrule{\with}]{\oseq{\octx |- A \with B}}{
      \oseq{\octx |- A} & \oseq{\octx |- B}}
    \and
    \infer[\jrule{$\with$D}_1]{A \with B \reduces A}{}
    \and
    \infer[\jrule{$\with$D}_2]{A \with B \reduces B}{}
    \\
    \infer[\rrule{\top}]{\oseq{\octx |- \top}}{}
    \and
    \text{(no $\jrule{$\top$D}$ rule)}
    \\
    \infer[\rrule{\limp}]{\oseq{\octx |- A \limp B}}{
      \oseq{A \oc \octx |- B}}
    \and
    \infer[\jrule{$\limp$D}]{A \oc (A \limp B) \reduces B}{}
    \\
    \infer[\rrule{\pmir}]{\oseq{\octx |- B \pmir A}}{
      \oseq{\octx \oc A |- B}}
    \and
    \infer[\jrule{$\pmir$D}]{(B \pmir A) \oc A \reduces B}{}
    \\
    \infer[\rrule{\plus}_1]{\oseq{\octx |- A \plus B}}{
      \oseq{\octx |- A}}
    \and
    \infer[\rrule{\plus}_2]{\oseq{\octx |- A \plus B}}{
      \oseq{\octx |- B}}
    \and
    \infer[\lrule{\plus}]{\oseq{\octx'_L \oc (A \plus B) \oc \octx'_R |- C}}{
      \oseq{\octx'_L \oc A \oc \octx'_R |- C} &
      \oseq{\octx'_L \oc B \oc \octx'_R |- C}}
    \\
    \text{(no $\rrule{\zero}$ rule)}
    \and
    \infer[\lrule{\zero}]{\oseq{\octx'_L \oc \zero \oc \octx'_R |- C}}{}
  \end{inferences}
  \caption{A refactoring of the ordered sequent calculus to emphasize that most left rules amount to resource decomposition}\label{fig:ordered-rewriting:decompose-seq-calc}
\end{figure}

\newthought{%
\Cref{fig:ordered-rewriting:decompose-seq-calc} presents%
}
the fully refactored sequent calculus for ordered logic.
This refactored calculus is sound and complete with respect to the ordered sequent calculus~\parencref{fig:ordered-logic:sequent-calculus}.
%
\begin{theorem}[Soundness]
  If\/ $\oseq{\octx |- A}$ is derivable in the refactored calculus of \cref{fig:ordered-rewriting:decompose-seq-calc}, then $\oseq{\octx |- A}$ is derivable in the ordered sequent calculus~\parencref{fig:ordered-logic:sequent-calculus}.
\end{theorem}
%
\begin{proof}
  By structural induction on the given derivation.
  The key lemma is the admissibility of $\jrule{CUT}^{\reduces}$ in the ordered sequent calculus:
  \begin{quotation}
    \normalsize If $\octx \reduces \octx'$ and $\oseq{\octx'_L \oc \octx' \oc \octx'_R |- C}$, then $\oseq{\octx'_L \oc \octx \oc \octx'_R |- C}$.
  \end{quotation}
  This lemma can be proved by case analysis of the decomposition $\octx \reduces \octx'$, reconstituting the corresponding left rule along the lines of the sketches from \cref{fig:ordered-rewriting:fuse-refactoring,fig:ordered-rewriting:limp-refactoring-2}.
\end{proof}
%
\begin{theorem}[Completeness]
  If\/ $\oseq{\octx |- A}$ is derivable in the ordered sequent calculus~\parencref{fig:ordered-logic:sequent-calculus}, then $\oseq{\octx |- A}$ is derivable in the refactored calculus of \cref{fig:ordered-rewriting:decompose-seq-calc}.
\end{theorem}
%
\begin{proof}
  By structural induction on the given derivation.
  The critical cases are the left rules; they are resolved along the lines of the sketches shown in \cref{fig:ordered-rewriting:fuse-refactoring,fig:ordered-rewriting:limp-refactoring-2}.
\end{proof}




\section{Decomposition as rewriting}

Thus far, we have used the decomposition judgment, $\octx \reduces \octx'$, and its rules as the basis for a reconfigured sequent-like calculus for ordered logic.
% But this refactoring also leads naturally to a rewriting system grounded in ordered logic.
% 
Instead,
% of taking the resource decomposition rules as a basis for a reconfigured sequent calculus,
we can also view decomposition as the foundation of a rewriting system grounded in ordered logic.
For example, the decomposition of resource $A \fuse B$ into $A \oc B$ by the $\jrule{$\fuse$D}$ rule
% \marginnote{%
%   \begin{equation*}
%     \infer[\jrule{$\fuse$D}]{A \fuse B \reduces A \oc B}{}
%   \end{equation*}
% }%
can also be seen as \emph{rewriting} $A \fuse B$ into $A \oc B$.
More generally, the decomposition judgment $\octx \reduces \octx'$ can be read as \enquote{$\octx$ rewrites to $\octx'$.}

\Cref{fig:ordered-rewriting:rewriting} summarizes the rewriting system that we obtain from the refactored sequent-like calculus of \cref{fig:ordered-rewriting:decompose-seq-calc}.
%
\begin{figure}[tbp]
  \vspace{\dimexpr-\abovedisplayskip-\abovecaptionskip\relax}
  \begin{inferences}
    \infer[\jrule{$\fuse$D}]{A \fuse B \reduces A \oc B}{}
    \and
    \infer[\jrule{$\one$D}]{\one \reduces \octxe}{}
    \\
    \infer[\jrule{$\with$D}_1]{A \with B \reduces A}{}
    \and
    \infer[\jrule{$\with$D}_2]{A \with B \reduces B}{}
    \and
    \text{(no $\jrule{$\top$D}$ rule)}
    \\
    \infer[\jrule{$\limp$D}]{A \oc (A \limp B) \reduces B}{}
    \and
    \infer[\jrule{$\pmir$D}]{(B \pmir A) \oc A \reduces B}{}
    \\
    \text{(no $\jrule{$\plus$D}$ and $\jrule{$\zero$D}$ rules)}
    \\
    \infer[\jrule{$\reduces$C}\smash{_{\jrule{L}}}]{\octx_1 \oc \octx_2 \reduces \octx'_1 \oc \octx_2}{
      \octx_1 \reduces \octx'_1}
    \and
    \infer[\jrule{$\reduces$C}\smash{_{\jrule{R}}}]{\octx_1 \oc \octx_2 \reduces \octx_1 \oc \octx'_2}{
      \octx_2 \reduces \octx'_2}
  \end{inferences}
  \begin{inferences}
    \infer[\jrule{$\Reduces$R}]{\octx \Reduces \octx}{}
    \and
    \infer[\jrule{$\Reduces$T}]{\octx \Reduces \octx''}{
      \octx \reduces \octx' & \octx' \Reduces \octx''}
  \end{inferences}
  \caption{A rewriting fragment of ordered logic, based on resource decomposition}\label{fig:ordered-rewriting:rewriting}
\end{figure}
%
Essentially, the ordered rewriting system is obtained by discarding all rules except for the decomposition rules.
However, if only the decomposition rules are used, rewritings cannot occur within a larger context.
For example, the $\jrule{$\limp$D}$ rule derives $A \oc (A \limp B) \reduces B$, but $\octx'_L \oc A \oc (A \limp B) \oc \octx'_R \reduces \octx'_L \oc B \oc \octx'_R$ would not be derivable in general.
In the refactored calculus of \cref{fig:ordered-rewriting:decompose-seq-calc}, this kind of framing is taken care of by the cut principle for decomposition, $\jrule{CUT}^{\reduces}$.
To express framing at the level of the $\octx \reduces \octx'$ judgment, we introduce two compatibility rules: together,
\begin{inferences}
  \infer[\jrule{$\reduces$C}\smash{_{\jrule{L}}}]{\octx_1 \oc \octx_2 \reduces \octx'_1 \oc \octx_2}{
    \octx_1 \reduces \octx'_1}
  \and\text{and}\and
  \infer[\jrule{$\reduces$C}\smash{_{\jrule{R}}}]{\octx_1 \oc \octx_2 \reduces \octx_1 \oc \octx'_2}{
    \octx_2 \reduces \octx'_2}
\end{inferences}
ensure that rewriting is compatible with concatenation of ordered contexts.%
\footnote[][-4\baselineskip]{%
  Because ordered contexts form a monoid, these compatibility rules are equivalent to the unified rule
  \begin{equation*}
    \infer[\jrule{$\reduces$C}]{\octx_L \oc \octx \oc \octx_R \reduces \octx_L \oc \octx' \oc \octx_R}{
      \octx \reduces \octx'}
    .
  \end{equation*}
  However, we prefer the two-rule formulation of compatibility because it better aligns with the syntactic structure of contexts.%
}

By forming the reflexive, transitive closure of $\reduces$, we may construct a multi-step rewriting relation, which we choose to write as $\Reduces$.%
\footnote[][0.5\baselineskip]{%
  Usually written as $\reduces^*$, we instead chose $\Reduces$ for the reflexive, transitive closure because of its similarity with process calculus notation for weak transitions, $\Reduces[\smash{\alpha}]$.
  Our reasons will become clearer in subsequent \lcnamecrefs{ch:ordered-bisimilarity}.%
}

Consistent with its [free] monoidal structure, there are two equivalent formulations of this reflexive, transitive closure: each rewriting sequence $\octx \Reduces \octx'$ can be viewed as either a list or tree of individual rewriting steps.
We prefer the list-based formulation shown in \cref{fig:ordered-rewriting:rewriting} because it tends to [...] proofs by structural induction, but, on the basis of the following \lcnamecref{fact:ordered-rewriting:transitivity}, we allow ourselves to freely switch between the two formulations as needed.
%
\begin{fact}[Transitivity of $\Reduces$]
  If \kern0.15em$\octx \Reduces \octx'$ and\/ $\octx' \Reduces \octx''$, then\/ $\octx \Reduces \octx''$.
\end{fact}
%
\begin{proof}
  By induction on the structure of the first trace, $\octx \Reduces \octx'$.
\end{proof}

\newthought{A few remarks} about these rewriting relations are in order.
%
First, interpreting the resource decomposition rules as rewriting only confirms our preference for the nullary $\jrule{$\limp$D}$ and $\jrule{$\pmir$D}$ rules.
% [over the $\jrule{$\limp$D}'$ and $\jrule{$\pmir$D}'$ rules.]
The $\jrule{$\limp$D}'$ and $\jrule{$\pmir$D}'$ rules, with their $\oseq{\octx |- A}$ premises, would be problematic as rewriting rules because they would introduce a dependence of ordered rewriting upon general provability%
% By instead using the $\jrule{$\limp$D}$ and $\jrule{$\pmir$D}$ rules, we ensures that ordered rewriting is a syntactic procedure that
% Instead, we want ordered rewriting to be a syntactic procedure, withou 
, and the concomitant[/attendant] proof search would take ordered rewriting too far afield from traditional, syntactic\fixnote{Is this the right word?} notions of string and multiset rewriting.
[mechanical, computational]

Second, multi-step rewriting is incomplete with respect to the ordered sequent calculus~\parencref{fig:ordered-logic:sequent-calculus} because all right rules have been discarded.
%
 \begin{falseclaim}[Completeness]
  If \kern0.15em$\oseq{\octx |- A}$, then\/ $\octx \Reduces A$.
\end{falseclaim}
%
\begin{proof}[Counterexample]
  The sequent $\oseq{A \limp (C \pmir B) |- (A \limp C) \pmir B}$ is provable, but $A \limp (C \pmir B) \Longarrownot\Reduces (A \limp C) \pmir B$ even though $A \oc (A \limp (C \pmir B)) \oc B \Reduces C$ does hold.
\end{proof}


As expected from the way in which it was developed, ordered rewriting is, however, sound.
Before stating and proving soundness, we must define an operation $\bigfuse \octx$ that reifies an ordered context as a single proposition (see adjacent \lcnamecref{fig:ordered-rewriting:bigfuse}).
%
\begin{marginfigure}
  \begin{align*}
    (\octx_1 \oc \octx_2) &= (\octx_1) \fuse (\octx_2) \\
    \mathord{\text{$\fuse$}} (\octxe) &= \one \\
    A &= A
  \end{align*}
  \begin{align*}
    \bigfuse (\octx_1 \oc \octx_2) &= (\bigfuse \octx_1) \fuse (\bigfuse \octx_2) \\
    \bigfuse (\octxe) &= \one \\
    \bigfuse A &= A
  \end{align*}
\begin{theorem}
  If \kern0.15em$\octx \reduces \octx'$, then\/ $\oseq{\octx |- \bigfuse \octx'}$.
  Also, if \kern0.15em$\octx \Reduces \octx'$, then\/ $\oseq{\octx |- \bigfuse \octx'}$.
\end{theorem}
  \caption{From ordered contexts to propositions}\label{fig:ordered-rewriting:bigfuse}
\end{marginfigure}
%
\begin{theorem}[Soundness]
  If \kern0.15em$\octx \reduces \octx'$, then\/ $\oseq{\octx |- \bigfuse \octx'}$.
  Also, if \kern0.15em$\octx \Reduces \octx'$, then\/ $\oseq{\octx |- \bigfuse \octx'}$.
\end{theorem}
%
\begin{proof}
  By induction on the structure of the given step or trace.
\end{proof}

Last, notice that every rewriting step, $\octx \reduces \octx'$, strictly decreases the number of logical connectives that occur in the ordered context.
More formally, let $\card{\octx}$ be a measure of the number of logical connectives that occur in $\octx$, as defined in the adjacent \lcnamecref{fig:ordered-rewriting:measure}.
%
\begin{marginfigure}
  \begin{align*}
    \card{\octx_1 \oc \octx_2} &= \card{\octx_1} + \card{\octx_2} \\
    \card{\octxe} &= 0 \\
    \card{A \star B} &= \begin{tabular}[t]{@{}l@{}}
                          $1 + \card{A} + \card{B}$ \\
                          \quad if $\mathord{\star} = \mathord{\fuse}$, $\mathord{\with}$, $\mathord{\limp}$, $\mathord{\pmir}$, or $\mathord{\plus}$
                         \end{tabular} \\
    \card{A} &= \mathrlap{1}
                    \quad \text{if $A = \alpha$, $\one$, $\top$, or $\zero$}
  \end{align*}
  \caption{A measure of the number of logical connectives within an ordered context}\label{fig:ordered-rewriting:measure}
\end{marginfigure}%
%
We may then prove the following \lcnamecref{fact:ordered-rewriting:reduction}.
%
\begin{fact}\label{fact:ordered-rewriting:reduction}
  If \kern0.15em$\octx \reduces \octx'$, then $\card{\octx} > \card{\octx'}$.
  Also, if \kern0.15em$\octx \Reduces \octx'$, then $\card{\octx} \geq \card{\octx'}$.
\end{fact}
%
\begin{proof}
  By induction on the structure of the rewriting step.
\end{proof}
%
\noindent
On the basis of this \lcnamecref{fact:ordered-rewriting:reduction}, we will frequently refer to the rewriting relation, $\reduces$, as reduction.


\section{}

\subsection{Binary counters}

\begin{equation*}
  \begin{lgathered}
    \bin{e} \oc \atmL{i} \Reduces \bin{e} \oc \bin{b}_1 \\
    \bin{b}_0 \oc \atmL{i} \Reduces \bin{b}_1 \\
    \bin{b}_1 \oc \atmL{i} \Reduces \atmL{i} \oc \bin{b}_0
  \end{lgathered}
\end{equation*}

\begin{equation*}
  \begin{lgathered}
    \bin{e} \oc \atmL{d} \Reduces \atmR{z} \\
    \bin{b}_0 \oc \atmL{d} \Reduces \atmL{d} \oc \bin{b}'_0 \\
    \bin{b}_1 \oc \atmL{d} \Reduces \bin{b}_0 \oc \atmR{s} \\
    \atmR{z} \oc \bin{b}'_0 \Reduces \atmR{z} \\
    \atmR{s} \oc \bin{b}'_0 \Reduces \bin{b}_1 \oc \atmR{s}
  \end{lgathered}
\end{equation*}

\begin{inferences}
  \infer{e \simu{R} \bin{e}}{}
  \and
  \infer{\octx \oc b_0 \simu{R} \octx' \oc \bin{b}_0}{
    \octx \simu{R} \octx'}
  \and
  \infer{\octx \oc b_1 \simu{R} \octx' \oc \bin{b}_1}{
    \octx \simu{R} \octx'}
  \and
  \infer{\octx \oc i \simu{R} \octx' \oc \atmL{i}}{
    \octx \simu{R} \octx'}
  \\
  \infer{\octx \oc i \oc b_0 \simu{R} \octx' \oc (\atmL{i} \fuse b_0)}{
    \octx \simu{R} \octx'}
\end{inferences}

\subsection{Automata}

\begin{inferences}
  \infer{a \oc \octx \simu{R} \atmR{a} \oc \octx'}{
    \octx \simu{R} \octx'}
  \and
  \infer{q \simu{R} \dfa{q}}{}
\end{inferences}

\begin{equation*}
  \infer{q \simu{R} \dfa{q}'}{
    q \asim q'}
\end{equation*}


\section{Ordered rewriting for specifications}

\subsection{\Aclp*{DFA}}

\begin{equation*}
  \infer[]{a \oc q \reduces q'_a}{}
\end{equation*}
for each \ac{DFA} transition $q \dfareduces[a] q'_a$, and 
\begin{equation*}
  \infer[]{\emp \oc q \reduces F(q)}{}
\end{equation*}
for each state $q$, where $F(q) = \one$ if $q$ is a final state and $F(q) = \top$ otherwise.

\begin{itemize}
\item $q \dfareduces[a] q'_a$ if, and only if, $a \oc q \reduces q'_a$; and 
\item $q \in F$ if, and only if, $\emp \oc q \reduces \one$.
\end{itemize}

\subsection{\Aclp*{NFA}}

Equally straightforward

\subsection{Binary counters}

Values

\paragraph*{An increment operation}
To use ordered rewriting to specify [...]
\begin{equation*}
  \infer[]{e \oc i \reduces e \oc b_1}{}
  \qquad
  \infer[]{b_0 \oc i \reduces b_1}{}
  \qquad
  \infer[]{b_1 \oc i \reduces i \oc b_0}{}
\end{equation*}

Small- and big-step adequacy theorems for increments
\begin{itemize}
\item Slightly simplified because there is no $\fuse$
\end{itemize}


\paragraph*{A decrement operation}
\begin{equation*}
  \infer[]{e \oc d \reduces z}{}
  \qquad
  \infer[]{b_0 \oc d \reduces d \oc b'_0}{}
  \qquad
  \infer[]{b_1 \oc d \reduces b_0 \oc s}{}
  \qquad
  \infer[]{z \oc b'_0 \reduces z}{}
  \qquad
  \infer[]{s \oc b'_0 \reduces b_1 \oc s}{}
\end{equation*}

\begin{itemize}
\item Significantly simpler because there is no $\with$, so we don't need (weak) focusing
\end{itemize}



\section{}

\subsection{Concurrency in ordered rewriting}

As an example of multi-step rewriting, observe that
\begin{equation*}
  % \octx = 
  \alpha_1 \oc (\alpha_1 \limp \alpha_2) \oc (\beta_2 \pmir \beta_1) \oc \beta_1 \Reduces \alpha_2 \oc \beta_2
  % = \octx''
  .
\end{equation*}
In fact, as shown in the adjacent \lcnamecref{fig:ordered-rewriting:concurrent-example},%
%
\begin{marginfigure}
  \begin{equation*}
  \begin{tikzcd}[row sep=large, column sep=tiny]
    &
    \makebox[1em][c]{$\alpha_1 \oc (\alpha_1 \limp \alpha_2) \oc (\beta_2 \pmir \beta_1) \oc \beta_1$}
      \dlar \drar \arrow[Reduces]{dd}
    &
    \\
    \alpha_2 \oc (\beta_2 \pmir \beta_1) \oc \beta_1
      \drar
    &&
    \alpha_1 \oc (\alpha_1 \limp \alpha_2) \oc \beta_2
      \dlar
    \\
    &
    \alpha_2 \oc \beta_2
    &
  \end{tikzcd}
\end{equation*}
  \caption{An example of concurrent ordered rewriting}\label{fig:ordered-rewriting:concurrent-example}
\end{marginfigure}
%
two sequences witness this rewriting: either
\begin{itemize*}[
  mode=unboxed,
  label=, afterlabel=
]
\item the initial state's left half, $\alpha_1 \oc (\alpha_1 \limp \alpha_2)$, is first rewritten to $\alpha_2$ and then its right half, $(\beta_2 \pmir \beta_1) \oc \beta_1$, is rewritten to $\beta_2$; or
\item \textit{vice versa}, the right half is first rewritten to $\beta_2$ and then the left half is rewritten to $\alpha_2$
\end{itemize*}.

Notice that these two sequences differ only in how non-overlapping, and therefore independent, rewritings of the initial state's two halves are interleaved.
Consequently, the two sequences can be -- and indeed should be -- considered essentially equivalent.
% In differing only by the order in which the non-overlapping left and right halves are rewritten, these two rewriting sequences are essentially equivalent.
The details of how the small-step rewrites are interleaved are irrelevant, so that
conceptually, at least, only the big-step trace from $\alpha_1 \oc (\alpha_1 \limp \alpha_2) \oc (\beta_2 \pmir \beta_1) \oc \beta_1$ to $\alpha_2 \oc \beta_2$ remains.
% The details of how the small-step rewrites are interleaved are -- and indeed should be -- swept away, so that conceptually only the big-step trace from $\alpha_1 \oc (\alpha_1 \limp \alpha_2) \oc (\beta_2 \pmir \beta_1) \oc \beta_1$ to $\alpha_2 \oc \beta_2$ remains.

More generally, this idea that the interleaving of independent actions is irrelevant is known as \vocab{concurrent equality}\autocite{Watkins+:CMU02}, and it forms the basis of concurrency.\autocite{??}
Concurrent equality also endows traces $\octx \Reduces \octx'$ with a free partially commutative monoid structure, \ie, traces form a trace monoid.


Because the two indivisual rewriting steps are independent, 
Nothing about the final result, $\alpha_2 \oc \beta_2$, suggests which rewriting sequence 


The rewritings of the left and right halves are not overlapping and therefore independent.
Their independence means that we may view the two rewriting sequences as equivalent -- the two rewriting steps

More generally, any non-overlapping rewritings are independent and may occur in any order.
Rewriting sequences that differ only by the order in which independent rewritings occur may be seen as equivalent sequences.
This equivalence relation, \vocab{concurrent equality}\autocite{Watkins+:CMU02}

because the left half of $\octx$ may be rewritten by the $\jrule{$\limp$D}$ rule to $\alpha_2$, and then the right half may be rewritten to $\beta_2$:

\subsection{Other properties of ordered rewriting}

As the relation $\Reduces$ forms a rewriting system, we may evaluate it along several standard dimensions: termination, confluence.


Because each rewriting step reduces the number of logical connectives present in the state~\parencref{fact:ordered-rewriting:reduction}, ordered rewriting is terminating.
%
\begin{theorem}[Termination]
  No infinite rewriting sequence $\octx_0 \reduces \octx_1 \reduces \octx_2 \reduces \dotsb$ exists.
\end{theorem}
%
\begin{proof}
  Beginning from state $\octx_0$, some state $\octx_i$ will eventually be reached such that either: $\octx_i \nreduces$; or $\card{\octx_i} = 0$ and $\octx_i \reduces \octx_{i+1}$.
  In the latter case, \cref{fact:ordered-rewriting:reduction} establishes $\card{\octx_{i+1}} < 0$, which is impossible.
\end{proof}

Although terminating, ordered rewriting is not confluent.
Confluence requires that all states with a common ancestor, \ie, states $\octx'_1$ and $\octx'_2$ such that $\octx'_1 \secudeR\Reduces \octx'_2$, be joinable, \ie, $\octx'_1 \Reduces\secudeR \octx'_2$.
Because ordered rewriting is directional\fixnote{Is this phrasing correct?} and the relation $\Reduces$ is not symmetric, some nondeterministic choices are irreversible.%
%
\begin{falseclaim}[Confluence]
  If\/ $\octx'_1 \secudeR\Reduces \octx'_2$, then $\octx'_1 \Reduces\secudeR \octx'_2$.
\end{falseclaim}
%
\begin{proof}[Counterexamples]
  Consider the state $\alpha \with \beta$.
  By the rewriting rules for additive conjunction, $\alpha \secuder \alpha \with \beta \reduces \beta$, and hence $\alpha \secudeR \alpha \with \beta \Reduces \beta$.
  However, being atoms, neither $\alpha$ nor $\beta$ reduces.
  And $\alpha \neq \beta$, so $\alpha \Reduces\secudeR \beta$ does \emph{not} hold.

  Even in the $\with$-free fragment, ordered rewriting is not confluent.
  For example,
  % consider the state $(\beta_1 \pmir \alpha) \oc \alpha \oc (\alpha \limp \beta_2)$.
  % By the rewriting rules for right- and left-handed implications,
  \begin{equation*}
    \nsecuder \beta_1 \oc (\alpha \limp \beta_2) \secudeR (\beta_1 \pmir \alpha) \oc \alpha \oc (\alpha \limp \beta_2) \Reduces (\beta_1 \pmir \alpha) \oc \beta_2 \nreduces
    .
    \qedhere
  \end{equation*}
\end{proof}


% Viewing the resource decomposition rules for left- and right-handed implications as rewriting rules is slightly problematic, however.%
% Notice that the premises of these rules both require proofs of $\oseq{\octx |-  A}$.
% In the refactored sequent calculus of \cref{fig:ordered-rewriting:decompose-seq-calc}, that dependence of judgments is fine.
% But for a rewriting system, including arbitrary[/general] proofs would be odd -- rewriting should be a syntax-directed process and should not depend on provability.



% We write the reflexive, transitive closure of $\reduces$ as $\Reduces$.%
% \footnote{This notation is adopted for its similarity with the standard $\pi$-calculus notation for weak transitions, $\cramped{\Reduces[\alpha]}$.}

% This rewriting system is a proper fragment of ordered logic.
% \begin{equation*}
%   \oseq{A \limp (C \pmir B) \dashv|- (A \limp C) \pmir B}
%   \enspace\text{but}\enspace
%   A \limp (C \pmir B) \Longarrownot\Reduces (A \limp C) \pmir B
% \end{equation*}


\section{Unbounded ordered rewriting}

\autocite{Aranda+:FMCO06}

Although a seemingly pleasant property, termination~\parencref{thm:ordered-rewriting:termination} significantly limits the expressiveness of ordered rewriting.
For example, without unbounded rewriting, we cannot even give ordered rewriting specifications of producer-consumer systems or finite automata.

As the proof of termination shows, rewriting is bounded
% $\card{\octx_0}$ is an upper bound on the length of any trace from state $\octx_0$,
precisely because
% $\octx_0$
states
consist of finitely many finite propositions.
To admit unbounded rewriting, we therefore choose to permit infinite propositions in the form of mutually recursive definitions, $\alpha \defd A$.
% could either permit states consisting of infinitely many finite propositions or states consisting of finitely many infinite propositions.
% We choose the latter route [...].
%%
%%
% Infinite propositions are described by mutually recursive definitions $\alpha \defd A$.
These definitions are collected into a signature, $\sig = (\alpha_i \defd A_i)_i$, which indexes the rewriting relations: $\reduces_{\sig}$ and $\Reduces_{\sig}$.%
\footnote{We frequently elide the indexing signature, as it is usually clear from context.} 
To rule out definitions like $\alpha \defd \alpha$ that do not correspond to sensible infinite propositions, we also require that definitions be \vocab{contractive}\autocite{Gay+Hole:AI05} -- \ie, that the body of each recursive definition begin with a logical connective at the top level.

By analogy with recursive types from functional programming\autocite{??}, we must now decide whether to treat definitions \emph{iso}\-re\-cur\-sively or \emph{equi}\-re\-cur\-sively.
Under an equirecursive interpretation, definitions $\alpha \defd A$ may be silently unrolled or rolled at will;
in other words, $\alpha$ is literally \emph{equal} to its unrolling, $A$.
In contrast, under an isorecursive interpretation, unrolling a recursively defined proposition would count as an explicit step of rewriting -- $\alpha \reduces A$, for example.

% Under the isorecursive interpretation, unrolling a recursively defined prop\-o\-sition counts as an explicit step of rewriting.
% We introduce the $\jrule{$\defd$D}$ rule to account for this unrolling:
% \begin{equation*}
%   \infer[\jrule{$\defd$D}]{\alpha \reduces_{\sig} A}{
%     \text{$(\alpha \defd A) \in \sig$}}
% \end{equation*}
% Because $A$ is seen as a proper subformula of [the recursively defined] $\alpha$, this unrolling rule aligns well with the rewriting-as-decomposition philosophy.%
% \footnote{In fact, we could have chosen to include recursive definitions in the sequent calculus, following \textcites{SchroederHeister:LICS93}{Tiu+Momigliano:JAL12} and others.
%   Had we done so, the $\jrule{$\defd$D}$ rule would be seen as the decomposition counterpart to the left rule
%   \begin{equation*}
%     \infer[\lrule{\defd}]{\oseq{\octx'_L \oc \alpha \oc \octx'_R |-_{\sig} C}}{
%       \bigl((\alpha \defd A) \in \sig\bigr) &
%       \oseq{\octx'_L \oc A \oc \octx'_R |-_{\sig} C}}
%   \end{equation*}
% }
% Conversely, there is no rule that permits the rolling of $A$ into $\alpha$, because such a rule would not be a decomposition.

We choose to interpret definitions equirecursively
because the equirecursive treatment, with its generous notion of equality, helps to minimize the overhead of recursively defined propositions.
As a simple example, under the equirecursive definition $\beta \defd a \limp \beta$, we have the trace
\begin{equation*}
  a \oc a \oc \beta = a \oc a \oc (a \limp \beta) \reduces a \oc \beta = a \oc (a \limp \beta) \reduces \beta
\end{equation*}
or, more concisely, $a \oc a \oc \beta \reduces a \oc \beta \reduces \beta$.
Had we chosen
% With
 an isorecursive treatment of the same definition, we would have only the more laborious
\begin{equation*}
  a \oc a \oc \beta \reduces a \oc a \oc (a \limp \beta) \reduces a \oc \beta \reduces a \oc (a \limp \beta) \reduces \beta
  .
\end{equation*}

% As a simple example of ordered rewriting with recursive definitions, consider rewriting under the definition $\beta \defd a \limp \beta$; we have the trace
% \begin{equation*}
%   a \oc a \oc \beta = a \oc a \oc (a \limp \beta) \reduces a \oc \beta = a \oc (a \limp \beta) \reduces \beta
%   .
% \end{equation*}



% Instead of allowing arbitrary infinite propositions, we require that infinite propositions have a regular, recursive structure:
% A signature of mutually recursive definitions
% \begin{equation*}
%   \sig = (\alpha_i \defd A_i)_i
%   ,
% \end{equation*}
% where the variables $\alpha_i$ may occur in the bodies $A_j$.
% %
% To rule out definitions like $\alpha \defd \alpha$ that do notcorrespond to sensible infinite propositions, we additionally require that definitions be \vocab{contractive}\autocite{Gay+Hole:AI05} -- that the body of each recursive definition begin with a logical connective at the top level.




% Contractivity justifies an \emph{equi}recursive treatment of propositions in which definitions may be silently unrolled (or rolled) at will.
% In other words, a proposition $\alpha \defd A$ is \emph{equal} to its unrolling, $[A/\alpha]A$.
% This stands in contrast with an \emph{iso}recursive treatment of definitions in which unrolling a recursively defined proposition would count as an explicit step of rewriting: isorecursively, $\alpha \defd A$ would not be equal to $[A/\alpha]A$, but $\alpha \reduces {[A/\alpha]A}$.

% The equirecursive treatment, with its generous notion of equality, helps to minimize the overhead of recursively defined propositions.
% As a simple example, under the equirecursive definition $\beta \defd a \limp \beta$, we have
% \begin{equation*}
%   a \oc a \oc \beta = a \oc a \oc (a \limp \beta) \reduces a \oc \beta = a \oc (a \limp \beta) \reduces \beta
% \end{equation*}
% or, more concisely, $a \oc a \oc \beta \reduces a \oc \beta \reduces \beta$.
% With an isorecursive treatment of the same definition, we would have only the more laborious
% \begin{equation*}
%   a \oc a \oc \beta \reduces a \oc a \oc (a \limp \beta) \reduces a \oc \beta \reduces a \oc (a \limp \beta) \reduces \beta
%   .
% \end{equation*}

% The proof of termination involves a finite upper bound on the number of rewriting steps that 
% Stated informally, termination means that As captured in \cref{fact:ordered-rewriting:reduction}, states $\octx$ that consist of finitely many finite propositions

% Although its development from the ordered sequent calculus, ordered rewriting as defined thus far is not terribly useful.
% Its main limitation is that finite states 
% With finite states $\octx$ consisting of 

% \subsection{Replication}

% \subsection{Recursively defined propositions}


\subsection{Replication}

In Milner's development of the $\pi$-calculus, there are two avenues to unbounded process behavior: recursive process definitions and replication.


\section{Extended examples of ordered rewriting}

\subsection{Encoding \aclp*{DFA}}

As an extended example, we will use ordered rewriting to specify how \iac{DFA} processes its input.
%
% \Acp{DFA} serve as an example of ordered rewriting,  can be used to specify how \iac{DFA} processes its input.
%
Given \iac{DFA} $\aut{A} = (Q, ?, F)$ over an input alphabet $\ialph$, the idea is to encode each state, $q \in Q$, as an ordered proposition, $\dfa{q}$, in such a way that the \ac{DFA}'s operational semantics are adequately captured by [ordered] rewriting.
%
% The basic idea is to define an encoding, $\dfa{q}$, of \ac{DFA} states as ordered propositions;
% this encoding should adequately reflect the \ac{DFA}'s operational semantics with ordered rewriting traces.
\fixnote{[In general, the behavior of \iac{DFA} state is recursive, so the proposition $\dfa{q}$ will be recursively defined.]}
%
% finite input words, $w \in \finwds{\ialph}$, are encoded as ordered contexts by $\emp \oc \rev{w}$

% \NewDocumentCommand \rev { s m } {
%   \IfBooleanTF {#1}
%     { (#2)^{\mathsf{R}} }
%     { #2^{\mathsf{R}} }
% }

% \begin{align*}
%   \rev{a} &= a \\
%   \rev*{w_1 \wc w_2} &= \rev{w_2} \oc \rev{w_1} \\
%   \rev{\emp} &= \octxe
% \end{align*}

Ideally, \ac{DFA} transitions $q \dfareduces[a] q'_a$ would be in bijective correspondence with rewriting steps $a \oc \dfa{q} \reduces \dfa{q}'_a$, where each input symbol $a$ is encoded by a matching [propositional] atom.
%
We will return to the possibility of this kind of tight correspondence in \cref{??}, but,
%
for now, we will content ourselves with a correspondence with traces rather than individual steps, adopting the following desiderata:
% Unfortunately, ordered rewriting's small step size turns out to be a poor match for [...], so in both cases we will instead content ourselves with corrspondances with \emph{traces}:
% a bijection between transitions $q \dfareduces[a] q'_a$ and \emph{traces} $a \oc \dfa{q} \Reduces \dfa{q}'_a$.
% Similarly, [...] a bijection between accepting states $q \in F$ and traces $\emp \oc \dfa{q} \Reduces \octxe$.
%
% This leads us to adopt the following as desiderata:
\begin{itemize}
\item
  $q \dfareduces[a] q'_a$ if, and only if, $a \oc \dfa{q} \Reduces \dfa{q}'_a$, for all input symbols $a \in \ialph$.
\item
  $q \in F$ if, and only if, $\emp \oc \dfa{q} \Reduces \one$, where the atom $\emp$ functions as an end-of-word marker.
% \item
%   $q \dfareduces[w] q'_w \in F$ if, and only if, $\emp \oc \rev{w} \oc \dfa{q} \Reduces \octxe$.
%   Also, $q \dfareduces[w] q'_w \notin F$ if, and only if, $\emp \oc \rev{w} \oc \dfa{q} \Reduces \top$.
\end{itemize}
Given the reversal (anti-)\-homo\-morph\-ism from finite words to ordered contexts defined in the adjacent \lcnamecref{fig:ordered-rewriting:reversal}%
\begin{marginfigure}
  \begin{align*}
    \rev*{w_1 \wc w_2} &= \rev{w_2} \oc \rev{w_1} \\
    \rev{\emp} &= \octxe \\
    \rev{a} &= a
  \end{align*}
  \caption{An (anti-)\-homo\-morph\-ism for reversal of finite words to ordered contexts}\label{fig:ordered-rewriting:reversal}
\end{marginfigure}%
, the first desideratum is subsumed by a third:
% property that covers finite words:
\begin{itemize}[resume*]
\item $q \dfareduces[w] q'$ if, and only if, $\rev{w} \oc \dfa{q} \Reduces \dfa{q}'$, for all finite words $w \in \finwds{\ialph}$.
\end{itemize}

From these desiderata [and the observation that \acp{DFA}' graphs frequently%
\fixnote{Actually, there is always at least one cycle in a well-formed \ac{DFA}.}
contain cycles], we arrive at the following encoding, in which each state is encoded by one of a collection of mutually recursive definitions:%
\fixnote{$q'_a$, using function or relation?}
\begin{gather*}
  \dfa{q} \defd
    \parens[size=big]{
      \bigwith_{a \in \ialph}(a \limp \dfa{q}'_a)}
    \with
    \parens[size=big]{\emp \limp \dfa{F}(q)}
  % \text{where
  %   $q \dfareduces[a] q'_a$ for all $a \in \ialph$
  %   and
  %   $\dfa{F}(q) = 
  %     \begin{cases*}
  %       \one & if $q \in F$ \\
  %       \top & if $q \notin F$
  %     \end{cases*}$%
  % }
  %
\shortintertext{where}
  %
  q \dfareduces[a] q'_a
  \text{, for all input symbols $a \in \ialph$,\quad and\quad}
  \dfa{F}(q) = 
    \begin{cases*}
      \one & if $q \in F$ \\
      \top & if $q \notin F$%
    \,.
    \end{cases*}
\end{gather*}
Just as each state $q$ has exactly one successor for each input symbol $a$, its encoding, $\dfa{q}$, has exactly one input clause, $(a \limp \dotsb)$, for each symbol $a$.



% The traces $a \oc \dfa{q} \Reduces \dfa{q}'_a$
% % for input symbols $a \in \ialph$
% suggest that $\dfa{q}$ should be a collection of clauses that input atoms $a$ from the left.
% And the traces $\emp \oc \dfa{q} \Reduces \octxe$ or $\emp \oc \dfa{q} \Reduces \top$ suggest that $\dfa{q}$ also contain a clause that inputs atom $\emp$ from the left.
% Thus, we arrive at the encoding


\newthought{For a concrete instance} of this encoding, recall from \cref{ch:automata} the \ac{DFA} (repeated in the adjacent \lcnamecref{fig:ordered-rewriting:dfa-example-ends-b})%
%
\begin{marginfigure}
  \begin{equation*}
    \mathllap{\aut{A}_2 = {}}
    \begin{tikzpicture}[baseline=(q_0.base)]
      \graph [automaton] {
        q_0
         -> [loop above, "a"]
        q_0
         -> ["b", bend left]
        q_1 [accepting]
         -> [loop above, "b"]
        q_1
         -> ["a", bend left]
        q_0;
      };
    \end{tikzpicture}
  \end{equation*}
  \caption{\Iac*{DFA} that accepts, from state $q_0$, exactly those words that end with $b$. (Repeated from \cref{fig:dfa-example-ends-b}.)}\label{fig:ordered-rewriting:dfa-example-ends-b}
\end{marginfigure}
%
that accepts exactly those words, over the alphabet $\ialph = \set{a,b}$, that end with $b$; that \ac{DFA} is encoded by the following definitions:
\begin{equation*}
  \begin{lgathered}
    \dfa{q}_0 \defd (a \limp \dfa{q}_0) \with (b \limp \dfa{q}_1) \with (\emp \limp \top) \\
    \dfa{q}_1 \defd (a \limp \dfa{q}_0) \with (b \limp \dfa{q}_1) \with (\emp \limp \one)
  \end{lgathered}
\end{equation*}
Indeed, just as the \ac{DFA} has a transition $q_0 \dfareduces[b] q_1$, its encoding admits a trace
\begin{align*}
  &b \oc \dfa{q}_0
     = b \oc \bigl((a \limp \dfa{q}_0) \with (b \limp \dfa{q}_1) \with (\emp \limp \top)\bigr)
     \Reduces b \oc (b \limp \dfa{q}_1)
     \reduces \dfa{q}_1
  \,.
\intertext{And, just as $q_1$ is an accepting state, its encoding also admits a trace}
  &\emp \oc \dfa{q}_1 = \emp \oc \bigl((a \limp \dfa{q}_0) \with (b \limp \dfa{q}_1) \with (\emp \limp \one)\bigr) \Reduces \emp \oc (\emp \limp \one) \reduces \one
  \,.
\end{align*}

\newthought{More generally}, this encoding is complete, in the sense that it simulates all \ac{DFA} transitions: $q \dfareduces[a] q'$ implies $a \oc \dfa{q} \Reduces \dfa{q}'$, for all states $q$ and $q'$ and input symbols $a$.

However, the converse does not hold -- the encoding is unsound because there are rewritings that cannot be simulated by \iac{DFA} transition.
% That is, $a \oc \dfa{q} \Reduces \dfa{q}'$ does \emph{not} imply $q \dfareduces[a] q'$.
% 
\begin{falseclaim}
  Let $\aut{A} = (Q, \mathord{\dfareduces}, F)$ be \iac{DFA} over the input alphabet $\ialph$.
  Then $a \oc \dfa{q} \Reduces \dfa{q}'$ implies $q \dfareduces[a] q'$, for all input symbols $a \in \ialph$.
\end{falseclaim}
%
\begin{marginfigure}
    \centering
    % \subfloat[][]{\label{fig:ordered-rewriting:dfa-counterexample:dfa}%
      \begin{equation*}
        \aut{A}'_2 = 
      \begin{tikzpicture}[baseline=(q_0.base)]
        \graph [automaton] {
          q_0
           -> [loop above, "a"]
          q_0
           -> ["b", bend left]
          q_1 [accepting]
           -> [loop above, "b"]
          q_1
           -> ["a", bend left]
          q_0;
          %
%          { [chain shift={(2,0)}]
            s_1 [accepting, below=1.5em of q_1.south]
             -> [loop right, "b"]
            s_1
             -> ["a", bend left]
            q_0;
%          };
        };
      \end{tikzpicture}
    \end{equation*}
    % }
    % \subfloat[][]{\label{fig:ordered-rewriting:dfa-counterexample:encoding}%
      $\!\begin{aligned}
        \dfa{q}_0 &\defd (a \limp \dfa{q}_0) \with (b \limp \dfa{q}_1) \with (\emp \limp \top) \\
        \dfa{q}_1 &\defd (a \limp \dfa{q}_0) \with (b \limp \dfa{q}_1) \with (\emp \limp \one) \\
        \dfa{s}_1 &\defd (a \limp \dfa{q}_0) \with (b \limp \dfa{s}_1) \with (\emp \limp \one)
      \end{aligned}$%
    % }
    \caption{{fig:ordered-rewriting:dfa-counterexample:dfa}~A slightly modified version of the \ac*{DFA} from \cref{fig:ordered-rewriting:dfa-example-ends-b}; and {fig:ordered-rewriting:dfa-counterexample:encoding}~its encoding}\label{fig:ordered-rewriting:dfa-counterexample}
  \end{marginfigure}%
\begin{proof}[Counterexample]
  Consider the \ac{DFA} and encoding shown in the adjacent \lcnamecref{fig:ordered-rewriting:dfa-counterexample}; it is the same \ac{DFA} as shown in \cref{fig:ordered-rewriting:dfa-example-ends-b}, but with one added state, $s_1$, that is unreachable from $q_0$ and $q_1$.
    %
  % When encoded as an ordered rewriting specification, it corresponds to the following definitions:
  % \begin{equation*}
  %   \begin{lgathered}
  %     \dfa{q}_0 \defd (a \limp \dfa{q}_0) \with (b \limp \dfa{q}_1) \with (\emp \limp \top) \\
  %     \dfa{q}_1 \defd (a \limp \dfa{q}_0) \with (b \limp \dfa{q}_1) \with (\emp \limp \one) \\
  %     \dfa{s}_1 \defd (a \limp \dfa{q}_0) \with (b \limp \dfa{s}_1) \with (\emp \limp \one)
  %   \end{lgathered}
  % \end{equation*}
  Notice that, as a coinductive consequence of the equirecursive treatment of definitions, $\dfa{q}_1 = \dfa{s}_1$.
  Previously, we saw that $b \oc \dfa{q}_0 \Reduces \dfa{q}_1$; hence $b \oc \dfa{q}_0 \Reduces \dfa{s}_1$.
  However, the \ac{DFA} has no $q_0 \dfareduces[b] s_1$ transition (because $q_1 \neq s_1$), and so this encoding is unsound with respect to the operational semantics of \acp{DFA}.
\end{proof}

As this counterexample shows, the lack of adequacy stems from attempting to use an encoding that is not injective -- here, $q_1 \neq s_1$ even though $\dfa{q}_1 = \dfa{s}_1$.
In other words, eqality of state encodings is a coarser eqvivalence than equality of the states themselves.

One possible remedy for this lack of adequacy might be to revise the encoding to have a stronger nominal character.
By tagging each state's encoding with an atom that is unique to that state, we can make the encoding manifestly injective.
For instance, given the pairwise distinct atoms $\Set{q \given q \in F}$ and $\Set{\bar{q} \given q \in Q - F}$ to tag final and non-final states, respectively, we could define an alternative encoding, $\check{q}$:
%
\begin{gather*}
  \check{q} \defd
    \parens[size=big]{
      \bigwith_{a \in \ialph}(a \limp \check{q}'_a)}
    \with
    \parens[size=big]{\emp \limp \check{F}(q)}
  %
  \shortintertext{where}
  %
  q \dfareduces[a] q'_a
  \text{, for all input symbols $a \in \ialph$,\quad and\quad}
  \check{F}(q) =
    \begin{cases*}
      q & if $q \in F$ \\
      \bar{q} & if $q \notin F$%
    \,.
    \end{cases*}
\end{gather*}
%
Under this alternative encoding, the states $q_1$ and $s_1$ of \cref{fig:ordered-rewriting:dfa-counterexample} are no longer a counterexample to injectivity:
Because $q_1$ and $s_1$ are distinct states, they correspond to distinct tags, and so $\check{q}_1 \neq \check{s}_1$.

% One possible remedy
% % for this apparent lack of adequacy
% might be to revise the encoding to have a stronger nominal character % .
% by tagging each state's encoding with an atom that is unique to that state.
% For instance, given the pairwise distinct atoms $\set{q \given q \in F}$ and $\set{\bar{q} \given q \in Q - F}$ to tag final and non-final states, respectively, we could define an alternative encoding, $\check{q}$, that is manifestly injective:
% %
% % \begin{marginfigure}
% \begin{gather*}
%   \check{q} \defd
%     \parens[size=big]{
%       \bigwith_{a \in \ialph}(a \limp \check{q}'_a)}
%     \with
%     \parens[size=big]{\emp \limp \check{F}(q)}
%   %
%   \shortintertext{where}
%   %
%   q \dfareduces[a] q'_a
%   \text{, for all input symbols $a \in \ialph$,\quad and\quad}
%   \check{F}(q) =
%     \begin{cases*}
%       q & if $q \in F$ \\
%       \bar{q} & if $q \notin F$%
%     \,.
%     \end{cases*}
% \end{gather*}
% % \end{marginfigure}%
% % , the encoding can be made to be injective.
% % With this change, the alternative encoding is now injective: $\check{q} = \check{s}$ implies $q = s$.

Although such a solution is certainly possible, it seems unsatisfyingly ad~hoc.
A closer examination of the preceding counterexample reveals that the states $q_1$ and $s_1$, while not equal, are in fact bisimilar~\parencref{??}.
In other words, although the encoding is not, strictly speaking, injective, it is injective \emph{up to bisimilarity}: $\dfa{q} = \dfa{s}$ implies $q \asim s$.
This suggests a more elegant solution to the apparent lack of adequacy: the encoding's adequacy should be judged up to \ac{DFA} bisimilarity.
%
\renewcommand{\dfaadequacybisimbody}{%
  Let $\aut{A} = (Q, ?, F)$ be \iac{DFA} over the input alphabet $\ialph$.
  Then, for all states $q$, $q'$, and $s$:
  \begin{enumerate}
  \item\label{enum:ordered-rewriting:dfa-adequacy:1}
    $q \asim s$ if, and only if, $\dfa{q} = \dfa{s}$.
  \item\label{enum:ordered-rewriting:dfa-adequacy:2}
    $q \asim\dfareduces[a]\asim q'$ if, and only if, $a \oc \dfa{q} \Reduces \dfa{q}'$, for all input symbols $a \in \ialph$.    
    More generally, $q \asim\dfareduces[w]\asim q'$ if, and only if, $\rev{w} \oc \dfa{q} \Reduces \dfa{q}'$, for all finite words $w \in \finwds{\ialph}$.
  \item\label{enum:ordered-rewriting:dfa-adequacy:3}
    $q \in F$ if, and only if, $\emp \oc \dfa{q} \Reduces \one$.
  \end{enumerate}%
}%
%  
\begin{restatable*}[
  name=\ac*{DFA} adequacy up to bisimilarity,
  label=thm:ordered-rewriting:dfa-adequacy-bisim
]{theorem}{dfaadequacybisim}
  \dfaadequacybisimbody
% Let $\aut{A} = (Q, \mathord{\dfareduces}, F)$ be \iac{DFA} over the input alphabet $\ialph$.
%   Then, for all states $q$, $q'$, and $s$:
%   \begin{enumerate}
%   \item\label{enum:ordered-rewriting:dfa-adequacy:1}
%     $q \asim s$ if, and only if, $\dfa{q} = \dfa{s}$.
%   \item\label{enum:ordered-rewriting:dfa-adequacy:2}
%     $q \asim\dfareduces[a]\asim q'$ if, and only if, $a \oc \dfa{q} \Reduces \dfa{q}'$, for all input symbols $a \in \ialph$.    
%     More generally, $q \asim\dfareduces[w]\asim q'$ if, and only if, $\rev{w} \oc \dfa{q} \Reduces \dfa{q}'$, for all finite words $w \in \finwds{\ialph}$.
%   \item\label{enum:ordered-rewriting:dfa-adequacy:3}
%     $q \in F$ if, and only if, $\emp \oc \dfa{q} \Reduces \one$.
%   \end{enumerate}
\end{restatable*}

Before proving this \lcnamecref{thm:ordered-rewriting:dfa-adequacy-bisim}, we must first prove a \lcnamecref{lem:ordered-rewriting:dfa-traces}: the only traces from one state's encoding to another's are the trivial traces.
%
\begin{lemma}\label{lem:ordered-rewriting:dfa-traces}
  Let $\aut{A} = (Q, ?, F)$ be \iac{DFA} over the input alphabet $\ialph$.
  For all states $q$ and $s$, if $\dfa{q} \Reduces \dfa{s}$, then $\dfa{q} = \dfa{s}$.
\end{lemma}
%
\begin{proof}
  Assume that a trace $\dfa{q} \Reduces \dfa{s}$ exists.
  If the trace is trivial, then $\dfa{q} = \dfa{s}$ is immediate.
  Otherwise, the trace is nontrivial and consists of a strictly positive number of rewriting steps.
  By inversion, those rewriting steps drop one or more conjuncts from $\dfa{q}$ to form $\dfa{s}$.
  Every \ac{DFA} state's encoding contains exactly $\card{\ialph} + 1$ conjuncts -- one for each input symbol $a$ and one for the end-of-word marker, $\emp$.
  % Being the encoding of \iac{DFA} state, $\dfa{q}$ contains one $(\emp \limp \dotsb)$ conjunct and exactly one $(a \limp \dotsb)$ conjunct for each input symbol $a$.
  % Similarly, $\dfa{s}$ must contain the same.
  If even one conjunct is dropped from $\dfa{q}$, not enough conjuncts will remain to form $\dfa{s}$.
  Thus, a nontrivial trace $\dfa{q} \Reduces \dfa{s}$ cannot exist.
\end{proof}
%
\noindent
It is important to differentiate this \lcnamecref{lem:ordered-rewriting:dfa-traces} from the false claim that a state's encoding can take no rewriting steps.
There certainly exist nontrivial traces from $\dfa{q}$, but they do not arrive at the encoding of any state.

With this \lcnamecref{lem:ordered-rewriting:dfa-traces} now in hand, we can proceed to proving adequacy up to bisimilarity.
%
\dfaadequacybisim
%
\begin{proof}
  Each part is proved in turn.
  The proof of part~\ref{enum:ordered-rewriting:dfa-adequacy:2} % and~\ref{enum:ordered-rewriting:dfa-adequacy:4}
  depends on the proof of part~\ref{enum:ordered-rewriting:dfa-adequacy:1}.
  \begin{enumerate}[parsep=0em, listparindent=\parindent]
  %% Part one
  \item
    We shall show that bisimilarity coincides with equality of encodings, proving each direction separately.
    \begin{itemize}[parsep=0em, listparindent=\parindent]
    \item
      To prove that bisimilar \ac{DFA} states have equal encodings -- \ie, that $q \asim s$ implies $\dfa{q} = \dfa{s}$ -- a fairly straightforward proof by coinduction suffices.

      Let $q$ and $s$ be bisimilar states.
      By the definition of bisimilarity~\parencref{??}, two properties hold:
      \begin{itemize}
      \item For all input symbols $a$, the unique $a$-successors of $q$ and $s$ are also bisimilar.
      \item States $q$ and $s$ have matching finalities -- \ie, $q \in F$ if and only if $s \in F$.
      \end{itemize}
      Applying the coinductive hypothesis to the former property, we may deduce that, for all symbols $a$, the $a$-successors of $q$ and $s$ also have equal encodings.
      From the latter property, it follows that $\dfa{F}(q) = \dfa{F}(s)$.
      Because definitions are interpreted equirecursively, these equalities together imply that $q$ and $s$ themselves have equal encodings.

    \item
      To prove the converse -- that states with equal encodings are bisimilar -- we will show that the relation $\mathord{\simu{R}} = \Set{(q, s) \given \dfa{q} = \dfa{s}}$, which relates states if they have equal encodings, is a bisimulation and is therefore included in bisimilarity.
      \begin{itemize}
      \item
        The relation $\simu{R}$ is symmetric.
      \item
        We must show that $\simu{R}$-related states have $\simu{R}$-related $a$-successors, for all input symbols $a$.

        Let $q$ and $s$ be $\simu{R}$-related states.
        Being $\simu{R}$-related, $q$ and $s$ have equal encodings;
        because definitions are interpreted equirecursively, the unrollings of those encodings are also equal.
        By definition of the encoding, it follows that, for each input symbol $a$, the unique $a$-successors of $q$ and $s$ have equal encodings.
        Therefore, for each $a$, the $a$-successors of $q$ and $s$ are themselves $\simu{R}$-related.

      \item
        We must show that $\simu{R}$-related states have matching finalities.

        Let $q$ and $s$ be $\simu{R}$-related states, with $q$ a final state.
        Being $\simu{R}$-related, $q$ and $s$ have equal encodings;
        because definitions are interpreted equirecursively, the unrollings of those encodings are also equal.
        It follows that $\dfa{F}(q) = \dfa{F}(s)$, and so $s$ is also a final state.
      \end{itemize}
    \end{itemize}

  %% Part two
  \item
    We would like to prove that $q \asim\dfareduces[a]\asim q'$ if, and only if, $a \oc \dfa{q} \Reduces \dfa{q}'$, or, more generally, that $q \asim\dfareduces[w]\asim q'$ if, and only if, $\rev{w} \oc \dfa{q} \Reduces \dfa{q}'$.
    Because bisimilar states have equal encodings (part~\ref{enum:ordered-rewriting:dfa-adequacy:1}) and bisimilarity is reflexive (\cref{??}), it suffices to show two stronger statements:
    \begin{enumerate*}
    \item that $q \dfareduces[w] q'$ implies $\rev{w} \oc \dfa{q} \Reduces \dfa{q}'$; and
    \item that $\rev{w} \oc \dfa{q} \Reduces \dfa{q}'$ implies $q \dfareduces[w]\asim q'$.
    \end{enumerate*}
    %
    We prove these in turn.
    %
    \begin{enumerate}
    %% Subpart (a)
    \item
      We shall prove that $q \dfareduces[w] q'$ implies $\rev{w} \oc \dfa{q} \Reduces \dfa{q}'$ by induction over the structure of word $w$.
      \begin{itemize}
      \item
        Consider the case of the empty word, $\emp$; we must show that $q = q'$ implies $\dfa{q} \Reduces \dfa{q}'$.
        Because the encoding is a function, this is immediate.
      \item
        Consider the case of a nonempty word, $a \wc w$; we must show that $q \dfareduces[a]\dfareduces[w] q'$ implies $\rev{w} \oc a \oc \dfa{q} \Reduces \dfa{q}'$.
        Let $q'_a$ be an $a$-successor of state $q$ that is itself $w$-succeeded by state $q'$.
        There exists, by definition of the encoding, a trace
        \begin{equation*}
          \rev{w} \oc a \oc \dfa{q}
            \Reduces \rev{w} \oc a \oc (a \limp \dfa{q}'_a)
            \reduces \rev{w} \oc \dfa{q}'_a
            \Reduces \dfa{q}'
          \,,
        \end{equation*}
        with the trace's tail justified by an appeal to the inductive hypothesis.
        % Because $q'$ is a $w$-successor of $q'_a$, an appeal to the inductive hypothesis yields a trace $\rev{w} \oc \dfa{q}'_a \Reduces \dfa{q}'$.
      \end{itemize}

      % Let $q'$ be an $a$-successor of state $q$.
      % There exists, by definition of the encoding, a trace
      % \begin{equation*}
      %   a \oc \dfa{q} \Reduces a \oc (a \limp \dfa{q}') \reduces \dfa{q}'
      % \,.
      % \end{equation*}

    %% Subpart (b)
    \item
      We shall prove that $\rev{w} \oc \dfa{q} \Reduces \dfa{q}'$ implies $q \dfareduces[w]\asim q'$ by induction over the structure of word $w$.
      \begin{itemize}
      \item
        Consider the case of the empty word, $\emp$;
        we must show that $\dfa{q} \Reduces \dfa{q}'$ implies $q \asim q'$.
        By \cref{lem:ordered-rewriting:dfa-traces}, $\dfa{q} \Reduces \dfa{q}'$ implies that $q$ and $q'$ have equal encodings.
        Part~\ref{enum:ordered-rewriting:dfa-adequacy:1} can then be used to establish that $q$ and $q'$ are bisimilar.
      \item
        Consider the case of a nonempty word, $a \wc w$;
        we must show that $\rev{w} \oc a \oc \dfa{q} \Reduces \dfa{q}'$ implies $q \dfareduces[a]\dfareduces[w]\asim q'$.
        By inversion\fixnote{Is this enough justification?}, the given trace can only begin by inputting $a$:
        \begin{equation*}
          \rev{w} \oc a \oc \dfa{q}
            \Reduces \rev{w} \oc a \oc (a \limp \dfa{q}'_a)
            \reduces \rev{w} \oc \dfa{q}'_a
            \Reduces \dfa{q}'
          \,,
        \end{equation*}
        where $q'_a$ is an $a$-successor of state $q$.
        An appeal to the inductive hypothesis on the trace's tail yields $q'_a \dfareduces[w]\asim q'$, and so the \ac{DFA} admits $q \dfareduces[a]\dfareduces[w]\asim q'$, as required.
      \end{itemize}
      % Assume that a trace $a \oc \dfa{q} \Reduces \dfa{q}'$ exists.
      % By the input lemma, $\dfa{q} \Reduces (a \limp A) \oc \octx'$ for some proposition $A$ and context $\octx'$ such that $A \oc \octx' \Reduces \dfa{q}'$.
      % Upon inversion of the trace from $\dfa{q}$, we conclude that $A = \dfa{q}'_a$, where $q'_a$ is an $a$-successor of $q$, and that $\octx'$ is empty -- in other words, we have a trace $\dfa{q}'_a \Reduces \dfa{q}'$.
      % Such a trace exists only if $\dfa{q}'_a = \dfa{q}'$.
      % By part~\ref{enum:ordered-rewriting:dfa-adequacy:1} of this \lcnamecref{thm:ordered-rewriting:dfa-adequacy-bisim}, it follows that $q'_a$ and $q'$ are bisimilar.
    \end{enumerate}

  %% Part three
  \item
    We shall prove that the final states are exactly those states $q$ such that $\emp \oc \dfa{q} \Reduces \one$.
    \begin{itemize}
    \item
      Let $q$ be a final state; accordingly, $\dfa{F}(q) = \one$.
      There exists, by definition of the encoding, a trace
      \begin{equation*}
        \emp \oc \dfa{q} \Reduces \emp \oc (\emp \limp \dfa{F}(q)) \reduces \dfa{F}(q) = \one
      \,.
      \end{equation*}

    \item
      Assume that a trace $\emp \oc \dfa{q} \Reduces \one$ exists.
      By inversion\fixnote{Is this enough justification?}, this trace can only begin by inputting $\emp$:
      \begin{equation*}
        \emp \oc \dfa{q} \Reduces \emp \oc (\emp \limp \dfa{F}(q)) \reduces \dfa{F}(q) \Reduces \one
      \,.
      \end{equation*}
      The tail of this trace, $\dfa{F}(q) \Reduces \one$, can exist only if $q$ is a final state.
    %
    \qedhere
    \end{itemize}

  % %% Part four
  % \item 
  %   We would like to prove that $q \asim\dfareduces[w]\asim q'$ if, and only if, $\rev{w} \oc \dfa{q} \Reduces \dfa{q}'$.
  %   Because bisimilar states have equal encodings (part~\ref{enum:ordered-rewriting:dfa-adequacy:1}) and bisimilarity is reflexive (\cref{??}), it suffices to show:
  %   \begin{enumerate*}
  %   \item that $q \dfareduces[w] q'$ implies $\rev{w} \oc \dfa{q} \Reduces \dfa{q}'$; and
  %   \item that $\rev{w} \oc \dfa{q} \Reduces \dfa{q}'$ implies $q \dfareduces[w]\asim q'$.
  %   \end{enumerate*}

  %   Both statements can be established by induction over the structure of word $w$.
  %   The latter proof is slightly more involved and deserves a bit of explanation.
  %   \begin{itemize}
  %   \item Consider the case in which $w$ is the empty word; we must show that $\dfa{q} \Reduces \dfa{q}'$ implies $q \asim q'$.
  %     By \cref{lem:ordered-rewriting:dfa-traces}, $\dfa{q} \Reduces \dfa{q}'$ implies that $\dfa{q} = \dfa{q}'$.
  %     Part~\ref{enum:ordered-rewriting:dfa-adequacy:1} can then be used to establish $q$ and $q'$ as bisimilar.

  %   \item Consider the case of a nonempty word, $a \wc w$.
  %     We must show that $\rev{w} \oc a \oc \dfa{q} \Reduces \dfa{q}'$ implies $q \dfareduces[a]\dfareduces[w]\asim q'$.
  %     By inversion, the given trace must begin by inputting $a$:
  %     \begin{equation*}
  %       \rev{w} \oc a \oc \dfa{q} \Reduces \rev{w} \oc a \oc (a \limp \dfa{q}'_a) \reduces \rev{w} \oc \dfa{q}'_a \Reduces \dfa{q}'
  %       \,,
  %     \end{equation*}
  %     where $q'_a$ is an $a$-successor of state $q$.
  %     Appealing to the inductive hypothesis on the trace's tail yields $q'_a \dfareduces[w]\asim q'$, and so $q \dfareduces[a]\dfareduces[w]\asim q'$, as required.
  %   %
  %   \qedhere
  %   \end{itemize}
  \end{enumerate}
\end{proof}


\subsection{Encoding \aclp*{NFA}?}

We would certainly be remiss if we did not attempt to generalize the rewriting specification of \acp{DFA} to one for their nondeterministic cousins.

Differently from \ac{DFA} states, \iac{NFA} state $q$ may have several nondeterministic successors for each input symbol $a$.
To encode the \ac{NFA} state $q$, all of its $a$-successors are collected in an alternative conjunction underneath the left-handed input of $a$.
Thus, the encoding of \iac{NFA} state $q$ becomes
\begin{equation*}
  \nfa{q} \defd
    \parens[size=auto]{\displaystyle
      \bigwith_{a \in \ialph}
        \parens[size=big]{a \limp \parens{\bigwith_{q'_a} \nfa{q}'_a}}
    }
    \with
    \parens[size=big]{\emp \limp \nfa{F}(q)}
  \,,
\end{equation*}
where $\nfa{F}(q)$ is defined as for \acp{DFA}.

The adjacent \lcnamecref{fig:ordered-rewriting:nfa-example}
\begin{marginfigure}
  \centering
  % \subfloat[][]{\label{fig:ordered-rewriting:nfa-example:nfa}%
    \begin{tikzpicture}
      \graph [automaton] {
        q_0
         -> ["a,b", loop above]
        q_0
         -> ["b"]
        q_1 [accepting]
         -> ["a,b"]
        q_2
         -> ["a,b", loop above]
        q_2;
      };
    \end{tikzpicture}
  % }

%   \subfloat[][]{\label{fig:ordered-rewriting:nfa-example:encoding}%
      $\!\begin{aligned}
        \nfa{q}_0 &\defd (a \limp \nfa{q}_0) \with \bigl(b \limp (\nfa{q}_0 \with \nfa{q}_1)\bigr) \with (\emp \limp \top) \\
        \nfa{q}_1 &\defd (a \limp \nfa{q}_2) \with (b \limp \nfa{q}_2) \with (\emp \limp \one) \\
        \nfa{q}_2 &\defd (a \limp \nfa{q}_2) \with (b \limp \nfa{q}_2) \with (\emp \limp \top)
      \end{aligned}$
%     }

  \caption{{fig:ordered-rewriting:nfa-example:nfa}~\Iac*{NFA} that accepts exactly those words, over the alphabet $\ialph = \set{a,b}$, that end with $b$; and {fig:ordered-rewriting:nfa-example:encoding}~its encoding}\label{fig:ordered-rewriting:nfa-example}
\end{marginfigure}%
recalls from \cref{ch:automata} \iac{NFA} that accepts exactly those words, over the alphabet $\ialph = \set{a,b}$, that end with $b$.
Using the above encoding of \acp{NFA}, ordered rewriting does indeed simulate this \ac{NFA}.
For example, just as there are transitions $q_0 \nfareduces[b] q_0$ and $q_0 \nfareduces[b] q_1$, there are traces
\begin{equation*}
  \begin{tikzcd}[
    cells={inner xsep=0.65ex,
           inner ysep=0.4ex},
         % nodes={draw},
    row sep=0em,
    column sep=scriptsize
  ]
    &[-0.2em] \nfa{q}_0
    \\
    b \oc \nfa{q}_0 \Reduces b \oc \bigl(b \limp (\nfa{q}_0 \with \nfa{q}_1)\bigr) \reduces \nfa{q}_0 \with \nfa{q}_1
      \urar[reduces, start anchor=east]
      \drar[reduces, start anchor=base east]
    \\
    & \nfa{q}_1
  \end{tikzcd}
\end{equation*}

Unfortunately, while it does simulate \ac{NFA} behavior, this encoding is not adequate.
Like \ac{DFA} states, \ac{NFA} states that have equal encodings are bisimilar.
% \begin{proof}
%   Define a relation $\mathord{\simu{R}} = \set{(q, s) \given \nfa{q} = \nfa{s}}$; we will show that $\simu{R}$ is a bisimulation.
%   \begin{itemize}
%   \item Assume that $s \simu{R}^{-1} q \nfareduces[a] q'_a$.
%     By definition, $a \oc \nfa{q} \Reduces \nfa{q}'_a$.
%     Because $\nfa{q} = \nfa{s}$, it follows that $s \nfareduces[a] s'_a$ for some state $s'_a$ such that $\nfa{q}'_a = \nfa{s}'_a$ -- that is, $q'_a \simu{R} s'_a$.
%     Thus, $s \nfareduces[a]\simu{R}^{-1} q'_a$.
%   \item Assume that $q \simu{R} s$.
%     It follows that $\nfa{F}(q) = \nfa{F}(s)$.
%     Thus, $q$ is an accepting state if and only if $s$ is.
%   \end{itemize}
% \end{proof}
However, for \acp{NFA}, the converse does not hold: bisimilar states do not necessarily have equal encodings.
%
\begin{falseclaim}
  Let $\aut{A} = (Q, ?, F)$ be \iac{NFA} over input alphabet $\ialph$.
  Then $q \asim s$ implies $\nfa{q} = \nfa{s}$, for all states $q$ and $s$.
\end{falseclaim}
%
\begin{proof}[Counterexample]
  Consider the \ac{NFA} and encoding depicted in the adjacent \lcnamecref{fig:ordered-rewriting:nfa-counterexample}.
  \begin{marginfigure}
    \begin{alignat*}{2}
      \begin{tikzpicture}
        \graph [automaton] {
          q_0 [accepting]
           -> ["a", loop above]
          q_0
           -> ["a", overlay]
          q_1 [accepting, overlay]
           -> ["a", loop above, overlay]
          q_1;
        };
      \end{tikzpicture}
      &\quad&&
      \\
      &\quad& \nfa{q}_0 &\defd \bigl(a \limp (\nfa{q}_0 \with \nfa{q}_1)\bigr) \with (\emp \limp \one) \\
      &\quad& \nfa{q}_1 &\defd (a \limp \nfa{q}_1) \with (\emp \limp \one)
    \end{alignat*}
    \caption{\Iac*{NFA} that accepts all finite words over the alphabet $\ialph = \set{a}$}\label{fig:ordered-rewriting:nfa-counterexample}
  \end{marginfigure}
  It is easy to verify that the relation $\set{q_1} \times \set{q_0,q_1}$ is a bisimulation; in particular, $q_1$ simulates the $q_0 \nfareduces[a] q_1$ transition by its self-loop, $q_1 \nfareduces[a] q_1$.
  Hence, $q_0$ and $s_0$ are bisimilar.
  %
  % These same \acp{NFA} are encoded by the following definitions.
  % \begin{align*}
  %   \nfa{q}_0 &\defd (a \limp \nfa{q}_0) \with (\emp \limp \one)
  % \shortintertext{and}
  %   \nfa{s}_0 &\defd \bigl(a \limp (\nfa{s}_0 \with \nfa{s}_1)\bigr) \with (\emp \limp \one) \\
  %   \nfa{s}_1 &\defd (a \limp \nfa{s}_1) \with (\emp \limp \one)
  % \end{align*}
  It is equally easy to verify, by unrolling the definitions used in the encoding, that $\nfa{q}_0 \neq \nfa{s}_0$.
\end{proof}

For \acp{DFA}, bisimilar states do have equal encodings because the inherent determinism \ac{DFA} bisimilarity is a rather fine-grained equivalence.
Because each \ac{DFA} state has exactly one successor for each input symbol
The additional flexibility entailed by nondeterminism

Once again, it would be possible to construct an adequate encoding, by tagging each state with a unique atom.
% with a stronger nominal character

For the moment, we will put aside the question of an adequate encoding of \acp{NFA}.

\subsection{Binary representation of natural numbers}

As a further example of ordered rewriting, consider a rewriting specification of binary counters: binary representations of natural numbers equipped with increment and decrement operations.

\paragraph*{Binary representations}
In this setting, we represent a natural number in binary by
% A binary representation of a natural number is
an ordered context that consists of a big-endian sequence of atoms $b_0$ and $b_1$, prefixed by the atom $e$; leading $b_0$s are permitted.
For example, both $\octx = e \oc b_1$ and $\octx' = e \oc b_0 \oc b_1$ are valid binary representations of the natural number $1$.

To be more precise, we inductively define a relation, $\aval{}{}$, 
% [between binary representations and the natural number [value]s that they represent.]
that assigns to each binary representation a unique natural number denotation.
% , defined inductively by the following rules.
% between ordered contexts and natural numbers that is inductively defined by the following rules.
When $\aval{\octx}{n}$, we say that $\octx$ denotes, or represents, natural number $n$ in binary.
%
\newcommand{\ooavalrules}{%
  \infer[\jrule{$e$-V}]{\aval{e}{0}}{}
  \and
  \infer[\jrule{$b_0$-V}]{\aval{\octx \oc b_0}{2n}}{
    \aval{\octx}{n}}
  \and
  \infer[\jrule{$b_1$-V}]{\aval{\octx \oc b_1}{2n+1}}{
    \aval{\octx}{n}}%
}%
\begin{inferences}
  \ooavalrules
\end{inferences}
% [In addition to assigning each binary representation a natural number value,]
Besides providing a denotational semantics of binary numbers, the $\aval{}{}$ relation also serves to implicitly characterize the well-formed binary numbers as those ordered contexts $\octx$ that form the relation's domain of definition.%
% Implicit in this definition is 
\footnote{Alternatively, the well-formed binary numbers could be described more explicitly by the grammar
% More precisely, the binary numbers are those contexts that are generated by the following grammar:
\begin{equation*}
  \octx \Coloneqq e \mid \octx \oc b_0 \mid \octx \oc b_1
  \,.
\end{equation*}%
}

These properties\fixnote{which properties?} of the $\aval{}{}$ relation are proved as the following adequacy \lcnamecref{thm:ordered-rewriting:binary-adequacy}.
%
\newcommand{\ooavaltheorem}{%
  \begin{theorem}[Adequacy of binary representations]\label{thm:ordered-rewriting:binary-adequacy}%
    \leavevmode
    \begin{thmdescription}
    \item[Functional]
      For every binary number $\octx$, there exists a unique natural number $n$ such that $\aval{\octx}{n}$.
      % [If $\aval{\octx}{n}$ and $\aval{\octx}{n'}$, then $n = n'$.]
    \item[Surjectivity]
      For every natural number $n$, there exists a binary number $\octx$ such that $\aval{\octx}{n}$.
      % Moreover, when the rule for $b_0$ is restricted to nonzero even numbers, the representation is unique.
    \item[Value]
      If $\aval{\octx}{n}$, then $\octx \nreduces$.
    \end{thmdescription}
  \end{theorem}%
}%
%
\ooavaltheorem
\begin{proof}
  The three claims may be proved by induction over the structure of $\octx$, and by induction on $n$, respectively.
\end{proof}

Notice that the above $\jrule{$e$-V}$ and $\jrule{$b_0$-V}$ rules overlap when the denotation\fixnote{represented natural number?} is $0$, giving rise to the leading $b_0$s that make the $\aval{}{}$ relation surjective:
for example, both $\aval{e \oc b_1}{1}$ and $\aval{e \oc b_0 \oc b_1}{1}$ hold.
However, if the rule for $b_0$ is restricted to \emph{nonzero} even numbers, then each natural number has a unique, canonical representation that is free of leading $b_0$s.%
\footnote{
  A restriction of the $b_0$ rule to nonzero even numbers is:
  \begin{equation*}
    \infer{\aval{\octx \oc b_0}{2n}}{
      \aval{\octx}{n} & \text{($n > 0$)}}
  \,.
  \end{equation*}
  The leading-$b_0$-free representations could alternatively be seen as the canonical representatives of the equivalence classes induced by the equivalence relation among binary numbers that have the same denotation: $\octx \equiv \octx'$ if $\aval{\octx}{n}$ and $\aval{\octx'}{n}$ for some $n$.}


% This leads to a nontrivial equivalence relation over binary numbers: binary numbers $\octx$ and $\octx'$ are equivalent up to leading $b_0$s if both $\octx$ and $\octx'$ represent the same $n$.
% % $\aval{\octx}{n}$ and $\aval{\octx}{n}$ for some $n$.
% Corresponding to the [...] of leading $b_0$s.

% Define a nontrivial equivalence relation on binary numbers. 

% If the rule for $\octx \oc b_0$ is restricted to apply to only strictly positive even numbers


% (\octx, n) R(~) (\octx', n) iff \octx ~ n and \octx' ~ n
% R(~) is an equivalence relation.
% ~/R(~) = {[(\octx, n)] | \octx ~ n}
% \octx ~/R(~) n iff \octx is leading-b0-free and \octx ~ n

% Let ~ be an equivalence relation over X.
% Let [-] : X -> X/~ be the surjection that maps to equivalence classes
% Let s : X/~ -> X be an injective function such that [s(c)] = c.

% This function describes an adequate representation because it forms a bijection (up to leading $b_0$s) between binary counters and natural numbers.
% %
% \begin{theorem}[Representational adequacy]
%   Up to leading $b_0$s, the $\aval{}{}$ relation is a bijection:
%   \begin{itemize}[noitemsep]
%   \item
%     For every binary counter $\octx$, there exists a unique natural number $n$ such that $\aval{\octx}{n}$.
%   \item
%     Conversely, for every natural number $n$, there exists a binary counter $\octx$, unique up to leading $b_0$s, such that $\aval{\octx}{n}$.
%   \item
%     For all binary counters $\octx$ and $\octx'$ that are syntactically equal modulo leading $b_0$s, $\aval{\octx}{n}$ if, and only if, $\aval{\octx'}{n}$.
%   \end{itemize}
% \end{theorem}
% %
% \begin{proof}
%   The two claims may be proved by induction over the structure of the binary counter $\octx$, and by induction on the natural number $n$, respectively.
% \end{proof}

% Alternatively, binary counters could be organized around leading-$b_0$--free representations.
% Leading-$b_0$--free representations form a retract within the binary counters, with the 
% \begin{equation*}
%   \begin{lgathered}
%     r(e) = e \\
%     r(\octx \oc b_0) =
%       \begin{cases*}
%         e & if $r(\octx) = e$ \\
%         r(\octx) \oc b_0 & otherwise
%       \end{cases*} \\
%     r(\octx \oc b_1) = r(\octx) \oc b_1
%   \end{lgathered}
% \end{equation*}

% Notice that the representations that are free of leading $b_0$s form a retract 

% \begin{inferences}
%   \infer{\aval{e}{0}}{}
%   \and
%   \infer{\aval{\octx \oc b_0}{2n+2}}{
%     \aval{\octx}{' n+1}}
%   \and
%   \infer{\aval{\octx \oc b_1}{2n+1}}{
%     \aval{\octx}{n}}
%   \and
%   \infer{\aval{\octx \oc b_0}{' 2n+2}}{
%     \aval{\octx}{' n+1}}
%   \and
%   \infer{\aval{\octx \oc b_1}{' 2n+1}}{
%     \aval{\octx}{n}}
% \end{inferences}

% % bin = +{ e: 1, b0: pos, b1: bin }
% % pos = +{ b0: pos, b1: bin }
% %
% % bin = &{ i: pos, d: +{ z: 1, s: bin  } }
% % pos = &{ i: pos, d: bin }
% %
% % e = (e * b1 / i) & (z / d) & (e * z / h)
% % b0 = (b1 / i) & (d * (s \ b1) * s) & (b0 * s / h)
% % b1 = (i * b0 / i) & (h * b1' * s / d) & (b1 * s / h)
% % b1' = (z \ 1) & (s \ b0)

% % C b0 d --> C d (s \ b1) s --> C' b1 s
% % 2(n+1) --> 2(n+1)-1 + 1 --> 2n+1 + 1

% % e b1 d --> e d b1' s --> z b1' s --> e s
% % 2(0)+1 --> 2(0) + 1 --> 2(0) + 1 --> 0 + 1

% % C b1 d --> C d b1' s --> C' s b1' s --> C' i b0 s
% % 2(n+1)+1 --> 2(n+1) + 1 --> 2(n + 1) + 1 --> 
% \begin{equation*}
%   \begin{lgathered}
%     e \defd (e \fuse b_1 \pmir i) \with (z \pmir d) \\
%     b_0 \defd (b_1 \pmir i) \with (d \fuse b_1 \fuse s \pmir d) \\
%     b_1 \defd (i \fuse b_0 \pmir i) \with (d \fuse b'_1 \pmir d) \\
%     p_0 \defd () \with (d \fuse b_1 \pmir d) \\
%     b'_1 \defd (z \limp e \fuse s) \with (s \limp i \fuse b_0 \fuse s)
%   \end{lgathered}
% \end{equation*}

% \begin{equation*}
%   \begin{lgathered}
%     e \defd (e \fuse b_1 \pmir i) \with (z \pmir d) \\
%     b_0 \defd (b_1 \pmir i) \with (d \fuse b'_0 \pmir d) \with (c \fuse b^c_0 \pmir c) \\
%     b^c_0 \defd (z \limp z) \with (s \limp b_0 \fuse s) \\
%     b'_0 \defd (z \limp z) \with (s \limp b_1 \fuse s) \\
%     b_1 \defd (b_0 \pmir i) \with (d \fuse ((z \limp e \fuse s) \with (s \limp b_0 \fuse s)) \pmir d)
%   \end{lgathered}
% \end{equation*}

% % e b1 d --> e s
% % e b1 b1 d --> e b1 d ... --> e s ... --> e i b0 s

% % 2(n+1)+1 - 1 --> 2(n+1-1)+1

% % e b1 d --> e d b1' --> z b1' --> e s
% % e b1 b0 d --> e s b0' --> e b1 s
% % e b1 b0 b0 d --> e b1 s b0' --> e b1 b1 s
% % e b1 b1 d --> e s b1' --> e i b0 s --> e b1 b0 s
% % e b1 b0 b1 d --> e b1 s b1' --> e b1 i b0 s --> e b1 b0 b0 s

% \begin{inferences}
%   \infer{\aval{e}{e}}{}
%   \and
%   \infer{\aval{\octx \oc b_0}{e}}{
%     \aval{\octx}{e}}
%   \and
%   \infer{\aval{\octx \oc b_0}{\octx' \oc b_0}}{
%     \aval{\octx}{\octx'}}
%   \and
%   \infer{\aval{\octx \oc b_1}{\octx' \oc b_1}}{
%     \aval{\octx}{\octx'}}
% \end{inferences}
% By analogy with functional computation, the ordered contexts $\octx$ that appear in this relation will serve as values -- end results of computations.

% The relation $\aval{}{}$ defines an adequate representation because it is, in fact, a bijection (up to leading $b_0$s) between ordered contexts and natural numbers.

% If we restrict our attention to counters $\octx$ that are free of leading $b_0$s, then the $\aval{}{}$ relation is a bijection with the natural numbers.
% %
% Right-unique: $\aval{\octx}{n}$ and $\aval{\octx}{n'}$, then $n = n'$.
% Left-total:

% \begin{theorem}[Representational adequacy]
%   For all natural numbers $n$, there exists a context $\octx$, unique up to leading $b_0$s, such that $\aval{\octx}{n}$.
%   Moreover, the relation $\aval{}{}$ is functional -- \ie, if $\aval{\octx}{n}$ and $\aval{\octx}{n'}$, then $n = n'$.
% \end{theorem}
% \begin{proof}
%   The first part follows by induction on the natural number $n$; the second part follows by induction on the structure of the context $\octx$.
% \end{proof}
% %
% In other words, the binary representations that are free of leading $b_0$s form a retract.


\paragraph{An increment operation}
To use ordered rewriting to describe an increment operation on binary representations, we introduce a new, uninterpreted atom $i$ that will serve as an increment instruction.

Given a binary number $\octx$ that represents $n$, we may append $i$ to form a computational state, $\octx \oc i$.
For $i$ to adequately represent the increment operation, the state $\octx \oc i$ must meet two conditions, captured by the following global desiderata:
% \lcnamecref{thm:increment-structural-adequacy}.
%global desiderata: 
\begin{theorem}\label{thm:increment-structural-adequacy}
  Let $\octx$ be a binary representation of $n$.
  Then:
  \begin{itemize}[nosep]
  \item
    \emph{some} computation from $\octx \oc i$ results in a binary representation of $n+1$ -- that is, $\octx \oc i \Reduces\aval{}{n+1}$; and
  \item
    \emph{any} computation from $\octx \oc i$ results in a binary representation of $n+1$ -- that is, $\octx \oc i \Reduces\aval{}{n'}$ only if $n' = n+1$.%
    \fixnote{Compare \enquote{If $\octx \oc i \Reduces \octx'$, then $\octx' \Reduces\aval{}{n+1}$.}}
    % Moreover, computations preserve an absence of leading $b_0$s.\fixnote{Is this necessary?}
  \end{itemize}
\end{theorem}
\noindent
For example, because $e \oc b_1$ denotes $1$, a computation $e \oc b_1 \oc i \Reduces\aval{}{2}$ must exist; moreover, every computation $e \oc b_1 \oc i \Reduces\aval{}{n'}$ must satisfy $n' = 2$.

\newthought{To implement these} global desiderata locally, the previously uninterpreted atoms $e$, $b_0$, and $b_1$ are now given mutually recursive definitions that describe how they may be rewritten when the increment instruction, $i$, is encountered.
\begin{description}[font=\color{structure}]
\item[$e \defd e \fuse b_1 \pmir i$]
  To increment $e$, append $b_1$ as a new most\fixnote{or least?} significant bit, resulting in $e \oc b_1$;
  the rewriting sequence $e \oc i \reduces e \fuse b_1 \reduces e \oc b_1$ is entailed by this definition.
\item[$b_0 \defd b_1 \pmir i$]
  To increment a binary number ending in $b_0$, flip that bit to $b_1$;
  the entailed rewriting step is $\octx \oc b_0 \oc i \reduces \octx \oc b_1$.
\item[$b_1 \defd i \fuse b_0 \pmir i$]
  To increment a binary number ending in $b_1$, flip that bit to $b_0$ and carry the increment over to the more significant bits;
  the entailed rewriting sequence is $\octx \oc b_1 \oc i \reduces \octx \oc (i \fuse b_0) \reduces \octx \oc i \oc b_0$.
\end{description}
Comfortingly, $1+1 = 2$: that is, a computation
% $e \oc b_1 \oc i \Reduces \aval{e \oc b_1 \oc b_0}{2}$, namely:
$e \oc b_1 \oc i \Reduces e \oc i \oc b_0 \Reduces e \oc b_1 \oc b_0$ indeed exists.

\newthought{It should also} be possible to permit several increments at once, such as in $e \oc b_1 \oc i \oc i$.
We could, of course, handle the increments sequentially from left to right, fully computing a binary value before moving on to the subsequent increment:
\begin{equation*}
  e \oc b_1 \oc i \oc i \Reduces e \oc b_1 \oc b_0 \oc i \reduces e \oc b_1 \oc b_1
  \,.
\end{equation*}
However, a strictly sequential treatment of increments would be rather disappointing.
Because the ordered rewriting framework\fixnote{wc?} is inherently concurrent, a truly concurrent treatment of multiple increments would be far more satisfying.

For example, consider the several computations of $(1+1)+1 = 3$ from $e \oc b_1 \oc i \oc i$:
\begin{equation*}
  \begin{tikzcd}[
    cells={inner xsep=0.65ex,
           inner ysep=0.4ex},
    row sep=0em,
    column sep=scriptsize
  ]
    &[-0.2em]
    e \oc b_1 \oc b_0 \oc i
      \drar[reduces, start anchor=base east,
                     end anchor=north west]
    &[-0.2em]
    \\
    e \oc b_1 \oc i \oc i \Reduces e \oc i \oc b_0 \oc i
      \urar[Reduces, start anchor=north east,
                     end anchor=base west]
      \ar[Reduces, gray, dashed]{rr}
      \drar[reduces, start anchor=base east,
                     end anchor=west]
    &&
    e \oc b_1 \oc b_1
    \\
    &
    e \oc i \oc b_1
      \urar[Reduces, start anchor=east,
                     end anchor=base west]
    &
  \end{tikzcd}
\end{equation*}
In other words, once the leftmost increment is carried past the least significant bit, the two increments can be processed concurrently -- the increments' rewriting steps can be interleaved, with no observable difference between the various interleavings.
We can even abstract from the interleavings by writing simply $e \oc i \oc b_0 \oc i \Reduces e \oc b_1 \oc b_1$.

Unfortunately, a concurrent treatment of increments falls outside the domain of \cref{??}.
Intermediate computational states, such as ...,
\begin{equation*}
  e \oc i \oc b_0 \oc i \reduces e \oc i \oc b_1
\end{equation*}
because $e \oc i \oc b_0$ is simply not a binary value.
An adequacy theorem stronger than \cref{??} is needed.

The situation here is roughly analogous to the desire, in a functional language, for stronger metatheorems than a big-step, natural sematics admits, and we adopt a similar solution.

\newthought{To this end}, we define a binary relation, $\ainc{}{}$, that assigns a natural number denotation to each intermediate computational state, not only to the terminal values as $\aval{}{}$ did..%
\footnote{Like the $\aval{}{}$ relation does for values, the $\ainc{}{}$ relation also serves to implicitly characterize the valid intermediate states as those contexts that form the relation's domain of definition.
As with values, the valid intermediate states could also be enumerated more explicitly and syntactically with a grammar:
\begin{equation*}
  \octx \Coloneqq e \mid \octx \oc b_0 \mid \octx \oc b_1 \mid \octx \oc i \mid e \fuse b_1 \mid \octx \oc (i \fuse b_0)
\end{equation*}}%
%
\newcommand{\aincrules}{%
  \infer[\jrule{$e$-I}]{\ainc{e}{0}}{}
  \and
  \infer[\jrule{$b_0$-I}]{\ainc{\octx \oc b_0}{2n}}{
    \ainc{\octx}{n}}
  \and
  \infer[\jrule{$b_1$-I}]{\ainc{\octx \oc b_1}{2n+1}}{
    \ainc{\octx}{n}}
  \and
  \infer[\jrule{$i$-I}]{\ainc{\octx \oc i}{n+1}}{
    \ainc{\octx}{n}}
  \\
  \infer[\jrule{$\fuse_1$-I}]{\ainc{e \fuse b_1}{1}}{}
  \and
  \infer[\jrule{$\fuse_2$-I}]{\ainc{\octx \oc (i \fuse b_0)}{2(n+1)}}{
    \ainc{\octx}{n}}%
}%
%
\begin{inferences}
  \aincrules
\end{inferences}
Binary values should themselves be valid, terminal computational states, so the first three rules are carried over from the $\aval{}{}$ relation.
The $\jrule{$i$-I}$ rule allows multiple increment instructions to be interspersed throughout the state.
Lastly, because the atomicity of ordered rewriting steps is very fine-grained, the $\jrule{$\fuse_1$-I}$ and $\jrule{$\fuse_2$-I}$ rules are needed to completely describe the valid intermediate states and their denotations.
For instance, the state $e \oc i$ first rewrites to the intermediate $e \fuse b_1$ before eventually rewriting to $e \oc b_1$; the state $\octx \oc (i \fuse b_0)$ has a similar status.

With this $\ainc{}{}$ relation in hand, we can now prove a stronger, small-step adequacy theorem.
%
\newcommand{\smallincadequacytheorem}{%
\begin{theorem}[Small-step adequacy of increments]%
  \leavevmode
  \begin{thmdescription}
  \item[Value soundness]
    If $\aval{\octx}{n}$, then $\ainc{\octx}{n}$ and $\octx \nreduces$.
  \item[Preservation]
    If $\ainc{\octx}{n}$ and $\octx \reduces \octx'$, then $\ainc{\octx'}{n}$.
  \item[Progress]
    If $\ainc{\octx}{n}$, then either
    \begin{itemize*}[
      mode=unboxed, label=, afterlabel=,
      before=\unskip:\space,
      itemjoin=;\space, itemjoin*=; or\space%
    ]
    \item $\octx \reduces \octx'$ for some $\octx'$; or
    \item $\aval{\octx}{n}$.\fixnote{Compare with \enquote{If $\ainc{\octx}{n}$, then $\aval{\octx}{n}$ if, and only if, $\octx \nreduces$.}}
    \end{itemize*}
  \item[Termination]
    If $\ainc{\octx}{n}$, then every rewriting sequence from $\octx$ is finite.
  \end{thmdescription}
\end{theorem}%
}%
%
\smallincadequacytheorem
\begin{proof}
  Each part is proved separately.
  \begin{description}[
    parsep=0pt, listparindent=\parindent,
    labelsep=0.35em
  ]
  \item[Value soundness]
    can be proved by structural induction on the derivation of $\aval{\octx}{n}$.
  \item[Preservation and progress]
    can likewise be proved by structural induction on the derivation of $\ainc{\octx}{n}$.
    In particular, the $e \fuse b_1$ and $\octx \oc (i \fuse b_0)$ rules
  \item[Termination]
    can be proved using an explicit termination measure, $\card{\octx}$, that is strictly decreasing across each rewriting, $\octx \reduces \octx'$.
    Specifically, we use a measure (see the adjacent \lcnamecref{fig:ordered-rewriting:binary-counter:measure}),
    % For valid states $\octx$, we define a measure $\card{\octx}$ that is strictly decreasing across each rewriting $\octx \reduces \octx'$ (see the adjacent \lcnamecref{fig:ordered-rewriting:binary-counter:measure}).
    \begin{marginfigure}
      \begin{equation*}
        \begin{lgathered}[t]
          \card{e} = 0 \\
          \card{\octx \oc b_0} = \card{\octx} \\
          \card{\octx \oc b_1} = \card{\octx} + 2 \\
          \card{\octx \oc i} = \card{\octx} + 4
        \end{lgathered}
        \qquad
        \begin{lgathered}[t]
          \card{e \fuse b_1} = 3 \\
          \card{\octx \oc (i \fuse b_0)} = \card{\octx} + 5
        \end{lgathered}
      \end{equation*}
      \caption{A termination measure, adapted from the standard amortized work analysis of increment for binary counters}\label{fig:ordered-rewriting:binary-counter:measure}
    \end{marginfigure}%
    adapted from the standard amortized work analysis of increment for binary counters\autocite{??}, for which $\octx \reduces \octx'$ implies $\card{\octx} > \card{\octx'}$.
    % That is, if $\octx$ is a valid state and $\octx \reduces \octx'$, then $\card{\octx} > \card{\octx'}$.
    Because the measure is always nonnegative, only finitely many such rewritings can occur.

    As an example case, consider the intermediate state $\octx \oc b_0 \oc i$ and its rewriting $\octx \oc b_0 \oc i \reduces \octx \oc b_1$.
    It follows that $\card{\octx \oc b_0 \oc i} = \card{\octx} + 4 > \card{\octx} + 2 = \card{\octx \oc b_1}$.
  \qedhere
  \end{description}
\end{proof}

\begin{corollary}[Big-step adequacy of increments]
  \leavevmode
  \begin{thmdescription}
  \item[Evaluation]
    If $\ainc{\octx}{n}$, then $\octx \Reduces\aval{}{n}$.
    In particular, if $\aval{\octx}{n}$, then $\octx \oc i \Reduces\aval{}{n+1}$.
  \item[Preservation]
    If $\ainc{\octx}{n}$ and $\octx \Reduces \octx'$, then $\ainc{\octx'}{n}$.
    In particular, if $\aval{\octx}{n}$ and $\octx \oc i \Reduces\aval{}{n'}$, then $n' = n+1$.
  \end{thmdescription}
\end{corollary}
\begin{proof}
  The two parts are proved separately.
  \begin{description}[labelsep=0.35em]
  \item[Evaluation] can be proved by repeatedly appealing to the progress and preservation results\parencref{??}.
    By the accompanying termination result, a binary value must eventually be reached.
  \item[Preservation] can be proved by structural induction on the given rewriting sequence.
  %
  \qedhere
  \end{description}
\end{proof}

\newthought{But, of course}, a few isolated examples do not make a proof.



By analogy with functional programming, the above adequacy conditions can be seen as stating evaluation and termination results for a big-step, evaluation semantics of increments, with $\aval{\octx}{n}$ acting as a kind of typing judgment -- admittedly, a very precise one.

In functional programming, big-step results like these are usually proved by first providing a small-step operational semantics, then characterizing the valid intermediate states that arise with small steps, and finally establishing type preservation, progress, and termination results for the small-step semantics.
We will adopt the same proof strategy here.

In this case, the small-step operational semantics already exists -- it is simply the individual rewriting steps entailed by the definitions of $e$, $b_0$, and $b_1$.
So our first task is to characterize the valid intermediate states that arise during a computation.
To this end, we define a binary relation, $\ainc{}{}$, that, like the $\aval{}{}$ relation, serves the dual purposes of enumerating the valid intermediate states and assigning to each state a natural number denotation.%
\footnote{As with values, we could also choose to enumerate the valid immediate states more explicitly and syntactically with a grammar:
  \begin{equation*}
    \octx \Coloneqq e \mid \octx \oc b_0 \mid \octx \oc b_1 \mid \octx \oc i \mid e \fuse b_1 \mid \octx \oc (i \fuse b_0)
  \end{equation*}}
% To this end, we define a binary relation, $\ainc{}{}$, between computational states and the natural numbers that they represent;
\begin{inferences}
  \infer{\ainc{e}{0}}{}
  \and
  \infer{\ainc{\octx \oc b_0}{2n}}{
    \ainc{\octx}{n}}
  \and
  \infer{\ainc{\octx \oc b_1}{2n+1}}{
    \ainc{\octx}{n}}
  \and
  \infer{\ainc{\octx \oc i}{n+1}}{
    \ainc{\octx}{n}}
  \\
  \infer{\ainc{e \fuse b_1}{1}}{}
  \and
  \infer{\ainc{\octx \oc (i \fuse b_0)}{2(n+1)}}{
    \ainc{\octx}{n}}
\end{inferences}
Binary values should themselves be valid, terminal computational states, so the first three rules are carried over from the $\aval{}{}$ relation.
The fourth rule, involving $i$, allows multiple increments to be interspersed throughout the counter.

Because ordered rewriting steps are quite fine-grained, two final rules are needed to completely describe the valid intermediate states and their denotations.
For instance, the state $e \oc i$ first rewrites to $e \fuse b_1$ before eventually rewriting to $e \oc b_1$.



Having characterized the valid intermediate states, we may state and prove the small-step adequacy of increments: preservation, progress, and termination.
%
\begin{theorem}[Small-step adequacy of increments]%
  \leavevmode
  \begin{thmdescription}[nosep]
  \item[Value inclusion]
    If $\aval{\octx}{n}$, then $\ainc{\octx}{n}$.
  \item[Preservation]
    If $\ainc{\octx}{n}$ and $\octx \reduces \octx'$, then $\ainc{\octx'}{n}$.
  \item[Progress]
    If $\ainc{\octx}{n}$, then either
    \begin{itemize*}[
      mode=unboxed, label=, afterlabel=,
      before=\unskip:\space,
      itemjoin=;\space, itemjoin*=; or\space%
    ]
    \item $\octx \reduces \octx'$ for some $\octx'$; or
    \item $\octx \nreduces$ and $\aval{\octx}{n}$.
    \end{itemize*}
  \item[Termination]
    If $\ainc{\octx}{n}$, then every rewriting sequence from $\octx$ is finite.
  \end{thmdescription}
\end{theorem}
%
\begin{proof}
  Each part is proved separately.
  \begin{description}[
    parsep=0pt, listparindent=\parindent,
    labelsep=0.35em
  ]
  \item[Value inclusion]
    can be proved by structural induction on the derivation of $\aval{\octx}{n}$.
  \item[Preservation and progress]
    can likewise be proved by structural induction on the derivation of $\ainc{\octx}{n}$.
    In particular, the $e \fuse b_1$ and $\octx \oc (i \fuse b_0)$ rules
  \item[Termination]
    can be proved using an explicit termination measure, $\card{\octx}$, that is strictly decreasing across each rewriting, $\octx \reduces \octx'$.
    Specifically, we use a measure (see the adjacent \lcnamecref{fig:ordered-rewriting:binary-counter:measure}),
    % For valid states $\octx$, we define a measure $\card{\octx}$ that is strictly decreasing across each rewriting $\octx \reduces \octx'$ (see the adjacent \lcnamecref{fig:ordered-rewriting:binary-counter:measure}).
    \begin{marginfigure}
      \begin{equation*}
        \begin{lgathered}[t]
          \card{e} = 0 \\
          \card{\octx \oc b_0} = \card{\octx} \\
          \card{\octx \oc b_1} = \card{\octx} + 2 \\
          \card{\octx \oc i} = \card{\octx} + 4
        \end{lgathered}
        \qquad
        \begin{lgathered}[t]
          \card{e \fuse b_1} = 3 \\
          \card{\octx \oc (i \fuse b_0)} = \card{\octx} + 5
        \end{lgathered}
      \end{equation*}
      \caption{A termination measure, adapted from the standard amortized work analysis of increment for binary counters}\label{fig:ordered-rewriting:binary-counter:measure}
    \end{marginfigure}%
    adapted from the standard amortized work analysis of increment for binary counters\autocite{??}, for which $\octx \reduces \octx'$ implies $\card{\octx} > \card{\octx'}$.
    % That is, if $\octx$ is a valid state and $\octx \reduces \octx'$, then $\card{\octx} > \card{\octx'}$.
    Because the measure is always nonnegative, only finitely many such rewritings can occur.

    As an example case, consider the intermediate state $\octx \oc b_0 \oc i$ and its rewriting $\octx \oc b_0 \oc i \reduces \octx \oc b_1$.
    It follows that $\card{\octx \oc b_0 \oc i} = \card{\octx} + 4 > \card{\octx} + 2 = \card{\octx \oc b_1}$.
  \qedhere
  \end{description}
\end{proof}

\begin{theorem}[Big-step adequacy of increments]\leavevmode
  \begin{thmdescription}
  \item[Preservation]
    If $\ainc{\octx}{n}$ and $\octx \Reduces\aval{}{n'}$, then $n' = n$.
  \item[Termination?]
    If $\ainc{\octx}{n}$, then $\octx \Reduces\aval{}{n}$.
  \end{thmdescription}
\end{theorem}
\begin{proof}
  Both parts are consequences of the small-step adequacy of increments \parencref{??}.
  \begin{description}
  \item[Preservation]
    is proved by structural induction on the given rewriting sequence.
    The base case follows [...] by an inner structural induction on the derivation of $\aval{\octx}{n'}$.
    The inductive case can be proved by first appealing to small-step preservation \parencref{??} and then to the inductive hypothesis.
  \item[Termination?]
    is proved by repeatedly appealing to small-step progress \parencref{??}.
    The small-step termination [...] \parencref{??} ensures that a value will be reached after finitely many such appeals.
  %
  \qedhere
  \end{description}
\end{proof}

\begin{corollary}[Structural adequacy of increments]
  If $\aval{\octx}{n}$, then $\octx \oc i \Reduces\aval{}{n'}$ if, and only if, $n' = n+1$.
\end{corollary}



% For example, incrementing [a representation of] $1$ should result in [a representation of] $2$, as evidenced by a trace $e \oc b_1 \oc i \Reduces e \oc b_1 \oc b_0$.

% Given a binary number $\octx$ that represents $n$, we may append $i$ to form a computational state, $\octx \oc i$, that should compute a binary representation of $n+1$ and thereby increment the number.
% For example, incrementing [a representation of] $1$ should result in [a representation of] $2$, as evidenced by a trace $e \oc b_1 \oc i \Reduces e \oc b_1 \oc b_0$.

% Conversely, any computation

% To describe\fixnote{implement?} the increment operation using ordered rewriting, the previously uninterpreted atoms $e$, $b_0$, and $b_1$ are now given mutually recursive definitions that describe how they may be rewritten when $i$ is encountered.
% \begin{description}[font=\color{structure}]
% \item[$e \defd e \fuse b_1 \pmir i$]
%   To increment $e$, append $b_1$ as a most significant bit, resulting in $e \oc b_1$.
% \item[$b_0 \defd b_1 \pmir i$]
%   To increment a binary number that has $b_0$ as its least significant bit, simply flip that bit to $b_1$.
% \item[$b_1 \defd i \fuse b_0 \pmir i$]
%   To increment a binary number that has $b_1$ as its least significant bit, flip that bit to $b_0$ and carry the increment over to the more significant bits.
% \end{description}


As an example computation, consider incrementing $e \oc b_1$ twice, as captured by the state $e \oc b_1 \oc i \oc i$.
\begin{equation*}
  \begin{tikzcd}[
    cells={inner xsep=0.65ex,
           inner ysep=0.4ex},
    row sep=0em,
    column sep=scriptsize
  ]
    &[-0.2em]
    e \oc b_1 \oc b_0 \oc i
      \drar[reduces, start anchor=base east,
                     end anchor=west]
    &[-0.2em]
    \\
    e \oc b_1 \oc i \oc i \Reduces e \oc i \oc b_0 \oc i
      \urar[Reduces, start anchor=east,
                     end anchor=base west]
      \drar[reduces, start anchor=base east,
                     end anchor=west]
    &&
    e \oc b_1 \oc b_1
    \\
    &
    e \oc i \oc b_1
      \urar[Reduces, start anchor=east,
                     end anchor=base west]
    &
  \end{tikzcd}
\end{equation*}

First, processing of the leftmost increment begins: the least significant bit is flipped, and the increment is carried over to the more significant bits.
This corresponds to the reduction $e \oc b_1 \oc i \oc i \Reduces e \oc i \oc b_0 \oc i$.
Next, either of the two remaining increments may be processed -- that is, either $e \oc i \oc b_0 \oc i \Reduces e \oc b_1 \oc b_0 \oc i$ or $e \oc i \oc b_0 \oc i \Reduces e \oc i \oc b_1$.


% Conversely, any complete computation from $\octx \oc i$ must have as its result a binary rrepresentation of $n+1$.

% These two properties ensure that the atom $i$ adequately characterizes an increment operation:
% \begin{itemize}
% \item 
% \end{itemize}

% For this to be an adequate description of an increment operation, it should satisfy two desiderata:
% \begin{enumerate*}
% \item
% \end{enumerate*}
% Formally, these desiderata are captured by the following adequacy theorem:
% \begin{itemize}
% \item If $\aval{\octx}{n}$, then $\octx \oc i \Reduces\aval{}{n+1}$.
% \item If $\aval{\octx}{n}$, then $\octx \oc i \Reduces\aval{}{n'}$ implies $n' = n+1$.
% \end{itemize}

% By analogy with functional programming, ...

% Appending $i$ to a counter will initiate an increment, with ordered rewriting used to compute, step by step, a binary representation of the incremented value.



% \begin{itemize}
% \item
%   If counter $\octx$ represents $n$, then $\octx \oc i$ can compute a representation of $n+1$ and, conversely, any computation from $\octx \oc i$ results in a representation of $n+1$.
%   That is, if $\aval{\octx}{n}$, then $\octx \oc i \Reduces\aval{}{n'}$ if, and only if, $n' = n+1$.
% \end{itemize}

% \begin{equation*}
%   \octx \Coloneqq e \mid \octx \oc b_0 \mid \octx \oc b_1 \mid \octx \oc i
% \end{equation*}
% The binary counters

% If counter $\octx$ represents $n$, then $\octx \oc i$ should compute to a representation of $n+1$.
% If $\octx$ represents $n$ and $\octx \oc i$ can reduce to $n'$, then $n' = n+1$.
% \begin{itemize}
% \item 
% \end{itemize}

% The basic idea is to assign $e$, $b_0$, and $b_1$ recursive definitions that enable them to interact with these atoms $i$.

% The basic idea is to This atom is appended to a counter to initiate an increment 
% By appending this atom to a counter, This atom is appended to a counter
% ordered rewriting of 
% Because of these increments,
% To initiate an increment of a counter $\octx$, we simply append an uninterpreted atom $i$ to the counter; the atom $i$
% %
% % \begin{desiderata*}[Computational adequacy -- increments]\label{des:ordered-rewriting:increments}\leavevmode
%   \begin{itemize}[noitemsep]
%   \item If $\aval{\octx}{n}$ and $\octx \oc i \Reduces\aval{}{n'}$, then $n' = n+1$.
%   \item In addition, if $\aval{\octx}{n}$, then $\octx \oc i \Reduces\aval{}{n+1}$.
%   \end{itemize}
% % \end{desiderata*}



% Because of the new increment operation, the previously uninterpreted atoms $e$, $b_0$, and $b_1$ are now given mutually recursive definitions that describe how they may be rewritten when encountering $i$:
% % \begin{equation*}
% %   \begin{lgathered}
% %     e \defd e \fuse b_1 \pmir i \\
% %     b_0 \defd b_1 \pmir i \\
% %     b_1 \defd i \fuse b_0 \pmir i
% %   \end{lgathered}
% % \end{equation*}
% \begin{description}[font=\color{structure}]
% \item[$e \defd e \fuse b_1 \pmir i$]
%   To increment $e$, append $b_1$ as a most significant bit, resulting in $e \oc b_1$.
% \item[$b_0 \defd b_1 \pmir i$]
%   To increment a binary number that has $b_0$ as its least significant bit, simply flip that bit to $b_1$.
% \item[$b_1 \defd i \fuse b_0 \pmir i$]
%   To increment a binary number that has $b_1$ as its least significant bit, flip that bit to $b_0$ and carry the increment over to the more significant bits.
% \end{description}

% \begin{description}
% \item[$e \defd e \fuse b_1 \pmir i$]
%   To increment the counter $e$, which represents $0$, introduce $b_1$ as a new most significant bit, resulting in the counter $e \oc b_1$, which represents $1$.
%   That is, because $\aval{e}{0}$, there should exist a trace $e \oc i \Reduces \aval{e \oc b_1}{1}$.
%   % Having started at value $0$ (\ie, $\aval{e}{0}$), an increment results in value $1$ (\ie, $\aval{e \oc b_1}{1}$).
% \item[$b_0 \defd b_1 \pmir i$]
%   Because $\aval{\octx \oc b_0}{2n}$ when $\aval{\octx}{n}$, there should exist a trace $\octx \oc b_0 \oc i \Reduces \aval{\octx \oc b_1}{2n+1}$.
%   To increment a counter that ends with least significant bit $b_0$, simply flip that bit to $b_1$.
%   That is, $\octx \oc b_0 \oc i \reduces \octx \oc b_1$.
%   % Having started at value $2n$ (\ie, $\aval{\octx \oc b_0}{2n}$ with $\aval{\octx}{n}$), an increment results in value $2n+1$ (\ie, $\aval{\octx \oc b_1}{2n+1}$).
% \item[$b_1 \defd i \fuse b_0 \pmir i$]
%   Because $\aval{\octx \oc b_1}{2n+1}$ when $\aval{\octx}{n}$, there should exist a trace $\octx \oc b_1 \oc i \Reduces \aval{\octx' \oc b_0}{2n+2}$, provided that there exists a trace $\octx \oc i \Reduces \aval{\octx'}{n+1}$.
%   To increment a counter that ends with least significant bit $b_1$, flip that bit to $b_0$ and propagate the increment to the more significant bits as a carry.
%   That is, $\octx \oc b_1 \oc i \Reduces \octx \oc i \oc b_0$.
%   % Having started at value $2n+1$ (\ie, $\cval{\octx \oc b_1} = 2\cval{\octx}+1$), an increment results in value $2n+2 = 2(n+1)$ (\ie, $\cval{\octx \oc i \oc b_0} = 2\cval{\octx}+1$).
% \end{description}

% As an example computation, consider incrementing $e \oc b_1$ twice, as captured by the state $e \oc b_1 \oc i \oc i$.
% \begin{equation*}
%   \begin{tikzcd}[
%     cells={inner xsep=0.65ex,
%            inner ysep=0.4ex},
%     row sep=0em,
%     column sep=scriptsize
%   ]
%     &[-0.2em]
%     e \oc b_1 \oc b_0 \oc i
%       \drar[reduces, start anchor=base east,
%                      end anchor=west]
%     &[-0.2em]
%     \\
%     e \oc b_1 \oc i \oc i \Reduces e \oc i \oc b_0 \oc i
%       \urar[Reduces, start anchor=east,
%                      end anchor=base west]
%       \drar[reduces, start anchor=base east,
%                      end anchor=west]
%     &&
%     e \oc b_1 \oc b_1
%     \\
%     &
%     e \oc i \oc b_1
%       \urar[Reduces, start anchor=east,
%                      end anchor=base west]
%     &
%   \end{tikzcd}
% \end{equation*}

% First, processing of the leftmost increment begins: the least significant bit is flipped, and the increment is carried over to the more significant bits.
% This corresponds to the reduction $e \oc b_1 \oc i \oc i \Reduces e \oc i \oc b_0 \oc i$.
% Next, either of the two remaining increments may be processed -- that is, either $e \oc i \oc b_0 \oc i \Reduces e \oc b_1 \oc b_0 \oc i$ or $e \oc i \oc b_0 \oc i \Reduces e \oc i \oc b_1$.


We should like to prove the correctness of this specification of increments by establishing a computational adequacy result:
%
\begin{theorem}[Adequacy of increments]\label{thm:ordered-rewriting:binary-counter:inc-adequacy}
  If $\aval{\octx}{n}$ and $\octx \oc i \Reduces\aval{}{n'}$, then $n' = n+1$.
  Moreover, if $\aval{\octx}{n}$, then $\octx \oc i \Reduces\aval{}{n+1}$.
\end{theorem}
%
By analogy with functional programming, this \lcnamecref{thm:ordered-rewriting:binary-counter:inc-adequacy} can be seen as stating evaluation and termination results for a big-step evaluation semantics of increments --
the judgment $\aval{\octx}{n}$ is acting as a kind of typing judgment, with $n$ being the \enquote{type} [abstract interpretation?] of the counter $\octx$.

In functional programming, these sorts of big-step results are proved by first providing a small-step operational semantics, then characterizing the valid intermediate states that arise with small steps, and finally establishing type preservation, progress, and termination results for the small-step semantics.
We will adopt the same strategy here.

First, we define a relation, $\ainc{}{}$, that characterizes the valid intermediate states that arise during increments.

To prove this \lcnamecref{thm:ordered-rewriting:increments}, we will first introduce an auxiliary relation, $\ainc{}{}$, that characterizes the valid states that arise during increments.
This relation is defined inductively by the following rules.
%
\begin{inferences}
  \infer{\ainc{e}{0}}{}
  \and
  \infer{\ainc{\octx \oc b_0}{2n}}{
    \ainc{\octx}{n}}
  \and
  \infer{\ainc{\octx \oc b_1}{2n+1}}{
    \ainc{\octx}{n}}
  \and
  \infer{\ainc{\octx \oc i}{n+1}}{
    \ainc{\octx}{n}}
  \\
  \infer{\ainc{e \fuse b_1}{1}}{}
  \and
  \infer{\ainc{\octx \oc (i \fuse b_0)}{2(n+1)}}{
    \ainc{\octx}{n}}
\end{inferences}
The latter two

% \begin{lemma}[Value inclusion]
%   If $\aval{\octx}{n}$, then $\ainc{\octx}{n}$.
% \end{lemma}
% %
% \begin{proof}
%   By structural induction on the derivation of $\aval{\octx}{n}$.
% \end{proof}

% \begin{theorem}[Preservation]
%   If $\ainc{\octx}{n}$ and $\octx \reduces \octx'$, then $\ainc{\octx'}{n}$.
% \end{theorem}
% %
% \begin{proof}
%   By structural induction on the derivation of $\ainc{\octx}{n}$.
%   % We will show a representative cases.
%   % \begin{itemize}
%   % % \item Consider the case in which
%   % %   \begin{equation*}
%   % %     \octx
%   % %     =
%   % %     \infer{\ainc{\octx_0 \oc i}{n_0+1}}{
%   % %       \ainc{\octx_0}{n_0}}
%   % %     =
%   % %     n
%   % %   \end{equation*}
%   % %   and $\octx = \octx_0 \oc i \reduces \octx'_0 \oc i = \octx'$ because $\octx_0 \reduces \octx'_0$, for some $\octx_0$, $\octx'_0$, and $n_0$.
%   % %   By the inductive hypothesis, $\ainc{\octx'_0}{n_0}$.
%   % %   And so, $\octx' = \ainc{\octx'_0 \oc i}{n_0+1} = n$, as required.
%   % 
%   % % \item Consider the case in which
%   % %   \begin{equation*}
%   % %     \octx
%   % %     =
%   % %     \infer{\ainc{\octx_0 \oc b_1 \oc i}{(2n_0+1)+1}}{
%   % %       \infer{\ainc{\octx_0 \oc b_1}{2n_0+1}}{
%   % %         \ainc{\octx_0}{n_0}}}
%   % %     =
%   % %     n
%   % %   \end{equation*}
%   % %   and $\octx = \octx_0 \oc b_1 \oc i \reduces \octx_0 \oc (i \fuse b_0) = \octx'$, for some $\octx_0$ and $n_0$.
%   % %   It immediately follows that $\octx' = \ainc{\octx_0 \oc (i \fuse b_0)}{2(n_0+1)} = 2n_0+2 = n$, as required.
%   % 
%   % \item Consider the case in which
%   %   \begin{equation*}
%   %     \octx
%   %     =
%   %     \infer{\ainc{\octx_0 \oc (i \fuse b_0)}{2(n_0+1)}}{
%   %       \ainc{\octx_0}{n_0}}
%   %     =
%   %     n
%   %   \end{equation*}
%   %   and $\octx = \octx_0 \oc (i \fuse b_0) \reduces \octx_0 \oc i \oc b_0 = \octx'$, for some $\octx_0$ and $n_0$.
%   %   It immediately follows that
%   %   \begin{equation*}
%   %     \octx'
%   %     =
%   %     \infer{\ainc{\octx_0 \oc i \oc b_0}{2(n_0+1)}}{
%   %       \infer{\ainc{\octx_0 \oc i}{n_0+1}}{
%   %         \ainc{\octx_0}{n_0}}}
%   %     =
%   %     n
%   %     \,,
%   %   \end{equation*}
%   %   as required.
%   % \qedhere
%   % \end{itemize}
% \end{proof}


% \begin{theorem}[Progress]
%   If $\ainc{\octx}{n}$, then either $\octx \reduces \octx'$ or $\aval{\octx}{n}$.
% \end{theorem}
% %
% \begin{proof}
%   By structural induction on the derivation of $\ainc{\octx}{n}$.
%   % \begin{itemize}
%   % \item Consider the case in which
%   %   \begin{equation*}
%   %     \octx
%   %     =
%   %     \infer{\ainc{\octx_0 \oc i}{n_0+1}}{
%   %       \ainc{\octx_0}{n_0}}
%   %     =
%   %     n
%   %   \end{equation*}
%   %   for some $\octx_0$ and $n_0$.
%   % \end{itemize}
% \end{proof}

% Because rewriting is nondeterministic, we cannot take \enquote{$\ainc{\octx}{n}$ implies $\octx \Reduces\aval{}{n}$} as a statement of termination.

% \begin{theorem}[Termination]
%   If $\ainc{\octx}{n}$, then there is no infinite rewriting of $\octx$.
% \end{theorem}
% %
% \begin{proof}
%   For valid states $\octx$, we define a measure $\card{\octx}$ that is strictly decreasing across each rewriting $\octx \reduces \octx'$ (see the adjacent \lcnamecref{fig:ordered-rewriitting:binary-counter:measure}).
%   \begin{marginfigure}
%     \begin{equation*}
%       \begin{lgathered}
%         \card{e} = 0 \\
%         \card{\octx \oc b_0} = \card{\octx} \\
%         \card{\octx \oc b_1} = \card{\octx} + 2 \\
%         \card{\octx \oc i} = \card{\octx} + 4 \\
%         \card{e \fuse b_1} = 3 \\
%         \card{\octx \oc (i \fuse b_0)} = \card{\octx} + 5
%       \end{lgathered}
%     \end{equation*}
%   \end{marginfigure}%
%   That is, if $\octx$ is a valid state and $\octx \reduces \octx'$, then $\card{\octx} > \card{\octx'}$.
%   Because the measure is nonnegative, only finitely many such rewrittings can occour. 

%   As an example case, consider the valid state $\octx \oc b_0 \oc i$ and its rewritting  $\octx \oc b_0 \oc i \reduces \octx \oc b_1$.
%   It follows from the definition that $\card{\octx \oc b_0 \oc i} = \card{\octx} + 4 > \card{\octx} + 2 = \card{\octx \oc b_1}$.
% \end{proof}


% %
% \begin{proof}[Counterexample]
%   Small-step preservation does \emph{not} hold for $\ainc{}{}$.
%   As a specific counterexample, notice that $\ainc{\octx \oc b_1 \oc i}{2n+2}$ and $\octx \oc b_1 \oc i \reduces \octx \oc (i \fuse b_0)$, but $\ainc{\octx \oc (i \fuse b_0) \not}{2n+2}$.
%   Similarly, $\ainc{e \oc i}{1}$ and $e \oc i \reduces e \fuse b_1$, but $\ainc{e \fuse b_1 \not}{1}$.
% \end{proof}

% \begin{theorem}[Big-step preservation]
%   If $\ainc{\octx}{n}$ and $\octx \Reduces \ainc{\octx'}{n'}$, then $n = n'$.
% \end{theorem}


%  Consider the case in which $\octx = \octx_0 \oc b_1 \oc i \reduces \octx_0 \oc (i \fuse b_0) \Reduces \ainc{\octx'}{n'}$ and $n = 2n_0+2$ for some $\octx_0$ and $ n_0$ such that $\ainc{\octx_0}{n_0}$.
%     By inversion, $\octx_0 \oc i \oc b_0 \Reduces \ainc{\octx'}{n'}$.

% \begin{theorem}[Big-step evaluation]
%   If $\ainc{\octx}{n}$, then $\octx \Reduces \aval{\octx'}{n}$.
% \end{theorem}
% %
% \begin{proof}
%   By nested innduction, first on the natural number $n$ and then on the context $\octx$.
%   \begin{itemize}
%   \item Consider the case in which $\octx = \octx_0 \oc b_1 \oc i$ and $n = 2n_0+2$ for some $\octx_0$ and $ n_0$ such that $\ainc{\octx_0}{n_0}$.
%     By the inductive hypothesis, $\octx_0 \Reduces \aval{\octx'_0}{n_0}$, for some $\octx'_0$.
%     Notice that $\octx'_0 \oc b_1 \oc i \Reduces \octx'_0 \oc i \oc b_0$.
%     By the inductive hypothesis again, $\octx'_0 \oc i \Reduces \aval{\octx''_0}{n_0+1}$.
%     Framing $b_0$ on to the right, $\octx \Reduces \octx'_0 \oc b_1 \oc i \Reduces \octx'_0 \oc i \oc b_0 \Reduces \aval{\octx''_0 \oc b_0}{2(n_0+1)} = n$.

%   \item Consider the case in which $\octx = \octx_0 \oc b_0$ and $n = 2n_0$ for some $\octx_0$ and $n_0$ such that $\ainc{\octx_0}{n_0}$.
%     By the inductive hypothesis, $\octx_0 \Reduces \aval{\octx'_0}{n_0}$ for some $\octx'_0$.
%     Framing $b_0$ on to the right, $\octx = \octx_0 \oc b_0 \Reduces \aval{\octx'_0 \oc b_0}{2n_0} = n$.

%   \item Consider the case in which $\octx = e \oc i$ and $n = 1$.
%     It follows that $\octx = e \oc i \Reduces \aval{e \oc b_1}{1} = n$.
%   \end{itemize}
% \end{proof}

% \begin{theorem}[Big-step determinism]
%   If $\ainc{\octx}{n}$, then $\octx \Reduces \aval{\octx'}{n}$.
% \end{theorem}


% To correct this, there are two choices.
% First, we could introduce the following rules.
% \begin{inferences}
%   \infer{\ainc{e \fuse b_1}{1}}{}
%   \and
%   \infer{\ainc{\octx \oc (i \fuse b_0)}{2n+2}}{
%     \ainc{\octx}{n}}
% \end{inferences}
% Second, we could prove a big-step preservation result:
% \begin{theorem}[Big-step preservation]
%   If $\ainc{\octx}{n}$ and $\octx \Reduces \ainc{\octx'}{n'}$, then $n = n'$.
% \end{theorem}
% %
% \begin{proof}
%   \begin{itemize}
%   \item Consider the case in which $\octx = e \oc i$ and $n = 1$ and $e \fuse b_1 \Reduces\ainc{}{n'}$.
%     By inversion, $e \fuse b_1 \reduces \ainc{e \oc b_1}{1} = n'$.
%   \item Consider the case in which $\octx = \octx_0 \oc b_0 \oc i$ and $n = 2n_0+1$ and $\octx_0 \oc b_1 \Reduces\ainc{}{n'}$ for some $\octx_0$ and $n_0$ such that $\ainc{\octx_0}{n_0}$.
%     Notice that $\ainc{\octx_0 \oc b_1}{2n_0+1}$, and so $n' = 2n_0+1 = n$, by the inductive hypothesis.
%   \item Consider the case in which $\octx = \octx_0 \oc b_1 \oc i$ and $n = 2n_0+2$ and $\octx_0 \oc (i \fuse b_0) \Reduces\ainc{}{n'}$ for some $\octx_0$ and $n_0$ such that $\ainc{\octx_0}{n_0}$.
%     By [...], $\octx_0 \oc i \oc b_0 \Reduces\ainc{}{n'}$.
%     Notice that $\ainc{\octx_0 \oc i \oc b_0}{2(n_0+1)}$, and so $n' = 2(n_0+1) = n$, by the inductive hypothesis.
%   \item Consider the case in which $\octx = \octx_0 \oc i$ and $n = n_0+1$ and $\octx_0 \reduces \octx'_0$ and $\octx'_0 \oc i \Reduces\ainc{}{n'}$ for some $\octx_0$, $\octx'_0$, and $n_0$ such that $\ainc{\octx_0}{n_0}$.
%   \end{itemize}
% \end{proof}


% \begin{theorem}[Preservation and progress]\leavevmode
%   \begin{description}[nosep]
% %  \item[Unicity] If $\ainc{\octx}{n}$ and $\ainc{\octx}{n'}$, then $n = n'$.
% %  \item[Preservation] If $\ainc{\octx}{n}$ and $\octx \Reduces \octx'$, then $\ainc{\octx'}{n}$.
%   \item[Weak preservation] If $\ainc{\octx}{n}$ and $\octx \Reduces \ainc{\octx'}{n'}$, then $n = '$.
% %  \item[Progress] If $\ainc{\octx}{n}$, then either $\octx \reduces \octx'$ or $\aval{\octx}{n}$.
%   \item[Termination] If $\ainc{\octx}{n}$, then $\octx \Reduces\aval{}{n}$.
%   \end{description}
% \end{theorem}
% %
% \begin{proof}
%   \begin{description}
%   \item[Termination]
%     Assume that $\ainc{\octx}{n}$; we must show that $\octx \Reduces\aval{}{n}$.
%     \begin{itemize}
%     \item Consider the case in which $\octx = \octx_0 \oc b_0$ and $n = 2n_0$ for some $\octx_0$ and $n_0$ such that $\ainc{\octx_0}{n_0}$.
%       By the inductive hypothesis, $\octx_0 \Reduces\aval{}{n_0}$.
%       It follows that $\octx = \octx_0 \oc b_0 \Reduces\aval{}{2n_0} = n$.

%     \item The case in which $\octx = \octx_0 \oc b_1$ and $n = 2n_0+1$ for some $\octx_0$ and $n_0$ such that $\ainc{\octx_0}{n_0}$ is analogous.

%     \item Consider the case in which $\octx = \octx_0 \oc b_0 \oc i$ and $n = 2n_0+1$ for some $\octx_0$ and $n_0$ such that $\ainc{\octx_0}{n_0}$.
%       By the inductive hypothesis, $\octx_0 \Reduces \aval{\octx'_0}{n_0}$ for some $\octx'_0$.
%       It follows that $\octx_0 \oc b_0 \oc i \Reduces \octx'_0 \oc b_0 \oc i \reduces \octx'_0 \oc b_1$, and moreover, $\aval{\octx'_0 \oc b_1}{2n_0+1}$.
%       So, indeed, $\octx \Reduces \aval{}{2n_0+1}$.

%     \item
%     \end{itemize}
%   \end{description}
% \end{proof}

\paragraph{A decrement operation}
Binary counters
% \newthought{These binary counters}
may also be equipped with a decrement operation.
Instead of examining decrements \emph{per se}, we will implement a closely related operation: the normalization of binary representations to what might be called \vocab{head-unary form}.%
\footnote{We will frequently abuse terminology, using \enquote*{head-unary normalization} and \enquote*{decrement} interchangably.}
An ordered context $\octx$ will be said to be in head-unary form if it has one of two forms: $\octx = z$; or $\octx = \octx' \oc s$, for some binary number $\octx'$.

Just as appending the atom $i$ to a counter initiates an increment, appending an uninterpreted atom $d$ will cause the counter to begin normalizing to head-unary form.
The following \lcnamecref{thm:decrement-adequacy} serves as a specification of head-unary normalization, relating a value's head-unary form to its denotation.
%
\begin{theorem}[Structural adequacy of decrements]
  If $\aval{\octx}{n}$, then:
  \begin{itemize}[nosep]
  \item $\octx \oc d \Reduces z$ if, and only if, $n=0$;
  \item $\octx \oc d \Reduces \octx' \oc s$ for some $\octx'$ such that $\aval{\octx'}{n-1}$, if $n > 0$; and
  \item $\octx \oc d \Reduces \octx' \oc s$ only if $n > 0$ and $\aval{\octx'}{n-1}$.
  \end{itemize}
\end{theorem}
%
\noindent
For example, because $e \oc b_1$ denotes $1$, a computation $e \oc b_1 \oc d \Reduces \octx' \oc s$ must exist, for some $\aval{\octx'}{0}$.

\newthought{Once again}, to implement these desiderata locally, the recursive definitions of $e$, $b_0$, and $b_1$ will be revised with an additional clause that handles decrements;
also, a recursively defined proposition $b'_0$ is introduced:
% 
% Similarly to the use of the atom $i$ to describe 
% Similarly to the way $i$ initiates increments, a decrement is triggered by appending an [uninterpreted] atom $d$ to the counter;
% $d$ is then processed from right to left by the counter's individual bits.
% To support this, the definitions of $e$, $b_0$, and $b_1$ are revised with an addition clause each:
\begin{description}[font=\color{structure}]
\item[$e \defd (\dotsb \pmir i) \with (z \pmir d)$]
  Because $e$ denotes $0$, its head-unary form is simply $z$.
  % Because $e$ denotes $0$, it may be put into head-unary form by replacing it with $z$.
\item[$b_0 \defd (\dotsb \pmir i) \with (d \fuse b'_0 \pmir d)$]
  Because $\octx \oc b_0$ denotes $2n$ if $\octx$ denotes $n$, its head-unary form can be contructed by recursively putting the more significant bits into head-unary form and appending $b'_0$ to process that result.
  % To put $\octx \oc b_0$ into head-unary form, recursively put the more significant bits into head-unary form and append $b'_0$ to process that result.
\item[$b'_0 \defd (z \limp z) \with (s \limp b_1 \fuse s)$]
  If the more significant bits have head-unary form $z$ and therefore denote $0$, then $\octx \oc b_0$ also denotes $0$ and has head-unary form $z$.
  Otherwise, if they have head-unary form $\octx' \oc s$ and therefore denote $n > 0$, then $\octx \oc b_0$ denotes $2n$ and has head-unary form $\octx' \oc b_1 \oc s$, which can be constructed by replacing $s$ with $b_1 \oc s$.
  % If the more significant bits, $\octx$, have $z$ as their head-unary form, then so does $\octx \oc b_0$; otherwise, if their head-unary form ends with $s$, then
\item[$b_1 \defd (\dotsb \pmir i) \with (b_0 \fuse s \pmir d)$]
  Because $\octx \oc b_1$ denotes $2n+1$ if $\octx$ denotes $n$, its head-unary form, $\octx \oc b_0 \oc s$, can be constructed by flipping the least significant bit to $b_0$ and appending $s$.
  % To put $\octx \oc b_1$ into head-unary form, decrement the least significant bit to $b_0$ and append $s$.
\end{description}
%
Comfortingly, $2-1 = 1$: the head-unary form of $e \oc b_1$ is $e \oc b_0 \oc b_1 \oc s$:
\begin{equation*}
  e \oc b_1 \oc b_0 \oc d \Reduces e \oc b_1 \oc d \oc b'_0 \Reduces e \oc b_0 \oc s \oc b'_0 \Reduces e \oc b_0 \oc b_1 \oc s
  \,.
\end{equation*}


\newthought{At this point}, we would like to prove the adequacy of decrements.
However, having just revised the definitions of $e$, $b_0$, and $b_1$, we must first recheck the adequacy of binary representation\parencref[see]{??}.
%
Unfortunately, the newly introduced alternative conjunctions, together with the fine-grained atomicity of ordered rewriting, cause [...].
%
\begin{falseclaim}[Adequacy of binary representations]%
  \leavevmode
  \begin{thmdescription}
  \item[Functional]
    For every binary number $\octx$, there exists a unique natural number $n$ such that $\aval{\octx}{n}$.
  \item[Surjectivity]
    For every natural number $n$, there exists a binary number $\octx$ such that $\aval{\octx}{n}$.
  \item[Values]
    If $\aval{\octx}{n}$, then $\octx \nreduces$.
  \end{thmdescription}
\end{falseclaim}
\begin{proof}[Counterexample]
  Although the $\aval{}{}$ relation remains functional and surjective, it does not satisfy [...].
  Because $\aval{e}{0}$, the counter $e$ is a value (with denotation $0$).
  However, because the atomicity of ordered rewriting is extremely fine-grained, $e$ can be rewritten:
  \begin{equation*}
    \begin{tikzcd}[
      cells={inner xsep=0.65ex,
             inner ysep=0.4ex},
      row sep=0em,
      column sep=scriptsize,
      /tikz/column 2/.append style={anchor=west}
    ]
      &[-0.2em] e \mathrlap{{} \fuse b_1 \pmir i}
      \\
      e = (e \fuse b_1 \pmir i) \with (z \pmir d)
        \urar[reduces, start anchor=east]
        \drar[reduces, start anchor=base east]
      \\
      & z \mathrlap{{} \pmir d}
    \end{tikzcd}
    \hphantom{{} \fuse b_1 \pmir i}
  \end{equation*}
  That $e$ is an active proposition violates our conception of values as inactive.
\end{proof}

\newthought{At this point}, we would like to prove the adequacy of decrements.
However, having just revised the definitions of $e$, $b_0$, and $b_1$, we must first recheck the adequacy of increments.
%
Unfortunately, the newly introduced alternative conjunctions, together with the fine-grained atomicity of ordered rewriting, cause the preservation and progress properties to fail.
%
\begin{falseclaim}[Small-step adequacy of increments]%
  \leavevmode
  \begin{thmdescription}
  \item[Value inclusion]
    If $\aval{\octx}{n}$, then $\ainc{\octx}{n}$.
  \item[Preservation]
    If $\ainc{\octx}{n}$ and $\octx \reduces \octx'$, then $\ainc{\octx'}{n}$.
  \item[Progress]
    If $\ainc{\octx}{n}$, then either%
    \begin{itemize*}[
      mode=unboxed, label=, afterlabel=,
      before=\unskip:\space,
      itemjoin=;\space, itemjoin*=; or\space%
    ]  
    \item $\octx \reduces \octx'$ for some $\octx'$
    \item $\octx \nreduces$ and $\aval{\octx}{n}$
    \end{itemize*}
  \item[Termination]
    If $\ainc{\octx}{n}$, then every rewriting sequence from $\octx$ is finite.
  \end{thmdescription}
\end{falseclaim}
\begin{proof}[Counterexample]
  As a counterexample to preservation, notice that $e \oc i$ denotes $1$ and that
\begin{equation*}
  e \oc i = (e \fuse b_1 \pmir i) \with (z \pmir d)
    \reduces (e \fuse b_1 \pmir i) \oc i
  \,,
\end{equation*}
but that $(e \fuse b_1 \pmir i) \oc i$ does not have a denotation under the $\ainc{}{}$ relation.

  Even worse, computations can now enter stuck states -- $e \oc i \reduces (z \pmir d) \oc i \nreduces$, for example.
  It's difficult to imagine assigning denotations to these stuck states, making them counterexamples to preservation.
  Even if denotations were somehow assigned to them, such states would anyway violate the desired progress theorem.
\end{proof}

In both cases, these counterexamples arise from the very fine-grained atomicity of ordered rewriting.
Now that the definitions of $e$, $b_0$, and $b_1$ include alternative conjunctions, [...].

\begin{theorem}
  \begin{thmdescription}
  \item[Evaluation]
    If $\ainc{\octx}{n}$, then $\octx \Reduces\aval{}{n}$.
    In particular, if $\aval{\octx}{n}$, then $\octx \Reduces\aval{n+1}$.
  \item[Preservation]
    If $\ainc{\octx}{n}$ and $\octx \Reduces\aval{}{n'}$, then $n' = n$.
  \end{thmdescription}
\end{theorem}
\begin{proof}
  By structural induction on the given derivation of $\ainc{\octx}{n}$.
\end{proof}

The solution is to chain several small rewriting steps together into a single, larger atomic step. 


\section{Weakly focused rewriting}

\Textcite{Andreoli:??}'s observation was that propositions can be partitioned into two classes, or \vocab{polarities}\fixnote{reference?}, according to the invertibility of their sequent calculus rules, and that [...].



The ordered propositions are polarized into two classes, the positive and negative propositions, according to the invertibility of their sequent calculus rules.
\begin{syntax*}
  Positive props. &
    \p{A} & \p{\alpha} \mid \p{A} \fuse \p{B} \mid \one \mid \dn \n{A}
  \\
  Negative props. &
    \n{A} & \n{\alpha} \mid
    % \begin{array}[t]{@{{} \mid {}}l@{}}
              \p{A} \limp \n{B} \mid \n{B} \pmir \p{A} \mid % \\
              \n{A} \with \n{B} \mid \top \mid \up \p{A}
            % \end{array}
\end{syntax*}
The positive propositions are those propositions that have invertible left rules, such as ordered conjunction;
the negative propositions are those that have invertible right rules, such as the ordered implications.

\begin{syntax*}
  Ordered contexts &
    \octx & \octx_1 \oc \octx_2 \mid \octxe \mid \p{A}
\end{syntax*}

Left rules for negative connectives may be chained together into a single \vocab{left-focusing phase}, reflected by the pattern judgment $\lfocus{\octx_L}{\n{A}}{\octx_R}{\p{C}}$.
Following \textcite{Zeilberger:??}, this judgment can be read as a function of an in-focus negative proposition, $\n{A}$, that produces the ordered contexts $\octx_L$ and $\octx_R$ and the positive consequent $\p{C}$ as outputs.

The left-focus judgment is defined inductively on the structure of the in-focus proposition by the following rules.
\begin{inferences}
  \infer[\lrule{\limp}']{\lfocus{\octx_L \oc \p{A}}{\p{A} \limp \n{B}}{\octx_R}{\p{C}}}{
    \lfocus{\octx_L}{\n{B}}{\octx_R}{\p{C}}}
  \and
  \infer[\lrule{\pmir}']{\lfocus{\octx_L}{\n{B} \pmir \p{A}}{\p{A} \oc \octx_R}{\p{C}}}{
    \lfocus{\octx_L}{\n{B}}{\octx_R}{\p{C}}}
  \\
  \infer[\lrule{\with}_1]{\lfocus{\octx_L}{\n{A} \with \n{B}}{\octx_R}{\p{C}}}{
    \lfocus{\octx_L}{\n{A}}{\octx_R}{\p{C}}}
  \and
  \infer[\lrule{\with}_2]{\lfocus{\octx_L}{\n{A} \with \n{B}}{\octx_R}{\p{C}}}{
    \lfocus{\octx_L}{\n{B}}{\octx_R}{\p{C}}}
  \and
  \text{(no $\lrule{\top}$ rule)}
  \\
  \infer[\lrule{\up}]{\lfocus{}{\up \p{A}}{}{\p{A}}}{}
\end{inferences}
These rules parallel the usual sequent calculus rules, maintaining focus on the subformulas of negative polarity.
First, the $\lrule{\up}$ rule finishes a left-focusing phase by producing the consequent $\p{A}$ from $\up \p{A}$.

Second, the $\lrule{\limp}'$ and $\lrule{\pmir}'$ rules diverge slightly from the usual left rules for left- and right-handed implication in that they have no premises decomposing [the antecedent\fixnote{wc?}] $\p{A}$.
This would mean that a weakly focused sequent calculus based on $\lrule{\limp}'$ and $\lrule{\pmir}'$ would be incomplete for provability.
It is possible to [...]\autocite{Simmons:CMU??}.
However, because our goal here is a rewriting framework and such a framework is inherently incomplete\fixnote{Is this right?}, [...].


\begin{equation*}
  \infer[\jrule{$\dn$D}]{\octx_L \oc \dn \n{A} \oc \octx_R \reduces \p{C}}{
    \lfocus{\octx_L}{\n{A}}{\octx_R}{\p{C}}}
\end{equation*}

Consider the recursively defined proposition $\alpha \defd (\beta \limp \alpha) \with (\gamma \limp \one)$.
Previously, in the unfocused rewriting framework, it took two steps to rewrite $\beta \oc \alpha$ into $\alpha$:
\begin{equation*}
  \beta \oc \alpha = \beta \oc \bigl((\beta \limp \alpha) \with (\gamma \limp \one)\bigr)
    \reduces \beta \oc (\beta \limp \alpha)
    \reduces \alpha
\end{equation*}
Now, in the polarized, weakly focused rewriting framework, the analogous recursive definition is $\n{\alpha} \defd (\p{\beta} \limp \up \dn \n{\alpha}) \with (\p{\gamma} \limp \up \one)$, and it takes only one step to rewrite $\p{\beta} \oc \dn \n{\alpha}$ into $\dn \n{\alpha}$:
\begin{equation*}
  \p{\beta} \oc \dn \n{\alpha} = \p{\beta} \oc \dn \bigl((\p{\beta} \limp \up \dn \n{\alpha}) \with (\p{\gamma} \limp \up \one)\bigr)
    \reduces \dn \n{\alpha}
\end{equation*}
because $\lfocus{\p{\beta}}{\n{\alpha}}{}{\dn \n{\alpha}}$.

Notice that, because the left-focus judgment is defined inductively, there are some recursively defined negative propositions that cannot successfully be put in focus.
For example, under the definition $\n{\alpha} \defd \p{\beta} \limp \n{\alpha}$, there are no contexts $\octx_L$ and $\octx_R$ and conseqeunt $\p{C}$ for which $\lfocus{\octx_L}{\n{\alpha}}{\octx_R}{\p{C}}$ is derivable.

In addition to the $\jrule{$\dn$D}$ rule for decomposing $\dn \n{A}$, weakly focused ordered rewriting retains the $\jrule{$\fuse$D}$ and $\jrule{$\one$D}$ rules for decomposing $\p{A} \fuse \p{B}$ and $\one$ and the compatability rules, $\jrule{$\reduces$C}_{\jrule{L}}$ and $\jrule{$\reduces$C}_{\jrule{R}}$.
Together, these five rules and the left focus\fixnote{focal?} rules comprise the weakly focused ordered rewriting framework; they are summarized in \cref{??}.
%
\begin{figure}
  \begin{syntax*}
    Positive props. &
      \p{A} & \p{\alpha} \mid \p{A} \fuse \p{B} \mid \one \mid \dn \n{A}
    \\
    Negative props. &
      \n{A} & \n{\alpha} \mid
      % \begin{array}[t]{@{{} \mid {}}l@{}}
                \p{A} \limp \n{B} \mid \n{B} \pmir \p{A} \mid % \\
                \n{A} \with \n{B} \mid \top \mid \up \p{A}
              % \end{array}
    \\
    Ordered contexts &
      \octx & \octx_1 \oc \octx_2 \mid \octxe \mid \p{A}
  \end{syntax*}
  \begin{inferences}[Rewriting: $\octx \reduces \octx'$ and $\octx \Reduces \octx'$]
    \infer[\jrule{$\dn$D}]{\octx_L \oc \dn \n{A} \oc \octx_R \reduces \p{C}}{
      \lfocus{\octx_L}{\n{A}}{\octx_R}{\p{C}}}
    \and
    \infer[\jrule{$\fuse$D}]{\p{A} \fuse \p{B} \reduces \p{A} \oc \p{B}}{}
    \and
    \infer[\jrule{$\one$D}]{\one \reduces \octxe}{}
    \\
    \text{(no $\jrule{$\plus$D}$ and $\jrule{$\zero$D}$ rules)}
    \\
    \infer[\jrule{$\reduces$C}_{\jrule{L}}]{\octx_1 \oc \octx_2 \reduces \octx'_1 \oc \octx_2}{
      \octx_1 \reduces \octx'_1}
    \and
    \infer[\jrule{$\reduces$C}_{\jrule{R}}]{\octx_1 \oc \octx_2 \reduces \octx'_1 \oc \octx_2}{
      \octx_1 \reduces \octx'_1}
  \end{inferences}
  \begin{inferences}
    \infer[\jrule{$\Reduces$R}]{\octx \Reduces \octx}{}
    \and
    \infer[\jrule{$\Reduces$T}]{\octx \Reduces \octx''}{
      \octx \reduces \octx' & \octx' \Reduces \octx''}
  \end{inferences}

  \begin{inferences}[Left focus: $\lfocus{\octx_L}{\n{A}}{\octx_R}{\p{C}}$]
    \infer[\lrule{\limp}']{\lfocus{\octx_L}{\p{A} \limp \n{B}}{\octx_R}{\p{C}}}{
      \lfocus{\octx_L \oc \p{A}}{\n{B}}{\octx_R}{\p{C}}}
    \and
    \infer[\lrule{\pmir}']{\lfocus{\octx_L}{\n{B} \pmir \p{A}}{\octx_R}{\p{C}}}{
      \lfocus{\octx_L}{\n{B}}{\p{A} \oc \octx_R}{\p{C}}}
    \\
    \infer[\lrule{\with}_1]{\lfocus{\octx_L}{\n{A} \with \n{B}}{\octx_R}{\p{C}}}{
      \lfocus{\octx_L}{\n{A}}{\octx_R}{\p{C}}}
    \and
    \infer[\lrule{\with}_2]{\lfocus{\octx_L}{\n{A} \with \n{B}}{\octx_R}{\p{C}}}{
      \lfocus{\octx_L}{\n{B}}{\octx_R}{\p{C}}}
    \and
    \text{(no $\lrule{\top}$ rule)}
    \\
    \infer[\lrule{\up}]{\lfocus{}{\up \p{A}}{}{\p{A}}}{}
  \end{inferences}
  \caption{A weakly focused ordered rewriting framework}
\end{figure}

Weakly focused ordered rewriting is sound with respect to the unfocused rewriting framework of \cref{??}.
Given a depolarization function $\erase{}$ that maps polarized propositions and contexts to their unpolarized counterparts,%
\begin{marginfigure}
  \begin{equation*}
    \begin{aligned}[t]
      \erase{\octx_1 \oc \octx_2}
        &= \erase*{\octx_1} \oc \erase*{\octx_2} \\
      \erase{\octxe} &= \octxe \\
      \erase{\p{A}} &= \erase{\p{A}}
    \end{aligned}
    \qquad
    \begin{gathered}[t]
      \erase{\dn \n{A}} = \erase{\n{A}}
      \\
      \erase{\up \p{A}} = \erase{\p{A}}
      % \\
      % \erase{\p{A} \fuse \p{B}} &= \erase{\p{A}} \fuse \erase{\p{B}}
      \\
      \begin{array}{@{}l@{}}
        \erase{\p{A} \limp \n{B}} \\
        \quad{} = \erase{\p{A}} \limp \erase{\n{B}}
      \end{array}
      \\
      \rlap{\emph{etc.}}
    \end{gathered}
  \end{equation*}
  \caption{Depolarization of propositions and contexts}
\end{marginfigure}%
%
we may state and prove the following soundness theorem for weakly focused rewriting.
%
\begin{theorem}[Soundness of weakly focused rewriting]
  If $\octx \Reduces \octx'$, then $\erase*{\octx} \Reduces \erase{\octx'}$.
\end{theorem}
\begin{proof}
  By structural induction on the given rewriting step, after generalizing the inductive hypothesis to include:
  \begin{itemize}[nosep]
  \item If $\octx \reduces \octx'$, then $\erase*{\octx} \Reduces \erase{\octx'}$.
  \item If $\lfocus{\octx_L}{\n{A}}{\octx_R}{\p{C}}$, then $\erase{\octx_L \oc \dn \n{A} \oc \octx_R} \Reduces \erase{\p{C}}$.
  %
  \qedhere
  \end{itemize}
\end{proof}
%
\noindent
A completeness theorem also holds, but we forgo its development because it is not essential to the remainder of this work.

Second, with the lone exception negative propositions are latent\autocite{??} -- 

\section{Revisiting automata}

\begin{gather*}
  \dfa{q} \defd
    \parens[size=big]{\bigwith_{a \in \ialph}(a \limp \up \dn \dfa{q}'_a)}
    \with
    (\emp \limp \up \dfa{F}(q))
\shortintertext{where}
  q \dfareduces[a] q'_a
  \text{, for all $a \in \ialph$\quad and\quad}
  \dfa{F}(q) = \begin{cases*}
                 \one & if $q \in F$ \\
                 \dn \top & if $q \notin F$
               \end{cases*}
\end{gather*}

\begin{theorem}[\ac*{DFA} adequacy up to bisimilarity]
  \dfaadequacybisimbody
\end{theorem}

Lemma\cref{??} is still needed, but now has a much different proof.
Previously, the proof of \cref{??} relied on a very specific and delicate property of \acp{DFA}, namely that each \ac{DFA} state has one and only one $a$-successor for each input symbol $a$.
Now, with weakly focused ordered rewriting, the \lcnamecref{??}'s proof is much less fragile.
With the larger granularity of individual rewriting steps that the weakly focused framework affords, a state's encoding is a latent proposition 

\section{Revisiting binary counters}

With ordered rewriting now based on a weakly focused sequent calculus, we can revisit our previous attempt to extend binary counters with support for decrements or head-unary normalization.

The propositions $e$, $b_0$, $b'_0$, and $b_1$ are recursively defined in nearly the same way as before.
With one exception discussed below, only the necessary shifts are inserted to consistently assign a negative polarity to the defined atoms $e$, $b_0$, $b'_0$, and $b_1$ and a positive polarity to the uninterpreted atoms $i$, $d$, $z$, and $s$.
\begin{equation*}
  \begin{lgathered}
    e \defd (e \fuse b_1 \pmir i) \with (z \pmir d) \\
    b_0 \defd (\up \dn b_1 \pmir i) \with (d \fuse b'_0 \pmir d) \\
    b'_0 \defd (z \limp z) \with (s \limp b_1 \fuse s) \\
    b_1 \defd (i \fuse b_0 \pmir i) \with (b_0 \fuse s \pmir d)
  \end{lgathered}
\end{equation*}

\paragraph*{Values}
Once again, we use the same $\aval{}{}$ relation to assign a unique natural number denotation to each binary representation.
\begin{inferences}
  \ooavalrules
\end{inferences}
Because the underlying ordered rewriting framework has changed, we must verify that $\aval{}{}$ is adequate -- inparticular, the [...] property that values cannot be independently rewritten.
%
\ooavaltheorem
\begin{proof}
  By induction over the structure of $\octx$.
  As an example, consider the case in which $\aval{e}{0}$.
  Indeed, $e \nreduces$ because $e = (e \fuse b_1 \pmir i) \with (z \pmir d)$ and
  \begin{equation*}
    \lfocus{\octx_L}{(e \fuse b_1 \pmir i) \with (z \pmir d)}{\octx_R}{\p{C}}
    \text{\ only if $\octx_L = \octxe$ and either $\octx_R = i$ or $\octx_R = d$.}
  \end{equation*}
  The other cases are similar.
\end{proof}

\paragraph*{Increment}
Previously, under the unfocused rewriting framework\fixnote{wc?}, rewriting $e \oc i$ into $e \fuse b_1$ took two small steps:
\begin{equation*}
  e \oc i = \bigl((e \fuse b_1 \pmir i) \with (z \pmir d)\bigr) \oc i
    \reduces (e \fuse b_1 \pmir i) \oc i
    \reduces e \fuse b_1
\end{equation*}
But now, with weakly focused rewriting, those two steps are combined into one atomic whole: $e \oc i \reduces e \fuse b_1$.

As for the unfocused rewriting implementation of binary increments, we use a $\ainc{}{}$ relation to assign a natural number denotation to each computational state.
In fact, the specific definition of the $\ainc{}{}$ remains unchanged from \cref{??}: 
\begin{inferences}
  \aincrules
\end{inferences}

The only exception to [...] is the appearance of $\up \dn b_1$ in the definition of $b_0$.
Without this double shift, $e \oc b_0 \oc i$ would be latent, unable to rewrite to a value until a second increment is appended, because the necessary $\lfocus{}{(b_1 \pmir i) \with (d \fuse b'_0 \pmir d)}{i}{b_1}$ is not derivable.
However, with the double shift, $e \oc b_0 \oc i \reduces e \oc b_1$ because $\lfocus{}{(\up \dn b_1 \pmir i) \with (d \fuse b'_0 \pmir d)}{i}{\dn b_1}$ is derivable.


With weakly focused rewriting, it is no longer possible to reach the stuck state $...$.
\smallincadequacytheorem
\begin{proof}
  As before, each part is proved separately.
  \begin{description}[
    parsep=0pt, listparindent=\parindent,
    labelsep=0.35em
  ]
  \item[Value soundness, preservation, and progress]
    can be proved by structural induction on the derivation of $\ainc{\octx}{n}$.
  \item[Termination]
    can be proved using the same explicit termination measure, $\card{\octx}$, as in \cref{??}.
  %
  \qedhere
  \end{description}
\end{proof}


\paragraph*{Decrements}

\begin{inferences}
  \infer[\jrule{$d$-D}]{\adec{\octx \oc d}{n}}{
    \ainc{\octx}{n}}
  \and
  \infer[\jrule{$b'_0$-D}]{\adec{\octx \oc b'_0}{2n}}{
    \adec{\octx}{n}}
  \and
  \infer[\jrule{$z$-D}]{\adec{z}{0}}{}
  \and
  \infer[\jrule{$s$-D}]{\adec{\octx \oc s}{n+1}}{
    \ainc{\octx}{n}}
  \\
  \infer[\jrule{$\fuse_1$-D}]{\adec{\octx \oc (d \fuse b'_0)}{2n}}{
    \ainc{\octx}{n}}
  \\
  \infer[\jrule{$\fuse_2$-D}]{\adec{\octx \oc (b_0 \fuse s)}{2n+1}}{
    \ainc{\octx}{n}}
  \and
  \infer[\jrule{$\fuse_3$-D}]{\adec{\octx \oc (b_1 \fuse s)}{2n+2}}{
    \ainc{\octx}{n}}
\end{inferences}

\begin{theorem}[Small-step adequacy of decrements]
  \leavevmode
  \begin{thmdescription}
  \item[Preservation]
    If $\adec{\octx}{n}$ and $\octx \reduces \octx'$, then $\adec{\octx'}{n}$.
  \item[Progress]
    If $\adec{\octx}{n}$, then [either]:
    \begin{itemize}[nosep]
    \item $\octx \reduces \octx'$, for some $\octx'$;
    \item $n = 0$ and $\octx = z$; or
    \item $n > 0$ and $\octx = \octx' \oc s$, for some $\octx'$ such that $\ainc{\octx'}{n-1}$.
    \end{itemize}
  \item[Termination]
    If $\adec{\octx}{n}$, then every rewriting sequence from $\octx$ is finite.
  \end{thmdescription}
\end{theorem}
\begin{proof}
  \begin{description}
  \item[Preservation and progress]
    are proved, as before, by structural induction on the given derivation of $\adec{\octx}{n}$.
  \item[Termination] is proved by exhibiting a measure, $\card[d]{}$, that is strictly decreasing across each rewriting.
    Following the example of termination for increment-only binary counters\parencref{??}, we could try to assign a constant amount of potential to each of the counter's constituents.
    Leaving these potentials as unknowns, we can generate a set of constraints from the allowed rewritings and then attempt to solve them.

    For instance, here are several rewritings and their corresponding potential constraints.
    \begin{center}
      \begin{tabular}{@{}c@{\quad}c@{}}
        \emph{Some selected rewritings} & \emph{Potential constraints}
        \\
        $\octx \oc b_1 \oc i \reduces \octx \oc (i \fuse b_0) \reduces \octx \oc i \oc b_0$
        & $b_1 + i > i + b_0 + 1$
        \\
        $\octx \oc b_0 \oc d \reduces \octx \oc (d \fuse b'_0) \reduces \octx \oc d \oc b'_0$
        & $b_0 + d > d + b'_0 + 1$
        \\
        $\octx \oc s \oc b'_0 \reduces \octx \oc (b_1 \fuse s) \reduces \octx \oc b_1 \oc s$
        & $s + b'_0 > b_1 + s + 1$
        % \\
        % $\octx \oc b_1 \oc d \reduces \octx \oc (b_0 \fuse s) \reduces \octx \oc b_0 \oc s$
        % & $b_1 + d > b_0 + s + 1$
      \end{tabular}
    \end{center}
    These constraints are satisfiable only if $b_1 > b_0 > b'_0 > b_1$, which is, of course, impossible.

    However, notice that each $b_1$ that arises from an interaction between $s$ and $b'_0$ will never participate in further rewritings because any increments remaining to the left of $b_1$ will only involve more significant bits, not this less significant $b_1$.
    A similar argument can be made for all bits that occur between the rightmost $i$ and the terminal $s$, suggesting that those bits be assigned no potential at all.

    This leads to the termination measure, $\card[d]{}$, and its auxiliary measures, $\card[i]{}$ and $\card[s]{}$, shown in the adjacent \lcnamecref{??}.
    (Note that the measure $\card[i]{}$ is not the same as the measure used for increment-only binary counters\parencref{??}.)x




    is proved by exhibiting a pair of measures, $\card{\octx}_d$ and $\card{\octx}_s$, ordered lexicographically:
    \begin{itemize}
    \item If $\adec{\octx}{n}$ and $\octx \reduces \octx'$, then either:
      \begin{itemize*}[label=, afterlabel=]
      \item $\card{\octx}_d > \card{\octx'}_d$; or
      \item $\card{\octx}_d = 0$ and $\card{\octx}_s > \card{\octx'}_s$.
      % \item $\card{\octx} > \card{\octx'}$; or
      % \item $n = 0$ and $\octx' = z$; or
      % \item $n > 0$ and $\octx' = \octx'' \oc s$, for some $\octx''$ such that $\ainc{\octx''}{n-1}$.
      % \item $\octx = \octx_0 \oc (b_1 \fuse s) \reduces \octx_0 \oc b_1 \oc s = \octx'$.
      \end{itemize*}
    \end{itemize}
    These measures are shown in the adjacent \lcnamecref{??}, %
    \begin{marginfigure}
      \begin{equation*}
        \begin{lgathered}[t]
          \card{\octx \oc s}_s = \card{\octx} \\
          \card{e} = 0 \\
          \card{\octx \oc b_0} = \card{\octx} + 4 \\
          \card{\octx \oc b_1} = \card{\octx} + 6 \\
          \card{\octx \oc i} = \card{\octx} + 8 \\
          \card{e \fuse b_1} = 7 \\
          \card{\octx \oc (i \fuse b_0)} = \card{\octx} + 13
        \end{lgathered}
        \qquad
        \begin{lgathered}[t]
          \card{\octx \oc d}_d = \card{\octx} + 1 \\
          \card{\octx \oc b'_0}_d = \card{\octx}_d + 2 \\
          \card{z}_d = 0 \\
          \card{\octx \oc s}_d = 0 \\
          \card{\octx \oc (d \fuse b'_0)}_d = \card{\octx} + 4 \\
          \card{\octx \oc (b_0 \fuse s)}_d = 1 \\
          \card{\octx \oc (b_1 \fuse s)}_d = 1
        \end{lgathered}
      \end{equation*}
    \end{marginfigure}
    rely on an auxiliary measure, $\card{\octx}$, for increment states.
    Unfortunately, it is not possible to simply reuse the measure from \cref{??}.
    In that measure, each $b_0$ bit was assigned no potential.
    With decrements, however, $b_0$ needs to carry enough potential to transfer to $b'_0$ in case a decrement instruction is encountered.

    For the rewritings $\octx \oc b_1 \oc i \reduces \octx \oc (i \fuse b_0) \reduces \octx \oc i \oc b_0$, the assigned potentials must satisfy $b_1 + i > i + b_0 + 1$


    No 

        
    % e d --> z   1 > 0
    % b0 d --> d * b0' --> d b0'  4 > 3 > 2
    % b1 d --> b0 * s --> b0 s    6 > 1 > 0
    % z b0' --> z                 2 > 0
    % s b0' --> b1* s --> b1 s    2 > 1 > 0
    % e i --> e * b1 --> e b1     8 > 7 > 6
    % b0 i --> b1                 12 > 6
    % b1 i --> i * b0 --> i b0    14 > 13 > 12

    % e + i > e + b1 + 1  
    % b0 + i > b1
    % b1 + i > i + b0 + 1
    % e + d > z
    % b0 + d > d + b0' + 1
    % b1 + d > s + 1
    % z + b0' > z
    % s + b0' > s + 1

    % b1 d --> b0 * s --> b0 s      2 + d > 1 > 0
    % b0 d --> d * b'0 --> d b'0    b0 + d > d + 2 + 1 > d + 2
    % z b'0 --> z
    % s b'0 --> b1 * s --> b1 s

    As an example case, consider the intermediate state $\octx \oc (b_1 \fuse s)$ and its rewriting $\octx \oc (b_1 \fuse s) \reduces \octx \oc b_1 \oc s$.
    It follows $\card{\octx \oc (b_1 \fuse s)}_d = 1 > 0 = \card{\octx \oc b_1 \oc s}_d$.
    Any subsequent rewritings of $\octx$ are justified by a decrease in $\card{\octx \oc b_1 \oc s}_s > \card{\octx}$.
  \end{description}
\end{proof}

\begin{corollary}[Big-step adequacy of decrements]
  If $\adec{\octx}{n}$, then:
  \begin{itemize}[nosep]
  \item $\octx \Reduces z$ if, and only if, $n = 0$;
  \item $\octx \Reduces \octx' \oc s$ for some $\octx'$ such that $\ainc{\octx'}{n-1}$, if $n > 0$; and
  \item $\octx \Reduces \octx' \oc s$ only if $n > 0$ and $\ainc{\octx'}{n-1}$.
  \end{itemize}
\end{corollary}
\begin{proof}
  From the small-step preservation result of \cref{??}, it is possible to prove, using a structural induction on the given trace, a big-step preservation result: namely, that $\adec{\octx}{n}$ and $\octx \Reduces \octx'$ only if $\adec{\octx'}{n}$.
  Each of the above claims then follows from either progress and termination\parencref{??} or big-step preservation together with inversion.
\end{proof}

%   \begin{itemize}
%   \item
%     In the left-to-right direction, preservation yields $\adec{z}{n}$; by inversion, $n$ can only be $0$.
%     The right-to-left direction follows immediately from productivity.
%   \item
%     This second statement follows immediately from productivity.
%   \item
%     Preservation yields $\adec{\octx' \oc s}{n}$; by inversion, $n$ must be strictly positive with $\ainc{\octx'}{n-1}$.
%   \end{itemize}
% \end{proof}

    \begin{equation*}
      \begin{lgathered}[t]
        \card[d]{\octx \oc d} = \card[i]{\octx} + 1 \\
        \card[d]{\octx \oc b'_0} = \card[d]{\octx} + 2 \\
        \card[d]{z} = 0 \\
        \card[d]{\octx \oc s} = \card[s]{\octx}
        \\[\jot]
        \card[d]{\octx \oc (d \fuse b'_0)} = \card[d]{\octx \oc d \oc b'_0} + 1 \\
        \card[d]{\octx \oc (b_0 \fuse s)} = \card[d]{\octx \oc b_0 \oc s} + 1 \\
        \card[d]{\octx \oc (b_1 \fuse s)} = \card[d]{\octx \oc b_1 \oc s} + 1
      \end{lgathered}
      \qquad
      \begin{lgathered}[t]
        \card[i]{e} = 0 \\
        \card[i]{\octx \oc b_0} = \card[i]{\octx} + 4 \\
        \card[i]{\octx \oc b_1} = \card[i]{\octx} + 6 \\
        \card[i]{\octx \oc i} = \card[i]{\octx} + 8
        \\[\jot]
        \card[i]{e \fuse b_1} = \card[i]{e \oc b_1} + 1 \\
        \card[i]{\octx \oc (i \fuse b_0)} = \card[i]{\octx \oc i \oc b_0} + 1
      \end{lgathered}
      \qquad
      \begin{lgathered}[t]
        \card[s]{e} = \card[i]{e} = 0 \\
        \card[s]{\octx \oc b_0} = \card[s]{\octx} \\
        \card[s]{\octx \oc b_1} = \card[s]{\octx} \\
        \card[s]{\octx \oc i} = \card[i]{\octx \oc i} = \card[i]{\octx} + 8
        \\[\jot]
        \card[s]{e \fuse b_1} = \card[s]{e \oc b_1} + 1 \\
        \card[s]{\octx \oc (i \fuse b_0)} = \card[s]{\octx \oc i \oc b_0} + 1
      \end{lgathered}
    \end{equation*}

    \begin{itemize}
    \item If $\adec{\octx}{n}$ and $\octx \reduces \octx'$, then $\card[d]{\octx} > \card[d]{\octx'}$.
    \item If $\ainc{\octx}{n}$ and $\octx \reduces \octx'$, then $\card[i]{\octx} > \card[i]{\octx'}$ and $\card[s]{\octx} > \card[s]{\octx'}$.
    \end{itemize}

% b1 d --> b0 s  |-|i + 7 > |-|s + 1
% s b0' --> b1 s  |-|s + 2 > |-|s + 1


\section{Temporary}

% \begin{theorem}[Behavioral adequacy of decrements]
%   If $\adec{\octx}{n}$, then:
%   \begin{itemize}[nosep]
%   \item $\octx \Reduces z$ if, and only if, $n = 0$;
%   \item $\octx \Reduces \octx' \oc s$ for some $\octx'$ such that $\ainc{\octx'}{n-1}$, if $n > 0$;
%   \item $\octx \Reduces \octx' \oc s$ only if $n > 0$ and $\ainc{\octx'}{n-1}$.
%   \end{itemize}
% \end{theorem}

\begin{inferences}
  \infer{\adec{\octx \oc d}{n}}{
    \ainc{\octx}{n}}
  \and
  \infer{\adec{\octx \oc b'_0}{2n}}{
    \adec{\octx}{n}}
  \and
  \infer{\adec{z}{0}}{}
  \and
  \infer{\adec{\octx \oc s}{n+1}}{
    \ainc{\octx}{n}}
  \\
  \infer{\adec{\octx \oc (d \fuse b'_0)}{2n}}{
    \adec{\octx}{n}}
  \and
  \infer{\adec{\octx \oc (b_1 \fuse s)}{2n+2}}{
    \ainc{\octx}{n}}
  \and
  \infer{\adec{\octx \oc (b_0 \fuse s)}{2n+1}}{
    \ainc{\octx}{n}}
\end{inferences}
As the first rule exhibits, a binary number and its head-unary form denote the same value.
The last three rules are included by analogy with the $e \fuse b_1$ and $i \fuse b_0$ rules of the $\ainc{}{}$ relation.

\begin{falseclaim}[Small-step adequacy of decrements]\leavevmode
  \begin{thmdescription}
  \item[Preservation]
    If $\adec{\octx}{n}$ and $\octx \reduces \octx'$, then $\adec{\octx'}{n}$.
  \item[Progress]
    If $\adec{\octx}{n}$, then either:
    \begin{itemize}[nosep]
    \item $\octx \reduces \octx'$;
    \item $n = 0$ and $\octx = z$; or
    \item $n > 0$ and $\octx = \octx' \oc s$ for some $\octx'$ such that $\ainc{\octx'}{n-1}$.
    \end{itemize}
  \item[Productivity]
    If $\adec{\octx}{n}$, then every rewriting sequence from $\octx$ has a finite prefix $\octx \Reduces \octx'$ such that either:
    \begin{itemize}[nosep]
    \item $n = 0$ and $\octx' = z$; or
    \item $n > 0$ and $\octx' = \octx'_0 \oc s$, for some $\octx'_0$ such that $\ainc{\octx'_0}{n-1}$.
    \end{itemize}
  \end{thmdescription}
\end{falseclaim}
\begin{proof}
  The fine-grained atomicity of ordered rewriting, together with the use of alternative conjunction in the recursively defined propositions $e$, $b_0$, $b'_0$, and $b_1$, causes both preservation and progress properties to fail.

  As a counterexample to preservation, $\adec{e \oc d}{0}$ and $e \oc d \reduces (z \pmir d) \oc d$, but $\adec{(z \pmir d) \oc d}{0}$ does \emph{not} hold.

  Even worse, the fine-grained atomicity of ordered rewriting means that computations can enter stuck states, which shouldn't have denotations and which would violate progress if they were somehow assigned denotations.
  For example, $\adec{e \oc d}{0}$ and $e \oc d \reduces (e \fuse b_1 \pmir i) \oc d \nreduces$.

  $\ainc{e \oc i}{0}$ and $e \oc i \reduces (z \pmir d) \oc i \nreduces$
\end{proof}


\newthought{These binary counters} may also be equipped with a decrement operation.
Although \enquote{decrement} is a convenient name for this operation, it is more accurate to implement decrements by converting the binary representation to what might be called \emph{head-unary form}: an ordered context $\octx$ is said to be in head-unary form if either: $\octx = z$; or $\octx = \octx_0 \oc s$ for some binary representation $\octx_0$.

Similar to how the atom $i$ is used to describe increments, a decrement is initiated by appending an atom $d$ to the counter; $d$ is then processed from right to left by the counter's bits.
To support this, the definitions of $e$, $b_0$, and $b_1$ are revised
\begin{equation*}
  \begin{lgathered}
    e \defd (e \fuse b_1 \pmir i) \with (\dotsb \pmir d) \\
    b_0 \defd (b_1 \pmir i) \with (\dotsb \pmir d) \\
    b_1 \defd (i \fuse b_0 \pmir i) \with (\dotsb \pmir d)
  \end{lgathered}
\end{equation*}

To initiate a decrement of a counter $\octx$, we append the uninterpreted atom $d$ to the counter, forming $\octx \oc d$.

To implement the decrement operation, we instead

Although \enquote{decrement} is a convenient name for this operation, it is perhaps more accurate to think of this operation as putting the binary representation into a head-unary form: either $z$ or $\octx' \oc s$ for some $\ainc{\octx'}{n-1}$.
\begin{itemize}
\item If $\ainc{\octx}{n}$, then:
  \begin{itemize}
  \item $n = 0$ if, and only if, $\octx \oc d \Reduces z$; and
  \item $n > 0$ implies $\octx \oc d \Reduces \octx' \oc s$ for some $\octx'$ such that $\ainc{\octx'}{n-1}$; and
  \item $\octx \oc d \Reduces \octx' \oc s$ implies $n > 0$ and $\ainc{\octx'}{n-1}$.
  \end{itemize}
\end{itemize}

\begin{equation*}
  \begin{lgathered}
    e \defd (e \fuse b_1 \pmir i) \with (z \pmir d) \\
    b_0 \defd (b_1 \pmir i) \with (d \fuse b'_0 \pmir d) \\
    b'_0 \defd (z \limp z) \with (s \limp b_1\fuse s) \\
    b_1 \defd (i \fuse b_0 \pmir i) \with (b_0 \fuse s \pmir d)
  \end{lgathered}
\end{equation*}

\begin{description}
\item[$e \defd \dotsb \with (z \pmir d)$]
  Because the counter $e$ represents $0$, its head-unary form is simply $z$.
%
\item[$b_1 \defd \dotsb \with (b_0 \fuse s \pmir d)$]
  Because the counter $\octx \oc b_1$ represents $2n+1 > 0$ when $\octx$ represents $n$, its head-unary form must then be the successor of a counter representing $2n$ -- that is, $\octx \oc b_0 \oc s$.
%
\item[$b_0 \defd \dotsb \with (d \fuse b'_0 \pmir d)$]
  The natural number that the counter $\octx \oc b_0$ represents could be either zero or positive, depending on whether $\octx$ represents zero or a positive natural number.
  Thus, to put $\octx \oc b_0$ into head-unary form, we first put $\octx$ into head-unary form and then use $b'_0$ to branch on the result.
%
\item[$b'_0 \defd (z \limp z) \with (s \limp b_1 \fuse s)$]
  If the head-unary form of $\octx$ is $z$, then $\octx \oc b_0$ also represents $0$ and has head-unary form $z$.
  Otherwise, if the head-unary form of $\octx$ is $\octx' \oc s$ for some $\ainc{\octx'}{n'}$, then $\octx \oc b_0$ represents $2n'+2$ and has head-unary form $\octx' \oc b_1 \oc s$.
\end{description}

Decrements actually do not literally decrement the counter, but instead put it into a \enquote{head unary} form in which the couter is either $z$ or $s$ with a binary counter beneath.


We will use the same strategy for proving the adequacy of decrements as we did for increments:
Characterize the valid states
\begin{inferences}
  \infer{\adec{\octx \oc d}{n}}{
    \ainc{\octx}{n}}
  \and
  \infer{\adec{\octx \oc b'_0}{2n}}{
    \adec{\octx}{n}}
  \and
  \infer{\adec{z}{0}}{}
  \and
  \infer{\adec{\octx \oc s}{n+1}}{
    \ainc{\octx}{n}}
  \\
  \infer{\ainc{e \fuse b_1 \pmir i}{0}}{}
  \and
  \infer{\ainc{z \pmir d}{0}}{}
  \\
  \infer{\ainc{\octx \oc (b_1 \pmir i)}{2n}}{
    \ainc{\octx}{n}}
  \and
  \infer{\ainc{\octx \oc (d \fuse b'_0 \pmir d)}{2n}}{
    \ainc{\octx}{n}}
  \and
  \infer{\adec{\octx \oc (d \fuse b'_0)}{2n}}{
    \ainc{\octx}{n}}
  \\
  \infer{\ainc{\octx \oc (i \fuse b_0 \pmir i)}{2n+1}}{
    \ainc{\octx}{n}}
  \and
  \infer{\ainc{\octx \oc (b_0 \fuse s \pmir d)}{2n+1}}{
    \ainc{\octx}{n}}
  \and
  \infer{\ainc{\octx \oc (i \fuse b_0)}{2n+2}}{
    \ainc{\octx}{n}}
  \and
  \infer{\adec{\octx \oc (b_0 \fuse s)}{2n+1}}{
    \ainc{\octx}{n}}
  \\
  \infer{\adec{\octx \oc (z \limp z)}{2n}}{
    \adec{\octx}{n}}
  \and
  \infer{\adec{\octx \oc (s \limp b_1 \fuse s)}{2n}}{
    \adec{\octx}{n}}
  \and
  \infer{\adec{\octx \oc (b_1 \fuse s)}{2n+2}}{
    \ainc{\octx}{n}}
\end{inferences}

Notice that $\adec{e \oc s \oc b'_0}{0}$ but $e \oc s \oc b'_0 \Reduces \adec{e \oc b_1 \oc s}{1}$.
If we revise the $s$ rule to use $n+1$, then a different problem arises: $\adec{e \oc b_1 \oc d}{0}$ but $e \oc b_1 \oc d \Reduces \adec{e \oc b_0 \oc s}{1}$.

\begin{theorem}[Adequacy]
  If $\ainc{\octx}{n}$, then:
  \begin{itemize}[nosep]
  \item $n = 0$ if and only if $\octx \oc d \Reduces z$; and
  \item $n > 0$ implies $\octx \oc d \Reduces \octx' \oc s$ and $\ainc{\octx'}{n-1}$;
  \item $\octx \oc d \Reduces \octx' \oc s$ implies $n > 0$ and $\ainc{\octx'}{n-1}$.
  \end{itemize}
\end{theorem}
%
\begin{proof}
  
\end{proof}

\begin{theorem}[Small-step adequacy]\leavevmode
  \begin{description}[nosep, font=\emph]
  \item[Preservation] If $\adec{\octx}{n}$ and $\octx \reduces \octx'$, then $\adec{\octx'}{n}$.
  \item[Progress] If $\adec{\octx}{n}$, then either:
    \begin{itemize}[nosep]
    \item $\octx \reduces \octx'$;
    \item $n = 0$ and $\octx = z$; or
    \item $n = n'+1$ and $\octx = \octx' \oc s$ for some $n'$ and $\octx'$ such that $\ainc{\octx'}{n'}$.
    \end{itemize}
  \end{description}
\end{theorem}



\section{}

\begin{corollary}[Big-step adequacy of decrements]
  If $\adec{\octx}{n}$, then:
  \begin{itemize}
  \item $\octx \Reduces \atmR{z}$ if, and only if, $n = 0$;
  \item $\octx \Reduces \octx' \oc \atmR{s}$ for some $\octx'$ such that $\ainc{\octx'}{n-1}$, if $n > 0$; and
  \item $\octx \Reduces \octx' \oc \atmR{s}$ only if $n > 0$ and $\ainc{\octx'}{n-1}$.
  \end{itemize}
\end{corollary}




\section{}

\subsection{Automata}

\begin{enumerate}
\item
  Traces do not imply DFA transitions:
  \begin{equation*}
    \begin{lgathered}
      \dfa{q}_0 \defd (a \limp \dfa{q}_0) \with (b \limp \dfa{q}_1) \with (\emp \limp \top) \\
      \dfa{q}_1 \defd (a \limp \dfa{q}_0) \with (b \limp \dfa{q}_1) \with (\emp \limp \one) \\
      \dfa{s}_1 \defd (a \limp \dfa{q}_0) \with (b \limp \dfa{s}_1) \with (\emp \limp \one)
    \end{lgathered}
  \end{equation*}
  \begin{marginfigure}
    \begin{equation*}
      \begin{tikzpicture}[baseline=(q_0.base)]
        \graph [automaton] {
          q_0
           -> [loop above, "a"]
          q_0
           -> ["b", bend left]
          q_1 [accepting]
           -> ["b", loop above]
          q_1
           -> ["a", bend left]
          q_0;
          s_1 [below=0.05 of q_1, accepting]
           -> [loop right, "b"]
          s_1
           -> ["a", bend left]
          q_0;
        };
      \end{tikzpicture}
    \end{equation*}
  \end{marginfigure}
  Notice that $b \oc \dfa{q}_0 \Reduces \dfa{q}_1 = \dfa{s}_1$ but $s_1$ is not reachable from $q_0$.
  ($\dfa{q}_1 = \dfa{s}_1$ is proved coinductively.)

\item
  NFA bisimilarity does not imply equality of encodings:
  \begin{equation*}
    \begin{lgathered}
      \nfa{q}_0 \defd (a \limp (\nfa{q}_0 \with \nfa{q}_1)) \with (\emp \limp \one) \\
      \nfa{q}_1 \defd (a \limp \nfa{q}_1) \with (\emp \limp \one)
    \end{lgathered}
  \end{equation*}
  \begin{marginfigure}
    \begin{equation*}
      \begin{tikzpicture}[baseline=(q_0.base)]
        \graph [automaton] {
          q_0 [accepting]
           -> [loop above, "a"]
          q_0
           -> ["a"]
          q_1 [accepting]
           -> ["a", loop above]
          q_1;
        };
      \end{tikzpicture}
    \end{equation*}
  \end{marginfigure}
  Notice that $q_0$ and $q_1$ are bisimilar, as witnessed by the reflexive closure of $\{(q_0,q_1)\}$.
  However, $\nfa{q}_0 \neq \nfa{q}_1$.

\item
  NFA similarity does not imply reduction.
  In the above example, NFA states $q_0$ and $q_1$ are bisimilar, andhence $q_1$ simulates $q_0$ (and vice versa).
  However, neither $\nfa{q}_0 \Reduces \nfa{q}_1$ nor $\nfa{q}_1 \Reduces \nfa{q}_0$ hold.

\item
  Even if an alternative, flatter encoding is used, NFA similarity does not imply reduction.
  Consider the following NFAs:
  \begin{align*}
   &\begin{lgathered}
      \nfa{q}_0 \defd (a \limp \nfa{q}_1) \with (\emp \limp \top) \\
      \nfa{q}_1 \defd (a \limp \nfa{q}_1) \with (a \limp \nfa{q}_2) \with (\emp \limp \one) \\
      \nfa{q}_2 \defd (a \limp \nfa{q}_2) \with (\emp \limp \one)
    \end{lgathered}
  \shortintertext{and}
   &\begin{lgathered}
      \nfa{s}_0 \defd (a \limp \nfa{s}_1) \with (\emp \limp \top) \\
      \nfa{s}_1 \defd (a \limp \nfa{s}_1) \with (\emp \limp \one)
    \end{lgathered}
  \end{align*}
  \begin{marginfigure}
    \begin{align*}
      \begin{tikzpicture}[baseline=(q_0.base)]
        \graph [automaton] {
          q_0
           -> ["a"]
          q_1 [accepting]
           -> [loop above, "a"]
          q_1
           -> ["a"]
          q_2 [accepting]
           -> ["a", loop above]
          q_2;
        };
      \end{tikzpicture}
      \\
      \begin{tikzpicture}[baseline=(s_0.base)]
        \graph [automaton] {
          s_0
           -> ["a"]
          s_1 [accepting]
           -> [loop above, "a"]
          s_1;
        };
      \end{tikzpicture}
    \end{align*}
  \end{marginfigure}
  As witnessed by the relation $\{(q_0,s_0), (q_1,s_1), (q_2,s_1)\}$, state $s_0$ simulates $q_0$.
  However, $\nfa{q}_0 \Longarrownot\Reduces \nfa{s}_0$.
  Essentially, similarity and reduction do not coincide because similarity is successor-congruent, whereas reduction is not $\limp$-congruent.

\item
  Focusing with eager inversion does not solve this problem.
  For DFAs, we would be able to prove:
  \begin{itemize}[nosep]
  \item $q$ and $s$ are bisimular if, and only if, $\dfa{q} = \dfa{s}$.
  \item $q \misa\dfareduces[a]\asim q'$ if, and only if, $a \oc \dfa{q} \reduces \dfa{q}'$.
  \end{itemize}

\item
  For NFAs, we will be able to prove:
  \begin{itemize}[nosep]
  \item $q$ and $s$ are bisimular if, and only if, $\nfa{q} \cong \nfa{s}$.
  \item $q \misa\nfareduces[a]\asim q'$ if, and only if, $a \oc \nfa{q} \cong^{-1}\reduces\cong \nfa{q}'$.
  \end{itemize}
\end{enumerate}


\subsection{Extended example: \Acp*{NFA}}

As an example of ordered rewriting, consider a specification of \acp{NFA}.
Recall from \cref{ch:automata} the \ac{NFA} (repeated in the adjacent \lcnamecref{fig:ordered-rewriting:nfa-example-ends-b})
%
\begin{marginfigure}
  \begin{equation*}
    \mathllap{\aut{A}_1 = {}}
    \begin{tikzpicture}[baseline=(q_0.base)]
      \graph [automaton] {
        q_0
         -> [loop above, "a,b"]
        q_0
         -> ["b"]
        q_1 [accepting]
         -> ["a,b"]
        q_2
         -> [loop above, "a,b"]
        q_2;
      };
    \end{tikzpicture}
  \end{equation*}
  \caption{\Iac*{NFA} that accepts, from state $q_0$, exactly those words that end with $b$. (Repeated from \cref{fig:nfa-example-ends-b}.)}\label{fig:ordered-rewriting:nfa-example-ends-b}
\end{marginfigure}%
%
that accepts exactly those words, over the alphabet $\ialph = \Set{a, b}$, that end with $b$.
We may represent that \ac{NFA} as a rewriting specification using a collection of recursive definitions, one for each of the \ac{NFA}'s states:%
\fixnote{Should I include ${} \with (\emp \limp \top)$?}
\begin{equation*}
  % \sig = \parens[size=auto]{
  \begin{lgathered}
    \nfa{q}_0 \defd (a \limp \nfa{q}_0) \with (b \limp (\nfa{q}_0 \with \nfa{q}_1)) \with (\emp \limp \top) \\
    \nfa{q}_1 \defd (a \limp \nfa{q}_2) \with (b \limp \nfa{q}_2) \with (\emp \limp \one) \\
    \nfa{q}_2 \defd (a \limp \nfa{q}_2) \with (b \limp \nfa{q}_2) \with (\emp \limp \top)
  \end{lgathered}
  % }
\end{equation*}
The \ac{NFA}'s acceptance of words is represented by the existence of traces.
For example, because the word $ab$ ends with $b$, a trace $\emp \oc b \oc a \oc \nfa{q}_0
% \Reduces \emp \oc b \oc \nfa{q}_0
% \Reduces \emp \oc \nfa{q}_1
\Reduces \octxe$ exists.
On the other hand, $\emp \oc a \oc b \oc \nfa{q}_0 \Longarrownot\Reduces \octxe$ because the word $ba$ does not end with $b$.

More generally, \iac{NFA} $\aut{A} = (Q, \mathord{\nfareduces}, F)$ over an input alphabet $\ialph$ can be represented as the ordered rewriting specification in which each state $q \in Q$ corresponds to a recursively defined proposition $\nfa{q}$:
\begin{equation*}
  \nfa{q} \defd
  \parens[size=auto]{\displaystyle
      \bigwith_{a \in \ialph}
        \parens[size=big]{a \limp \bigwith_{q'_a \in \nfapow(q,a)} \nfa{q}'_a}
    }
    \with
    \parens[size=big]{\emp \limp \nfa{F}(q)}
  \enspace\text{where\enspace
    $\nfa{F}(q) =
       \begin{cases*}
         \one & if $q \in F$ \\
         \top & if $q \notin F$\rlap{ .}
       \end{cases*}$}
\end{equation*}
After defining a representation, $\nfawds{w}$, of words $w$ (see adjacent \lcnamecref{fig:ordered-rewriting:words-represent})%
%
\begin{marginfigure}
  \begin{align*}
    \nfawds{\emp} &= \octxe \\
    \nfawds{a \wc w} &= \nfawds{w} \oc a
  \end{align*}
  \caption{Words as ordered contexts}\label{fig:ordered-rewriting:words-represent}
\end{marginfigure}%
%
, we may state and prove that ordered rewriting under these definitions is sound and complete with respect to the \ac{NFA} semantics given in \cref{ch:automata}.


\begin{theorem}
  \begin{itemize}
  \item $q \nfareduces[a] q'$ if, and only if, $a \oc \nfa{q} \Reduces \nfa{q}'$.
  \item $q \in F$ if, and only if, $\emp \oc \nfa{q} \Reduces \octxe$.
  \end{itemize}
\end{theorem}


\clearpage


\begin{falseclaim}
  Let $\aut{A} = (Q, \mathord{\nfareduces}, F)$ be \iac{NFA} over the input alphabet $\ialph$.
  Then:
  \begin{itemize}[nosep]
  \item $q \nfareduces[a]\asim s'$ if, and only if, $a \oc \nfa{q} \Reduces \nfa{s}'$.
  \item $q \asim s$ if, and only if, $\nfa{q} = \nfa{s}$.
  \end{itemize}
\end{falseclaim}
%
\begin{proof}[Counterexample]
  First, $q \asim s$ does not imply $\nfa{q} = \nfa{s}$.
  Consider the following \ac{NFA} and its corresponding definitions:
  \begin{equation*}
    \begin{tikzpicture}
      \graph [automaton] {
        q [accepting]
         -> ["a"]
        { s_1 [accepting] ->[loop right, "a" right] s_1 ,
          s_2 [accepting] ->[loop right, "a" right] s_2 };
      };
    \end{tikzpicture}
    \qquad
    \begin{lgathered}[b]
      \nfa{q} \defd (a \limp \nfa{s}_1) \with (a \limp \nfa{s}_2) \with (\emp \limp \one) \\
      \nfa{s}_1 \defd (a \limp \nfa{s}_1) \with (\emp \limp \one) \\
      \nfa{s}_2 \defd (a \limp \nfa{s}_2) \with (\emp \limp \one)
    \end{lgathered}
  \end{equation*}
  Observe that the universal binary relation on states is a bisimulation: every state has an $a$-successor and every state is an accepting state.
  Therefore, all pairs of states are bisimilar; in particular, $q \asim s_1$.
  However, $\nfa{q} \neq \nfa{s}_1$.

  Second, $a \oc \nfa{q} \Reduces \nfa{s}'$ does not imply $q \nfareduces[a]\asim s'$.
  Consider the following \ac{NFA} and its corresponding definitions:
  \begin{equation*}
    \begin{tikzpicture}
      \graph [automaton] {
        q_1 -> ["a"] q_2 [accepting]
         -> ["a"]
        { s_1 [accepting] ->[loop right, "a" right] s_1 ,
          s_2             ->[loop right, "a" right] s_2 };
      };
    \end{tikzpicture}
    \qquad
    \begin{lgathered}[b]
      \nfa{q}_1 \defd (a \limp \nfa{q}_2) \with (\emp \limp \top) \\
      \nfa{q}_2 \defd (a \limp \nfa{s}_1) \with (a \limp \nfa{s}_2) \with (\emp \limp \one) \\
      \nfa{s}_1 \defd (a \limp \nfa{s}_1) \with (\emp \limp \one) \\
      \nfa{s}_2 \defd (a \limp \nfa{s}_2) \with (\emp \limp \top)
    \end{lgathered}
  \end{equation*}
  Observe that $a \oc \nfa{q}_1 \Reduces \nfa{q}_2 \Reduces \nfa{s}_1$.
  However, $q_2 \nsim s_1$, and so $q_1 \nfareduces[a]\asim s_1$ does \emph{not} hold.
  To see why $q_2 \nsim s_1$, notice that $q_2 \nfareduces[a] s_2 \notin F$ is not matched from $s_1$, which has only $s_1 \nfareduces[a] s_1 \in F$.
\end{proof}


\begin{definition}
  A binary relation $\simu{R}$ on states is a simulation if:
  \begin{itemize}
  \item $s \simu{R}^{-1}\nfareduces[a] q'$ implies $s \nfareduces[a]\simu{R}^{-1} q'$; and
  \item $s \simu{R}^{-1} q \in F$ implies $s \in F$.
  \end{itemize}
  Similarity, $\lesssim$, is the largest simulation.
\end{definition}


\begin{lemma}
  If $\nfa{q} \secudeR \nfa{s}$, then $q \lesssim s$.
\end{lemma}
\begin{proof}
  We must check two properties:
  \begin{itemize}
  \item Suppose that $\nfa{s} \Reduces \nfa{q}$ and $q \nfareduces[a] q'_a$ for some state $q'_a$; we must show that $s \nfareduces[a] s'_a$ and $\nfa{s}'_a \secudeR \nfa{q}'_a$, for some state $s'_a$.
    According to the definition, the definiens of $\nfa{q}$ contains a clause $(a \limp \nfa{q}'_a)$.
    Because $\nfa{s} \Reduces \nfa{q}$, the definiens of $\nfa{s}$ also contains the clause $(a \limp \nfa{q}'_a)$.
    It follows that $s \nfareduces[a] q'_a$ and $\nfa{q}'_a \secudeR \nfa{q}'_a$.
  \item Suppose that $\nfa{s} \Reduces \nfa{q}$ and $q \in F$; we must show that $s \in F$.
    According to the definition, the definiens of $\nfa{q}$ contains a clause $(\emp \limp \one)$.
    Because $\nfa{s} \Reduces \nfa{q}$, the definiens of $\nfa{s}$ also contains the clause $(\emp \limp \one)$.
    It follows that $s \in F$.
  \qedhere
  \end{itemize}
\end{proof}


\begin{theorem}[Adequacy]
  Let $\aut{A} = (Q, \mathord{\nfareduces}, F)$ be \iac{NFA} over the input alphabet $\ialph$.
  If $q \nfareduces[a] q'$, then $a \oc \nfa{q} \Reduces \nfa{q}'$.
  Moreover, if $a \oc \nfa{q} \Reduces \nfa{s}'$, then $q \nfareduces[a]\gtrsim s'$.
\end{theorem}
%
\begin{proof}
  The first part follows by construction.

  To prove the second part, suppose $a \oc \nfa{q} \Reduces \nfa{s}'$.
  By the lemma, $\nfa{q} \Reduces (a \limp B) \oc \octx'_a$ and $B \oc \octx'_a \Reduces \nfa{s}'$ for some $B$ and $\octx'_a$.
  By inversion, $\octx'_a = \octxe$ and $B = \nfa{q}'_a$ for some state $q'_a$ such that $q \nfareduces[a] q'_a$.
  Therefore, $\nfa{q}'_a \Reduces \nfa{s}'$.
  By the lemma, $s' \lesssim q'_a$ and so $q \nfareduces[a]\gtrsim s'$.
\end{proof}


\begin{theorem}[Adequacy]
  Let $\aut{A} = (Q, \mathord{\nfareduces}, F)$ be \iac{NFA} over the input alphabet $\ialph$.
  Then:
  \begin{enumerate}
  \item If $q \nfareduces[a]\asim s'$, then $a \oc \nfa{q} \Reduces \nfa{s}'$.
  \item If $q \asim s$, then $\nfa{q} = \nfa{s}$.
  \item If $\nfa{q} = \nfa{s}$, then $q \asim s$.
  \item If $a \oc \nfa{q} \Reduces \nfa{s}'$, then $q \nfareduces[a]\asim s'$.
  \end{enumerate}
\end{theorem}
%
\begin{proof}
  \begin{enumerate}
  \item Suppose that $q \nfareduces[a] q' \asim s'$; we must show that $a \oc \nfa{q} \Reduces \nfa{s}'$.
    By construction, $a \oc \nfa{q} \Reduces \nfa{q}'$.
    It follows from part [...] that $\nfa{q}' = \nfa{s}'$, and so $a \oc \nfa{q} \Reduces \nfa{s}'$.

  \item Suppose $q \asim s$; we must show that $\nfa{q} = \nfa{s}$.
    \begin{itemize}
    \item Choose an arbitrary symbol $a \in \ialph$.
      If $q \nfareduces[a] q'_a$, then there exists \iac{NFA} state $s'_a$ such that $s \nfareduces[a] s'_a \misa q'_a$, and, by the coinductive hypothesis, $\nfa{q}'_a = \nfa{s}'_a$.
      Conversely, if $s \nfareduces[a] s'_a$, then there exists \iac{NFA} state $q'_a$ such that $q \nfareduces[a] q'_a$ and $\nfa{q}'_a = \nfa{s}'_a$.
    \item Also, $q$ is an accepting state if and only if $s$ is an accepting state.
    \end{itemize}
    Therefore, the definientia of $\nfa{q}$ and $\nfa{s}$ are equal, and, by the equirecursive interpretation of definitions, so are the definienda $\nfa{q}$ and $\nfa{s}$.

  \item Suppose that $\nfa{s} = \nfa{q}$ and $q \nfareduces[a] q'$; we must show that $s \nfareduces[a] s'$ and $\nfa{s}' = \nfa{q}'$, for some \ac{NFA} state $s'$.
    By its definition, the definiens of $\nfa{q}$ therefore contains the clause $(a \limp \nfa{q}')$.
    Because $\nfa{s} = \nfa{q}$, the definiens of $\nfa{s}$ must also contain a clause $(a \limp \nfa{s}')$ for some state $s'$ such that $s \nfareduces[a] s'$ and $\nfa{s}' = \nfa{q}'$.

    Symmetrically, if $\nfa{q} = \nfa{s}$ and $s \nfareduces[a] s'$, then $q \nfareduces[a] q'$ and $\nfa{q}' = \nfa{s}'$, for some state $q'$.

    
  \item Suppose $a \oc \nfa{q} \Reduces \nfa{q}'$.
    By the lemma, $a \oc \nfa{q} \Reduces (a \limp B) \oc \octx'_a$ and $B \oc \octx'_a \Reduces \nfa{q}'$ for some $B$ and $\octx'_a$.
    By inversion, $B = \nfa{q}'_a$ and $\octx'_a = \octxe$.
    Therefore, $\nfa{q}'_a \Reduces \nfa{q}'$.
    How to show that $q'_a \asim q'$?
  \end{enumerate}
\end{proof}
%
\begin{proof}
  By coinduction on $q \asim s$.
  \begin{itemize}
  \item Suppose $\nfa{s} = \nfa{q}$ and $q \nfareduces[a] q'$; we must show that $s \nfareduces[a] s'$ and $\nfa{s}' = \nfa{q}'$ for some \ac{NFA} state $s'$.
    It follows from the coinductive hypothesis that $a \oc \nfa{s} = a \oc \nfa{q} \Reduces \nfa{q}'$.
  \end{itemize}
\end{proof}
%
\begin{proof}
  In the left-to-right directions, by unrolling the definition of $\nfa{q}$ (and a structural induction on the word $w$).

  In the right-to-left directions, by structural induction on the given trace, using the following lemma:
  \begin{quotation}
    \normalsize
    If $a \oc \octx \Reduces \octx''$ and there is no $\octx''_0$ for which $\octx'' = a \oc \octx''_0$, then\\ $\octx \Reduces (a \limp B) \oc \octx'$ for some $B$ and $\octx'$ such that $B \oc \octx' \Reduces \octx''$.
  \end{quotation}

  Assume that $a \oc \nfa{q} \Reduces \nfa{q}'$.
  Using the above lemma, $\nfa{q} \Reduces (a \limp B) \oc \octx'$ for some $B$ and $\octx'$ such that $B \oc \octx' \Reduces \nfa{q}'$.
  By inversion on the trace from $\nfa{q}$, it must be that $B = \bigwith_{q'_a \in \nfapow(q,a)} \nfa{q}'_a$ and $\octx' = \octxe$.
  Further inversion on the trace from $B \oc \octx'$ establishes that $q' \in \nfapow(q,a)$ and hence $q \nfareduces[\smash{a}] q'$.
\end{proof}






\begin{equation*}
  \nfa{q} \defd
    \parens[size=auto]{\displaystyle
      \bigwith_{a \in \ialph}
        \parens[size=big]{a \limp \nfa{q}'_a \fuse \nfa{v}_a}
    }
    \with
    \parens[size=big]{\emp \limp \nfa{\sftterm}(q)}
    \enspace\text{where\enspace
      $q'_a = \sftnext(q,a)$ and
      $v_a = \sftout(q,a)$ and $v = \sftterm(q)$}
\end{equation*}


\subsection{Extended example: Binary representation of natural numbers}

As a second example, consider a rewriting specification of the binary representation of natural numbers with increment and decrement operations.

% \NewDocumentCommand \aval { m m } { #1 \approx_{\text{\normalfont\scshape v}} #2 }
% \NewDocumentCommand \ainc { m m } { #1 \approx_{\text{\normalfont\scshape i}} #2 }
% \NewDocumentCommand \adec { m m } { #1 \approx_{\text{\normalfont\scshape d}} #2 }

\NewDocumentCommand \cinc { m } { \mathbb{I}(#1) }
\NewDocumentCommand \cnat { m } { \cinc{#1} }
\NewDocumentCommand \cdec { m } { \mathbb{D}(#1) }

For this specification, a natural number is represented in binary by
% A binary representation of a natural number is
an ordered context consisting of a big-endian sequence of atoms $b_0$ and $b_1$, prefixed by the atom $e$; leading $b_0$s are permitted.
For example, both $\octx = e \oc b_1$ and $\octx = e \oc b_0 \oc b_1$ are valid binary representations of the natural number $1$.

More generally, let $\cval{}$ be the partial function from ordered contexts to natural numbers defined as follows; we say that the ordered context $\octx$ \emph{represents} natural number $n$ if $\cval{\octx} = n$.
\begin{equation*}
  \begin{lgathered}
    \cval{e} = 0 \\
    \cval{\octx \oc b_0} = 2\cval{\octx} \\
    \cval{\octx \oc b_1} = 2\cval{\octx} + 1
  \end{lgathered}
\end{equation*}
The partial function \(\cval{}\) defines an adequate representation because, up to leading $b_0$s, the natural numbers and valid binary representations (\ie, the domain of definition of $\cval{}$) are in bijective correspondence.
%
\begin{theorem}[Representational adequacy]
  For all natural numbers \(n \in \mathbb{N}\), there exists a context \(\octx\) such that \(\cval{\octx} = n\).
  Moreover, if \(\cval{\octx_1} = n\) and \(\cval{\octx_2} = n\), then \(\octx_1\) and \(\octx_2\) are identical up to leading \(b_0\)s.
\end{theorem}
\begin{proof}
  The first part follows by induction on the natural number \(n\); the second part follows by induction on the structure of the contexts \(\octx_1\) and \(\octx_2\).
\end{proof}

Next, we may describe an increment operation on these binary representations as an ordered rewriting specification; because of these increments, [...].
To indicate that an increment should be performed, a new, uninterpreted atom $i$ is introduced.
The previously uninterpreted atoms $e$, $b_0$, and $b_1$ are now given mutually recursive definitions that describe their interactions with $i$.
\begin{description}
\item[$e \defd e \fuse b_1 \pmir i$]
  To increment the counter $e$, introduce $b_1$ as a new most significant bit, resulting in the counter $e \oc b_1$.
  That is, $e \oc i \Reduces e \oc b_1$.
  Having started at value $0$ (\ie, $\cval{e} = 0$), an increment results in value $1$ (\ie, $\cval{e \oc b_1} = 1$).
\item[$b_0 \defd b_1 \pmir i$]
  To increment a counter that ends with least significant bit $b_0$, simply flip that bit to $b_1$.
  That is, $\octx \oc b_0 \oc i \Reduces \octx \oc b_1$.
  Having started at value $2n$ (\ie, $\cval{\octx \oc b_0} = 2\cval{\octx}$), an increment results in value $2n+1$ (\ie, $\cval{\octx \oc b_1} = 2\cval{\octx}+1$).
\item[$b_1 \defd i \fuse b_0 \pmir i$]
  To increment a counter that ends with least significant bit $b_1$, flip that bit to $b_0$ and propagate the increment on to the more significant bits as a carry.
  That is, $\octx \oc b_1 \oc i \Reduces \octx \oc i \oc b_0$.
  Having started at value $2n+1$ (\ie, $\cval{\octx \oc b_1} = 2\cval{\octx}+1$), an increment results in value $2n+2 = 2(n+1)$ (\ie, $\cval{\octx \oc i \oc b_0} = 2\cval{\octx}+1$).
\end{description}

As an example, consider incrementing $e \oc b_1$ twice, as captured by the state $e \oc b_1 \oc i \oc i$.
First, processing of the leftmost increment begins: the least significant bit is flipped, and the increment is carried over to the more significant bits.
This corresponds to the reduction $e \oc b_1 \oc i \oc i \Reduces e \oc i \oc b_0 \oc i$.
Next, either of the two remaining increments may be processed -- that is, either $e \oc i \oc b_0 \oc i \Reduces e \oc b_1 \oc b_0 \oc i$ or $e \oc i \oc b_0 \oc i \Reduces e \oc i \oc b_1$.

\begin{tikzcd}[]
  && e \oc b_1 \oc b_0 \oc i \drar[Reduces] &
  \\
  e \oc b_1 \oc i \oc i \rar[Reduces]
   & e \oc i \oc b_0 \oc i \urar[Reduces] \drar[Reduces] \arrow[Reduces, gray, dashed]{rr}
   && e \oc b_1 \oc b_1
  \\
   && e \oc i \oc b_1 \urar[Reduces] &
\end{tikzcd}

\begin{equation*}
  \begin{aligned}
  \MoveEqLeft[.5]
  e \oc b_1 \oc i \oc i \\
   &\Reduces e \oc i \oc b_0 \oc i \\
   &\Reduces e \oc b_1 \oc b_0 \oc i \\
   &\Reduces e \oc b_1 \oc b_1
\end{aligned}
\begin{aligned}
  \MoveEqLeft[.5]
  e \oc b_1 \oc i \oc i \\
   &\reduces e \oc (i \fuse b_0 \pmir i) \oc i \oc i
    \reduces e \oc (i \fuse b_0) \oc i
    \reduces e \oc i \oc b_0 \oc i \\
   &\reduces (e \fuse b_1 \pmir i) \oc i \oc b_0 \oc i
    \reduces (e \fuse b_1) \oc b_0 \oc i
    \reduces e \oc b_1 \oc b_0 \oc i \\
   &\reduces e \oc b_1 \oc (b_1 \pmir i) \oc i
    \reduces e \oc b_1 \oc b_1
\end{aligned}
\end{equation*}

% \begin{equation*}
%   \begin{lgathered}
%     e \defd e \fuse b_1 \pmir i \\
%     b_0 \defd b_1 \pmir i \\
%     b_1 \defd i \fuse b_0 \pmir i
%   \end{lgathered}
% \end{equation*}

% First representation, then computation.

% \begin{equation*}
%   \begin{lgathered}
%     e \defd (e \fuse b_1 \pmir i) \with (z \pmir d) \\
%     b_0 \defd (b_1 \pmir i) \with (d \fuse b'_0 \pmir d) \\
%     b_1 \defd (i \fuse b_0 \pmir i) \with (b_0 \fuse s \pmir d) \\
%     b'_0 \defd (z \limp z) \with (s \limp b_1 \fuse s)
%   \end{lgathered}
% \end{equation*}



% \begin{theorem}
%   % If $\cval{\octx} = n$, then $\octx \oc i \Reduces \octx'$ for some $\octx'$ such that $\cval{\octx'} = n+1$.
%   If $\cval{\octx} = n$ and $\octx \oc i \Reduces \octx'$, then $\octx' \Reduces \octx''$ for some $\octx''$ such that $\cval{\octx''} = n+1$.
% \end{theorem}

% \begin{equation*}
%   \begin{lgathered}
%     \cnat{e} = 0 \\
%     \cnat{\octx \oc b_0} = 2\cnat{\octx} \\
%     \cnat{\octx \oc b_1} = 2\cnat{\octx} + 1 \\
%     \cnat{\octx \oc i} = \cnat{\octx} + 1
%   \end{lgathered}
% \end{equation*}

% \begin{theorem}[Preservation]
%   If $\cnat{\octx} = n$ and $\octx \Reduces \octx'$, then $\cnat{\octx'} = n$.
% \end{theorem}

% \begin{theorem}[Progress]
%   If $\cnat{\octx} = n$, then either: $\octx \reduces \octx'$ for some $\octx'$; or $\cval{\octx} = n$.
% \end{theorem}

\clearpage

\begin{inferences}
  \infer{\aval{e}{0}}{}
  \and
  \infer{\aval{\octx \oc b_0}{2n}}{
    \aval{\octx}{n}}
  \and
  \infer{\aval{\octx \oc b_1}{2n+1}}{
    \aval{\octx}{n}}
\end{inferences}

\begin{theorem}[Adequacy]
  If \(\aval{\octx}{n}\) and \(\octx \oc i \Reduces \octx'\), then \(\octx' \Reduces \aval{}{n+1}\).
\end{theorem}
\begin{proof}
  \begin{itemize}
  \item Suppose that \(e \oc i \Reduces \octx'\); we must show that \(\octx' \Reduces \aval{}{1}\).
  \item Suppose that \(\octx \oc b_0 \oc i \Reduces \octx'\) and \(\aval{\octx}{n}\); we must show that \(\octx' \Reduces \aval{}{2n}\).
  \end{itemize}
\end{proof}


\begin{inferences}
  \infer{\ainc{\octx}{n}}{
    \aval{\octx}{n}}
  \and
  \infer{\ainc{\octx \oc i}{n+1}}{
    \ainc{\octx}{n}}
  \and
  \infer{\ainc{\octx \oc b_0}{2n}}{
    \ainc{\octx}{n}}
  \and
  \infer{\ainc{\octx \oc b_1}{2n+1}}{
    \ainc{\octx}{n}}
  \\
  \infer{\ainc{\octx_L \oc A \oc \octx_R}{n}}{
    \ainc{\octx_L \oc \alpha \oc \octx_R}{n} & (\alpha \defd A) \in \sig}
\end{inferences}

\begin{theorem}[Preservation]
  If \(\ainc{\octx}{n}\) and \(\octx \reduces \octx'\), then \(\octx' \Reduces \ainc{}{n}\).
\end{theorem}
%
\begin{proof}
  \begin{itemize}
  \item Suppose that \(\ainc{\octx_0}{n}\) and \(\octx = \octx_0 \oc i \reduces \octx'\); we must show that \(\octx' \Reduces \ainc{}{n+1}\).
    \begin{itemize}
    \item Consider the case in which \(\octx_0 \reduces \octx'_0\) and \(\octx' = \octx'_0 \oc i\).
      By the inductive hypothesis, \(\octx'_0 \Reduces \ainc{}{n}\).
      From the increment rule, it follows that \(\octx' = \octx'_0 \oc i \Reduces \ainc{}{n+1}\).
    \item Consider the case in which \(\octx_0 = \octx_L \oc (A_0 \pmir i)\) and \(\ainc{\octx_L \oc \alpha}{n}\) and \(\octx' = \octx_L \oc A_0\) such that \((\alpha \defd A_0 \pmir i) \in \sig\).
      There are three subcases:
      \begin{itemize}
      \item Consider the subcase in which \(\alpha = b_0\) and \(n = 2n_0\) and \(\ainc{\octx_L}{n_0}\).
        By inversion on the signature, \(A_0 = b_1\).
        It follows that \(\octx' = \ainc{\octx_L \oc b_1}{2n_0+1} = n+1\).
      \item Consider the subcase in which \(\alpha = b_1\) and \(n = 2n_0+1\) and \(\ainc{\octx_L}{n_0}\).
        By inversion on the signature, \(A_0 = i \fuse b_0\).
        It follows that \(\octx' = \octx_L \oc (i \fuse b_0) \reduces \ainc{\octx_L \oc i \oc b_0}{2(n_0+1)} = n+1\).
      \item Consider the subcase in which \(\alpha = e\) and \(n = 0\) and \(\octx_L = \octxe\).
        By inversion on the signature, \(A_0 = e \fuse b_1\).
        It follows that \(\octx' = e \fuse b_1 \reduces \ainc{e \oc b_1}{1} = n+1\).
      \end{itemize}
    \end{itemize}
  \end{itemize}
\end{proof}

\begin{theorem}[Progress]
  If \(\ainc{\octx}{n}\), then either: \(\octx \reduces \octx'\) for some \(\octx'\); or \(\aval{\octx}{n}\).
\end{theorem}



% \begin{equation*}
%   \begin{lgathered}
%     \cdec{\octx \oc b'_0} = 2\cdec{\octx} \\
%     \cdec{\octx \oc d} = \cnat{\octx} \\
%     \cdec{\octx \oc s} = \cnat{\octx} + 1 \\
%     \cdec{z} = 0
%   \end{lgathered}
% \end{equation*}

% \begin{theorem}
%   If \(\cinc{\octx} = n\) and \(\octx \oc d \Reduces \octx'\), then: \(\octx' \Reduces z\) if \(n = 0\); and \(\octx' \Reduces \octx'' \oc s\) for some \(\octx''\) such that \(\cinc{\octx''} = n-1\), if \(n > 0\).
% \end{theorem}

% \(\cdec{\octx'} = n\) if, and only if, \(\octx \oc d \Reduces \octx'\) for some \(\octx\) such that \(\cinc{\octx} = n\).


\begin{inferences}
  \infer{\adec{z}{0}}{}
  \and
  \infer{\adec{\octx \oc s}{n+1}}{
    \ainc{\octx}{n}}
  \and
  \infer{\adec{\octx \oc d}{n}}{
    \ainc{\octx}{n}}
  \and
  \infer{\adec{\octx \oc b'_0}{2n}}{
    \adec{\octx}{n}}
  \\
  \infer{\adec{\octx_L \oc A \oc \octx_R}{n}}{
    \adec{\octx_L \oc \alpha \oc \octx_R}{n} & (\alpha \defd A) \in \sig}
\end{inferences}

\(\adec{\octx'}{n}\) if, and only if, \(\octx \oc d \Reduces \octx'\) for some \(\octx\) such that \(\ainc{\octx}{n}\).

% \begin{theorem}
%   If \(\cinc{\octx} = n\) and \(\octx \oc d \Reduces \octx'\), then:
%   \begin{itemize}[nosep]
%   \item \(n = 2n_0\) and \(\octx' = \octx'_0 \oc b'_0 \reduces \octx''\)
%     and \(\cinc{\octx_0} = n_0\) and \(\octx_0 \oc d \Reduces \octx'_0\);
%   \item \(\cinc{\octx'_0} = n\) and \(\octx' = \octx'_0 \oc d \reduces \octx''\);
%   \item \(n = 0\) and \(\octx' = z\); or
%   \item \(n > 0\) and \(\octx' = \octx'' \oc s\) for some \(\octx''\) such that \(\cinc{\octx''} = n-1\).
%   \end{itemize}
% \end{theorem}
% %
% \begin{proof}
%   \begin{itemize}
%   \item Suppose \(\octx = e\) and \(n = 0\) and \(e \oc d \Reduces \octx'\).
%     \begin{itemize}
%     \item If the reduction is trivial, then choose \(\octx'_0 = e\) and \(\octx'' = (z \pmir d) \oc d\).
%     \end{itemize}
%   \end{itemize}
% \end{proof}


\begin{theorem}[Preservation]
  If $\adec{\octx}{n}$ and $\octx \Reduces \octx'$, then $\adec{\octx'}{n}$.
\end{theorem}

% \begin{theorem}[Preservation]
%   If $\cdec{\octx} = n$ and $\octx \Reduces \octx'$, then $\cdec{\octx'} = n$.
% \end{theorem}

% \begin{theorem}[Progress]
%   If $\cdec{\octx} = n$, then either:
%   \begin{itemize}[nosep]
%   \item $\octx \reduces \octx'$ for some $\octx'$;
%   \item $\octx = \octx' \oc s$ and $n = n' + 1$ and $\cnat{\octx'} = n'$ for some $\octx'$ and $n'$; or
%   \item $\octx = z$ and $n = 0$.
%   \end{itemize}
% \end{theorem}

\begin{theorem}[Progress]
  If $\adec{\octx}{n}$, then either:
  \begin{itemize}[nosep]
  \item $\octx \reduces \octx'$ for some $\octx'$;
  \item $\octx = \octx' \oc s$ and $n = n' + 1$ and $\ainc{\octx'}{n'}$ for some $\octx'$ and $n'$; or
  \item $\octx = z$ and $n = 0$.
  \end{itemize}
\end{theorem}


% \section{Propositional ordered rewriting}

% In this \lcnamecref{sec:ordered-rewriting:general}, we develop a rewriting interpretation of the ordered sequent calculus from the previous \lcnamecref{ch:ordered-logic}.
% This development closely follows \citeauthor{Cervesato+Scedrov:IC09}'s work on intuitionistic linear logic as a multiset rewriting framework.\autocite{Cervesato+Scedrov:IC09}

% Just as their linear logical rewriting framework is more expressive than multiset rewriting, ordered rewriting framework presented in this chapter can be seen as an extension of traditional notions of string rewriting.


% \begin{equation*}
%   \infer*{\oseq{\octx |- A}}{
%     \oseq{\octx' |- A'}}
% \end{equation*}


% Many of the ordered sequent calculus's left rules consist of a single major premise with the same consequent as in the rule's conclusion [sequent], as well as a minor premise in the case of the $\lrule{\limp}$ and $\lrule{\pmir}$ rules.
% \begin{inferences}
%   \infer[\lrule{\fuse}]{\oseq{\octx'_L \oc (A \fuse B) \oc \octx'_R |- C}}{
%     \oseq{\octx'_L \oc A \oc B \oc \octx'_R |- C}}
%   \and
%   \infer[\lrule{\with}_1]{\oseq{\octx'_L \oc (A \with B) \oc \octx'_R |- C}}{
%     \oseq{\octx'_L \oc A \oc \octx'_R |- C}}
% \end{inferences}
% Both rules, at their core, decompose resources -- the resource $A \fuse B$ into the separate resources $A \oc B$; and the resource $A \with B$ into the resource $A$.
% The resource decomposition is somewhat obscured 
% Notice that much of these two rules is devoted to shared scaffolding/boilerplate -- the framing contexts $\octx'_L$ and $\octx'_R$, and goal consequent $C$ that remain unchanged from conclusion to premise.

% Because so many rules share this scaffolding, it might be worthwhile to restructure the ordered sequent calculus to expose this shared scaffolding.
% \begin{equation*}
%   \infer{\oseq{\octx |- C}}{
%     \octx \reduces \octx' & \oseq{\octx' |- C}}
% \end{equation*}
% For instance, if $\octx_L \oc (A \fuse B) \oc \octx_R \reduces \octx_L \oc A \oc B \oc \octx_R$ holds, then the usual $\lrule{\fuse}$ rule is a derivable instance of this generalized left rule.


% \begin{theorem}
%   $\oseq{\octx |- A}$ in ... if and only if $\oseq{\octx |- A}$ in ...
% \end{theorem}
% \begin{proof}
%   The two directions are proved separately, each by induction on the structure of the given derivation.
%   \begin{gather*}
%     \infer[\lrule{\with}_1]{\oseq{\octx'_L \oc (A \with B) \oc \octx'_R |- C}}{
%       \oseq{\octx'_L \oc A \oc \octx'_R |- C}}
%     \\\rightsquigarrow\\
%     \infer[]{\oseq{\octx'_L \oc (A \with B) \oc \octx'_R |- C}}{
%       \infer[]{\octx'_L \oc (A \with B) \oc \octx'_R \reduces \octx'_L \oc A \oc \octx'_R}{
%         \infer[]{(A \with B) \oc \octx'_R \reduces A \oc \octx'_R}{
%         \infer[\lrule{\with}'_1]{A \with B \reduces A}{}}} &
%       \oseq{\octx'_L \oc A \oc \octx'_R |- C}}
%   \end{gather*}

%   \begin{equation*}
%     \begin{lgathered}
%       \bigfuse (\octx_1 \oc \octx_2) = (\bigfuse \octx_1) \fuse (\bigfuse \octx_2) \\
%       \bigfuse (\octxe) = \one \\
%       \bigfuse A = A
%     \end{lgathered}
%   \end{equation*}

%   \begin{lemma}
%     If\/ $\octx \reduces \octx'$, then $\oseq{\octx |- \bigfuse \octx'}$.
%     $\oseq{\octx' |- \bigfuse \octx'}$ for all $\octx'$.
%   \end{lemma}
% \end{proof}

% \begin{theorem}
%   If $\oseq{\octx |- A}$ and $\octx'_L \oc A \oc \octx'_R \reduces \octx'$, then $\oseq{\octx'_L \oc \octx \oc \octx'_R |- \bigfuse \octx'}$.
% \end{theorem}

% \begin{syntax*}
%   Propositions &
%     A & p \mid A \limp B \mid B \pmir A
%           \mid A \fuse B \mid \one
%           \mid A \with B \mid \top
%   \\
%   Ordered contexts & 
%     \octx & \octxe \mid \octx_1 \oc \octx_2 \mid A
% \end{syntax*}

% \begin{itemize}
% \item Lambek calculus and rewriting; compare to multiset rewriting; compare to string rewriting
% \item Explain why $\plus$ and $\zero$ (and $\bot$) are undesirable here.
% \item Connections to left rules
% \end{itemize}

% The rewriting relation is the smallest compatible relation that satisfies:
% \begin{inferences}
%   \infer{A \oc (A \limp B) \reduces B}{}
%   \and
%   \infer{(B \pmir A) \oc A \reduces B}{}
%   \\
%   \infer{A \with B \reduces A}{}
%   \and
%   \infer{A \with B \reduces B}{}
%   \and
%   \text{(no rule for $\top$)}
%   \\
%   \infer{A \fuse B \reduces A \oc B}{}
%   \and
%   \infer{\one \reduces \octxe}{}
% \end{inferences}
% We will also refer to this relation as \vocab{reduction}%
% \footnote{Input transitions are postponed to \cref{ch:ordered-bisimilarity}.}%
% .

% $\Reduces$ is the reflexive-transitive closure of $\reduces$


% \begin{equation*}
%   \infer[\lrule{\with}_1]{\oseq{\octx'_L \oc (A \with B) \oc \octx'_R |- \gamma}}{
%     \oseq{\octx'_L \oc A \oc \octx'_R |- \gamma}}
%   \leftrightsquigarrow
%   \infer{\octx'_L \oc (A \with B) \oc \octx'_R \reduces \octx'_L \oc A \oc \octx'_R}{
%     \infer{A \with B \reduces A}{}}
% \end{equation*}

% \begin{equation*}
%   \infer[\lrule{\limp}]{\oseq{\octx'_L \oc \octx \oc (A \limp B) \oc \octx'_R |- \gamma}}{
%     \oseq{\octx |- A} & \oseq{\octx'_L \oc B \oc \octx'_R |- \gamma}}
%   \rightsquigarrow
%   \infer[\lrule{\limp}']{\oseq{\octx'_L \oc A \oc (A \limp B) \oc \octx'_R |- \gamma}}{
%     \oseq{\octx'_L \oc B \oc \octx'_R |- \gamma}}
%   \leftrightsquigarrow
%   \infer{\octx'_L \oc A \oc (A \limp B) \oc \octx'_R \reduces \octx'_L \oc B \oc \octx'_R}{
%     \infer{A \oc (A \limp B) \reduces B}{}}
% \end{equation*}


% \subsection{Definitions}

% \begin{itemize}
% \item not very interesting without recursion
% \end{itemize}

\subsection{Examples}

% \paragraph*{Automata and transducers}

% \begin{equation*}
%   \begin{lgathered}[t]
%     q_0 \defd (a \limp q_0) \with (b \limp q_0 \with q_1) \\
%     q_1 \defd (a \limp q_2) \with (b \limp q_2) \with (\emp \limp \one) \\
%     q_2 \defd (a \limp q_2) \with (b \limp q_2)
%   \end{lgathered}
%   \qquad
%   \begin{lgathered}[t]
%     s_0 \defd (a \limp s_0) \with (b \limp s_1) \\
%     s_1 \defd (a \limp s_0) \with (b \limp s_1) \with (\emp \limp \one)
%   \end{lgathered}
% \end{equation*}

% \begin{equation*}
%   \nfa{q} \defd \bigwith_{a \in \ialph} \bigl({\textstyle a \limp \bigwith_{q'_a} \nfa{q}'_a}\bigr)
% \end{equation*}

% \begin{theorem}
%   Let $\aut{A} = (Q, \mathord{\nfareduces}, F)$ be \iac{NFA} over an input alphabet $\ialph$.
%   Then:
%   \begin{itemize}[nosep]
%   \item $q \nfareduces[a] q'$ if and only if $\atm{a} \oc \nfa{q} \Reduces \nfa{q}'$.
%   \item $q \in F$ if and only if $\atm{\emp} \oc \nfa{q} \Reduces \octxe$.
%   \item $q \notin F$ if and only if $\atm{\emp} \oc \nfa{q} \longarrownot\reduces$.\alertnote{Careful -- depends on focusing!}
%   \end{itemize}
%   % \item
%   %   If $\atm{a} \oc \nfa{q} \Reduces \nfa{q}'$, then $q \nfareduces[a] q'$.
%   %   If $\atm{\emp} \oc \nfa{q} \Reduces \octxe$, then $q \in F$.
% \end{theorem}


% \paragraph*{Binary counter}

% \begin{equation*}
%   \begin{lgathered}
%     e \defd (e \fuse b_1 \pmir i) \with (z \pmir d) \\
%     b_0 \defd (b_1 \pmir i) \with (d \fuse b'_0 \pmir d) \\
%     b_1 \defd (i \fuse b_0 \pmir i) \with (b_0 \fuse s \pmir d) \\
%     b'_0 \defd (z \limp z) \with (s \limp b_1 \fuse s)
%   \end{lgathered}
% \end{equation*}

\begin{itemize}
\item Alternative choreography -- how are these related?
\begin{equation*}
  \begin{lgathered}
    p \defd (i \fuse p \pmir \atmL{i}) \with (d \fuse p' \pmir \atmL{d}) \\
    p' \defd (\atmR{z} \limp \atmR{z}) \with (\atmR{s} \limp p \fuse \atmR{s}) \\
    i \defd (\atmR{e} \limp \atmR{e} \fuse \atmR{b}_1) \with (\atmR{b}_0 \limp \atmR{b}_1) \with (\atmR{b}_1 \limp i \fuse \atmR{b}_0) \\
    d \defd (\atmR{e} \limp \atmR{z}) \with (\atmR{b}_0 \limp d \fuse b'_0) \with (\atmR{b}_1 \limp \atmR{b}_0 \fuse \atmR{s}) \\
    b'_0 \defd (\atmR{z} \limp \atmR{z}) \with (\atmR{s} \limp \atmR{b}_1 \fuse \atmR{s})
  \end{lgathered}
\end{equation*}

\begin{inferences}
  \infer{\adec{\octx \oc d}{n}}{
    \ainc{\octx}{n}}
\end{inferences}

If $\octx \oc \atmL{i} \reduces \octx'$, then $\atmR{\octx} \oc i \reduces \atmR{\octx}'$.
\end{itemize}


\section{}



%%% Local Variables:
%%% mode: latex
%%% TeX-master: "thesis"
%%% End:

%% \chapter{Ordered rewriting}\label{ch:ordered-rewriting}

In this \lcnamecref{ch:ordered-rewriting}, we develop a rewriting interpretation of the ordered sequent calculus from the previous \lcnamecref{ch:ordered-logic}.

\section{Propositional ordered rewriting}

In this \lcnamecref{sec:ordered-rewriting:general}, we develop a rewriting interpretation of the ordered sequent calculus from the previous \lcnamecref{ch:ordered-logic}.
This development closely follows \citeauthor{Cervesato+Scedrov:IC09}'s work on intuitionistic linear logic as a multiset rewriting framework.

Just as their linear logical rewriting framework is more expressive than multiset rewriting, ordered rewriting framework presented in this chapter can be seen as an extension of tranditional notions of string rewriting.


\begin{equation*}
  \infer*{\oseq{\octx |- A}}{
    \oseq{\octx' |- A'}}
\end{equation*}


Many of the ordered sequent calculus's left rules consist of a single major premise with the same consequent as in the rule's conclusion [sequent], as well as a minor premise in the case of the $\lrule{\limp}$ and $\lrule{\pmir}$ rules.

Because so many rules share this scaffolding, it might be worthwhile to restructure the ordered sequent calculus to expose this shared scaffolding.
\begin{equation*}
  \infer{\oseq{\octx |- C}}{
    \octx \reduces \octx' & \oseq{\octx' |- C}}
\end{equation*}
For instance, if $\octx_L \oc (A \fuse B) \oc \octx_R \reduces \octx_L \oc A \oc B \oc \octx_R$ holds, then the usual $\lrule{\fuse}$ rule is a derivable instance of this generalized left rule.

\begin{inferences}
  \infer{\octx_1 \oc \octx_2 \reduces \octx'_1 \oc \octx_2}{
    \octx_1 \reduces \octx'_1}
  \and
  \infer{\octx_1 \oc \octx_2 \reduces \octx_1 \oc \octx'_2}{
    \octx_2 \reduces \octx'_2}
  \\
  \infer{\octx \oc (A \limp B) \reduces B}{
    \oseq{\octx |- A}}
  \and
  \infer{(B \pmir A) \oc \octx \reduces B}{
    \oseq{\octx |- A}}
  \\
  \infer{A \fuse B \reduces A \oc B}{}
  \and
  \infer{A \esuf B \reduces B \oc A}{}
  \and
  \infer{\one \reduces \octxe}{}
  \\
  \infer{A \with B \reduces A}{}
  \and
  \infer{A \with B \reduces B}{}
  \and
  \text{(no rule for $\top$)}
\end{inferences}


\begin{syntax*}
  Propositions &
    A & p \mid A \limp B \mid B \pmir A
          \mid A \fuse B \mid \one
          \mid A \with B \mid \top
  \\
  Ordered contexts & 
    \octx & \octxe \mid \octx_1 \oc \octx_2 \mid A
\end{syntax*}

\begin{itemize}
\item Lambek calculus and rewriting; compare to multiset rewriting; compare to string rewriting
\item Explain why $\plus$ and $\zero$ (and $\bot$) are undesirable here.
\item Connections to left rules
\end{itemize}

The rewriting relation is the smallest compatible relation that satisfies:
\begin{inferences}
  \infer{A \oc (A \limp B) \reduces B}{}
  \and
  \infer{(B \pmir A) \oc A \reduces B}{}
  \\
  \infer{A \with B \reduces A}{}
  \and
  \infer{A \with B \reduces B}{}
  \and
  \text{(no rule for $\top$)}
  \\
  \infer{A \fuse B \reduces A \oc B}{}
  \and
  \infer{\one \reduces \octxe}{}
\end{inferences}
We will also refer to this relation as \vocab{reduction}%
\footnote{Input transitions are postponed to \cref{ch:ordered-bisimilarity}.}%
.

$\Reduces$ is the reflexive-transitive closure of $\reduces$


\begin{equation*}
  \infer[\lrule{\with}_1]{\oseq{\octx'_L \oc (A \with B) \oc \octx'_R |- \gamma}}{
    \oseq{\octx'_L \oc A \oc \octx'_R |- \gamma}}
  \leftrightsquigarrow
  \infer{\octx'_L \oc (A \with B) \oc \octx'_R \reduces \octx'_L \oc A \oc \octx'_R}{
    \infer{A \with B \reduces A}{}}
\end{equation*}

\begin{equation*}
  \infer[\lrule{\limp}]{\oseq{\octx'_L \oc \octx \oc (A \limp B) \oc \octx'_R |- \gamma}}{
    \oseq{\octx |- A} & \oseq{\octx'_L \oc B \oc \octx'_R |- \gamma}}
  \rightsquigarrow
  \infer[\lrule{\limp}']{\oseq{\octx'_L \oc A \oc (A \limp B) \oc \octx'_R |- \gamma}}{
    \oseq{\octx'_L \oc B \oc \octx'_R |- \gamma}}
  \leftrightsquigarrow
  \infer{\octx'_L \oc A \oc (A \limp B) \oc \octx'_R \reduces \octx'_L \oc B \oc \octx'_R}{
    \infer{A \oc (A \limp B) \reduces B}{}}
\end{equation*}


\subsection{Definitions}

\begin{itemize}
\item not very interesting without recursion
\end{itemize}

\subsection{Examples}

\subsubsection{Automata and transducers}

\begin{equation*}
  \begin{lgathered}[t]
    q_0 \defd (a \limp q_0) \with (b \limp q_0 \with q_1) \\
    q_1 \defd (a \limp q_2) \with (b \limp q_2) \with (\emp \limp \one) \\
    q_2 \defd (a \limp q_2) \with (b \limp q_2)
  \end{lgathered}
  \qquad
  \begin{lgathered}[t]
    s_0 \defd (a \limp s_0) \with (b \limp s_1) \\
    s_1 \defd (a \limp s_0) \with (b \limp s_1) \with (\emp \limp \one)
  \end{lgathered}
\end{equation*}

\begin{equation*}
  \nfa{q} \defd \bigwith_{a \in \ialph} \bigl({\textstyle a \limp \bigwith_{q'_a} \nfa{q}'_a}\bigr)
\end{equation*}

\begin{theorem}
  Let $\aut{A} = (Q, \mathord{\nfareduces}, F)$ be \iac{NFA} over an input alphabet $\ialph$.
  Then:
  \begin{itemize}[nosep]
  \item $q \nfareduces[a] q'$ if and only if $\atm{a} \oc \nfa{q} \Reduces \nfa{q}'$.
  \item $q \in F$ if and only if $\atm{\emp} \oc \nfa{q} \Reduces \octxe$.
  \item $q \notin F$ if and only if $\atm{\emp} \oc \nfa{q} \longarrownot\reduces$.\alertnote{Careful -- depends on focusing!}
  \end{itemize}
  % \item
  %   If $\atm{a} \oc \nfa{q} \Reduces \nfa{q}'$, then $q \nfareduces[a] q'$.
  %   If $\atm{\emp} \oc \nfa{q} \Reduces \octxe$, then $q \in F$.
\end{theorem}


\subsubsection{Binary counter}

\begin{equation*}
  \begin{lgathered}
    e \defd (e \fuse b_1 \pmir i) \with (z \pmir d) \\
    b_0 \defd (b_1 \pmir i) \with (d \fuse b'_0 \pmir d) \\
    b_1 \defd (i \fuse b_0 \pmir i) \with (b_0 \fuse s \pmir d) \\
    b'_0 \defd (z \limp z) \with (s \limp b_1 \fuse s)
  \end{lgathered}
\end{equation*}

\begin{itemize}
\item Concurrency that is not message-passing
\item Alternative choreography -- how are these related?
\begin{equation*}
  \begin{lgathered}
    i \defd (e \limp e \fuse b_1) \with (b_0 \limp b_1) \with (b_1 \limp i \fuse b_0) \\
    d \defd (e \limp z) \with (b_0 \limp d \fuse b'_0) \with (b_1 \limp b_0 \fuse s) \\
    b'_0 \defd (z \limp z) \with (s \limp b_1 \fuse s)
  \end{lgathered}
\end{equation*}

\begin{equation*}
  \begin{lgathered}
    i \defd (e \limp e \fuse b_1) \with (b_0 \limp b_1) \with (b_1 \limp i \fuse b_0) \\
    d \defd (e \limp z) \with (b_0 \limp d \fuse b'_0) \with (b_1 \limp b_0 \fuse s) \\
    z \defd z \pmir b'_0 \\
    s \defd b_1 \fuse s \pmir b'_0
  \end{lgathered}
\end{equation*}
\end{itemize}


\section{}



%%% Local Variables:
%%% mode: latex
%%% TeX-master: "thesis"
%%% End:

% 

\chapter{MOVE THESE}

\section{Choreographies}

Recall the string rewriting specification
\begin{equation*}
  \infer{a \oc b \reduces b}{}
  \qquad\text{and}\qquad
  \infer{b \reduces \emp}{}
  \:.
\end{equation*}

A choreography is a refinement of this specification in which each symbol $a$ of the rewriting alphabet is mapped to an ordered proposition: either an atomic proposition, $\atmL{a}$ or $\atmR{a}$, or a recursively defined proposition, $\proc{a}$.
In other words, a choreography is an injection from symbols to propositions.
\begin{equation*}
  \begin{tikzcd}
    w \rar[reduces] \dar[dash] & w' \dar[dash, exists]
    \\
    \mathllap{\theta}(w) \rar[Reduces, exists] & \theta(w')
  \end{tikzcd}
  \begin{tikzcd}
    \mathllap{\theta}(w) \rar[reduces] \dar[dash] & \octx' \rar[Reduces, exists] & \theta(w') \dar[dash, exists]
    \\
    w \arrow[reduces, exists]{rr} && w'
  \end{tikzcd}
\end{equation*}

$\atmR{a} \oc \proc{b}$

Suppose that $\theta$ is the mapping $a \mapsto \atmR{a}$ and $b \mapsto \proc{b}$.
and the choreography
\begin{equation*}
  \proc{b} \defd (\atmR{a} \limp \proc{b}) \with \one
  \,.
\end{equation*}
Notice that 
\begin{alignat*}{2}
  &a \oc b \reduces b
  &&\quad\text{and}\quad
  b \reduces \emp
\shortintertext{as well as}
  &\atmR{a} \oc \proc{b} \reduces \atmR{a} \oc (\atmR{a} \limp \proc{b}) \reduces \proc{b}
  &&\quad\text{and}\quad
  \proc{b} \reduces \one \reduces \octxe
  \:.
\end{alignat*}


\begin{equation*}
  \infer{a \oc b \reduces b}{}
  \qquad\text{and}\qquad
  \infer{c \oc b \reduces b}{}
\end{equation*}

\begin{equation*}
  \proc{b} \defd (\atmR{a} \limp \proc{b}) \with (\atmR{c} \limp \proc{b})
\end{equation*}

\begin{equation*}
  a \oc b \reduces w' \text{ implies $w' = b$}
  \quad\text{but}\quad
  \atmR{a} \oc \proc{b} \reduces \atmR{a} \oc (\atmR{c} \limp \proc{b}) \nreduces
\end{equation*}

\autocite{McDowell+:TCS03}


Judgments $\chorsig{\theta}{\sig}{\sig'}$ and $\chorax{\theta}{w \reduces w'}{\proc{a}}{A}$.
In both judgments, all terms before the $\chorarrow$ are inputs; all terms after the $\chorarrow$ are outputs.


\begin{inferences}
  \infer{\chorsig{\theta}{\sige}{\sige}}{}
  \and
  \infer{\chorsig{\theta}{\sig, w \reduces w'}{\sig', \proc{a} \defd A_1 \with A_2}}{
    \chorsig{\theta}{\sig}{\sig'} &
    \chorax{\theta}{w \reduces w'}{\proc{a}}{A_2} &
    \text{($\sig'(\proc{a}) = A_1$)}}
  \\
  \infer{\chorsig{\theta}{\sig, w \reduces w'}{\sig', \proc{a} \defd A}}{
    \chorsig{\theta}{\sig}{\sig'} &
    \chorax{\theta}{w \reduces w'}{\proc{a}}{A} &
    \text{($\proc{a} \notin \dom{\sig'}$)}}
  \\
  \infer{\chorax{\theta}{a \reduces w'}{\proc{a}}{\up (\bigfuse \octx')}}{
    \text{($\theta(a) = \proc{a}$)} &
    \text{($\theta(w') = \octx'$)}}
  \\
  \infer{\chorax{\theta}{b \oc w \reduces w'}{\proc{a}}{\atmR{b} \limp A}}{
    \chorax{\theta}{w \reduces w'}{\proc{a}}{A} &
    \text{($\theta(b) = \atmR{b}$)}}
  \and
  \infer{\chorax{\theta}{w \oc b \reduces w'}{\proc{a}}{A \pmir \atmL{b}}}{
    \chorax{\theta}{w \reduces w'}{\proc{a}}{A} &
    \text{($\theta(b) = \atmL{b}$)}}
\end{inferences}

\begin{theorem}
  \begin{itemize}
  \item If $\chorsig{\theta}{\sig}{\sig'}$ and $w \reduces_{\sig} w'$, then $\theta(w) \reduces_{\sig'} \theta(w')$.
    If $\chorsig{\theta}{\sig}{\sig'}$ and $\octx \reduces_{\sig'} \octx'$, then $\theta^{-1}(\octx) \reduces_{\sig} \theta^{-1}(\octx')$.
  \item If $\chorax{\theta}{w \reduces w'}{\proc{a}}{A}$, then $\theta(w) \reduces_{\proc{a} \defd A} \theta(w')$.
    If $\chorax{\theta}{w \reduces w'}{\proc{a}}{A}$ and $\octx \reduces_{\proc{a} \defd A} \octx'$, then $\theta^{-1}(\octx) \reduces \theta^{-1}(\octx')$.
  \end{itemize}
\end{theorem}
\begin{proof}
  $\proc{a} \reduces \bigfuse \theta(w')$

  $\atmR{b} \oc \theta(w) \reduces_{\proc{a} \defd \atmR{b} \limp A} \theta(w')$ if $\theta(w) \reduces_{\proc{a} \defd A} \theta(w')$

  $\theta(w) \oc \atmL{b} \reduces_{\proc{a} \defd A \pmir \atmL{b}} \theta(w')$ if $\theta(w) \reduces_{\proc{a} \defd A} \theta(w')$


  
\end{proof}

When $\theta = \Set{(a, \atmR{a}), (b, \proc{b})}$, the judgment $\chorsig{\theta}{\sig}{\proc{b} \defd (\atmR{a} \limp \proc{b}) \with \one}$ holds.
However, $b \reduces \emp$ but $\proc{b} \nreduces \octxe$.



\begin{equation*}
  \begin{lgathered}
    \proc{e} \defd \up (\dn \proc{e} \fuse \dn \proc{b}_1) \pmir \atmL{i} \\
    \proc{b}_0 \defd \up \dn \proc{b}_1 \pmir \atmL{i} \\
    \proc{b}_1 \defd \up (\atmL{i} \fuse \dn \proc{b}_0) \pmir \atmL{i}
  \end{lgathered}
\end{equation*}

\begin{equation*}
  \begin{lgathered}
    \proc{e} \defd \bigl(\up (\dn \proc{e} \fuse \dn \proc{b}_1) \pmir \atmL{i}\bigr) \with (\up \atmR{z} \pmir \atmL{d})
  \end{lgathered}
\end{equation*}

\begin{equation*}
  \begin{lgathered}
    \proc{\imath} \defd (\atmR{e} \limp \up (\atmR{e} \fuse \atmR{b}_1)) \with (\atmR{b}_0 \limp \up \atmR{b}_1) \with (\atmR{b}_1 \limp \up (\dn \proc{\imath} \fuse \atmR{b}_0))
  \end{lgathered}
\end{equation*}



\subsection{}



\section{}

Atomic ordered propositions are viewed as messages; compound ordered propositions, as processes; and ordered contexts, as configurations of processes.

The ordered contexts form a monoid over the positive propositions and are given by
\begin{equation*}
  \octx \Coloneqq \octx_1 \oc \octx_2 \mid \octxe \mid \p{A}
  \,.
\end{equation*}
In keeping with the monoid laws, we treat $(\octx_1 \oc \octx_2) \oc \octx_3$ and $\octx_1 \oc (\octx_2 \oc \octx_3)$ as syntactically indistinguishable, as we also do for $\octx \oc (\octxe)$ and $\octx$ and $(\octxe) \oc \octx$.

Each atom is consistently assigned a direction, either left-directed, $\atmL{a}$, or right-directed, $\atmR{a}$.

An atom's direction and position within the larger context together indicate whether, when viewed as a message, it is being sent or received.
In the context $\octx_1 \oc \atmR{a} \oc \octx_2$, the atom $\atmR{a}$ is a message being sent from $\octx_1$ to $\octx_2$.
Symmetrically, in the context $\octx_1 \oc \atmL{a} \oc \octx_2$, the atom $\atmL{a}$ is a message being sent from $\octx_1$ to $\octx_2$.

The context $\octx = \octx' \oc \atmR{a}$ is a process configuration that sends $\atmR{a}$ to its right and continues as $\octx'$.
Conversely, $\atmR{a} \oc \octx$ is a process configuration in which $\octx$ is the intended recipient of the message $\atmR{a}$.


\begin{align*}
  \p{A} &\Coloneqq \atmL{a} \mid \atmR{a} \mid \p{\hat{p}} \mid \p{A} \fuse \p{B} \mid \one \mid \dn \n{A} \\
  \n{A} &\Coloneqq \n{\hat{p}} \mid \atmR{a} \limp \n{B} \mid \n{B} \pmir \atmL{a} \mid \n{A} \with \n{B} \mid \top \mid \up \p{A}
\end{align*}

\section{Choreographing specifications}

\begin{equation*}
  \infer{a \oc b \reduces b}{}
  \qquad\text{and}\qquad
  \infer{b \reduces \octxe}{}
\end{equation*}

As a specification, these string rewriting axioms are quite reasonable.
However, as a [...], [...].

Toward our ultimate goal of relating the proof-construction and proof-reduction appraches to concurrency, we would like a description of this concurrent system that is slightly more concrete.

\begin{align*}
  \atmR{a} \oc \hat{b} &\reduces \hat{b} \\
  \hat{b} &\reduces \octxe
\end{align*}

\begin{equation*}
  \hat{b} \defd (\atmR{a} \limp \up \dn \hat{b}) \with \one
\end{equation*}


\begin{inferences}
  \infer{e \oc i \reduces e \oc b_1}{}
  \and
  \infer{b_0 \oc i \reduces b_1}{}
  \and\text{and}\and
  \infer{b_1 \oc i \reduces i \oc b_0}{}
\end{inferences}

\begin{equation*}
  \begin{lgathered}
    \bin{e} \defd \bin{e} \fuse \bin{b}_1 \pmir \atmL{i} \\
    \bin{b}_0 \defd \bin{b}_1 \pmir \atmL{i} \\
    \bin{b}_1 \defd \atmL{i} \fuse \bin{b}_0 \pmir \atmL{i}
  \end{lgathered}
\end{equation*}

\begin{equation*}
  \begin{lgathered}
    e \simu{R} \bin{e} \\
    b_0 \simu{R} \bin{b}_0 \\
    b_1 \simu{R} \bin{b}_1 \\
    i \simu{R} \atmL{i} \\
    e \oc b_1 \simu{R} \bin{e} \fuse \bin{b}_1 \\
    i \oc b_0 \simu{R} \atmL{i} \fuse \bin{b}_0
  \end{lgathered}
\end{equation*}
$\simu{R}$ is a reduction bisimulation.

$\bin{e} \oc \atmL{i} \Reduces \bin{e} \oc \bin{b}_1$
and $e \oc i \Reduces e \oc b_1$
and $e \oc b_1 \Reduces e \oc b_1$

\begin{equation*}
  \begin{lgathered}
    \bin{\imath} \defd (\atmR{e} \limp \atmR{e} \fuse \atmR{b}_1)
               \with (\atmR{b}_0 \limp \atmR{b}_1)
               \with (\atmR{b}_1 \limp \bin{\imath} \fuse \atmR{b}_0)
  \end{lgathered}
\end{equation*}

$\atmR{e} \oc \bin{\imath} \Reduces \atmR{e} \oc \atmR{b}_1$
and $e \oc i \Reduces e \oc b_1$
and $e \oc b_1 \Reduces e \oc b_1$


\begin{equation*}
  \dfa{q} \defd \bigwith_{a \in \ialph}(\atmR{a} \limp \dfa{q}'_a)
\end{equation*}

Compare:
\begin{itemize}
\item $q \dfareduces[a] q'_a$ if, and only if, $\atmR{a} \oc \dfa{q} \reduces \dfa{q}'_a$.
\item $q \dfareduces[a] q'_a$ if, and only if, $\atmR{a} \oc \dfa{q} \reduces \octx'$ for some $\octx' = \dfa{q}'_a$.
\item $q \dfareduces[a] q'_a$ and $\dfa{q}'_a = \octx'$ for some $q'_a$ if, and only if, $\atmR{a} \oc \dfa{q} \reduces \octx'$.
\end{itemize}
These differ in the placement of the existential quantifier.
The first pair are, in fact, false.

\emph{For the former:}
Assume that $q \dfareduces[a] q''_a$ and $\dfa{q}''_a = \octx' = \dfa{q}'_a$.
It might be that the states $q''_a$ and $q'_a$ are only bisimilar, not equal.
In that case, $q \dfareduces[a]\asim q'_a$ but, in general, not $q \dfareduces[a] q'_a$ directly.

\emph{For the latter:}
Assume that $q \dfareduces[a] q''_a$ and $\dfa{q}''_a = \octx'$.
Choosing $q'_a \coloneqq q''_a$, we indeed have $q \dfareduces[a] q'_a$ and $\dfa{q}'_a = \octx'$. 

\begin{equation*}
\begin{tikzcd}
  q \rar[reduces, "a", exists] \dar[dash, "\simu{R}"'] & q'_a \dar[dash, "\simu{R}"]
  \\
  \atmR{a} \oc \dfa{q} \rar[reduces] & \octx' \mathrlap{{} = \dfa{q}'_a}
\end{tikzcd}
\qquad\qquad
\begin{tikzcd}
  q \rar[reduces, "a", exists] \dar[dash, "\simu{R}"'] & q'_a \dar[dash, "\simu{R}", exists]
  \\
  \atmR{a} \oc \dfa{q} \rar[reduces] & \octx' \mathrlap{{} = \dfa{q}'_a}
\end{tikzcd}
\end{equation*}



\section{Encoding \aclp*{DFA}}

Recall from \cref{??} our string rewriting specification of how \iac{NFA} processes its input.
Given \iac{DFA} $\aut{A} = (Q, ?, F)$ over an input alphabet $\ialph$, the \ac{NFA}'s operational semantics are adequately captured by the folllwing string rewriting axioms:
\begin{equation*}
  \infer{a \oc q \reduces q'_a}{}
  \enspace\text{for each transition $q \nfareduces[a] q'_a$.}
\end{equation*}
\begin{equation*}
  \infer{\emp \oc q \reduces F(q)}{}
  \enspace\text{for each state $q$, where}\enspace
  F(q) = \begin{cases*}
           (\octxe) & if $q \in F$ \\
           \symrej & if $q \notin F$\,.
         \end{cases*}
\end{equation*}

\clearpage
\subsection{A functional choreography}

One possible choreography for this specification treats the input symbols $a \in \ialph$ as atomic propositions $\atmR{a}$; states $q \in Q$ as recursively defined propostions $\proc{q}$;and the end-of-word marker $\emp$ as an atomic proposition $\atmR{\emp}$.
In other words, the \ac{NFA}'s input is treated as a sequence of messages, $\atmR{\emp} \oc \atmR{a}_n \dotsm \atmR{a}_2 \oc \atmR{a}_1$, and the \ac{NFA}'s states are treated as [recursive] processes.

$a \mapsto \atmR{a}$ for all $a \in \ialph$; $q \mapsto \proc{q}$ for all $q \in Q$; and $\emp \mapsto \atmR{\emp}$.

Using this assignment, the choreography constructed from the specification consists of the following definition, one for each \ac{NFA} state $q \in Q$:
\begin{equation*}
  \proc{q} \defd \bigwith_{a \in \ialph} \bigwith_{q\smash{'_a}} (\atmR{a} \limp \proc{q}'_a) \with (\atmR{\emp} \limp \nfa{F}(q))
  \,.
\end{equation*}

\begin{corollary}
  If $a \oc q \reduces q'_a$, then $\atmR{a} \oc \proc{q} \Reduces \proc{q}'_a$.
  If $\atmR{a} \oc \proc{q} \reduces \octx'$, then $a \oc q \reduces w'$ and $\octx' \Reduces \theta(w')$.
\end{corollary}

\begin{corollary}
  If $q \nfareduces[a] q'_a$, then $\atmR{a} \oc \proc{q} \Reduces \proc{q}'_a$.
  If $\atmR{a} \oc \proc{q} \reduces \proc{q}'_a$, then $q \nfareduces[a] q''_a$ for some $q''_a$ such that $\proc{q}'_a = \proc{q}''_a$.
\end{corollary}

As an extended example, we will use ordered rewriting to specify how \iac{DFA} processes its input.
%
% \Acp{DFA} serve as an example of ordered rewriting,  can be used to specify how \iac{DFA} processes its input.
%
Given \iac{DFA} $\aut{A} = (Q, ?, F)$ over an input alphabet $\ialph$, the idea is to encode each state, $q \in Q$, as an ordered proposition, $\dfa{q}$, in such a way that the \ac{DFA}'s operational semantics are adequately captured by [ordered] rewriting.
%
% The basic idea is to define an encoding, $\dfa{q}$, of \ac{DFA} states as ordered propositions;
% this encoding should adequately reflect the \ac{DFA}'s operational semantics with ordered rewriting traces.
\fixnote{[In general, the behavior of \iac{DFA} state is recursive, so the proposition $\dfa{q}$ will be recursively defined.]}
%
% finite input words, $w \in \finwds{\ialph}$, are encoded as ordered contexts by $\emp \oc \rev{w}$

% \NewDocumentCommand \rev { s m } {
%   \IfBooleanTF {#1}
%     { (#2)^{\mathsf{R}} }
%     { #2^{\mathsf{R}} }
% }

% \begin{align*}
%   \rev{a} &= a \\
%   \rev*{w_1 \wc w_2} &= \rev{w_2} \oc \rev{w_1} \\
%   \rev{\emp} &= \octxe
% \end{align*}

Ideally, \ac{DFA} transitions $q \dfareduces[a] q'_a$ would be in bijective correspondence with rewriting steps $a \oc \dfa{q} \reduces \dfa{q}'_a$, where each input symbol $a$ is encoded by a matching [propositional] atom.
%
We will return to the possibility of this kind of tight correspondence in \cref{??}, but,
%
for now, we will content ourselves with a correspondence with traces rather than individual steps, adopting the following desiderata:
% Unfortunately, ordered rewriting's small step size turns out to be a poor match for [...], so in both cases we will instead content ourselves with corrspondances with \emph{traces}:
% a bijection between transitions $q \dfareduces[a] q'_a$ and \emph{traces} $a \oc \dfa{q} \Reduces \dfa{q}'_a$.
% Similarly, [...] a bijection between accepting states $q \in F$ and traces $\emp \oc \dfa{q} \Reduces \octxe$.
%
% This leads us to adopt the following as desiderata:
\begin{itemize}
\item
  $q \dfareduces[a] q'_a$ if, and only if, $a \oc \dfa{q} \Reduces \dfa{q}'_a$, for all input symbols $a \in \ialph$.
\item
  $q \in F$ if, and only if, $\emp \oc \dfa{q} \Reduces \one$, where the atom $\emp$ functions as an end-of-word marker.
% \item
%   $q \dfareduces[w] q'_w \in F$ if, and only if, $\emp \oc \rev{w} \oc \dfa{q} \Reduces \octxe$.
%   Also, $q \dfareduces[w] q'_w \notin F$ if, and only if, $\emp \oc \rev{w} \oc \dfa{q} \Reduces \top$.
\end{itemize}
Given the reversal (anti-)\-homo\-morph\-ism from finite words to ordered contexts defined in the adjacent \lcnamecref{fig:ordered-rewriting:reversal}%
\begin{marginfigure}
  \begin{align*}
    \rev*{w_1 \wc w_2} &= \rev{w_2} \oc \rev{w_1} \\
    \rev{\emp} &= \octxe \\
    \rev{a} &= a
  \end{align*}
  \caption{An (anti-)\-homo\-morph\-ism for reversal of finite words to ordered contexts}\label{fig:ordered-rewriting:reversal}
\end{marginfigure}%
, the first desideratum is subsumed by a third:
% property that covers finite words:
\begin{itemize}[resume*]
\item $q \dfareduces[w] q'$ if, and only if, $\rev{w} \oc \dfa{q} \Reduces \dfa{q}'$, for all finite words $w \in \finwds{\ialph}$.
\end{itemize}

From these desiderata [and the observation that \acp{DFA}' graphs frequently%
\fixnote{Actually, there is always at least one cycle in a well-formed \ac{DFA}.}
contain cycles], we arrive at the following encoding, in which each state is encoded by one of a collection of mutually recursive definitions:%
\fixnote{$q'_a$, using function or relation?}
\begin{gather*}
  \dfa{q} \defd
    \parens[size=big]{
      \bigwith_{a \in \ialph}(a \limp \dfa{q}'_a)}
    \with
    \parens[size=big]{\emp \limp \dfa{F}(q)}
  % \text{where
  %   $q \dfareduces[a] q'_a$ for all $a \in \ialph$
  %   and
  %   $\dfa{F}(q) = 
  %     \begin{cases*}
  %       \one & if $q \in F$ \\
  %       \top & if $q \notin F$
  %     \end{cases*}$%
  % }
  %
\shortintertext{where}
  %
  q \dfareduces[a] q'_a
  \text{, for all input symbols $a \in \ialph$,\quad and\quad}
  \dfa{F}(q) = 
    \begin{cases*}
      \one & if $q \in F$ \\
      \top & if $q \notin F$%
    \,.
    \end{cases*}
\end{gather*}
Just as each state $q$ has exactly one successor for each input symbol $a$, its encoding, $\dfa{q}$, has exactly one input clause, $(a \limp \dotsb)$, for each symbol $a$.



% The traces $a \oc \dfa{q} \Reduces \dfa{q}'_a$
% % for input symbols $a \in \ialph$
% suggest that $\dfa{q}$ should be a collection of clauses that input atoms $a$ from the left.
% And the traces $\emp \oc \dfa{q} \Reduces \octxe$ or $\emp \oc \dfa{q} \Reduces \top$ suggest that $\dfa{q}$ also contain a clause that inputs atom $\emp$ from the left.
% Thus, we arrive at the encoding


\newthought{For a concrete instance} of this encoding, recall from \cref{ch:automata} the \ac{DFA} (repeated in the adjacent \lcnamecref{fig:ordered-rewriting:dfa-example-ends-b})%
%
\begin{marginfigure}
  \begin{equation*}
    \mathllap{\aut{A}_2 = {}}
    \begin{tikzpicture}[baseline=(q_0.base)]
      \graph [automaton] {
        q_0
         -> [loop above, "a"]
        q_0
         -> ["b", bend left]
        q_1 [accepting]
         -> [loop above, "b"]
        q_1
         -> ["a", bend left]
        q_0;
      };
    \end{tikzpicture}
  \end{equation*}
  \caption{\Iac*{DFA} that accepts, from state $q_0$, exactly those words that end with $b$. (Repeated from \cref{fig:dfa-example-ends-b}.)}\label{fig:ordered-rewriting:dfa-example-ends-b}
\end{marginfigure}
%
that accepts exactly those words, over the alphabet $\ialph = \set{a,b}$, that end with $b$; that \ac{DFA} is encoded by the following definitions:
\begin{equation*}
  \begin{lgathered}
    \dfa{q}_0 \defd (a \limp \dfa{q}_0) \with (b \limp \dfa{q}_1) \with (\emp \limp \top) \\
    \dfa{q}_1 \defd (a \limp \dfa{q}_0) \with (b \limp \dfa{q}_1) \with (\emp \limp \one)
  \end{lgathered}
\end{equation*}
Indeed, just as the \ac{DFA} has a transition $q_0 \dfareduces[b] q_1$, its encoding admits a trace
\begin{align*}
  &b \oc \dfa{q}_0
     = b \oc \bigl((a \limp \dfa{q}_0) \with (b \limp \dfa{q}_1) \with (\emp \limp \top)\bigr)
     \Reduces b \oc (b \limp \dfa{q}_1)
     \reduces \dfa{q}_1
  \,.
\intertext{And, just as $q_1$ is an accepting state, its encoding also admits a trace}
  &\emp \oc \dfa{q}_1 = \emp \oc \bigl((a \limp \dfa{q}_0) \with (b \limp \dfa{q}_1) \with (\emp \limp \one)\bigr) \Reduces \emp \oc (\emp \limp \one) \reduces \one
  \,.
\end{align*}

\newthought{More generally}, this encoding is complete, in the sense that it simulates all \ac{DFA} transitions: $q \dfareduces[a] q'$ implies $a \oc \dfa{q} \Reduces \dfa{q}'$, for all states $q$ and $q'$ and input symbols $a$.

However, the converse does not hold -- the encoding is unsound because there are rewritings that cannot be simulated by \iac{DFA} transition.
% That is, $a \oc \dfa{q} \Reduces \dfa{q}'$ does \emph{not} imply $q \dfareduces[a] q'$.
% 
\begin{falseclaim}
  Let $\aut{A} = (Q, \mathord{\dfareduces}, F)$ be \iac{DFA} over the input alphabet $\ialph$.
  Then $a \oc \dfa{q} \Reduces \dfa{q}'$ implies $q \dfareduces[a] q'$, for all input symbols $a \in \ialph$.
\end{falseclaim}
%
\begin{marginfigure}
    \centering
    % \subfloat[][]{\label{fig:ordered-rewriting:dfa-counterexample:dfa}%
      \begin{equation*}
        \aut{A}'_2 = 
      \begin{tikzpicture}[baseline=(q_0.base)]
        \graph [automaton] {
          q_0
           -> [loop above, "a"]
          q_0
           -> ["b", bend left]
          q_1 [accepting]
           -> [loop above, "b"]
          q_1
           -> ["a", bend left]
          q_0;
          %
%          { [chain shift={(2,0)}]
            s_1 [accepting, below=1.5em of q_1.south]
             -> [loop right, "b"]
            s_1
             -> ["a", bend left]
            q_0;
%          };
        };
      \end{tikzpicture}
    \end{equation*}
    % }
    % \subfloat[][]{\label{fig:ordered-rewriting:dfa-counterexample:encoding}%
      $\!\begin{aligned}
        \dfa{q}_0 &\defd (a \limp \dfa{q}_0) \with (b \limp \dfa{q}_1) \with (\emp \limp \top) \\
        \dfa{q}_1 &\defd (a \limp \dfa{q}_0) \with (b \limp \dfa{q}_1) \with (\emp \limp \one) \\
        \dfa{s}_1 &\defd (a \limp \dfa{q}_0) \with (b \limp \dfa{s}_1) \with (\emp \limp \one)
      \end{aligned}$%
    % }
    \caption{{fig:ordered-rewriting:dfa-counterexample:dfa}~A slightly modified version of the \ac*{DFA} from \cref{fig:ordered-rewriting:dfa-example-ends-b}; and {fig:ordered-rewriting:dfa-counterexample:encoding}~its encoding}\label{fig:ordered-rewriting:dfa-counterexample}
  \end{marginfigure}%
\begin{proof}[Counterexample]
  Consider the \ac{DFA} and encoding shown in the adjacent \lcnamecref{fig:ordered-rewriting:dfa-counterexample}; it is the same \ac{DFA} as shown in \cref{fig:ordered-rewriting:dfa-example-ends-b}, but with one added state, $s_1$, that is unreachable from $q_0$ and $q_1$.
    %
  % When encoded as an ordered rewriting specification, it corresponds to the following definitions:
  % \begin{equation*}
  %   \begin{lgathered}
  %     \dfa{q}_0 \defd (a \limp \dfa{q}_0) \with (b \limp \dfa{q}_1) \with (\emp \limp \top) \\
  %     \dfa{q}_1 \defd (a \limp \dfa{q}_0) \with (b \limp \dfa{q}_1) \with (\emp \limp \one) \\
  %     \dfa{s}_1 \defd (a \limp \dfa{q}_0) \with (b \limp \dfa{s}_1) \with (\emp \limp \one)
  %   \end{lgathered}
  % \end{equation*}
  Notice that, as a coinductive consequence of the equirecursive treatment of definitions, $\dfa{q}_1 = \dfa{s}_1$.
  Previously, we saw that $b \oc \dfa{q}_0 \Reduces \dfa{q}_1$; hence $b \oc \dfa{q}_0 \Reduces \dfa{s}_1$.
  However, the \ac{DFA} has no $q_0 \dfareduces[b] s_1$ transition (because $q_1 \neq s_1$), and so this encoding is unsound with respect to the operational semantics of \acp{DFA}.
\end{proof}

As this counterexample shows, the lack of adequacy stems from attempting to use an encoding that is not injective -- here, $q_1 \neq s_1$ even though $\dfa{q}_1 = \dfa{s}_1$.
In other words, eqality of state encodings is a coarser eqvivalence than equality of the states themselves.

One possible remedy for this lack of adequacy might be to revise the encoding to have a stronger nominal character.
By tagging each state's encoding with an atom that is unique to that state, we can make the encoding manifestly injective.
For instance, given the pairwise distinct atoms $\Set{q \given q \in F}$ and $\Set{\bar{q} \given q \in Q - F}$ to tag final and non-final states, respectively, we could define an alternative encoding, $\check{q}$:
%
\begin{gather*}
  \check{q} \defd
    \parens[size=big]{
      \bigwith_{a \in \ialph}(a \limp \check{q}'_a)}
    \with
    \parens[size=big]{\emp \limp \check{F}(q)}
  %
  \shortintertext{where}
  %
  q \dfareduces[a] q'_a
  \text{, for all input symbols $a \in \ialph$,\quad and\quad}
  \check{F}(q) =
    \begin{cases*}
      q & if $q \in F$ \\
      \bar{q} & if $q \notin F$%
    \,.
    \end{cases*}
\end{gather*}
%
Under this alternative encoding, the states $q_1$ and $s_1$ of \cref{fig:ordered-rewriting:dfa-counterexample} are no longer a counterexample to injectivity:
Because $q_1$ and $s_1$ are distinct states, they correspond to distinct tags, and so $\check{q}_1 \neq \check{s}_1$.

% One possible remedy
% % for this apparent lack of adequacy
% might be to revise the encoding to have a stronger nominal character % .
% by tagging each state's encoding with an atom that is unique to that state.
% For instance, given the pairwise distinct atoms $\set{q \given q \in F}$ and $\set{\bar{q} \given q \in Q - F}$ to tag final and non-final states, respectively, we could define an alternative encoding, $\check{q}$, that is manifestly injective:
% %
% % \begin{marginfigure}
% \begin{gather*}
%   \check{q} \defd
%     \parens[size=big]{
%       \bigwith_{a \in \ialph}(a \limp \check{q}'_a)}
%     \with
%     \parens[size=big]{\emp \limp \check{F}(q)}
%   %
%   \shortintertext{where}
%   %
%   q \dfareduces[a] q'_a
%   \text{, for all input symbols $a \in \ialph$,\quad and\quad}
%   \check{F}(q) =
%     \begin{cases*}
%       q & if $q \in F$ \\
%       \bar{q} & if $q \notin F$%
%     \,.
%     \end{cases*}
% \end{gather*}
% % \end{marginfigure}%
% % , the encoding can be made to be injective.
% % With this change, the alternative encoding is now injective: $\check{q} = \check{s}$ implies $q = s$.

Although such a solution is certainly possible, it seems unsatisfyingly ad~hoc.
A closer examination of the preceding counterexample reveals that the states $q_1$ and $s_1$, while not equal, are in fact bisimilar~\parencref{??}.
In other words, although the encoding is not, strictly speaking, injective, it is injective \emph{up to bisimilarity}: $\dfa{q} = \dfa{s}$ implies $q \asim s$.
This suggests a more elegant solution to the apparent lack of adequacy: the encoding's adequacy should be judged up to \ac{DFA} bisimilarity.
%
\newcommand{\dfaadequacybisimbody}{%
  Let $\aut{A} = (Q, ?, F)$ be \iac{DFA} over the input alphabet $\ialph$.
  Then, for all states $q$, $q'$, and $s$:
  \begin{enumerate}
  \item\label{enum:ordered-rewriting:dfa-adequacy:1}
    $q \asim s$ if, and only if, $\dfa{q} = \dfa{s}$.
  \item\label{enum:ordered-rewriting:dfa-adequacy:2}
    $q \asim\dfareduces[a]\asim q'$ if, and only if, $a \oc \dfa{q} \Reduces \dfa{q}'$, for all input symbols $a \in \ialph$.    
    More generally, $q \asim\dfareduces[w]\asim q'$ if, and only if, $\rev{w} \oc \dfa{q} \Reduces \dfa{q}'$, for all finite words $w \in \finwds{\ialph}$.
  \item\label{enum:ordered-rewriting:dfa-adequacy:3}
    $q \in F$ if, and only if, $\emp \oc \dfa{q} \Reduces \one$.
  \end{enumerate}%
}%
%  
\begin{restatable*}[
  name=\ac*{DFA} adequacy up to bisimilarity,
  label=thm:ordered-rewriting:dfa-adequacy-bisim
]{theorem}{dfaadequacybisim}
  \dfaadequacybisimbody
% Let $\aut{A} = (Q, \mathord{\dfareduces}, F)$ be \iac{DFA} over the input alphabet $\ialph$.
%   Then, for all states $q$, $q'$, and $s$:
%   \begin{enumerate}
%   \item\label{enum:ordered-rewriting:dfa-adequacy:1}
%     $q \asim s$ if, and only if, $\dfa{q} = \dfa{s}$.
%   \item\label{enum:ordered-rewriting:dfa-adequacy:2}
%     $q \asim\dfareduces[a]\asim q'$ if, and only if, $a \oc \dfa{q} \Reduces \dfa{q}'$, for all input symbols $a \in \ialph$.    
%     More generally, $q \asim\dfareduces[w]\asim q'$ if, and only if, $\rev{w} \oc \dfa{q} \Reduces \dfa{q}'$, for all finite words $w \in \finwds{\ialph}$.
%   \item\label{enum:ordered-rewriting:dfa-adequacy:3}
%     $q \in F$ if, and only if, $\emp \oc \dfa{q} \Reduces \one$.
%   \end{enumerate}
\end{restatable*}

Before proving this \lcnamecref{thm:ordered-rewriting:dfa-adequacy-bisim}, we must first prove a \lcnamecref{lem:ordered-rewriting:dfa-traces}: the only traces from one state's encoding to another's are the trivial traces.
%
\begin{lemma}\label{lem:ordered-rewriting:dfa-traces}
  Let $\aut{A} = (Q, ?, F)$ be \iac{DFA} over the input alphabet $\ialph$.
  For all states $q$ and $s$, if $\dfa{q} \Reduces \dfa{s}$, then $\dfa{q} = \dfa{s}$.
\end{lemma}
%
\begin{proof}
  Assume that a trace $\dfa{q} \Reduces \dfa{s}$ exists.
  If the trace is trivial, then $\dfa{q} = \dfa{s}$ is immediate.
  Otherwise, the trace is nontrivial and consists of a strictly positive number of rewriting steps.
  By inversion, those rewriting steps drop one or more conjuncts from $\dfa{q}$ to form $\dfa{s}$.
  Every \ac{DFA} state's encoding contains exactly $\card{\ialph} + 1$ conjuncts -- one for each input symbol $a$ and one for the end-of-word marker, $\emp$.
  % Being the encoding of \iac{DFA} state, $\dfa{q}$ contains one $(\emp \limp \dotsb)$ conjunct and exactly one $(a \limp \dotsb)$ conjunct for each input symbol $a$.
  % Similarly, $\dfa{s}$ must contain the same.
  If even one conjunct is dropped from $\dfa{q}$, not enough conjuncts will remain to form $\dfa{s}$.
  Thus, a nontrivial trace $\dfa{q} \Reduces \dfa{s}$ cannot exist.
\end{proof}
%
\noindent
It is important to differentiate this \lcnamecref{lem:ordered-rewriting:dfa-traces} from the false claim that a state's encoding can take no rewriting steps.
There certainly exist nontrivial traces from $\dfa{q}$, but they do not arrive at the encoding of any state.

With this \lcnamecref{lem:ordered-rewriting:dfa-traces} now in hand, we can proceed to proving adequacy up to bisimilarity.
%
\dfaadequacybisim
%
\begin{proof}
  Each part is proved in turn.
  The proof of part~\ref{enum:ordered-rewriting:dfa-adequacy:2} % and~\ref{enum:ordered-rewriting:dfa-adequacy:4}
  depends on the proof of part~\ref{enum:ordered-rewriting:dfa-adequacy:1}.
  \begin{enumerate}[parsep=0em, listparindent=\parindent]
  %% Part one
  \item
    We shall show that bisimilarity coincides with equality of encodings, proving each direction separately.
    \begin{itemize}[parsep=0em, listparindent=\parindent]
    \item
      To prove that bisimilar \ac{DFA} states have equal encodings -- \ie, that $q \asim s$ implies $\dfa{q} = \dfa{s}$ -- a fairly straightforward proof by coinduction suffices.

      Let $q$ and $s$ be bisimilar states.
      By the definition of bisimilarity~\parencref{??}, two properties hold:
      \begin{itemize}
      \item For all input symbols $a$, the unique $a$-successors of $q$ and $s$ are also bisimilar.
      \item States $q$ and $s$ have matching finalities -- \ie, $q \in F$ if and only if $s \in F$.
      \end{itemize}
      Applying the coinductive hypothesis to the former property, we may deduce that, for all symbols $a$, the $a$-successors of $q$ and $s$ also have equal encodings.
      From the latter property, it follows that $\dfa{F}(q) = \dfa{F}(s)$.
      Because definitions are interpreted equirecursively, these equalities together imply that $q$ and $s$ themselves have equal encodings.

    \item
      To prove the converse -- that states with equal encodings are bisimilar -- we will show that the relation $\mathord{\simu{R}} = \Set{(q, s) \given \dfa{q} = \dfa{s}}$, which relates states if they have equal encodings, is a bisimulation and is therefore included in bisimilarity.
      \begin{itemize}
      \item
        The relation $\simu{R}$ is symmetric.
      \item
        We must show that $\simu{R}$-related states have $\simu{R}$-related $a$-successors, for all input symbols $a$.

        Let $q$ and $s$ be $\simu{R}$-related states.
        Being $\simu{R}$-related, $q$ and $s$ have equal encodings;
        because definitions are interpreted equirecursively, the unrollings of those encodings are also equal.
        By definition of the encoding, it follows that, for each input symbol $a$, the unique $a$-successors of $q$ and $s$ have equal encodings.
        Therefore, for each $a$, the $a$-successors of $q$ and $s$ are themselves $\simu{R}$-related.

      \item
        We must show that $\simu{R}$-related states have matching finalities.

        Let $q$ and $s$ be $\simu{R}$-related states, with $q$ a final state.
        Being $\simu{R}$-related, $q$ and $s$ have equal encodings;
        because definitions are interpreted equirecursively, the unrollings of those encodings are also equal.
        It follows that $\dfa{F}(q) = \dfa{F}(s)$, and so $s$ is also a final state.
      \end{itemize}
    \end{itemize}

  %% Part two
  \item
    We would like to prove that $q \asim\dfareduces[a]\asim q'$ if, and only if, $a \oc \dfa{q} \Reduces \dfa{q}'$, or, more generally, that $q \asim\dfareduces[w]\asim q'$ if, and only if, $\rev{w} \oc \dfa{q} \Reduces \dfa{q}'$.
    Because bisimilar states have equal encodings (part~\ref{enum:ordered-rewriting:dfa-adequacy:1}) and bisimilarity is reflexive (\cref{??}), it suffices to show two stronger statements:
    \begin{enumerate*}
    \item that $q \dfareduces[w] q'$ implies $\rev{w} \oc \dfa{q} \Reduces \dfa{q}'$; and
    \item that $\rev{w} \oc \dfa{q} \Reduces \dfa{q}'$ implies $q \dfareduces[w]\asim q'$.
    \end{enumerate*}
    %
    We prove these in turn.
    %
    \begin{enumerate}
    %% Subpart (a)
    \item
      We shall prove that $q \dfareduces[w] q'$ implies $\rev{w} \oc \dfa{q} \Reduces \dfa{q}'$ by induction over the structure of word $w$.
      \begin{itemize}
      \item
        Consider the case of the empty word, $\emp$; we must show that $q = q'$ implies $\dfa{q} \Reduces \dfa{q}'$.
        Because the encoding is a function, this is immediate.
      \item
        Consider the case of a nonempty word, $a \wc w$; we must show that $q \dfareduces[a]\dfareduces[w] q'$ implies $\rev{w} \oc a \oc \dfa{q} \Reduces \dfa{q}'$.
        Let $q'_a$ be an $a$-successor of state $q$ that is itself $w$-succeeded by state $q'$.
        There exists, by definition of the encoding, a trace
        \begin{equation*}
          \rev{w} \oc a \oc \dfa{q}
            \Reduces \rev{w} \oc a \oc (a \limp \dfa{q}'_a)
            \reduces \rev{w} \oc \dfa{q}'_a
            \Reduces \dfa{q}'
          \,,
        \end{equation*}
        with the trace's tail justified by an appeal to the inductive hypothesis.
        % Because $q'$ is a $w$-successor of $q'_a$, an appeal to the inductive hypothesis yields a trace $\rev{w} \oc \dfa{q}'_a \Reduces \dfa{q}'$.
      \end{itemize}

      % Let $q'$ be an $a$-successor of state $q$.
      % There exists, by definition of the encoding, a trace
      % \begin{equation*}
      %   a \oc \dfa{q} \Reduces a \oc (a \limp \dfa{q}') \reduces \dfa{q}'
      % \,.
      % \end{equation*}

    %% Subpart (b)
    \item
      We shall prove that $\rev{w} \oc \dfa{q} \Reduces \dfa{q}'$ implies $q \dfareduces[w]\asim q'$ by induction over the structure of word $w$.
      \begin{itemize}
      \item
        Consider the case of the empty word, $\emp$;
        we must show that $\dfa{q} \Reduces \dfa{q}'$ implies $q \asim q'$.
        By \cref{lem:ordered-rewriting:dfa-traces}, $\dfa{q} \Reduces \dfa{q}'$ implies that $q$ and $q'$ have equal encodings.
        Part~\ref{enum:ordered-rewriting:dfa-adequacy:1} can then be used to establish that $q$ and $q'$ are bisimilar.
      \item
        Consider the case of a nonempty word, $a \wc w$;
        we must show that $\rev{w} \oc a \oc \dfa{q} \Reduces \dfa{q}'$ implies $q \dfareduces[a]\dfareduces[w]\asim q'$.
        By inversion\fixnote{Is this enough justification?}, the given trace can only begin by inputting $a$:
        \begin{equation*}
          \rev{w} \oc a \oc \dfa{q}
            \Reduces \rev{w} \oc a \oc (a \limp \dfa{q}'_a)
            \reduces \rev{w} \oc \dfa{q}'_a
            \Reduces \dfa{q}'
          \,,
        \end{equation*}
        where $q'_a$ is an $a$-successor of state $q$.
        An appeal to the inductive hypothesis on the trace's tail yields $q'_a \dfareduces[w]\asim q'$, and so the \ac{DFA} admits $q \dfareduces[a]\dfareduces[w]\asim q'$, as required.
      \end{itemize}
      % Assume that a trace $a \oc \dfa{q} \Reduces \dfa{q}'$ exists.
      % By the input lemma, $\dfa{q} \Reduces (a \limp A) \oc \octx'$ for some proposition $A$ and context $\octx'$ such that $A \oc \octx' \Reduces \dfa{q}'$.
      % Upon inversion of the trace from $\dfa{q}$, we conclude that $A = \dfa{q}'_a$, where $q'_a$ is an $a$-successor of $q$, and that $\octx'$ is empty -- in other words, we have a trace $\dfa{q}'_a \Reduces \dfa{q}'$.
      % Such a trace exists only if $\dfa{q}'_a = \dfa{q}'$.
      % By part~\ref{enum:ordered-rewriting:dfa-adequacy:1} of this \lcnamecref{thm:ordered-rewriting:dfa-adequacy-bisim}, it follows that $q'_a$ and $q'$ are bisimilar.
    \end{enumerate}

  %% Part three
  \item
    We shall prove that the final states are exactly those states $q$ such that $\emp \oc \dfa{q} \Reduces \one$.
    \begin{itemize}
    \item
      Let $q$ be a final state; accordingly, $\dfa{F}(q) = \one$.
      There exists, by definition of the encoding, a trace
      \begin{equation*}
        \emp \oc \dfa{q} \Reduces \emp \oc (\emp \limp \dfa{F}(q)) \reduces \dfa{F}(q) = \one
      \,.
      \end{equation*}

    \item
      Assume that a trace $\emp \oc \dfa{q} \Reduces \one$ exists.
      By inversion\fixnote{Is this enough justification?}, this trace can only begin by inputting $\emp$:
      \begin{equation*}
        \emp \oc \dfa{q} \Reduces \emp \oc (\emp \limp \dfa{F}(q)) \reduces \dfa{F}(q) \Reduces \one
      \,.
      \end{equation*}
      The tail of this trace, $\dfa{F}(q) \Reduces \one$, can exist only if $q$ is a final state.
    %
    \qedhere
    \end{itemize}

  % %% Part four
  % \item 
  %   We would like to prove that $q \asim\dfareduces[w]\asim q'$ if, and only if, $\rev{w} \oc \dfa{q} \Reduces \dfa{q}'$.
  %   Because bisimilar states have equal encodings (part~\ref{enum:ordered-rewriting:dfa-adequacy:1}) and bisimilarity is reflexive (\cref{??}), it suffices to show:
  %   \begin{enumerate*}
  %   \item that $q \dfareduces[w] q'$ implies $\rev{w} \oc \dfa{q} \Reduces \dfa{q}'$; and
  %   \item that $\rev{w} \oc \dfa{q} \Reduces \dfa{q}'$ implies $q \dfareduces[w]\asim q'$.
  %   \end{enumerate*}

  %   Both statements can be established by induction over the structure of word $w$.
  %   The latter proof is slightly more involved and deserves a bit of explanation.
  %   \begin{itemize}
  %   \item Consider the case in which $w$ is the empty word; we must show that $\dfa{q} \Reduces \dfa{q}'$ implies $q \asim q'$.
  %     By \cref{lem:ordered-rewriting:dfa-traces}, $\dfa{q} \Reduces \dfa{q}'$ implies that $\dfa{q} = \dfa{q}'$.
  %     Part~\ref{enum:ordered-rewriting:dfa-adequacy:1} can then be used to establish $q$ and $q'$ as bisimilar.

  %   \item Consider the case of a nonempty word, $a \wc w$.
  %     We must show that $\rev{w} \oc a \oc \dfa{q} \Reduces \dfa{q}'$ implies $q \dfareduces[a]\dfareduces[w]\asim q'$.
  %     By inversion, the given trace must begin by inputting $a$:
  %     \begin{equation*}
  %       \rev{w} \oc a \oc \dfa{q} \Reduces \rev{w} \oc a \oc (a \limp \dfa{q}'_a) \reduces \rev{w} \oc \dfa{q}'_a \Reduces \dfa{q}'
  %       \,,
  %     \end{equation*}
  %     where $q'_a$ is an $a$-successor of state $q$.
  %     Appealing to the inductive hypothesis on the trace's tail yields $q'_a \dfareduces[w]\asim q'$, and so $q \dfareduces[a]\dfareduces[w]\asim q'$, as required.
  %   %
  %   \qedhere
  %   \end{itemize}
  \end{enumerate}
\end{proof}


\subsection{Encoding \aclp*{NFA}?}

We would certainly be remiss if we did not attempt to generalize the rewriting specification of \acp{DFA} to one for their nondeterministic cousins.

Differently from \ac{DFA} states, \iac{NFA} state $q$ may have several nondeterministic successors for each input symbol $a$.
To encode the \ac{NFA} state $q$, all of its $a$-successors are collected in an alternative conjunction underneath the left-handed input of $a$.
Thus, the encoding of \iac{NFA} state $q$ becomes
\begin{equation*}
  \nfa{q} \defd
    \parens[size=auto]{\displaystyle
      \bigwith_{a \in \ialph}
        \parens[size=big]{a \limp \parens{\bigwith_{q'_a} \nfa{q}'_a}}
    }
    \with
    \parens[size=big]{\emp \limp \nfa{F}(q)}
  \,,
\end{equation*}
where $\nfa{F}(q)$ is defined as for \acp{DFA}.

The adjacent \lcnamecref{fig:ordered-rewriting:nfa-example}
\begin{marginfigure}
  \centering
  % \subfloat[][]{\label{fig:ordered-rewriting:nfa-example:nfa}%
    \begin{tikzpicture}
      \graph [automaton] {
        q_0
         -> ["a,b", loop above]
        q_0
         -> ["b"]
        q_1 [accepting]
         -> ["a,b"]
        q_2
         -> ["a,b", loop above]
        q_2;
      };
    \end{tikzpicture}
  % }

%   \subfloat[][]{\label{fig:ordered-rewriting:nfa-example:encoding}%
      $\!\begin{aligned}
        \nfa{q}_0 &\defd (a \limp \nfa{q}_0) \with \bigl(b \limp (\nfa{q}_0 \with \nfa{q}_1)\bigr) \with (\emp \limp \top) \\
        \nfa{q}_1 &\defd (a \limp \nfa{q}_2) \with (b \limp \nfa{q}_2) \with (\emp \limp \one) \\
        \nfa{q}_2 &\defd (a \limp \nfa{q}_2) \with (b \limp \nfa{q}_2) \with (\emp \limp \top)
      \end{aligned}$
%     }

  \caption{{fig:ordered-rewriting:nfa-example:nfa}~\Iac*{NFA} that accepts exactly those words, over the alphabet $\ialph = \set{a,b}$, that end with $b$; and {fig:ordered-rewriting:nfa-example:encoding}~its encoding}\label{fig:ordered-rewriting:nfa-example}
\end{marginfigure}%
recalls from \cref{ch:automata} \iac{NFA} that accepts exactly those words, over the alphabet $\ialph = \set{a,b}$, that end with $b$.
Using the above encoding of \acp{NFA}, ordered rewriting does indeed simulate this \ac{NFA}.
For example, just as there are transitions $q_0 \nfareduces[b] q_0$ and $q_0 \nfareduces[b] q_1$, there are traces
\begin{equation*}
  \begin{tikzcd}[
    cells={inner xsep=0.65ex,
           inner ysep=0.4ex},
         % nodes={draw},
    row sep=0em,
    column sep=scriptsize
  ]
    &[-0.2em] \nfa{q}_0
    \\
    b \oc \nfa{q}_0 \Reduces b \oc \bigl(b \limp (\nfa{q}_0 \with \nfa{q}_1)\bigr) \reduces \nfa{q}_0 \with \nfa{q}_1
      \urar[reduces, start anchor=east]
      \drar[reduces, start anchor=base east]
    \\
    & \nfa{q}_1
  \end{tikzcd}
\end{equation*}

Unfortunately, while it does simulate \ac{NFA} behavior, this encoding is not adequate.
Like \ac{DFA} states, \ac{NFA} states that have equal encodings are bisimilar.
% \begin{proof}
%   Define a relation $\mathord{\simu{R}} = \set{(q, s) \given \nfa{q} = \nfa{s}}$; we will show that $\simu{R}$ is a bisimulation.
%   \begin{itemize}
%   \item Assume that $s \simu{R}^{-1} q \nfareduces[a] q'_a$.
%     By definition, $a \oc \nfa{q} \Reduces \nfa{q}'_a$.
%     Because $\nfa{q} = \nfa{s}$, it follows that $s \nfareduces[a] s'_a$ for some state $s'_a$ such that $\nfa{q}'_a = \nfa{s}'_a$ -- that is, $q'_a \simu{R} s'_a$.
%     Thus, $s \nfareduces[a]\simu{R}^{-1} q'_a$.
%   \item Assume that $q \simu{R} s$.
%     It follows that $\nfa{F}(q) = \nfa{F}(s)$.
%     Thus, $q$ is an accepting state if and only if $s$ is.
%   \end{itemize}
% \end{proof}
However, for \acp{NFA}, the converse does not hold: bisimilar states do not necessarily have equal encodings.
%
\begin{falseclaim}
  Let $\aut{A} = (Q, ?, F)$ be \iac{NFA} over input alphabet $\ialph$.
  Then $q \asim s$ implies $\nfa{q} = \nfa{s}$, for all states $q$ and $s$.
\end{falseclaim}
%
\begin{proof}[Counterexample]
  Consider the \ac{NFA} and encoding depicted in the adjacent \lcnamecref{fig:ordered-rewriting:nfa-counterexample}.
  \begin{marginfigure}
    \begin{alignat*}{2}
      \begin{tikzpicture}
        \graph [automaton] {
          q_0 [accepting]
           -> ["a", loop above]
          q_0
           -> ["a", overlay]
          q_1 [accepting, overlay]
           -> ["a", loop above, overlay]
          q_1;
        };
      \end{tikzpicture}
      &\quad&&
      \\
      &\quad& \nfa{q}_0 &\defd \bigl(a \limp (\nfa{q}_0 \with \nfa{q}_1)\bigr) \with (\emp \limp \one) \\
      &\quad& \nfa{q}_1 &\defd (a \limp \nfa{q}_1) \with (\emp \limp \one)
    \end{alignat*}
    \caption{\Iac*{NFA} that accepts all finite words over the alphabet $\ialph = \set{a}$}\label{fig:ordered-rewriting:nfa-counterexample}
  \end{marginfigure}
  It is easy to verify that the relation $\set{q_1} \times \set{q_0,q_1}$ is a bisimulation; in particular, $q_1$ simulates the $q_0 \nfareduces[a] q_1$ transition by its self-loop, $q_1 \nfareduces[a] q_1$.
  Hence, $q_0$ and $s_0$ are bisimilar.
  %
  % These same \acp{NFA} are encoded by the following definitions.
  % \begin{align*}
  %   \nfa{q}_0 &\defd (a \limp \nfa{q}_0) \with (\emp \limp \one)
  % \shortintertext{and}
  %   \nfa{s}_0 &\defd \bigl(a \limp (\nfa{s}_0 \with \nfa{s}_1)\bigr) \with (\emp \limp \one) \\
  %   \nfa{s}_1 &\defd (a \limp \nfa{s}_1) \with (\emp \limp \one)
  % \end{align*}
  It is equally easy to verify, by unrolling the definitions used in the encoding, that $\nfa{q}_0 \neq \nfa{s}_0$.
\end{proof}

For \acp{DFA}, bisimilar states do have equal encodings because the inherent determinism \ac{DFA} bisimilarity is a rather fine-grained equivalence.
Because each \ac{DFA} state has exactly one successor for each input symbol
The additional flexibility entailed by nondeterminism

Once again, it would be possible to construct an adequate encoding, by tagging each state with a unique atom.
% with a stronger nominal character

For the moment, we will put aside the question of an adequate encoding of \acp{NFA}.



\section{Introduction}

In the previous \lcnamecref{ch:ordered-logic}, we saw that the ordered sequent calculus can be given a resource interpretation in which sequents $\oseq{\octx |- A}$ may be read as \enquote{From resources $\octx$, resource goal $A$ is achievable.}
For instance, the left rule for ordered conjunction ($\lrule{\fuse}$, see adjacent display)%
\marginnote{%
  $\infer[\lrule{\fuse}]{\oseq{\octx'_L \oc (A \fuse B) \oc \octx'_R |- C}}{
     \oseq{\octx'_L \oc A \oc B \oc \octx'_R |- C}}$%
}
was read \enquote{Goal $C$ is achievable from resource $A \fuse B$ if it is achievable from the separate resources $A \oc B$.}

As alluded in the previous \lcnamecref{ch:ordered-logic}'s discussion of ordered conjunction\footnote{See \cpageref{p:ordered-logic:ordered-conjunction}.}, this $\lrule{\fuse}$ rule is essentially a rule of resource decomposition: it decomposes [the resource] $A \fuse B$ into the separate resources $A \oc B$ and relegates the unchanged goal $C$ to a secondary role.

\newthought{%
This \lcnamecref{ch:ordered-rewriting}%
}
begins by exploring a refactoring of the ordered sequent calculus's left rules around this idea of resource decomposition~\parencref{sec:ordered-rewriting:??}.
Most of the left rules can be easily refactored in this way, although a few will prove resistant to the change.

Emphasizing resource decomposition naturally leads us to a rewriting interpretation of (a fragment of) ordered logic~\parencref{sec:ordered-rewriting:??}.
This rewriting system is closely related to traditional notions of string rewriting\autocite{??}, but simultaneously restricts and generalizes [...] along distinct axes.

The connection of ordered logic and the Lambek calculus to rewriting is certainly not new.
\Citeauthor{Lambek:AMM58}'s original article\autocite{Lambek:AMM58}

This development borrows from \citeauthor{Cervesato+Scedrov:IC09}'s work on intuitionistic linear logic as an expressive rewriting framework that generalizes traditional notions of multiset rewriting.\autocite{Cervesato+Scedrov:IC09}



\newthought{Most} of the left rules could be seen as decomposing resources.
The left rules were seen as decomposing resources, such as the $\lrule{\fuse}$~rule%
\marginnote{%
  $\infer[\lrule{\fuse}]{\oseq{\octx'_L \oc (A \fuse B) \oc \octx'_R |- C}}{
     \oseq{\octx'_L \oc A \oc B \oc \octx'_R |- C}}$%
}
decomposing $A \fuse B$ into the resources $A \oc B$.
The right rules, on the other hand, were seen as ...

Replacing the left rules with a single, common rule ... and a new judgment, $\octx \reduces \octx'$, that exposes [makes [more] explicit] the decomposition of resources/state transformation aspect.


\section{Most left rules decompose ordered resources}

Recall two of the ordered sequent calculus's left rules: $\lrule{\fuse}$ and $\lrule{\with}_1$.
\begin{inferences}
  \infer[\lrule{\fuse}]{\oseq{\octx'_L \oc (A \fuse B) \oc \octx'_R |- C}}{
    \oseq{\octx'_L \oc A \oc B \oc \octx'_R |- C}}
  \and
  \infer[\lrule{\with}_1]{\oseq{\octx'_L \oc (A \with B) \oc \octx'_R |- C}}{
    \oseq{\octx'_L \oc A \oc \octx'_R |- C}}
\end{inferences}
Both rules decompose the principal resource: in the $\lrule{\fuse}$ rule, $A \fuse B$ into the separate resources $A \oc B$; and, in the $\lrule{\with}_1$ rule, $A \with B$ into $A$.
However, in both cases, the resource decomposition is somewhat obscured by boilerplate.
The framed contexts $\octx'_L$ and $\octx'_R$ and goal $C$ serve to enable the rules to be applied anywhere [in the string of resources], without restriction;
these concerns are not specific to the $\lrule{\fuse}$ and $\lrule{\with}_1$ rules, but are general boilerplate that arguably should be factored out.

To decouple the resource decomposition from the surrounding boilerplate, we will introduce a new judgment, $\octx \reduces \octx'$, meaning \enquote{Resources $\octx$ may be decomposed into [resources] $\octx'$.}
% With this judgment in hand, the boilerplate can be factored into a uniform left rule, $\lrule{\star}$:
With this new judgment comes a cut principle, $\jrule{CUT}^{\reduces}$, into which all of the boilerplate is factored:
\begin{equation*}
  \infer[\jrule{CUT}\smash{^{\reduces}}]{\oseq{\octx'_L \oc \octx \oc \octx'_R |- C}}{
    \octx \reduces \octx' &
    \oseq{\octx'_L \oc \octx' \oc \octx'_R |- C}}
  .
\end{equation*}

The standard left rules can be recovered from resource decomposition rules using this cut principle.
For example, the decomposition of $A \fuse B$ into $A \oc B$ is captured by
\begin{equation*}
  \infer[\jrule{$\fuse$D}]{A \fuse B \reduces A \oc B}{}
  ,
\end{equation*}
and the standard $\lrule{\fuse}$ rule can then be recovered as shown in the neighboring \lcnamecref{fig:ordered-rewriting:fuse-refactoring}.%
\begin{marginfigure}[-8\baselineskip]
  \begin{gather*}
    \infer[\lrule{\fuse}]{\oseq{\octx'_L \oc (A \fuse B) \oc \octx'_R |- C}}{
      \oseq{\octx'_L \oc A \oc B \oc \octx'_R |- C}}
    %
    \\\leftrightsquigarrow\\
    %
    \infer[\jrule{CUT}\smash{^{\reduces}}]{\oseq{\octx'_L \oc (A \fuse B) \oc \octx'_R |- C}}{
      \infer[\jrule{$\fuse$D}]{A \fuse B \reduces A \oc B}{} &
      \oseq{\octx'_L \oc A \oc B \oc \octx'_R |- C}}
    .
  \end{gather*}
  \caption{A refactoring of the $\lrule{\fuse}$ rule as resource decomposition}\label{fig:ordered-rewriting:fuse-refactoring}
\end{marginfigure}
The left rules for $\one$ and $A \with B$ can be similarly refactored into resource decomposition rules.

Even the left rules for left- and right-handed implications can be refactored in this way, despite the additional, minor premises that those rules carry.
To keep the correspondence between resource decomposition rules and left rules close, we could introduce the decomposition rules
\begin{inferences}
  \infer[\jrule{$\limp$D}']{\octx \oc (A \limp B) \reduces B}{
    \oseq{\octx |- A}}
  \and\text{and}\and
  \infer[\jrule{$\pmir$D}']{(B \pmir A) \oc \octx \reduces B}{
    \oseq{\octx |- A}}
  .
\end{inferences}
Just as for ordered conjunction, the left rules for left- and right-handed implication are then recovered by combining a decomposition rule with the $\jrule{CUT}^{\reduces}$ rule~(see adjacent \lcnamecref{fig:ordered-rewriting:limp-refactoring-1}).%
\begin{marginfigure}[-8\baselineskip]
  \begin{gather*}
    \infer[\lrule{\limp}]{\oseq{\octx'_L \oc \octx \oc (A \limp B) \oc \octx'_R |- C}}{
      \oseq{\octx |- A} &
      \oseq{\octx'_L \oc B \oc \octx'_R |- C}}
    %
    \\\leftrightsquigarrow\\
    %
    \infer[\jrule{CUT}\smash{^{\reduces}}]{\oseq{\octx'_L \oc \octx \oc (A \limp B) \oc \octx'_R |- C}}{
      \infer[\jrule{$\limp$D}']{\octx \oc (A \limp B) \reduces B}{
        \oseq{\octx |- A}} &
      \oseq{\octx'_L \oc B \oc \octx'_R |- C}}
  \end{gather*}
  \caption{A refactoring of the $\lrule{\limp}$ rule using a resource decomposition rule}\label{fig:ordered-rewriting:limp-refactoring-1}
\end{marginfigure}

Although these $\jrule{$\limp$D}'$ and $\jrule{$\pmir$D}'$ rules keep the correspondence between resource decomposition rules and left rules close, they differ from the other decomposition rules in two significant ways.
First, the above $\jrule{$\limp$D}'$ and $\jrule{$\pmir$D}'$ rules have premises, and those premises create a dependence of the decomposition judgment upon general provability.
Second, the above $\jrule{$\limp$D}'$ and $\jrule{$\pmir$D}'$ rules do not decompose the principal proposition into immediate subformulas.
This contrasts with, for example, the $\jrule{$\fuse$D}$ rule that decomposes $A \fuse B$ into the immediate subformulas $A \oc B$.

For these reasons, the above $\jrule{$\limp$D}'$ and $\jrule{$\pmir$D}'$ rules are somewhat undesirable.
Fortunately, there is an alternative.
Filling in the $\oseq{\octx |- A}$ premises with the $\jrule{ID}^A$ rule, we arrive at the derivable rules
\begin{inferences}
  \infer[\jrule{$\limp$D}]{A \oc (A \limp B) \reduces B}{}
  \and\text{and}\and
  \infer[\jrule{$\pmir$D}]{(B \pmir A) \oc A \reduces B}{}
  .
\end{inferences}
The standard $\lrule{\limp}$ and $\lrule{\pmir}$ rules can still be recovered from these more specific decomposition rules, thanks to $\jrule{CUT}$ (see adjacent \lcnamecref{fig:ordered-rewriting:limp-refactoring-2}).%
\begin{marginfigure}[-10\baselineskip]
  \begin{gather*}
    \infer[\lrule{\limp}]{\oseq{\octx'_L \oc \octx \oc (A \limp B) \oc \octx'_R |- C}}{
      \oseq{\octx |- A} &
      \oseq{\octx'_L \oc B \oc \octx'_R |- C}}
    %
    \\\leftrightsquigarrow\\
    %
    \infer[\jrule{CUT}\smash{^A}]{\oseq{\octx'_L \oc \octx \oc (A \limp B) \oc \octx'_R |- C}}{
      \oseq{\octx |- A} &
      \infer[\jrule{CUT}\smash{^{\reduces}}]{\oseq{\octx'_L \oc A \oc (A \limp B) \oc \octx'_R |- C}}{
        \infer[\jrule{$\limp$D}]{A \oc (A \limp B) \reduces B}{} &
        \oseq{\octx'_L \oc B \oc \octx'_R |- C}}}
  \end{gather*}
  \caption{A refactoring of the $\lrule{\limp}$ rule using an alternative resource decomposition rule}\label{fig:ordered-rewriting:limp-refactoring-2}
\end{marginfigure}
These revised, nullary decomposition rules correct the earlier drawbacks: like the other decomposition rules, they now have no premises and only refer to immediate subformulas.
Moreover, these rules have the advantage of matching two of the axioms from \citeauthor{Lambek:AMM58}'s original article.\autocite{Lambek:AMM58}


% For many of the ordered logical connectives, this approach  works perfectly.
% The decomposition of $A \fuse B$ into $A \oc B$ is, for example, captured by
% \begin{equation*}
%   \infer[\lrule{\fuse}']{A \fuse B \reduces A \oc B}{}
%   ,
% \end{equation*}
% so that the ordered sequent calculus's standard $\lrule{\fuse}$ rule
% % left rule for multiplicative conjunction
% is then derivable from the uniform left rule:
% \begin{equation*}
%   \infer[\lrule{\fuse}]{\oseq{\octx'_L \oc (A \fuse B) \oc \octx'_R |- C}}{
%     \oseq{\octx'_L \oc A \oc B \oc \octx'_R |- C}}
%   %
%   \enspace\leftrightsquigarrow\enspace
%   %
%   \infer[\lrule{\star}]{\oseq{\octx'_L \oc (A \fuse B) \oc \octx'_R |- C}}{
%     \infer[\lrule{\fuse}']{A \fuse B \reduces A \oc B}{} &
%     \oseq{\octx'_L \oc A \oc B \oc \octx'_R |- C}}
%   .
% \end{equation*}
% The left rules for $\one$ and $A \with B$ can be refactored in a similar way.
% Despite their additional, minor premises, even the left rules for left- and right-handed implications can be refactored in this way.
% \begin{inferences}
%   \infer[\lrule{\limp}']{\octx \oc (A \limp B) \reduces B}{
%     \oseq{\octx |- A}}
%   \and
%   \infer[\lrule{\pmir}']{(B \pmir A) \oc \octx \reduces B}{
%     \oseq{\octx |- A}}
% \end{inferences}

% \begin{equation*}
%   \infer[\lrule{\limp}]{\oseq{\octx'_L \oc \octx \oc (A \limp B) \oc \octx'_R |- C}}{
%     \oseq{\octx |- A} &
%     \oseq{\octx'_L \oc B \oc \octx'_R |- C}}
%   %
%   \enspace\leftrightsquigarrow\enspace
%   %
%   \infer[\lrule{\star}]{\oseq{\octx'_L \oc \octx \oc (A \limp B) \oc \octx'_R |- C}}{
%     \infer[\lrule{\limp}']{\octx \oc (A \limp B) \reduces B}{
%       \oseq{\octx |- A}} &
%     \oseq{\octx'_L \oc B \oc \octx'_R |- C}}
% \end{equation*}


\newthought{%
So, for most%
}
ordered logical connectives, this approach works perfectly.
Unfortunately, the left rules for additive disjunction, $A \plus B$, and its unit, $\zero$, are resistant to this kind of refactoring.
The difficulty with additive disjunction isn't that its left rule, $\lrule{\plus}$,%
\marginnote{%
  \begin{equation*}
    \infer[\lrule{\plus}]{\oseq{\octx'_L \oc (A \plus B) \oc \octx'_R |-  C}}{
      \oseq{\octx'_L \oc A \oc \octx'_R |-  C} &
      \oseq{\octx'_L \oc B \oc \octx'_R |-  C}}
  \end{equation*} 
}
doesn't decompose the resource $A \plus B$.
The $\lrule{\plus}$ rule certainly does decompose $A \plus B$, but it does so [...].
$A \plus B \reduces A \mid B$
[...] retain the standard $\lrule{\plus}$ and $\lrule{\zero}$ rules.

\begin{figure}[tbp]
  \begin{inferences}
    \infer[\jrule{CUT}\smash{^A}]{\oseq{\octx'_L \oc \octx \oc \octx'_R |- C}}{
      \oseq{\octx |- A} & \oseq{\octx'_L \oc A \oc \octx'_R |- C}}
    \and 
    \infer[\jrule{ID}\smash{^A}]{\oseq{A |- A}}{}
    \\
    \infer[\jrule{CUT}\smash{^{\reduces}}]{\oseq{\octx'_L \oc \octx \oc \octx'_R |- C}}{
      \octx \reduces \octx' & \oseq{\octx'_L \oc \octx' \oc \octx'_R |- C}}
    \\
    \infer[\rrule{\fuse}]{\oseq{\octx_1 \oc \octx_2 |- A \fuse B}}{
      \oseq{\octx_1 |- A} & \oseq{\octx_2 |- B}}
    \and
    \infer[\jrule{$\fuse$D}]{A \fuse B \reduces A \oc B}{}
    \\
    \infer[\rrule{\one}]{\oseq{\octxe |- \one}}{}
    \and
    \infer[\jrule{$\one$D}]{\one \reduces \octxe}{}
    \\
    \infer[\rrule{\with}]{\oseq{\octx |- A \with B}}{
      \oseq{\octx |- A} & \oseq{\octx |- B}}
    \and
    \infer[\jrule{$\with$D}_1]{A \with B \reduces A}{}
    \and
    \infer[\jrule{$\with$D}_2]{A \with B \reduces B}{}
    \\
    \infer[\rrule{\top}]{\oseq{\octx |- \top}}{}
    \and
    \text{(no $\jrule{$\top$D}$ rule)}
    \\
    \infer[\rrule{\limp}]{\oseq{\octx |- A \limp B}}{
      \oseq{A \oc \octx |- B}}
    \and
    \infer[\jrule{$\limp$D}]{A \oc (A \limp B) \reduces B}{}
    \\
    \infer[\rrule{\pmir}]{\oseq{\octx |- B \pmir A}}{
      \oseq{\octx \oc A |- B}}
    \and
    \infer[\jrule{$\pmir$D}]{(B \pmir A) \oc A \reduces B}{}
    \\
    \infer[\rrule{\plus}_1]{\oseq{\octx |- A \plus B}}{
      \oseq{\octx |- A}}
    \and
    \infer[\rrule{\plus}_2]{\oseq{\octx |- A \plus B}}{
      \oseq{\octx |- B}}
    \and
    \infer[\lrule{\plus}]{\oseq{\octx'_L \oc (A \plus B) \oc \octx'_R |- C}}{
      \oseq{\octx'_L \oc A \oc \octx'_R |- C} &
      \oseq{\octx'_L \oc B \oc \octx'_R |- C}}
    \\
    \text{(no $\rrule{\zero}$ rule)}
    \and
    \infer[\lrule{\zero}]{\oseq{\octx'_L \oc \zero \oc \octx'_R |- C}}{}
  \end{inferences}
  \caption{A refactoring of the ordered sequent calculus to emphasize that most left rules amount to resource decomposition}\label{fig:ordered-rewriting:decompose-seq-calc}
\end{figure}

\newthought{%
\Cref{fig:ordered-rewriting:decompose-seq-calc} presents%
}
the fully refactored sequent calculus for ordered logic.
This refactored calculus is sound and complete with respect to the ordered sequent calculus~\parencref{fig:ordered-logic:sequent-calculus}.
%
\begin{theorem}[Soundness]
  If\/ $\oseq{\octx |- A}$ is derivable in the refactored calculus of \cref{fig:ordered-rewriting:decompose-seq-calc}, then $\oseq{\octx |- A}$ is derivable in the ordered sequent calculus~\parencref{fig:ordered-logic:sequent-calculus}.
\end{theorem}
%
\begin{proof}
  By structural induction on the given derivation.
  The key lemma is the admissibility of $\jrule{CUT}^{\reduces}$ in the ordered sequent calculus:
  \begin{quotation}
    \normalsize If $\octx \reduces \octx'$ and $\oseq{\octx'_L \oc \octx' \oc \octx'_R |- C}$, then $\oseq{\octx'_L \oc \octx \oc \octx'_R |- C}$.
  \end{quotation}
  This lemma can be proved by case analysis of the decomposition $\octx \reduces \octx'$, reconstituting the corresponding left rule along the lines of the sketches from \cref{fig:ordered-rewriting:fuse-refactoring,fig:ordered-rewriting:limp-refactoring-2}.
\end{proof}
%
\begin{theorem}[Completeness]
  If\/ $\oseq{\octx |- A}$ is derivable in the ordered sequent calculus~\parencref{fig:ordered-logic:sequent-calculus}, then $\oseq{\octx |- A}$ is derivable in the refactored calculus of \cref{fig:ordered-rewriting:decompose-seq-calc}.
\end{theorem}
%
\begin{proof}
  By structural induction on the given derivation.
  The critical cases are the left rules; they are resolved along the lines of the sketches shown in \cref{fig:ordered-rewriting:fuse-refactoring,fig:ordered-rewriting:limp-refactoring-2}.
\end{proof}




\section{Decomposition as rewriting}

Thus far, we have used the decomposition judgment, $\octx \reduces \octx'$, and its rules as the basis for a reconfigured sequent-like calculus for ordered logic.
% But this refactoring also leads naturally to a rewriting system grounded in ordered logic.
% 
Instead,
% of taking the resource decomposition rules as a basis for a reconfigured sequent calculus,
we can also view decomposition as the foundation of a rewriting system grounded in ordered logic.
For example, the decomposition of resource $A \fuse B$ into $A \oc B$ by the $\jrule{$\fuse$D}$ rule
% \marginnote{%
%   \begin{equation*}
%     \infer[\jrule{$\fuse$D}]{A \fuse B \reduces A \oc B}{}
%   \end{equation*}
% }%
can also be seen as \emph{rewriting} $A \fuse B$ into $A \oc B$.
More generally, the decomposition judgment $\octx \reduces \octx'$ can be read as \enquote{$\octx$ rewrites to $\octx'$.}

\Cref{fig:ordered-rewriting:rewriting} summarizes the rewriting system that we obtain from the refactored sequent-like calculus of \cref{fig:ordered-rewriting:decompose-seq-calc}.
%
\begin{figure}[tbp]
  \vspace{\dimexpr-\abovedisplayskip-\abovecaptionskip\relax}
  \begin{inferences}
    \infer[\jrule{$\fuse$D}]{A \fuse B \reduces A \oc B}{}
    \and
    \infer[\jrule{$\one$D}]{\one \reduces \octxe}{}
    \\
    \infer[\jrule{$\with$D}_1]{A \with B \reduces A}{}
    \and
    \infer[\jrule{$\with$D}_2]{A \with B \reduces B}{}
    \and
    \text{(no $\jrule{$\top$D}$ rule)}
    \\
    \infer[\jrule{$\limp$D}]{A \oc (A \limp B) \reduces B}{}
    \and
    \infer[\jrule{$\pmir$D}]{(B \pmir A) \oc A \reduces B}{}
    \\
    \text{(no $\jrule{$\plus$D}$ and $\jrule{$\zero$D}$ rules)}
    \\
    \infer[\jrule{$\reduces$C}\smash{_{\jrule{L}}}]{\octx_1 \oc \octx_2 \reduces \octx'_1 \oc \octx_2}{
      \octx_1 \reduces \octx'_1}
    \and
    \infer[\jrule{$\reduces$C}\smash{_{\jrule{R}}}]{\octx_1 \oc \octx_2 \reduces \octx_1 \oc \octx'_2}{
      \octx_2 \reduces \octx'_2}
  \end{inferences}
  \begin{inferences}
    \infer[\jrule{$\Reduces$R}]{\octx \Reduces \octx}{}
    \and
    \infer[\jrule{$\Reduces$T}]{\octx \Reduces \octx''}{
      \octx \reduces \octx' & \octx' \Reduces \octx''}
  \end{inferences}
  \caption{A rewriting fragment of ordered logic, based on resource decomposition}\label{fig:ordered-rewriting:rewriting}
\end{figure}
%
Essentially, the ordered rewriting system is obtained by discarding all rules except for the decomposition rules.
However, if only the decomposition rules are used, rewritings cannot occur within a larger context.
For example, the $\jrule{$\limp$D}$ rule derives $A \oc (A \limp B) \reduces B$, but $\octx'_L \oc A \oc (A \limp B) \oc \octx'_R \reduces \octx'_L \oc B \oc \octx'_R$ would not be derivable in general.
In the refactored calculus of \cref{fig:ordered-rewriting:decompose-seq-calc}, this kind of framing is taken care of by the cut principle for decomposition, $\jrule{CUT}^{\reduces}$.
To express framing at the level of the $\octx \reduces \octx'$ judgment, we introduce two compatibility rules: together,
\begin{inferences}
  \infer[\jrule{$\reduces$C}\smash{_{\jrule{L}}}]{\octx_1 \oc \octx_2 \reduces \octx'_1 \oc \octx_2}{
    \octx_1 \reduces \octx'_1}
  \and\text{and}\and
  \infer[\jrule{$\reduces$C}\smash{_{\jrule{R}}}]{\octx_1 \oc \octx_2 \reduces \octx_1 \oc \octx'_2}{
    \octx_2 \reduces \octx'_2}
\end{inferences}
ensure that rewriting is compatible with concatenation of ordered contexts.%
\footnote[][-4\baselineskip]{%
  Because ordered contexts form a monoid, these compatibility rules are equivalent to the unified rule
  \begin{equation*}
    \infer[\jrule{$\reduces$C}]{\octx_L \oc \octx \oc \octx_R \reduces \octx_L \oc \octx' \oc \octx_R}{
      \octx \reduces \octx'}
    .
  \end{equation*}
  However, we prefer the two-rule formulation of compatibility because it better aligns with the syntactic structure of contexts.%
}

By forming the reflexive, transitive closure of $\reduces$, we may construct a multi-step rewriting relation, which we choose to write as $\Reduces$.%
\footnote[][0.5\baselineskip]{%
  Usually written as $\reduces^*$, we instead chose $\Reduces$ for the reflexive, transitive closure because of its similarity with process calculus notation for weak transitions, $\Reduces[\smash{\alpha}]$.
  Our reasons will become clearer in subsequent \lcnamecrefs{ch:ordered-bisimilarity}.%
}

Consistent with its [free] monoidal structure, there are two equivalent formulations of this reflexive, transitive closure: each rewriting sequence $\octx \Reduces \octx'$ can be viewed as either a list or tree of individual rewriting steps.
We prefer the list-based formulation shown in \cref{fig:ordered-rewriting:rewriting} because it tends to [...] proofs by structural induction, but, on the basis of the following \lcnamecref{fact:ordered-rewriting:transitivity}, we allow ourselves to freely switch between the two formulations as needed.
%
\begin{fact}[Transitivity of $\Reduces$]
  If \kern0.15em$\octx \Reduces \octx'$ and\/ $\octx' \Reduces \octx''$, then\/ $\octx \Reduces \octx''$.
\end{fact}
%
\begin{proof}
  By induction on the structure of the first trace, $\octx \Reduces \octx'$.
\end{proof}

\newthought{A few remarks} about these rewriting relations are in order.
%
First, interpreting the resource decomposition rules as rewriting only confirms our preference for the nullary $\jrule{$\limp$D}$ and $\jrule{$\pmir$D}$ rules.
% [over the $\jrule{$\limp$D}'$ and $\jrule{$\pmir$D}'$ rules.]
The $\jrule{$\limp$D}'$ and $\jrule{$\pmir$D}'$ rules, with their $\oseq{\octx |- A}$ premises, would be problematic as rewriting rules because they would introduce a dependence of ordered rewriting upon general provability%
% By instead using the $\jrule{$\limp$D}$ and $\jrule{$\pmir$D}$ rules, we ensures that ordered rewriting is a syntactic procedure that
% Instead, we want ordered rewriting to be a syntactic procedure, withou 
, and the concomitant[/attendant] proof search would take ordered rewriting too far afield from traditional, syntactic\fixnote{Is this the right word?} notions of string and multiset rewriting.
[mechanical, computational]

Second, multi-step rewriting is incomplete with respect to the ordered sequent calculus~\parencref{fig:ordered-logic:sequent-calculus} because all right rules have been discarded.
%
 \begin{falseclaim}[Completeness]
  If \kern0.15em$\oseq{\octx |- A}$, then\/ $\octx \Reduces A$.
\end{falseclaim}
%
\begin{proof}[Counterexample]
  The sequent $\oseq{A \limp (C \pmir B) |- (A \limp C) \pmir B}$ is provable, but $A \limp (C \pmir B) \Longarrownot\Reduces (A \limp C) \pmir B$ even though $A \oc (A \limp (C \pmir B)) \oc B \Reduces C$ does hold.
\end{proof}


As expected from the way in which it was developed, ordered rewriting is, however, sound.
Before stating and proving soundness, we must define an operation $\bigfuse \octx$ that reifies an ordered context as a single proposition (see adjacent \lcnamecref{fig:ordered-rewriting:bigfuse}).
%
\begin{marginfigure}
  \begin{align*}
    (\octx_1 \oc \octx_2) &= (\octx_1) \fuse (\octx_2) \\
    \mathord{\text{$\fuse$}} (\octxe) &= \one \\
    A &= A
  \end{align*}
  \begin{align*}
    \bigfuse (\octx_1 \oc \octx_2) &= (\bigfuse \octx_1) \fuse (\bigfuse \octx_2) \\
    \bigfuse (\octxe) &= \one \\
    \bigfuse A &= A
  \end{align*}
\begin{theorem}
  If \kern0.15em$\octx \reduces \octx'$, then\/ $\oseq{\octx |- \bigfuse \octx'}$.
  Also, if \kern0.15em$\octx \Reduces \octx'$, then\/ $\oseq{\octx |- \bigfuse \octx'}$.
\end{theorem}
  \caption{From ordered contexts to propositions}\label{fig:ordered-rewriting:bigfuse}
\end{marginfigure}
%
\begin{theorem}[Soundness]
  If \kern0.15em$\octx \reduces \octx'$, then\/ $\oseq{\octx |- \bigfuse \octx'}$.
  Also, if \kern0.15em$\octx \Reduces \octx'$, then\/ $\oseq{\octx |- \bigfuse \octx'}$.
\end{theorem}
%
\begin{proof}
  By induction on the structure of the given step or trace.
\end{proof}

Last, notice that every rewriting step, $\octx \reduces \octx'$, strictly decreases the number of logical connectives that occur in the ordered context.
More formally, let $\card{\octx}$ be a measure of the number of logical connectives that occur in $\octx$, as defined in the adjacent \lcnamecref{fig:ordered-rewriting:measure}.
%
\begin{marginfigure}
  \begin{align*}
    \card{\octx_1 \oc \octx_2} &= \card{\octx_1} + \card{\octx_2} \\
    \card{\octxe} &= 0 \\
    \card{A \star B} &= \begin{tabular}[t]{@{}l@{}}
                          $1 + \card{A} + \card{B}$ \\
                          \quad if $\mathord{\star} = \mathord{\fuse}$, $\mathord{\with}$, $\mathord{\limp}$, $\mathord{\pmir}$, or $\mathord{\plus}$
                         \end{tabular} \\
    \card{A} &= \mathrlap{1}
                    \quad \text{if $A = \alpha$, $\one$, $\top$, or $\zero$}
  \end{align*}
  \caption{A measure of the number of logical connectives within an ordered context}\label{fig:ordered-rewriting:measure}
\end{marginfigure}%
%
We may then prove the following \lcnamecref{fact:ordered-rewriting:reduction}.
%
\begin{fact}\label{fact:ordered-rewriting:reduction}
  If \kern0.15em$\octx \reduces \octx'$, then $\card{\octx} > \card{\octx'}$.
  Also, if \kern0.15em$\octx \Reduces \octx'$, then $\card{\octx} \geq \card{\octx'}$.
\end{fact}
%
\begin{proof}
  By induction on the structure of the rewriting step.
\end{proof}
%
\noindent
On the basis of this \lcnamecref{fact:ordered-rewriting:reduction}, we will frequently refer to the rewriting relation, $\reduces$, as reduction.


\section{}

\subsection{Binary counters}

\begin{equation*}
  \begin{lgathered}
    \bin{e} \oc \atmL{i} \Reduces \bin{e} \oc \bin{b}_1 \\
    \bin{b}_0 \oc \atmL{i} \Reduces \bin{b}_1 \\
    \bin{b}_1 \oc \atmL{i} \Reduces \atmL{i} \oc \bin{b}_0
  \end{lgathered}
\end{equation*}

\begin{equation*}
  \begin{lgathered}
    \bin{e} \oc \atmL{d} \Reduces \atmR{z} \\
    \bin{b}_0 \oc \atmL{d} \Reduces \atmL{d} \oc \bin{b}'_0 \\
    \bin{b}_1 \oc \atmL{d} \Reduces \bin{b}_0 \oc \atmR{s} \\
    \atmR{z} \oc \bin{b}'_0 \Reduces \atmR{z} \\
    \atmR{s} \oc \bin{b}'_0 \Reduces \bin{b}_1 \oc \atmR{s}
  \end{lgathered}
\end{equation*}

\begin{inferences}
  \infer{e \simu{R} \bin{e}}{}
  \and
  \infer{\octx \oc b_0 \simu{R} \octx' \oc \bin{b}_0}{
    \octx \simu{R} \octx'}
  \and
  \infer{\octx \oc b_1 \simu{R} \octx' \oc \bin{b}_1}{
    \octx \simu{R} \octx'}
  \and
  \infer{\octx \oc i \simu{R} \octx' \oc \atmL{i}}{
    \octx \simu{R} \octx'}
  \\
  \infer{\octx \oc i \oc b_0 \simu{R} \octx' \oc (\atmL{i} \fuse b_0)}{
    \octx \simu{R} \octx'}
\end{inferences}

\subsection{Automata}

\begin{inferences}
  \infer{a \oc \octx \simu{R} \atmR{a} \oc \octx'}{
    \octx \simu{R} \octx'}
  \and
  \infer{q \simu{R} \dfa{q}}{}
\end{inferences}

\begin{equation*}
  \infer{q \simu{R} \dfa{q}'}{
    q \asim q'}
\end{equation*}


\section{Ordered rewriting for specifications}

\subsection{\Aclp*{DFA}}

\begin{equation*}
  \infer[]{a \oc q \reduces q'_a}{}
\end{equation*}
for each \ac{DFA} transition $q \dfareduces[a] q'_a$, and 
\begin{equation*}
  \infer[]{\emp \oc q \reduces F(q)}{}
\end{equation*}
for each state $q$, where $F(q) = \one$ if $q$ is a final state and $F(q) = \top$ otherwise.

\begin{itemize}
\item $q \dfareduces[a] q'_a$ if, and only if, $a \oc q \reduces q'_a$; and 
\item $q \in F$ if, and only if, $\emp \oc q \reduces \one$.
\end{itemize}

\subsection{\Aclp*{NFA}}

Equally straightforward

\subsection{Binary counters}

Values

\paragraph*{An increment operation}
To use ordered rewriting to specify [...]
\begin{equation*}
  \infer[]{e \oc i \reduces e \oc b_1}{}
  \qquad
  \infer[]{b_0 \oc i \reduces b_1}{}
  \qquad
  \infer[]{b_1 \oc i \reduces i \oc b_0}{}
\end{equation*}

Small- and big-step adequacy theorems for increments
\begin{itemize}
\item Slightly simplified because there is no $\fuse$
\end{itemize}


\paragraph*{A decrement operation}
\begin{equation*}
  \infer[]{e \oc d \reduces z}{}
  \qquad
  \infer[]{b_0 \oc d \reduces d \oc b'_0}{}
  \qquad
  \infer[]{b_1 \oc d \reduces b_0 \oc s}{}
  \qquad
  \infer[]{z \oc b'_0 \reduces z}{}
  \qquad
  \infer[]{s \oc b'_0 \reduces b_1 \oc s}{}
\end{equation*}

\begin{itemize}
\item Significantly simpler because there is no $\with$, so we don't need (weak) focusing
\end{itemize}



\section{}

\subsection{Concurrency in ordered rewriting}

As an example of multi-step rewriting, observe that
\begin{equation*}
  % \octx = 
  \alpha_1 \oc (\alpha_1 \limp \alpha_2) \oc (\beta_2 \pmir \beta_1) \oc \beta_1 \Reduces \alpha_2 \oc \beta_2
  % = \octx''
  .
\end{equation*}
In fact, as shown in the adjacent \lcnamecref{fig:ordered-rewriting:concurrent-example},%
%
\begin{marginfigure}
  \begin{equation*}
  \begin{tikzcd}[row sep=large, column sep=tiny]
    &
    \makebox[1em][c]{$\alpha_1 \oc (\alpha_1 \limp \alpha_2) \oc (\beta_2 \pmir \beta_1) \oc \beta_1$}
      \dlar \drar \arrow[Reduces]{dd}
    &
    \\
    \alpha_2 \oc (\beta_2 \pmir \beta_1) \oc \beta_1
      \drar
    &&
    \alpha_1 \oc (\alpha_1 \limp \alpha_2) \oc \beta_2
      \dlar
    \\
    &
    \alpha_2 \oc \beta_2
    &
  \end{tikzcd}
\end{equation*}
  \caption{An example of concurrent ordered rewriting}\label{fig:ordered-rewriting:concurrent-example}
\end{marginfigure}
%
two sequences witness this rewriting: either
\begin{itemize*}[
  mode=unboxed,
  label=, afterlabel=
]
\item the initial state's left half, $\alpha_1 \oc (\alpha_1 \limp \alpha_2)$, is first rewritten to $\alpha_2$ and then its right half, $(\beta_2 \pmir \beta_1) \oc \beta_1$, is rewritten to $\beta_2$; or
\item \textit{vice versa}, the right half is first rewritten to $\beta_2$ and then the left half is rewritten to $\alpha_2$
\end{itemize*}.

Notice that these two sequences differ only in how non-overlapping, and therefore independent, rewritings of the initial state's two halves are interleaved.
Consequently, the two sequences can be -- and indeed should be -- considered essentially equivalent.
% In differing only by the order in which the non-overlapping left and right halves are rewritten, these two rewriting sequences are essentially equivalent.
The details of how the small-step rewrites are interleaved are irrelevant, so that
conceptually, at least, only the big-step trace from $\alpha_1 \oc (\alpha_1 \limp \alpha_2) \oc (\beta_2 \pmir \beta_1) \oc \beta_1$ to $\alpha_2 \oc \beta_2$ remains.
% The details of how the small-step rewrites are interleaved are -- and indeed should be -- swept away, so that conceptually only the big-step trace from $\alpha_1 \oc (\alpha_1 \limp \alpha_2) \oc (\beta_2 \pmir \beta_1) \oc \beta_1$ to $\alpha_2 \oc \beta_2$ remains.

More generally, this idea that the interleaving of independent actions is irrelevant is known as \vocab{concurrent equality}\autocite{Watkins+:CMU02}, and it forms the basis of concurrency.\autocite{??}
Concurrent equality also endows traces $\octx \Reduces \octx'$ with a free partially commutative monoid structure, \ie, traces form a trace monoid.


Because the two indivisual rewriting steps are independent, 
Nothing about the final result, $\alpha_2 \oc \beta_2$, suggests which rewriting sequence 


The rewritings of the left and right halves are not overlapping and therefore independent.
Their independence means that we may view the two rewriting sequences as equivalent -- the two rewriting steps

More generally, any non-overlapping rewritings are independent and may occur in any order.
Rewriting sequences that differ only by the order in which independent rewritings occur may be seen as equivalent sequences.
This equivalence relation, \vocab{concurrent equality}\autocite{Watkins+:CMU02}

because the left half of $\octx$ may be rewritten by the $\jrule{$\limp$D}$ rule to $\alpha_2$, and then the right half may be rewritten to $\beta_2$:

\subsection{Other properties of ordered rewriting}

As the relation $\Reduces$ forms a rewriting system, we may evaluate it along several standard dimensions: termination, confluence.


Because each rewriting step reduces the number of logical connectives present in the state~\parencref{fact:ordered-rewriting:reduction}, ordered rewriting is terminating.
%
\begin{theorem}[Termination]
  No infinite rewriting sequence $\octx_0 \reduces \octx_1 \reduces \octx_2 \reduces \dotsb$ exists.
\end{theorem}
%
\begin{proof}
  Beginning from state $\octx_0$, some state $\octx_i$ will eventually be reached such that either: $\octx_i \nreduces$; or $\card{\octx_i} = 0$ and $\octx_i \reduces \octx_{i+1}$.
  In the latter case, \cref{fact:ordered-rewriting:reduction} establishes $\card{\octx_{i+1}} < 0$, which is impossible.
\end{proof}

Although terminating, ordered rewriting is not confluent.
Confluence requires that all states with a common ancestor, \ie, states $\octx'_1$ and $\octx'_2$ such that $\octx'_1 \secudeR\Reduces \octx'_2$, be joinable, \ie, $\octx'_1 \Reduces\secudeR \octx'_2$.
Because ordered rewriting is directional\fixnote{Is this phrasing correct?} and the relation $\Reduces$ is not symmetric, some nondeterministic choices are irreversible.%
%
\begin{falseclaim}[Confluence]
  If\/ $\octx'_1 \secudeR\Reduces \octx'_2$, then $\octx'_1 \Reduces\secudeR \octx'_2$.
\end{falseclaim}
%
\begin{proof}[Counterexamples]
  Consider the state $\alpha \with \beta$.
  By the rewriting rules for additive conjunction, $\alpha \secuder \alpha \with \beta \reduces \beta$, and hence $\alpha \secudeR \alpha \with \beta \Reduces \beta$.
  However, being atoms, neither $\alpha$ nor $\beta$ reduces.
  And $\alpha \neq \beta$, so $\alpha \Reduces\secudeR \beta$ does \emph{not} hold.

  Even in the $\with$-free fragment, ordered rewriting is not confluent.
  For example,
  % consider the state $(\beta_1 \pmir \alpha) \oc \alpha \oc (\alpha \limp \beta_2)$.
  % By the rewriting rules for right- and left-handed implications,
  \begin{equation*}
    \nsecuder \beta_1 \oc (\alpha \limp \beta_2) \secudeR (\beta_1 \pmir \alpha) \oc \alpha \oc (\alpha \limp \beta_2) \Reduces (\beta_1 \pmir \alpha) \oc \beta_2 \nreduces
    .
    \qedhere
  \end{equation*}
\end{proof}


% Viewing the resource decomposition rules for left- and right-handed implications as rewriting rules is slightly problematic, however.%
% Notice that the premises of these rules both require proofs of $\oseq{\octx |-  A}$.
% In the refactored sequent calculus of \cref{fig:ordered-rewriting:decompose-seq-calc}, that dependence of judgments is fine.
% But for a rewriting system, including arbitrary[/general] proofs would be odd -- rewriting should be a syntax-directed process and should not depend on provability.



% We write the reflexive, transitive closure of $\reduces$ as $\Reduces$.%
% \footnote{This notation is adopted for its similarity with the standard $\pi$-calculus notation for weak transitions, $\cramped{\Reduces[\alpha]}$.}

% This rewriting system is a proper fragment of ordered logic.
% \begin{equation*}
%   \oseq{A \limp (C \pmir B) \dashv|- (A \limp C) \pmir B}
%   \enspace\text{but}\enspace
%   A \limp (C \pmir B) \Longarrownot\Reduces (A \limp C) \pmir B
% \end{equation*}


\section{Unbounded ordered rewriting}

\autocite{Aranda+:FMCO06}

Although a seemingly pleasant property, termination~\parencref{thm:ordered-rewriting:termination} significantly limits the expressiveness of ordered rewriting.
For example, without unbounded rewriting, we cannot even give ordered rewriting specifications of producer-consumer systems or finite automata.

As the proof of termination shows, rewriting is bounded
% $\card{\octx_0}$ is an upper bound on the length of any trace from state $\octx_0$,
precisely because
% $\octx_0$
states
consist of finitely many finite propositions.
To admit unbounded rewriting, we therefore choose to permit infinite propositions in the form of mutually recursive definitions, $\alpha \defd A$.
% could either permit states consisting of infinitely many finite propositions or states consisting of finitely many infinite propositions.
% We choose the latter route [...].
%%
%%
% Infinite propositions are described by mutually recursive definitions $\alpha \defd A$.
These definitions are collected into a signature, $\sig = (\alpha_i \defd A_i)_i$, which indexes the rewriting relations: $\reduces_{\sig}$ and $\Reduces_{\sig}$.%
\footnote{We frequently elide the indexing signature, as it is usually clear from context.} 
To rule out definitions like $\alpha \defd \alpha$ that do not correspond to sensible infinite propositions, we also require that definitions be \vocab{contractive}\autocite{Gay+Hole:AI05} -- \ie, that the body of each recursive definition begin with a logical connective at the top level.

By analogy with recursive types from functional programming\autocite{??}, we must now decide whether to treat definitions \emph{iso}\-re\-cur\-sively or \emph{equi}\-re\-cur\-sively.
Under an equirecursive interpretation, definitions $\alpha \defd A$ may be silently unrolled or rolled at will;
in other words, $\alpha$ is literally \emph{equal} to its unrolling, $A$.
In contrast, under an isorecursive interpretation, unrolling a recursively defined proposition would count as an explicit step of rewriting -- $\alpha \reduces A$, for example.

% Under the isorecursive interpretation, unrolling a recursively defined prop\-o\-sition counts as an explicit step of rewriting.
% We introduce the $\jrule{$\defd$D}$ rule to account for this unrolling:
% \begin{equation*}
%   \infer[\jrule{$\defd$D}]{\alpha \reduces_{\sig} A}{
%     \text{$(\alpha \defd A) \in \sig$}}
% \end{equation*}
% Because $A$ is seen as a proper subformula of [the recursively defined] $\alpha$, this unrolling rule aligns well with the rewriting-as-decomposition philosophy.%
% \footnote{In fact, we could have chosen to include recursive definitions in the sequent calculus, following \textcites{SchroederHeister:LICS93}{Tiu+Momigliano:JAL12} and others.
%   Had we done so, the $\jrule{$\defd$D}$ rule would be seen as the decomposition counterpart to the left rule
%   \begin{equation*}
%     \infer[\lrule{\defd}]{\oseq{\octx'_L \oc \alpha \oc \octx'_R |-_{\sig} C}}{
%       \bigl((\alpha \defd A) \in \sig\bigr) &
%       \oseq{\octx'_L \oc A \oc \octx'_R |-_{\sig} C}}
%   \end{equation*}
% }
% Conversely, there is no rule that permits the rolling of $A$ into $\alpha$, because such a rule would not be a decomposition.

We choose to interpret definitions equirecursively
because the equirecursive treatment, with its generous notion of equality, helps to minimize the overhead of recursively defined propositions.
As a simple example, under the equirecursive definition $\beta \defd a \limp \beta$, we have the trace
\begin{equation*}
  a \oc a \oc \beta = a \oc a \oc (a \limp \beta) \reduces a \oc \beta = a \oc (a \limp \beta) \reduces \beta
\end{equation*}
or, more concisely, $a \oc a \oc \beta \reduces a \oc \beta \reduces \beta$.
Had we chosen
% With
 an isorecursive treatment of the same definition, we would have only the more laborious
\begin{equation*}
  a \oc a \oc \beta \reduces a \oc a \oc (a \limp \beta) \reduces a \oc \beta \reduces a \oc (a \limp \beta) \reduces \beta
  .
\end{equation*}

% As a simple example of ordered rewriting with recursive definitions, consider rewriting under the definition $\beta \defd a \limp \beta$; we have the trace
% \begin{equation*}
%   a \oc a \oc \beta = a \oc a \oc (a \limp \beta) \reduces a \oc \beta = a \oc (a \limp \beta) \reduces \beta
%   .
% \end{equation*}



% Instead of allowing arbitrary infinite propositions, we require that infinite propositions have a regular, recursive structure:
% A signature of mutually recursive definitions
% \begin{equation*}
%   \sig = (\alpha_i \defd A_i)_i
%   ,
% \end{equation*}
% where the variables $\alpha_i$ may occur in the bodies $A_j$.
% %
% To rule out definitions like $\alpha \defd \alpha$ that do notcorrespond to sensible infinite propositions, we additionally require that definitions be \vocab{contractive}\autocite{Gay+Hole:AI05} -- that the body of each recursive definition begin with a logical connective at the top level.




% Contractivity justifies an \emph{equi}recursive treatment of propositions in which definitions may be silently unrolled (or rolled) at will.
% In other words, a proposition $\alpha \defd A$ is \emph{equal} to its unrolling, $[A/\alpha]A$.
% This stands in contrast with an \emph{iso}recursive treatment of definitions in which unrolling a recursively defined proposition would count as an explicit step of rewriting: isorecursively, $\alpha \defd A$ would not be equal to $[A/\alpha]A$, but $\alpha \reduces {[A/\alpha]A}$.

% The equirecursive treatment, with its generous notion of equality, helps to minimize the overhead of recursively defined propositions.
% As a simple example, under the equirecursive definition $\beta \defd a \limp \beta$, we have
% \begin{equation*}
%   a \oc a \oc \beta = a \oc a \oc (a \limp \beta) \reduces a \oc \beta = a \oc (a \limp \beta) \reduces \beta
% \end{equation*}
% or, more concisely, $a \oc a \oc \beta \reduces a \oc \beta \reduces \beta$.
% With an isorecursive treatment of the same definition, we would have only the more laborious
% \begin{equation*}
%   a \oc a \oc \beta \reduces a \oc a \oc (a \limp \beta) \reduces a \oc \beta \reduces a \oc (a \limp \beta) \reduces \beta
%   .
% \end{equation*}

% The proof of termination involves a finite upper bound on the number of rewriting steps that 
% Stated informally, termination means that As captured in \cref{fact:ordered-rewriting:reduction}, states $\octx$ that consist of finitely many finite propositions

% Although its development from the ordered sequent calculus, ordered rewriting as defined thus far is not terribly useful.
% Its main limitation is that finite states 
% With finite states $\octx$ consisting of 

% \subsection{Replication}

% \subsection{Recursively defined propositions}


\subsection{Replication}

In Milner's development of the $\pi$-calculus, there are two avenues to unbounded process behavior: recursive process definitions and replication.


\section{Extended examples of ordered rewriting}

\subsection{Encoding \aclp*{DFA}}

As an extended example, we will use ordered rewriting to specify how \iac{DFA} processes its input.
%
% \Acp{DFA} serve as an example of ordered rewriting,  can be used to specify how \iac{DFA} processes its input.
%
Given \iac{DFA} $\aut{A} = (Q, ?, F)$ over an input alphabet $\ialph$, the idea is to encode each state, $q \in Q$, as an ordered proposition, $\dfa{q}$, in such a way that the \ac{DFA}'s operational semantics are adequately captured by [ordered] rewriting.
%
% The basic idea is to define an encoding, $\dfa{q}$, of \ac{DFA} states as ordered propositions;
% this encoding should adequately reflect the \ac{DFA}'s operational semantics with ordered rewriting traces.
\fixnote{[In general, the behavior of \iac{DFA} state is recursive, so the proposition $\dfa{q}$ will be recursively defined.]}
%
% finite input words, $w \in \finwds{\ialph}$, are encoded as ordered contexts by $\emp \oc \rev{w}$

% \NewDocumentCommand \rev { s m } {
%   \IfBooleanTF {#1}
%     { (#2)^{\mathsf{R}} }
%     { #2^{\mathsf{R}} }
% }

% \begin{align*}
%   \rev{a} &= a \\
%   \rev*{w_1 \wc w_2} &= \rev{w_2} \oc \rev{w_1} \\
%   \rev{\emp} &= \octxe
% \end{align*}

Ideally, \ac{DFA} transitions $q \dfareduces[a] q'_a$ would be in bijective correspondence with rewriting steps $a \oc \dfa{q} \reduces \dfa{q}'_a$, where each input symbol $a$ is encoded by a matching [propositional] atom.
%
We will return to the possibility of this kind of tight correspondence in \cref{??}, but,
%
for now, we will content ourselves with a correspondence with traces rather than individual steps, adopting the following desiderata:
% Unfortunately, ordered rewriting's small step size turns out to be a poor match for [...], so in both cases we will instead content ourselves with corrspondances with \emph{traces}:
% a bijection between transitions $q \dfareduces[a] q'_a$ and \emph{traces} $a \oc \dfa{q} \Reduces \dfa{q}'_a$.
% Similarly, [...] a bijection between accepting states $q \in F$ and traces $\emp \oc \dfa{q} \Reduces \octxe$.
%
% This leads us to adopt the following as desiderata:
\begin{itemize}
\item
  $q \dfareduces[a] q'_a$ if, and only if, $a \oc \dfa{q} \Reduces \dfa{q}'_a$, for all input symbols $a \in \ialph$.
\item
  $q \in F$ if, and only if, $\emp \oc \dfa{q} \Reduces \one$, where the atom $\emp$ functions as an end-of-word marker.
% \item
%   $q \dfareduces[w] q'_w \in F$ if, and only if, $\emp \oc \rev{w} \oc \dfa{q} \Reduces \octxe$.
%   Also, $q \dfareduces[w] q'_w \notin F$ if, and only if, $\emp \oc \rev{w} \oc \dfa{q} \Reduces \top$.
\end{itemize}
Given the reversal (anti-)\-homo\-morph\-ism from finite words to ordered contexts defined in the adjacent \lcnamecref{fig:ordered-rewriting:reversal}%
\begin{marginfigure}
  \begin{align*}
    \rev*{w_1 \wc w_2} &= \rev{w_2} \oc \rev{w_1} \\
    \rev{\emp} &= \octxe \\
    \rev{a} &= a
  \end{align*}
  \caption{An (anti-)\-homo\-morph\-ism for reversal of finite words to ordered contexts}\label{fig:ordered-rewriting:reversal}
\end{marginfigure}%
, the first desideratum is subsumed by a third:
% property that covers finite words:
\begin{itemize}[resume*]
\item $q \dfareduces[w] q'$ if, and only if, $\rev{w} \oc \dfa{q} \Reduces \dfa{q}'$, for all finite words $w \in \finwds{\ialph}$.
\end{itemize}

From these desiderata [and the observation that \acp{DFA}' graphs frequently%
\fixnote{Actually, there is always at least one cycle in a well-formed \ac{DFA}.}
contain cycles], we arrive at the following encoding, in which each state is encoded by one of a collection of mutually recursive definitions:%
\fixnote{$q'_a$, using function or relation?}
\begin{gather*}
  \dfa{q} \defd
    \parens[size=big]{
      \bigwith_{a \in \ialph}(a \limp \dfa{q}'_a)}
    \with
    \parens[size=big]{\emp \limp \dfa{F}(q)}
  % \text{where
  %   $q \dfareduces[a] q'_a$ for all $a \in \ialph$
  %   and
  %   $\dfa{F}(q) = 
  %     \begin{cases*}
  %       \one & if $q \in F$ \\
  %       \top & if $q \notin F$
  %     \end{cases*}$%
  % }
  %
\shortintertext{where}
  %
  q \dfareduces[a] q'_a
  \text{, for all input symbols $a \in \ialph$,\quad and\quad}
  \dfa{F}(q) = 
    \begin{cases*}
      \one & if $q \in F$ \\
      \top & if $q \notin F$%
    \,.
    \end{cases*}
\end{gather*}
Just as each state $q$ has exactly one successor for each input symbol $a$, its encoding, $\dfa{q}$, has exactly one input clause, $(a \limp \dotsb)$, for each symbol $a$.



% The traces $a \oc \dfa{q} \Reduces \dfa{q}'_a$
% % for input symbols $a \in \ialph$
% suggest that $\dfa{q}$ should be a collection of clauses that input atoms $a$ from the left.
% And the traces $\emp \oc \dfa{q} \Reduces \octxe$ or $\emp \oc \dfa{q} \Reduces \top$ suggest that $\dfa{q}$ also contain a clause that inputs atom $\emp$ from the left.
% Thus, we arrive at the encoding


\newthought{For a concrete instance} of this encoding, recall from \cref{ch:automata} the \ac{DFA} (repeated in the adjacent \lcnamecref{fig:ordered-rewriting:dfa-example-ends-b})%
%
\begin{marginfigure}
  \begin{equation*}
    \mathllap{\aut{A}_2 = {}}
    \begin{tikzpicture}[baseline=(q_0.base)]
      \graph [automaton] {
        q_0
         -> [loop above, "a"]
        q_0
         -> ["b", bend left]
        q_1 [accepting]
         -> [loop above, "b"]
        q_1
         -> ["a", bend left]
        q_0;
      };
    \end{tikzpicture}
  \end{equation*}
  \caption{\Iac*{DFA} that accepts, from state $q_0$, exactly those words that end with $b$. (Repeated from \cref{fig:dfa-example-ends-b}.)}\label{fig:ordered-rewriting:dfa-example-ends-b}
\end{marginfigure}
%
that accepts exactly those words, over the alphabet $\ialph = \set{a,b}$, that end with $b$; that \ac{DFA} is encoded by the following definitions:
\begin{equation*}
  \begin{lgathered}
    \dfa{q}_0 \defd (a \limp \dfa{q}_0) \with (b \limp \dfa{q}_1) \with (\emp \limp \top) \\
    \dfa{q}_1 \defd (a \limp \dfa{q}_0) \with (b \limp \dfa{q}_1) \with (\emp \limp \one)
  \end{lgathered}
\end{equation*}
Indeed, just as the \ac{DFA} has a transition $q_0 \dfareduces[b] q_1$, its encoding admits a trace
\begin{align*}
  &b \oc \dfa{q}_0
     = b \oc \bigl((a \limp \dfa{q}_0) \with (b \limp \dfa{q}_1) \with (\emp \limp \top)\bigr)
     \Reduces b \oc (b \limp \dfa{q}_1)
     \reduces \dfa{q}_1
  \,.
\intertext{And, just as $q_1$ is an accepting state, its encoding also admits a trace}
  &\emp \oc \dfa{q}_1 = \emp \oc \bigl((a \limp \dfa{q}_0) \with (b \limp \dfa{q}_1) \with (\emp \limp \one)\bigr) \Reduces \emp \oc (\emp \limp \one) \reduces \one
  \,.
\end{align*}

\newthought{More generally}, this encoding is complete, in the sense that it simulates all \ac{DFA} transitions: $q \dfareduces[a] q'$ implies $a \oc \dfa{q} \Reduces \dfa{q}'$, for all states $q$ and $q'$ and input symbols $a$.

However, the converse does not hold -- the encoding is unsound because there are rewritings that cannot be simulated by \iac{DFA} transition.
% That is, $a \oc \dfa{q} \Reduces \dfa{q}'$ does \emph{not} imply $q \dfareduces[a] q'$.
% 
\begin{falseclaim}
  Let $\aut{A} = (Q, \mathord{\dfareduces}, F)$ be \iac{DFA} over the input alphabet $\ialph$.
  Then $a \oc \dfa{q} \Reduces \dfa{q}'$ implies $q \dfareduces[a] q'$, for all input symbols $a \in \ialph$.
\end{falseclaim}
%
\begin{marginfigure}
    \centering
    % \subfloat[][]{\label{fig:ordered-rewriting:dfa-counterexample:dfa}%
      \begin{equation*}
        \aut{A}'_2 = 
      \begin{tikzpicture}[baseline=(q_0.base)]
        \graph [automaton] {
          q_0
           -> [loop above, "a"]
          q_0
           -> ["b", bend left]
          q_1 [accepting]
           -> [loop above, "b"]
          q_1
           -> ["a", bend left]
          q_0;
          %
%          { [chain shift={(2,0)}]
            s_1 [accepting, below=1.5em of q_1.south]
             -> [loop right, "b"]
            s_1
             -> ["a", bend left]
            q_0;
%          };
        };
      \end{tikzpicture}
    \end{equation*}
    % }
    % \subfloat[][]{\label{fig:ordered-rewriting:dfa-counterexample:encoding}%
      $\!\begin{aligned}
        \dfa{q}_0 &\defd (a \limp \dfa{q}_0) \with (b \limp \dfa{q}_1) \with (\emp \limp \top) \\
        \dfa{q}_1 &\defd (a \limp \dfa{q}_0) \with (b \limp \dfa{q}_1) \with (\emp \limp \one) \\
        \dfa{s}_1 &\defd (a \limp \dfa{q}_0) \with (b \limp \dfa{s}_1) \with (\emp \limp \one)
      \end{aligned}$%
    % }
    \caption{{fig:ordered-rewriting:dfa-counterexample:dfa}~A slightly modified version of the \ac*{DFA} from \cref{fig:ordered-rewriting:dfa-example-ends-b}; and {fig:ordered-rewriting:dfa-counterexample:encoding}~its encoding}\label{fig:ordered-rewriting:dfa-counterexample}
  \end{marginfigure}%
\begin{proof}[Counterexample]
  Consider the \ac{DFA} and encoding shown in the adjacent \lcnamecref{fig:ordered-rewriting:dfa-counterexample}; it is the same \ac{DFA} as shown in \cref{fig:ordered-rewriting:dfa-example-ends-b}, but with one added state, $s_1$, that is unreachable from $q_0$ and $q_1$.
    %
  % When encoded as an ordered rewriting specification, it corresponds to the following definitions:
  % \begin{equation*}
  %   \begin{lgathered}
  %     \dfa{q}_0 \defd (a \limp \dfa{q}_0) \with (b \limp \dfa{q}_1) \with (\emp \limp \top) \\
  %     \dfa{q}_1 \defd (a \limp \dfa{q}_0) \with (b \limp \dfa{q}_1) \with (\emp \limp \one) \\
  %     \dfa{s}_1 \defd (a \limp \dfa{q}_0) \with (b \limp \dfa{s}_1) \with (\emp \limp \one)
  %   \end{lgathered}
  % \end{equation*}
  Notice that, as a coinductive consequence of the equirecursive treatment of definitions, $\dfa{q}_1 = \dfa{s}_1$.
  Previously, we saw that $b \oc \dfa{q}_0 \Reduces \dfa{q}_1$; hence $b \oc \dfa{q}_0 \Reduces \dfa{s}_1$.
  However, the \ac{DFA} has no $q_0 \dfareduces[b] s_1$ transition (because $q_1 \neq s_1$), and so this encoding is unsound with respect to the operational semantics of \acp{DFA}.
\end{proof}

As this counterexample shows, the lack of adequacy stems from attempting to use an encoding that is not injective -- here, $q_1 \neq s_1$ even though $\dfa{q}_1 = \dfa{s}_1$.
In other words, eqality of state encodings is a coarser eqvivalence than equality of the states themselves.

One possible remedy for this lack of adequacy might be to revise the encoding to have a stronger nominal character.
By tagging each state's encoding with an atom that is unique to that state, we can make the encoding manifestly injective.
For instance, given the pairwise distinct atoms $\Set{q \given q \in F}$ and $\Set{\bar{q} \given q \in Q - F}$ to tag final and non-final states, respectively, we could define an alternative encoding, $\check{q}$:
%
\begin{gather*}
  \check{q} \defd
    \parens[size=big]{
      \bigwith_{a \in \ialph}(a \limp \check{q}'_a)}
    \with
    \parens[size=big]{\emp \limp \check{F}(q)}
  %
  \shortintertext{where}
  %
  q \dfareduces[a] q'_a
  \text{, for all input symbols $a \in \ialph$,\quad and\quad}
  \check{F}(q) =
    \begin{cases*}
      q & if $q \in F$ \\
      \bar{q} & if $q \notin F$%
    \,.
    \end{cases*}
\end{gather*}
%
Under this alternative encoding, the states $q_1$ and $s_1$ of \cref{fig:ordered-rewriting:dfa-counterexample} are no longer a counterexample to injectivity:
Because $q_1$ and $s_1$ are distinct states, they correspond to distinct tags, and so $\check{q}_1 \neq \check{s}_1$.

% One possible remedy
% % for this apparent lack of adequacy
% might be to revise the encoding to have a stronger nominal character % .
% by tagging each state's encoding with an atom that is unique to that state.
% For instance, given the pairwise distinct atoms $\set{q \given q \in F}$ and $\set{\bar{q} \given q \in Q - F}$ to tag final and non-final states, respectively, we could define an alternative encoding, $\check{q}$, that is manifestly injective:
% %
% % \begin{marginfigure}
% \begin{gather*}
%   \check{q} \defd
%     \parens[size=big]{
%       \bigwith_{a \in \ialph}(a \limp \check{q}'_a)}
%     \with
%     \parens[size=big]{\emp \limp \check{F}(q)}
%   %
%   \shortintertext{where}
%   %
%   q \dfareduces[a] q'_a
%   \text{, for all input symbols $a \in \ialph$,\quad and\quad}
%   \check{F}(q) =
%     \begin{cases*}
%       q & if $q \in F$ \\
%       \bar{q} & if $q \notin F$%
%     \,.
%     \end{cases*}
% \end{gather*}
% % \end{marginfigure}%
% % , the encoding can be made to be injective.
% % With this change, the alternative encoding is now injective: $\check{q} = \check{s}$ implies $q = s$.

Although such a solution is certainly possible, it seems unsatisfyingly ad~hoc.
A closer examination of the preceding counterexample reveals that the states $q_1$ and $s_1$, while not equal, are in fact bisimilar~\parencref{??}.
In other words, although the encoding is not, strictly speaking, injective, it is injective \emph{up to bisimilarity}: $\dfa{q} = \dfa{s}$ implies $q \asim s$.
This suggests a more elegant solution to the apparent lack of adequacy: the encoding's adequacy should be judged up to \ac{DFA} bisimilarity.
%
\renewcommand{\dfaadequacybisimbody}{%
  Let $\aut{A} = (Q, ?, F)$ be \iac{DFA} over the input alphabet $\ialph$.
  Then, for all states $q$, $q'$, and $s$:
  \begin{enumerate}
  \item\label{enum:ordered-rewriting:dfa-adequacy:1}
    $q \asim s$ if, and only if, $\dfa{q} = \dfa{s}$.
  \item\label{enum:ordered-rewriting:dfa-adequacy:2}
    $q \asim\dfareduces[a]\asim q'$ if, and only if, $a \oc \dfa{q} \Reduces \dfa{q}'$, for all input symbols $a \in \ialph$.    
    More generally, $q \asim\dfareduces[w]\asim q'$ if, and only if, $\rev{w} \oc \dfa{q} \Reduces \dfa{q}'$, for all finite words $w \in \finwds{\ialph}$.
  \item\label{enum:ordered-rewriting:dfa-adequacy:3}
    $q \in F$ if, and only if, $\emp \oc \dfa{q} \Reduces \one$.
  \end{enumerate}%
}%
%  
\begin{restatable*}[
  name=\ac*{DFA} adequacy up to bisimilarity,
  label=thm:ordered-rewriting:dfa-adequacy-bisim
]{theorem}{dfaadequacybisim}
  \dfaadequacybisimbody
% Let $\aut{A} = (Q, \mathord{\dfareduces}, F)$ be \iac{DFA} over the input alphabet $\ialph$.
%   Then, for all states $q$, $q'$, and $s$:
%   \begin{enumerate}
%   \item\label{enum:ordered-rewriting:dfa-adequacy:1}
%     $q \asim s$ if, and only if, $\dfa{q} = \dfa{s}$.
%   \item\label{enum:ordered-rewriting:dfa-adequacy:2}
%     $q \asim\dfareduces[a]\asim q'$ if, and only if, $a \oc \dfa{q} \Reduces \dfa{q}'$, for all input symbols $a \in \ialph$.    
%     More generally, $q \asim\dfareduces[w]\asim q'$ if, and only if, $\rev{w} \oc \dfa{q} \Reduces \dfa{q}'$, for all finite words $w \in \finwds{\ialph}$.
%   \item\label{enum:ordered-rewriting:dfa-adequacy:3}
%     $q \in F$ if, and only if, $\emp \oc \dfa{q} \Reduces \one$.
%   \end{enumerate}
\end{restatable*}

Before proving this \lcnamecref{thm:ordered-rewriting:dfa-adequacy-bisim}, we must first prove a \lcnamecref{lem:ordered-rewriting:dfa-traces}: the only traces from one state's encoding to another's are the trivial traces.
%
\begin{lemma}\label{lem:ordered-rewriting:dfa-traces}
  Let $\aut{A} = (Q, ?, F)$ be \iac{DFA} over the input alphabet $\ialph$.
  For all states $q$ and $s$, if $\dfa{q} \Reduces \dfa{s}$, then $\dfa{q} = \dfa{s}$.
\end{lemma}
%
\begin{proof}
  Assume that a trace $\dfa{q} \Reduces \dfa{s}$ exists.
  If the trace is trivial, then $\dfa{q} = \dfa{s}$ is immediate.
  Otherwise, the trace is nontrivial and consists of a strictly positive number of rewriting steps.
  By inversion, those rewriting steps drop one or more conjuncts from $\dfa{q}$ to form $\dfa{s}$.
  Every \ac{DFA} state's encoding contains exactly $\card{\ialph} + 1$ conjuncts -- one for each input symbol $a$ and one for the end-of-word marker, $\emp$.
  % Being the encoding of \iac{DFA} state, $\dfa{q}$ contains one $(\emp \limp \dotsb)$ conjunct and exactly one $(a \limp \dotsb)$ conjunct for each input symbol $a$.
  % Similarly, $\dfa{s}$ must contain the same.
  If even one conjunct is dropped from $\dfa{q}$, not enough conjuncts will remain to form $\dfa{s}$.
  Thus, a nontrivial trace $\dfa{q} \Reduces \dfa{s}$ cannot exist.
\end{proof}
%
\noindent
It is important to differentiate this \lcnamecref{lem:ordered-rewriting:dfa-traces} from the false claim that a state's encoding can take no rewriting steps.
There certainly exist nontrivial traces from $\dfa{q}$, but they do not arrive at the encoding of any state.

With this \lcnamecref{lem:ordered-rewriting:dfa-traces} now in hand, we can proceed to proving adequacy up to bisimilarity.
%
\dfaadequacybisim
%
\begin{proof}
  Each part is proved in turn.
  The proof of part~\ref{enum:ordered-rewriting:dfa-adequacy:2} % and~\ref{enum:ordered-rewriting:dfa-adequacy:4}
  depends on the proof of part~\ref{enum:ordered-rewriting:dfa-adequacy:1}.
  \begin{enumerate}[parsep=0em, listparindent=\parindent]
  %% Part one
  \item
    We shall show that bisimilarity coincides with equality of encodings, proving each direction separately.
    \begin{itemize}[parsep=0em, listparindent=\parindent]
    \item
      To prove that bisimilar \ac{DFA} states have equal encodings -- \ie, that $q \asim s$ implies $\dfa{q} = \dfa{s}$ -- a fairly straightforward proof by coinduction suffices.

      Let $q$ and $s$ be bisimilar states.
      By the definition of bisimilarity~\parencref{??}, two properties hold:
      \begin{itemize}
      \item For all input symbols $a$, the unique $a$-successors of $q$ and $s$ are also bisimilar.
      \item States $q$ and $s$ have matching finalities -- \ie, $q \in F$ if and only if $s \in F$.
      \end{itemize}
      Applying the coinductive hypothesis to the former property, we may deduce that, for all symbols $a$, the $a$-successors of $q$ and $s$ also have equal encodings.
      From the latter property, it follows that $\dfa{F}(q) = \dfa{F}(s)$.
      Because definitions are interpreted equirecursively, these equalities together imply that $q$ and $s$ themselves have equal encodings.

    \item
      To prove the converse -- that states with equal encodings are bisimilar -- we will show that the relation $\mathord{\simu{R}} = \Set{(q, s) \given \dfa{q} = \dfa{s}}$, which relates states if they have equal encodings, is a bisimulation and is therefore included in bisimilarity.
      \begin{itemize}
      \item
        The relation $\simu{R}$ is symmetric.
      \item
        We must show that $\simu{R}$-related states have $\simu{R}$-related $a$-successors, for all input symbols $a$.

        Let $q$ and $s$ be $\simu{R}$-related states.
        Being $\simu{R}$-related, $q$ and $s$ have equal encodings;
        because definitions are interpreted equirecursively, the unrollings of those encodings are also equal.
        By definition of the encoding, it follows that, for each input symbol $a$, the unique $a$-successors of $q$ and $s$ have equal encodings.
        Therefore, for each $a$, the $a$-successors of $q$ and $s$ are themselves $\simu{R}$-related.

      \item
        We must show that $\simu{R}$-related states have matching finalities.

        Let $q$ and $s$ be $\simu{R}$-related states, with $q$ a final state.
        Being $\simu{R}$-related, $q$ and $s$ have equal encodings;
        because definitions are interpreted equirecursively, the unrollings of those encodings are also equal.
        It follows that $\dfa{F}(q) = \dfa{F}(s)$, and so $s$ is also a final state.
      \end{itemize}
    \end{itemize}

  %% Part two
  \item
    We would like to prove that $q \asim\dfareduces[a]\asim q'$ if, and only if, $a \oc \dfa{q} \Reduces \dfa{q}'$, or, more generally, that $q \asim\dfareduces[w]\asim q'$ if, and only if, $\rev{w} \oc \dfa{q} \Reduces \dfa{q}'$.
    Because bisimilar states have equal encodings (part~\ref{enum:ordered-rewriting:dfa-adequacy:1}) and bisimilarity is reflexive (\cref{??}), it suffices to show two stronger statements:
    \begin{enumerate*}
    \item that $q \dfareduces[w] q'$ implies $\rev{w} \oc \dfa{q} \Reduces \dfa{q}'$; and
    \item that $\rev{w} \oc \dfa{q} \Reduces \dfa{q}'$ implies $q \dfareduces[w]\asim q'$.
    \end{enumerate*}
    %
    We prove these in turn.
    %
    \begin{enumerate}
    %% Subpart (a)
    \item
      We shall prove that $q \dfareduces[w] q'$ implies $\rev{w} \oc \dfa{q} \Reduces \dfa{q}'$ by induction over the structure of word $w$.
      \begin{itemize}
      \item
        Consider the case of the empty word, $\emp$; we must show that $q = q'$ implies $\dfa{q} \Reduces \dfa{q}'$.
        Because the encoding is a function, this is immediate.
      \item
        Consider the case of a nonempty word, $a \wc w$; we must show that $q \dfareduces[a]\dfareduces[w] q'$ implies $\rev{w} \oc a \oc \dfa{q} \Reduces \dfa{q}'$.
        Let $q'_a$ be an $a$-successor of state $q$ that is itself $w$-succeeded by state $q'$.
        There exists, by definition of the encoding, a trace
        \begin{equation*}
          \rev{w} \oc a \oc \dfa{q}
            \Reduces \rev{w} \oc a \oc (a \limp \dfa{q}'_a)
            \reduces \rev{w} \oc \dfa{q}'_a
            \Reduces \dfa{q}'
          \,,
        \end{equation*}
        with the trace's tail justified by an appeal to the inductive hypothesis.
        % Because $q'$ is a $w$-successor of $q'_a$, an appeal to the inductive hypothesis yields a trace $\rev{w} \oc \dfa{q}'_a \Reduces \dfa{q}'$.
      \end{itemize}

      % Let $q'$ be an $a$-successor of state $q$.
      % There exists, by definition of the encoding, a trace
      % \begin{equation*}
      %   a \oc \dfa{q} \Reduces a \oc (a \limp \dfa{q}') \reduces \dfa{q}'
      % \,.
      % \end{equation*}

    %% Subpart (b)
    \item
      We shall prove that $\rev{w} \oc \dfa{q} \Reduces \dfa{q}'$ implies $q \dfareduces[w]\asim q'$ by induction over the structure of word $w$.
      \begin{itemize}
      \item
        Consider the case of the empty word, $\emp$;
        we must show that $\dfa{q} \Reduces \dfa{q}'$ implies $q \asim q'$.
        By \cref{lem:ordered-rewriting:dfa-traces}, $\dfa{q} \Reduces \dfa{q}'$ implies that $q$ and $q'$ have equal encodings.
        Part~\ref{enum:ordered-rewriting:dfa-adequacy:1} can then be used to establish that $q$ and $q'$ are bisimilar.
      \item
        Consider the case of a nonempty word, $a \wc w$;
        we must show that $\rev{w} \oc a \oc \dfa{q} \Reduces \dfa{q}'$ implies $q \dfareduces[a]\dfareduces[w]\asim q'$.
        By inversion\fixnote{Is this enough justification?}, the given trace can only begin by inputting $a$:
        \begin{equation*}
          \rev{w} \oc a \oc \dfa{q}
            \Reduces \rev{w} \oc a \oc (a \limp \dfa{q}'_a)
            \reduces \rev{w} \oc \dfa{q}'_a
            \Reduces \dfa{q}'
          \,,
        \end{equation*}
        where $q'_a$ is an $a$-successor of state $q$.
        An appeal to the inductive hypothesis on the trace's tail yields $q'_a \dfareduces[w]\asim q'$, and so the \ac{DFA} admits $q \dfareduces[a]\dfareduces[w]\asim q'$, as required.
      \end{itemize}
      % Assume that a trace $a \oc \dfa{q} \Reduces \dfa{q}'$ exists.
      % By the input lemma, $\dfa{q} \Reduces (a \limp A) \oc \octx'$ for some proposition $A$ and context $\octx'$ such that $A \oc \octx' \Reduces \dfa{q}'$.
      % Upon inversion of the trace from $\dfa{q}$, we conclude that $A = \dfa{q}'_a$, where $q'_a$ is an $a$-successor of $q$, and that $\octx'$ is empty -- in other words, we have a trace $\dfa{q}'_a \Reduces \dfa{q}'$.
      % Such a trace exists only if $\dfa{q}'_a = \dfa{q}'$.
      % By part~\ref{enum:ordered-rewriting:dfa-adequacy:1} of this \lcnamecref{thm:ordered-rewriting:dfa-adequacy-bisim}, it follows that $q'_a$ and $q'$ are bisimilar.
    \end{enumerate}

  %% Part three
  \item
    We shall prove that the final states are exactly those states $q$ such that $\emp \oc \dfa{q} \Reduces \one$.
    \begin{itemize}
    \item
      Let $q$ be a final state; accordingly, $\dfa{F}(q) = \one$.
      There exists, by definition of the encoding, a trace
      \begin{equation*}
        \emp \oc \dfa{q} \Reduces \emp \oc (\emp \limp \dfa{F}(q)) \reduces \dfa{F}(q) = \one
      \,.
      \end{equation*}

    \item
      Assume that a trace $\emp \oc \dfa{q} \Reduces \one$ exists.
      By inversion\fixnote{Is this enough justification?}, this trace can only begin by inputting $\emp$:
      \begin{equation*}
        \emp \oc \dfa{q} \Reduces \emp \oc (\emp \limp \dfa{F}(q)) \reduces \dfa{F}(q) \Reduces \one
      \,.
      \end{equation*}
      The tail of this trace, $\dfa{F}(q) \Reduces \one$, can exist only if $q$ is a final state.
    %
    \qedhere
    \end{itemize}

  % %% Part four
  % \item 
  %   We would like to prove that $q \asim\dfareduces[w]\asim q'$ if, and only if, $\rev{w} \oc \dfa{q} \Reduces \dfa{q}'$.
  %   Because bisimilar states have equal encodings (part~\ref{enum:ordered-rewriting:dfa-adequacy:1}) and bisimilarity is reflexive (\cref{??}), it suffices to show:
  %   \begin{enumerate*}
  %   \item that $q \dfareduces[w] q'$ implies $\rev{w} \oc \dfa{q} \Reduces \dfa{q}'$; and
  %   \item that $\rev{w} \oc \dfa{q} \Reduces \dfa{q}'$ implies $q \dfareduces[w]\asim q'$.
  %   \end{enumerate*}

  %   Both statements can be established by induction over the structure of word $w$.
  %   The latter proof is slightly more involved and deserves a bit of explanation.
  %   \begin{itemize}
  %   \item Consider the case in which $w$ is the empty word; we must show that $\dfa{q} \Reduces \dfa{q}'$ implies $q \asim q'$.
  %     By \cref{lem:ordered-rewriting:dfa-traces}, $\dfa{q} \Reduces \dfa{q}'$ implies that $\dfa{q} = \dfa{q}'$.
  %     Part~\ref{enum:ordered-rewriting:dfa-adequacy:1} can then be used to establish $q$ and $q'$ as bisimilar.

  %   \item Consider the case of a nonempty word, $a \wc w$.
  %     We must show that $\rev{w} \oc a \oc \dfa{q} \Reduces \dfa{q}'$ implies $q \dfareduces[a]\dfareduces[w]\asim q'$.
  %     By inversion, the given trace must begin by inputting $a$:
  %     \begin{equation*}
  %       \rev{w} \oc a \oc \dfa{q} \Reduces \rev{w} \oc a \oc (a \limp \dfa{q}'_a) \reduces \rev{w} \oc \dfa{q}'_a \Reduces \dfa{q}'
  %       \,,
  %     \end{equation*}
  %     where $q'_a$ is an $a$-successor of state $q$.
  %     Appealing to the inductive hypothesis on the trace's tail yields $q'_a \dfareduces[w]\asim q'$, and so $q \dfareduces[a]\dfareduces[w]\asim q'$, as required.
  %   %
  %   \qedhere
  %   \end{itemize}
  \end{enumerate}
\end{proof}


\subsection{Encoding \aclp*{NFA}?}

We would certainly be remiss if we did not attempt to generalize the rewriting specification of \acp{DFA} to one for their nondeterministic cousins.

Differently from \ac{DFA} states, \iac{NFA} state $q$ may have several nondeterministic successors for each input symbol $a$.
To encode the \ac{NFA} state $q$, all of its $a$-successors are collected in an alternative conjunction underneath the left-handed input of $a$.
Thus, the encoding of \iac{NFA} state $q$ becomes
\begin{equation*}
  \nfa{q} \defd
    \parens[size=auto]{\displaystyle
      \bigwith_{a \in \ialph}
        \parens[size=big]{a \limp \parens{\bigwith_{q'_a} \nfa{q}'_a}}
    }
    \with
    \parens[size=big]{\emp \limp \nfa{F}(q)}
  \,,
\end{equation*}
where $\nfa{F}(q)$ is defined as for \acp{DFA}.

The adjacent \lcnamecref{fig:ordered-rewriting:nfa-example}
\begin{marginfigure}
  \centering
  % \subfloat[][]{\label{fig:ordered-rewriting:nfa-example:nfa}%
    \begin{tikzpicture}
      \graph [automaton] {
        q_0
         -> ["a,b", loop above]
        q_0
         -> ["b"]
        q_1 [accepting]
         -> ["a,b"]
        q_2
         -> ["a,b", loop above]
        q_2;
      };
    \end{tikzpicture}
  % }

%   \subfloat[][]{\label{fig:ordered-rewriting:nfa-example:encoding}%
      $\!\begin{aligned}
        \nfa{q}_0 &\defd (a \limp \nfa{q}_0) \with \bigl(b \limp (\nfa{q}_0 \with \nfa{q}_1)\bigr) \with (\emp \limp \top) \\
        \nfa{q}_1 &\defd (a \limp \nfa{q}_2) \with (b \limp \nfa{q}_2) \with (\emp \limp \one) \\
        \nfa{q}_2 &\defd (a \limp \nfa{q}_2) \with (b \limp \nfa{q}_2) \with (\emp \limp \top)
      \end{aligned}$
%     }

  \caption{{fig:ordered-rewriting:nfa-example:nfa}~\Iac*{NFA} that accepts exactly those words, over the alphabet $\ialph = \set{a,b}$, that end with $b$; and {fig:ordered-rewriting:nfa-example:encoding}~its encoding}\label{fig:ordered-rewriting:nfa-example}
\end{marginfigure}%
recalls from \cref{ch:automata} \iac{NFA} that accepts exactly those words, over the alphabet $\ialph = \set{a,b}$, that end with $b$.
Using the above encoding of \acp{NFA}, ordered rewriting does indeed simulate this \ac{NFA}.
For example, just as there are transitions $q_0 \nfareduces[b] q_0$ and $q_0 \nfareduces[b] q_1$, there are traces
\begin{equation*}
  \begin{tikzcd}[
    cells={inner xsep=0.65ex,
           inner ysep=0.4ex},
         % nodes={draw},
    row sep=0em,
    column sep=scriptsize
  ]
    &[-0.2em] \nfa{q}_0
    \\
    b \oc \nfa{q}_0 \Reduces b \oc \bigl(b \limp (\nfa{q}_0 \with \nfa{q}_1)\bigr) \reduces \nfa{q}_0 \with \nfa{q}_1
      \urar[reduces, start anchor=east]
      \drar[reduces, start anchor=base east]
    \\
    & \nfa{q}_1
  \end{tikzcd}
\end{equation*}

Unfortunately, while it does simulate \ac{NFA} behavior, this encoding is not adequate.
Like \ac{DFA} states, \ac{NFA} states that have equal encodings are bisimilar.
% \begin{proof}
%   Define a relation $\mathord{\simu{R}} = \set{(q, s) \given \nfa{q} = \nfa{s}}$; we will show that $\simu{R}$ is a bisimulation.
%   \begin{itemize}
%   \item Assume that $s \simu{R}^{-1} q \nfareduces[a] q'_a$.
%     By definition, $a \oc \nfa{q} \Reduces \nfa{q}'_a$.
%     Because $\nfa{q} = \nfa{s}$, it follows that $s \nfareduces[a] s'_a$ for some state $s'_a$ such that $\nfa{q}'_a = \nfa{s}'_a$ -- that is, $q'_a \simu{R} s'_a$.
%     Thus, $s \nfareduces[a]\simu{R}^{-1} q'_a$.
%   \item Assume that $q \simu{R} s$.
%     It follows that $\nfa{F}(q) = \nfa{F}(s)$.
%     Thus, $q$ is an accepting state if and only if $s$ is.
%   \end{itemize}
% \end{proof}
However, for \acp{NFA}, the converse does not hold: bisimilar states do not necessarily have equal encodings.
%
\begin{falseclaim}
  Let $\aut{A} = (Q, ?, F)$ be \iac{NFA} over input alphabet $\ialph$.
  Then $q \asim s$ implies $\nfa{q} = \nfa{s}$, for all states $q$ and $s$.
\end{falseclaim}
%
\begin{proof}[Counterexample]
  Consider the \ac{NFA} and encoding depicted in the adjacent \lcnamecref{fig:ordered-rewriting:nfa-counterexample}.
  \begin{marginfigure}
    \begin{alignat*}{2}
      \begin{tikzpicture}
        \graph [automaton] {
          q_0 [accepting]
           -> ["a", loop above]
          q_0
           -> ["a", overlay]
          q_1 [accepting, overlay]
           -> ["a", loop above, overlay]
          q_1;
        };
      \end{tikzpicture}
      &\quad&&
      \\
      &\quad& \nfa{q}_0 &\defd \bigl(a \limp (\nfa{q}_0 \with \nfa{q}_1)\bigr) \with (\emp \limp \one) \\
      &\quad& \nfa{q}_1 &\defd (a \limp \nfa{q}_1) \with (\emp \limp \one)
    \end{alignat*}
    \caption{\Iac*{NFA} that accepts all finite words over the alphabet $\ialph = \set{a}$}\label{fig:ordered-rewriting:nfa-counterexample}
  \end{marginfigure}
  It is easy to verify that the relation $\set{q_1} \times \set{q_0,q_1}$ is a bisimulation; in particular, $q_1$ simulates the $q_0 \nfareduces[a] q_1$ transition by its self-loop, $q_1 \nfareduces[a] q_1$.
  Hence, $q_0$ and $s_0$ are bisimilar.
  %
  % These same \acp{NFA} are encoded by the following definitions.
  % \begin{align*}
  %   \nfa{q}_0 &\defd (a \limp \nfa{q}_0) \with (\emp \limp \one)
  % \shortintertext{and}
  %   \nfa{s}_0 &\defd \bigl(a \limp (\nfa{s}_0 \with \nfa{s}_1)\bigr) \with (\emp \limp \one) \\
  %   \nfa{s}_1 &\defd (a \limp \nfa{s}_1) \with (\emp \limp \one)
  % \end{align*}
  It is equally easy to verify, by unrolling the definitions used in the encoding, that $\nfa{q}_0 \neq \nfa{s}_0$.
\end{proof}

For \acp{DFA}, bisimilar states do have equal encodings because the inherent determinism \ac{DFA} bisimilarity is a rather fine-grained equivalence.
Because each \ac{DFA} state has exactly one successor for each input symbol
The additional flexibility entailed by nondeterminism

Once again, it would be possible to construct an adequate encoding, by tagging each state with a unique atom.
% with a stronger nominal character

For the moment, we will put aside the question of an adequate encoding of \acp{NFA}.

\subsection{Binary representation of natural numbers}

As a further example of ordered rewriting, consider a rewriting specification of binary counters: binary representations of natural numbers equipped with increment and decrement operations.

\paragraph*{Binary representations}
In this setting, we represent a natural number in binary by
% A binary representation of a natural number is
an ordered context that consists of a big-endian sequence of atoms $b_0$ and $b_1$, prefixed by the atom $e$; leading $b_0$s are permitted.
For example, both $\octx = e \oc b_1$ and $\octx' = e \oc b_0 \oc b_1$ are valid binary representations of the natural number $1$.

To be more precise, we inductively define a relation, $\aval{}{}$, 
% [between binary representations and the natural number [value]s that they represent.]
that assigns to each binary representation a unique natural number denotation.
% , defined inductively by the following rules.
% between ordered contexts and natural numbers that is inductively defined by the following rules.
When $\aval{\octx}{n}$, we say that $\octx$ denotes, or represents, natural number $n$ in binary.
%
\newcommand{\ooavalrules}{%
  \infer[\jrule{$e$-V}]{\aval{e}{0}}{}
  \and
  \infer[\jrule{$b_0$-V}]{\aval{\octx \oc b_0}{2n}}{
    \aval{\octx}{n}}
  \and
  \infer[\jrule{$b_1$-V}]{\aval{\octx \oc b_1}{2n+1}}{
    \aval{\octx}{n}}%
}%
\begin{inferences}
  \ooavalrules
\end{inferences}
% [In addition to assigning each binary representation a natural number value,]
Besides providing a denotational semantics of binary numbers, the $\aval{}{}$ relation also serves to implicitly characterize the well-formed binary numbers as those ordered contexts $\octx$ that form the relation's domain of definition.%
% Implicit in this definition is 
\footnote{Alternatively, the well-formed binary numbers could be described more explicitly by the grammar
% More precisely, the binary numbers are those contexts that are generated by the following grammar:
\begin{equation*}
  \octx \Coloneqq e \mid \octx \oc b_0 \mid \octx \oc b_1
  \,.
\end{equation*}%
}

These properties\fixnote{which properties?} of the $\aval{}{}$ relation are proved as the following adequacy \lcnamecref{thm:ordered-rewriting:binary-adequacy}.
%
\newcommand{\ooavaltheorem}{%
  \begin{theorem}[Adequacy of binary representations]\label{thm:ordered-rewriting:binary-adequacy}%
    \leavevmode
    \begin{thmdescription}
    \item[Functional]
      For every binary number $\octx$, there exists a unique natural number $n$ such that $\aval{\octx}{n}$.
      % [If $\aval{\octx}{n}$ and $\aval{\octx}{n'}$, then $n = n'$.]
    \item[Surjectivity]
      For every natural number $n$, there exists a binary number $\octx$ such that $\aval{\octx}{n}$.
      % Moreover, when the rule for $b_0$ is restricted to nonzero even numbers, the representation is unique.
    \item[Value]
      If $\aval{\octx}{n}$, then $\octx \nreduces$.
    \end{thmdescription}
  \end{theorem}%
}%
%
\ooavaltheorem
\begin{proof}
  The three claims may be proved by induction over the structure of $\octx$, and by induction on $n$, respectively.
\end{proof}

Notice that the above $\jrule{$e$-V}$ and $\jrule{$b_0$-V}$ rules overlap when the denotation\fixnote{represented natural number?} is $0$, giving rise to the leading $b_0$s that make the $\aval{}{}$ relation surjective:
for example, both $\aval{e \oc b_1}{1}$ and $\aval{e \oc b_0 \oc b_1}{1}$ hold.
However, if the rule for $b_0$ is restricted to \emph{nonzero} even numbers, then each natural number has a unique, canonical representation that is free of leading $b_0$s.%
\footnote{
  A restriction of the $b_0$ rule to nonzero even numbers is:
  \begin{equation*}
    \infer{\aval{\octx \oc b_0}{2n}}{
      \aval{\octx}{n} & \text{($n > 0$)}}
  \,.
  \end{equation*}
  The leading-$b_0$-free representations could alternatively be seen as the canonical representatives of the equivalence classes induced by the equivalence relation among binary numbers that have the same denotation: $\octx \equiv \octx'$ if $\aval{\octx}{n}$ and $\aval{\octx'}{n}$ for some $n$.}


% This leads to a nontrivial equivalence relation over binary numbers: binary numbers $\octx$ and $\octx'$ are equivalent up to leading $b_0$s if both $\octx$ and $\octx'$ represent the same $n$.
% % $\aval{\octx}{n}$ and $\aval{\octx}{n}$ for some $n$.
% Corresponding to the [...] of leading $b_0$s.

% Define a nontrivial equivalence relation on binary numbers. 

% If the rule for $\octx \oc b_0$ is restricted to apply to only strictly positive even numbers


% (\octx, n) R(~) (\octx', n) iff \octx ~ n and \octx' ~ n
% R(~) is an equivalence relation.
% ~/R(~) = {[(\octx, n)] | \octx ~ n}
% \octx ~/R(~) n iff \octx is leading-b0-free and \octx ~ n

% Let ~ be an equivalence relation over X.
% Let [-] : X -> X/~ be the surjection that maps to equivalence classes
% Let s : X/~ -> X be an injective function such that [s(c)] = c.

% This function describes an adequate representation because it forms a bijection (up to leading $b_0$s) between binary counters and natural numbers.
% %
% \begin{theorem}[Representational adequacy]
%   Up to leading $b_0$s, the $\aval{}{}$ relation is a bijection:
%   \begin{itemize}[noitemsep]
%   \item
%     For every binary counter $\octx$, there exists a unique natural number $n$ such that $\aval{\octx}{n}$.
%   \item
%     Conversely, for every natural number $n$, there exists a binary counter $\octx$, unique up to leading $b_0$s, such that $\aval{\octx}{n}$.
%   \item
%     For all binary counters $\octx$ and $\octx'$ that are syntactically equal modulo leading $b_0$s, $\aval{\octx}{n}$ if, and only if, $\aval{\octx'}{n}$.
%   \end{itemize}
% \end{theorem}
% %
% \begin{proof}
%   The two claims may be proved by induction over the structure of the binary counter $\octx$, and by induction on the natural number $n$, respectively.
% \end{proof}

% Alternatively, binary counters could be organized around leading-$b_0$--free representations.
% Leading-$b_0$--free representations form a retract within the binary counters, with the 
% \begin{equation*}
%   \begin{lgathered}
%     r(e) = e \\
%     r(\octx \oc b_0) =
%       \begin{cases*}
%         e & if $r(\octx) = e$ \\
%         r(\octx) \oc b_0 & otherwise
%       \end{cases*} \\
%     r(\octx \oc b_1) = r(\octx) \oc b_1
%   \end{lgathered}
% \end{equation*}

% Notice that the representations that are free of leading $b_0$s form a retract 

% \begin{inferences}
%   \infer{\aval{e}{0}}{}
%   \and
%   \infer{\aval{\octx \oc b_0}{2n+2}}{
%     \aval{\octx}{' n+1}}
%   \and
%   \infer{\aval{\octx \oc b_1}{2n+1}}{
%     \aval{\octx}{n}}
%   \and
%   \infer{\aval{\octx \oc b_0}{' 2n+2}}{
%     \aval{\octx}{' n+1}}
%   \and
%   \infer{\aval{\octx \oc b_1}{' 2n+1}}{
%     \aval{\octx}{n}}
% \end{inferences}

% % bin = +{ e: 1, b0: pos, b1: bin }
% % pos = +{ b0: pos, b1: bin }
% %
% % bin = &{ i: pos, d: +{ z: 1, s: bin  } }
% % pos = &{ i: pos, d: bin }
% %
% % e = (e * b1 / i) & (z / d) & (e * z / h)
% % b0 = (b1 / i) & (d * (s \ b1) * s) & (b0 * s / h)
% % b1 = (i * b0 / i) & (h * b1' * s / d) & (b1 * s / h)
% % b1' = (z \ 1) & (s \ b0)

% % C b0 d --> C d (s \ b1) s --> C' b1 s
% % 2(n+1) --> 2(n+1)-1 + 1 --> 2n+1 + 1

% % e b1 d --> e d b1' s --> z b1' s --> e s
% % 2(0)+1 --> 2(0) + 1 --> 2(0) + 1 --> 0 + 1

% % C b1 d --> C d b1' s --> C' s b1' s --> C' i b0 s
% % 2(n+1)+1 --> 2(n+1) + 1 --> 2(n + 1) + 1 --> 
% \begin{equation*}
%   \begin{lgathered}
%     e \defd (e \fuse b_1 \pmir i) \with (z \pmir d) \\
%     b_0 \defd (b_1 \pmir i) \with (d \fuse b_1 \fuse s \pmir d) \\
%     b_1 \defd (i \fuse b_0 \pmir i) \with (d \fuse b'_1 \pmir d) \\
%     p_0 \defd () \with (d \fuse b_1 \pmir d) \\
%     b'_1 \defd (z \limp e \fuse s) \with (s \limp i \fuse b_0 \fuse s)
%   \end{lgathered}
% \end{equation*}

% \begin{equation*}
%   \begin{lgathered}
%     e \defd (e \fuse b_1 \pmir i) \with (z \pmir d) \\
%     b_0 \defd (b_1 \pmir i) \with (d \fuse b'_0 \pmir d) \with (c \fuse b^c_0 \pmir c) \\
%     b^c_0 \defd (z \limp z) \with (s \limp b_0 \fuse s) \\
%     b'_0 \defd (z \limp z) \with (s \limp b_1 \fuse s) \\
%     b_1 \defd (b_0 \pmir i) \with (d \fuse ((z \limp e \fuse s) \with (s \limp b_0 \fuse s)) \pmir d)
%   \end{lgathered}
% \end{equation*}

% % e b1 d --> e s
% % e b1 b1 d --> e b1 d ... --> e s ... --> e i b0 s

% % 2(n+1)+1 - 1 --> 2(n+1-1)+1

% % e b1 d --> e d b1' --> z b1' --> e s
% % e b1 b0 d --> e s b0' --> e b1 s
% % e b1 b0 b0 d --> e b1 s b0' --> e b1 b1 s
% % e b1 b1 d --> e s b1' --> e i b0 s --> e b1 b0 s
% % e b1 b0 b1 d --> e b1 s b1' --> e b1 i b0 s --> e b1 b0 b0 s

% \begin{inferences}
%   \infer{\aval{e}{e}}{}
%   \and
%   \infer{\aval{\octx \oc b_0}{e}}{
%     \aval{\octx}{e}}
%   \and
%   \infer{\aval{\octx \oc b_0}{\octx' \oc b_0}}{
%     \aval{\octx}{\octx'}}
%   \and
%   \infer{\aval{\octx \oc b_1}{\octx' \oc b_1}}{
%     \aval{\octx}{\octx'}}
% \end{inferences}
% By analogy with functional computation, the ordered contexts $\octx$ that appear in this relation will serve as values -- end results of computations.

% The relation $\aval{}{}$ defines an adequate representation because it is, in fact, a bijection (up to leading $b_0$s) between ordered contexts and natural numbers.

% If we restrict our attention to counters $\octx$ that are free of leading $b_0$s, then the $\aval{}{}$ relation is a bijection with the natural numbers.
% %
% Right-unique: $\aval{\octx}{n}$ and $\aval{\octx}{n'}$, then $n = n'$.
% Left-total:

% \begin{theorem}[Representational adequacy]
%   For all natural numbers $n$, there exists a context $\octx$, unique up to leading $b_0$s, such that $\aval{\octx}{n}$.
%   Moreover, the relation $\aval{}{}$ is functional -- \ie, if $\aval{\octx}{n}$ and $\aval{\octx}{n'}$, then $n = n'$.
% \end{theorem}
% \begin{proof}
%   The first part follows by induction on the natural number $n$; the second part follows by induction on the structure of the context $\octx$.
% \end{proof}
% %
% In other words, the binary representations that are free of leading $b_0$s form a retract.


\paragraph{An increment operation}
To use ordered rewriting to describe an increment operation on binary representations, we introduce a new, uninterpreted atom $i$ that will serve as an increment instruction.

Given a binary number $\octx$ that represents $n$, we may append $i$ to form a computational state, $\octx \oc i$.
For $i$ to adequately represent the increment operation, the state $\octx \oc i$ must meet two conditions, captured by the following global desiderata:
% \lcnamecref{thm:increment-structural-adequacy}.
%global desiderata: 
\begin{theorem}\label{thm:increment-structural-adequacy}
  Let $\octx$ be a binary representation of $n$.
  Then:
  \begin{itemize}[nosep]
  \item
    \emph{some} computation from $\octx \oc i$ results in a binary representation of $n+1$ -- that is, $\octx \oc i \Reduces\aval{}{n+1}$; and
  \item
    \emph{any} computation from $\octx \oc i$ results in a binary representation of $n+1$ -- that is, $\octx \oc i \Reduces\aval{}{n'}$ only if $n' = n+1$.%
    \fixnote{Compare \enquote{If $\octx \oc i \Reduces \octx'$, then $\octx' \Reduces\aval{}{n+1}$.}}
    % Moreover, computations preserve an absence of leading $b_0$s.\fixnote{Is this necessary?}
  \end{itemize}
\end{theorem}
\noindent
For example, because $e \oc b_1$ denotes $1$, a computation $e \oc b_1 \oc i \Reduces\aval{}{2}$ must exist; moreover, every computation $e \oc b_1 \oc i \Reduces\aval{}{n'}$ must satisfy $n' = 2$.

\newthought{To implement these} global desiderata locally, the previously uninterpreted atoms $e$, $b_0$, and $b_1$ are now given mutually recursive definitions that describe how they may be rewritten when the increment instruction, $i$, is encountered.
\begin{description}[font=\color{structure}]
\item[$e \defd e \fuse b_1 \pmir i$]
  To increment $e$, append $b_1$ as a new most\fixnote{or least?} significant bit, resulting in $e \oc b_1$;
  the rewriting sequence $e \oc i \reduces e \fuse b_1 \reduces e \oc b_1$ is entailed by this definition.
\item[$b_0 \defd b_1 \pmir i$]
  To increment a binary number ending in $b_0$, flip that bit to $b_1$;
  the entailed rewriting step is $\octx \oc b_0 \oc i \reduces \octx \oc b_1$.
\item[$b_1 \defd i \fuse b_0 \pmir i$]
  To increment a binary number ending in $b_1$, flip that bit to $b_0$ and carry the increment over to the more significant bits;
  the entailed rewriting sequence is $\octx \oc b_1 \oc i \reduces \octx \oc (i \fuse b_0) \reduces \octx \oc i \oc b_0$.
\end{description}
Comfortingly, $1+1 = 2$: that is, a computation
% $e \oc b_1 \oc i \Reduces \aval{e \oc b_1 \oc b_0}{2}$, namely:
$e \oc b_1 \oc i \Reduces e \oc i \oc b_0 \Reduces e \oc b_1 \oc b_0$ indeed exists.

\newthought{It should also} be possible to permit several increments at once, such as in $e \oc b_1 \oc i \oc i$.
We could, of course, handle the increments sequentially from left to right, fully computing a binary value before moving on to the subsequent increment:
\begin{equation*}
  e \oc b_1 \oc i \oc i \Reduces e \oc b_1 \oc b_0 \oc i \reduces e \oc b_1 \oc b_1
  \,.
\end{equation*}
However, a strictly sequential treatment of increments would be rather disappointing.
Because the ordered rewriting framework\fixnote{wc?} is inherently concurrent, a truly concurrent treatment of multiple increments would be far more satisfying.

For example, consider the several computations of $(1+1)+1 = 3$ from $e \oc b_1 \oc i \oc i$:
\begin{equation*}
  \begin{tikzcd}[
    cells={inner xsep=0.65ex,
           inner ysep=0.4ex},
    row sep=0em,
    column sep=scriptsize
  ]
    &[-0.2em]
    e \oc b_1 \oc b_0 \oc i
      \drar[reduces, start anchor=base east,
                     end anchor=north west]
    &[-0.2em]
    \\
    e \oc b_1 \oc i \oc i \Reduces e \oc i \oc b_0 \oc i
      \urar[Reduces, start anchor=north east,
                     end anchor=base west]
      \ar[Reduces, gray, dashed]{rr}
      \drar[reduces, start anchor=base east,
                     end anchor=west]
    &&
    e \oc b_1 \oc b_1
    \\
    &
    e \oc i \oc b_1
      \urar[Reduces, start anchor=east,
                     end anchor=base west]
    &
  \end{tikzcd}
\end{equation*}
In other words, once the leftmost increment is carried past the least significant bit, the two increments can be processed concurrently -- the increments' rewriting steps can be interleaved, with no observable difference between the various interleavings.
We can even abstract from the interleavings by writing simply $e \oc i \oc b_0 \oc i \Reduces e \oc b_1 \oc b_1$.

Unfortunately, a concurrent treatment of increments falls outside the domain of \cref{??}.
Intermediate computational states, such as ...,
\begin{equation*}
  e \oc i \oc b_0 \oc i \reduces e \oc i \oc b_1
\end{equation*}
because $e \oc i \oc b_0$ is simply not a binary value.
An adequacy theorem stronger than \cref{??} is needed.

The situation here is roughly analogous to the desire, in a functional language, for stronger metatheorems than a big-step, natural sematics admits, and we adopt a similar solution.

\newthought{To this end}, we define a binary relation, $\ainc{}{}$, that assigns a natural number denotation to each intermediate computational state, not only to the terminal values as $\aval{}{}$ did..%
\footnote{Like the $\aval{}{}$ relation does for values, the $\ainc{}{}$ relation also serves to implicitly characterize the valid intermediate states as those contexts that form the relation's domain of definition.
As with values, the valid intermediate states could also be enumerated more explicitly and syntactically with a grammar:
\begin{equation*}
  \octx \Coloneqq e \mid \octx \oc b_0 \mid \octx \oc b_1 \mid \octx \oc i \mid e \fuse b_1 \mid \octx \oc (i \fuse b_0)
\end{equation*}}%
%
\newcommand{\aincrules}{%
  \infer[\jrule{$e$-I}]{\ainc{e}{0}}{}
  \and
  \infer[\jrule{$b_0$-I}]{\ainc{\octx \oc b_0}{2n}}{
    \ainc{\octx}{n}}
  \and
  \infer[\jrule{$b_1$-I}]{\ainc{\octx \oc b_1}{2n+1}}{
    \ainc{\octx}{n}}
  \and
  \infer[\jrule{$i$-I}]{\ainc{\octx \oc i}{n+1}}{
    \ainc{\octx}{n}}
  \\
  \infer[\jrule{$\fuse_1$-I}]{\ainc{e \fuse b_1}{1}}{}
  \and
  \infer[\jrule{$\fuse_2$-I}]{\ainc{\octx \oc (i \fuse b_0)}{2(n+1)}}{
    \ainc{\octx}{n}}%
}%
%
\begin{inferences}
  \aincrules
\end{inferences}
Binary values should themselves be valid, terminal computational states, so the first three rules are carried over from the $\aval{}{}$ relation.
The $\jrule{$i$-I}$ rule allows multiple increment instructions to be interspersed throughout the state.
Lastly, because the atomicity of ordered rewriting steps is very fine-grained, the $\jrule{$\fuse_1$-I}$ and $\jrule{$\fuse_2$-I}$ rules are needed to completely describe the valid intermediate states and their denotations.
For instance, the state $e \oc i$ first rewrites to the intermediate $e \fuse b_1$ before eventually rewriting to $e \oc b_1$; the state $\octx \oc (i \fuse b_0)$ has a similar status.

With this $\ainc{}{}$ relation in hand, we can now prove a stronger, small-step adequacy theorem.
%
\newcommand{\smallincadequacytheorem}{%
\begin{theorem}[Small-step adequacy of increments]%
  \leavevmode
  \begin{thmdescription}
  \item[Value soundness]
    If $\aval{\octx}{n}$, then $\ainc{\octx}{n}$ and $\octx \nreduces$.
  \item[Preservation]
    If $\ainc{\octx}{n}$ and $\octx \reduces \octx'$, then $\ainc{\octx'}{n}$.
  \item[Progress]
    If $\ainc{\octx}{n}$, then either
    \begin{itemize*}[
      mode=unboxed, label=, afterlabel=,
      before=\unskip:\space,
      itemjoin=;\space, itemjoin*=; or\space%
    ]
    \item $\octx \reduces \octx'$ for some $\octx'$; or
    \item $\aval{\octx}{n}$.\fixnote{Compare with \enquote{If $\ainc{\octx}{n}$, then $\aval{\octx}{n}$ if, and only if, $\octx \nreduces$.}}
    \end{itemize*}
  \item[Termination]
    If $\ainc{\octx}{n}$, then every rewriting sequence from $\octx$ is finite.
  \end{thmdescription}
\end{theorem}%
}%
%
\smallincadequacytheorem
\begin{proof}
  Each part is proved separately.
  \begin{description}[
    parsep=0pt, listparindent=\parindent,
    labelsep=0.35em
  ]
  \item[Value soundness]
    can be proved by structural induction on the derivation of $\aval{\octx}{n}$.
  \item[Preservation and progress]
    can likewise be proved by structural induction on the derivation of $\ainc{\octx}{n}$.
    In particular, the $e \fuse b_1$ and $\octx \oc (i \fuse b_0)$ rules
  \item[Termination]
    can be proved using an explicit termination measure, $\card{\octx}$, that is strictly decreasing across each rewriting, $\octx \reduces \octx'$.
    Specifically, we use a measure (see the adjacent \lcnamecref{fig:ordered-rewriting:binary-counter:measure}),
    % For valid states $\octx$, we define a measure $\card{\octx}$ that is strictly decreasing across each rewriting $\octx \reduces \octx'$ (see the adjacent \lcnamecref{fig:ordered-rewriting:binary-counter:measure}).
    \begin{marginfigure}
      \begin{equation*}
        \begin{lgathered}[t]
          \card{e} = 0 \\
          \card{\octx \oc b_0} = \card{\octx} \\
          \card{\octx \oc b_1} = \card{\octx} + 2 \\
          \card{\octx \oc i} = \card{\octx} + 4
        \end{lgathered}
        \qquad
        \begin{lgathered}[t]
          \card{e \fuse b_1} = 3 \\
          \card{\octx \oc (i \fuse b_0)} = \card{\octx} + 5
        \end{lgathered}
      \end{equation*}
      \caption{A termination measure, adapted from the standard amortized work analysis of increment for binary counters}\label{fig:ordered-rewriting:binary-counter:measure}
    \end{marginfigure}%
    adapted from the standard amortized work analysis of increment for binary counters\autocite{??}, for which $\octx \reduces \octx'$ implies $\card{\octx} > \card{\octx'}$.
    % That is, if $\octx$ is a valid state and $\octx \reduces \octx'$, then $\card{\octx} > \card{\octx'}$.
    Because the measure is always nonnegative, only finitely many such rewritings can occur.

    As an example case, consider the intermediate state $\octx \oc b_0 \oc i$ and its rewriting $\octx \oc b_0 \oc i \reduces \octx \oc b_1$.
    It follows that $\card{\octx \oc b_0 \oc i} = \card{\octx} + 4 > \card{\octx} + 2 = \card{\octx \oc b_1}$.
  \qedhere
  \end{description}
\end{proof}

\begin{corollary}[Big-step adequacy of increments]
  \leavevmode
  \begin{thmdescription}
  \item[Evaluation]
    If $\ainc{\octx}{n}$, then $\octx \Reduces\aval{}{n}$.
    In particular, if $\aval{\octx}{n}$, then $\octx \oc i \Reduces\aval{}{n+1}$.
  \item[Preservation]
    If $\ainc{\octx}{n}$ and $\octx \Reduces \octx'$, then $\ainc{\octx'}{n}$.
    In particular, if $\aval{\octx}{n}$ and $\octx \oc i \Reduces\aval{}{n'}$, then $n' = n+1$.
  \end{thmdescription}
\end{corollary}
\begin{proof}
  The two parts are proved separately.
  \begin{description}[labelsep=0.35em]
  \item[Evaluation] can be proved by repeatedly appealing to the progress and preservation results\parencref{??}.
    By the accompanying termination result, a binary value must eventually be reached.
  \item[Preservation] can be proved by structural induction on the given rewriting sequence.
  %
  \qedhere
  \end{description}
\end{proof}

\newthought{But, of course}, a few isolated examples do not make a proof.



By analogy with functional programming, the above adequacy conditions can be seen as stating evaluation and termination results for a big-step, evaluation semantics of increments, with $\aval{\octx}{n}$ acting as a kind of typing judgment -- admittedly, a very precise one.

In functional programming, big-step results like these are usually proved by first providing a small-step operational semantics, then characterizing the valid intermediate states that arise with small steps, and finally establishing type preservation, progress, and termination results for the small-step semantics.
We will adopt the same proof strategy here.

In this case, the small-step operational semantics already exists -- it is simply the individual rewriting steps entailed by the definitions of $e$, $b_0$, and $b_1$.
So our first task is to characterize the valid intermediate states that arise during a computation.
To this end, we define a binary relation, $\ainc{}{}$, that, like the $\aval{}{}$ relation, serves the dual purposes of enumerating the valid intermediate states and assigning to each state a natural number denotation.%
\footnote{As with values, we could also choose to enumerate the valid immediate states more explicitly and syntactically with a grammar:
  \begin{equation*}
    \octx \Coloneqq e \mid \octx \oc b_0 \mid \octx \oc b_1 \mid \octx \oc i \mid e \fuse b_1 \mid \octx \oc (i \fuse b_0)
  \end{equation*}}
% To this end, we define a binary relation, $\ainc{}{}$, between computational states and the natural numbers that they represent;
\begin{inferences}
  \infer{\ainc{e}{0}}{}
  \and
  \infer{\ainc{\octx \oc b_0}{2n}}{
    \ainc{\octx}{n}}
  \and
  \infer{\ainc{\octx \oc b_1}{2n+1}}{
    \ainc{\octx}{n}}
  \and
  \infer{\ainc{\octx \oc i}{n+1}}{
    \ainc{\octx}{n}}
  \\
  \infer{\ainc{e \fuse b_1}{1}}{}
  \and
  \infer{\ainc{\octx \oc (i \fuse b_0)}{2(n+1)}}{
    \ainc{\octx}{n}}
\end{inferences}
Binary values should themselves be valid, terminal computational states, so the first three rules are carried over from the $\aval{}{}$ relation.
The fourth rule, involving $i$, allows multiple increments to be interspersed throughout the counter.

Because ordered rewriting steps are quite fine-grained, two final rules are needed to completely describe the valid intermediate states and their denotations.
For instance, the state $e \oc i$ first rewrites to $e \fuse b_1$ before eventually rewriting to $e \oc b_1$.



Having characterized the valid intermediate states, we may state and prove the small-step adequacy of increments: preservation, progress, and termination.
%
\begin{theorem}[Small-step adequacy of increments]%
  \leavevmode
  \begin{thmdescription}[nosep]
  \item[Value inclusion]
    If $\aval{\octx}{n}$, then $\ainc{\octx}{n}$.
  \item[Preservation]
    If $\ainc{\octx}{n}$ and $\octx \reduces \octx'$, then $\ainc{\octx'}{n}$.
  \item[Progress]
    If $\ainc{\octx}{n}$, then either
    \begin{itemize*}[
      mode=unboxed, label=, afterlabel=,
      before=\unskip:\space,
      itemjoin=;\space, itemjoin*=; or\space%
    ]
    \item $\octx \reduces \octx'$ for some $\octx'$; or
    \item $\octx \nreduces$ and $\aval{\octx}{n}$.
    \end{itemize*}
  \item[Termination]
    If $\ainc{\octx}{n}$, then every rewriting sequence from $\octx$ is finite.
  \end{thmdescription}
\end{theorem}
%
\begin{proof}
  Each part is proved separately.
  \begin{description}[
    parsep=0pt, listparindent=\parindent,
    labelsep=0.35em
  ]
  \item[Value inclusion]
    can be proved by structural induction on the derivation of $\aval{\octx}{n}$.
  \item[Preservation and progress]
    can likewise be proved by structural induction on the derivation of $\ainc{\octx}{n}$.
    In particular, the $e \fuse b_1$ and $\octx \oc (i \fuse b_0)$ rules
  \item[Termination]
    can be proved using an explicit termination measure, $\card{\octx}$, that is strictly decreasing across each rewriting, $\octx \reduces \octx'$.
    Specifically, we use a measure (see the adjacent \lcnamecref{fig:ordered-rewriting:binary-counter:measure}),
    % For valid states $\octx$, we define a measure $\card{\octx}$ that is strictly decreasing across each rewriting $\octx \reduces \octx'$ (see the adjacent \lcnamecref{fig:ordered-rewriting:binary-counter:measure}).
    \begin{marginfigure}
      \begin{equation*}
        \begin{lgathered}[t]
          \card{e} = 0 \\
          \card{\octx \oc b_0} = \card{\octx} \\
          \card{\octx \oc b_1} = \card{\octx} + 2 \\
          \card{\octx \oc i} = \card{\octx} + 4
        \end{lgathered}
        \qquad
        \begin{lgathered}[t]
          \card{e \fuse b_1} = 3 \\
          \card{\octx \oc (i \fuse b_0)} = \card{\octx} + 5
        \end{lgathered}
      \end{equation*}
      \caption{A termination measure, adapted from the standard amortized work analysis of increment for binary counters}\label{fig:ordered-rewriting:binary-counter:measure}
    \end{marginfigure}%
    adapted from the standard amortized work analysis of increment for binary counters\autocite{??}, for which $\octx \reduces \octx'$ implies $\card{\octx} > \card{\octx'}$.
    % That is, if $\octx$ is a valid state and $\octx \reduces \octx'$, then $\card{\octx} > \card{\octx'}$.
    Because the measure is always nonnegative, only finitely many such rewritings can occur.

    As an example case, consider the intermediate state $\octx \oc b_0 \oc i$ and its rewriting $\octx \oc b_0 \oc i \reduces \octx \oc b_1$.
    It follows that $\card{\octx \oc b_0 \oc i} = \card{\octx} + 4 > \card{\octx} + 2 = \card{\octx \oc b_1}$.
  \qedhere
  \end{description}
\end{proof}

\begin{theorem}[Big-step adequacy of increments]\leavevmode
  \begin{thmdescription}
  \item[Preservation]
    If $\ainc{\octx}{n}$ and $\octx \Reduces\aval{}{n'}$, then $n' = n$.
  \item[Termination?]
    If $\ainc{\octx}{n}$, then $\octx \Reduces\aval{}{n}$.
  \end{thmdescription}
\end{theorem}
\begin{proof}
  Both parts are consequences of the small-step adequacy of increments \parencref{??}.
  \begin{description}
  \item[Preservation]
    is proved by structural induction on the given rewriting sequence.
    The base case follows [...] by an inner structural induction on the derivation of $\aval{\octx}{n'}$.
    The inductive case can be proved by first appealing to small-step preservation \parencref{??} and then to the inductive hypothesis.
  \item[Termination?]
    is proved by repeatedly appealing to small-step progress \parencref{??}.
    The small-step termination [...] \parencref{??} ensures that a value will be reached after finitely many such appeals.
  %
  \qedhere
  \end{description}
\end{proof}

\begin{corollary}[Structural adequacy of increments]
  If $\aval{\octx}{n}$, then $\octx \oc i \Reduces\aval{}{n'}$ if, and only if, $n' = n+1$.
\end{corollary}



% For example, incrementing [a representation of] $1$ should result in [a representation of] $2$, as evidenced by a trace $e \oc b_1 \oc i \Reduces e \oc b_1 \oc b_0$.

% Given a binary number $\octx$ that represents $n$, we may append $i$ to form a computational state, $\octx \oc i$, that should compute a binary representation of $n+1$ and thereby increment the number.
% For example, incrementing [a representation of] $1$ should result in [a representation of] $2$, as evidenced by a trace $e \oc b_1 \oc i \Reduces e \oc b_1 \oc b_0$.

% Conversely, any computation

% To describe\fixnote{implement?} the increment operation using ordered rewriting, the previously uninterpreted atoms $e$, $b_0$, and $b_1$ are now given mutually recursive definitions that describe how they may be rewritten when $i$ is encountered.
% \begin{description}[font=\color{structure}]
% \item[$e \defd e \fuse b_1 \pmir i$]
%   To increment $e$, append $b_1$ as a most significant bit, resulting in $e \oc b_1$.
% \item[$b_0 \defd b_1 \pmir i$]
%   To increment a binary number that has $b_0$ as its least significant bit, simply flip that bit to $b_1$.
% \item[$b_1 \defd i \fuse b_0 \pmir i$]
%   To increment a binary number that has $b_1$ as its least significant bit, flip that bit to $b_0$ and carry the increment over to the more significant bits.
% \end{description}


As an example computation, consider incrementing $e \oc b_1$ twice, as captured by the state $e \oc b_1 \oc i \oc i$.
\begin{equation*}
  \begin{tikzcd}[
    cells={inner xsep=0.65ex,
           inner ysep=0.4ex},
    row sep=0em,
    column sep=scriptsize
  ]
    &[-0.2em]
    e \oc b_1 \oc b_0 \oc i
      \drar[reduces, start anchor=base east,
                     end anchor=west]
    &[-0.2em]
    \\
    e \oc b_1 \oc i \oc i \Reduces e \oc i \oc b_0 \oc i
      \urar[Reduces, start anchor=east,
                     end anchor=base west]
      \drar[reduces, start anchor=base east,
                     end anchor=west]
    &&
    e \oc b_1 \oc b_1
    \\
    &
    e \oc i \oc b_1
      \urar[Reduces, start anchor=east,
                     end anchor=base west]
    &
  \end{tikzcd}
\end{equation*}

First, processing of the leftmost increment begins: the least significant bit is flipped, and the increment is carried over to the more significant bits.
This corresponds to the reduction $e \oc b_1 \oc i \oc i \Reduces e \oc i \oc b_0 \oc i$.
Next, either of the two remaining increments may be processed -- that is, either $e \oc i \oc b_0 \oc i \Reduces e \oc b_1 \oc b_0 \oc i$ or $e \oc i \oc b_0 \oc i \Reduces e \oc i \oc b_1$.


% Conversely, any complete computation from $\octx \oc i$ must have as its result a binary rrepresentation of $n+1$.

% These two properties ensure that the atom $i$ adequately characterizes an increment operation:
% \begin{itemize}
% \item 
% \end{itemize}

% For this to be an adequate description of an increment operation, it should satisfy two desiderata:
% \begin{enumerate*}
% \item
% \end{enumerate*}
% Formally, these desiderata are captured by the following adequacy theorem:
% \begin{itemize}
% \item If $\aval{\octx}{n}$, then $\octx \oc i \Reduces\aval{}{n+1}$.
% \item If $\aval{\octx}{n}$, then $\octx \oc i \Reduces\aval{}{n'}$ implies $n' = n+1$.
% \end{itemize}

% By analogy with functional programming, ...

% Appending $i$ to a counter will initiate an increment, with ordered rewriting used to compute, step by step, a binary representation of the incremented value.



% \begin{itemize}
% \item
%   If counter $\octx$ represents $n$, then $\octx \oc i$ can compute a representation of $n+1$ and, conversely, any computation from $\octx \oc i$ results in a representation of $n+1$.
%   That is, if $\aval{\octx}{n}$, then $\octx \oc i \Reduces\aval{}{n'}$ if, and only if, $n' = n+1$.
% \end{itemize}

% \begin{equation*}
%   \octx \Coloneqq e \mid \octx \oc b_0 \mid \octx \oc b_1 \mid \octx \oc i
% \end{equation*}
% The binary counters

% If counter $\octx$ represents $n$, then $\octx \oc i$ should compute to a representation of $n+1$.
% If $\octx$ represents $n$ and $\octx \oc i$ can reduce to $n'$, then $n' = n+1$.
% \begin{itemize}
% \item 
% \end{itemize}

% The basic idea is to assign $e$, $b_0$, and $b_1$ recursive definitions that enable them to interact with these atoms $i$.

% The basic idea is to This atom is appended to a counter to initiate an increment 
% By appending this atom to a counter, This atom is appended to a counter
% ordered rewriting of 
% Because of these increments,
% To initiate an increment of a counter $\octx$, we simply append an uninterpreted atom $i$ to the counter; the atom $i$
% %
% % \begin{desiderata*}[Computational adequacy -- increments]\label{des:ordered-rewriting:increments}\leavevmode
%   \begin{itemize}[noitemsep]
%   \item If $\aval{\octx}{n}$ and $\octx \oc i \Reduces\aval{}{n'}$, then $n' = n+1$.
%   \item In addition, if $\aval{\octx}{n}$, then $\octx \oc i \Reduces\aval{}{n+1}$.
%   \end{itemize}
% % \end{desiderata*}



% Because of the new increment operation, the previously uninterpreted atoms $e$, $b_0$, and $b_1$ are now given mutually recursive definitions that describe how they may be rewritten when encountering $i$:
% % \begin{equation*}
% %   \begin{lgathered}
% %     e \defd e \fuse b_1 \pmir i \\
% %     b_0 \defd b_1 \pmir i \\
% %     b_1 \defd i \fuse b_0 \pmir i
% %   \end{lgathered}
% % \end{equation*}
% \begin{description}[font=\color{structure}]
% \item[$e \defd e \fuse b_1 \pmir i$]
%   To increment $e$, append $b_1$ as a most significant bit, resulting in $e \oc b_1$.
% \item[$b_0 \defd b_1 \pmir i$]
%   To increment a binary number that has $b_0$ as its least significant bit, simply flip that bit to $b_1$.
% \item[$b_1 \defd i \fuse b_0 \pmir i$]
%   To increment a binary number that has $b_1$ as its least significant bit, flip that bit to $b_0$ and carry the increment over to the more significant bits.
% \end{description}

% \begin{description}
% \item[$e \defd e \fuse b_1 \pmir i$]
%   To increment the counter $e$, which represents $0$, introduce $b_1$ as a new most significant bit, resulting in the counter $e \oc b_1$, which represents $1$.
%   That is, because $\aval{e}{0}$, there should exist a trace $e \oc i \Reduces \aval{e \oc b_1}{1}$.
%   % Having started at value $0$ (\ie, $\aval{e}{0}$), an increment results in value $1$ (\ie, $\aval{e \oc b_1}{1}$).
% \item[$b_0 \defd b_1 \pmir i$]
%   Because $\aval{\octx \oc b_0}{2n}$ when $\aval{\octx}{n}$, there should exist a trace $\octx \oc b_0 \oc i \Reduces \aval{\octx \oc b_1}{2n+1}$.
%   To increment a counter that ends with least significant bit $b_0$, simply flip that bit to $b_1$.
%   That is, $\octx \oc b_0 \oc i \reduces \octx \oc b_1$.
%   % Having started at value $2n$ (\ie, $\aval{\octx \oc b_0}{2n}$ with $\aval{\octx}{n}$), an increment results in value $2n+1$ (\ie, $\aval{\octx \oc b_1}{2n+1}$).
% \item[$b_1 \defd i \fuse b_0 \pmir i$]
%   Because $\aval{\octx \oc b_1}{2n+1}$ when $\aval{\octx}{n}$, there should exist a trace $\octx \oc b_1 \oc i \Reduces \aval{\octx' \oc b_0}{2n+2}$, provided that there exists a trace $\octx \oc i \Reduces \aval{\octx'}{n+1}$.
%   To increment a counter that ends with least significant bit $b_1$, flip that bit to $b_0$ and propagate the increment to the more significant bits as a carry.
%   That is, $\octx \oc b_1 \oc i \Reduces \octx \oc i \oc b_0$.
%   % Having started at value $2n+1$ (\ie, $\cval{\octx \oc b_1} = 2\cval{\octx}+1$), an increment results in value $2n+2 = 2(n+1)$ (\ie, $\cval{\octx \oc i \oc b_0} = 2\cval{\octx}+1$).
% \end{description}

% As an example computation, consider incrementing $e \oc b_1$ twice, as captured by the state $e \oc b_1 \oc i \oc i$.
% \begin{equation*}
%   \begin{tikzcd}[
%     cells={inner xsep=0.65ex,
%            inner ysep=0.4ex},
%     row sep=0em,
%     column sep=scriptsize
%   ]
%     &[-0.2em]
%     e \oc b_1 \oc b_0 \oc i
%       \drar[reduces, start anchor=base east,
%                      end anchor=west]
%     &[-0.2em]
%     \\
%     e \oc b_1 \oc i \oc i \Reduces e \oc i \oc b_0 \oc i
%       \urar[Reduces, start anchor=east,
%                      end anchor=base west]
%       \drar[reduces, start anchor=base east,
%                      end anchor=west]
%     &&
%     e \oc b_1 \oc b_1
%     \\
%     &
%     e \oc i \oc b_1
%       \urar[Reduces, start anchor=east,
%                      end anchor=base west]
%     &
%   \end{tikzcd}
% \end{equation*}

% First, processing of the leftmost increment begins: the least significant bit is flipped, and the increment is carried over to the more significant bits.
% This corresponds to the reduction $e \oc b_1 \oc i \oc i \Reduces e \oc i \oc b_0 \oc i$.
% Next, either of the two remaining increments may be processed -- that is, either $e \oc i \oc b_0 \oc i \Reduces e \oc b_1 \oc b_0 \oc i$ or $e \oc i \oc b_0 \oc i \Reduces e \oc i \oc b_1$.


We should like to prove the correctness of this specification of increments by establishing a computational adequacy result:
%
\begin{theorem}[Adequacy of increments]\label{thm:ordered-rewriting:binary-counter:inc-adequacy}
  If $\aval{\octx}{n}$ and $\octx \oc i \Reduces\aval{}{n'}$, then $n' = n+1$.
  Moreover, if $\aval{\octx}{n}$, then $\octx \oc i \Reduces\aval{}{n+1}$.
\end{theorem}
%
By analogy with functional programming, this \lcnamecref{thm:ordered-rewriting:binary-counter:inc-adequacy} can be seen as stating evaluation and termination results for a big-step evaluation semantics of increments --
the judgment $\aval{\octx}{n}$ is acting as a kind of typing judgment, with $n$ being the \enquote{type} [abstract interpretation?] of the counter $\octx$.

In functional programming, these sorts of big-step results are proved by first providing a small-step operational semantics, then characterizing the valid intermediate states that arise with small steps, and finally establishing type preservation, progress, and termination results for the small-step semantics.
We will adopt the same strategy here.

First, we define a relation, $\ainc{}{}$, that characterizes the valid intermediate states that arise during increments.

To prove this \lcnamecref{thm:ordered-rewriting:increments}, we will first introduce an auxiliary relation, $\ainc{}{}$, that characterizes the valid states that arise during increments.
This relation is defined inductively by the following rules.
%
\begin{inferences}
  \infer{\ainc{e}{0}}{}
  \and
  \infer{\ainc{\octx \oc b_0}{2n}}{
    \ainc{\octx}{n}}
  \and
  \infer{\ainc{\octx \oc b_1}{2n+1}}{
    \ainc{\octx}{n}}
  \and
  \infer{\ainc{\octx \oc i}{n+1}}{
    \ainc{\octx}{n}}
  \\
  \infer{\ainc{e \fuse b_1}{1}}{}
  \and
  \infer{\ainc{\octx \oc (i \fuse b_0)}{2(n+1)}}{
    \ainc{\octx}{n}}
\end{inferences}
The latter two

% \begin{lemma}[Value inclusion]
%   If $\aval{\octx}{n}$, then $\ainc{\octx}{n}$.
% \end{lemma}
% %
% \begin{proof}
%   By structural induction on the derivation of $\aval{\octx}{n}$.
% \end{proof}

% \begin{theorem}[Preservation]
%   If $\ainc{\octx}{n}$ and $\octx \reduces \octx'$, then $\ainc{\octx'}{n}$.
% \end{theorem}
% %
% \begin{proof}
%   By structural induction on the derivation of $\ainc{\octx}{n}$.
%   % We will show a representative cases.
%   % \begin{itemize}
%   % % \item Consider the case in which
%   % %   \begin{equation*}
%   % %     \octx
%   % %     =
%   % %     \infer{\ainc{\octx_0 \oc i}{n_0+1}}{
%   % %       \ainc{\octx_0}{n_0}}
%   % %     =
%   % %     n
%   % %   \end{equation*}
%   % %   and $\octx = \octx_0 \oc i \reduces \octx'_0 \oc i = \octx'$ because $\octx_0 \reduces \octx'_0$, for some $\octx_0$, $\octx'_0$, and $n_0$.
%   % %   By the inductive hypothesis, $\ainc{\octx'_0}{n_0}$.
%   % %   And so, $\octx' = \ainc{\octx'_0 \oc i}{n_0+1} = n$, as required.
%   % 
%   % % \item Consider the case in which
%   % %   \begin{equation*}
%   % %     \octx
%   % %     =
%   % %     \infer{\ainc{\octx_0 \oc b_1 \oc i}{(2n_0+1)+1}}{
%   % %       \infer{\ainc{\octx_0 \oc b_1}{2n_0+1}}{
%   % %         \ainc{\octx_0}{n_0}}}
%   % %     =
%   % %     n
%   % %   \end{equation*}
%   % %   and $\octx = \octx_0 \oc b_1 \oc i \reduces \octx_0 \oc (i \fuse b_0) = \octx'$, for some $\octx_0$ and $n_0$.
%   % %   It immediately follows that $\octx' = \ainc{\octx_0 \oc (i \fuse b_0)}{2(n_0+1)} = 2n_0+2 = n$, as required.
%   % 
%   % \item Consider the case in which
%   %   \begin{equation*}
%   %     \octx
%   %     =
%   %     \infer{\ainc{\octx_0 \oc (i \fuse b_0)}{2(n_0+1)}}{
%   %       \ainc{\octx_0}{n_0}}
%   %     =
%   %     n
%   %   \end{equation*}
%   %   and $\octx = \octx_0 \oc (i \fuse b_0) \reduces \octx_0 \oc i \oc b_0 = \octx'$, for some $\octx_0$ and $n_0$.
%   %   It immediately follows that
%   %   \begin{equation*}
%   %     \octx'
%   %     =
%   %     \infer{\ainc{\octx_0 \oc i \oc b_0}{2(n_0+1)}}{
%   %       \infer{\ainc{\octx_0 \oc i}{n_0+1}}{
%   %         \ainc{\octx_0}{n_0}}}
%   %     =
%   %     n
%   %     \,,
%   %   \end{equation*}
%   %   as required.
%   % \qedhere
%   % \end{itemize}
% \end{proof}


% \begin{theorem}[Progress]
%   If $\ainc{\octx}{n}$, then either $\octx \reduces \octx'$ or $\aval{\octx}{n}$.
% \end{theorem}
% %
% \begin{proof}
%   By structural induction on the derivation of $\ainc{\octx}{n}$.
%   % \begin{itemize}
%   % \item Consider the case in which
%   %   \begin{equation*}
%   %     \octx
%   %     =
%   %     \infer{\ainc{\octx_0 \oc i}{n_0+1}}{
%   %       \ainc{\octx_0}{n_0}}
%   %     =
%   %     n
%   %   \end{equation*}
%   %   for some $\octx_0$ and $n_0$.
%   % \end{itemize}
% \end{proof}

% Because rewriting is nondeterministic, we cannot take \enquote{$\ainc{\octx}{n}$ implies $\octx \Reduces\aval{}{n}$} as a statement of termination.

% \begin{theorem}[Termination]
%   If $\ainc{\octx}{n}$, then there is no infinite rewriting of $\octx$.
% \end{theorem}
% %
% \begin{proof}
%   For valid states $\octx$, we define a measure $\card{\octx}$ that is strictly decreasing across each rewriting $\octx \reduces \octx'$ (see the adjacent \lcnamecref{fig:ordered-rewriitting:binary-counter:measure}).
%   \begin{marginfigure}
%     \begin{equation*}
%       \begin{lgathered}
%         \card{e} = 0 \\
%         \card{\octx \oc b_0} = \card{\octx} \\
%         \card{\octx \oc b_1} = \card{\octx} + 2 \\
%         \card{\octx \oc i} = \card{\octx} + 4 \\
%         \card{e \fuse b_1} = 3 \\
%         \card{\octx \oc (i \fuse b_0)} = \card{\octx} + 5
%       \end{lgathered}
%     \end{equation*}
%   \end{marginfigure}%
%   That is, if $\octx$ is a valid state and $\octx \reduces \octx'$, then $\card{\octx} > \card{\octx'}$.
%   Because the measure is nonnegative, only finitely many such rewrittings can occour. 

%   As an example case, consider the valid state $\octx \oc b_0 \oc i$ and its rewritting  $\octx \oc b_0 \oc i \reduces \octx \oc b_1$.
%   It follows from the definition that $\card{\octx \oc b_0 \oc i} = \card{\octx} + 4 > \card{\octx} + 2 = \card{\octx \oc b_1}$.
% \end{proof}


% %
% \begin{proof}[Counterexample]
%   Small-step preservation does \emph{not} hold for $\ainc{}{}$.
%   As a specific counterexample, notice that $\ainc{\octx \oc b_1 \oc i}{2n+2}$ and $\octx \oc b_1 \oc i \reduces \octx \oc (i \fuse b_0)$, but $\ainc{\octx \oc (i \fuse b_0) \not}{2n+2}$.
%   Similarly, $\ainc{e \oc i}{1}$ and $e \oc i \reduces e \fuse b_1$, but $\ainc{e \fuse b_1 \not}{1}$.
% \end{proof}

% \begin{theorem}[Big-step preservation]
%   If $\ainc{\octx}{n}$ and $\octx \Reduces \ainc{\octx'}{n'}$, then $n = n'$.
% \end{theorem}


%  Consider the case in which $\octx = \octx_0 \oc b_1 \oc i \reduces \octx_0 \oc (i \fuse b_0) \Reduces \ainc{\octx'}{n'}$ and $n = 2n_0+2$ for some $\octx_0$ and $ n_0$ such that $\ainc{\octx_0}{n_0}$.
%     By inversion, $\octx_0 \oc i \oc b_0 \Reduces \ainc{\octx'}{n'}$.

% \begin{theorem}[Big-step evaluation]
%   If $\ainc{\octx}{n}$, then $\octx \Reduces \aval{\octx'}{n}$.
% \end{theorem}
% %
% \begin{proof}
%   By nested innduction, first on the natural number $n$ and then on the context $\octx$.
%   \begin{itemize}
%   \item Consider the case in which $\octx = \octx_0 \oc b_1 \oc i$ and $n = 2n_0+2$ for some $\octx_0$ and $ n_0$ such that $\ainc{\octx_0}{n_0}$.
%     By the inductive hypothesis, $\octx_0 \Reduces \aval{\octx'_0}{n_0}$, for some $\octx'_0$.
%     Notice that $\octx'_0 \oc b_1 \oc i \Reduces \octx'_0 \oc i \oc b_0$.
%     By the inductive hypothesis again, $\octx'_0 \oc i \Reduces \aval{\octx''_0}{n_0+1}$.
%     Framing $b_0$ on to the right, $\octx \Reduces \octx'_0 \oc b_1 \oc i \Reduces \octx'_0 \oc i \oc b_0 \Reduces \aval{\octx''_0 \oc b_0}{2(n_0+1)} = n$.

%   \item Consider the case in which $\octx = \octx_0 \oc b_0$ and $n = 2n_0$ for some $\octx_0$ and $n_0$ such that $\ainc{\octx_0}{n_0}$.
%     By the inductive hypothesis, $\octx_0 \Reduces \aval{\octx'_0}{n_0}$ for some $\octx'_0$.
%     Framing $b_0$ on to the right, $\octx = \octx_0 \oc b_0 \Reduces \aval{\octx'_0 \oc b_0}{2n_0} = n$.

%   \item Consider the case in which $\octx = e \oc i$ and $n = 1$.
%     It follows that $\octx = e \oc i \Reduces \aval{e \oc b_1}{1} = n$.
%   \end{itemize}
% \end{proof}

% \begin{theorem}[Big-step determinism]
%   If $\ainc{\octx}{n}$, then $\octx \Reduces \aval{\octx'}{n}$.
% \end{theorem}


% To correct this, there are two choices.
% First, we could introduce the following rules.
% \begin{inferences}
%   \infer{\ainc{e \fuse b_1}{1}}{}
%   \and
%   \infer{\ainc{\octx \oc (i \fuse b_0)}{2n+2}}{
%     \ainc{\octx}{n}}
% \end{inferences}
% Second, we could prove a big-step preservation result:
% \begin{theorem}[Big-step preservation]
%   If $\ainc{\octx}{n}$ and $\octx \Reduces \ainc{\octx'}{n'}$, then $n = n'$.
% \end{theorem}
% %
% \begin{proof}
%   \begin{itemize}
%   \item Consider the case in which $\octx = e \oc i$ and $n = 1$ and $e \fuse b_1 \Reduces\ainc{}{n'}$.
%     By inversion, $e \fuse b_1 \reduces \ainc{e \oc b_1}{1} = n'$.
%   \item Consider the case in which $\octx = \octx_0 \oc b_0 \oc i$ and $n = 2n_0+1$ and $\octx_0 \oc b_1 \Reduces\ainc{}{n'}$ for some $\octx_0$ and $n_0$ such that $\ainc{\octx_0}{n_0}$.
%     Notice that $\ainc{\octx_0 \oc b_1}{2n_0+1}$, and so $n' = 2n_0+1 = n$, by the inductive hypothesis.
%   \item Consider the case in which $\octx = \octx_0 \oc b_1 \oc i$ and $n = 2n_0+2$ and $\octx_0 \oc (i \fuse b_0) \Reduces\ainc{}{n'}$ for some $\octx_0$ and $n_0$ such that $\ainc{\octx_0}{n_0}$.
%     By [...], $\octx_0 \oc i \oc b_0 \Reduces\ainc{}{n'}$.
%     Notice that $\ainc{\octx_0 \oc i \oc b_0}{2(n_0+1)}$, and so $n' = 2(n_0+1) = n$, by the inductive hypothesis.
%   \item Consider the case in which $\octx = \octx_0 \oc i$ and $n = n_0+1$ and $\octx_0 \reduces \octx'_0$ and $\octx'_0 \oc i \Reduces\ainc{}{n'}$ for some $\octx_0$, $\octx'_0$, and $n_0$ such that $\ainc{\octx_0}{n_0}$.
%   \end{itemize}
% \end{proof}


% \begin{theorem}[Preservation and progress]\leavevmode
%   \begin{description}[nosep]
% %  \item[Unicity] If $\ainc{\octx}{n}$ and $\ainc{\octx}{n'}$, then $n = n'$.
% %  \item[Preservation] If $\ainc{\octx}{n}$ and $\octx \Reduces \octx'$, then $\ainc{\octx'}{n}$.
%   \item[Weak preservation] If $\ainc{\octx}{n}$ and $\octx \Reduces \ainc{\octx'}{n'}$, then $n = '$.
% %  \item[Progress] If $\ainc{\octx}{n}$, then either $\octx \reduces \octx'$ or $\aval{\octx}{n}$.
%   \item[Termination] If $\ainc{\octx}{n}$, then $\octx \Reduces\aval{}{n}$.
%   \end{description}
% \end{theorem}
% %
% \begin{proof}
%   \begin{description}
%   \item[Termination]
%     Assume that $\ainc{\octx}{n}$; we must show that $\octx \Reduces\aval{}{n}$.
%     \begin{itemize}
%     \item Consider the case in which $\octx = \octx_0 \oc b_0$ and $n = 2n_0$ for some $\octx_0$ and $n_0$ such that $\ainc{\octx_0}{n_0}$.
%       By the inductive hypothesis, $\octx_0 \Reduces\aval{}{n_0}$.
%       It follows that $\octx = \octx_0 \oc b_0 \Reduces\aval{}{2n_0} = n$.

%     \item The case in which $\octx = \octx_0 \oc b_1$ and $n = 2n_0+1$ for some $\octx_0$ and $n_0$ such that $\ainc{\octx_0}{n_0}$ is analogous.

%     \item Consider the case in which $\octx = \octx_0 \oc b_0 \oc i$ and $n = 2n_0+1$ for some $\octx_0$ and $n_0$ such that $\ainc{\octx_0}{n_0}$.
%       By the inductive hypothesis, $\octx_0 \Reduces \aval{\octx'_0}{n_0}$ for some $\octx'_0$.
%       It follows that $\octx_0 \oc b_0 \oc i \Reduces \octx'_0 \oc b_0 \oc i \reduces \octx'_0 \oc b_1$, and moreover, $\aval{\octx'_0 \oc b_1}{2n_0+1}$.
%       So, indeed, $\octx \Reduces \aval{}{2n_0+1}$.

%     \item
%     \end{itemize}
%   \end{description}
% \end{proof}

\paragraph{A decrement operation}
Binary counters
% \newthought{These binary counters}
may also be equipped with a decrement operation.
Instead of examining decrements \emph{per se}, we will implement a closely related operation: the normalization of binary representations to what might be called \vocab{head-unary form}.%
\footnote{We will frequently abuse terminology, using \enquote*{head-unary normalization} and \enquote*{decrement} interchangably.}
An ordered context $\octx$ will be said to be in head-unary form if it has one of two forms: $\octx = z$; or $\octx = \octx' \oc s$, for some binary number $\octx'$.

Just as appending the atom $i$ to a counter initiates an increment, appending an uninterpreted atom $d$ will cause the counter to begin normalizing to head-unary form.
The following \lcnamecref{thm:decrement-adequacy} serves as a specification of head-unary normalization, relating a value's head-unary form to its denotation.
%
\begin{theorem}[Structural adequacy of decrements]
  If $\aval{\octx}{n}$, then:
  \begin{itemize}[nosep]
  \item $\octx \oc d \Reduces z$ if, and only if, $n=0$;
  \item $\octx \oc d \Reduces \octx' \oc s$ for some $\octx'$ such that $\aval{\octx'}{n-1}$, if $n > 0$; and
  \item $\octx \oc d \Reduces \octx' \oc s$ only if $n > 0$ and $\aval{\octx'}{n-1}$.
  \end{itemize}
\end{theorem}
%
\noindent
For example, because $e \oc b_1$ denotes $1$, a computation $e \oc b_1 \oc d \Reduces \octx' \oc s$ must exist, for some $\aval{\octx'}{0}$.

\newthought{Once again}, to implement these desiderata locally, the recursive definitions of $e$, $b_0$, and $b_1$ will be revised with an additional clause that handles decrements;
also, a recursively defined proposition $b'_0$ is introduced:
% 
% Similarly to the use of the atom $i$ to describe 
% Similarly to the way $i$ initiates increments, a decrement is triggered by appending an [uninterpreted] atom $d$ to the counter;
% $d$ is then processed from right to left by the counter's individual bits.
% To support this, the definitions of $e$, $b_0$, and $b_1$ are revised with an addition clause each:
\begin{description}[font=\color{structure}]
\item[$e \defd (\dotsb \pmir i) \with (z \pmir d)$]
  Because $e$ denotes $0$, its head-unary form is simply $z$.
  % Because $e$ denotes $0$, it may be put into head-unary form by replacing it with $z$.
\item[$b_0 \defd (\dotsb \pmir i) \with (d \fuse b'_0 \pmir d)$]
  Because $\octx \oc b_0$ denotes $2n$ if $\octx$ denotes $n$, its head-unary form can be contructed by recursively putting the more significant bits into head-unary form and appending $b'_0$ to process that result.
  % To put $\octx \oc b_0$ into head-unary form, recursively put the more significant bits into head-unary form and append $b'_0$ to process that result.
\item[$b'_0 \defd (z \limp z) \with (s \limp b_1 \fuse s)$]
  If the more significant bits have head-unary form $z$ and therefore denote $0$, then $\octx \oc b_0$ also denotes $0$ and has head-unary form $z$.
  Otherwise, if they have head-unary form $\octx' \oc s$ and therefore denote $n > 0$, then $\octx \oc b_0$ denotes $2n$ and has head-unary form $\octx' \oc b_1 \oc s$, which can be constructed by replacing $s$ with $b_1 \oc s$.
  % If the more significant bits, $\octx$, have $z$ as their head-unary form, then so does $\octx \oc b_0$; otherwise, if their head-unary form ends with $s$, then
\item[$b_1 \defd (\dotsb \pmir i) \with (b_0 \fuse s \pmir d)$]
  Because $\octx \oc b_1$ denotes $2n+1$ if $\octx$ denotes $n$, its head-unary form, $\octx \oc b_0 \oc s$, can be constructed by flipping the least significant bit to $b_0$ and appending $s$.
  % To put $\octx \oc b_1$ into head-unary form, decrement the least significant bit to $b_0$ and append $s$.
\end{description}
%
Comfortingly, $2-1 = 1$: the head-unary form of $e \oc b_1$ is $e \oc b_0 \oc b_1 \oc s$:
\begin{equation*}
  e \oc b_1 \oc b_0 \oc d \Reduces e \oc b_1 \oc d \oc b'_0 \Reduces e \oc b_0 \oc s \oc b'_0 \Reduces e \oc b_0 \oc b_1 \oc s
  \,.
\end{equation*}


\newthought{At this point}, we would like to prove the adequacy of decrements.
However, having just revised the definitions of $e$, $b_0$, and $b_1$, we must first recheck the adequacy of binary representation\parencref[see]{??}.
%
Unfortunately, the newly introduced alternative conjunctions, together with the fine-grained atomicity of ordered rewriting, cause [...].
%
\begin{falseclaim}[Adequacy of binary representations]%
  \leavevmode
  \begin{thmdescription}
  \item[Functional]
    For every binary number $\octx$, there exists a unique natural number $n$ such that $\aval{\octx}{n}$.
  \item[Surjectivity]
    For every natural number $n$, there exists a binary number $\octx$ such that $\aval{\octx}{n}$.
  \item[Values]
    If $\aval{\octx}{n}$, then $\octx \nreduces$.
  \end{thmdescription}
\end{falseclaim}
\begin{proof}[Counterexample]
  Although the $\aval{}{}$ relation remains functional and surjective, it does not satisfy [...].
  Because $\aval{e}{0}$, the counter $e$ is a value (with denotation $0$).
  However, because the atomicity of ordered rewriting is extremely fine-grained, $e$ can be rewritten:
  \begin{equation*}
    \begin{tikzcd}[
      cells={inner xsep=0.65ex,
             inner ysep=0.4ex},
      row sep=0em,
      column sep=scriptsize,
      /tikz/column 2/.append style={anchor=west}
    ]
      &[-0.2em] e \mathrlap{{} \fuse b_1 \pmir i}
      \\
      e = (e \fuse b_1 \pmir i) \with (z \pmir d)
        \urar[reduces, start anchor=east]
        \drar[reduces, start anchor=base east]
      \\
      & z \mathrlap{{} \pmir d}
    \end{tikzcd}
    \hphantom{{} \fuse b_1 \pmir i}
  \end{equation*}
  That $e$ is an active proposition violates our conception of values as inactive.
\end{proof}

\newthought{At this point}, we would like to prove the adequacy of decrements.
However, having just revised the definitions of $e$, $b_0$, and $b_1$, we must first recheck the adequacy of increments.
%
Unfortunately, the newly introduced alternative conjunctions, together with the fine-grained atomicity of ordered rewriting, cause the preservation and progress properties to fail.
%
\begin{falseclaim}[Small-step adequacy of increments]%
  \leavevmode
  \begin{thmdescription}
  \item[Value inclusion]
    If $\aval{\octx}{n}$, then $\ainc{\octx}{n}$.
  \item[Preservation]
    If $\ainc{\octx}{n}$ and $\octx \reduces \octx'$, then $\ainc{\octx'}{n}$.
  \item[Progress]
    If $\ainc{\octx}{n}$, then either%
    \begin{itemize*}[
      mode=unboxed, label=, afterlabel=,
      before=\unskip:\space,
      itemjoin=;\space, itemjoin*=; or\space%
    ]  
    \item $\octx \reduces \octx'$ for some $\octx'$
    \item $\octx \nreduces$ and $\aval{\octx}{n}$
    \end{itemize*}
  \item[Termination]
    If $\ainc{\octx}{n}$, then every rewriting sequence from $\octx$ is finite.
  \end{thmdescription}
\end{falseclaim}
\begin{proof}[Counterexample]
  As a counterexample to preservation, notice that $e \oc i$ denotes $1$ and that
\begin{equation*}
  e \oc i = (e \fuse b_1 \pmir i) \with (z \pmir d)
    \reduces (e \fuse b_1 \pmir i) \oc i
  \,,
\end{equation*}
but that $(e \fuse b_1 \pmir i) \oc i$ does not have a denotation under the $\ainc{}{}$ relation.

  Even worse, computations can now enter stuck states -- $e \oc i \reduces (z \pmir d) \oc i \nreduces$, for example.
  It's difficult to imagine assigning denotations to these stuck states, making them counterexamples to preservation.
  Even if denotations were somehow assigned to them, such states would anyway violate the desired progress theorem.
\end{proof}

In both cases, these counterexamples arise from the very fine-grained atomicity of ordered rewriting.
Now that the definitions of $e$, $b_0$, and $b_1$ include alternative conjunctions, [...].

\begin{theorem}
  \begin{thmdescription}
  \item[Evaluation]
    If $\ainc{\octx}{n}$, then $\octx \Reduces\aval{}{n}$.
    In particular, if $\aval{\octx}{n}$, then $\octx \Reduces\aval{n+1}$.
  \item[Preservation]
    If $\ainc{\octx}{n}$ and $\octx \Reduces\aval{}{n'}$, then $n' = n$.
  \end{thmdescription}
\end{theorem}
\begin{proof}
  By structural induction on the given derivation of $\ainc{\octx}{n}$.
\end{proof}

The solution is to chain several small rewriting steps together into a single, larger atomic step. 


\section{Weakly focused rewriting}

\Textcite{Andreoli:??}'s observation was that propositions can be partitioned into two classes, or \vocab{polarities}\fixnote{reference?}, according to the invertibility of their sequent calculus rules, and that [...].



The ordered propositions are polarized into two classes, the positive and negative propositions, according to the invertibility of their sequent calculus rules.
\begin{syntax*}
  Positive props. &
    \p{A} & \p{\alpha} \mid \p{A} \fuse \p{B} \mid \one \mid \dn \n{A}
  \\
  Negative props. &
    \n{A} & \n{\alpha} \mid
    % \begin{array}[t]{@{{} \mid {}}l@{}}
              \p{A} \limp \n{B} \mid \n{B} \pmir \p{A} \mid % \\
              \n{A} \with \n{B} \mid \top \mid \up \p{A}
            % \end{array}
\end{syntax*}
The positive propositions are those propositions that have invertible left rules, such as ordered conjunction;
the negative propositions are those that have invertible right rules, such as the ordered implications.

\begin{syntax*}
  Ordered contexts &
    \octx & \octx_1 \oc \octx_2 \mid \octxe \mid \p{A}
\end{syntax*}

Left rules for negative connectives may be chained together into a single \vocab{left-focusing phase}, reflected by the pattern judgment $\lfocus{\octx_L}{\n{A}}{\octx_R}{\p{C}}$.
Following \textcite{Zeilberger:??}, this judgment can be read as a function of an in-focus negative proposition, $\n{A}$, that produces the ordered contexts $\octx_L$ and $\octx_R$ and the positive consequent $\p{C}$ as outputs.

The left-focus judgment is defined inductively on the structure of the in-focus proposition by the following rules.
\begin{inferences}
  \infer[\lrule{\limp}']{\lfocus{\octx_L \oc \p{A}}{\p{A} \limp \n{B}}{\octx_R}{\p{C}}}{
    \lfocus{\octx_L}{\n{B}}{\octx_R}{\p{C}}}
  \and
  \infer[\lrule{\pmir}']{\lfocus{\octx_L}{\n{B} \pmir \p{A}}{\p{A} \oc \octx_R}{\p{C}}}{
    \lfocus{\octx_L}{\n{B}}{\octx_R}{\p{C}}}
  \\
  \infer[\lrule{\with}_1]{\lfocus{\octx_L}{\n{A} \with \n{B}}{\octx_R}{\p{C}}}{
    \lfocus{\octx_L}{\n{A}}{\octx_R}{\p{C}}}
  \and
  \infer[\lrule{\with}_2]{\lfocus{\octx_L}{\n{A} \with \n{B}}{\octx_R}{\p{C}}}{
    \lfocus{\octx_L}{\n{B}}{\octx_R}{\p{C}}}
  \and
  \text{(no $\lrule{\top}$ rule)}
  \\
  \infer[\lrule{\up}]{\lfocus{}{\up \p{A}}{}{\p{A}}}{}
\end{inferences}
These rules parallel the usual sequent calculus rules, maintaining focus on the subformulas of negative polarity.
First, the $\lrule{\up}$ rule finishes a left-focusing phase by producing the consequent $\p{A}$ from $\up \p{A}$.

Second, the $\lrule{\limp}'$ and $\lrule{\pmir}'$ rules diverge slightly from the usual left rules for left- and right-handed implication in that they have no premises decomposing [the antecedent\fixnote{wc?}] $\p{A}$.
This would mean that a weakly focused sequent calculus based on $\lrule{\limp}'$ and $\lrule{\pmir}'$ would be incomplete for provability.
It is possible to [...]\autocite{Simmons:CMU??}.
However, because our goal here is a rewriting framework and such a framework is inherently incomplete\fixnote{Is this right?}, [...].


\begin{equation*}
  \infer[\jrule{$\dn$D}]{\octx_L \oc \dn \n{A} \oc \octx_R \reduces \p{C}}{
    \lfocus{\octx_L}{\n{A}}{\octx_R}{\p{C}}}
\end{equation*}

Consider the recursively defined proposition $\alpha \defd (\beta \limp \alpha) \with (\gamma \limp \one)$.
Previously, in the unfocused rewriting framework, it took two steps to rewrite $\beta \oc \alpha$ into $\alpha$:
\begin{equation*}
  \beta \oc \alpha = \beta \oc \bigl((\beta \limp \alpha) \with (\gamma \limp \one)\bigr)
    \reduces \beta \oc (\beta \limp \alpha)
    \reduces \alpha
\end{equation*}
Now, in the polarized, weakly focused rewriting framework, the analogous recursive definition is $\n{\alpha} \defd (\p{\beta} \limp \up \dn \n{\alpha}) \with (\p{\gamma} \limp \up \one)$, and it takes only one step to rewrite $\p{\beta} \oc \dn \n{\alpha}$ into $\dn \n{\alpha}$:
\begin{equation*}
  \p{\beta} \oc \dn \n{\alpha} = \p{\beta} \oc \dn \bigl((\p{\beta} \limp \up \dn \n{\alpha}) \with (\p{\gamma} \limp \up \one)\bigr)
    \reduces \dn \n{\alpha}
\end{equation*}
because $\lfocus{\p{\beta}}{\n{\alpha}}{}{\dn \n{\alpha}}$.

Notice that, because the left-focus judgment is defined inductively, there are some recursively defined negative propositions that cannot successfully be put in focus.
For example, under the definition $\n{\alpha} \defd \p{\beta} \limp \n{\alpha}$, there are no contexts $\octx_L$ and $\octx_R$ and conseqeunt $\p{C}$ for which $\lfocus{\octx_L}{\n{\alpha}}{\octx_R}{\p{C}}$ is derivable.

In addition to the $\jrule{$\dn$D}$ rule for decomposing $\dn \n{A}$, weakly focused ordered rewriting retains the $\jrule{$\fuse$D}$ and $\jrule{$\one$D}$ rules for decomposing $\p{A} \fuse \p{B}$ and $\one$ and the compatability rules, $\jrule{$\reduces$C}_{\jrule{L}}$ and $\jrule{$\reduces$C}_{\jrule{R}}$.
Together, these five rules and the left focus\fixnote{focal?} rules comprise the weakly focused ordered rewriting framework; they are summarized in \cref{??}.
%
\begin{figure}
  \begin{syntax*}
    Positive props. &
      \p{A} & \p{\alpha} \mid \p{A} \fuse \p{B} \mid \one \mid \dn \n{A}
    \\
    Negative props. &
      \n{A} & \n{\alpha} \mid
      % \begin{array}[t]{@{{} \mid {}}l@{}}
                \p{A} \limp \n{B} \mid \n{B} \pmir \p{A} \mid % \\
                \n{A} \with \n{B} \mid \top \mid \up \p{A}
              % \end{array}
    \\
    Ordered contexts &
      \octx & \octx_1 \oc \octx_2 \mid \octxe \mid \p{A}
  \end{syntax*}
  \begin{inferences}[Rewriting: $\octx \reduces \octx'$ and $\octx \Reduces \octx'$]
    \infer[\jrule{$\dn$D}]{\octx_L \oc \dn \n{A} \oc \octx_R \reduces \p{C}}{
      \lfocus{\octx_L}{\n{A}}{\octx_R}{\p{C}}}
    \and
    \infer[\jrule{$\fuse$D}]{\p{A} \fuse \p{B} \reduces \p{A} \oc \p{B}}{}
    \and
    \infer[\jrule{$\one$D}]{\one \reduces \octxe}{}
    \\
    \text{(no $\jrule{$\plus$D}$ and $\jrule{$\zero$D}$ rules)}
    \\
    \infer[\jrule{$\reduces$C}_{\jrule{L}}]{\octx_1 \oc \octx_2 \reduces \octx'_1 \oc \octx_2}{
      \octx_1 \reduces \octx'_1}
    \and
    \infer[\jrule{$\reduces$C}_{\jrule{R}}]{\octx_1 \oc \octx_2 \reduces \octx'_1 \oc \octx_2}{
      \octx_1 \reduces \octx'_1}
  \end{inferences}
  \begin{inferences}
    \infer[\jrule{$\Reduces$R}]{\octx \Reduces \octx}{}
    \and
    \infer[\jrule{$\Reduces$T}]{\octx \Reduces \octx''}{
      \octx \reduces \octx' & \octx' \Reduces \octx''}
  \end{inferences}

  \begin{inferences}[Left focus: $\lfocus{\octx_L}{\n{A}}{\octx_R}{\p{C}}$]
    \infer[\lrule{\limp}']{\lfocus{\octx_L}{\p{A} \limp \n{B}}{\octx_R}{\p{C}}}{
      \lfocus{\octx_L \oc \p{A}}{\n{B}}{\octx_R}{\p{C}}}
    \and
    \infer[\lrule{\pmir}']{\lfocus{\octx_L}{\n{B} \pmir \p{A}}{\octx_R}{\p{C}}}{
      \lfocus{\octx_L}{\n{B}}{\p{A} \oc \octx_R}{\p{C}}}
    \\
    \infer[\lrule{\with}_1]{\lfocus{\octx_L}{\n{A} \with \n{B}}{\octx_R}{\p{C}}}{
      \lfocus{\octx_L}{\n{A}}{\octx_R}{\p{C}}}
    \and
    \infer[\lrule{\with}_2]{\lfocus{\octx_L}{\n{A} \with \n{B}}{\octx_R}{\p{C}}}{
      \lfocus{\octx_L}{\n{B}}{\octx_R}{\p{C}}}
    \and
    \text{(no $\lrule{\top}$ rule)}
    \\
    \infer[\lrule{\up}]{\lfocus{}{\up \p{A}}{}{\p{A}}}{}
  \end{inferences}
  \caption{A weakly focused ordered rewriting framework}
\end{figure}

Weakly focused ordered rewriting is sound with respect to the unfocused rewriting framework of \cref{??}.
Given a depolarization function $\erase{}$ that maps polarized propositions and contexts to their unpolarized counterparts,%
\begin{marginfigure}
  \begin{equation*}
    \begin{aligned}[t]
      \erase{\octx_1 \oc \octx_2}
        &= \erase*{\octx_1} \oc \erase*{\octx_2} \\
      \erase{\octxe} &= \octxe \\
      \erase{\p{A}} &= \erase{\p{A}}
    \end{aligned}
    \qquad
    \begin{gathered}[t]
      \erase{\dn \n{A}} = \erase{\n{A}}
      \\
      \erase{\up \p{A}} = \erase{\p{A}}
      % \\
      % \erase{\p{A} \fuse \p{B}} &= \erase{\p{A}} \fuse \erase{\p{B}}
      \\
      \begin{array}{@{}l@{}}
        \erase{\p{A} \limp \n{B}} \\
        \quad{} = \erase{\p{A}} \limp \erase{\n{B}}
      \end{array}
      \\
      \rlap{\emph{etc.}}
    \end{gathered}
  \end{equation*}
  \caption{Depolarization of propositions and contexts}
\end{marginfigure}%
%
we may state and prove the following soundness theorem for weakly focused rewriting.
%
\begin{theorem}[Soundness of weakly focused rewriting]
  If $\octx \Reduces \octx'$, then $\erase*{\octx} \Reduces \erase{\octx'}$.
\end{theorem}
\begin{proof}
  By structural induction on the given rewriting step, after generalizing the inductive hypothesis to include:
  \begin{itemize}[nosep]
  \item If $\octx \reduces \octx'$, then $\erase*{\octx} \Reduces \erase{\octx'}$.
  \item If $\lfocus{\octx_L}{\n{A}}{\octx_R}{\p{C}}$, then $\erase{\octx_L \oc \dn \n{A} \oc \octx_R} \Reduces \erase{\p{C}}$.
  %
  \qedhere
  \end{itemize}
\end{proof}
%
\noindent
A completeness theorem also holds, but we forgo its development because it is not essential to the remainder of this work.

Second, with the lone exception negative propositions are latent\autocite{??} -- 

\clearpage
\section{Revisiting automata}

\begin{gather*}
  \dfa{q} \defd
    \parens[size=big]{\bigwith_{a \in \ialph}(a \limp \up \dn \dfa{q}'_a)}
    \with
    (\emp \limp \up \dfa{F}(q))
\shortintertext{where}
  q \dfareduces[a] q'_a
  \text{, for all $a \in \ialph$\quad and\quad}
  \dfa{F}(q) = \begin{cases*}
                 \one & if $q \in F$ \\
                 \dn \top & if $q \notin F$
               \end{cases*}
\end{gather*}

\begin{theorem}[\ac*{DFA} adequacy up to bisimilarity]
  \dfaadequacybisimbody
\end{theorem}

Lemma\cref{??} is still needed, but now has a much different proof.
Previously, the proof of \cref{??} relied on a very specific and delicate property of \acp{DFA}, namely that each \ac{DFA} state has one and only one $a$-successor for each input symbol $a$.
Now, with weakly focused ordered rewriting, the \lcnamecref{??}'s proof is much less fragile.
With the larger granularity of individual rewriting steps that the weakly focused framework affords, a state's encoding is a latent proposition 

\section{Revisiting binary counters}

With ordered rewriting now based on a weakly focused sequent calculus, we can revisit our previous attempt to extend binary counters with support for decrements or head-unary normalization.

The propositions $e$, $b_0$, $b'_0$, and $b_1$ are recursively defined in nearly the same way as before.
With one exception discussed below, only the necessary shifts are inserted to consistently assign a negative polarity to the defined atoms $e$, $b_0$, $b'_0$, and $b_1$ and a positive polarity to the uninterpreted atoms $i$, $d$, $z$, and $s$.
\begin{equation*}
  \begin{lgathered}
    e \defd (e \fuse b_1 \pmir i) \with (z \pmir d) \\
    b_0 \defd (\up \dn b_1 \pmir i) \with (d \fuse b'_0 \pmir d) \\
    b'_0 \defd (z \limp z) \with (s \limp b_1 \fuse s) \\
    b_1 \defd (i \fuse b_0 \pmir i) \with (b_0 \fuse s \pmir d)
  \end{lgathered}
\end{equation*}

\paragraph*{Values}
Once again, we use the same $\aval{}{}$ relation to assign a unique natural number denotation to each binary representation.
\begin{inferences}
  \ooavalrules
\end{inferences}
Because the underlying ordered rewriting framework has changed, we must verify that $\aval{}{}$ is adequate -- inparticular, the [...] property that values cannot be independently rewritten.
%
\ooavaltheorem
\begin{proof}
  By induction over the structure of $\octx$.
  As an example, consider the case in which $\aval{e}{0}$.
  Indeed, $e \nreduces$ because $e = (e \fuse b_1 \pmir i) \with (z \pmir d)$ and
  \begin{equation*}
    \lfocus{\octx_L}{(e \fuse b_1 \pmir i) \with (z \pmir d)}{\octx_R}{\p{C}}
    \text{\ only if $\octx_L = \octxe$ and either $\octx_R = i$ or $\octx_R = d$.}
  \end{equation*}
  The other cases are similar.
\end{proof}

\paragraph*{Increment}
Previously, under the unfocused rewriting framework\fixnote{wc?}, rewriting $e \oc i$ into $e \fuse b_1$ took two small steps:
\begin{equation*}
  e \oc i = \bigl((e \fuse b_1 \pmir i) \with (z \pmir d)\bigr) \oc i
    \reduces (e \fuse b_1 \pmir i) \oc i
    \reduces e \fuse b_1
\end{equation*}
But now, with weakly focused rewriting, those two steps are combined into one atomic whole: $e \oc i \reduces e \fuse b_1$.

As for the unfocused rewriting implementation of binary increments, we use a $\ainc{}{}$ relation to assign a natural number denotation to each computational state.
In fact, the specific definition of the $\ainc{}{}$ remains unchanged from \cref{??}: 
\begin{inferences}
  \aincrules
\end{inferences}

The only exception to [...] is the appearance of $\up \dn b_1$ in the definition of $b_0$.
Without this double shift, $e \oc b_0 \oc i$ would be latent, unable to rewrite to a value until a second increment is appended, because the necessary $\lfocus{}{(b_1 \pmir i) \with (d \fuse b'_0 \pmir d)}{i}{b_1}$ is not derivable.
However, with the double shift, $e \oc b_0 \oc i \reduces e \oc b_1$ because $\lfocus{}{(\up \dn b_1 \pmir i) \with (d \fuse b'_0 \pmir d)}{i}{\dn b_1}$ is derivable.


With weakly focused rewriting, it is no longer possible to reach the stuck state $...$.
\smallincadequacytheorem
\begin{proof}
  As before, each part is proved separately.
  \begin{description}[
    parsep=0pt, listparindent=\parindent,
    labelsep=0.35em
  ]
  \item[Value soundness, preservation, and progress]
    can be proved by structural induction on the derivation of $\ainc{\octx}{n}$.
  \item[Termination]
    can be proved using the same explicit termination measure, $\card{\octx}$, as in \cref{??}.
  %
  \qedhere
  \end{description}
\end{proof}


\paragraph*{Decrements}

\begin{inferences}
  \infer[\jrule{$d$-D}]{\adec{\octx \oc d}{n}}{
    \ainc{\octx}{n}}
  \and
  \infer[\jrule{$b'_0$-D}]{\adec{\octx \oc b'_0}{2n}}{
    \adec{\octx}{n}}
  \and
  \infer[\jrule{$z$-D}]{\adec{z}{0}}{}
  \and
  \infer[\jrule{$s$-D}]{\adec{\octx \oc s}{n+1}}{
    \ainc{\octx}{n}}
  \\
  \infer[\jrule{$\fuse_1$-D}]{\adec{\octx \oc (d \fuse b'_0)}{2n}}{
    \ainc{\octx}{n}}
  \\
  \infer[\jrule{$\fuse_2$-D}]{\adec{\octx \oc (b_0 \fuse s)}{2n+1}}{
    \ainc{\octx}{n}}
  \and
  \infer[\jrule{$\fuse_3$-D}]{\adec{\octx \oc (b_1 \fuse s)}{2n+2}}{
    \ainc{\octx}{n}}
\end{inferences}

\begin{theorem}[Small-step adequacy of decrements]
  \leavevmode
  \begin{thmdescription}
  \item[Preservation]
    If $\adec{\octx}{n}$ and $\octx \reduces \octx'$, then $\adec{\octx'}{n}$.
  \item[Progress]
    If $\adec{\octx}{n}$, then [either]:
    \begin{itemize}[nosep]
    \item $\octx \reduces \octx'$, for some $\octx'$;
    \item $n = 0$ and $\octx = z$; or
    \item $n > 0$ and $\octx = \octx' \oc s$, for some $\octx'$ such that $\ainc{\octx'}{n-1}$.
    \end{itemize}
  \item[Termination]
    If $\adec{\octx}{n}$, then every rewriting sequence from $\octx$ is finite.
  \end{thmdescription}
\end{theorem}
\begin{proof}
  \begin{description}
  \item[Preservation and progress]
    are proved, as before, by structural induction on the given derivation of $\adec{\octx}{n}$.
  \item[Termination] is proved by exhibiting a measure, $\card[d]{}$, that is strictly decreasing across each rewriting.
    Following the example of termination for increment-only binary counters\parencref{??}, we could try to assign a constant amount of potential to each of the counter's constituents.
    Leaving these potentials as unknowns, we can generate a set of constraints from the allowed rewritings and then attempt to solve them.

    For instance, here are several rewritings and their corresponding potential constraints.
    \begin{center}
      \begin{tabular}{@{}c@{\quad}c@{}}
        \emph{Some selected rewritings} & \emph{Potential constraints}
        \\
        $\octx \oc b_1 \oc i \reduces \octx \oc (i \fuse b_0) \reduces \octx \oc i \oc b_0$
        & $b_1 + i > i + b_0 + 1$
        \\
        $\octx \oc b_0 \oc d \reduces \octx \oc (d \fuse b'_0) \reduces \octx \oc d \oc b'_0$
        & $b_0 + d > d + b'_0 + 1$
        \\
        $\octx \oc s \oc b'_0 \reduces \octx \oc (b_1 \fuse s) \reduces \octx \oc b_1 \oc s$
        & $s + b'_0 > b_1 + s + 1$
        % \\
        % $\octx \oc b_1 \oc d \reduces \octx \oc (b_0 \fuse s) \reduces \octx \oc b_0 \oc s$
        % & $b_1 + d > b_0 + s + 1$
      \end{tabular}
    \end{center}
    These constraints are satisfiable only if $b_1 > b_0 > b'_0 > b_1$, which is, of course, impossible.

    However, notice that each $b_1$ that arises from an interaction between $s$ and $b'_0$ will never participate in further rewritings because any increments remaining to the left of $b_1$ will only involve more significant bits, not this less significant $b_1$.
    A similar argument can be made for all bits that occur between the rightmost $i$ and the terminal $s$, suggesting that those bits be assigned no potential at all.

    This leads to the termination measure, $\card[d]{}$, and its auxiliary measures, $\card[i]{}$ and $\card[s]{}$, shown in the adjacent \lcnamecref{??}.
    (Note that the measure $\card[i]{}$ is not the same as the measure used for increment-only binary counters\parencref{??}.)x




    is proved by exhibiting a pair of measures, $\card{\octx}_d$ and $\card{\octx}_s$, ordered lexicographically:
    \begin{itemize}
    \item If $\adec{\octx}{n}$ and $\octx \reduces \octx'$, then either:
      \begin{itemize*}[label=, afterlabel=]
      \item $\card{\octx}_d > \card{\octx'}_d$; or
      \item $\card{\octx}_d = 0$ and $\card{\octx}_s > \card{\octx'}_s$.
      % \item $\card{\octx} > \card{\octx'}$; or
      % \item $n = 0$ and $\octx' = z$; or
      % \item $n > 0$ and $\octx' = \octx'' \oc s$, for some $\octx''$ such that $\ainc{\octx''}{n-1}$.
      % \item $\octx = \octx_0 \oc (b_1 \fuse s) \reduces \octx_0 \oc b_1 \oc s = \octx'$.
      \end{itemize*}
    \end{itemize}
    These measures are shown in the adjacent \lcnamecref{??}, %
    \begin{marginfigure}
      \begin{equation*}
        \begin{lgathered}[t]
          \card{\octx \oc s}_s = \card{\octx} \\
          \card{e} = 0 \\
          \card{\octx \oc b_0} = \card{\octx} + 4 \\
          \card{\octx \oc b_1} = \card{\octx} + 6 \\
          \card{\octx \oc i} = \card{\octx} + 8 \\
          \card{e \fuse b_1} = 7 \\
          \card{\octx \oc (i \fuse b_0)} = \card{\octx} + 13
        \end{lgathered}
        \qquad
        \begin{lgathered}[t]
          \card{\octx \oc d}_d = \card{\octx} + 1 \\
          \card{\octx \oc b'_0}_d = \card{\octx}_d + 2 \\
          \card{z}_d = 0 \\
          \card{\octx \oc s}_d = 0 \\
          \card{\octx \oc (d \fuse b'_0)}_d = \card{\octx} + 4 \\
          \card{\octx \oc (b_0 \fuse s)}_d = 1 \\
          \card{\octx \oc (b_1 \fuse s)}_d = 1
        \end{lgathered}
      \end{equation*}
    \end{marginfigure}
    rely on an auxiliary measure, $\card{\octx}$, for increment states.
    Unfortunately, it is not possible to simply reuse the measure from \cref{??}.
    In that measure, each $b_0$ bit was assigned no potential.
    With decrements, however, $b_0$ needs to carry enough potential to transfer to $b'_0$ in case a decrement instruction is encountered.

    For the rewritings $\octx \oc b_1 \oc i \reduces \octx \oc (i \fuse b_0) \reduces \octx \oc i \oc b_0$, the assigned potentials must satisfy $b_1 + i > i + b_0 + 1$


    No 

        
    % e d --> z   1 > 0
    % b0 d --> d * b0' --> d b0'  4 > 3 > 2
    % b1 d --> b0 * s --> b0 s    6 > 1 > 0
    % z b0' --> z                 2 > 0
    % s b0' --> b1* s --> b1 s    2 > 1 > 0
    % e i --> e * b1 --> e b1     8 > 7 > 6
    % b0 i --> b1                 12 > 6
    % b1 i --> i * b0 --> i b0    14 > 13 > 12

    % e + i > e + b1 + 1  
    % b0 + i > b1
    % b1 + i > i + b0 + 1
    % e + d > z
    % b0 + d > d + b0' + 1
    % b1 + d > s + 1
    % z + b0' > z
    % s + b0' > s + 1

    % b1 d --> b0 * s --> b0 s      2 + d > 1 > 0
    % b0 d --> d * b'0 --> d b'0    b0 + d > d + 2 + 1 > d + 2
    % z b'0 --> z
    % s b'0 --> b1 * s --> b1 s

    As an example case, consider the intermediate state $\octx \oc (b_1 \fuse s)$ and its rewriting $\octx \oc (b_1 \fuse s) \reduces \octx \oc b_1 \oc s$.
    It follows $\card{\octx \oc (b_1 \fuse s)}_d = 1 > 0 = \card{\octx \oc b_1 \oc s}_d$.
    Any subsequent rewritings of $\octx$ are justified by a decrease in $\card{\octx \oc b_1 \oc s}_s > \card{\octx}$.
  \end{description}
\end{proof}

\begin{corollary}[Big-step adequacy of decrements]
  If $\adec{\octx}{n}$, then:
  \begin{itemize}[nosep]
  \item $\octx \Reduces z$ if, and only if, $n = 0$;
  \item $\octx \Reduces \octx' \oc s$ for some $\octx'$ such that $\ainc{\octx'}{n-1}$, if $n > 0$; and
  \item $\octx \Reduces \octx' \oc s$ only if $n > 0$ and $\ainc{\octx'}{n-1}$.
  \end{itemize}
\end{corollary}
\begin{proof}
  From the small-step preservation result of \cref{??}, it is possible to prove, using a structural induction on the given trace, a big-step preservation result: namely, that $\adec{\octx}{n}$ and $\octx \Reduces \octx'$ only if $\adec{\octx'}{n}$.
  Each of the above claims then follows from either progress and termination\parencref{??} or big-step preservation together with inversion.
\end{proof}

%   \begin{itemize}
%   \item
%     In the left-to-right direction, preservation yields $\adec{z}{n}$; by inversion, $n$ can only be $0$.
%     The right-to-left direction follows immediately from productivity.
%   \item
%     This second statement follows immediately from productivity.
%   \item
%     Preservation yields $\adec{\octx' \oc s}{n}$; by inversion, $n$ must be strictly positive with $\ainc{\octx'}{n-1}$.
%   \end{itemize}
% \end{proof}

    \begin{equation*}
      \begin{lgathered}[t]
        \card[d]{\octx \oc d} = \card[i]{\octx} + 1 \\
        \card[d]{\octx \oc b'_0} = \card[d]{\octx} + 2 \\
        \card[d]{z} = 0 \\
        \card[d]{\octx \oc s} = \card[s]{\octx}
        \\[\jot]
        \card[d]{\octx \oc (d \fuse b'_0)} = \card[d]{\octx \oc d \oc b'_0} + 1 \\
        \card[d]{\octx \oc (b_0 \fuse s)} = \card[d]{\octx \oc b_0 \oc s} + 1 \\
        \card[d]{\octx \oc (b_1 \fuse s)} = \card[d]{\octx \oc b_1 \oc s} + 1
      \end{lgathered}
      \qquad
      \begin{lgathered}[t]
        \card[i]{e} = 0 \\
        \card[i]{\octx \oc b_0} = \card[i]{\octx} + 4 \\
        \card[i]{\octx \oc b_1} = \card[i]{\octx} + 6 \\
        \card[i]{\octx \oc i} = \card[i]{\octx} + 8
        \\[\jot]
        \card[i]{e \fuse b_1} = \card[i]{e \oc b_1} + 1 \\
        \card[i]{\octx \oc (i \fuse b_0)} = \card[i]{\octx \oc i \oc b_0} + 1
      \end{lgathered}
      \qquad
      \begin{lgathered}[t]
        \card[s]{e} = \card[i]{e} = 0 \\
        \card[s]{\octx \oc b_0} = \card[s]{\octx} \\
        \card[s]{\octx \oc b_1} = \card[s]{\octx} \\
        \card[s]{\octx \oc i} = \card[i]{\octx \oc i} = \card[i]{\octx} + 8
        \\[\jot]
        \card[s]{e \fuse b_1} = \card[s]{e \oc b_1} + 1 \\
        \card[s]{\octx \oc (i \fuse b_0)} = \card[s]{\octx \oc i \oc b_0} + 1
      \end{lgathered}
    \end{equation*}

    \begin{itemize}
    \item If $\adec{\octx}{n}$ and $\octx \reduces \octx'$, then $\card[d]{\octx} > \card[d]{\octx'}$.
    \item If $\ainc{\octx}{n}$ and $\octx \reduces \octx'$, then $\card[i]{\octx} > \card[i]{\octx'}$ and $\card[s]{\octx} > \card[s]{\octx'}$.
    \end{itemize}

% b1 d --> b0 s  |-|i + 7 > |-|s + 1
% s b0' --> b1 s  |-|s + 2 > |-|s + 1


\section{Temporary}

% \begin{theorem}[Behavioral adequacy of decrements]
%   If $\adec{\octx}{n}$, then:
%   \begin{itemize}[nosep]
%   \item $\octx \Reduces z$ if, and only if, $n = 0$;
%   \item $\octx \Reduces \octx' \oc s$ for some $\octx'$ such that $\ainc{\octx'}{n-1}$, if $n > 0$;
%   \item $\octx \Reduces \octx' \oc s$ only if $n > 0$ and $\ainc{\octx'}{n-1}$.
%   \end{itemize}
% \end{theorem}

\begin{inferences}
  \infer{\adec{\octx \oc d}{n}}{
    \ainc{\octx}{n}}
  \and
  \infer{\adec{\octx \oc b'_0}{2n}}{
    \adec{\octx}{n}}
  \and
  \infer{\adec{z}{0}}{}
  \and
  \infer{\adec{\octx \oc s}{n+1}}{
    \ainc{\octx}{n}}
  \\
  \infer{\adec{\octx \oc (d \fuse b'_0)}{2n}}{
    \adec{\octx}{n}}
  \and
  \infer{\adec{\octx \oc (b_1 \fuse s)}{2n+2}}{
    \ainc{\octx}{n}}
  \and
  \infer{\adec{\octx \oc (b_0 \fuse s)}{2n+1}}{
    \ainc{\octx}{n}}
\end{inferences}
As the first rule exhibits, a binary number and its head-unary form denote the same value.
The last three rules are included by analogy with the $e \fuse b_1$ and $i \fuse b_0$ rules of the $\ainc{}{}$ relation.

\begin{falseclaim}[Small-step adequacy of decrements]\leavevmode
  \begin{thmdescription}
  \item[Preservation]
    If $\adec{\octx}{n}$ and $\octx \reduces \octx'$, then $\adec{\octx'}{n}$.
  \item[Progress]
    If $\adec{\octx}{n}$, then either:
    \begin{itemize}[nosep]
    \item $\octx \reduces \octx'$;
    \item $n = 0$ and $\octx = z$; or
    \item $n > 0$ and $\octx = \octx' \oc s$ for some $\octx'$ such that $\ainc{\octx'}{n-1}$.
    \end{itemize}
  \item[Productivity]
    If $\adec{\octx}{n}$, then every rewriting sequence from $\octx$ has a finite prefix $\octx \Reduces \octx'$ such that either:
    \begin{itemize}[nosep]
    \item $n = 0$ and $\octx' = z$; or
    \item $n > 0$ and $\octx' = \octx'_0 \oc s$, for some $\octx'_0$ such that $\ainc{\octx'_0}{n-1}$.
    \end{itemize}
  \end{thmdescription}
\end{falseclaim}
\begin{proof}
  The fine-grained atomicity of ordered rewriting, together with the use of alternative conjunction in the recursively defined propositions $e$, $b_0$, $b'_0$, and $b_1$, causes both preservation and progress properties to fail.

  As a counterexample to preservation, $\adec{e \oc d}{0}$ and $e \oc d \reduces (z \pmir d) \oc d$, but $\adec{(z \pmir d) \oc d}{0}$ does \emph{not} hold.

  Even worse, the fine-grained atomicity of ordered rewriting means that computations can enter stuck states, which shouldn't have denotations and which would violate progress if they were somehow assigned denotations.
  For example, $\adec{e \oc d}{0}$ and $e \oc d \reduces (e \fuse b_1 \pmir i) \oc d \nreduces$.

  $\ainc{e \oc i}{0}$ and $e \oc i \reduces (z \pmir d) \oc i \nreduces$
\end{proof}


\newthought{These binary counters} may also be equipped with a decrement operation.
Although \enquote{decrement} is a convenient name for this operation, it is more accurate to implement decrements by converting the binary representation to what might be called \emph{head-unary form}: an ordered context $\octx$ is said to be in head-unary form if either: $\octx = z$; or $\octx = \octx_0 \oc s$ for some binary representation $\octx_0$.

Similar to how the atom $i$ is used to describe increments, a decrement is initiated by appending an atom $d$ to the counter; $d$ is then processed from right to left by the counter's bits.
To support this, the definitions of $e$, $b_0$, and $b_1$ are revised
\begin{equation*}
  \begin{lgathered}
    e \defd (e \fuse b_1 \pmir i) \with (\dotsb \pmir d) \\
    b_0 \defd (b_1 \pmir i) \with (\dotsb \pmir d) \\
    b_1 \defd (i \fuse b_0 \pmir i) \with (\dotsb \pmir d)
  \end{lgathered}
\end{equation*}

To initiate a decrement of a counter $\octx$, we append the uninterpreted atom $d$ to the counter, forming $\octx \oc d$.

To implement the decrement operation, we instead

Although \enquote{decrement} is a convenient name for this operation, it is perhaps more accurate to think of this operation as putting the binary representation into a head-unary form: either $z$ or $\octx' \oc s$ for some $\ainc{\octx'}{n-1}$.
\begin{itemize}
\item If $\ainc{\octx}{n}$, then:
  \begin{itemize}
  \item $n = 0$ if, and only if, $\octx \oc d \Reduces z$; and
  \item $n > 0$ implies $\octx \oc d \Reduces \octx' \oc s$ for some $\octx'$ such that $\ainc{\octx'}{n-1}$; and
  \item $\octx \oc d \Reduces \octx' \oc s$ implies $n > 0$ and $\ainc{\octx'}{n-1}$.
  \end{itemize}
\end{itemize}

\begin{equation*}
  \begin{lgathered}
    e \defd (e \fuse b_1 \pmir i) \with (z \pmir d) \\
    b_0 \defd (b_1 \pmir i) \with (d \fuse b'_0 \pmir d) \\
    b'_0 \defd (z \limp z) \with (s \limp b_1\fuse s) \\
    b_1 \defd (i \fuse b_0 \pmir i) \with (b_0 \fuse s \pmir d)
  \end{lgathered}
\end{equation*}

\begin{description}
\item[$e \defd \dotsb \with (z \pmir d)$]
  Because the counter $e$ represents $0$, its head-unary form is simply $z$.
%
\item[$b_1 \defd \dotsb \with (b_0 \fuse s \pmir d)$]
  Because the counter $\octx \oc b_1$ represents $2n+1 > 0$ when $\octx$ represents $n$, its head-unary form must then be the successor of a counter representing $2n$ -- that is, $\octx \oc b_0 \oc s$.
%
\item[$b_0 \defd \dotsb \with (d \fuse b'_0 \pmir d)$]
  The natural number that the counter $\octx \oc b_0$ represents could be either zero or positive, depending on whether $\octx$ represents zero or a positive natural number.
  Thus, to put $\octx \oc b_0$ into head-unary form, we first put $\octx$ into head-unary form and then use $b'_0$ to branch on the result.
%
\item[$b'_0 \defd (z \limp z) \with (s \limp b_1 \fuse s)$]
  If the head-unary form of $\octx$ is $z$, then $\octx \oc b_0$ also represents $0$ and has head-unary form $z$.
  Otherwise, if the head-unary form of $\octx$ is $\octx' \oc s$ for some $\ainc{\octx'}{n'}$, then $\octx \oc b_0$ represents $2n'+2$ and has head-unary form $\octx' \oc b_1 \oc s$.
\end{description}

Decrements actually do not literally decrement the counter, but instead put it into a \enquote{head unary} form in which the couter is either $z$ or $s$ with a binary counter beneath.


We will use the same strategy for proving the adequacy of decrements as we did for increments:
Characterize the valid states
\begin{inferences}
  \infer{\adec{\octx \oc d}{n}}{
    \ainc{\octx}{n}}
  \and
  \infer{\adec{\octx \oc b'_0}{2n}}{
    \adec{\octx}{n}}
  \and
  \infer{\adec{z}{0}}{}
  \and
  \infer{\adec{\octx \oc s}{n+1}}{
    \ainc{\octx}{n}}
  \\
  \infer{\ainc{e \fuse b_1 \pmir i}{0}}{}
  \and
  \infer{\ainc{z \pmir d}{0}}{}
  \\
  \infer{\ainc{\octx \oc (b_1 \pmir i)}{2n}}{
    \ainc{\octx}{n}}
  \and
  \infer{\ainc{\octx \oc (d \fuse b'_0 \pmir d)}{2n}}{
    \ainc{\octx}{n}}
  \and
  \infer{\adec{\octx \oc (d \fuse b'_0)}{2n}}{
    \ainc{\octx}{n}}
  \\
  \infer{\ainc{\octx \oc (i \fuse b_0 \pmir i)}{2n+1}}{
    \ainc{\octx}{n}}
  \and
  \infer{\ainc{\octx \oc (b_0 \fuse s \pmir d)}{2n+1}}{
    \ainc{\octx}{n}}
  \and
  \infer{\ainc{\octx \oc (i \fuse b_0)}{2n+2}}{
    \ainc{\octx}{n}}
  \and
  \infer{\adec{\octx \oc (b_0 \fuse s)}{2n+1}}{
    \ainc{\octx}{n}}
  \\
  \infer{\adec{\octx \oc (z \limp z)}{2n}}{
    \adec{\octx}{n}}
  \and
  \infer{\adec{\octx \oc (s \limp b_1 \fuse s)}{2n}}{
    \adec{\octx}{n}}
  \and
  \infer{\adec{\octx \oc (b_1 \fuse s)}{2n+2}}{
    \ainc{\octx}{n}}
\end{inferences}

Notice that $\adec{e \oc s \oc b'_0}{0}$ but $e \oc s \oc b'_0 \Reduces \adec{e \oc b_1 \oc s}{1}$.
If we revise the $s$ rule to use $n+1$, then a different problem arises: $\adec{e \oc b_1 \oc d}{0}$ but $e \oc b_1 \oc d \Reduces \adec{e \oc b_0 \oc s}{1}$.

\begin{theorem}[Adequacy]
  If $\ainc{\octx}{n}$, then:
  \begin{itemize}[nosep]
  \item $n = 0$ if and only if $\octx \oc d \Reduces z$; and
  \item $n > 0$ implies $\octx \oc d \Reduces \octx' \oc s$ and $\ainc{\octx'}{n-1}$;
  \item $\octx \oc d \Reduces \octx' \oc s$ implies $n > 0$ and $\ainc{\octx'}{n-1}$.
  \end{itemize}
\end{theorem}
%
\begin{proof}
  
\end{proof}

\begin{theorem}[Small-step adequacy]\leavevmode
  \begin{description}[nosep, font=\emph]
  \item[Preservation] If $\adec{\octx}{n}$ and $\octx \reduces \octx'$, then $\adec{\octx'}{n}$.
  \item[Progress] If $\adec{\octx}{n}$, then either:
    \begin{itemize}[nosep]
    \item $\octx \reduces \octx'$;
    \item $n = 0$ and $\octx = z$; or
    \item $n = n'+1$ and $\octx = \octx' \oc s$ for some $n'$ and $\octx'$ such that $\ainc{\octx'}{n'}$.
    \end{itemize}
  \end{description}
\end{theorem}



\section{}

\begin{corollary}[Big-step adequacy of decrements]
  If $\adec{\octx}{n}$, then:
  \begin{itemize}
  \item $\octx \Reduces \atmR{z}$ if, and only if, $n = 0$;
  \item $\octx \Reduces \octx' \oc \atmR{s}$ for some $\octx'$ such that $\ainc{\octx'}{n-1}$, if $n > 0$; and
  \item $\octx \Reduces \octx' \oc \atmR{s}$ only if $n > 0$ and $\ainc{\octx'}{n-1}$.
  \end{itemize}
\end{corollary}




\section{}

\subsection{Automata}

\begin{enumerate}
\item
  Traces do not imply DFA transitions:
  \begin{equation*}
    \begin{lgathered}
      \dfa{q}_0 \defd (a \limp \dfa{q}_0) \with (b \limp \dfa{q}_1) \with (\emp \limp \top) \\
      \dfa{q}_1 \defd (a \limp \dfa{q}_0) \with (b \limp \dfa{q}_1) \with (\emp \limp \one) \\
      \dfa{s}_1 \defd (a \limp \dfa{q}_0) \with (b \limp \dfa{s}_1) \with (\emp \limp \one)
    \end{lgathered}
  \end{equation*}
  \begin{marginfigure}
    \begin{equation*}
      \begin{tikzpicture}[baseline=(q_0.base)]
        \graph [automaton] {
          q_0
           -> [loop above, "a"]
          q_0
           -> ["b", bend left]
          q_1 [accepting]
           -> ["b", loop above]
          q_1
           -> ["a", bend left]
          q_0;
          s_1 [below=0.05 of q_1, accepting]
           -> [loop right, "b"]
          s_1
           -> ["a", bend left]
          q_0;
        };
      \end{tikzpicture}
    \end{equation*}
  \end{marginfigure}
  Notice that $b \oc \dfa{q}_0 \Reduces \dfa{q}_1 = \dfa{s}_1$ but $s_1$ is not reachable from $q_0$.
  ($\dfa{q}_1 = \dfa{s}_1$ is proved coinductively.)

\item
  NFA bisimilarity does not imply equality of encodings:
  \begin{equation*}
    \begin{lgathered}
      \nfa{q}_0 \defd (a \limp (\nfa{q}_0 \with \nfa{q}_1)) \with (\emp \limp \one) \\
      \nfa{q}_1 \defd (a \limp \nfa{q}_1) \with (\emp \limp \one)
    \end{lgathered}
  \end{equation*}
  \begin{marginfigure}
    \begin{equation*}
      \begin{tikzpicture}[baseline=(q_0.base)]
        \graph [automaton] {
          q_0 [accepting]
           -> [loop above, "a"]
          q_0
           -> ["a"]
          q_1 [accepting]
           -> ["a", loop above]
          q_1;
        };
      \end{tikzpicture}
    \end{equation*}
  \end{marginfigure}
  Notice that $q_0$ and $q_1$ are bisimilar, as witnessed by the reflexive closure of $\{(q_0,q_1)\}$.
  However, $\nfa{q}_0 \neq \nfa{q}_1$.

\item
  NFA similarity does not imply reduction.
  In the above example, NFA states $q_0$ and $q_1$ are bisimilar, andhence $q_1$ simulates $q_0$ (and vice versa).
  However, neither $\nfa{q}_0 \Reduces \nfa{q}_1$ nor $\nfa{q}_1 \Reduces \nfa{q}_0$ hold.

\item
  Even if an alternative, flatter encoding is used, NFA similarity does not imply reduction.
  Consider the following NFAs:
  \begin{align*}
   &\begin{lgathered}
      \nfa{q}_0 \defd (a \limp \nfa{q}_1) \with (\emp \limp \top) \\
      \nfa{q}_1 \defd (a \limp \nfa{q}_1) \with (a \limp \nfa{q}_2) \with (\emp \limp \one) \\
      \nfa{q}_2 \defd (a \limp \nfa{q}_2) \with (\emp \limp \one)
    \end{lgathered}
  \shortintertext{and}
   &\begin{lgathered}
      \nfa{s}_0 \defd (a \limp \nfa{s}_1) \with (\emp \limp \top) \\
      \nfa{s}_1 \defd (a \limp \nfa{s}_1) \with (\emp \limp \one)
    \end{lgathered}
  \end{align*}
  \begin{marginfigure}
    \begin{align*}
      \begin{tikzpicture}[baseline=(q_0.base)]
        \graph [automaton] {
          q_0
           -> ["a"]
          q_1 [accepting]
           -> [loop above, "a"]
          q_1
           -> ["a"]
          q_2 [accepting]
           -> ["a", loop above]
          q_2;
        };
      \end{tikzpicture}
      \\
      \begin{tikzpicture}[baseline=(s_0.base)]
        \graph [automaton] {
          s_0
           -> ["a"]
          s_1 [accepting]
           -> [loop above, "a"]
          s_1;
        };
      \end{tikzpicture}
    \end{align*}
  \end{marginfigure}
  As witnessed by the relation $\{(q_0,s_0), (q_1,s_1), (q_2,s_1)\}$, state $s_0$ simulates $q_0$.
  However, $\nfa{q}_0 \Longarrownot\Reduces \nfa{s}_0$.
  Essentially, similarity and reduction do not coincide because similarity is successor-congruent, whereas reduction is not $\limp$-congruent.

\item
  Focusing with eager inversion does not solve this problem.
  For DFAs, we would be able to prove:
  \begin{itemize}[nosep]
  \item $q$ and $s$ are bisimular if, and only if, $\dfa{q} = \dfa{s}$.
  \item $q \misa\dfareduces[a]\asim q'$ if, and only if, $a \oc \dfa{q} \reduces \dfa{q}'$.
  \end{itemize}

\item
  For NFAs, we will be able to prove:
  \begin{itemize}[nosep]
  \item $q$ and $s$ are bisimular if, and only if, $\nfa{q} \cong \nfa{s}$.
  \item $q \misa\nfareduces[a]\asim q'$ if, and only if, $a \oc \nfa{q} \cong^{-1}\reduces\cong \nfa{q}'$.
  \end{itemize}
\end{enumerate}


\subsection{Extended example: \Acp*{NFA}}

As an example of ordered rewriting, consider a specification of \acp{NFA}.
Recall from \cref{ch:automata} the \ac{NFA} (repeated in the adjacent \lcnamecref{fig:ordered-rewriting:nfa-example-ends-b})
%
\begin{marginfigure}
  \begin{equation*}
    \mathllap{\aut{A}_1 = {}}
    \begin{tikzpicture}[baseline=(q_0.base)]
      \graph [automaton] {
        q_0
         -> [loop above, "a,b"]
        q_0
         -> ["b"]
        q_1 [accepting]
         -> ["a,b"]
        q_2
         -> [loop above, "a,b"]
        q_2;
      };
    \end{tikzpicture}
  \end{equation*}
  \caption{\Iac*{NFA} that accepts, from state $q_0$, exactly those words that end with $b$. (Repeated from \cref{fig:nfa-example-ends-b}.)}\label{fig:ordered-rewriting:nfa-example-ends-b}
\end{marginfigure}%
%
that accepts exactly those words, over the alphabet $\ialph = \Set{a, b}$, that end with $b$.
We may represent that \ac{NFA} as a rewriting specification using a collection of recursive definitions, one for each of the \ac{NFA}'s states:%
\fixnote{Should I include ${} \with (\emp \limp \top)$?}
\begin{equation*}
  % \sig = \parens[size=auto]{
  \begin{lgathered}
    \nfa{q}_0 \defd (a \limp \nfa{q}_0) \with (b \limp (\nfa{q}_0 \with \nfa{q}_1)) \with (\emp \limp \top) \\
    \nfa{q}_1 \defd (a \limp \nfa{q}_2) \with (b \limp \nfa{q}_2) \with (\emp \limp \one) \\
    \nfa{q}_2 \defd (a \limp \nfa{q}_2) \with (b \limp \nfa{q}_2) \with (\emp \limp \top)
  \end{lgathered}
  % }
\end{equation*}
The \ac{NFA}'s acceptance of words is represented by the existence of traces.
For example, because the word $ab$ ends with $b$, a trace $\emp \oc b \oc a \oc \nfa{q}_0
% \Reduces \emp \oc b \oc \nfa{q}_0
% \Reduces \emp \oc \nfa{q}_1
\Reduces \octxe$ exists.
On the other hand, $\emp \oc a \oc b \oc \nfa{q}_0 \Longarrownot\Reduces \octxe$ because the word $ba$ does not end with $b$.

More generally, \iac{NFA} $\aut{A} = (Q, \mathord{\nfareduces}, F)$ over an input alphabet $\ialph$ can be represented as the ordered rewriting specification in which each state $q \in Q$ corresponds to a recursively defined proposition $\nfa{q}$:
\begin{equation*}
  \nfa{q} \defd
  \parens[size=auto]{\displaystyle
      \bigwith_{a \in \ialph}
        \parens[size=big]{a \limp \bigwith_{q'_a \in \nfapow(q,a)} \nfa{q}'_a}
    }
    \with
    \parens[size=big]{\emp \limp \nfa{F}(q)}
  \enspace\text{where\enspace
    $\nfa{F}(q) =
       \begin{cases*}
         \one & if $q \in F$ \\
         \top & if $q \notin F$\rlap{ .}
       \end{cases*}$}
\end{equation*}
After defining a representation, $\nfawds{w}$, of words $w$ (see adjacent \lcnamecref{fig:ordered-rewriting:words-represent})%
%
\begin{marginfigure}
  \begin{align*}
    \nfawds{\emp} &= \octxe \\
    \nfawds{a \wc w} &= \nfawds{w} \oc a
  \end{align*}
  \caption{Words as ordered contexts}\label{fig:ordered-rewriting:words-represent}
\end{marginfigure}%
%
, we may state and prove that ordered rewriting under these definitions is sound and complete with respect to the \ac{NFA} semantics given in \cref{ch:automata}.


\begin{theorem}
  \begin{itemize}
  \item $q \nfareduces[a] q'$ if, and only if, $a \oc \nfa{q} \Reduces \nfa{q}'$.
  \item $q \in F$ if, and only if, $\emp \oc \nfa{q} \Reduces \octxe$.
  \end{itemize}
\end{theorem}


\clearpage


\begin{falseclaim}
  Let $\aut{A} = (Q, \mathord{\nfareduces}, F)$ be \iac{NFA} over the input alphabet $\ialph$.
  Then:
  \begin{itemize}[nosep]
  \item $q \nfareduces[a]\asim s'$ if, and only if, $a \oc \nfa{q} \Reduces \nfa{s}'$.
  \item $q \asim s$ if, and only if, $\nfa{q} = \nfa{s}$.
  \end{itemize}
\end{falseclaim}
%
\begin{proof}[Counterexample]
  First, $q \asim s$ does not imply $\nfa{q} = \nfa{s}$.
  Consider the following \ac{NFA} and its corresponding definitions:
  \begin{equation*}
    \begin{tikzpicture}
      \graph [automaton] {
        q [accepting]
         -> ["a"]
        { s_1 [accepting] ->[loop right, "a" right] s_1 ,
          s_2 [accepting] ->[loop right, "a" right] s_2 };
      };
    \end{tikzpicture}
    \qquad
    \begin{lgathered}[b]
      \nfa{q} \defd (a \limp \nfa{s}_1) \with (a \limp \nfa{s}_2) \with (\emp \limp \one) \\
      \nfa{s}_1 \defd (a \limp \nfa{s}_1) \with (\emp \limp \one) \\
      \nfa{s}_2 \defd (a \limp \nfa{s}_2) \with (\emp \limp \one)
    \end{lgathered}
  \end{equation*}
  Observe that the universal binary relation on states is a bisimulation: every state has an $a$-successor and every state is an accepting state.
  Therefore, all pairs of states are bisimilar; in particular, $q \asim s_1$.
  However, $\nfa{q} \neq \nfa{s}_1$.

  Second, $a \oc \nfa{q} \Reduces \nfa{s}'$ does not imply $q \nfareduces[a]\asim s'$.
  Consider the following \ac{NFA} and its corresponding definitions:
  \begin{equation*}
    \begin{tikzpicture}
      \graph [automaton] {
        q_1 -> ["a"] q_2 [accepting]
         -> ["a"]
        { s_1 [accepting] ->[loop right, "a" right] s_1 ,
          s_2             ->[loop right, "a" right] s_2 };
      };
    \end{tikzpicture}
    \qquad
    \begin{lgathered}[b]
      \nfa{q}_1 \defd (a \limp \nfa{q}_2) \with (\emp \limp \top) \\
      \nfa{q}_2 \defd (a \limp \nfa{s}_1) \with (a \limp \nfa{s}_2) \with (\emp \limp \one) \\
      \nfa{s}_1 \defd (a \limp \nfa{s}_1) \with (\emp \limp \one) \\
      \nfa{s}_2 \defd (a \limp \nfa{s}_2) \with (\emp \limp \top)
    \end{lgathered}
  \end{equation*}
  Observe that $a \oc \nfa{q}_1 \Reduces \nfa{q}_2 \Reduces \nfa{s}_1$.
  However, $q_2 \nsim s_1$, and so $q_1 \nfareduces[a]\asim s_1$ does \emph{not} hold.
  To see why $q_2 \nsim s_1$, notice that $q_2 \nfareduces[a] s_2 \notin F$ is not matched from $s_1$, which has only $s_1 \nfareduces[a] s_1 \in F$.
\end{proof}


\begin{definition}
  A binary relation $\simu{R}$ on states is a simulation if:
  \begin{itemize}
  \item $s \simu{R}^{-1}\nfareduces[a] q'$ implies $s \nfareduces[a]\simu{R}^{-1} q'$; and
  \item $s \simu{R}^{-1} q \in F$ implies $s \in F$.
  \end{itemize}
  Similarity, $\lesssim$, is the largest simulation.
\end{definition}


\begin{lemma}
  If $\nfa{q} \secudeR \nfa{s}$, then $q \lesssim s$.
\end{lemma}
\begin{proof}
  We must check two properties:
  \begin{itemize}
  \item Suppose that $\nfa{s} \Reduces \nfa{q}$ and $q \nfareduces[a] q'_a$ for some state $q'_a$; we must show that $s \nfareduces[a] s'_a$ and $\nfa{s}'_a \secudeR \nfa{q}'_a$, for some state $s'_a$.
    According to the definition, the definiens of $\nfa{q}$ contains a clause $(a \limp \nfa{q}'_a)$.
    Because $\nfa{s} \Reduces \nfa{q}$, the definiens of $\nfa{s}$ also contains the clause $(a \limp \nfa{q}'_a)$.
    It follows that $s \nfareduces[a] q'_a$ and $\nfa{q}'_a \secudeR \nfa{q}'_a$.
  \item Suppose that $\nfa{s} \Reduces \nfa{q}$ and $q \in F$; we must show that $s \in F$.
    According to the definition, the definiens of $\nfa{q}$ contains a clause $(\emp \limp \one)$.
    Because $\nfa{s} \Reduces \nfa{q}$, the definiens of $\nfa{s}$ also contains the clause $(\emp \limp \one)$.
    It follows that $s \in F$.
  \qedhere
  \end{itemize}
\end{proof}


\begin{theorem}[Adequacy]
  Let $\aut{A} = (Q, \mathord{\nfareduces}, F)$ be \iac{NFA} over the input alphabet $\ialph$.
  If $q \nfareduces[a] q'$, then $a \oc \nfa{q} \Reduces \nfa{q}'$.
  Moreover, if $a \oc \nfa{q} \Reduces \nfa{s}'$, then $q \nfareduces[a]\gtrsim s'$.
\end{theorem}
%
\begin{proof}
  The first part follows by construction.

  To prove the second part, suppose $a \oc \nfa{q} \Reduces \nfa{s}'$.
  By the lemma, $\nfa{q} \Reduces (a \limp B) \oc \octx'_a$ and $B \oc \octx'_a \Reduces \nfa{s}'$ for some $B$ and $\octx'_a$.
  By inversion, $\octx'_a = \octxe$ and $B = \nfa{q}'_a$ for some state $q'_a$ such that $q \nfareduces[a] q'_a$.
  Therefore, $\nfa{q}'_a \Reduces \nfa{s}'$.
  By the lemma, $s' \lesssim q'_a$ and so $q \nfareduces[a]\gtrsim s'$.
\end{proof}


\begin{theorem}[Adequacy]
  Let $\aut{A} = (Q, \mathord{\nfareduces}, F)$ be \iac{NFA} over the input alphabet $\ialph$.
  Then:
  \begin{enumerate}
  \item If $q \nfareduces[a]\asim s'$, then $a \oc \nfa{q} \Reduces \nfa{s}'$.
  \item If $q \asim s$, then $\nfa{q} = \nfa{s}$.
  \item If $\nfa{q} = \nfa{s}$, then $q \asim s$.
  \item If $a \oc \nfa{q} \Reduces \nfa{s}'$, then $q \nfareduces[a]\asim s'$.
  \end{enumerate}
\end{theorem}
%
\begin{proof}
  \begin{enumerate}
  \item Suppose that $q \nfareduces[a] q' \asim s'$; we must show that $a \oc \nfa{q} \Reduces \nfa{s}'$.
    By construction, $a \oc \nfa{q} \Reduces \nfa{q}'$.
    It follows from part [...] that $\nfa{q}' = \nfa{s}'$, and so $a \oc \nfa{q} \Reduces \nfa{s}'$.

  \item Suppose $q \asim s$; we must show that $\nfa{q} = \nfa{s}$.
    \begin{itemize}
    \item Choose an arbitrary symbol $a \in \ialph$.
      If $q \nfareduces[a] q'_a$, then there exists \iac{NFA} state $s'_a$ such that $s \nfareduces[a] s'_a \misa q'_a$, and, by the coinductive hypothesis, $\nfa{q}'_a = \nfa{s}'_a$.
      Conversely, if $s \nfareduces[a] s'_a$, then there exists \iac{NFA} state $q'_a$ such that $q \nfareduces[a] q'_a$ and $\nfa{q}'_a = \nfa{s}'_a$.
    \item Also, $q$ is an accepting state if and only if $s$ is an accepting state.
    \end{itemize}
    Therefore, the definientia of $\nfa{q}$ and $\nfa{s}$ are equal, and, by the equirecursive interpretation of definitions, so are the definienda $\nfa{q}$ and $\nfa{s}$.

  \item Suppose that $\nfa{s} = \nfa{q}$ and $q \nfareduces[a] q'$; we must show that $s \nfareduces[a] s'$ and $\nfa{s}' = \nfa{q}'$, for some \ac{NFA} state $s'$.
    By its definition, the definiens of $\nfa{q}$ therefore contains the clause $(a \limp \nfa{q}')$.
    Because $\nfa{s} = \nfa{q}$, the definiens of $\nfa{s}$ must also contain a clause $(a \limp \nfa{s}')$ for some state $s'$ such that $s \nfareduces[a] s'$ and $\nfa{s}' = \nfa{q}'$.

    Symmetrically, if $\nfa{q} = \nfa{s}$ and $s \nfareduces[a] s'$, then $q \nfareduces[a] q'$ and $\nfa{q}' = \nfa{s}'$, for some state $q'$.

    
  \item Suppose $a \oc \nfa{q} \Reduces \nfa{q}'$.
    By the lemma, $a \oc \nfa{q} \Reduces (a \limp B) \oc \octx'_a$ and $B \oc \octx'_a \Reduces \nfa{q}'$ for some $B$ and $\octx'_a$.
    By inversion, $B = \nfa{q}'_a$ and $\octx'_a = \octxe$.
    Therefore, $\nfa{q}'_a \Reduces \nfa{q}'$.
    How to show that $q'_a \asim q'$?
  \end{enumerate}
\end{proof}
%
\begin{proof}
  By coinduction on $q \asim s$.
  \begin{itemize}
  \item Suppose $\nfa{s} = \nfa{q}$ and $q \nfareduces[a] q'$; we must show that $s \nfareduces[a] s'$ and $\nfa{s}' = \nfa{q}'$ for some \ac{NFA} state $s'$.
    It follows from the coinductive hypothesis that $a \oc \nfa{s} = a \oc \nfa{q} \Reduces \nfa{q}'$.
  \end{itemize}
\end{proof}
%
\begin{proof}
  In the left-to-right directions, by unrolling the definition of $\nfa{q}$ (and a structural induction on the word $w$).

  In the right-to-left directions, by structural induction on the given trace, using the following lemma:
  \begin{quotation}
    \normalsize
    If $a \oc \octx \Reduces \octx''$ and there is no $\octx''_0$ for which $\octx'' = a \oc \octx''_0$, then\\ $\octx \Reduces (a \limp B) \oc \octx'$ for some $B$ and $\octx'$ such that $B \oc \octx' \Reduces \octx''$.
  \end{quotation}

  Assume that $a \oc \nfa{q} \Reduces \nfa{q}'$.
  Using the above lemma, $\nfa{q} \Reduces (a \limp B) \oc \octx'$ for some $B$ and $\octx'$ such that $B \oc \octx' \Reduces \nfa{q}'$.
  By inversion on the trace from $\nfa{q}$, it must be that $B = \bigwith_{q'_a \in \nfapow(q,a)} \nfa{q}'_a$ and $\octx' = \octxe$.
  Further inversion on the trace from $B \oc \octx'$ establishes that $q' \in \nfapow(q,a)$ and hence $q \nfareduces[\smash{a}] q'$.
\end{proof}






\begin{equation*}
  \nfa{q} \defd
    \parens[size=auto]{\displaystyle
      \bigwith_{a \in \ialph}
        \parens[size=big]{a \limp \nfa{q}'_a \fuse \nfa{v}_a}
    }
    \with
    \parens[size=big]{\emp \limp \nfa{\sftterm}(q)}
    \enspace\text{where\enspace
      $q'_a = \sftnext(q,a)$ and
      $v_a = \sftout(q,a)$ and $v = \sftterm(q)$}
\end{equation*}


\subsection{Extended example: Binary representation of natural numbers}

As a second example, consider a rewriting specification of the binary representation of natural numbers with increment and decrement operations.

% \NewDocumentCommand \aval { m m } { #1 \approx_{\text{\normalfont\scshape v}} #2 }
% \NewDocumentCommand \ainc { m m } { #1 \approx_{\text{\normalfont\scshape i}} #2 }
% \NewDocumentCommand \adec { m m } { #1 \approx_{\text{\normalfont\scshape d}} #2 }

\NewDocumentCommand \cinc { m } { \mathbb{I}(#1) }
\NewDocumentCommand \cnat { m } { \cinc{#1} }
\NewDocumentCommand \cdec { m } { \mathbb{D}(#1) }

For this specification, a natural number is represented in binary by
% A binary representation of a natural number is
an ordered context consisting of a big-endian sequence of atoms $b_0$ and $b_1$, prefixed by the atom $e$; leading $b_0$s are permitted.
For example, both $\octx = e \oc b_1$ and $\octx = e \oc b_0 \oc b_1$ are valid binary representations of the natural number $1$.

More generally, let $\cval{}$ be the partial function from ordered contexts to natural numbers defined as follows; we say that the ordered context $\octx$ \emph{represents} natural number $n$ if $\cval{\octx} = n$.
\begin{equation*}
  \begin{lgathered}
    \cval{e} = 0 \\
    \cval{\octx \oc b_0} = 2\cval{\octx} \\
    \cval{\octx \oc b_1} = 2\cval{\octx} + 1
  \end{lgathered}
\end{equation*}
The partial function \(\cval{}\) defines an adequate representation because, up to leading $b_0$s, the natural numbers and valid binary representations (\ie, the domain of definition of $\cval{}$) are in bijective correspondence.
%
\begin{theorem}[Representational adequacy]
  For all natural numbers \(n \in \mathbb{N}\), there exists a context \(\octx\) such that \(\cval{\octx} = n\).
  Moreover, if \(\cval{\octx_1} = n\) and \(\cval{\octx_2} = n\), then \(\octx_1\) and \(\octx_2\) are identical up to leading \(b_0\)s.
\end{theorem}
\begin{proof}
  The first part follows by induction on the natural number \(n\); the second part follows by induction on the structure of the contexts \(\octx_1\) and \(\octx_2\).
\end{proof}

Next, we may describe an increment operation on these binary representations as an ordered rewriting specification; because of these increments, [...].
To indicate that an increment should be performed, a new, uninterpreted atom $i$ is introduced.
The previously uninterpreted atoms $e$, $b_0$, and $b_1$ are now given mutually recursive definitions that describe their interactions with $i$.
\begin{description}
\item[$e \defd e \fuse b_1 \pmir i$]
  To increment the counter $e$, introduce $b_1$ as a new most significant bit, resulting in the counter $e \oc b_1$.
  That is, $e \oc i \Reduces e \oc b_1$.
  Having started at value $0$ (\ie, $\cval{e} = 0$), an increment results in value $1$ (\ie, $\cval{e \oc b_1} = 1$).
\item[$b_0 \defd b_1 \pmir i$]
  To increment a counter that ends with least significant bit $b_0$, simply flip that bit to $b_1$.
  That is, $\octx \oc b_0 \oc i \Reduces \octx \oc b_1$.
  Having started at value $2n$ (\ie, $\cval{\octx \oc b_0} = 2\cval{\octx}$), an increment results in value $2n+1$ (\ie, $\cval{\octx \oc b_1} = 2\cval{\octx}+1$).
\item[$b_1 \defd i \fuse b_0 \pmir i$]
  To increment a counter that ends with least significant bit $b_1$, flip that bit to $b_0$ and propagate the increment on to the more significant bits as a carry.
  That is, $\octx \oc b_1 \oc i \Reduces \octx \oc i \oc b_0$.
  Having started at value $2n+1$ (\ie, $\cval{\octx \oc b_1} = 2\cval{\octx}+1$), an increment results in value $2n+2 = 2(n+1)$ (\ie, $\cval{\octx \oc i \oc b_0} = 2\cval{\octx}+1$).
\end{description}

As an example, consider incrementing $e \oc b_1$ twice, as captured by the state $e \oc b_1 \oc i \oc i$.
First, processing of the leftmost increment begins: the least significant bit is flipped, and the increment is carried over to the more significant bits.
This corresponds to the reduction $e \oc b_1 \oc i \oc i \Reduces e \oc i \oc b_0 \oc i$.
Next, either of the two remaining increments may be processed -- that is, either $e \oc i \oc b_0 \oc i \Reduces e \oc b_1 \oc b_0 \oc i$ or $e \oc i \oc b_0 \oc i \Reduces e \oc i \oc b_1$.

\begin{tikzcd}[]
  && e \oc b_1 \oc b_0 \oc i \drar[Reduces] &
  \\
  e \oc b_1 \oc i \oc i \rar[Reduces]
   & e \oc i \oc b_0 \oc i \urar[Reduces] \drar[Reduces] \arrow[Reduces, gray, dashed]{rr}
   && e \oc b_1 \oc b_1
  \\
   && e \oc i \oc b_1 \urar[Reduces] &
\end{tikzcd}

\begin{equation*}
  \begin{aligned}
  \MoveEqLeft[.5]
  e \oc b_1 \oc i \oc i \\
   &\Reduces e \oc i \oc b_0 \oc i \\
   &\Reduces e \oc b_1 \oc b_0 \oc i \\
   &\Reduces e \oc b_1 \oc b_1
\end{aligned}
\begin{aligned}
  \MoveEqLeft[.5]
  e \oc b_1 \oc i \oc i \\
   &\reduces e \oc (i \fuse b_0 \pmir i) \oc i \oc i
    \reduces e \oc (i \fuse b_0) \oc i
    \reduces e \oc i \oc b_0 \oc i \\
   &\reduces (e \fuse b_1 \pmir i) \oc i \oc b_0 \oc i
    \reduces (e \fuse b_1) \oc b_0 \oc i
    \reduces e \oc b_1 \oc b_0 \oc i \\
   &\reduces e \oc b_1 \oc (b_1 \pmir i) \oc i
    \reduces e \oc b_1 \oc b_1
\end{aligned}
\end{equation*}

% \begin{equation*}
%   \begin{lgathered}
%     e \defd e \fuse b_1 \pmir i \\
%     b_0 \defd b_1 \pmir i \\
%     b_1 \defd i \fuse b_0 \pmir i
%   \end{lgathered}
% \end{equation*}

% First representation, then computation.

% \begin{equation*}
%   \begin{lgathered}
%     e \defd (e \fuse b_1 \pmir i) \with (z \pmir d) \\
%     b_0 \defd (b_1 \pmir i) \with (d \fuse b'_0 \pmir d) \\
%     b_1 \defd (i \fuse b_0 \pmir i) \with (b_0 \fuse s \pmir d) \\
%     b'_0 \defd (z \limp z) \with (s \limp b_1 \fuse s)
%   \end{lgathered}
% \end{equation*}



% \begin{theorem}
%   % If $\cval{\octx} = n$, then $\octx \oc i \Reduces \octx'$ for some $\octx'$ such that $\cval{\octx'} = n+1$.
%   If $\cval{\octx} = n$ and $\octx \oc i \Reduces \octx'$, then $\octx' \Reduces \octx''$ for some $\octx''$ such that $\cval{\octx''} = n+1$.
% \end{theorem}

% \begin{equation*}
%   \begin{lgathered}
%     \cnat{e} = 0 \\
%     \cnat{\octx \oc b_0} = 2\cnat{\octx} \\
%     \cnat{\octx \oc b_1} = 2\cnat{\octx} + 1 \\
%     \cnat{\octx \oc i} = \cnat{\octx} + 1
%   \end{lgathered}
% \end{equation*}

% \begin{theorem}[Preservation]
%   If $\cnat{\octx} = n$ and $\octx \Reduces \octx'$, then $\cnat{\octx'} = n$.
% \end{theorem}

% \begin{theorem}[Progress]
%   If $\cnat{\octx} = n$, then either: $\octx \reduces \octx'$ for some $\octx'$; or $\cval{\octx} = n$.
% \end{theorem}

\clearpage

\begin{inferences}
  \infer{\aval{e}{0}}{}
  \and
  \infer{\aval{\octx \oc b_0}{2n}}{
    \aval{\octx}{n}}
  \and
  \infer{\aval{\octx \oc b_1}{2n+1}}{
    \aval{\octx}{n}}
\end{inferences}

\begin{theorem}[Adequacy]
  If \(\aval{\octx}{n}\) and \(\octx \oc i \Reduces \octx'\), then \(\octx' \Reduces \aval{}{n+1}\).
\end{theorem}
\begin{proof}
  \begin{itemize}
  \item Suppose that \(e \oc i \Reduces \octx'\); we must show that \(\octx' \Reduces \aval{}{1}\).
  \item Suppose that \(\octx \oc b_0 \oc i \Reduces \octx'\) and \(\aval{\octx}{n}\); we must show that \(\octx' \Reduces \aval{}{2n}\).
  \end{itemize}
\end{proof}


\begin{inferences}
  \infer{\ainc{\octx}{n}}{
    \aval{\octx}{n}}
  \and
  \infer{\ainc{\octx \oc i}{n+1}}{
    \ainc{\octx}{n}}
  \and
  \infer{\ainc{\octx \oc b_0}{2n}}{
    \ainc{\octx}{n}}
  \and
  \infer{\ainc{\octx \oc b_1}{2n+1}}{
    \ainc{\octx}{n}}
  \\
  \infer{\ainc{\octx_L \oc A \oc \octx_R}{n}}{
    \ainc{\octx_L \oc \alpha \oc \octx_R}{n} & (\alpha \defd A) \in \sig}
\end{inferences}

\begin{theorem}[Preservation]
  If \(\ainc{\octx}{n}\) and \(\octx \reduces \octx'\), then \(\octx' \Reduces \ainc{}{n}\).
\end{theorem}
%
\begin{proof}
  \begin{itemize}
  \item Suppose that \(\ainc{\octx_0}{n}\) and \(\octx = \octx_0 \oc i \reduces \octx'\); we must show that \(\octx' \Reduces \ainc{}{n+1}\).
    \begin{itemize}
    \item Consider the case in which \(\octx_0 \reduces \octx'_0\) and \(\octx' = \octx'_0 \oc i\).
      By the inductive hypothesis, \(\octx'_0 \Reduces \ainc{}{n}\).
      From the increment rule, it follows that \(\octx' = \octx'_0 \oc i \Reduces \ainc{}{n+1}\).
    \item Consider the case in which \(\octx_0 = \octx_L \oc (A_0 \pmir i)\) and \(\ainc{\octx_L \oc \alpha}{n}\) and \(\octx' = \octx_L \oc A_0\) such that \((\alpha \defd A_0 \pmir i) \in \sig\).
      There are three subcases:
      \begin{itemize}
      \item Consider the subcase in which \(\alpha = b_0\) and \(n = 2n_0\) and \(\ainc{\octx_L}{n_0}\).
        By inversion on the signature, \(A_0 = b_1\).
        It follows that \(\octx' = \ainc{\octx_L \oc b_1}{2n_0+1} = n+1\).
      \item Consider the subcase in which \(\alpha = b_1\) and \(n = 2n_0+1\) and \(\ainc{\octx_L}{n_0}\).
        By inversion on the signature, \(A_0 = i \fuse b_0\).
        It follows that \(\octx' = \octx_L \oc (i \fuse b_0) \reduces \ainc{\octx_L \oc i \oc b_0}{2(n_0+1)} = n+1\).
      \item Consider the subcase in which \(\alpha = e\) and \(n = 0\) and \(\octx_L = \octxe\).
        By inversion on the signature, \(A_0 = e \fuse b_1\).
        It follows that \(\octx' = e \fuse b_1 \reduces \ainc{e \oc b_1}{1} = n+1\).
      \end{itemize}
    \end{itemize}
  \end{itemize}
\end{proof}

\begin{theorem}[Progress]
  If \(\ainc{\octx}{n}\), then either: \(\octx \reduces \octx'\) for some \(\octx'\); or \(\aval{\octx}{n}\).
\end{theorem}



% \begin{equation*}
%   \begin{lgathered}
%     \cdec{\octx \oc b'_0} = 2\cdec{\octx} \\
%     \cdec{\octx \oc d} = \cnat{\octx} \\
%     \cdec{\octx \oc s} = \cnat{\octx} + 1 \\
%     \cdec{z} = 0
%   \end{lgathered}
% \end{equation*}

% \begin{theorem}
%   If \(\cinc{\octx} = n\) and \(\octx \oc d \Reduces \octx'\), then: \(\octx' \Reduces z\) if \(n = 0\); and \(\octx' \Reduces \octx'' \oc s\) for some \(\octx''\) such that \(\cinc{\octx''} = n-1\), if \(n > 0\).
% \end{theorem}

% \(\cdec{\octx'} = n\) if, and only if, \(\octx \oc d \Reduces \octx'\) for some \(\octx\) such that \(\cinc{\octx} = n\).


\begin{inferences}
  \infer{\adec{z}{0}}{}
  \and
  \infer{\adec{\octx \oc s}{n+1}}{
    \ainc{\octx}{n}}
  \and
  \infer{\adec{\octx \oc d}{n}}{
    \ainc{\octx}{n}}
  \and
  \infer{\adec{\octx \oc b'_0}{2n}}{
    \adec{\octx}{n}}
  \\
  \infer{\adec{\octx_L \oc A \oc \octx_R}{n}}{
    \adec{\octx_L \oc \alpha \oc \octx_R}{n} & (\alpha \defd A) \in \sig}
\end{inferences}

\(\adec{\octx'}{n}\) if, and only if, \(\octx \oc d \Reduces \octx'\) for some \(\octx\) such that \(\ainc{\octx}{n}\).

% \begin{theorem}
%   If \(\cinc{\octx} = n\) and \(\octx \oc d \Reduces \octx'\), then:
%   \begin{itemize}[nosep]
%   \item \(n = 2n_0\) and \(\octx' = \octx'_0 \oc b'_0 \reduces \octx''\)
%     and \(\cinc{\octx_0} = n_0\) and \(\octx_0 \oc d \Reduces \octx'_0\);
%   \item \(\cinc{\octx'_0} = n\) and \(\octx' = \octx'_0 \oc d \reduces \octx''\);
%   \item \(n = 0\) and \(\octx' = z\); or
%   \item \(n > 0\) and \(\octx' = \octx'' \oc s\) for some \(\octx''\) such that \(\cinc{\octx''} = n-1\).
%   \end{itemize}
% \end{theorem}
% %
% \begin{proof}
%   \begin{itemize}
%   \item Suppose \(\octx = e\) and \(n = 0\) and \(e \oc d \Reduces \octx'\).
%     \begin{itemize}
%     \item If the reduction is trivial, then choose \(\octx'_0 = e\) and \(\octx'' = (z \pmir d) \oc d\).
%     \end{itemize}
%   \end{itemize}
% \end{proof}


\begin{theorem}[Preservation]
  If $\adec{\octx}{n}$ and $\octx \Reduces \octx'$, then $\adec{\octx'}{n}$.
\end{theorem}

% \begin{theorem}[Preservation]
%   If $\cdec{\octx} = n$ and $\octx \Reduces \octx'$, then $\cdec{\octx'} = n$.
% \end{theorem}

% \begin{theorem}[Progress]
%   If $\cdec{\octx} = n$, then either:
%   \begin{itemize}[nosep]
%   \item $\octx \reduces \octx'$ for some $\octx'$;
%   \item $\octx = \octx' \oc s$ and $n = n' + 1$ and $\cnat{\octx'} = n'$ for some $\octx'$ and $n'$; or
%   \item $\octx = z$ and $n = 0$.
%   \end{itemize}
% \end{theorem}

\begin{theorem}[Progress]
  If $\adec{\octx}{n}$, then either:
  \begin{itemize}[nosep]
  \item $\octx \reduces \octx'$ for some $\octx'$;
  \item $\octx = \octx' \oc s$ and $n = n' + 1$ and $\ainc{\octx'}{n'}$ for some $\octx'$ and $n'$; or
  \item $\octx = z$ and $n = 0$.
  \end{itemize}
\end{theorem}


% \section{Propositional ordered rewriting}

% In this \lcnamecref{sec:ordered-rewriting:general}, we develop a rewriting interpretation of the ordered sequent calculus from the previous \lcnamecref{ch:ordered-logic}.
% This development closely follows \citeauthor{Cervesato+Scedrov:IC09}'s work on intuitionistic linear logic as a multiset rewriting framework.\autocite{Cervesato+Scedrov:IC09}

% Just as their linear logical rewriting framework is more expressive than multiset rewriting, ordered rewriting framework presented in this chapter can be seen as an extension of traditional notions of string rewriting.


% \begin{equation*}
%   \infer*{\oseq{\octx |- A}}{
%     \oseq{\octx' |- A'}}
% \end{equation*}


% Many of the ordered sequent calculus's left rules consist of a single major premise with the same consequent as in the rule's conclusion [sequent], as well as a minor premise in the case of the $\lrule{\limp}$ and $\lrule{\pmir}$ rules.
% \begin{inferences}
%   \infer[\lrule{\fuse}]{\oseq{\octx'_L \oc (A \fuse B) \oc \octx'_R |- C}}{
%     \oseq{\octx'_L \oc A \oc B \oc \octx'_R |- C}}
%   \and
%   \infer[\lrule{\with}_1]{\oseq{\octx'_L \oc (A \with B) \oc \octx'_R |- C}}{
%     \oseq{\octx'_L \oc A \oc \octx'_R |- C}}
% \end{inferences}
% Both rules, at their core, decompose resources -- the resource $A \fuse B$ into the separate resources $A \oc B$; and the resource $A \with B$ into the resource $A$.
% The resource decomposition is somewhat obscured 
% Notice that much of these two rules is devoted to shared scaffolding/boilerplate -- the framing contexts $\octx'_L$ and $\octx'_R$, and goal consequent $C$ that remain unchanged from conclusion to premise.

% Because so many rules share this scaffolding, it might be worthwhile to restructure the ordered sequent calculus to expose this shared scaffolding.
% \begin{equation*}
%   \infer{\oseq{\octx |- C}}{
%     \octx \reduces \octx' & \oseq{\octx' |- C}}
% \end{equation*}
% For instance, if $\octx_L \oc (A \fuse B) \oc \octx_R \reduces \octx_L \oc A \oc B \oc \octx_R$ holds, then the usual $\lrule{\fuse}$ rule is a derivable instance of this generalized left rule.


% \begin{theorem}
%   $\oseq{\octx |- A}$ in ... if and only if $\oseq{\octx |- A}$ in ...
% \end{theorem}
% \begin{proof}
%   The two directions are proved separately, each by induction on the structure of the given derivation.
%   \begin{gather*}
%     \infer[\lrule{\with}_1]{\oseq{\octx'_L \oc (A \with B) \oc \octx'_R |- C}}{
%       \oseq{\octx'_L \oc A \oc \octx'_R |- C}}
%     \\\rightsquigarrow\\
%     \infer[]{\oseq{\octx'_L \oc (A \with B) \oc \octx'_R |- C}}{
%       \infer[]{\octx'_L \oc (A \with B) \oc \octx'_R \reduces \octx'_L \oc A \oc \octx'_R}{
%         \infer[]{(A \with B) \oc \octx'_R \reduces A \oc \octx'_R}{
%         \infer[\lrule{\with}'_1]{A \with B \reduces A}{}}} &
%       \oseq{\octx'_L \oc A \oc \octx'_R |- C}}
%   \end{gather*}

%   \begin{equation*}
%     \begin{lgathered}
%       \bigfuse (\octx_1 \oc \octx_2) = (\bigfuse \octx_1) \fuse (\bigfuse \octx_2) \\
%       \bigfuse (\octxe) = \one \\
%       \bigfuse A = A
%     \end{lgathered}
%   \end{equation*}

%   \begin{lemma}
%     If\/ $\octx \reduces \octx'$, then $\oseq{\octx |- \bigfuse \octx'}$.
%     $\oseq{\octx' |- \bigfuse \octx'}$ for all $\octx'$.
%   \end{lemma}
% \end{proof}

% \begin{theorem}
%   If $\oseq{\octx |- A}$ and $\octx'_L \oc A \oc \octx'_R \reduces \octx'$, then $\oseq{\octx'_L \oc \octx \oc \octx'_R |- \bigfuse \octx'}$.
% \end{theorem}

% \begin{syntax*}
%   Propositions &
%     A & p \mid A \limp B \mid B \pmir A
%           \mid A \fuse B \mid \one
%           \mid A \with B \mid \top
%   \\
%   Ordered contexts & 
%     \octx & \octxe \mid \octx_1 \oc \octx_2 \mid A
% \end{syntax*}

% \begin{itemize}
% \item Lambek calculus and rewriting; compare to multiset rewriting; compare to string rewriting
% \item Explain why $\plus$ and $\zero$ (and $\bot$) are undesirable here.
% \item Connections to left rules
% \end{itemize}

% The rewriting relation is the smallest compatible relation that satisfies:
% \begin{inferences}
%   \infer{A \oc (A \limp B) \reduces B}{}
%   \and
%   \infer{(B \pmir A) \oc A \reduces B}{}
%   \\
%   \infer{A \with B \reduces A}{}
%   \and
%   \infer{A \with B \reduces B}{}
%   \and
%   \text{(no rule for $\top$)}
%   \\
%   \infer{A \fuse B \reduces A \oc B}{}
%   \and
%   \infer{\one \reduces \octxe}{}
% \end{inferences}
% We will also refer to this relation as \vocab{reduction}%
% \footnote{Input transitions are postponed to \cref{ch:ordered-bisimilarity}.}%
% .

% $\Reduces$ is the reflexive-transitive closure of $\reduces$


% \begin{equation*}
%   \infer[\lrule{\with}_1]{\oseq{\octx'_L \oc (A \with B) \oc \octx'_R |- \gamma}}{
%     \oseq{\octx'_L \oc A \oc \octx'_R |- \gamma}}
%   \leftrightsquigarrow
%   \infer{\octx'_L \oc (A \with B) \oc \octx'_R \reduces \octx'_L \oc A \oc \octx'_R}{
%     \infer{A \with B \reduces A}{}}
% \end{equation*}

% \begin{equation*}
%   \infer[\lrule{\limp}]{\oseq{\octx'_L \oc \octx \oc (A \limp B) \oc \octx'_R |- \gamma}}{
%     \oseq{\octx |- A} & \oseq{\octx'_L \oc B \oc \octx'_R |- \gamma}}
%   \rightsquigarrow
%   \infer[\lrule{\limp}']{\oseq{\octx'_L \oc A \oc (A \limp B) \oc \octx'_R |- \gamma}}{
%     \oseq{\octx'_L \oc B \oc \octx'_R |- \gamma}}
%   \leftrightsquigarrow
%   \infer{\octx'_L \oc A \oc (A \limp B) \oc \octx'_R \reduces \octx'_L \oc B \oc \octx'_R}{
%     \infer{A \oc (A \limp B) \reduces B}{}}
% \end{equation*}


% \subsection{Definitions}

% \begin{itemize}
% \item not very interesting without recursion
% \end{itemize}

\subsection{Examples}

% \paragraph*{Automata and transducers}

% \begin{equation*}
%   \begin{lgathered}[t]
%     q_0 \defd (a \limp q_0) \with (b \limp q_0 \with q_1) \\
%     q_1 \defd (a \limp q_2) \with (b \limp q_2) \with (\emp \limp \one) \\
%     q_2 \defd (a \limp q_2) \with (b \limp q_2)
%   \end{lgathered}
%   \qquad
%   \begin{lgathered}[t]
%     s_0 \defd (a \limp s_0) \with (b \limp s_1) \\
%     s_1 \defd (a \limp s_0) \with (b \limp s_1) \with (\emp \limp \one)
%   \end{lgathered}
% \end{equation*}

% \begin{equation*}
%   \nfa{q} \defd \bigwith_{a \in \ialph} \bigl({\textstyle a \limp \bigwith_{q'_a} \nfa{q}'_a}\bigr)
% \end{equation*}

% \begin{theorem}
%   Let $\aut{A} = (Q, \mathord{\nfareduces}, F)$ be \iac{NFA} over an input alphabet $\ialph$.
%   Then:
%   \begin{itemize}[nosep]
%   \item $q \nfareduces[a] q'$ if and only if $\atm{a} \oc \nfa{q} \Reduces \nfa{q}'$.
%   \item $q \in F$ if and only if $\atm{\emp} \oc \nfa{q} \Reduces \octxe$.
%   \item $q \notin F$ if and only if $\atm{\emp} \oc \nfa{q} \longarrownot\reduces$.\alertnote{Careful -- depends on focusing!}
%   \end{itemize}
%   % \item
%   %   If $\atm{a} \oc \nfa{q} \Reduces \nfa{q}'$, then $q \nfareduces[a] q'$.
%   %   If $\atm{\emp} \oc \nfa{q} \Reduces \octxe$, then $q \in F$.
% \end{theorem}


% \paragraph*{Binary counter}

% \begin{equation*}
%   \begin{lgathered}
%     e \defd (e \fuse b_1 \pmir i) \with (z \pmir d) \\
%     b_0 \defd (b_1 \pmir i) \with (d \fuse b'_0 \pmir d) \\
%     b_1 \defd (i \fuse b_0 \pmir i) \with (b_0 \fuse s \pmir d) \\
%     b'_0 \defd (z \limp z) \with (s \limp b_1 \fuse s)
%   \end{lgathered}
% \end{equation*}

\begin{itemize}
\item Alternative choreography -- how are these related?
\begin{equation*}
  \begin{lgathered}
    p \defd (i \fuse p \pmir \atmL{i}) \with (d \fuse p' \pmir \atmL{d}) \\
    p' \defd (\atmR{z} \limp \atmR{z}) \with (\atmR{s} \limp p \fuse \atmR{s}) \\
    i \defd (\atmR{e} \limp \atmR{e} \fuse \atmR{b}_1) \with (\atmR{b}_0 \limp \atmR{b}_1) \with (\atmR{b}_1 \limp i \fuse \atmR{b}_0) \\
    d \defd (\atmR{e} \limp \atmR{z}) \with (\atmR{b}_0 \limp d \fuse b'_0) \with (\atmR{b}_1 \limp \atmR{b}_0 \fuse \atmR{s}) \\
    b'_0 \defd (\atmR{z} \limp \atmR{z}) \with (\atmR{s} \limp \atmR{b}_1 \fuse \atmR{s})
  \end{lgathered}
\end{equation*}

\begin{inferences}
  \infer{\adec{\octx \oc d}{n}}{
    \ainc{\octx}{n}}
\end{inferences}

If $\octx \oc \atmL{i} \reduces \octx'$, then $\atmR{\octx} \oc i \reduces \atmR{\octx}'$.
\end{itemize}


\section{}



\chapter{A formula-as-process interpretation of ordered rewriting}\label{ch:formula-as-process}\label{ch:choreographies}


% In \cref{ch:string-rewriting}, we saw that string rewriting can be used to specify the dynamics of concurrent systems, but that those specifications are quite abstract.
% Even the operational semantics is left completely abstract: permitted rewritings just \emph{happen}, as if a central, meta-level actor schedules and otherwise coordinates rewriting.

In \cref{ch:string-rewriting}, we saw that string rewriting can be used to specify the dynamics of concurrent systems, but that those specifications are quite abstract.
Even the operational semantics is left completely abstract:
String rewriting is a state-transformation model of concurrency, with axioms $w \reduces w'$
% are strictly global in their phrasing, stating
stating merely that a substring of the form $w$ may be replaced, en masse, with $w'$.
Nothing is said about how this replacement is achieved -- permitted rewritings just \emph{happen}, as if a central, meta-level actor schedules and otherwise coordinates rewriting, with substrings and their constituent symbols as mere passive accessories.

In the previous \lcnamecref{ch:ordered-rewriting}, we presented a different rewriting framework, derived from the (focused) ordered sequent calculus and closely related to the \citeauthor{Lambek:AMM58} calculus\autocite{Lambek:AMM58}.
Ordered rewriting, in both its unfocused and focused variants, continues to leave the operational semantics abstract, as if a central, meta-level actor governs rewriting.

The string rewriting and (strongly) focused ordered rewriting frameworks are both expressive enough to exhibit concurrency% behaviors%
\footnote{See \cref{sec:string-rewriting:concurrency,sec:ordered-rewriting:concurrency}.}%
, and the ordered rewriting framework's logical foundations make it a proof-construction approach to concurrency.
But without a concrete operational semantics neither framework is yet suitable for\fixnote{suited to} our ultimate goal -- 
% a method for extracting local, message-passing implementations from concurrent specifications.
to establish the relationship between proof-construction and proof-reduction descriptions of concurrency.
%  operating at such a high level of abstraction that neither framework's operational semantics can be cleanly implemented\fixnote{word choice?} in terms of local, message-passing communication.

This \lcnamecref{ch:formula-as-process} takes three significant steps toward this end, using the focused ordered rewriting framework of the previous \lcnamecref{ch:ordered-rewriting} as a stepping stone.
\begin{itemize}[listparindent=\parindent, itemsep=\dimexpr\itemsep+\parsep\relax, parsep=0pt]
\item
  In \cref{sec:formula-as-process:interpretation}, we refine the focused ordered rewriting framework of the previous \lcnamecref{ch:ordered-rewriting} into one that can be given a \emph{formula-as-process} interpretation%
  \footnote{This interpretation is \emph{very} closely related to the process-as-formula view of concurrency put forth by \textcites{Miller:??}{Cervesato+Scedrov:IC09}.
    For us, however, the logical aspects, and propositions in particular, are conceptually prior to any notion of process, hence our use of the reversed \emph{formula-as-process} terminology.}
  in which rewriting faithfully represents message-passing communication among concurrent processes that are arranged in a linear topology.
  In this way, the formula-as-process interpretation assigns a concrete operational semantics to ordered rewriting, nudging it away from a state-transformation model of concurrency and toward a process-based model.

  Specifically, we show that atomic propositions may be interpreted as messages;
  the other, non-atomic propositions, as processes;
  contexts, as configurations comprised of those messages and processes;
  and
  rewriting, as mes\-sage-passing communication among a configuration's con\-stit\-u\-ent processes.
  %
  Perhaps surprisingly, only three small tweaks to the structure of propositions are needed to make this formula-as-process interpretation viable.


  % In \cref{sec:formula-as-process:interpretation}, we refine the focused ordered rewriting framework of the previous \lcnamecref{ch:ordered-rewriting} into one that can be cleanly given a \emph{formula-as-process} interpretation%
  % \footnote{This interpretation is \emph{very} closely related to the process-as-formula view of concurrency put forth by \textcites{Miller:??}{Cervesato+Scedrov:IC09}.
  %   For us, however, the logical aspects, and propositions in particular, are conceptually prior to any notion of process, hence our use of the reversed \emph{formula-as-process} terminology.}
  % that assigns to focused ordered rewriting a concrete operational semantics in terms of message-passing communication among concurrent processes.

  % Under this formula-as-process interpretation, atomic propositions may be viewed as messages;
  % (compound)\fixnote{$\top$ and $\one$ aren't exactly compound.} propositions, as processes;
  % contexts, as configurations comprised of those messages and processes;
  % and
  % rewriting, as message-passing communication among a configuration's con\-stit\-u\-ent processes.
  % %
  % Somewhat surprisingly, only three small tweaks to the structure of propositions are needed to make the formula-as-process interpretation viable.

\item
  With this new formula-as-process perspective and its accompanying mes\-sage-passing semantics, (focused) ordered rewriting can be understood in terms of local interactions alone.
  By analogy with the $\pi$-calculus's operational semantics, the existing rewriting relation, $\reduces$, serves as a reduction semantics, but now an equivalent, labeled transition semantics can also be given~\parencref{sec:formula-as-process:transition-semantics}.

  The labeled transition semantics describes how ordered contexts, now understood as process configurations, interact with neighboring contexts.
  It thus goes hand-in-hand with the formula-as-process interpretation in establishing a concrete and local operational semantics for ordered rewriting.

\item
  Having established a formula-as-process refinement of the focused ordered rewriting framework that permits only local, message-passing interactions, we then revisit string rewriting specifications.

  In \cref{sec:formula-as-process:choreographies}, we describe, first informally and then formally, a method for operationalizing, or \emph{choreographing}, string rewriting specifications within the formula-as-process ordered rewriting framework.
  %
  Symbols in the specification's alphabet are uniquely mapped to propositions, thereby casting each symbol in one of two roles -- either a message role or a process role.
  % Each symbol in the specification's alphabet is uniquely mapped to either an atomic or compound\fixnote{terminology?} proposition -- hence casting each symbol in either a message or process role.

  Not all such role assignments give rise to well-formed choreographies, however.
  But, for those that do, the resulting choreography adequately embeds the string rewriting specification:
  % But those well-formed choreographies adequately [...] the string rewriting specification:
  the specification's axioms are in one-to-one correspondence with the choreography's derivable ordered rewritings, as we prove in \cref{cor:formula-as-process:choreography-adequacy}.
  Stated differently, the string rewriting specification and choreography will be (strongly) bisimilar, with the role assignment being a bisimulation that witnesses their bisimilarity.

%   we prove that the role assignment that underlies a well-formed choreography is always a (strong) bisimulation between the string rewriting specification and its choreography within the formula-as-process ordered rewriting framework.

%   By mapping each symbol in a string rewriting specification's alphabet to either an 

%   String rewriting specifications are \emph{choreographed} by mapping each symbol to either an atomic proposition or a recursively defined proposition -- hence casting each symbol in either a message or process role.
%   Not all such role assignments result in meaningful choreographies, however.
%   A meaningful choreography is one in which the string rewriting specification's axioms are in one-to-one correspondence with derivable ordered rewritings of the propositions.
%   In other words, a meaningful choreography is one in which the underlying role assignment is a (strong) bisimulation between 




%   By mapping symbols to propositions, each symbol is uniquely cast in one of two roles -- either a message or a process -- in such a way that the string rewriting specification's axioms are in one-to-one correspondence with derivable ordered rewritings of the propositions.

%   We revisit string rewriting specifications and introduce a notion of \vocab{choreography}.
%   String rewriting specifications are choreographed by mapping each symbol to either an atomic proposition or a recursively defined proposition -- hence casting each symbol in either a message or process role.
%   Not all such role assignments result in meaningful choreographies, however.
%   A meaningful choreography is 

% , so we describe a judgment for evaluating a mapping.
%   We also prove that any mapping that results in a meaningful choreography acts as a (strong) bisimulation between the string rewriting specification and ordered rewriting in the choreography.
\end{itemize}

% At such a high level of abstraction, neither string rewriting specifications nor ordered rewriting [...] 

% In \cref{ch:string-rewriting}, we saw that string rewriting can be used to specify the dynamics of concurrent systems, but that those specifications are quite abstract.
% Even the [...] is left completely abstract: permitted rewritings just \emph{happen}, as if a central, meta-level actor schedules and otherwise coordinates rewriting.
% At this high level of abstraction, string rewriting specifications are not amenable to 

% In the previous \lcnamecref{ch:ordered-rewriting}, we developed a focused ordered rewriting framework, \acs{FOR}, that has more expressive power than string rewriting.\fixnote{Can I really say this?}
% Still, the [...] is left abstract, with a central, meta-level actor governing rewriting.

% Ultimately, however, our goal is a more concrete [...] that can be implemented using local message-passing communication.
% This \lcnamecref{ch:formula-as-process} takes three significant steps in that direction.




% As demonstrated in \cref{ch:string-rewriting,ch:ordered-rewriting}, string rewriting and the more expressive (focused) ordered rewriting are suitable frameworks for describing the dynamics of concurrent systems whose components have a monoidal structure.
% But, as formulated thus far, these rewriting descriptions are only abstract specifications -- they lack the clear notion of local communication and decentralized execution

% String rewriting axioms, $w \reduces w'$, and ordered rewriting steps\fixnote{wc}, $\octx \reduces \octx'$, are strictly global in their phrasing, stating merely that any substate of the form $w$ or $\octx$ may be transformed, \foreigntext{en masse}, into $w'$ or $\octx'$.
% Because nothing at all is said about how this transformation is achieved, the rewritings specified by these frameworks are [...].
% It might as well be assumed that a meta-level actor is responsible for coordinating and conducting the rewriting, with the (sub-)states and their constituent symbols or propositions as mere passive objects.

% As outlined in \cref{ch:introduction}, our eventual goal is to relate these rewriting-based specifications to (session-typed) message-passing concurrent processes.
% With an eye on this goal, the contributions of this \lcnamecref{ch:formula-as-process} push ordered rewriting toward a lower level of abstraction.

% First, in \cref{sec:formula-as-process:interpretation}, we refine the focused ordered rewriting framework of the previous \lcnamecref{ch:ordered-rewriting} into one that can be cleanly given a \emph{formula-as-process} interpretation\autocites{Miller:??}{Cervesato+Scedrov:IC09}.
% Under this interpretation, atomic propositions may be viewed as messages; (compound) propositions, as processes; contexts, as configurations of those processes; and rewriting, as message-passing communication [among the processes in a configuration].\fixnote{fix}
% Somewhat surprisingly, only two small tweaks (and one optional addition) to the structure of propositions are needed to make this formula-as-process interpretation viable.

% Second, with the new perspective that this formula-as-process interpretation supplies, we can now understand ordered rewriting in terms of local interactions.
% \Cref{sec:formula-as-process:interaction} presents a labeled transition semantics for

% \ac{FOR}
% \Ac{FOR}

% Finally, in \cref{sec:formula-as-process:choreographies}, we introduce a notion of \emph{choreography} that relates a string rewriting specification to 

% Another way to view this chapter is that it nudges ordered rewriting away from a state-transformation model of concurrency toward a process-based model.
Even though this \lcnamecref{ch:formula-as-process} introduces a notion of process, it should be noted that computation is still driven by derivability and proof construction, not by proof reduction.
Only in \cref{part:proof-reduction} will we begin to examine a proof-reduction account of concurrency.


% \newthought{In process calculi} like the $\pi$-calculus, a reduction semantics often plays a lesser role than a labeled transition semantics does.
% But here the roles are reversed, with the labeled transition semantics taking a back seat, because of the reduction semantics's clearer connection to rewriting.

% The other contribution of this \lcnamecref{ch:choreographies} is the idea of choreographing string rewriting specifications into this formula-as-process refinement of ordered rewriting.

% This formula-as-process interpretation nudges ordered rewriting away from a state-transformation model of concurrency toward a process-based model.
% Computation is still driven by derivability



% \newthought{As shown in} \cref{ch:string-rewriting}, string rewriting is a suitable framework for describing the dynamics of concurrent systems whose components have a monoidal structure.
% But these string rewriting descriptions are only abstract specifications from a state-transformation perspective -- they lack a clear notion of local, decentralized execution and therefore give only a global characterization of the interactions between components.

% String rewriting axioms $w \reduces w'$ are strictly global in their phrasing, stating merely that any substring of the form $w$ may be replaced, en masse, with $w'$ -- nothing is said about how this replacement is achieved.
% The string rewriting framework therefore implicitly suggests that a meta-level actor is responsible for coordinating and conducting the rewriting, with substrings and their constituent symbols as mere passive accessories\fixnote{word choice?}.

% In this \lcnamecref{ch:choreographies}, our goal is to move toward a lower level of abstraction.


% As an example, recall from \cref{ch:string-rewriting} the string rewriting specification of a system that may transform strings that end with $b$ into the empty string:
% \begin{equation}
%   \infer{a \wc b \reduces b}{}
%   \qquad
%   \infer{b \reduces \emp}{}
% \end{equation}


% Accordingly, we refine the focused ordered rewriting framework of the previous \lcnamecref{ch:ordered-rewriting} into one that can be given a \vocab{formula-as-process} interpretation in which [ordered] rewriting faithfully represents message-passing communication among processes that are arranged in a linear topology.
% Then, in \cref{??}, we describe how a string rewriting specification may be transformed into an ordered rewriting \vocab{choreography}.

% In this \lcnamecref{ch:choreographies}, we refine the focused ordered rewriting framework of the previous \lcnamecref{ch:ordered-rewriting} into one that can be given a \vocab{formula-as-process} interpretation in which rewriting faithfully represents message-passing communication among 

% As argued 

% Then we show how, given a mapping of symbols to either [...], choreographies can be generated from a string rewriting specification.



\section{Refining ordered rewriting: A formula-as-process interpretation}\label{sec:formula-as-process:interpretation}

In this section, we present the formula-as-process interpretation of focused ordered rewriting sketched above.

More specifically,
% under this formula-as-process interpretation,
(positive) atoms, $\p{a}$, may be viewed as messages, and negative propositions, $\n{A}$, as processes that receive and react to those messages.
Ordered contexts, $\np{\octx}$, which consist of negative propositions and (positive) atoms, are then linear-topology run-time configurations of processes and messages.
And positive propositions, $\p{A}$, which reify ordered contexts as propositions, are process expressions that reify configurations.
Lastly, but most importantly, the rewriting relation, $\reduces$, is viewed as a reduction semantics for message-passing communication among the processes in a configuration.%
\begin{margintable}
  \begin{center}
    \begin{tabular}{@{}l@{\enspace\ }>{\itshape}l@{}}
      $\p{\atmL{a}}$ & left-directed message \\
      $\p{\atmR{a}}$ & right-directed message \\
      $\n{A}$ & message-passing process \\
      $\np{\octx}$ & run-time process configuration \\
      $\p{A}$ & configuration reified as an expression
    \end{tabular}
  \end{center}
  \caption{A formula-as-process interpretation of polarized ordered propositions and contexts}\label{fig:choreographies:propctx-table}
\end{margintable}%

Perhaps surprisingly, only three small tweaks to the structure of propositions are needed to make this formula-as-process reading viable.
\begin{itemize}
\item
  The (positive) atoms are now partitioned into two classes, left- and right-directed atoms, to allow us to identify the direction in which a message is flowing.

\item
  The left- and right-handed implications are now restricted to have atomic premises with a complementary direction
  %
  , so that they may then be cleanly interpreted as input processes that receive  individual incoming messages.
%  Instead of the more general $\p{A} \limp \n{B}$ and $\n{B} \pmir \p{A}$, now only implications of the forms $\p{\atmR{a}} \limp \n{B}$ and $\n{B} \pmir \p{\atmL{a}}$ are permitted.
%  These restricted implications can then be cleanly interpreted as input processes that receive a single incoming message.
  % The proposition $\p{\atmR{a}} \limp \n{B}$ is interpreted as a process that inputs 

\item 
  The negative propositions are extended with coinductively defined propositions, $\n{\defp{p}} \defd \n{A}$, that will correspond to recursive processes.
\end{itemize}
Together, the first two of these tweaks serve to provide a modicum of static typing for the otherwise untyped processes, as we will discuss in further detail in \cref{sec:formula-as-process:typing}.

\newthought{Positive atoms}, as mentioned previously, are now partitioned into two classes, left- and right-directed atoms, to allow us to identify the direction in which a message is flowing.
Using a partition, we ensure that each atom has a unique direction that is consistently used across all instances of that atom.

These directions are denoted by an arrow placed below the atom:
Left-directed atoms, $\p{\atmL{a}}$, are messages that are being sent to the left; right-directed atoms, $\p{\atmR{a}}$, are messages that are being sent to the right.
We will often elide the polarity annotation, writing $\atmL{a}$ and $\atmR{a}$ in place of $\p{\atmL{a}}$ and $\p{\atmR{a}}$.

\newthought{Negative propositions}, $\n{A}$, are processes that receive and react to messages.
\begin{equation*}
  \n{A} , \n{B} \Coloneqq \p{\atmR{a}} \limp \n{B} \mid \n{B} \pmir \p{\atmL{a}} \mid \n{A} \with \n{B} \mid \top \mid \up \p{A} \mid \n{\defp{p}}
\end{equation*}%
%
\begin{margintable}
  \begin{center}
    \begin{tabular}{@{}r@{\enspace}>{\itshape}l@{}}
      $\p{\atmR{a}} \limp \n{B}$ & receive message $\p{\atmR{a}}$ from the right \\
      $\n{B} \pmir \p{\atmL{a}}$ & receive message $\p{\atmL{a}}$ from the left \\
      $\n{A} \with \n{B}$ & nondeterministic branching \\% continue as $\n{A}$ or $\n{B}$ \\
      $\top\hphantom{\n{}}$ & stuck process \\
      $\up \p{A}$ & quoted configuration \\
      $\n{\defp{p}}$ & call a recursively defined process
    \end{tabular}
  \end{center}
  \caption{A formula-as-process interpretation of negative propositions}\label{fig:choreographies:negprop-table}
\end{margintable}%
%
% \begin{itemize}
% \item
  Instead of the more general $\p{A} \limp \n{B}$ and $\n{B} \pmir \p{A}$, left- and right-handed implications are now restricted to be only $\p{\atmR{a}} \limp \n{B}$ and $\n{B} \pmir \p{\atmL{a}}$.
  These propositions are then interpreted as input processes:
  $\p{\atmR{a}} \limp \n{B}$ is a process that waits to receive a message, $\p{\atmR{a}}$, from its left-hand neighbor and then continues as $\n{B}$; symmetrically, $\n{B} \pmir \p{\atmL{a}}$ is a process that awaits message $\p{\atmL{a}}$ from its right-hand neighbor.
  Because implications are restricted to atoms with \emph{complementary} direction, processes cannot re-capture messages that they just sent to other processes.%
\footnote{See \cref{sec:formula-as-process:comments} for more discussion.}

% \item
  The proposition $\n{A} \with \n{B}$ is interpreted as a process that branches nondeterministically, continuing as either $\n{A}$ or $\n{B}$.
  And $\top$, as the nullary form of $\with$, is a stuck process that cannot continue.
% \item
  The proposition $\up \p{A}$ is interpreted as a process that holds a suspended, or quoted, configuration:
  when the process $\up \p{A}$ is executed, it unfolds to that configuration.
% \item
  Lastly, $\n{\defp{p}}$ is a coinductively defined negative proposition that is interpreted as a recursive process, which we discuss in more detail in \cref{sec:formula-as-process:coinductive}.
% \end{itemize}

\newthought{Ordered contexts}, $\np{\octx}$, are interpreted as linear-topology run-time configurations of processes and the messages that pass between them.
\begin{align*}
  \np{\octx} , \np{\lctx} &\Coloneqq \np{\octx}_1 \oc \np{\octx}_2 \mid \octxe \mid \oante \\
  \oante &\Coloneqq \n{A} \mid \p{\atmL{a}} \mid \p{\atmR{a}}
\end{align*}
Just as ordered contexts form a monoid over negative propositions and positive atoms, their formula-as-process interpretation forms a monoid over processes and messages.
%
\begin{margintable}
  \begin{center}
    \begin{tabular}{@{}r@{\enspace}>{\itshape}l@{}}
      $\np{\octx}_1 \oc \np{\octx}_2$ & parallel composition of configurations \\
      $(\octxe)$ & empty configuration \\
      $\n{A}$ & single-process configuration \\
      $\p{\atmL{a}}$ & left-directed message \\
      $\p{\atmR{a}}$ & right-directed message
    \end{tabular}
  \end{center}
  \caption{A formula-as-process interpretation of contexts}\label{fig:choreographies:ctxprop-table}
\end{margintable}%
%
The monoid operation is now parallel, end-to-end composition of process configurations: $\np{\octx}_1 \oc \np{\octx}_2$ composes the configurations $\np{\octx}_1$ and $\np{\octx}_2$ so that they may interact along their mutual interface.
The empty context, $(\octxe)$, is now the empty configuration.

As usual, we do not distinguish configurations that are equivalent up to the monoid's associativity and unit laws.
This equivalence acts as an implicit structural congruence, of the sort found more explicitly in the $\pi$-calculus.

With the introduction of atom directions, it will often be useful to describe \emph{message contexts}, contexts that contain only atoms of one direction or the other.
We will use the metavariables $\atmL{\octx}$ and $\atmR{\octx}$ for those contexts that contain only left- and right-directed atoms, respectively.
More precisely, these are generated by the grammars 
\begin{equation*}
  \atmL{\octx} , \atmL{\lctx} \Coloneqq \atmL{\octx}_1 \oc \atmL{\octx}_2 \mid \octxe \mid \p{\atmL{a}}
  \qquad\text{and}\qquad
  \atmR{\octx} , \atmR{\lctx} \Coloneqq \atmR{\octx}_1 \oc \atmR{\octx}_2 \mid \octxe \mid \p{\atmR{a}}
  \,.
\end{equation*}


\newthought{Positive propositions}, $\p{A}$, are process expressions that reify run-time configurations $\np{\octx}$ as static objects.
\begin{equation*}
  \p{A} , \p{B} \Coloneqq \p{\atmL{a}} \mid \p{\atmR{a}} \mid \p{A} \fuse \p{B} \mid \one \mid \dn \n{A}
\end{equation*}
This reification is expressed as $\bigfuse \np{\octx} = \p{A}$, as defined in the adjacent \lcnamecref{fig:formula-as-process:reify-configuration}.%
\begin{marginfigure}
  \begin{align*}
    \bigfuse \p{\atmL{a}} &= \p{\atmL{a}} \\
    \bigfuse \p{\atmR{a}} &= \p{\atmR{a}} \\
    \bigfuse (\np{\octx}_1 \oc \np{\octx}_2) &= (\bigfuse \np{\octx}_1) \fuse (\bigfuse \np{\octx}_2) \\
    \bigfuse (\octxe) &= \one \\
    \bigfuse \n{A} &= \dn \n{A}
  \end{align*}
  \caption{Reifying a configuration as a process}\label{fig:formula-as-process:reify-configuration}
\end{marginfigure}%
This operation is the same as the one defined in \cref{fig:ordered-rewriting:polarized-bigfuse}, except that atom directions are preserved.

The propositions $\p{\atmL{a}}$ and $\p{\atmR{a}}$ are the expressions for left- and right-directed messages.
The proposition $\p{A} \fuse \p{B}$ reifies parallel, end-to-end composition of configurations: $\p{A} \fuse \p{B}$ is now interpreted as the expression for a process that spawns a new process, $\p{A}$, and then continues as $\p{B}$.
And $\one$ is interpreted as the expression for a process that immediately terminates, thereby reifying the empty configuration, $(\octxe)$.
Lastly, the proposition $\dn \n{A}$ is interpreted as the expression for a quoted process: executing that expression will result in the running process $\n{A}$.
%
\begin{margintable}
  \begin{center}
    \begin{tabular}{@{}r@{\enspace}>{\itshape}l@{}}
      $\p{\atmL{a}}$ & left-directed message \\
      $\p{\atmR{a}}$ & right-directed message \\
      $\p{A} \fuse \p{B}$ & parallel composition of $\p{A}$ and $\p{B}$ \\
      $\one\hphantom{\p{}}$ & terminating process \\
      $\dn \n{A}$ & quoted process
    \end{tabular}
  \end{center}
  \caption{A formula-as-process interpretation of positive propositions}\label{fig:choreographies:posprop-table}
\end{margintable}

Notice that there is no propositional connective that corresponds to a \emph{send} operation.
Sending a message is instead accomplished by spawning a process that is itself only a message: $\p{\atmL{a}} \fuse \p{B}$, $\p{B} \fuse \p{\atmR{a}}$, $\p{\atmR{a}} \fuse \p{B}$, and $\p{B} \fuse \p{\atmL{a}}$ are all possible but built from other propositional forms.
This is analogous to the way that the asynchronous $\pi$-calculus sends a message by parallel composition, as in $\bar{x}\langle y\rangle \mid P$.
Treating send operations this way makes the formula-as-process ordered rewriting framework an asynchronous calculus.

\subsection{Focused ordered rewriting as message-passing communication}

The three tweaks introduced by the formula-as-process interpretation to the structure of propositions -- especially atom directions and atomic premises for implications -- trickle down through the right- and left-focus judgments used to define rewriting:
\begin{itemize}[listparindent=\parindent, itemsep=\dimexpr\itemsep+\parsep\relax, parsep=0pt]
\item
First, because each positive atom is now marked with a direction, the $\jrule{ID}\smash{^{\p{a}}}$ rule%
\marginnote{\qquad%
  $\infer[\jrule{ID}\smash{^{\p{a}}}]{\rfocus{\p{a}}{\p{a}}}{}$%
}
that was previously part of the right-focus judgment's definition is replaced by two similar rules -- one for each direction:
\begin{equation*}
  \infer[\jrule{ID}\smash{^{\p{\atmL{a}}}}]{\rfocus{\p{\atmL{a}}}{\p{\atmL{a}}}}{}
  \qquad\text{and}\qquad
  \infer[\jrule{ID}\smash{^{\p{\atmR{a}}}}]{\rfocus{\p{\atmR{a}}}{\p{\atmR{a}}}}{}
  \,.
\end{equation*}
The other right-focusing rules remain unchanged.

\item
Second, because $\p{\atmR{a}} \limp \p{B}$ and $\n{B} \pmir \p{\atmL{a}}$ are now the only valid forms of implications, the left-focus judgment and its rules may be refined.
% left-handed implications now have only right-directed atoms as premises and right-handed implications have only left-directed atoms as premises, the left-focus judgment and its rules must be revised.
Instead of $\lfocus{\np{\octx}_L}{\n{A}}{\np{\octx}_R}{\p{C}}$, which has arbitrary contexts to the left and right of $\n{A}$, the judgment is now $\lfocus{\atmR{\octx}_L}{\n{A}}{\atmL{\octx}_R}{\p{C}}$ -- the left-hand context consists only of right-directed atoms, hence $\atmR{\octx}_L$; symmetrically, the right-hand context consists only of left-directed atoms, hence $\atmL{\octx}_R$.
The left-focus rules for the left- and right-handed implications are also revised to
\begin{inferences}
  \infer[\lrule{\limp}']{\lfocus{\atmR{\octx}_L \oc \p{\atmR{a}}}{\p{\atmR{a}} \limp \n{B}}{\atmL{\octx}_R}{\p{C}}}{
    \lfocus{\atmR{\octx}_L}{\n{B}}{\atmL{\octx}_R}{\p{C}}}
  \and\!\raisebox{1.25ex}{and}\and
  \infer[\lrule{\pmir}']{\lfocus{\atmR{\octx}_L}{\n{B} \pmir \p{\atmL{a}}}{\p{\atmL{a}} \oc \atmL{\octx}_R}{\p{C}}}{
    \lfocus{\atmR{\octx}_L}{\n{B}}{\atmL{\octx}_R}{\p{C}}}
  \,,
\end{inferences}
which are the derivable from the earlier $\lrule{\limp}$ and $\lrule{\pmir}$ rules, as shown in the adjacent \lcnamecref{fig:formula-as-process:limp-focus}.%
\begin{marginfigure}
  \begin{gather*}
    \infer[\mathrlap{\lrule{\limp}}]{\lfocus{\atmR{\octx}_L \oc \p{\atmR{a}}}{\p{\atmR{a}} \limp \n{B}}{\atmL{\octx}_R}{\p{C}}}{
      \infer[\jrule{ID}\smash{^{\p{\atmR{a}}}}]{\rfocus{\p{\atmR{a}}}{\p{\atmR{a}}}}{} &
      \lfocus{\atmR{\octx}_L}{\n{B}}{\atmL{\octx}_R}{\p{C}}}
    %
    \\\downsquigarrow\\
    %
    \infer[\mathrlap{\lrule{\limp}'}]{\lfocus{\atmR{\octx}_L \oc \p{\atmR{a}}}{\p{\atmR{a}} \limp \n{B}}{\atmL{\octx}_R}{\p{C}}}{
      \lfocus{\atmR{\octx}_L}{\n{B}}{\atmL{\octx}_R}{\p{C}}}
  \end{gather*}
  \caption{Deriving the $\lrule{\limp}'$ left focus rule}\label{fig:formula-as-process:limp-focus}
\end{marginfigure}

% With the restriction to atomic premises, we need to reconsider the left-focus rules for the left- and right-handed implications.
% By inversion under the previous set of left-focus rules, any derivation focused on $\atmR{a} \limp \n{B}$ would end with
% By similar reasoning, we arrive at a $\lrule{\pmir}'$ rule, as well:
% \begin{equation*}
%   \infer[\lrule{\pmir}']{\lfocus{\atmR{\octx}_L}{\n{B} \pmir \atmL{a}}{\atmL{a} \oc \atmL{\octx}_R}{\p{C}}}{
%     \lfocus{\atmR{\octx}_L}{\n{B}}{\atmL{\octx}_R}{\p{C}}}
% \end{equation*}
The other rules for the left-focus judgment remain fundamentally unchanged, save for the fact that the left- and right-hand contexts now contain only atoms of the complementary direction.
\end{itemize}
% \Cref{fig:??} summarizes the revised rules for the right- and left-focus judgments.
% 
Having refined the left-focus judgment to use message contexts, we may similarly refine the principal reduction rule, $\jrule{$\reduces$I}$:
% and $\jrule{$\reduces$C}$:
\begin{equation*}
  \infer[\jrule{$\reduces$I}]{\atmR{\octx}_L \oc \n{A} \oc \atmL{\octx}_R \reduces \np{\octx'{}}}{
    \lfocus{\atmR{\octx}_L}{\n{A}}{\atmL{\octx}_R}{\p{B}} &
    \rfocus{\np{\octx'{}}}{\p{B}}}
  % \and
  % \infer[\jrule{$\reduces$C}]{\np{\octx}_L \oc \np{\octx} \oc \np{\octx}_R \reduces \np{\octx}_L \oc \np{\octx'{}} \oc \np{\octx}_R}{
  %   \np{\octx} \reduces \np{\octx'{}}}
  \,.
\end{equation*}
The compatibility rule, $\jrule{$\reduces$C}$, remains unchanged.
\Cref{fig:formula-as-process:framework} summarizes the revised rules for the formula-as-process ordered rewriting framework.

\begin{figure}[tbp]
  \vspace*{\dimexpr-\abovedisplayskip-\abovecaptionskip\relax}
  \begin{syntax*}
    Positive props. &
      \p{A} , \p{B} & \p{\atmL{a}} \mid \p{\atmR{a}} \mid \p{A} \fuse \p{B} \mid \one \mid \dn \n{A}
    \\
    Negative props. &
      \n{A} , \n{B} & \p{\atmR{a}} \limp \n{B} \mid \n{B} \pmir \p{\atmL{a}} \mid \n{A} \with \n{B} \mid \top \mid \up \p{A} \mid \n{\defp{p}}
    \\
    Contexts &
      \np{\octx} , \np{\lctx} & \np{\octx}_1 \oc \np{\octx}_2 \mid \octxe \mid \n{A} \mid \p{\atmL{a}} \mid \p{\atmR{a}}
    \\
    Signatures &
      \orsig & \orsige \mid \orsig, \n{\defp{p}} \defd \n{A}
  \end{syntax*}
  \begin{inferences}
    \infer[\rrule{\fuse}]{\rfocus{\np{\octx}_1 \oc \np{\octx}_2}{\p{A} \fuse \p{B}}}{
      \rfocus{\np{\octx}_1}{\p{A}} & \rfocus{\np{\octx}_2}{\p{B}}}
    \and
    \infer[\rrule{\one}]{\rfocus{\octxe}{\one}}{}
    \\
    \infer[\jrule{ID}\smash{^{\p{\atmL{a}}}}]{\rfocus{\p{\atmL{a}}}{\p{\atmL{a}}}}{}
    \and
    \infer[\jrule{ID}\smash{^{\p{\atmR{a}}}}]{\rfocus{\p{\atmR{a}}}{\p{\atmR{a}}}}{}
    \and
    \infer[\rrule{\dn}]{\rfocus{\n{A}}{\dn \n{A}}}{}
  \end{inferences}
  \begin{inferences}
    \infer[\lrule{\limp}']{\lfocus{\atmR{\octx}_L \oc \p{\atmR{a}}}{\p{\atmR{a}} \limp \n{B}}{\atmL{\octx}_R}{\p{C}}}{
      \lfocus{\atmR{\octx}_L}{\n{B}}{\atmL{\octx}_R}{\p{C}}}
    \and
    \infer[\lrule{\pmir}']{\lfocus{\atmR{\octx}_L}{\n{B} \pmir \p{\atmL{a}}}{\atmL{\octx}_R}{\p{C}}}{
      \lfocus{\atmR{\octx}_L}{\n{B}}{\atmL{\octx}_R}{\p{C}}}
    \\
    \infer[\lrule{\with}_1]{\lfocus{\atmR{\octx}_L}{\n{A} \with \n{B}}{\atmL{\octx}_R}{\p{C}}}{
      \lfocus{\atmR{\octx}_L}{\n{A}}{\atmL{\octx}_R}{\p{C}}}
    \and
    \infer[\lrule{\with}_2]{\lfocus{\atmR{\octx}_L}{\n{A} \with \n{B}}{\atmL{\octx}_R}{\p{C}}}{
      \lfocus{\atmR{\octx}_L}{\n{B}}{\atmL{\octx}_R}{\p{C}}}
    \and
    \text{(no $\lrule{\top}$ rule)}
    \\
    \infer[\lrule{\up}]{\lfocus{}{\up \p{A}}{}{\p{A}}}{}
  \end{inferences}
  \begin{inferences}
    \infer[\jrule{$\reduces$I}]{\atmR{\octx}_L \oc \n{A} \oc \atmL{\octx}_R \reduces \octx'}{
      \lfocus{\atmR{\octx}_L}{\n{A}}{\atmL{\octx}_R}{\p{C}} &
      \rfocus{\octx'}{\p{C}}}
    \and
    \infer[\jrule{$\reduces$C}]{\octx_L \oc \octx \oc \octx_R \reduces \octx_L \oc \octx' \oc \octx_R}{
      \octx \reduces \octx'}
  \end{inferences}
  \begin{inferences}
    \infer[\jrule{$\Reduces$R}]{\octx \Reduces \octx}{}
    \and
    \infer[\jrule{$\Reduces$T}]{\octx \Reduces \octx''}{
      \octx \reduces \octx' & \octx' \Reduces \octx''}
  \end{inferences}
  \caption{A formula-as-process ordered rewriting framework}\label{fig:formula-as-process:framework}
\end{figure}

\clearpage
\subsection{Comments}\label{sec:formula-as-process:comments}\label{sec:formula-as-process:typing}

Now we are in a position to understand how the two principal syntactic changes -- atom directions and atomic premises for implications -- combine to endow the otherwise untyped processes with a modicum of static typing.

In the expression $\dn \n{A} \fuse \p{\atmR{a}}$, the atom $\p{\atmR{a}}$ is an outgoing message, owing to its direction away from the (quoted) process $\n{A}$.
If the premises of left- and right-handed implications were \emph{not} restricted to atoms of complementary direction, then $\n{A}$ might possibly be the input process $\up \dn \n{B} \pmir \p{\atmR{a}}$, which could incorrectly (re-)capture the outgoing message, $\p{\atmR{a}}$, that it just sent:
\begin{equation*}
  (\up \dn \n{B} \pmir \p{\atmR{a}}) \fuse \p{\atmR{a}}
    \reduces (\up \dn \n{B} \pmir \p{\atmR{a}}) \oc \p{\atmR{a}}
    \reduces \n{B}
  \,.
\end{equation*}
However, because the premises of left- and right-handed implications are indeed restricted to atoms of complementary direction, this scenario is impossible -- $\up \dn \n{B} \pmir \p{\atmR{a}}$ is not even a well-formed proposition!%

Formally, this property that outgoing messages cannot be recaptured is established the following \lcnamecref{thm:formula-as-process:no-recapture-outputs}.
\begin{theorem}\label{thm:formula-as-process:no-recapture-outputs}
  If $\octx = \atmL{\octx}_L \oc \octx_0 \oc \atmR{\octx}_R \Reduces \octx'$, then $\octx' = \atmL{\octx}_L \oc \octx_0 \oc \atmR{\octx}_R$ for some context $\octx'_0$ such that $\octx_0 \Reduces \octx'_0$.
\end{theorem}
\begin{proof}
  By induction on the structure of the given rewriting sequence, using inversion on the individual rewriting steps.
\end{proof}

As a related consequence of these syntactic restrictions, there is no contention for messages.
Without these restrictions, the above trace could be adapted to one in which a race could arise between two processes contending for the same message:\fixnote{fix}
\begin{equation*}
  \hphantom{(\up \dn \n{B} \pmir \p{a}) \oc \p{a} \oc (\p{a} \limp \up \dn \n{C})}
  \begin{tikzcd}[row sep=tiny, column sep=small]
    & \n{B} \\
    \mathllap{(\up \dn \n{B} \pmir \p{\atmR{a}}) \oc \p{\atmR{a}} \oc (\p{\atmR{a}} \limp \up \dn \n{C})}
      \urar[reduces] \drar[reduces] \\
    & \n{C}
  \end{tikzcd}
\end{equation*}
However, with these restrictions in place, there is no message -- neither $\p{\atmL{a}}$ nor $\p{\atmR{a}}$ -- that can cause contention between $\up \n{B} \pmir \p{\atmL{a}}$ and $\p{\atmR{a}} \limp \up \n{C}$ because $\p{\atmL{a}} \neq \p{\atmR{a}}$, that is, there is no one atom that may have both directions.
% \footnote{%
  % That is:
  % \begin{itemize}
  % \item $(\up \n{B} \pmir \p{\atmL{a}}) \oc \p{\atmL{a}} \oc (\p{\atmR{a}} \limp \up \n{C}) \reduces \np{\octx'{}}$ only if $\np{\octx'{}} = \n{B} \oc (\p{\atmR{a}} \limp \up \n{C})$; and
  % \item $(\up \n{B} \pmir \p{\atmL{a}}) \oc \p{\atmR{a}} \oc (\p{\atmR{a}} \limp \up \n{C}) \reduces \np{\octx'{}}$ only if $\np{\octx'{}} = (\up \n{B} \pmir \p{\atmL{a}}) \oc \n{C}$.
  % \end{itemize}}

Even with the restriction of left- and right-handed implication premises to atoms of complementary direction, it is nevertheless possible for a process to send \emph{itself} a message, as in
\begin{equation*}
  (\up \dn \n{B} \pmir \p{\atmL{a}}) \fuse \p{\atmL{a}}
    \reduces (\up \dn \n{B} \pmir \p{\atmL{a}}) \oc \p{\atmL{a}}
    \reduces \n{B}
  \,,
\end{equation*}
but this is not troubling because the intended recipient -- the process itself -- does indeed receive the message.

\newthought{Ordered conjunctions} are dual to the left- and right-handed implications.
So one might think that ordered conjunctions ought to be restricted to those of the form $\p{\atmL{a}} \fuse \p{B}$ and $\p{B} \fuse \p{\atmR{a}}$, as a kind of dual restriction to those placed on implications.
Just as the implications are restricted to receive only incoming messages, these ordered conjunctions would restrict processes to sending only outgoing messages.

Although certainly possible, such restrictions would limit the expressiveness of formula-as-process ordered rewriting by precluding a process from sending itself a message -- $(\up \dn \n{B} \pmir \p{\atmL{a}}) \fuse \p{\atmL{a}}$ would not be well-formed, for example.
Moreover, in \cref{ch:correspond}, we will present a correspondence between ordered propositions and the not-yet-introduced singleton proofs~\parencref{ch:singleton-logic,ch:process-chains}, which will turn out to be most direct if we retain the general $\p{A} \fuse \p{B}$ form for ordered conjunctions.
And finally, the general $\p{A} \fuse \p{B}$ form is more in line with the asynchronous nature of ordered rewriting.
For all of these reasons, we choose not to impose any restrictions on ordered conjunctions.


% \subsection{Left- and right-directed atoms as directed messages}

% The first tweak to the structure of propositions is that the positive atoms are now partitioned into two classes: left- and right-directed atoms.
% These directions, which we denote by an arrow placed below the atom, indicate the direction in which the corresponding message flows.
% Left-directed atoms, $\p{\atmL{a}}$, are messages that are being sent to the left; right-directed atoms, $\p{\atmR{a}}$, are messages that are being sent to the right.

% \subsection{Implications restricted to atomic premises as input processes}

\subsection{Coinductively defined negative propositions}\label{sec:formula-as-process:coinductive}

Recall from \cref{ch:ordered-rewriting} that \acl{FOR} is terminating: for all ordered contexts $\np{\octx}$, every rewriting sequence from $\np{\octx}$ is finite~\parencref{thm:ordered-rewriting:termination-focused}.
Although a seemingly pleasant property, termination significantly limits the expressiveness of \acl{FOR}.
For example, without unbounded rewriting, we cannot even describe producer--consumer systems or finite automata.

As the proof of termination shows, rewriting is bounded precisely because contexts consist of finitely many \emph{finite} propositions.
In multiset and ordered rewriting, unbounded behavior is traditionally introduced by way of persistent propositions that may be replicated as much as needed\autocites{Polakow:CMU01}{Watkins+:CMU02}{Simmons:CMU12}.
This is related to \citeauthor{Milner:??}'s use of replication, $!P$, in the $\pi$-calculus\autocite{Milner:??}.
(See \cref{??} for a more detailed discussion of replication.)

However, another option -- and the one that we pursue here -- is to permit circular negative propositions in the form of mutually coinductive definitions, $\n{\defp{p}} \defd \n{A}$, where the grammar of negative propositions is extended to include these coinductively defined propositions:
\begin{equation*}
  \n{A}, \n{B} \Coloneqq \p{\atmR{a}} \limp \n{B} \mid \n{B} \pmir \p{\atmL{a}} \mid \n{A} \with \n{B} \mid \top \mid \n{\defp{p}}
  \,.
\end{equation*}
Sequent calculi with recursive definitions of this kind have been studied previously\autocites{Hallnas:TCS91}{Eriksson:ELP91}{Schroeder-Heister:LICS93}{McDowell+Miller:TCS00}{Tiu+Momigliano:JAL12}, but, to the best of our knowledge, the use of coinductive definitions in the context of logically motivated rewriting systems is new.

That the definitions $\n{\defp{p}} \defd \n{A}$ are indeed coinductive is guaranteed by imposing the requirement that along every cycle among defined propositions there is a logical connective.%
\footnote{This generalizes the local \emph{contractivity} condition described by \textcite{Gay+Hole:AI05}.}
For example, the definition $\n{\defp{p}} \defd \p{\atmR{a}} \limp \n{\defp{p}}$ or even the definitions $\n{\defp{p}} \defd \n{\defp{q}}$ and $\n{\defp{q}} \defd \p{\atmR{a}} \limp \n{\defp{p}}$ are acceptable because $(\p{\atmR{a}} \limp \mathord{-})$ occurs along the cycle from $\n{\defp{p}}$ to itself; but the definitions $\n{\defp{p}} \defd \n{\defp{q}}$ and $\n{\defp{q}} \defd \n{\defp{p}}$ are forbidden because no logical connective occurs along the cycle.

To clarify, these definitions are coinductive in only a syntactic sense; when interpreted as processes, they are not (necessarily) behaviorally coinductive.
For example, the proposition $\n{\defp{p}}$ given by $\n{\defp{p}} \defd \atmR{a} \limp \atmR{a} \fuse \n{\defp{p}}$ does not correspond to a productive process when viewed from the formula-as-process perspective -- after receiving an initial message $\atmR{a}$, the process $\n{\defp{p}}$ diverges without sending or receiving any further messages.
Therefore, our coinductively defined propositions are \emph{not} greatest fixed points (or coinductive proofs) in the sense of \citeauthor{Fortier+Santocanale:CSL13}\autocite{Fortier+Santocanale:CSL13}.

% That the definitions $\n{\defp{p}} \defd \n{A}$ are indeed coinductive is guaranteed by imposing the requirement that all definitions be \emph{contractive}\autocite{??} -- \ie, that the body of each definition begin with a logical connective, logical constant, or atomic proposition at the top level.
% Contractivity rules out definitions like $\n{\defp{p}} \defd \n{\defp{p}}$ that do not correspond to sensible coinductive propositions and also to rule out sensible but inessential definitions like $\n{\defp{p}} \defd \n{\defp{q}}$.

% To rule out definitions like $\n{\defp{p}} \defd \n{\defp{p}}$ that do not correspond to sensible infinite propositions and also to rule out sensible but inessential definitions like $\n{\defp{p}} \defd \n{\defp{q}}$, we require that all definitions be \emph{contractive}\autocite{??} -- \ie, that the body of each recursive definition begin with a logical connective, logical constant, or atomic proposition at the top level.

Coinductive definitions are collected into a signature, $\orsig$, that indexes the rewriting relations: $\reduces_{\orsig}$ and $\Reduces_{\orsig}$.%
\footnote{We often elide the index, as it is usually clear from context.}
Syntactically, these signatures are given by the grammar
\begin{equation*}
  \orsig \Coloneqq \orsige \mid \orsig, (\n{\defp{p}} \defd \n{A})
  \,.
\end{equation*}
A signature $\orsig$ is well-formed if every $\n{\defp{p}}$ that occurs in the body of a definition itself has a definition in $\orsig$.

\newthought{By analogy with} recursive types from functional programming\autocite{??}, we must then decide whether to treat the coinductive definitions $\n{\defp{p}} \defd \n{A}$ \emph{iso}\-recursively or \emph{equi}\-recursively.\fixnote{equi-coinductively?}
Under an equirecursive treatment, definitions may be silently unrolled or rolled at will;
in other words, $\n{\defp{p}}$ is literally \emph{equal} to its unrolling: $\n{\defp{p}} = \n{A}$.
In contrast, under an isorecursive treatment, unrolling a coinductively defined proposition would count only as an explicit rule for the left-focus judgment: $\n{\defp{p}} \neq \n{A}$ but the adjacent $\lrule{\defd}$ rule would be present.%
\marginnote{%
  $\infer[\lrule{\defd}]{\lfocus{\atmR{\octx}_L}{\n{\defp{p}}}{\atmL{\octx}_R}{_{\orsig} \p{C}}}{
    \text{($(\n{\defp{p}} \defd \n{A}) \in \orsig$)} &
    \lfocus{\atmR{\octx}_L}{\n{A}}{\atmL{\octx}_R}{_{\orsig} \p{C}}}$}

Because these coinductively defined propositions are not generative\autocite{??}, there is not much difference between the equirecursive and isorecursive treatments.
We choose an equirecursive treatment of definitions simply because the accompanying generous notion of equality helps to minimize the conceptual overhead of coinductively defined propositions.
% As a simple example, under the equirecursive definition $\n{\defp{p}} \defd \p{\atmR{a}} \limp \up \dn \n{\defp{p}}$, we have the trace
% \begin{equation*}
%   \p{\atmR{a}} \oc \p{\atmR{a}} \oc \n{\defp{p}}
%     = \p{\atmR{a}} \oc \p{\atmR{a}} \oc (\p{\atmR{a}} \limp \up \dn \n{\defp{p}})
%     \reduces \p{\atmR{a}} \oc \n{\defp{p}}
% \end{equation*}

\newthought{How do these} coinductively defined negative propositions interact with the left-focus judgment, which is defined inductively?
The answer is that not all coinductively defined propositions can be successfully put into focus.
As previously mentioned, the proposition $\n{\defp{p}}$ given by $\n{\defp{p}} \defd \p{\atmR{a}} \limp \n{\defp{p}}$ is certainly a well-defined coinductive proposition, owing to the existence of $(\p{\atmR{a}} \limp \mathord{-})$ along the cycle.
Yet it cannot be successfully put into left focus -- there are no contexts $\atmR{\octx}_L$ and $\atmL{\octx}_R$ and positive consequent $\p{C}$ for which $\lfocus{\atmR{\octx}_L}{\n{\defp{p}}}{\atmL{\octx}_R}{\p{C}}$ is derivable.
To derive a left-focus judgment on $\n{\defp{p}}$, the finite context $\atmR{\octx}_L$ would need to hold an infinite stream of $\p{\atmR{a}}$ atoms -- an impossible feat for the inductively defined, and hence finite, contexts like $\atmR{\octx}_L$ that we consider here.%
\footnote{%
  \Textcite{Miller:??} presents a system with infinite focusing phases, which would certainly permit definitions like $\n{\defp{p}} \defd \p{\atmR{a}} \limp \n{\defp{p}}$ to be put into focus.
  We do not pursue that generalization here because we want rewriting to occur on finite contexts only.}

% However, because the left-focus judgment is defined inductively, not coinductively, there are some recursively defined negative propositions that cannot successfully be put into focus.
% Under the definition $\n{\defp{p}} \defd \p{\atmR{a}} \limp \n{\defp{p}}$, for example, there are no contexts $\atmR{\octx}_L$ and $\atmL{\octx}_R$ and positive consequent $\p{C}$ for which $\lfocus{\atmR{\octx}_L}{\n{\defp{p}}}{\atmL{\octx}_R}{\p{C}}$ is derivable.
% To derive a left-focus judgment on $\n{\defp{p}}$, the finite context $\atmR{\octx}_L$ would need to hold an infinite stream of $\p{\atmR{a}}$ atoms, which is impossible in an inductively defined, and hence finite, context.

However, by inserting $\up \dn$ as a double shift to blur focus -- in a way similar to how double shifts were used in the embedding of unfocused rewriting~\parencref{sec:ordered-rewriting:embed-unfocused} -- the definition can be revised to one that admits a left-focus judgment.
Specifically, if $\n{\defp{p}}$ is instead given by $\n{\defp{p}} \defd \p{\atmR{a}} \limp \up \dn \n{\defp{p}}$, then $\lfocus{\p{\atmR{a}}}{\n{\defp{p}}}{}{\dn \n{\defp{p}}}$ is derivable, and so $\p{\atmR{a}} \oc \n{\defp{p}} \reduces \n{\defp{p}}$.
More generally, any coinductively defined proposition that has an $\up$ shift along \emph{some} cycle can be successfully put into focus.

One might consider elevating the $\up$-shift property to a requirement on coinductively defined propositions -- \ie, demanding that every coinductively proposition have an $\up$ shift along some cycle.
This would forbid any definitions that cannot be put into left focus, such as $\n{\defp{p}} \defd \p{\atmR{a}} \limp \n{\defp{p}}$.
Although perhaps well-intentioned, such a requirement seems somewhat under-motivated after observing that even the proposition $\top$ cannot be successfully put into focus.

% \clearpage
% \section{}


% This interpretation is summarized in the adjacent \lcnamecref{fig:choreographies:propctx-table}.%
% \begin{margintable}
%   \begin{center}
%     \begin{tabular}{@{}l@{\enspace\ }>{\itshape}l@{}}
%       $\p{\atmL{a}}$ & left-directed message \\
%       $\p{\atmR{a}}$ & right-directed message \\
%       $\n{A}$ & message-passing process \\
%       $\np{\octx}$ & process configuration \\
%       $\p{A}$ & configuration reified as a process
%     \end{tabular}
%   \end{center}
%   \caption{A formula-as-process interpretation of polarized ordered propositions and contexts}\label{fig:choreographies:propctx-table}
% \end{margintable}%

% \newthought{}
% Under the formula-as-process interpretation, the rewriting judgment, $\np{\octx} \reduces \np{\octx'{}}$, is viewed as message-passing communication within the process configuration $\np{\octx}$.


% \begin{equation*}
%   \n{A}, \n{B} \Coloneqq \p{\atmR{a}} \limp \up \p{B} \mid \up \p{B} \pmir \p{\atmL{a}} \mid \n{A} \with \n{B} \mid \top \mid \up \p{A}
% \end{equation*}

% \begin{equation*}
%   \p{A}, \p{B} \Coloneqq \p{\atmL{a}} \mid \p{\atmR{a}} \mid \p{A} \fuse \p{B} \mid \one \mid \dn \n{A}
% \end{equation*}



% \begin{align*}
%   \n{A} &\Coloneqq \p{\atmR{a}} \limp \n{B} \mid \n{B} \pmir \p{\atmL{a}} \mid \n{A} \with \n{B} \mid \top \mid \up \p{A} \mid \n{\defp{p}}
% \end{align*}

% \begin{margintable}
%   \begin{center}
%     \begin{tabular}{@{}r@{\enspace}>{\itshape}l@{}}
%       $\p{\atmR{a}} \limp \n{B}$ & receive message $\p{\atmR{a}}$ from the right \\
%       $\n{B} \pmir \p{\atmL{a}}$ & receive message $\p{\atmL{a}}$ from the left \\
%       $\n{A} \with \n{B}$ & nondeterministic branching \\% continue as $\n{A}$ or $\n{B}$ \\
%       $\top\hphantom{\n{}}$ & \\
%       $\up \p{A}$ & \\
%       $\n{\defp{p}}$ & call a recursively defined process
%     \end{tabular}
%   \end{center}
%   \caption{A formula-as-process interpretation of negative propositions}\label{fig:choreographies:negprop-table}
% \end{margintable}

% \begin{margintable}
%   \begin{center}
%     \begin{tabular}{@{}r@{\enspace}>{\itshape}l@{}}
%       $\np{\octx}_1 \oc \np{\octx}_2$ & parallel composition of configurations \\
%       $(\octxe)$ & empty configuration \\
%       $\n{A}$ & single-process configuration \\
%       $\p{\atmL{a}}$ & left-directed message \\
%       $\p{\atmR{a}}$ & right-directed message
%     \end{tabular}
%   \end{center}
%   \caption{A formula-as-process interpretation of contexts}\label{fig:choreographies:ctxprop-table}
% \end{margintable}

% \begin{margintable}
%   \begin{center}
%     \begin{tabular}{@{}r@{\enspace}>{\itshape}l@{}}
%       $\p{\atmL{a}}$ & left-directed message \\
%       $\p{\atmR{a}}$ & right-directed message \\
%       $\p{A} \fuse \p{B}$ & parallel composition of $\p{A}$ and $\p{B}$ \\
%       $\one\hphantom{\p{}}$ & forwarding process \\
%       $\dn \n{A}$ & 
%     \end{tabular}
%   \end{center}
%   \caption{A formula-as-process interpretation of positive propositions}\label{fig:choreographies:posprop-table}
% \end{margintable}

% In focused ordered rewriting, ordered contexts consist of positive atoms, $\p{a}$, and negative propositions, $\n{A}$.
% Under the formula-as-process interpretation, positive atoms will be viewed as messages, and negative propositions will be viewed as processes that receive and react to those messages.
% Ordered contexts are then configurations of processes and messages, arranged in a linear topology.





% Positive atoms, $\p{a}$


% Negative propositions, $\n{A}$, are interpreted as message-passing processes, with positive atoms, $\p{a}$, as messages passed between them.
% Ordered contexts, $\octx$, are then configurations of processes and messages arranged in a linear topology.
% Finally, the positive propositions, $\p{A}$, reify ordered contexts, and so they can be interpreted as process expressions that reify process configurations.

% The ordered implications $\p{A} \limp \n{B}$ and $\n{B} \pmir \p{A}$ are restricted to $\p{a} \limp \n{B}$ and $\n{B} \pmir \p{a}$, respectively, so that they may cleanly be interpreted as processes that input a message $\p{a}$ from the left and right, respectively.


% Each positive atom $\p{a}$ is assigned a direction, either $\atmL{a}$ or $\atmR{a}$, that indicates 

% $\atmL{a}$ and $\atmR{a}$; and $\p{A} \limp \n{B}$ restricted to $\atmR{a} \limp \n{B}$ and similarly for right-handed implication.

% \subsection{Formula-as-process}

% Discuss here?

% Example of $\proc{b} \defd (\atmR{a} \limp \up \dn \proc{b}) \with \up \one$, without explicitly relating it to the specification.

% \subsection{Focused ordered rewriting, revisited}

% The two changes introduced by the formula-as-process interpretation -- atom directions and atomic premises for implications -- trickle down to the focused ordered rewriting framework.

% First, because each positive atom is now marked with a direction, the $\jrule{ID}\smash{^{\p{a}}}$ rule%
% \marginnote{\qquad%
%   $\infer[\jrule{ID}\smash{^{\p{a}}}]{\rfocus{\p{a}}{\p{a}}}{}$%
% }
% that was previously part of the right-focus judgment's definition is replaced by two similar rules:
% \begin{equation*}
%   \infer[\jrule{ID}\smash{^{\atmR{a}}}]{\rfocus{\atmR{a}}{\atmR{a}}}{}
%   \qquad\text{and}\qquad
%   \infer[\jrule{ID}\smash{^{\atmL{a}}}]{\rfocus{\atmL{a}}{\atmL{a}}}{}
%   \,.
% \end{equation*}
% The other right-focusing rules remain unchanged.

% Second, because left-handed implications now have only right-directed atoms as premises and right-handed implications have only left-directed atoms as premises, the left-focus judgment and its rules must be revised.
% Instead of $\lfocus{\np{\octx}_L}{\n{A}}{\np{\octx}_R}{\p{C}}$, which has arbitrary contexts to the left and right of $\n{A}$, the judgment is now $\lfocus{\atmR{\octx}_L}{\n{A}}{\atmL{\octx}_R}{\p{C}}$ -- the left-hand context consists only of right-directed atoms, hence $\atmR{\octx}_L$; symmetrically, the right-hand context consists only of left-directed atoms, hence $\atmL{\octx}_R$.

% With the restriction to atomic premises, we need to reconsider the left-focus rules for the left- and right-handed implications.
% By inversion under the previous set of left-focus rules, any derivation focused on $\atmR{a} \limp \n{B}$ would end with
% \begin{equation*}
%   \infer[\lrule{\limp}]{\lfocus{\atmR{\octx}_L \oc \atmR{a}}{\atmR{a} \limp \n{B}}{\atmL{\octx}_R}{\p{C}}}{
%     \infer[\jrule{ID}\smash{^{\atmR{a}}}]{\rfocus{\atmR{a}}{\atmR{a}}}{} &
%     \lfocus{\atmR{\octx}_L}{\n{B}}{\atmL{\octx}_R}{\p{C}}}
%   %
%   \qquad
%   %
%   \infer[\lrule{\limp}']{\lfocus{\atmR{\octx}_L \oc \atmR{a}}{\atmR{a} \limp \n{B}}{\atmL{\octx}_R}{\p{C}}}{
%     \lfocus{\atmR{\octx}_L}{\n{B}}{\atmL{\octx}_R}{\p{C}}}
% \end{equation*}
% By similar reasoning, we arrive at a $\lrule{\pmir}'$ rule, as well:
% \begin{equation*}
%   \infer[\lrule{\pmir}']{\lfocus{\atmR{\octx}_L}{\n{B} \pmir \atmL{a}}{\atmL{a} \oc \atmL{\octx}_R}{\p{C}}}{
%     \lfocus{\atmR{\octx}_L}{\n{B}}{\atmL{\octx}_R}{\p{C}}}
% \end{equation*}
% The other rules for the left-focus judgment remain unchanged, save for the fact that the left- and right-hand context now contain only atoms of the appropriate direction.
% \Cref{fig:??} summarizes the revised rules for the right- and left-focus judgments.

% Having refined the left-focus judgement to use input message contexts, we may similarly refine the reduction rules, $\jrule{$\reduces$I}$ and $\jrule{$\reduces$C}$:
% \begin{inferences}
%   \infer[\jrule{$\reduces$I}]{\atmR{\octx}_L \oc \n{A} \oc \atmL{\octx}_R \reduces \np{\octx'{}}}{
%     \lfocus{\atmR{\octx}_L}{\n{A}}{\atmL{\octx}_R}{\p{B}} &
%     \rfocus{\np{\octx'{}}}{\p{B}}}
%   \and
%   \infer[\jrule{$\reduces$C}]{\np{\octx}_L \oc \np{\octx} \oc \np{\octx}_R \reduces \np{\octx}_L \oc \np{\octx'{}} \oc \np{\octx}_R}{
%     \np{\octx} \reduces \np{\octx'{}}}
% \end{inferences}

% \clearpage
% \subsection{}

% In unfocused and focused ordered rewriting of the \ac{OR}~\parencref{??} and \ac{FOR}~\parencref{??} frameworks, the rewriting relation, $\reduces$, described a purely internal operation: $\octx \reduces \octx'$ held independently of any environment that might surround $\octx$.
% Ordered rewriting's isolationism was affirmed by its compatibility rule, $\jrule{$\reduces$C}$%
% \marginnote{%
% $\infer[\jrule{$\reduces$C}]{\octx_L \oc \octx \oc \octx_R \reduces \octx_L \oc \octx' \oc \octx_R}{
%   \octx \reduces \octx'}$}%
% , which shows that the environment remains unaffected by the rewriting of its [...].

% Now that we have a formula-as-process interpretation to ordered rewriting, we should reconsider this strict isolationism.
% Under our formula-as-process reading, this isolationist rewriting judgment corresponds to a reduction semantics for processes.
% But now, messages, as represented by the directed $\p{\atmL{a}}$ and $\p{\atmR{a}}$ atoms, make it possible to describe the interactions that a configuration $\octx$ offers to its surroundings. 


\section{A local interaction semantics}\label{sec:formula-as-process:local-interaction}% \label{sec:formula-as-process:transition-semantics}

For the formula-as-process interpretation, we have thus far examined the rewriting judgment, $\octx \reduces \octx'$, and suggested that it represents a kind of reduction semantics for the underlying processes.
% Now that we are ascribing a formula-as-process interpretation to ordered rewriting, this judgment characterizes internal reductions.

But a reduction semantics is not the only way to describe the operational semantics of a process calculus.
For example, in the $\pi$-calculus, labeled transition systems are frequently used as an alternative to a reduction semantics, particularly when an understanding of how processes interact with their surroundings is needed.

For the formula-as-process ordered rewriting framework, we can similarly conceive of a local interaction semantics of this sort.
All communication occurs through message passing, so there are just two ways a process configuration can interact with its surrounding environment -- either send messages or receive them; either make an output transition or make an input transition.
% The ability of a configuration to make these two forms of transition is captured by two distinct judgments.
A process configuration can also forgo interacting with its environment and make a silent, internal transition as the configuration's components interact with each other.

Traditionally, these three forms of transition -- internal, output, and input -- are expressed with a unified labeled transition judgment in which the labels distinguish among the three forms of trnsition.
% output transitions from input transitions.
Here we instead prefer to use distinct syntax for each form of transition.

\paragraph*{Internal transitions}

In labeled transition systems for process calculi, internal $\tau$-transitions express interaction of a process configuration, not with its environment, but within itself among its constituent processes.
In the $\pi$-calculus, these internal transitions coincide with the notion of reduction but are defined as $\overset{\tau}{\reduces}$ in such a way that the explicit and sometimes cumbersome structural congruence is not needed, thereby simplifying proofs.

In our setting, however, we have no explicit structural congruence; the implicit monoid laws do not complicate proofs, so we can get away without defining a distinct notion of internal $\tau$-transition.
Whenever we want to describe an internal transition, the $\reduces$ reduction relation can be used instead.

In the $\pi$-calculus, there is also a notion of weak internal $\tau$-transition, $\overset{\tau}{\Reduces}$, which is the reflexive, transitive closure of $\overset{\tau}{\reduces}$.
Just as we use the $\reduces$ reduction relation whenever we want to describe an internal transition, we will use its reflexive, transitive closure, $\Reduces$, whenever we want to describe a weak internal transition.
% Similarly, the $\Reduces$ multi-step reduction relation can be used wherever we want to describe a weak internal transition, which is merely the reflexive, transitive closure of the internal transition relation.




\paragraph*{Output transitions}




% Traditionally, these interactions would be described by the actions that label a transition.

% We say that $\octx$ sends messages $\atmL{\octx}_L$ and $\atmR{\octx}_R$ to its left- and right-hand surrounding when 


% A process configuration $\octx$ can interact with its surrounding environment along either (or both) of two interfaces

% Interactions [...]
Similar to our treatment of internal transitions, we do not adopt
% Rather than adopting
an explicit judgment for output interactions but instead make use of context equality. 
We say that the context $\octx$ outputs messages $\atmL{\octx}_L$ to the left and messages $\atmR{\octx}_R$ to the right exactly when $\octx = \atmL{\octx}_L \oc \octx' \oc \atmR{\octx}_R$ for some context $\octx'$.
This equality expresses an immediate, or strong, output of messages $\atmL{\octx}'_L$ and $\atmR{\octx}'_R$ from the context $\octx$.%
\footnote{It is roughly analogous to $\overset{x\langle y\rangle}{\reduces}$, the $\pi$-calculus's output transition relation.}
We will sometimes refer to the context $\octx'$ here as the \vocab{continuation context} because it represents the context that remains after the output of $\atmL{\octx}_L$ and $\atmR{\octx}_R$ occurs.

As an example, both $\atmL{a} \oc \atmL{b} \oc \n{C}$ and $\atmL{a} \oc \n{C} \oc \atmL{b}$ output $\atmL{a}$ to the left (and nothing to the right), but more precisely, the former outputs $\atmL{a} \oc \atmL{b}$, whereas the latter does not output $\atmL{b}$ at all.

In addition to immediate, or strong, output transitions, it is also typical in a labeled transition semantics to express eventual, or weak, output transitions.
A weak output transition consists of finitely many internal transitions, followed by a single strong output transition, along with finitely many internal transitions on the continuation.

In the $\pi$-calculus, the weak output transition relation is $\mathord{\overset{\smash{\bar{x}\langle y\rangle}}{\Reduces}} = \mathord{\overset{\smash{\tau}}{\Reduces}\overset{\smash{\bar{x}\langle y\rangle}}{\reduces}\overset{\smash{\tau}}{\Reduces}}$.
%
In the formula-as-process ordered rewriting framework, a weak output transition could be expressed by $\octx \Reduces \atmL{\octx}'_L \oc \octx'_0 \oc \atmR{\octx}'_R$ and $\octx'_0 \Reduces \octx'$ together -- the context $\octx$ eventually outputs $\atmL{\octx}'_L$ and $\atmR{\octx}'_R$ and eventually arrives at the continuation context $\octx'$.
Based on \cref{thm:formula-as-process:no-recapture-outputs}, we can more concisely express the same weak output transition as $\octx \Reduces \atmL{\octx}'_L \oc \octx' \oc \atmR{\octx}'_R$.
This is the form in which we will usually express weak output transitions.

\paragraph*{Input transitions}

Unlike internal and output transitions, 
we use an explicit judgment for input interactions.
The judgment $\ireduces{\atmR{\octx}_L \oc #1 \oc \atmL{\octx}_R}{\octx}{\octx'}$ indicates that, upon receiving messages $\atmR{\octx}_L$ from the left and $\atmL{\octx}_R$ from the right, the context $\octx$ may evolve to $\octx'$ in a single step.%
\footnote{It is roughly analogous to $\overset{x(y)}{\reduces}$, the $\pi$-calculus's input transition relation.}
In other words, for each such judgment there should be a corresponding reduction:
\begin{restatable}[
  name=Soundness,
  label=thm:formula-as-process:ireduces-soundness
]{theorem}{ireducessoundness}
  If $\ireduces{\atmR{\octx}_L \oc ##1 \oc \atmL{\octx}_R}{\octx}{\octx'}$, then $\atmR{\octx}_L \oc \octx \oc \atmL{\octx}_R \reduces \octx'$.
\end{restatable}

In terms of the judgment's input/output mode, $\octx$ is the sole input to the judgment, whereas it produces the contexts $\atmR{\octx}_L$, $\atmL{\octx}_R$, and $\octx'$ as outputs of the judgments.
Thus, the input transition judgment answers the question \enquote{What input messages suffice for $\octx$ to make a reduction?}

\newthought{As the notation} is intended to suggest, each input transition at its heart derives from focusing on a single negative proposition, $\n{A}$, as captured by the $\jrule{$[]{\reduces}$I}$ rule:%
\footnote{Notice that it is quite possible in this rule for both $\atmR{\octx}_L$ and $\atmL{\octx}_R$ to be empty and for the judgment to express the input of no messages at all.
But that happens only if $\n{A}$ has an $\up$ shift as its top-level connective.}
\begin{equation*}
  \infer[\jrule{$[]{\reduces}$I}]{\ireduces{\atmR{\octx}_L \oc #1 \oc \atmL{\octx}_R}{\n{A}}{\octx'}}{
    \lfocus{\atmR{\octx}_L}{\n{A}}{\atmL{\octx}_R}{\p{C}} &
    \rfocus{\octx'}{\p{C}}}
  \,.
\end{equation*}
Aside from the change of judgment in the rule's conclusion, this $\jrule{$[]{\reduces}$I}$ rule is identical to the core $\jrule{$\reduces$I}$ rule for reduction.
How can we claim that the input transition judgment is distinct from the reduction judgment?

The difference between the judgments is twofold.
First, and most importantly, this input transition differs from a reduction in terms of its input/output mode.
In a reduction $\atmR{\octx}_L \oc \n{A} \oc \atmL{\octx}_R \reduces \np{\octx'{}}$, the entire $\atmR{\octx}_L \oc \n{A} \oc \atmL{\octx}_R$ context is treated as an input to the reduction judgment, and $\octx'$ is treated as an output made by the judgment.
In the input transition $\ireduces{\atmR{\octx}_L \oc #1 \oc \atmL{\octx}_R}{\n{A}}{\np{\octx'{}}}$, on the other hand, only the proposition $\n{A}$ is treated as an input to the judgment, and the contexts $\atmR{\octx}_L$, $\atmL{\octx}_R$, and $\np{\octx'{}}$ are all treated as outputs made by the input transition judgment.

The second difference is that, unlike the reduction judgment, the input transition judgment is enriched with several other rules.
In addition to the core input transition rule, $\jrule{$[]{\reduces}$I}$, other compatibility rules exist.

Two of these rules allow the external inputs expected by an input transition to be (partially) satisfied internally by the context itself.
\begin{inferences}
  \infer[\jrule{$[\atmR{a}]$C}]{\ireduces{\atmR{\octx}_L \oc #1 \oc \atmL{\octx}_R}{\atmR{a} \oc \octx}{\octx'}}{
    \ireduces{\atmR{\octx}_L \oc \atmR{a} \oc #1 \oc \atmL{\octx}_R}{\octx}{\octx'} &
    \text{($\atmR{\octx}_L \neq (\octxe)$ or $\atmL{\octx}_R \neq (\octxe)$)}}
  \\
  \infer[\jrule{$[\atmL{a}]$C}]{\ireduces{\atmR{\octx}_L \oc #1 \oc \atmL{\octx}_R}{\octx \oc \atmL{a}}{\octx'}}{
    \ireduces{\atmR{\octx}_L \oc #1 \oc \atmL{a} \oc \atmL{\octx}_R}{\octx}{\octx'} &
    \text{($\atmR{\octx}_L \neq (\octxe)$ or $\atmL{\octx}_R \neq (\octxe)$)}}
\end{inferences}
For example, the $\jrule{$[\atmR{a}]$C}$ rule: if $\octx$ can reduce to $\octx'$ upon input of surrounding $\atmR{\octx}_L \oc \atmR{a}$ and $\atmL{\octx}_R$, then $\atmR{a} \oc \octx$ can reduce to $\octx'$ upon input of surrounding $\atmR{\octx}_L$ and $\atmL{\octx}_R$.
In other words, in the context $\atmR{a} \oc \octx$, the atom $\atmR{a}$ already, internally satisfies $\octx$'s demand for $\atmR{a}$.
The $\jrule{$[\atmL{a}]$C}$ rule is symmetric, involving $\atmL{a}$ on the right.
Algebraically, these two rules express a form of associativity.



Read top-down, these $\jrule{$[\atmR{a}]$C}$ and $\jrule{$[\atmL{a}]$C}$ rules allow an input message to be absorbed by an input transition.
In addition, the input transition judgment is equipped with several (limited) compatibility rules.
Instead of absorbing a message like the $\jrule{$[\atmR{a}]$C}$ and $\jrule{$[\atmL{a}]$C}$ rules do, these compatibility rules frame a message or process $\oante$ onto an input transition, passing $\oante$ through.\footnote{Recall from \cref{sec:formula-as-process:interpretation} that $\oante \Coloneqq \atmL{a} \mid \atmR{a} \mid \n{A}$.}
\begin{inferences}
   \infer[\jrule{$[\oante]$C}_1]{\ireduces{#1 \oc \atmL{\octx}_R}{\oante \oc \octx}{\oante \oc \octx'}}{
    \ireduces{#1 \oc \atmL{\octx}_R}{\octx}{\octx'}}
  \and
  \infer[\jrule{$[\oante]$C}_2]{\ireduces{\atmR{\octx}_L \oc #1}{\octx \oc \oante}{\octx' \oc \oante}}{
    \ireduces{\atmR{\octx}_L \oc #1}{\octx}{\octx'}}
\end{inferences}
Notice that these rules apply only to one-sided input transitions: $\octx$ must require no inputs at the side at which $\oante$ is added.
This is because these rules pass $\oante$ through the input transition unaffected, and so $\oante$ serves as an interaction barrier at the end at which it appears.

% \begin{inferences}
%    \infer{\ireduces{#1 \oc \atmL{\octx}_R}{\atmL{a} \oc \octx}{\atmL{a} \oc \octx'}}{
%     \ireduces{#1 \oc \atmL{\octx}_R}{\octx}{\octx'}}
%     \and
%    \infer{\ireduces{#1 \oc \atmL{\octx}_R}{\atmR{a} \oc \octx}{\atmR{a} \oc \octx'}}{
%     \ireduces{#1 \oc \atmL{\octx}_R}{\octx}{\octx'}}
%   \and
%    \infer{\ireduces{#1 \oc \atmL{\octx}_R}{\n{A} \oc \octx}{\n{A} \oc \octx'}}{
%     \ireduces{#1 \oc \atmL{\octx}_R}{\octx}{\octx'}}
%   \\
%   \infer{\ireduces{\atmR{\octx}_L \oc #1}{\octx \oc \atmL{a}}{\octx' \oc \atmL{a}}}{
%     \ireduces{\atmR{\octx}_L \oc #1}{\octx}{\octx'}}
%   \and
%   \infer{\ireduces{\atmR{\octx}_L \oc #1}{\octx \oc \atmR{a}}{\octx' \oc \atmR{a}}}{
%     \ireduces{\atmR{\octx}_L \oc #1}{\octx}{\octx'}}
%   \and
%   \infer{\ireduces{\atmR{\octx}_L \oc #1}{\octx \oc \n{A}}{\octx' \oc \n{A}}}{
%     \ireduces{\atmR{\octx}_L \oc #1}{\octx}{\octx'}}
% \end{inferences}

\begin{figure}[tbp]
  \begin{inferences}
    \infer[\jrule{$[]{\reduces}$C}]{\ireduces{\atmR{\octx}_L \oc #1 \oc \atmL{\octx}_R}{\n{A}}{\octx'}}{
      \lfocus{\atmR{\octx}_L}{\n{A}}{\atmL{\octx}_R}{\p{C}} &
      \rfocus{\octx'}{\p{C}}}
    \\
    \infer[\jrule{$[\atmR{a}]$C}]{\ireduces{\atmR{\octx}_L \oc #1 \oc \atmL{\octx}_R}{\atmR{a} \oc \octx}{\octx'}}{
      \ireduces{\atmR{\octx}_L \oc \atmR{a} \oc #1 \oc \atmL{\octx}_R}{\octx}{\octx'} &
      \text{($\atmR{\octx}_L \neq (\octxe)$ or $\atmL{\octx}_R \neq (\octxe)$)}}
    \and
    \infer[\jrule{$[\atmL{a}]$C}]{\ireduces{\atmR{\octx}_L \oc #1 \oc \atmL{\octx}_R}{\octx \oc \atmL{a}}{\octx'}}{
      \ireduces{\atmR{\octx}_L \oc #1 \oc \atmL{a} \oc \atmL{\octx}_R}{\octx}{\octx'} &
      \text{($\atmR{\octx}_L \neq (\octxe)$ or $\atmL{\octx}_R \neq (\octxe)$)}}
    \\
    \infer[\jrule{$[\oante]$C}_1]{\ireduces{#1 \oc \atmL{\octx}_R}{\oante \oc \octx}{\oante \oc \octx'}}{
    \ireduces{#1 \oc \atmL{\octx}_R}{\octx}{\octx'}}
  \and
  \infer[\jrule{$[\oante]$C}_2]{\ireduces{\atmR{\octx}_L \oc #1}{\octx \oc \oante}{\octx' \oc \oante}}{
    \ireduces{\atmR{\octx}_L \oc #1}{\octx}{\octx'}}
  \end{inferences}
  \caption{An input transition judgment}\label{fig:formula-as-process:ireduces}
\end{figure}

The full complement of input transition rules is summarized in \cref{fig:formula-as-process:ireduces}.
Now we may finally prove the previously stated claim of soundness for input transitions -- that each input transition has a corresponding reduction.
%
\ireducessoundness
\begin{proof}
  By induction on the structure of the given input transition.
\end{proof}


% \paragraph*{Internal transitions}

% In labeled transition systems for process calculi, there is usually a third kind of transition: the internal $\tau$-transition.
% In the $\pi$-calculus, these internal transitions coincide with the notion of reduction but are defined in such a way that the explicit and sometimes cumbersome structural congruence is not needed, thereby simplifying proofs.

% In our setting, we have no explicit structural congruence; the implicit monoid laws do not complicate proofs, so we can get away without defining a notion of internal $\tau$-transition.
% Wherever we want to describe an internal transition, the $\reduces$ reduction relation can be used instead.

Also, output and input transitions are together complete, in the sense that each reduction can be broken down into an input transition with complementary output transitions:
\begin{theorem}[Completeness]
%  If $\ireduces{\atmR{\octx}_L \oc #1 \oc \atmL{\octx}_R}{\octx}{\octx'}$, then $\atmR{\octx}_L \oc \octx \oc \atmL{\octx}_R \reduces \octx'$.
  % Conversely, if $\octx \reduces \octx'$, then there exist contexts $\octx_L$, $\atmR{\octx}_L$, $\atmL{\octx}_R$, $\octx_R$, and $\octx'_0$ and a proposition $\n{A}$ such that: $\octx = \octx_L \oc \atmR{\octx}_L \oc \n{A} \oc \atmL{\octx}_R \oc \octx_R$ and $\ireduces{\atmR{\octx}_L \oc #1 \oc \atmL{\octx}_R}{\n{A}}{\octx'_0}$ and $\octx' = \octx_L \oc \octx'_0 \oc \octx_R$.
%  $\ireduces{#1}{\octx}{\octx'}$.
%   and there exist contexts ... such that $\octx = \octx_L \oc \atmR{\lctx}_L \oc \octx_M \oc \atmL{\lctx}_R \oc \octx_R$ and $\octx' = \lctx_L \oc \octx'_M \oc \lctx_R$ and $\ireduces{\atmR{\lctx}_L \oc #1 \oc \atmL{\lctx}_R}{\octx_M}{\octx'_M}$.
% \\\\
%   If $\ireduces{\atmR{\octx}_L \oc #1 \oc \atmL{\octx}_R}{\octx}{\octx'}$, then $\octx = \octx'_L \oc \atmR{\octx}^*_L \oc \octx_0 \oc \atmL{\octx}^*_R \oc \octx'_R$ and $\ireduces{\atmR{\octx}_L \oc \atmR{\octx}^*_L \oc #1 \oc \atmL{\octx}^*_R \oc \atmL{\octx}_R}{\octx_0}{\octx'_0}$ and $\octx' = \octx'_L \oc \octx'_0 \oc \octx'_R$.
% \\\\
%   If $\octx \reduces \octx'$, then 
  If $\octx \reduces \octx'$, then there exist contexts $\octx'_L$, $\atmR{\octx}_L$, $\octx_0$, $\atmL{\octx}_R$, $\octx'_R$, and $\octx'_0$ such that: $\octx = (\octx'_L \oc \atmR{\octx}_L) \oc \octx_0 \oc (\atmL{\octx}_R \oc \octx'_R)$ and $\ireduces{\atmR{\octx}_L \oc #1 \oc \atmL{\octx}_R}{\octx_0}{\octx'_0}$ and $\octx' = \octx'_L \oc \octx'_0 \oc \octx'_R$.
\end{theorem}
\begin{proof}
  By induction on the structure of the given reduction.
\end{proof}
%
Together, these soundness and completeness results may also be thought of as establishing the admissibility and invertibility of the following rule. 
\begin{equation*}
  \infer-{\octx \reduces \octx'}{
    \octx = \octx_L \oc \lctx \oc \octx_R &
    \octx_L = \octx'_L \oc \atmR{\lctx}_L &
    \ireduces{\atmR{\lctx}_L \oc #1 \oc \atmL{\lctx}_R}{\lctx}{\lctx'} &
    \atmL{\lctx}_R \oc \octx'_R = \octx_R &
    \octx'_L \oc \lctx' \oc \octx'_R = \octx'}
\end{equation*}

We could also consider a notion of eventual, or weak, input transition, akin to the $\pi$-calculus's $\overset{\smash{x(y)}}{\Reduces}$ relation.
There is not an especially concise way to express weak input transitions using our notation for strong input transitions, and weak input transitions in and of themselves will not prove to be particularly useful to us, so we do not pursue them here.

% \begin{proof}
%   By induction on the structure of the given reduction.
  % , after first proving an easy lemma:
  % \begin{itemize}
  % \item If $\ireduces{\atmR{\octx}_L \oc #1 \oc \atmL{\octx}_R}{\octx}{\octx'}$, then $\ireduces{#1}{\atmR{\octx}_L \oc \octx \oc \atmL{\octx}_R}{\octx'}$.
  % \qedhere
  % \end{itemize}
  % $\octx'_L \oc \atmR{\octx}^*_L \oc \octx_0 \oc \atmL{\octx}^*_R \oc \octx'_R$ and $\ireduces{\atmR{\octx}^*_L \oc #1 \oc \atmL{\octx}^*_R \oc \atmL{\octx}_R}{\octx_0}{\octx'_0}$
% \end{proof}

\clearpage
\section{Choreographing string rewriting specifications}\label{sec:formula-as-process:choreographies}\label{sec:choreographies:choreographies}

So far in this \lcnamecref{ch:formula-as-process}, we have presented a formula-as-process refinement of the focused ordered rewriting framework and given it a local interaction semantics based on an implicit labeled transition system for output and input transitions.
With this local interaction semantics in hand, we can return to our goal of assigning a concrete operational semantics to string rewriting specifications.
We will show how to operationalize, or \emph{choreograph}, these specifications by embedding them within the formula-as-process ordered rewriting framework.

% Not all specifications can be choreographed, but 

To choreograph a string rewriting specification $(\sralph, \srsig)$, we would like to map each symbol $a \in \sralph$ to a proposition such that the string rewriting axioms $\srsig$ are mapped to derivable rewritings in our formula-as-process ordered rewriting framework.
In other words, to choreograph $(\sralph, \srsig)$, we would like to find a map $\theta$ from symbols to propositions and a signature $\orsig$ of coinductive definitions such that $\theta$ is a witness to the (strong) bisimilarity of string rewriting under the axioms $\srsig$ and formula-as-process ordered rewriting under the definitions $\orsig$.
That is, we would like to find a pair $(\theta, \orsig)$ for which we can complete the diagrams
\begin{equation*}
  \begin{tikzcd}
    w \rar[reduces, subscript=\srsig] \dar[relation][swap]{\theta}
      & w\mathrlap{'} \dar[relation, exists]{\theta}
    \\
    \octx \rar[reduces, exists, subscript=\orsig]
      & \octx\mathrlap{'}
  \end{tikzcd}
  \hphantom{'}
  \qquad\text{and}\qquad
  \begin{tikzcd}
    w \rar[reduces, exists, subscript=\srsig] \dar[relation][swap]{\theta}
      & w\mathrlap{'} \dar[relation, exists]{\theta}
    \\
    \octx \rar[reduces, subscript=\orsig]
      & \octx\mathrlap{' % = \theta(w') 
\,,}
  \end{tikzcd}
   \hphantom{' % = \theta(w')
 \,,}
\end{equation*}
where $\begin{tikzcd}[cramped, sep=small] w \rar[relation]{\smash{\theta}} & \octx \end{tikzcd}$ holds exactly when $\octx = \theta(w)$.
Only if the pair $(\theta, \orsig)$ satisfies these diagrams does it constitute a \emph{choreography} of the specification $(\sralph, \srsig)$.%
\footnote{An arbitrary pair $(\theta, \orsig)$ might be called a \emph{pre-choreography}.}

Because ordered rewriting in our formula-as-process framework permits only sensibly local interactions, we can be sure that the choreography $(\theta, \orsig)$ explains the \emph{how}, not just the what, of the concurrent system's dynamics.
The map $\theta$ is key to the how.
It 
% This map, $\theta$,
serves as a \emph{role assignment} for the string rewriting symbols, casting each symbol $a \in \sralph$ in the role of either a message, $\atmL{a}$ or $\atmR{a}$, or a coinductively defined process, $\defp{a}$.
(More formally, we will require that role assignments be injective monoid homomorphisms with this property.)

For a given specification, there will often be several role assignments that give rise to distinct choreographies, each one implying a different message-passing operationalization of the specification.
Without applying other, external criteria, no one choreography has more desirable lower-level behavior than another -- only the programmer is in a position to choose among choreographies.

Most of the $3^{\lvert\sralph\rvert}$ role assignments for a specification's alphabet do not lead to adequate choreographies.
Sometimes none of the possible role assignments produce a choreography.

\subsection{Choreographies by example}\label{sec:choreographies:informal}

% Given a string rewriting alphabet $\sralph$, we say that a (total) map $\theta$ from finite strings over $\sralph$ to ordered contexts is a \emph{role assignment} for $\sralph$ if it is an injective monoid homomorphism from the finite strings over $\sralph$ to ordered contexts that casts each symbol $a$ in the role of either a message, $\atmL{a}$ or $\atmR{a}$, or a coinductively defined process, $\defp{a}$.

Recall from \cref{ch:string-rewriting} the string rewriting specification $(\sralph, \srsig)$ of a system that can rewrite strings over $\sralph = \Set{a,b}$ into the empty string if the initial string ends in $b$.
\begin{equation*}
  \begin{lgathered}
    \sralph = \Set{a,b} \\
    \srsig = (a \wc \reduces b) \,, (b \reduces \emp)
  \end{lgathered}
\end{equation*}

Let $\theta$ be the injective monoid homomorphism generated by mapping $a$ to the right-directed message $\atmR{a}$ and $b$ to the coinductively defined process $\defp{b}$.%
\marginnote{$\theta = \Set{ a \mapsto \atmR{a} , b \mapsto \defp{b} }$}
The map $\theta$ is indeed a role assignment, but does it yield a meaningful choreography for the specification $(\sralph, \srsig)$?

We must determine if $\defp{b}$ can be given a definition $\orsig = (\defp{b} \defd \n{B})$ such that the above, strong bisimulation diagrams can be completed.
Because $\theta$ is injective, those diagrams are equivalent to the following ones:
In the first diagram, the right-hand edge $\begin{tikzcd}[cramped, sep=small] w' \rar[relation, exists]{\smash{\theta}} & \octx' \end{tikzcd}$ can be replaced with $\begin{tikzcd}[cramped, sep=small] w' \rar[relation]{\smash{\theta}} & \theta(w') \end{tikzcd}$, but we cannot make a similar replacement for the second diagram because $\theta$ is not bijective, only injective.
\begin{equation*}
  \begin{tikzcd}
    w \rar[reduces, subscript=\srsig] \dar[relation][swap]{\theta}
      & w\mathrlap{'} \dar[relation]{\theta}
    \\
    \theta(w) \rar[reduces, exists, subscript=\orsig]
      & \theta(w')
  \end{tikzcd}
  \qquad\text{and}\qquad
  \begin{tikzcd}
    w \rar[reduces, exists, subscript=\srsig] \dar[relation][swap]{\theta}
      & w\mathrlap{'}
      \dar[relation, exists]{\theta}
    \\
    \theta(w) \rar[reduces, subscript=\orsig]
      & \octx\mathrlap{' \,.} % = \theta(w') \,.}
  \end{tikzcd}
  \hphantom{' \,.}
\end{equation*}


The first diagram gives us a way forward to a choreography: for each axiom $(w \reduces w') \in \srsig$, the rewriting $\theta(w) \reduces_{\orsig} \theta(w')$ must be derivable under the definitions $\orsig$.
In other words, these rewritings serve as constraints upon the definitions $\orsig$ that must be fulfilled if $(\theta, \orsig)$ is to be a meaningful choreography for the specification $(\sralph, \srsig)$.

In this example, the axioms $a \wc b \reduces b$ and $b \reduces \emp$ induce the constraints
\begin{equation*}
  \begin{tikzcd}[cramped]
    \atmR{a} \oc \defp{b} \rar[reduces, exists, subscript=\orsig] & \defp{b}
  \end{tikzcd}
  \qquad\text{and}\qquad
  \begin{tikzcd}[cramped]
    \defp{b} \rar[reduces, exists, subscript=\orsig] & (\octxe)
  \end{tikzcd}
  \,.
\end{equation*}
\begin{marginfigure}
\begin{equation*}
  \begin{tikzcd}
    a \wc b \arrow[reduces, subscript=\srsig]{rrr} \dar[relation][swap]{\theta}
      &[-2.4em] &&[-2.4em] b \dar[relation]{\theta}
    \\
    \theta(a \wc b) & = \atmR{a} \oc \defp{b} \rar[reduces, exists, subscript=\orsig]
      & \defp{b} = & \theta(b)
  \end{tikzcd}
\end{equation*}
\begin{equation*}
\begin{tikzcd}
    b \arrow[reduces, subscript=\srsig]{rrr} \dar[relation][swap]{\theta}
      &[-2.4em] &&[-2.4em] \emp \dar[relation]{\theta}
    \\
    \theta(b) & = \defp{b} \rar[reduces, exists, subscript=\orsig]
      & (\octxe) = & \theta(\emp)
  \end{tikzcd}
\end{equation*}
\caption{Axioms induce rewritings as constraints on a choreography}\label{fig:formula-as-process:induced-constraints}
\end{marginfigure}%
Well, a definition $\defp{b} \defd \atmR{a} \limp \up \dn \defp{b}$ would satisfy the first constraint but not the second, because $\atmR{a} \oc \defp{b} = \atmR{a} \oc (\atmR{a} \limp \up \dn \defp{b}) \reduces_{\orsig} \defp{b}$.
And a definition $\defp{b} \defd \up \one$ would satisfy the second constraint but not the first, because $\defp{b} = \up \one \reduces_{\orsig} (\octxe)$.
Fortunately, we can form a kind of greatest lower bound of these definitions using alternative conjunction%
\footnote{This is possible because the left-focus rule for alternative conjunction preserves focus.}%
: the definition $\defp{b} \defd (\atmR{a} \limp \up \dn \defp{b}) \with \up \one$ satisfies \emph{both} constraints,
\begin{equation*}
  \atmR{a} \oc \defp{b} = \atmR{a} \oc ((\atmR{a} \limp \up \dn \defp{b}) \with \up \one) \reduces_{\orsig} \defp{b}
  \qquad\text{and}\qquad
  \defp{b} = (\atmR{a} \limp \up \dn \defp{b}) \with \up \one \reduces_{\orsig} (\octxe)
  \,.
\end{equation*}
And the second diagram holds because of the universal properties of the logical connectives.

\newthought{Not all role assignments} yield meaningful choreographies, however.
This happens when there is no solution to the constraints on $\orsig$ induced by the axioms and chosen role assignment.
For a set of constraints to be satisfiable, three conditions must hold.
\begin{itemize}
\item
  \emph{Each induced rewriting must have at least one process in its premise.}
  In the above example, for instance, role assignments $\theta'$ such that either $b \mapsto \atmL{b}$ or $b \mapsto \atmR{b}$ do not yield meaningful choreographies.
  Under such assignments, the axiom $b \reduces_{\srsig} \emp$ induces either $\ereduces[\orsig']{\atmL{b}}{(\octxe)}$ or $\ereduces[\orsig']{\atmR{b}}{(\octxe)}$ as constraints.
  There are, however, no definitions that satisfy either constraint because the formula-as-process framework has no rules that permit an atom alone to be rewritten: messages are passive objects.

\item
  \emph{Each induced rewriting must have at most one process in its premise.}
  In the above example, for instance, the role assignment $\theta'$ such that $a \mapsto \defp{a}$ and $b \mapsto \defp{b}$ does not yield a meaningful choreography.
  The axiom $a \wc b \reduces b$ induces the constraint $\ereduces[\orsig']{\defp{a} \oc \defp{b}}{\defp{b}}$.
  There are, however, no definitions for $\defp{a}$ and $\defp{b}$ that satisfy this constraint because the formula-as-process framework proscribes implications from having non-atomic premises: a process can input only messages, not other processes.

\item
  \emph{Each message in a premise must be directed inward, toward the premise's process.}
  In the above example, for instance, the role assignment $\theta'$ such that $a \mapsto \atmL{a}$ and $b \mapsto \defp{b}$ does not yield a meaningful choreography.
  The axiom $a \wc b \reduces b$ induces the constraint $\ereduces[\orsig']{\atmL{a} \oc \defp{b}}{\defp{b}}$.
  There is, however, no defintion for $\defp{b}$ that satisfies this constraint because the formula-as-process framework requires that implications have atomic premises of \emph{complementary} direction: a process can only receive messages intended for itself.
\end{itemize}

More generally, these observations suggest that only constraints of the form $\ereduces[\orsig]{\atmR{\octx}_L \oc \defp{a} \oc \atmL{\octx}_R}{\octx'}$ are satisfiable, and that these constraints are induced by axioms of the form $w_1 \wc a \wc w_2 \reduces w'$.
In the following \lcnamecref{sec:formula-as-process:choreograph-formal}, we leverage these ideas to present a more formal description of the above procedure for choreographing string rewriting specifications within the formula-as-process ordered rewriting framework.

\subsection{A formal description of choreographing specifications}\label{sec:formula-as-process:choreograph-formal}\label{sec:choreographies:choreograph-formal}

To give a formal description of choreographing specifications, we define a judgment $\chorsig{\theta}{\srsig}{\orsig}$ that, when given a string rewriting specification $(\sralph, \srsig)$ and a role assignment $\theta$, yields formula-as-process definitions $\orsig$ that make string rewriting under $\srsig$ and formula-as-process ordered rewriting under $\orsig$ bisimilar, if such definitions exist:
% \begin{equation*}
%   ??
% \end{equation*}
% This principal judgment also relies on an auxiliary judgment, $?$.
% 
\begin{equation*}
  \chorsig{\theta}{\srsig}{\orsig}
  \quad\text{only if}\enspace
  \begin{tikzcd}
    w \rar[reduces, subscript=\srsig] \dar[relation][swap]{\theta}
     & w\mathrlap{'} \dar[relation]{\theta}
    \\
    \theta(w) \rar[reduces, exists, subscript=\orsig]
     & \theta(w')
  \end{tikzcd}
  \hphantom{'}
  \enspace\text{and}\quad
  \begin{tikzcd}
    w \rar[reduces, exists, subscript=\srsig] \dar[relation][swap]{\theta}
     & w\mathrlap{'}
         \dar[relation, exists]{\theta}
    \\
    \theta(w) \rar[reduces, subscript=\orsig]
     & \octx\mathrlap{' \,.% \footnote{Once again, the injectivity of $\theta$ simplifies these diagrams.}
}
  \end{tikzcd}
  \hphantom{' \,.}
\end{equation*}
Stated differently, the judgment will be such that the following adequacy result will hold: \enquote{If $\chorsig{\theta}{\srsig}{\orsig}$, then $\theta(w) \reduces_{\orsig} \octx'$ if, and only if, there exists a string $w'$ such that $w \reduces_{\srsig} w'$ and $\theta(w') = \octx'$.}

This principal judgment
% is $\chorsig{\theta}{\srsig}{\orsig}$, and it
also relies on an auxiliary elaboration judgment, $\qimp{\atmR{\octx}_L}{\up \p{A}}{\atmL{\octx}_R}{\n{B}}$, which we describe first.
% Before giving the rules for the principal judgment, we will [...].

\newthought{The auxiliary judgment} $\qimp{\atmR{\octx}_L}{\up \p{A}}{\atmL{\octx}_R}{\n{B}}$ elaborates the quasi-propo\-si\-tion $\atmR{\octx}_L \limp \up \p{A} \pmir \atmL{\octx}_R$ into a well-formed proposition $\n{B}$ by nondeterministically abstracting one-by-one from either the left or right contexts.%
\footnote{This procedure could be made deterministic by preferring one side over the other, but we refrain from doing so because the choice of side to prefer is completely arbitrary.}
This proposition $\n{B}$ is semantically equivalent to the quasi-proposition $\atmR{\octx}_L \limp \up \p{A} \pmir \atmL{\octx}_R$ in the sense that the two intuitively satisfy the \enquote{same} left-focus judgments:
% $\n{B}$ satisfies $\atmR{\octx}_L \oc \n{B} \oc \atmL{\octx}_R \reduces \octx'$ if, and only if, $\rfocus{\octx'}{\p{A}}$.
We would expect the quasi-proposition to satisfy $\lfocus{\atmR{\octx}_L}{\atmR{\octx}_L \limp \up \p{A} \pmir \atmL{\octx}_R}{\atmL{\octx}_R}{\p{A}}$, and indeed, when $\qimp{\atmR{\octx}_L}{\up \p{A}}{\atmL{\octx}_R}{\n{B}}$, we have
$\lfocus{\atmR{\lctx}_L}{\n{B}}{\atmL{\lctx}_R}{\p{C}}$ if, and only if, $\atmR{\lctx}_L = \atmR{\octx}_L$ and $\atmL{\lctx}_R = \atmL{\octx}_R$ and $\p{C} = \p{A}$.
This is proved below as \cref{lem:qimp-correct}.

This auxiliary judgment is defined inductively by the following rules.
\begin{inferences}
  \infer[\jrule{$\up$EL}]{\qimp{(\octxe)}{\up \p{A}}{(\octxe)}{\up \p{A}}}{}
  \\
  \infer[\jrule{$\limp$EL}]{\qimp{(\atmR{\octx}_L \oc \atmR{a})}{\up \p{A}}{\atmL{\octx}_R}{\atmR{a} \limp \n{B}}}{
    \qimp{\atmR{\octx}_L}{\up \p{A}}{\atmL{\octx}_R}{\n{B}}}
  \and
  \infer[\jrule{$\pmir$EL}]{\qimp{\atmR{\octx}_L}{\up \p{A}}{(\atmL{a} \oc \atmL{\octx}_R)}{\n{B} \pmir \atmL{a}}}{
    \qimp{\atmR{\octx}_L}{\up \p{A}}{\atmL{\octx}_R}{\n{B}}}
\end{inferences}
The $\jrule{$\limp$EL}$ rule states that the quasi-proposition $(\atmR{\octx}_L \oc \atmR{a}) \limp \up \p{A} \pmir \atmL{\octx}_R$ is equivalent to $\atmR{a} \limp \n{B}$ if $\atmR{\octx}_L \limp \up \p{A} \pmir \atmL{\octx}_R$ is equivalent to $\n{B}$.
Notice that the $\jrule{$\limp$EL}$ rule moves $\atmR{a}$ from the right of $\atmR{\octx}_L$ to the left of $\n{B}$;
this is admittedly counterintuitive, but it is closely related to the equally counterintuitive currying law for left-handed implication in ordered logic:
% Likewise, the quasi-proposition $\atmR{\octx}_L \limp \up \p{A} \pmir (\p{\atmL{a}} \oc \atmL{\octx}_R)$ is equivalent to $(\atmR{\octx}_L \limp \up \p{A} \pmir \atmL{\octx}_R) \pmir \p{\atmL{a}}$.
$(B \fuse A) \limp C \dashv\vdash A \limp (B \limp C)$.
Symmetrically, the $\jrule{$\pmir$EL}$ rule is closely related to the currying law for right-handed implication: $C \pmir (A \fuse B) \dashv\vdash (C \pmir B) \pmir A$.

This intuition is captured in the proof of the following \lcnamecref{lem:qimp-correct}.
\begin{lemma}\label{lem:qimp-correct}
  % If $\chorax{\theta}{w_1}{\p{C}}{w_2}{\n{B}}$, then $\lfocus{\theta(w_1)}{\n{B}}{\theta(w_2)}{\p{C}}$.
  If $\qimp{\atmR{\octx}_L}{\up \p{A}}{\atmL{\octx}_R}{\n{B}}$, then $\lfocus{\atmR{\lctx}_L}{\n{B}}{\atmL{\lctx}_R}{\p{C}}$ if, and only if, $\atmR{\lctx}_L = \atmR{\octx}_L$ and $\atmL{\lctx}_R = \atmL{\octx}_R$ and $\p{C} = \p{A}$.
  %
  % Moreover, if $\qimp{\atmR{\octx}_L}{\up \p{A}}{\atmL{\octx}_R}{\n{B}}$, then $\atmR{\lctx}_L \oc \n{B} \oc \atmL{\lctx}_R \reduces \lctx'$ if, and only if, there exist contexts $\atmR{\lctx}'_L$, $\lctx'_0$, and$\atmL{\lctx}'_R$ such that $\atmR{\lctx}'_L \oc \atmR{\octx}_L = \atmR{\lctx}_L$ and $\atmL{\octx}_R \oc \atmL{\lctx}'_R = \atmL{\lctx}_R$ and $\rfocus{\lctx'_0}{\p{A}}$ and $\lctx' = \atmR{\lctx}'_L \oc \lctx'_0 \oc \atmL{\lctx}'_R$.
\end{lemma}
\begin{proof}
  By induction over the structure of the given elaboration.

  As an example case, consider
  % \begin{itemize}
  % \item Consider the case in which
  %   \begin{equation*}
  %     \infer{\chorax{\theta}{\emp}{\p{A}}{\emp}{\up \p{A}}}{}
  %     \,.
  %   \end{equation*}
  %   We must show that $\lfocus{\atmR{\octx}_L}{\up \p{A}}{\atmL{\octx}_R}{\p{C}}$ if, and only if, $\atmR{\octx}_L = \atmL{\octx}_R = \theta(\emp)$ and $\p{A} = \p{C}$.
  %   Indeed, the $\lrule{\up}$ rule is the unique rule for left-focusing on $\up \p{A}$, and $\octxe = \theta(\emp)$ because $\theta$ is a monoid homomorphism.
  % 
  % \item Consider the case in which
    \begin{equation*}
      \infer[\jrule{$\limp$EL}]{\qimp{(\atmR{\octx}_L \oc \p{\atmR{a}})}{\up \p{A}}{\atmL{\octx}_R}{\p{\atmR{a}} \limp \n{B}}}{
        \qimp{\atmR{\octx}_L}{\up \p{A}}{\atmL{\octx}_R}{\n{B}}}
      \,.
    \end{equation*}
    We must show that $\lfocus{\atmR{\lctx}_L}{\p{\atmR{a}} \limp \n{B}}{\atmL{\lctx}_R}{\p{C}}$ if, and only if, $\atmR{\lctx}_L = \atmR{\octx}_L \oc \p{\atmR{a}}$ and $\atmL{\lctx}_R = \atmL{\octx}_R$ and $\p{C} = \p{A}$.
    Indeed, the $\lrule{\limp}$ rule is the unique rule for left-focusing on the proposition $\p{\atmR{a}} \limp \n{B}$, so $\lfocus{\atmR{\lctx}_L}{\p{\atmR{a}} \limp \n{B}}{\atmL{\lctx}_R}{\p{C}}$ if, and only if, $\atmR{\lctx}_L = \atmR{\lctx}'_L \oc \p{\atmR{a}}\!\!$ and $\lfocus{\atmR{\lctx}'_L}{\n{B}}{\atmL{\lctx}_R}{\p{C}}$ for some $\atmR{\lctx}'_L$.
    By the inductive hypothesis, we have $\lfocus{\atmR{\lctx}'_L}{\n{B}}{\atmL{\lctx}_R}{\p{C}}$ if, and only if, $\atmR{\lctx}'_L = \atmR{\octx}_L$ and $\atmL{\lctx}_R = \atmL{\octx}_R$ and $\p{C} = \p{A}$.
    Putting everything together, $\lfocus{\atmR{\lctx}_L}{\p{\atmR{a}} \limp \n{B}}{\atmL{\lctx}_R}{\p{C}}$ if, and only if, $\atmR{\lctx}_L = \atmR{\octx}_L \oc \p{\atmR{a}}$ and $\atmL{\lctx}_R = \atmL{\octx}_R$ and $\p{C} = \p{A}$, as required.
  % 
  % \item
  %   The case in which
  % \begin{equation*}
  %   \infer{\chorax{\theta}{w_1}{\p{A}}{b \oc w_2}{\n{B} \pmir \atmL{b}}}{
  %     \chorax{\theta}{w_1}{\p{A}}{w_2}{\n{B}} &
  %     \text{($\theta(b) = \atmL{b}$)}}
  % \end{equation*}
  %   is symmtric to the previous one.
  % %
  % \qedhere
  % \end{itemize}
\end{proof}



\newthought{The principal judgment} is $\chorsig{\theta}{_{\sralph} \srsig}{\orsig}$.%
\footnote{Because the alphabet $\sralph$ never changes within a derivation, we nearly always elide it.}
Given a string rewriting specification $(\sralph, \srsig)$ and a role assignment $\theta$, this judgment produces formula-as-process definitions $\orsig$ that, together with $\theta$, constitute a choreography of $(\sralph, \srsig)$.
%
In other words, when $\chorsig{\theta}{_{\sralph} \srsig}{\orsig}$ holds, the definitions $\orsig$ are a solution to the constraints induced by axioms $\srsig$ under the role assignment $\theta$, such that $\theta$ is a (strong) bisimulation between $\reduces_{\srsig}$ and $\reduces_{\orsig}$.%
\marginnote{%
  \begin{tabular}{@{}c@{}}
    $\chorsig{\theta}{\srsig}{\orsig}$
    \\only if\\
    $\!
  \begin{tikzcd}[ampersand replacement=\&]
    w \rar[reduces, subscript=\srsig] \dar[relation][swap]{\theta}
     \& w\mathrlap{'} \dar[relation]{\theta}
    \\
    \theta(w) \rar[reduces, exists, subscript=\orsig]
     \& \theta(w')
  \end{tikzcd}
  $and$
  \begin{tikzcd}[ampersand replacement=\&]
    w \rar[reduces, exists, subscript=\srsig] \dar[relation][swap]{\theta}
     \& w\mathrlap{'} \dar[relation, exists]{\theta}
    \\
    \theta(w) \rar[reduces, subscript=\orsig]
     \& \octx \mathrlap{'}
  \end{tikzcd}\hphantom{'}
  $
  \end{tabular}}%
% \footnote{%
%   Actually, we end up proving a stronger soundness result in \cref{??}.}
% The exact converse -- that $\theta(w) \reduces_{\orsig} \theta(w')$ implies $w \reduces_{\srsig} w'$ -- does hold, but we can prove an even stronger soundness result.
%
If there is no $\orsig$ for which $\chorsig{\theta}{_{\sralph} \srsig}{\orsig}$ holds, then the role assignment $\theta$ yields no choreography of the specification $(\sralph, \srsig)$.

% This judgment relies on an auxiliary judgment, $\qimp{\atmR{\octx}_L}{\up \p{A}}{\atmL{\octx}_R}{\n{B}}$, that transforms the quasi-proposition $\atmR{\octx}_L \limp \up \p{A} \pmir \atmL{\octx}_R$ into a well-formed proposition $\n{B}$ by nondeterministically abstracting atoms one-by-one from either the left or right contexts.
% The proposition $\n{B}$ is semantically equivalent to the quasi-proposition $\atmR{\octx}_L \limp \up \p{A} \pmir \atmL{\octx}_R$ in the sense that $\n{B}$ satisfies $\atmR{\octx}_L \oc \n{B} \oc \atmL{\octx}_R \reduces \octx'$ if, and only if, $\rfocus{\octx'}{\p{A}}$.

This principal choreographing judgment is defined by just two rules:
\begin{gather*}
  \infer{\chorsig{\theta}{\srsige}{\orsige}}{}
  \\
  \infer{\chorsig{\theta}{\srsig_0, \bigl(w^L_i \wc a \wc w^R_i \reduces w'_i\bigr)_{i \in \mathcal{I}}}{\orsig_0, \bigl(\defp{a} \defd \bigwith_{i \in \mathcal{I}} \n{A}_i\bigr)}}{
    \begin{array}[b]{@{}c@{}}
      \text{($\theta(a) = \defp{a}$)} \quad
      \chorsig{\theta}{\srsig_0}{\orsig_0} \quad
      \text{($\defp{a} \notin \dom{\orsig_0}$)}
      \\
      \multipremise{i \in \mathcal{I}}{
        \text{$\bigl(\theta(w^L_i) = \atmR{\octx}^L_i\bigr)$} \quad
        \text{$\bigl(\theta(w^R_i) = \atmL{\octx}^R_i\bigr)$} \quad
        \qimp{\atmR{\octx}^L_i}{\up \bigfuse \theta(w'_i)}{\atmL{\octx}^R_i}{\n{A}_i}}
    \end{array}}
\end{gather*}
The first of these rules is straightforward: an empty set of string rewriting axioms choreographs as an empty set of coinductive formula-as-process definitions.
The second rule is quite a lot to parse and needs to be broken down step by step:
\begin{enumerate}
\item
  Choose a symbol $a$ that is mapped by $\theta$ to a coinductively defined proposition, $\defp{a}$.
  Then reorganize the axioms $\srsig$, collecting together all axioms in $\srsig$ that have an $a$ in their premises.
  Let $\bigl(w^L_i \wc a \wc w^R_i \reduces w'_i\bigr)_{i \in \mathcal{I}}$ be those axioms, so that $\srsig = \srsig_0 , \bigl(w^L_i \wc a \wc w^R_i \reduces w'_i\bigr)_{i \in \mathcal{I}}$ for some $\srsig_0$.
\item
  Inductively construct definitions $\orsig_0$ from $\srsig_0$ and $\theta$, using the judgment $\chorsig{\theta}{\srsig_0}{\orsig_0}$.
  Check that $\orsig_0$ gives no definition for $\defp{a}$, otherwise there is some axiom in $\srsig_0$ that contains $a$ in its premise and $\bigl(w^L_i \wc a \wc w^R_i \reduces w'_i\bigr)_{i \in \mathcal{I}}$ does not correctly constitute all such axioms.
\item
  Check, using the side condition $\theta(w^L_i) = \atmR{\octx}^L_i$, that each $w^L_i$ contains only those symbols that map to right-directed atoms.
  Symmetrically, check, using the side condition $\theta(w^R_i) = \atmL{\octx}^R_i$, that each $w^R_i$ contains only symbols that map to left-directed atoms.
\item
  Elaborate each quasi-proposition $\atmR{\octx}^L_i \limp \up \bigfuse \theta(w'_i) \pmir \atmL{\octx}^R_i$ into a semantically equivalent proposition $\n{A}_i$.
  Based on \cref{lem:qimp-correct}, the left-focus judgment $\lfocus{\theta(w^L_i)}{\n{A}_i}{\theta(w^R_i)}{\bigfuse \theta(w'_i)}$ holds, and so this proposition acts as the image of the axiom $w^L_i \wc a \wc w^R_i \reduces w'_i$ under the role assignment $\theta$ -- that is, $\theta(w^L_i) \oc \n{A}_i \oc \theta(w^R_i) \reduces \theta(w'_i)$.
\item
  Collect the $\n{A}_i$s into a single definition, $\defp{a} \defd \bigwith_{i \in \mathcal{I}} \n{A}_i$, which, based on steps 2, 3, and 4, describes all of the axioms from $\srsig$ that contain $a$ in their premises -- that is, $\theta(w^L_i) \oc \defp{a} \oc \theta(w^R_i) \reduces_{\Set{\defp{a} \defd \with_{i \in \mathcal{I}} \n{A}_i}} \theta(w'_i)$.
\end{enumerate}

We shall now prove that this judgment produces definitions $\orsig$ that constitute a formula-as-process choreography $(\theta, \orsig)$ of the string rewriting specification $(\sralph, \srsig)$ -- that is, that $\chorsig{\theta}{\srsig}{\orsig}$ implies that $\theta$ is a (strong) bisimulation between $\reduces_{\srsig}$ and $\reduces_{\orsig}$.
As previously mentioned, because $\theta$ is injective, this amounts to proving that
\begin{equation*}
  \chorsig{\theta}{\srsig}{\orsig}
  \quad\text{only if}\enspace
  \begin{tikzcd}
    w \rar[reduces, subscript=\srsig] \dar[relation][swap]{\theta}
      & w\mathrlap{'} \dar[relation]{\theta}
    \\
    \theta(w) \rar[reduces, exists, subscript=\orsig] & \theta(w')
  \end{tikzcd}
  \enspace\text{and}\quad
  \begin{tikzcd}
    w \rar[reduces, exists, subscript=\srsig] \dar[relation][swap]{\theta}
      & w\mathrlap{'} \dar[relation, exists]{\theta}
    \\
    \theta(w) \rar[reduces, subscript=\orsig] & \octx \mathrlap{' \,.}
  \end{tikzcd}
  \hphantom{' \,.}
\end{equation*}
% As stated earlier, when $\chorsig{\theta}{\srsig}{\orsig}$, the string rewriting step $w \reduces_{\srsig} w'$ holds if, and only if, the ordered rewriting step $\theta(w) \reduces_{\orsig} \theta(w')$ holds.
We prove the first diagram as the following completeness \lcnamecref{thm:formula-as-process:choreograph-completeness} and then prove a stronger soundness \lcnamecref{thm:formula-as-process:choreograph-soundness} that implies the second diagram.

\begin{lemma}[Definition weakening]\label{lem:formula-as-process:definition-weakening}
  If $\octx \reduces_{\orsig} \octx'$ and $\dom{\orsig} \cap \dom{\orsig'} = \emptyset$, then $\octx \reduces_{\orsig, \orsig'} \octx'$.
  % Similarly, if $\octx \reduces_{\orsig, (\defp{a} \defd \n{A})} \octx'$ or $\octx \reduces_{\orsig, (\defp{a} \defd \n{B})} \octx'$, then $\octx \reduces_{\orsig, (\defp{a} \defd \n{A} \with \n{B})} \octx'$.
\end{lemma}
\begin{proof}
  By induction over the structure of the given rewriting step.
\end{proof}

% \begin{lemma}
%   If $(w \reduces w') \in \srsig$ and $\chorsig{\theta}{\srsig}{\orsig}$, then $\theta(w) \reduces_{\orsig} \theta(w')$.
% \end{lemma}
% \begin{proof}
%   By induction over the structure of the given choreographing derivation, $\chorsig{\theta}{\srsig}{\orsig}$.
%   \begin{itemize}
%   \item Consider the case in which $w = w_1 \oc a \oc w_2 \reduces w'$ is the axiom in question and
%     \begin{equation*}
%       \infer{\chorsig{\theta}{\srsig, w_1 \oc a \oc w_2 \reduces w'}{\orsig, \defp{a} \defd \n{A} \with \n{B}}}{
%         \text{($\theta(a) = \defp{a}$)} &
%         \chorax{\theta}{w_1}{\bigfuse \theta(w')}{w_2}{\n{B}} &
%         \chorsig{\theta}{\srsig}{\orsig, \defp{a} = \n{A}}}
%     \end{equation*}
%     It follows from \cref{??} that $\lfocus{\theta(w_1)}{\n{B}}{\theta(w_2)}{\bigfuse \theta(w')}$, and hence $\lfocus{\theta(w_1)}{\n{A} \with \n{B}}{\theta(w_2)}{\bigfuse \theta(w')}$.
%     And because $\rfocus{\theta(w')}{\bigfuse \theta(w')}$ and $\defp{a} \defd \n{A} \with \n{B}$, we have $\theta(w) = \theta(w_1) \oc \defp{a} \oc \theta(w_2) \reduces \theta(w')$.
  
%   \item Consider the case in which
%     \begin{equation*}
%       \infer{\chorsig{\theta}{\sig, v_1 \oc a \oc v_2 \reduces v'}{\sig', \hat{a} \defd \n{A} \with \n{B}}}{
%         \text{($\theta(a) = \hat{a}$)} &
%         \chorax{\theta}{v_1}{\bigfuse \theta(v')}{v_2}{\n{B}} &
%         \chorsig{\theta}{\sig}{\sig'} &
%         \text{($\sig'(\hat{a}) = \n{A}$)}}
%     \end{equation*}
%     and the axiom $w \reduces w'$ comes from $\sig$.
%     By the inductive hypothesis, $\theta(w) \reduces_{\sig'} \theta(w')$.
%     By \cref{??}, $\theta(w) \reduces_{\sig', \hat{a} \defd \n{A} \with \n{B}} \theta(w')$.
  
%   \item Consider the case in which
%     \begin{equation*}
%       \infer{\chorsig{\theta}{\sig, v_1 \oc a \oc v_2 \reduces v'}{\sig', \hat{a} \defd \n{B}}}{
%         \text{($\theta(a) = \hat{a}$)} &
%         \chorax{\theta}{v_1}{\bigfuse \theta(v')}{v_2}{\n{B}} &
%         \chorsig{\theta}{\sig}{\sig'} &
%         \text{($\hat{a} \notin \dom{\sig'}$)}}
%     \end{equation*}
%     and the axiom $w \reduces w'$ comes from $\sig$.
%     By the inductive hypothesis, $\theta(w) \reduces_{\sig'} \theta(w')$.
%     By \cref{??}, $\theta(w) \reduces_{\sig', \hat{a} \defd \n{B}} \theta(w')$.
%   %
%   \qedhere
%   \end{itemize}
% \end{proof}

\begin{theorem}\label{thm:formula-as-process:choreograph-completeness}\leavevmode
  If $\chorsig{\theta}{\srsig}{\orsig}$, then $w \reduces_{\srsig} w'$ implies $\theta(w) \reduces_{\orsig} \theta(w')$.%
  \marginnote{%
    \begin{tabular}{@{}c@{}}
      $\chorsig{\theta}{\srsig}{\orsig}$
      \\only if\\
    \begin{tikzcd}[ampersand replacement=\&]
      w \rar[reduces, subscript=\srsig] \dar[relation][swap]{\theta}
       \& w\mathrlap{'} \dar[relation]{\theta}
      \\
      \theta(w) \rar[reduces, exists, subscript=\orsig]
       \& \theta(w')
    \end{tikzcd}
    \end{tabular}}%
\end{theorem}
\begin{proof}
  By simultaneous structural induction on the given choreographing derivation, $\chorsig{\theta}{\srsig}{\orsig}$, and ordered rewriting step, $w \reduces_{\srsig} w'$.
  \begin{itemize}[listparindent=\parindent, itemsep=\dimexpr\itemsep+\parsep\relax, parsep=0pt]
  \item
    Consider the case in which
    \begin{equation*}
      \chorsig{\theta}{\srsig}{\orsig}
      \qquad\text{and}\qquad
      w =
      \infer[\jrule{$\reduces$C}]{w_1 \wc w_0 \wc w_2 \reduces_{\srsig} w_1 \wc w'_0 \wc w_2}{
        w_0 \reduces_{\srsig} w'_0}
      = w'
      \,.
    \end{equation*}
    By the inductive hypothesis, $\theta(w_0) \reduces_{\orsig} \theta(w'_0)$.
    It follows from ordered rewriting's $\jrule{$\reduces$C}$ rule that
    \begin{equation*}
      \theta(w) = \theta(w_1) \oc \theta(w_0) \oc \theta(w_2) \reduces_{\orsig} \theta(w_1) \oc \theta(w'_0) \oc \theta(w_2) = \theta(w')
      \,.
    \end{equation*}

  \item
    Consider the case in which
    \begin{gather*}
      \infer{\chorsig{\theta}{\srsig_0, \bigl(w^L_i \wc a \wc w^R_i \reduces w'_i\bigr)_{i \in \mathcal{I}}}{\orsig_0, \bigl(\defp{a} \defd \bigwith_{i \in \mathcal{I}} \n{A}_i\bigr)}}{
        \begin{array}[b]{@{}c@{}}
          \text{($\theta(a) = \defp{a}$)} \quad
          \chorsig{\theta}{\srsig_0}{\orsig_0} \quad
          \text{($\defp{a} \notin \dom{\orsig_0}$)}
          \\
          \multipremise{i \in \mathcal{I}}{
            \text{$\bigl(\theta(w^L_i) = \atmR{\octx}^L_i\bigr)$} \quad
            \text{$\bigl(\theta(w^R_i) = \atmL{\octx}^R_i\bigr)$} \quad
            \qimp{\atmR{\octx}^L_i}{\up \bigfuse \theta(w'_i)}{\atmL{\octx}^R_i}{\n{A}_i}}
        \end{array}}
    %
    \shortintertext{and}
    %
      w = \infer[\jrule{$\reduces$AX}]{w^L_k \wc a \wc w^R_k \reduces_{\srsig} w'_k}{(w^L_k \wc a \wc w^R_k \reduces w'_k) \in \srsig} = w'
    \end{gather*}
    for some $k \in \mathcal{I}$, where the axioms are $\srsig = \srsig_0, (w^L_i \wc a \wc w^R_i \reduces w'_i)_{i \in \mathcal{I}}$, and the definitions are $\orsig = \orsig_0 , (\bigwith_{i \in \mathcal{I}} \n{A}_i)$.

    By \cref{lem:qimp-correct}, $\lfocus{\theta(w^L_k)}{\n{A}_k}{\theta(w^R_k)}{\bigfuse \theta(w'_k)}$ holds.
    Appending a $\lrule{\with}$ rule, we have 
    $\lfocus{\theta(w^L_k)}{\bigwith_{i \in \mathcal{I}} \n{A}_i}{\theta(w^R_k)}{\bigfuse \theta(w'_k)}$.
    Because $\rfocus{\theta(w'_k)}{\bigfuse \theta(w'_k)}$, it follows by the $\jrule{$\reduces$I}$ rule that $\theta(w^L_k) \oc \bigl(\bigwith_{i \in \mathcal{I}} \n{A}_i\bigr) \oc \theta(w^R_k) \reduces_{\orsig} \theta(w'_k)$, and so $\theta(w) = \theta(w^L_k) \oc \defp{a} \oc \theta(w^R_k) \reduces_{\orsig} \theta(w'_k) = \theta(w')$.

  \item
    Consider the case in which
    \begin{gather*}
      \infer{\chorsig{\theta}{\srsig_0, \bigl(v^L_i \wc a \wc v^R_i \reduces v'_i\bigr)_{i \in \mathcal{I}}}{\orsig_0, \bigl(\defp{a} \defd \bigwith_{i \in \mathcal{I}} \n{A}_i\bigr)}}{
        \begin{array}[b]{@{}c@{}}
          \text{($\theta(a) = \defp{a}$)} \quad
          \chorsig{\theta}{\srsig_0}{\orsig_0} \quad
          \text{($\defp{a} \notin \dom{\orsig_0}$)}
          \\
          \multipremise{i \in \mathcal{I}}{
            \text{$\bigl(\theta(v^L_i) = \atmR{\octx}^L_i\bigr)$} \quad
            \text{$\bigl(\theta(v^R_i) = \atmL{\octx}^R_i\bigr)$} \quad
            \qimp{\atmR{\octx}^L_i}{\up \bigfuse \theta(v'_i)}{\atmL{\octx}^R_i}{\n{A}_i}}
        \end{array}}
    %
    \shortintertext{and}
    %
      \infer[\jrule{$\reduces$AX}]{w \reduces_{\srsig} w'}{
        (w \reduces w') \in \srsig_0}
    \end{gather*}
    where $(w \reduces w') \in \srsig_0$; the axioms are $\srsig = \srsig_0, (v^L_i \wc a \wc v^R_i \reduces v'_i)_{i \in \mathcal{I}}$; and the definitions are $\orsig = \orsig_0 , (\bigwith_{i \in \mathcal{I}} \n{A}_i)$.

    By the inductive hypothesis, $\theta(w) \reduces_{\orsig_0} \theta(w')$.
    It follows from weakening~\parencref{lem:formula-as-process:definition-weakening} that $\theta(w) \reduces_{\orsig} \theta(w')$.    

  \item 
    The case in which
    \begin{equation*}
      \infer{\chorsig{\theta}{\srsige}{\orsige}}{}
      \qquad\text{and}\qquad
      \infer[\jrule{$\reduces$AX}]{w \reduces_{\srsig} w'}{
        (w \reduces w') \in \srsig}
    \end{equation*}
    where $\srsig = \srsige$ and $\orsig = \orsige$ is vacuous.
  \qedhere
  % \item
  %   Consider the case in which
  %   \begin{gather*}
  %     \infer{\chorsig{\theta}{\srsig_0, (w_1 \wc a \wc w_2 \reduces w')}{\orsig_0, (\defp{a} \defd \n{A} \with \n{B})}}{
  %       \begin{array}[b]{@{}c@{}}
  %         \text{($\theta(w_1) = \atmR{\octx}_L$)} \quad
  %         \text{($\theta(a) = \defp{a}$)} \quad
  %         \text{($\theta(w_2) = \atmL{\octx}_R$)} \\
  %         \atmR{\octx}_L \limp \up \bigfuse \theta(w') \pmir \atmL{\octx}_R \rightsquigarrow \n{B} \quad
  %         \chorsig{\theta}{\srsig_0}{\orsig_0, (\defp{a} \defd \n{A})}
  %       \end{array}}
  %   \shortintertext{and}
  %     w =
  %     \infer[\jrule{$\reduces$AX}]{w_1 \wc a \wc w_2 \reduces_{\srsig} w'}{}
  %   \end{gather*}
  %   where $\srsig = \srsig_0 , (w_1 \wc a \wc w_2 \reduces w')$ and $\orsig = \orsig_0, (\defp{a} \defd \n{A} \with \n{B})$.

  %   By \cref{lem:chorax-sound-complete}, $\lfocus{\theta(w_1)}{\n{B}}{\theta(w_2)}{\bigfuse \theta(w')}$.
  %   Upon adding the $\lrule{\with}_2$ rule, $\lfocus{\theta(w_1)}{\n{A} \with \n{B}}{\theta(w_2)}{\bigfuse \theta(w')}$.
  %   Because $\rfocus{\theta(w')}{\bigfuse \theta(w')}$~\parencref{??}, it follows by the $\jrule{$\reduces$I}$ rule that $\theta(w_1) \oc (\n{A} \with \n{B}) \oc \theta(w_2) \reduces_{\orsig} \theta(w')$, and so $\theta(w) = \theta(w_1) \oc \defp{a} \oc \theta(w_2) = \theta(w_1) \oc (\n{A} \with \n{B}) \oc \theta(w_2) \reduces_{\orsig} \theta(w')$


  % \item
  %   Consider the case in which
  %   \begin{gather*}
  %     \infer{\chorsig{\theta}{\srsig_0, (w_1 \wc a \wc w_2 \reduces w')}{\orsig_0, (\defp{a} \defd \n{B})}}{
  %       \begin{array}[b]{@{}c@{}}
  %         \text{($\theta(w_1) = \atmR{\octx}_L$)} \quad
  %         \text{($\theta(a) = \defp{a}$)} \quad
  %         \text{($\theta(w_2) = \atmL{\octx}_R$)} \\
  %         \atmR{\octx}_L \limp \up \bigfuse \theta(w') \pmir \atmL{\octx}_R \rightsquigarrow \n{B} \quad
  %         \chorsig{\theta}{\srsig_0}{\orsig_0} \quad
  %         \text{($\defp{a} \notin \dom{\orsig_0}$)}
  %       \end{array}}
  %   \shortintertext{and}
  %     w =
  %     \infer[\jrule{$\reduces$AX}]{w_1 \wc a \wc w_2 \reduces_{\srsig} w'}{}
  %   \end{gather*}
  %   where $\srsig = \srsig_0 , (w_1 \wc a \wc w_2 \reduces w')$ and $\orsig = \orsig_0 , (\defp{a} \defd \n{B})$.

  %   By \cref{lem:chorax-sound-complete}, $\lfocus{\theta(w_1)}{\n{B}}{\theta(w_2)}{\bigfuse \theta(w')}$.
  %   Because $\rfocus{\theta(w')}{\bigfuse \theta(w')}$, it follows by the $\jrule{$\reduces$I}$ rule that $\theta(w_1) \oc \n{B} \oc \theta(w_2) \reduces_{\orsig} \theta(w')$, and so $\theta(w) = \theta(w_1) \oc \defp{a} \oc \theta(w_2) = \theta(w_1) \oc \n{B} \oc \theta(w_2) \reduces_{\orsig} \theta(w')$.
    
  % \item
  %   Consider the case in which
  %   \begin{gather*}
  %     \infer{\chorsig{\theta}{\srsig_0, (v_1 \wc b \wc v_2 \reduces v')}{\orsig_0, (\defp{b} \defd \n{A} \with \n{B})}}{
  %       \begin{array}[b]{@{}c@{}}
  %         \text{($\theta(v_1) = \atmR{\octx}_L$)} \quad
  %         \text{($\theta(b) = \defp{b}$)} \quad
  %         \text{($\theta(v_2) = \atmL{\octx}_R$)} \\
  %         \atmR{\octx}_L \limp \up \bigfuse \theta(v') \pmir \atmL{\octx}_R \rightsquigarrow \n{B} \quad
  %         \chorsig{\theta}{\srsig_0}{\orsig_0, (\defp{b} \defd \n{A})}
  %       \end{array}}
  %   \shortintertext{and}
  %     \infer[\jrule{$\reduces$AX}]{w \reduces_{\srsig} w'}{
  %       w \reduces w' \in \srsig_0}
  %   \end{gather*}
  %   where $\srsig = \srsig_0, (v_1 \wc b \wc v_2 \reduces v')$ and $\orsig = \orsig_0 , (\defp{b} \defd \n{A} \with \n{B})$.

  %   By the inductive hypothesis, $\theta(w) \reduces_{\orsig_0, (\defp{b} \defd \n{A})} \theta(w')$.
  %   It follows from the weakening \lcnamecref{??}~\parencref{??} that $\theta(w) \reduces_{\orsig} \theta(w')$, as required.

  % \item
  %   The case in which
  %   \begin{gather*}
  %     \infer{\chorsig{\theta}{\srsig_0, (v_1 \wc b \wc v_2 \reduces v')}{\orsig_0, (\defp{b} \defd \n{B})}}{
  %       \begin{array}[b]{@{}c@{}}
  %         \text{($\theta(v_1) = \atmR{\octx}_L$)} \quad
  %         \text{($\theta(b) = \defp{b}$)} \quad
  %         \text{($\theta(v_2) = \atmL{\octx}_R$)} \\
  %         \atmR{\octx}_L \limp \up \bigfuse \theta(v') \pmir \atmL{\octx}_R \rightsquigarrow \n{B} \quad
  %         \chorsig{\theta}{\srsig_0}{\orsig_0} \quad
  %         \text{($\defp{b} \notin \dom{\orsig_0}$)}
  %       \end{array}}
  %   \shortintertext{and}
  %     \infer[\jrule{$\reduces$AX}]{w \reduces_{\srsig} w'}{
  %       w \reduces w' \in \srsig_0}
  %   \end{gather*}
  %   where $\srsig = \srsig_0, (v_1 \wc b \wc v_2 \reduces v')$ and $\orsig = \orsig_0 , (\defp{b} \defd \n{B})$ is similar to the previous one.
    %
  \end{itemize}
\end{proof}




% \begin{lemma}
%   If $\chorax{\theta}{w_1}{\p{A}}{w_2}{\n{B}}$ and $\lfocus{\atmR{\octx}_L}{\n{B}}{\atmL{\octx}_R}{\p{C}}$, then $\atmR{\octx}_L = \theta(w_1)$ and $\atmL{\octx}_R = \theta(w_2)$ and $\p{A} = \p{C}$.
% \end{lemma}
% \begin{proof}
%   By induction over the structure of the given choreographing derivation, $\chorax{\theta}{w_1}{\p{A}}{w_2}{\n{B}}$.
%   \begin{itemize}
%   \item
%     Consider the case in which
%   \begin{equation*}
%     \infer{\chorax{\theta}{\emp}{\p{A}}{\emp}{\up \p{A}}}{}
%     \qquad\text{and}\qquad
%     \lfocus{\atmR{\octx}_1}{\up \p{A}}{\atmL{\octx}_2}{\p{C}}
%     \,.
%   \end{equation*}
%   By inversion on the left-focus derivation, $\atmR{\octx}_L = \octxe = \theta(\emp)$ and $\atmL{\octx}_R = \octxe = \theta(\emp)$, as well as $\p{A} = \p{C}$, as required.

%   \item
%     Consider the case in which
%   \begin{equation*}
%     \infer{\chorax{\theta}{w_1 \oc b}{\p{A}}{w_2}{\atmR{b} \limp \n{B}}}{
%       \text{($\theta(b) = \atmR{b}$)} &
%       \chorax{\theta}{w_1}{\p{A}}{w_2}{\n{B}}}
%     \qquad\text{and}\qquad
%     \lfocus{\atmR{\octx}_L}{\atmR{b} \limp \n{B}}{\atmL{\octx}_R}{\p{C}}
%     \,.
%   \end{equation*}
%   By inversion on the left-focus derivation for $\atmR{b} \limp \n{B}$, there exists $\atmR{\octx}'_1$ such that $\atmR{\octx}_L = \atmR{\octx}'_L \oc \atmR{b}$ and $\lfocus{\atmR{\octx}'_L}{\n{B}}{\atmL{\octx}_R}{\p{C}}$.
%   It follows from the inductive hypothesis that $\atmR{\octx}'_L = \theta(w_1)$ and $\atmL{\octx}_R = \theta(w_2)$ and $\p{A} = \p{C}$.
%   So $\atmR{\octx}_L = \theta(w_1) \oc \atmR{b} = \theta(w_1 \oc b)$.

%   \item
%     The case in which
% \begin{equation*}
%     \infer{\chorax{\theta}{w_1 \oc b}{\p{A}}{w_2}{\n{B} \pmir \atmL{b}}}{
%       \text{($\theta(b) = \atmL{b}$)} &
%       \chorax{\theta}{w_1}{\p{A}}{w_2}{\n{B}}}
%     \qquad\text{and}\qquad
%     \lfocus{\atmR{\octx}_L}{\n{B} \pmir \atmL{b}}{\atmL{\octx}_R}{\p{C}}
%   \end{equation*}
%   is symmetric.
%   \qedhere
%   \end{itemize}
% \end{proof}


\begin{lemma}\label{lem:chorsig-atom-correct}
  If $\chorsig{\theta}{\srsig}{\orsig}$ and $\lfocus{\atmR{\octx}_L}{\defp{a}}{\atmL{\octx}_R}{_{\orsig} \p{C}}$, then there exists an axiom $(w_1 \oc a \oc w_2 \reduces w') \in \srsig$ such that $\atmR{\octx}_L = \theta(w_1)$, $\atmL{\octx}_R = \theta(w_2)$, and $\p{C} = \bigfuse \theta(w')$.
\end{lemma}
\begin{proof}
  By induction over the structure of the given choreographing derivation, $\chorsig{\theta}{\srsig}{\orsig}$.
  \begin{itemize}
  \item
    Consider the case in which
    \begin{gather*}
      \infer{\chorsig{\theta}{\srsig_0, \bigl(w^L_i \wc a \wc w^R_i \reduces w'_i\bigr)_{i \in \mathcal{I}}}{\orsig_0, \bigl(\defp{a} \defd \bigwith_{i \in \mathcal{I}} \n{A}_i\bigr)}}{
        \begin{array}[b]{@{}c@{}}
          \chorsig{\theta}{\srsig_0}{\orsig_0} \quad
          \text{($\theta(a) = \defp{a}$)} \quad
          \text{($\defp{a} \notin \dom{\orsig_0}$)}
          \\
          \multipremise{i \in \mathcal{I}}{
            \text{$\bigl(\theta(w^L_i) = \atmR{\lctx}^L_i\bigr)$} \quad
            \text{$\bigl(\theta(w^R_i) = \atmL{\lctx}^R_i\bigr)$} \quad
            \qimp{\atmR{\lctx}^L_i}{\up \bigfuse \theta(w'_i)}{\atmL{\lctx}^R_i}{\n{A}_i}}
        \end{array}}
    %
    \shortintertext{and}
    %
      \lfocus{\atmR{\octx}_L}{\defp{a} = \textstyle\bigwith_{i \in \mathcal{I}} \n{A}_i}{\atmL{\octx}_R}{_{\orsig} \p{C}}
    \end{gather*}
    where $\srsig = \srsig_0, \bigl(w^L_i \wc a \wc w^R_i \reduces w'_i\bigr)_{i \in \mathcal{I}}$ and $\orsig = \orsig_0, \bigl(\defp{a} \defd \bigwith_{i \in \mathcal{I}} \n{A}_i\bigr)$.

    By inversion on the left-focus derivation, either: $\lfocus{\atmR{\octx}_L}{\n{A}_k}{\atmL{\octx}_R}{\p{C}}$ for some $k \in \mathcal{I}$; or $\mathcal{I}$ is empty.
    \begin{itemize}
    \item
      If $\lfocus{\atmR{\octx}_L}{\n{A}_k}{\atmL{\octx}_R}{\p{C}}$ for some $k \in \mathcal{I}$, then \cref{lem:qimp-correct} allows us to conclude that $\atmR{\octx}_L = \atmR{\lctx}^L_k = \theta(w^L_k)$ and $\atmL{\octx}_R = \atmL{\lctx}^R_k = \theta(w^R_k)$ and $\p{C} = \bigfuse \theta(w'_k)$.
      Also, the axiom $w^L_k \wc a \wc w^R_k \reduces w'_k$ is contained in $\srsig$.
    \item
      Otherwise, if $\mathcal{I}$ is empty, then $\bigwith_{i \in \mathcal{I}} \n{A}_i = \top$.
      There is no $\lrule{\top}$ rule to derive $\lfocus{\atmR{\octx}_L}{\defp{a} = \top}{\atmL{\octx}_R}{_{\orsig} \p{C}}$, so this case is vacuous.
    \end{itemize}




  \item
    Consider the case in which
    \begin{gather*}
      \infer{\chorsig{\theta}{\srsig_0, \bigl(v^L_i \wc b \wc v^R_i \reduces v'_i\bigr)_{i \in \mathcal{I}}}{\orsig_0, \bigl(\defp{b} \defd \bigwith_{i \in \mathcal{I}} \n{B}_i\bigr)}}{
        \begin{array}[b]{@{}c@{}}
          \chorsig{\theta}{\srsig_0}{\orsig_0} \quad
          \text{($\theta(b) = \defp{b}$)} \quad
          \text{($\defp{b} \notin \dom{\orsig_0}$)}
          \\
          \multipremise{i \in \mathcal{I}}{
            \text{$\bigl(\theta(v^L_i) = \atmR{\lctx}^L_i\bigr)$} \quad
            \text{$\bigl(\theta(v^R_i) = \atmL{\lctx}^R_i\bigr)$} \quad
            \qimp{\atmR{\lctx}^L_i}{\up \bigfuse \theta(v'_i)}{\atmL{\lctx}^R_i}{\n{B}_i}}
        \end{array}}
    %
    \shortintertext{and}
    %
      \lfocus{\atmR{\octx}_L}{\defp{a}}{\atmL{\octx}_R}{_{\orsig} \p{C}}
    \end{gather*}
    where $a \neq b$ and $\srsig = \srsig_0, \bigl(v^L_i \wc b \wc v^R_i \reduces v'_i\bigr)_{i \in \mathcal{I}}$ and $\orsig = \orsig_0, \bigl(\defp{b} \defd \bigwith_{i \in \mathcal{I}} \n{B}_i\bigr)$.

    By the inductive hypthesis, there exists a string rewriting axiom $(w_1 \wc a \wc w_2 \reduces w') \in \srsig_0$ such that $\atmR{\octx}_L = \theta(w_1)$ and $\atmL{\octx}_R = \theta(w_2)$ and $\p{C} = \bigfuse \theta(w')$.
    The same axiom is contained in the signature $\srsig$.


  \item 
    The case in which
    \begin{equation*}
      \infer{\chorsig{\theta}{\srsige}{\orsige}}{}
      \qquad\text{and}\qquad
      \lfocus{\atmR{\octx}_L}{\defp{a}}{\atmL{\octx}_R}{_{\orsig} \p{C}}
    \end{equation*}
    where $\srsig = \srsige$ and $\orsig = \orsige$ is vacuous because there is no definition for $\defp{a}$ in the signature $\orsig$.
  \qedhere

  % \item
  %   Consider the case in which
  %   \begin{gather*}
  %     \infer{\chorsig{\theta}{\srsig_0, (v_1 \wc b \wc v_2 \reduces v')}{\orsig_0, (\defp{b} \defd \n{A} \with \n{B})}}{
  %       \begin{array}[b]{@{}c@{}}
  %         \text{($\theta(v_1) = \atmR{\lctx}_L$)} \quad
  %         \text{($\theta(b) = \defp{b}$)} \quad
  %         \text{($\theta(v_2) = \atmL{\lctx}_R$)} \\
  %         \atmR{\lctx}_L \limp \up \bigfuse \theta(v') \pmir \atmL{\lctx}_R \rightsquigarrow \n{B} \quad
  %         \chorsig{\theta}{\srsig_0}{\orsig_0, (\defp{b} \defd \n{A})}
  %       \end{array}}
  %   \shortintertext{and}
  %     \lfocus{\atmR{\octx}_L}{\defp{a}}{\atmL{\octx}_R}{_{\orsig} \p{C}}
  %   \end{gather*}
  %   where $a \neq b$ and $\srsig = \srsig_0, (v_1 \wc b \wc v_2 \reduces v')$ and $\orsig = \orsig_0, (\defp{b} \defd \n{A} \with \n{B})$.

  %   By the inductive hypothesis, there exists a string rewriting axiom $(w_1 \wc a \wc w_2 \reduces w') \in \srsig_0$ such that $\atmR{\octx}_L = \theta(w_1)$, $\atmL{\octx}_R = \theta(w_2)$, and $\p{C} = \bigfuse \theta(w')$.
  % The same axiom is contained in the signature $\srsig$.

  % \item
  %   Consider the case in which
  %   \begin{gather*}
  %     \infer{\chorsig{\theta}{\srsig_0, (v_1 \wc b \wc v_2 \reduces v')}{\orsig_0, (\defp{b} \defd \n{B})}}{
  %       \begin{array}[b]{@{}c@{}}
  %         \text{($\theta(v_1) = \atmR{\lctx}_L$)} \quad
  %         \text{($\theta(b) = \defp{b}$)} \quad
  %         \text{($\theta(v_2) = \atmL{\lctx}_R$)} \\
  %         \atmR{\lctx}_L \limp \up \bigfuse \theta(v') \pmir \atmL{\lctx}_R \rightsquigarrow \n{B} \quad
  %         \chorsig{\theta}{\srsig_0}{\orsig_0} \quad
  %         \text{($\defp{b} \notin \dom{\orsig_0}$)}
  %       \end{array}}
  %   \shortintertext{and}
  %     \lfocus{\atmR{\octx}_L}{\defp{a}}{\atmL{\octx}_R}{_{\orsig} \p{C}}
  %   \end{gather*}
  %   where $a \neq b$ and $\srsig = \srsig_0, (v_1 \wc b \wc v_2 \reduces v')$ and $\orsig = \orsig_0, (\defp{b} \defd \n{B})$.

  %   By the inductive hypothesis, there exists a string rewriting axiom $(w_1 \wc a \wc w_2 \reduces w') \in \srsig_0$ such that $\atmR{\octx}_L = \theta(w_1)$, $\atmL{\octx}_R = \theta(w_2)$, and $\p{C} = \bigfuse \theta(w')$.
  % The same axiom is contained in the signature $\srsig$.

  % \item
  %   Consider the case in which
  %   \begin{gather*}
  %     \infer{\chorsig{\theta}{\srsig_0, (w_1 \wc a \wc w_2 \reduces w')}{\orsig_0, (\defp{a} \defd \n{A} \with \n{B})}}{
  %       \begin{array}[b]{@{}c@{}}
  %         \text{($\theta(w_1) = \atmR{\lctx}_L$)} \quad
  %         \text{($\theta(a) = \defp{a}$)} \quad
  %         \text{($\theta(w_2) = \atmL{\lctx}_R$)} \\
  %         \atmR{\lctx}_L \limp \up \bigfuse \theta(w') \pmir \atmL{\lctx}_R \rightsquigarrow \n{B} \quad
  %         \chorsig{\theta}{\srsig_0}{\orsig_0, (\defp{a} \defd \n{A})}
  %       \end{array}}
  %   \shortintertext{and}
  %     \infer[\lrule{\with}_2]{\lfocus{\atmR{\octx}_L}{\defp{a} = \n{A} \with \n{B}}{\atmL{\octx}_R}{_{\orsig} \p{C}}}{
  %       \lfocus{\atmR{\octx}_L}{\n{B}}{\atmL{\octx}_R}{_{\orsig} \p{C}}}
  %   \end{gather*}
  %   where $\srsig = \srsig_0, (w_1 \wc a \wc w_2 \reduces w')$ and $\orsig = \orsig_0, (\defp{a} \defd \n{A} \with \n{B})$.

  %   By \cref{??}, $\atmR{\octx}_L = \atmR{\lctx}_L = \theta(w_1)$ and $\atmL{\octx}_R = \atmL{\lctx}_R = \theta(w_2)$ and $\p{C} = \bigfuse \theta(w')$.
  %   And the axiom $w_1 \wc a \wc w_2 \reduces w'$ is contained in the signature $\orsig$.

  % \item
  %   Consider the case in which
  %   \begin{gather*}
  %     \infer{\chorsig{\theta}{\srsig_0, (v_1 \wc a \wc v_2 \reduces v')}{\orsig_0, (\defp{a} \defd \n{A} \with \n{B})}}{
  %       \begin{array}[b]{@{}c@{}}
  %         \text{($\theta(v_1) = \atmR{\lctx}_L$)} \quad
  %         \text{($\theta(a) = \defp{a}$)} \quad
  %         \text{($\theta(v_2) = \atmL{\lctx}_R$)} \\
  %         \atmR{\lctx}_L \limp \up \bigfuse \theta(v') \pmir \atmL{\lctx}_R \rightsquigarrow \n{B} \quad
  %         \chorsig{\theta}{\srsig_0}{\orsig_0, (\defp{a} \defd \n{A})}
  %       \end{array}}
  %   \shortintertext{and}
  %     \infer[\lrule{\with}_1]{\lfocus{\atmR{\octx}_L}{\defp{a} = \n{A} \with \n{B}}{\atmL{\octx}_R}{_{\orsig} \p{C}}}{
  %       \lfocus{\atmR{\octx}_L}{\n{A}}{\atmL{\octx}_R}{_{\orsig} \p{C}}}
  %   \end{gather*}
  %   where $\srsig = \srsig_0, (v_1 \wc a \wc v_2 \reduces v')$ and $\orsig = \orsig_0, (\defp{a} \defd \n{A} \with \n{B})$.

  %   Let $\orsig' = \orsig_0 , (\defp{a} \defd \n{A})$.
  %   Then $\lfocus{\atmR{\octx}_L}{\defp{a} = \n{A}}{\atmL{\octx}_R}{_{\orsig'} \p{C}}$.
  %   By inductive hypothesis, there exists an axiom $(w_1 \wc a \wc w_2 \reduces w') \in \srsig_0$ such that $\atmR{\octx}_L = \theta(w_1)$ and $\atmL{\octx}_R = \theta(w_2)$ and $\p{C} = \bigfuse \theta(w')$.
  %   That same axiom is also contained in the signature $\srsig$.

  % \item
  %   Consider the case in which
  %   \begin{gather*}
  %     \infer{\chorsig{\theta}{\srsig_0, (w_1 \wc a \wc w_2 \reduces w')}{\orsig_0, (\defp{a} \defd \n{B})}}{
  %       \begin{array}[b]{@{}c@{}}
  %         \text{($\theta(w_1) = \atmR{\lctx}_L$)} \quad
  %         \text{($\theta(a) = \defp{a}$)} \quad
  %         \text{($\theta(w_2) = \atmL{\lctx}_R$)} \\
  %         \atmR{\lctx}_L \limp \up \bigfuse \theta(w') \pmir \atmL{\lctx}_R \rightsquigarrow \n{B} \quad
  %         \chorsig{\theta}{\srsig_0}{\orsig_0} \quad
  %         \text{($\defp{a} \notin \dom{\orsig_0}$)}
  %       \end{array}}
  %   \shortintertext{and}
  %     \lfocus{\atmR{\octx}_L}{\defp{a} = \n{B}}{\atmL{\octx}_R}{_{\orsig} \p{C}}
  %   \end{gather*}
  %   where $\srsig = \srsig_0, (w_1 \wc a \wc w_2 \reduces w')$ and $\orsig = \orsig_0, (\defp{a} \defd \n{B})$.

  %   By \cref{??}, $\atmR{\octx}_L = \atmR{\lctx}_L = \theta(w_1)$ and $\atmL{\octx}_R = \atmL{\lctx}_R = \theta(w_2)$ and $\p{C} = \bigfuse \theta(w')$.
  %   And the axiom $w_1 \wc a \wc w_2 \reduces w'$ is contained in the signature $\orsig$.


  % % \item
  % %   The case in which
  % % \begin{gather*}
  % %   \infer{\chorsig{\theta}{\srsig, v_1 \oc b \oc v_2 \reduces v'}{\orsig, \defp{a} \defd \n{A}, \defp{b} \defd \n{B}}}{
  % %     \text{($\theta(b) = \defp{b}$)} &
  % %     \chorax{\theta}{v_1}{\bigfuse \theta(v')}{v_2}{\n{B}} &
  % %     \chorsig{\theta}{\srsig}{\orsig, \defp{a} \defd \n{A}} &
  % %     \text{($\defp{b} \notin \dom{\orsig}$)}}
  % %   \\\text{and}\\
  % %   \lfocus{\atmR{\octx}_L}{\defp{a}}{\atmL{\octx}_R}{\p{C}}
  % % \end{gather*}
  % % is similar.

  % % \item
  % % Consider the case in which
  % % \begin{gather*}
  % %   \infer{\chorsig{\theta}{\srsig, v_1 \oc a \oc v_2 \reduces v'}{\orsig, \defp{a} \defd \n{A}_1 \with \n{A}_2}}{
  % %     \text{($\theta(a) = \defp{a}$)} &
  % %     \chorax{\theta}{v_1}{\bigfuse \theta(v')}{v_2}{\n{A}_2} &
  % %     \chorsig{\theta}{\srsig}{\orsig, \defp{a} \defd \n{A}_1}}
  % %   \\\text{and}\\
  % %   \lfocus{\atmR{\octx}_L}{\defp{a}}{\atmL{\octx}_R}{\p{C}}
  % %   \,.
  % % \end{gather*}
  % % There are two cases, according to whether the $\lfocus{\atmR{\octx}_L}{\defp{a}}{\atmL{\octx}_R}{\p{C}}$ derivation ends with the $\lrule{\with}_1$ or $\lrule{\with}_2$ rule.
  % %   \begin{itemize}
  % %   \item If the left-focus derivation ends with the $\lrule{\with}_2$ rule, then $\lfocus{\atmR{\octx}_L}{\n{A}_2}{\atmL{\octx}_R}{\p{C}}$.
  % %     Because $\chorax{\theta}{v_1}{\bigfuse \theta(v')}{v_2}{\n{A}_2}$, it follows from \cref{??} that $\atmR{\octx}_L = \theta(v_1)$ and $\atmL{\octx}_R = \theta(v_2)$ and $\p{C} = \bigfuse \theta(v')$.
  % %     Choose the axiom $w_1 \oc a \oc w_2 \reduces w'$ to be $v_1 \oc a \oc v_2 \reduces v'$.

  % %   \item Otherwise, if the left-focus derivation instead ends with the $\lrule{\with}_1$ rule, then $\lfocus{\atmR{\octx}_L}{\n{A}_1}{\atmL{\octx}_R}{\p{C}}$.
  % %     By the inductive hypothesis, $\atmR{\octx}_L = \theta(w_1)$, $\atmL{\octx}_R = \theta(w_2)$, and $\p{C} = \bigfuse \theta(w')$ for some string rewriting axiom $(w_1 \oc a \oc w_2 \reduces w') \in \srsig$.
  % %     The same axiom is contained in the signuare $\srsig, v_1 \oc a \oc v_2 \reduces v'$.
  % %   \end{itemize}

  % % \item
  % % Consider the case in which 
  % % \begin{equation*}
  % %   \infer{\chorsig{\theta}{\srsig, w_1 \oc a \oc w_2 \reduces w'}{\orsig, \defp{a} \defd \n{A}}}{
  % %     \text{($\theta(a) = \defp{a}$)} &
  % %     \chorax{\theta}{w_1}{\bigfuse \theta(w')}{w_2}{\n{A}} &
  % %     \chorsig{\theta}{\srsig}{\orsig} &
  % %     \text{($\defp{a} \notin \dom{\orsig}$)}}
  % % \end{equation*}
  % % Because $\chorax{\theta}{w_1}{\bigfuse \theta(w')}{w_2}{\n{A}_2}$, it follows from \cref{??} that $\atmR{\octx}_L = \theta(w_1)$ and $\atmL{\octx}_R = \theta(w_2)$ and $\p{C} = \bigfuse \theta(w')$.
  %
  \end{itemize}
\end{proof}

\begin{theorem}\label{thm:formula-as-process:choreograph-soundness}
  If $\chorsig{\theta}{\srsig}{\orsig}$ and $\theta(a) = \defp{a}$ and $\octx_L \oc \defp{a} \oc \octx_R \reduces_{\orsig} \octx'$, then either:
  \begin{itemize}
  \item $\octx_L = \octx'_L \oc \theta(w_1)$ and $\octx_R = \theta(w_2) \oc \octx'_R$ and $\octx' = \octx'_L \oc \theta(w') \oc \octx'_R$ for some contexts $\octx'_L$ and $\octx'_R$ and some strings $w_1$, $w_2$, and $w'$ such that $(w_1 \wc a \wc w_2 \reduces w') \in \srsig$ and $\lfocus{\theta(w_1)}{\defp{a}}{\theta(w_2)}{\bigfuse \theta(w')}$;
  \item $\octx_L \reduces_{\orsig} \octx'_L$ for some context $\octx'_L$ such that $\octx' = \octx'_L \oc \defp{a} \oc \octx_R$; or
  \item $\octx_R \reduces_{\orsig} \octx'_R$ for some context $\octx'_R$ such that $\octx' = \octx_L \oc \defp{a} \oc \octx'_R$.
  \end{itemize}
\end{theorem}
\begin{proof}
  As a negative proposition, $\defp{a}$ serves as a barrier for interactions between $\octx_L$ and $\octx_R$ -- in \ac{PFOR}, implications cannot consume negative propositions.
  Thus, any reduction on $\octx_L \oc \defp{a} \oc \octx_R$ must occur within either $\octx_L$ or $\octx_R$ alone or must arise from $\defp{a}$.

  If the reduction on $\octx_L \oc \defp{a} \oc \octx_R$ arises from $\defp{a}$, then it arises from a bipole that begins by focusing on $\defp{a}$.
  In other words, $\octx_L = \octx'_L \oc \atmR{\lctx}_L$ and $\octx_R = \atmL{\lctx}_R \oc \octx'_R$ and $\octx' = \octx'_L \oc \lctx' \oc \octx'_R$ for some contexts $\atmR{\lctx}_L$, $\atmL{\lctx}_R$, and $\lctx'$ and positive proposition $\p{C}$ such that $\lfocus{\atmR{\lctx}_L}{\defp{a}}{\atmL{\lctx}_R}{\p{C}}$ and $\rfocus{\lctx'}{\p{C}}$.
  By \cref{lem:chorsig-atom-correct}, there exists an axiom $(w_1 \wc a \wc w_2 \reduces w') \in \srsig$ such that $\atmR{\lctx}_L = \theta(w_1)$ and $\atmL{\lctx}_R = \theta(w_2)$ and $\p{C} = \bigfuse \theta(w')$.
  It follows that $\lctx' = \theta(w')$.
\end{proof}

% \begin{corollary}[Soundness]\label{cor:formula-as-process:choreograph-soundness}
%   If $\chorsig{\theta}{\srsig}{\orsig}$ and $\theta(w) \reduces_{\orsig} \octx'$, then there exists $w'$ such that $\octx' = \theta(w')$ and $w \reduces_{\srsig} w'$.%
%   \marginnote{%
%     \begin{tabular}{@{}c@{}}
%       $\chorsig{\theta}{\srsig}{\orsig}$
%       \\implies\\
%       $\begin{tikzcd}[ampersand replacement=\&]
%       w \rar[reduces, exists, subscript=\srsig] \dar[relation][swap]{\theta}
%        \& w\mathrlap{'} \dar[relation, exists]{\theta}
%       \\
%       \theta(w) \rar[reduces, subscript=\orsig]
%        \& \octx\mathrlap{'}
%     \end{tikzcd}
%     \hphantom{'}$
%     \end{tabular}}%
% \end{corollary}

\begin{corollary}[Choreography adequacy]\label{cor:formula-as-process:choreography-adequacy}
  If $\chorsig{\theta}{\srsig}{\orsig}$, then $\theta(w) \reduces_{\orsig} \octx'$ if, and only if, there exists $w'$ such that $w \reduces_{\orsig} w'$ and $\theta(w') = \octx'$.
\end{corollary}

\section{Extended example: Choreographing binary counters}\label{sec:formula-as-process:counters}

In this \lcnamecref{sec:formula-as-process:counters}, we revisit binary counters, \ie, binary representations of natural numbers equipped with increment and decrement operations.
Here we use them as an extended example of choreographing string rewriting specifications.

Recall from \cref{sec:string-rewriting:binary-counter} a string rewriting specification $(\sralph, \srsig)$ of binary counters where the alphabet $\sralph$ and the axioms $\srsig$ are:
\begin{equation*}
  \begin{lgathered}
    \sralph = \Set{e, b_0, b_1, i, d, z, s, b'_0}
    \\
    \srsig
      = \begin{array}[t]{@{}l@{}l@{}l@{}}
          (e \wc i \reduces e \wc b_1) \,, {} &
          (b_0 \wc i \reduces b_1) \,, &
          (b_1 \wc i \reduces i \wc b_0) \,, \\
          %
          (e \wc d \reduces z) \,, &
          (b_0 \wc d \reduces b'_0) \,, {} &
          (b_1 \wc d \reduces b_0 \wc s) \,, \\
          % 
          (z \wc b'_0 \reduces z) \,, &
          (s \wc b'_0 \reduces b_1 \wc s)
        \end{array}
  \end{lgathered}
\end{equation*}
% In this \lcnamecref{sec:formula-as-process:counters},
We will present several distinct choreographies of this specification, including an object-oriented choreography that treats the increment and decrement operations as messages, and a functional choreography that instead treats those operations as processes. 


\subsection{An object-oriented choreography}\label{sec:formula-as-process:counters-oo}

Let $\theta$ be%
\marginnote{$
  \theta = \{
    \begin{lgathered}[t]
      e \mapsto \smash{\defp{e}} , b_0 \mapsto \smash{\defp{b}_0} , b_1 \mapsto \smash{\defp{b}_1} , \\
      i \mapsto \atmL{i} , d \mapsto \atmL{d} , \\
      z \mapsto \atmR{z} , s \mapsto \atmR{s} , b'_0 \mapsto \smash{\defp{b}'_0} \}
    \end{lgathered}
$}
the role assignment that maps the bits $e$, $b_0$, and $b_1$ to coinductively defined processes $\defp{e}$, $\defp{b}_0$, and $\defp{b}_1$; increments $i$ and decrements $d$ to left-directed messages $\atmL{i}$ and $\atmL{d}$; unary constructors $z$ and $s$ to right-directed messages $\atmR{z}$ and $\atmR{s}$; and $b'_0$ to coinductively defined process $\defp{b}'_0$.%

Two axioms in $\srsig$ mention $e$ in their premises: $e \wc i \reduces e \wc b_1$ and $e \wc d \reduces z$.
Under the role assignment $\theta$, these axioms induce the rewritings
\begin{equation*}
  \ereduces[\orsig]{\defp{e} \oc \atmL{i}}{\defp{e} \oc \defp{b}_1}
  \qquad\text{and}\qquad
  \ereduces[\orsig]{\defp{e} \oc \atmL{d}}{\atmR{z}}
\end{equation*}
as constraints on $\orsig$ that must be satisfied if $(\theta, \orsig)$ is to be a meaningful choreography of the binary counter specification.
Solving these for $\defp{e}$, we obtain the definition
\begin{equation*}
  \defp{e} \defd (\defp{e} \fuse \defp{b}_1 \pmir \atmL{i}) \with (\atmR{z} \pmir \atmL{d})
  \,.
\end{equation*}
Similar reasoning allows us to construct coinductive definitions for $\defp{b}_0$, $\defp{b}_1$, and $\smash{\defp{b}'_0}\vphantom{b}$ as the solutions of the other constraints induced from the axioms $\srsig$ by $\theta$.
(See \cref{tbl:formula-as-process:deriving-oo-counter} for a sketch.)
%
\begin{table*}[tbp]
  \renewcommand{\arraystretch}{1.2}
  \begin{tabular}{@{}l@{\qquad}l@{\qquad}l@{}}
    \toprule
    \emph{Axioms, $\srsig$} &
    \emph{Rewriting constraints on $\orsig$} & \emph{Solution, $\orsig$}
    \\ \midrule
    $e \wc i \reduces e \wc b_1$ and $e \wc d \reduces z$ &
    $\ereduces[\orsig]{\defp{e} \oc \atmL{i}}{\defp{e} \oc \defp{b}_1}$ and $\ereduces[\orsig]{\defp{e} \oc \atmL{d}}{\atmR{z}}$
      & $\defp{e} \defd (\defp{e} \fuse \defp{b}_1 \pmir \atmL{i}) \with (\atmR{z} \pmir \atmL{d})$
    \\
    $b_0 \wc i \reduces b_1$ and $b_0 \wc d \reduces d \wc b'_0$ &
    $\ereduces[\orsig]{\defp{b}_0 \oc \atmL{i}}{\defp{b}_1}$ and $\ereduces[\orsig]{\defp{b}_0 \oc \atmL{d}}{\atmL{d} \oc \defp{b}'_0}$
      & $\defp{b}_0 \defd (\up \dn \defp{b}_1 \pmir \atmL{i}) \with (\atmL{d} \fuse \defp{b}'_0 \pmir \atmL{d})$
    \\
    $b_1 \wc i \reduces i \wc b_0$ and $b_1 \wc d \reduces b_0 \wc s$ &
    $\ereduces[\orsig]{\defp{b}_1 \oc \atmL{i}}{\atmL{i} \oc \defp{b}_0}$ and $\ereduces[\orsig]{\defp{b}_1 \oc \atmL{d}}{\defp{b}_0 \oc \atmR{s}}$
      & $\defp{b}_1 \defd (\atmL{i} \fuse \defp{b}_0 \pmir \atmL{i}) \with (\defp{b}_0 \fuse \atmR{s} \pmir \atmL{d})$
    \\
    $z \wc b'_0 \reduces z$ and $s \wc b'_0 \reduces b_1 \wc s$ &
    $\ereduces[\orsig]{\atmR{z} \oc \defp{b}'_0}{\atmR{z}}$ and $\ereduces[\orsig]{\atmR{s} \oc \defp{b}'_0}{\defp{b}_1 \oc \atmR{s}}$
      & $\defp{b}'_0 \defd (\atmR{z} \limp \atmR{z}) \with (\atmR{s} \limp \defp{b}_1 \fuse \atmR{s})$
    \\ \addlinespace \bottomrule
  \end{tabular}
  \caption{Deriving an object-oriented choreography of binary counters}\label{tbl:formula-as-process:deriving-oo-counter}
\end{table*}
%
In full, the solution to these constraints is the definitions $\orsig$:
\begin{equation*}
  \orsig =
  \begin{lgathered}[t]
    \bigl(\defp{e} \defd (\defp{e} \fuse \defp{b}_1 \pmir \atmL{i}) \with (\atmR{z} \pmir \atmL{d})\bigr) \,, \\
    \bigl(\defp{b}_0 \defd (\up \dn \defp{b}_1 \pmir \atmL{i}) \with (\atmL{d} \fuse \defp{b}'_0 \pmir \atmL{d})\bigr) \,, \\
    \bigl(\defp{b}_1 \defd (\atmL{i} \fuse \defp{b}_0 \pmir \atmL{i}) \with (\defp{b}_0 \fuse \atmR{s} \pmir \atmL{d})\bigr) \,, \\
    \bigl(\defp{b}'_0 \defd (\atmR{z} \limp \atmR{z}) \with (\atmR{s} \limp \defp{b}_1 \fuse \atmR{s})\bigr)
  \,.
  \end{lgathered}
\end{equation*}



In other words, under the role assignment $\theta$, the string rewriting axioms for the binary counter are choreographed to the coinductive propositions defined in $\orsig$.
It is easy, if tedious, to verify that 
% the judgment $\chorsig{\theta}{\srsig}{\orsig}$ holds -- 
the formal construction described in \lcnamecref{sec:formula-as-process:choreograph-formal} generates the same definitions, $\orsig$:
%
\begin{proposition}
  For the above string rewriting specification $(\sralph, \srsig)$ and role assignment $\theta$, the judgment $\chorsig{\theta}{\srsig}{\orsig}$ holds.
\end{proposition}

\newthought{This choreography} might be called \emph{object-oriented} for its similarity to the eponymous programming paradigm.
In that paradigm, computation is centered around message exchange between stateful objects -- data are stored by objects, and those data are manipulated by exchanging messages with the relevant objects.

This choreography of the binary counter specification behaves similarly:%
\footnote{For a study of the relationship between (session-typed) processes and objects, see \textcite{Balzer+Pfenning:AGERE15}.}
its data -- the bits $e$, $b_0$, and $b_1$ -- are represented as processes, and its operations -- the increments $i$ and decrements $d$ -- are represented as messages that the processes dispatch on.
%
% We refer to this choreography as \emph{object-oriented} for its similarity to the object-oriented progamming paradigm: because its data -- the bits $e$, $b_0$, and $b_1$ -- are represented as processes and its operations -- the increments $i$ and decrements $d$ -- are represented as messages.
%
For example, $\defp{e}$ is the coinductively defined process that waits to receive either the increment message $\atmL{i}$ or the decrement message $\atmL{d}$ from its right-hand neighbor.
If $\atmL{i}$ is received, then $\defp{e}$ spawns a new process, $\defp{b}_1$, to its right and then continues recursively as $\defp{e}$.
Otherwise, if $\atmL{d}$ is received, then $\defp{e}$ sends the message $\atmR{z}$ as a response.





\newthought{Recall from \cref{sec:string-rewriting:binary-counter}} that string rewriting specifications of binary counters were assigned natural number denotations according to the relations $\aval{}{}$, $\ainc{}{}$, and $\adec{}{}$.
Based on the role assignment that underlies this choreography, we can lift these denotations to choreographed counters:
For instance, $\octx$ is an increment context that denotes $n$ exactly when $\ainc{\theta^{-1}(\octx)}{n}$; and so $\defp{e} \oc \atmL{i} \oc \defp{b}_1$ denotes $3$ because $\theta^{-1}(\defp{e} \oc \atmL{i} \oc \defp{b}_1) = \ainc{e \wc i \wc b_1}{3}$.
We could even assign denotations directly to choreographed contexts by defining new $\aval{}{}$, $\ainc{}{}$, and $\adec{}{}$ relations on choreographed contexts.
\begin{inferences}
  \infer[\jrule{$\defp{e}$-V}]{\aval{\defp{e}}{0}}{}
  \and
  \infer[\jrule{$\defp{b}_0$-V}]{\aval{\octx \oc \defp{b}_0}{2n}}{
    \aval{\octx}{n}}
  \and
  \infer[\jrule{$\defp{b}_1$-V}]{\aval{\octx \oc \defp{b}_1}{2n+1}}{
    \aval{\octx}{n}}
  \\
  \infer[\jrule{$\defp{e}$-I}]{\ainc{\defp{e}}{0}}{}
  \and
  \infer[\jrule{$\defp{b}_0$-I}]{\ainc{\octx \oc \defp{b}_0}{2n}}{
    \ainc{\octx}{n}}
  \and
  \infer[\jrule{$\defp{b}_1$-I}]{\ainc{\octx \oc \defp{b}_1}{2n+1}}{
    \ainc{\octx}{n}}
  \and
  \infer[\jrule{$\atmL{i}$-I}]{\ainc{\octx \oc \atmL{i}}{n+1}}{
    \ainc{\octx}{n}}
  \\
  \infer[\jrule{$\atmL{d}$-D}]{\adec{\octx \oc \atmL{d}}{n}}{
    \ainc{\octx}{n}}
  \and
  \infer[\jrule{$\atmR{z}$-D}]{\adec{\atmR{z}}{0}}{}
  \and
  \infer[\jrule{$\atmR{s}$-D}]{\adec{\octx \oc \atmR{s}}{n+1}}{
    \ainc{\octx}{n}}
  \and
  \infer[\jrule{$\defp{b}'_0$-D}]{\adec{\octx \oc \defp{b}'_0}{2n}}{
    \adec{\octx}{n}}
\end{inferences}
We will say that a context $\octx$ is an \vocab{increment counter} or \vocab{increment context} if $\ainc{\octx}{n}$ for some natural number $n$;
likewise, we will say that a context $\octx$ is an \vocab{decrement counter} or \vocab{decrement context} if $\adec{\octx}{n}$ for some $n$.

\begin{theorem}
  The following hold for all $\octx$ and $n$.
  \begin{itemize}[nosep]
  \item $\aval{\octx}{n}$ if, and only if, $\octx = \theta(w)$ for some $w$ such that $\aval{w}{n}$.
  \item $\ainc{\octx}{n}$ if, and only if, $\octx = \theta(w)$ for some $w$ such that $\ainc{w}{n}$.
  \item $\adec{\octx}{n}$ if, and only if, $\octx = \theta(w)$ for some $w$ such that $\adec{w}{n}$.
  \end{itemize}
\end{theorem}
\begin{proof}
  In each direction, by structural induction on the derivation of the denotation.
\end{proof}

Just as we proved the string rewriting specification of binary counters to be adequate with respect to the natural number denotation, we can also show this object-oriented choreography to be adequate.
What is interesting is that the adequacy of this choreography comes for nearly free -- it can be established by composing the string rewriting specification's adequacy theorem with the theorems that show an arbitrary choreography to be sound and complete with respect to its underlying string rewriting specification.
% The adequacy of this choreography can be established by composing the adequacy of the string rewriting specification with the adequacy of the choreographing procedure.




Recall from \cref{??} that \cref{??} show that the string rewriting specification adequately describes increments and decrements:
\thmadequacysmalldecstring*
%
\coradequacydecstring*

Combining these \lcnamecrefs{??} with \cref{thm:formula-as-process:choreograph-completeness}, we have the immediate \lcnamecref{cor:choreographies:oo-counter-adequacy}:
\begin{restatable}[
  name=Adequacy of object-oriented choreography,
  label=cor:choreographies:oo-counter-adequacy
]{corollary}{coroocounteradequacy}
  The following hold.
  \begin{thmdescription}
  \item[Preservation]
    If $\ainc{\octx}{n}$ and $\octx \reduces_{\orsig} \octx'$, then $\ainc{\octx'}{n}$.
    If $\adec{\octx}{n}$ and $\octx \reduces_{\orsig} \octx'$, then $\adec{\octx'}{n}$.

  \item[Big-step]
    If $\adec{\octx}{n}$, then:
    \begin{itemize}[nosep]
    \item $\octx \Reduces_{\orsig} \atmR{z}$ if, and only if, $n = 0$;
    \item $\octx \Reduces_{\orsig} \octx' \oc \atmR{s}$ for some $\octx'$ such that $\ainc{\octx'}{n-1}$, if $n > 0$; and
    \item $\octx \Reduces_{\orsig} \octx' \oc \atmR{s}$ only if $n > 0$ and $\ainc{\octx'}{n-1}$.
    \end{itemize}
  \end{thmdescription}
\end{restatable}



\subsection{A functional choreography}\label{sec:formula-as-process:counters-functional}

The object-oriented choreography is not the only choreography possible for the binary counter specification, however.

Let $\theta'$ be%
\marginnote{$
  \theta' = \{
    \begin{lgathered}[t]
      e \mapsto \atmR{e} , b_0 \mapsto \atmR{b}_0 , b_1 \mapsto \atmR{b}_1 , \\
      i \mapsto \smash{\defp{\imath}} , d \mapsto \smash{\defp{d}} , \\
      z \mapsto \atmR{z} , s \mapsto \atmR{s} , b'_0 \mapsto \smash{\defp{b}'_0} \}
    \end{lgathered}
$}
a role assignment that is (roughly) dual to $\theta$ -- that is, let $\theta'$ map the bits $e$, $b_0$, and $b_1$ to right-directed messages $\atmR{e}$, $\atmR{b}_0$, and $\atmR{b}_1$; increments $i$ and decrements $d$ to coinductively defined processes $\defp{\imath}$ and $\defp{d}$; unary constructors $z$ and $s$ to right-directed messages $\atmR{z}$ and $\atmR{s}$; and $b'_0$ to the coinductively defined process $\defp{b}'_0$.%

Once again, we can construct a choreography from the string rewriting axioms $\srsig$ by solving constraints in the form of rewritings. 
Three axioms from $\srsig$ mention $i$ in their premises: $e \wc i \reduces e \wc b_1$, $b_0 \wc i \reduces b_1$, and $b_1 \wc i \reduces i \wc b_0$.
Under the role assignment $\theta'$, these axioms induce the rewritings
\begin{equation*}
  \ereduces[\orsig']{\atmR{e} \oc \defp{\imath}}{\atmR{e} \oc \atmR{b}_1}
  \qquad\text{and}\qquad
  \ereduces[\orsig']{\atmR{b}_0 \oc \defp{\imath}}{\atmR{b}_1}
  \qquad\text{and}\qquad
  \ereduces[\orsig']{\atmR{b}_1 \oc \defp{\imath}}{\defp{\imath} \oc \atmR{b}_0}
\end{equation*}
as constraints on $\orsig'$ that must be satisfied if $(\theta', \orsig')$ is to be a choreography of the binary counter specification.
Solving these constraints for $\defp{\imath}$, we obtain the definition
\begin{equation*}
  \defp{\imath} \defd (\atmR{e} \limp \atmR{e} \fuse \atmR{b}_1) \with (\atmR{b}_0 \limp \atmR{b}_1) \with (\atmR{b}_1 \limp \defp{\imath} \fuse \atmR{b}_0)
  \,.
\end{equation*}
Upon solving the remaining constraints for the other coinductively defined propositions, $\defp{d}$ and $\defp{b}'_0$,%
%
\begin{table*}[tbp]
  \renewcommand{\arraystretch}{1.3}
  \begin{tabular}{@{}l@{\qquad\enspace}l@{\qquad\enspace}l@{}}
    \toprule
    \emph{Axioms, $\srsig$} &
    \emph{Rewriting constraints on $\orsig'$} & \emph{Solution, $\orsig'$}
    \\ \midrule
    $e \wc i \reduces e \wc b_1$ and $b_0 \wc i \reduces b_1$ &
    $\ereduces[\orsig']{\atmR{e} \oc \defp{\imath}}{\atmR{e} \oc \atmR{b}_1}$ and $\ereduces[\orsig']{\atmR{b}_0 \oc \defp{\imath}}{\atmR{b}_1}$
    & $\defp{\imath} \defd (\atmR{e} \limp \atmR{e} \fuse \atmR{b}_1) \with (\atmR{b}_0 \limp \atmR{b}_1)$
    \\[-0.75ex]
    \quad and $b_1 \wc i \reduces i \wc b_0$ &
    \quad and $\ereduces[\orsig']{\atmR{b}_1 \oc \defp{\imath}}{\defp{\imath} \oc \atmR{b}_0}$ &
    $\hphantom{\defp{\imath} \defd {}} \with (\atmR{b}_1 \limp \defp{\imath} \fuse \atmR{b}_0)$
    \\
    $e \wc d \reduces z$ and $b_0 \wc d \reduces d \wc b'_0$ &
    $\ereduces[\orsig']{\atmR{e} \oc \defp{d}}{\atmR{z}}$ and $\ereduces[\orsig']{\atmR{b}_0 \oc \defp{d}}{\defp{d} \oc \defp{b}'_0}$
      & $\defp{d} \defd (\atmR{e} \limp \atmR{z}) \with (\atmR{b}_0 \limp \defp{d} \fuse \defp{b}'_0)$
    \\[-0.75ex]
    \quad and $b_1 \wc d \reduces b_0 \wc s$ &
    \quad and $\ereduces[\orsig']{\atmR{b}_1 \oc \defp{d}}{\atmR{b}_0 \oc \atmR{s}}$ &
    $\hphantom{\defp{d} \defd {}} \with (\atmR{b}_1 \limp \atmR{b}_0 \fuse \atmR{s})$
    \\
    $z \wc b'_0 \reduces z$ and $s \wc b'_0 \reduces b_1 \wc s$ &
    $\ereduces[\orsig']{\atmR{z} \oc \defp{b}'_0}{\atmR{z}}$ and $\ereduces[\orsig']{\atmR{s} \oc \defp{b}'_0}{\defp{b}_1 \oc \atmR{s}}$
      & $\defp{b}'_0 \defd (\atmR{z} \limp \atmR{z}) \with (\atmR{s} \limp \atmR{b}_1 \fuse \atmR{s})$
    \\ \addlinespace \bottomrule
  \end{tabular}
  \caption{Deriving a functional choreography of binary counters}\label{tbl:formula-as-process:deriving-functional-choreography}
\end{table*}%
%
\footnote{See \cref{tbl:formula-as-process:deriving-functional-choreography} for a sketch.}
we arrive at the definitions
\begin{equation*}
  \orsig' =
  \begin{lgathered}[t]
    \bigl(\defp{\imath} \defd (\atmR{e} \limp \atmR{e} \fuse \atmR{b}_1) \with (\atmR{b}_0 \limp \atmR{b}_1) \with (\atmR{b}_1 \limp \defp{\imath} \fuse \atmR{b}_0)\bigr) \,, \\
    \bigl(\defp{d} \defd (\atmR{e} \limp \atmR{z}) \with (\atmR{b}_0 \limp \defp{d} \fuse \defp{b}'_0) \with (\atmR{b}_1 \limp \atmR{b}_0 \fuse \atmR{s})\bigr) \,, \\
    \bigl(\defp{b}'_0 \defd (\atmR{z} \limp \atmR{z}) \with (\atmR{s} \limp \defp{b}_1 \fuse \atmR{s})\bigr)
    \,.
  \end{lgathered}
\end{equation*}
Again, it is easy to verify that these definitions are exactly those that are constructed by the formal description of the choreographing algorithm:
\begin{proposition}
  For the above string rewriting specification $(\sralph, \srsig)$ and role assignment $\theta'$, the judgment $\chorsig{\theta'}{\srsig}{\orsig'}$ holds.
\end{proposition}

In contrast with the previous, object-oriented choreography, this choreography treats its data -- the bits $e$, $b_0$, and $b_1$ -- as messages that are manipulated by processes that represent the operations -- increments $i$ and decrements $d$.
For this reason, the choreography $(\theta', \orsig')$ might be called \emph{functional} for its similarity to functional programming.

\clearpage
\subsection{Duality and other choreographies}

These two (roughly) dual object-oriented and functional choreographies hint at a fundamental duality between the object-oriented and functional programming paradigms.

It is briefly tempting to think that a general duality theorem for choreographies might exist.
Perhaps if $(\theta, \orsig)$ is a choreography of the specification $(\sralph, \srsig)$, there exists a dual choreography $(\theta^{\bot}, \orsig^{\bot})$ in which $\theta^{\bot}$ maps a symbol $a$ to a message exactly when $\theta$ mapped it to a process?

Such a theorem does not exist.
As a counterexample, recall the string rewriting specification $(\sralph, \srsig)$ and choreography $(\theta, \orsig)$ given by
\begin{equation*}
  \begin{lgathered}
    \sralph = \Set{a, b} \\
    \srsig = (a \wc b \reduces b) \,, (b \reduces \emp)
  \end{lgathered}
  \qquad\text{and}\qquad
  \begin{lgathered}
    \theta = \Set{ a \mapsto \atmR{a} , b \mapsto \defp{b} } \\
    \orsig = \bigl(\defp{b} \defd (\atmR{a} \limp \up \dn \defp{b}) \with \up \one\bigr)
    \,.
  \end{lgathered}
\end{equation*}
For this choreography, the dual role assignment, $\theta^{\bot}$, would map $b$ to a message, either $\atmL{b}$ or $\atmR{b}$.
And, the axiom $b \reduces \emp$ would, under $\theta^{\bot}$, induce either $\ereduces[\orsig^{\bot}]{\atmL{b}}{(\octxe)}$ or $\ereduces[\orsig^{\bot}]{\atmR{b}}{(\octxe)}$ as a constraint.
Neither of these possible constraints is satisfiable in the formula-as-process ordered rewriting framework because the premises contain only passive messages.%
\footnote{See \cref{??}.}

One might also ask if a theorem is possible if some additional conditions are imposed on the specification.
% It appeasr that the conditions necessary for such a theorem are so extremely narrow as to make any obtained theorem of little value.
For instance, at first glance, a duality theorem might seem possible for those specifications in which all axioms' premises contain exactly two symbols.
Unfortunately, this is not the case.
Consider, as a counterexample, the string rewriting specification $(\sralph, \srsig)$ and the choreography $(\theta, \orsig)$ given by
\begin{equation*}
  \begin{lgathered}
    \sralph = \Set{ a, b, c} \\
    \srsig = (a \wc b \reduces b) \,, (b \wc c \reduces b)
  \end{lgathered}
  \qquad\text{and}\qquad
  \begin{lgathered}
    \theta = \Set{ a \mapsto \atmR{a} , b \mapsto \defp{b} , c \mapsto \atmL{c} } \\
    \orsig = \bigl(\defp{b} \defd (\atmR{a} \limp \up \dn \defp{b}) \with (\up \dn \defp{b} \pmir \atmL{c})\bigr)
    \,.
  \end{lgathered}
\end{equation*}
For this choreography, the dual role assignment, $\theta^{\bot}$, would map $b$ to a message, either $\atmL{b}$ or $\atmR{b}$.
But either choice leads to unsatisfiable constraints.
Depending on whether $\theta^{\bot}$ maps $b$ to $\atmL{b}$ or $\atmR{b}$, the induced constraints are either:
\begin{equation*}
  \ereduces[\orsig^{\bot}]{\defp{a} \oc \atmL{b}}{\atmL{b}}
  \enspace\text{and}\enspace
  \ereduces[\orsig^{\bot}]{\atmL{b} \oc \defp{c}}{\atmL{b}}
  \qquad\text{or}\qquad
  \ereduces[\orsig^{\bot}]{\defp{a} \oc \atmR{b}}{\atmR{b}}
  \enspace\text{and}\enspace
  \ereduces[\orsig^{\bot}]{\atmR{b} \oc \defp{c}}{\atmR{b}}
  \,,
\end{equation*}
respectively.
In either case, the constraints are unsatisfiable because one premise in each group involves a message directed outward, away from the premise's process.%
\footnote{See \cref{??}.}

We will return to this idea of a duality theorem in \cref{??}, where we will see that session types provide just the right conditions for such a theorem.
% a b --> ...
% b c --> ...

% b = (a \ ..) & (... / c)

% Suppose that $\theta$ induces $\atmR{a} \oc \defp{b} \oc \atmL{c} \reduces_{\orsig} \mathord{\dotsm}$ as a constraint.
% Then $\theta^{\bot}$ would assign the symbols $a$ and $c$ process roles and the symbol $b$ a message role, resulting in either $\defp{a} \oc \atmL{b} \oc \defp{c} \reduces_{\orsig^{\bot}} \mathord{\dotsm}$ or $\defp{a} \oc \atmR{b} \oc \defp{c} \reduces_{\orsig^{\bot}} \mathord{\dotsm}$ as a constraint.
% Neither of these possible constraints is satisfiable in the formula-as-process ordered rewriting framework because the premises contain more than one process.%
% \footnote{See \cref{??}.}

% A general duality theorem does not exist: even if the constraint $\atmR{a} \oc \defp{b} \oc \atmL{c} \reduces_{\orsig} \theta(w')$ is satisfiable, the dual\fixnote{?} $\defp{a} \oc \atmL{b} \oc \defp{c} \reduces \theta^{\bot}(w')$ nor $\defp{a} \oc \atmR{b} \oc \defp{c} \reduces \theta^{\bot}(w')$ are [...].


\newthought{Besides these} object-oriented and functional choreographies, the binary counter specification has two other, related choreographies.
The two alternatives are broadly similar to the object-oriented and functional choreographies, with two exceptions: the unary constructors $z$ and $s$ are treated as processes, not messages; and $b'_0$ is treated as a message, not a process.
Instead of responding to a decrement with either a $\atmR{z}$ or $\atmR{s}$ message, these choreographies transform the binary counter into head-unary form where the head is a process -- either $\defp{z}$ or $\defp{s}$.

One problem with these choreographies, however, is that, without a $\atmR{z}$ or $\atmR{s}$ response message, there is no way to observe the counter's state.
Under these choreographies, all counters are, in some sense, equivalent because they can't be distinguished by any of the nonexistent observations.
We will return to the idea of observational equivalence in the following \lcnamecref{ch:??}.

\begin{table*}[tb]
  \renewcommand{\arraystretch}{1.2}
  \begin{tabular}{@{}ll@{}}
    \toprule
    \emph{Object-oriented--like alternative}
    & \emph{Functional-like alternative}
    \\ \midrule
    $\begin{aligned}[t]
       \theta^* &= \theta[z \mapsto \defp{z} , s \mapsto \defp{s} , b'_0 \mapsto \atmL{b}'_0]
       \\
       \orsig^* &=
       \begin{lgathered}[t]
         \bigl(\proc{e} \defd (\proc{e} \fuse \proc{b}_1 \pmir \atmL{i}) \with (\up \dn \proc{z} \pmir \atmL{d})\bigr) \,, \\
         \bigl(\proc{b}_0 \defd (\up \dn \proc{b}_1 \pmir \atmL{i}) \with (\atmL{d} \fuse \atmL{b}'_0 \pmir \atmL{d})\bigr) \,, \\
         \bigl(\proc{b}_1 \defd (\atmL{i} \fuse \proc{b}_0 \pmir \atmL{i}) \with (\proc{b}_0 \fuse \proc{s} \pmir \atmL{d})\bigr) \,, \\
         \bigl(\proc{z} \defd \up \dn \proc{z} \pmir \atmL{b}'_0\bigr) \,, \\
         \bigl(\proc{s} \defd \proc{b}_1 \fuse \proc{s} \pmir \atmL{b}'_0\bigr)
       \end{lgathered}
     \end{aligned}$
    &
    $\begin{aligned}[t]
       \theta^\dag &= \theta'[z \mapsto \defp{z} , s \mapsto \defp{s} , b'_0 \mapsto \atmL{b}'_0]
       \\
       \orsig^\dag &=
       \begin{lgathered}[t]
         \bigl(\defp{\imath} \defd (\atmR{e} \limp \atmR{e} \fuse \atmR{b}_1) \with (\atmR{b}_0 \limp \atmR{b}_1) \with (\atmR{b}_1 \limp \defp{\imath} \fuse \atmR{b}_0)\bigr) \,, \\
         \bigl(\defp{d} \defd (\atmR{e} \limp \up \dn \defp{z}) \with (\atmR{b}_0 \limp \defp{d} \fuse \atmL{b}'_0) \with (\atmR{b}_1 \limp \atmR{b}_0 \fuse \defp{s})\bigr) \,, \\
         \bigl(\defp{z} \defd \up \dn \defp{z} \pmir \atmL{b}'_0\bigr) \,, \\
         \bigl(\defp{s} \defd \atmR{b}_1 \fuse \defp{s} \pmir \atmL{b}'_0\bigr)
       \end{lgathered}
     \end{aligned}$
    \\ \addlinespace \bottomrule
  \end{tabular}
  \caption{Two other choreographies for the binary counter specification}
\end{table*}


\section{Extended example: Choreographing \aclp*{NFA}}\label{sec:formula-as-process:nfa}

Recall from \cref{??} our string rewriting specification of how \iac{NFA} processes its input.
Given \iac{NFA} $\aut{A} = (Q, \nfapow, F)$\fixnote{symbol?} over an input alphabet $\ialph$, the \ac{NFA}'s operational semantics is adequately captured by the string rewriting specification $(\ialph \dunion \Set{\eow, \symrej}, \srsig)$, where the axioms $\srsig$ are given by
\begin{equation*}
  \!\begin{aligned}
    \srsig = {}
      &\Set{ a \wc q \reduces q'_a \given (a \in \ialph) \land (q \in Q) \land (q'_a \in \nfapow(q, a)) } \\
      &{} \union \Set{ \eow \wc q \reduces F(q) \given q \in Q }
  \end{aligned}
\text{ where }
  F(q) = \begin{cases*}
           \emp & if $q \in F$ \\
           \symrej & if $q \notin F$\,.
         \end{cases*}
\end{equation*}
As a second extended example of a choreography, we would now like to choreograph this specification in the formula-as-process ordered rewriting framework.
As with the binary counter specification, there are, in fact, two distinct choreographies for this string rewriting specification of \acp{NFA} -- one functional and one object-oriented.

\subsection{A functional choreography}\label{ch:formula-as-process:nfa-functional}

Let $\theta$ be%
\marginnote{$
  \theta =
    \!\begin{lgathered}[t]
      \Set{ a \mapsto \atmR{a} \given a \in \ialph } \union \Set{ \eow \mapsto \atmR{\eow} } \\
      {} \union \Set{ q \mapsto \defp{q} \given q \in Q } \\
      {} \union \Set{ \symrej \mapsto \atmR{\symrej} }
    \end{lgathered}
$}
the role assignment that maps each input symbol $a \in \ialph$ to a right-directed message, $\atmR{a}$; the end-of-word marker, $\eow$, to a right-directed message, $\atmR{\eow}$; each state $q \in Q$ to a coinductively defined proposition, $\defp{q}$; and the rejection symbol, $\symrej$, to a right-directed message, $\atmR{\symrej}$.
In other words, the input word is transmitted as a sequence of messages to a process $\defp{q}$ that tracks the \ac{NFA}'s current state.

% One possible choreography for this specification interprets each input symbol $a \in \ialph$ as a right-directed atom, $\atmR{a}$; each state $q \in Q$ as a recursively defined proposition, $\defp{q}$; and the end-of-word marker, $\emp$, as a right-directed atom, $\atmR{\emp}$.
% In other words, the input string is transmitted as a sequence of messages to a process $\defp{q}$ that tracks the \ac{NFA}'s current state.

% In other words, the \ac{NFA}'s input is treated as a sequence of messages, $\atmR{\emp} \oc \atmR{a}_n \dotsm \atmR{a}_2 \oc \atmR{a}_1$, and the \ac{NFA}'s states are treated as [recursive] processes.

Choose an arbitrary state $q \in Q$.
Under the role assignment $\theta$, the axioms in $\srsig$ that mention $q$ in their premises induce the rewritings
\begin{equation*}
  \begin{lgathered}
    \Set{ \ereduces[\orsig]{\atmR{\eow} \oc \defp{q}}{\atmR{F}(q)} } \\
    \,{} \union {\textstyle \bigunion_{a \in \ialph} \Set{ \ereduces[\orsig]{\atmR{a} \oc \defp{q}}{\defp{q}'_a} \given q'_a \in \nfapow(q, a) }}
  \end{lgathered}
  \enspace\text{where }
  \atmR{F}(q) = \begin{cases*}
                  (\octxe) & if $q \in F$ \\
                  \atmR{\symrej} & if $q \notin F$
                \end{cases*}
\end{equation*}
as constraints on $\orsig$ that must be satisfied if $(\theta, \orsig)$ is to be a meaningful choreography of the \ac{NFA} specification $(\ialph \dunion \Set{\eow, \symrej}, \srsig)$.
Solving these constraints for $\defp{q}$, we obtain the definition
\begin{gather*}
  \defp{q} \defd
    (\atmR{\eow} \limp \up \bigfuse \atmR{F}(q)) \with
    \bigwith_{a \in \ialph} \bigl(\atmR{a} \limp (
      \textstyle \bigwith_{q'_a \in \nfapow(q, a)} \up \dn \defp{q}'_a)
    \bigr)
  \,,
\intertext{and therefore the full choreographing signature is}
  \orsig = \left(
    \defp{q} \defd
      (\atmR{\eow} \limp \up \bigfuse \atmR{F}(q)) \with
      \bigwith_{a \in \ialph} \bigl(\atmR{a} \limp (
        \textstyle \bigwith_{q'_a \in \nfapow(q, a)} \up \dn \defp{q}'_a)
      \bigr)
    \right)_{q \in Q}
\end{gather*}
As a concrete example, the adjacent \lcnamecref{fig:formula-as-process:nfa-example}
%
\begin{marginfigure}
  \centering
  \begin{tikzpicture}
    \graph [automaton] {
      q_0
       -> ["a,b", loop above]
      q_0
       -> ["b"]
      q_1 [accepting]
       -> ["a,b"]
      q_2
       -> ["a,b", loop above]
      q_2;
    };
  \end{tikzpicture}

  \begin{equation*}
    \orsig =
    \begin{lgathered}[t]
      \bigl(\defp{q}_0 \defd (\atmR{a} \limp \up \dn \defp{q}_0) \with (\atmR{b} \limp (\up \dn \defp{q}_0 \with \up \dn \defp{q}_1)) \with (\atmR{\eow} \limp \up \atmR{\symrej})\bigr) \,, \\
      \bigl(\defp{q}_1 \defd (\atmR{a} \limp \up \dn \defp{q}_2) \with (\atmR{b} \limp \up \dn \defp{q}_2) \with (\atmR{\eow} \limp \up \one)\bigr) \,, \\
      \bigl(\defp{q}_2 \defd (\atmR{a} \limp \up \dn \defp{q}_2) \with (\atmR{b} \limp \up \dn \defp{q}_2) \with (\atmR{\eow} \limp \up \atmR{\symrej})\bigr)
    \end{lgathered}
  \end{equation*}

  \caption{\Iac*{NFA} that accepts exactly those words, over the alphabet $\ialph = \set{a,b}$, that end with $b$; and a choreography}\label{fig:formula-as-process:nfa-example}
\end{marginfigure}%
%
recalls from \cref{??} \iac{NFA} that accepts those words, over the alphabet $\ialph = \Set{a,b}$, that end with $b$, and also gives a choreographing signature for that \ac{NFA}.

% \begin{equation*}
%   \orsig =
%     \bigl(
%       \defp{q} \defd
%         (\atmR{\eow} \limp \atmR{F}(q)) \with
%         \bigwith_{a \in \ialph} \bigl(
%           \textstyle \bigwith_{q'_a} (\atmR{a} \limp \up \dn \defp{q}'_a)
%         \bigr)
%     \bigr)
%   \,.
% \end{equation*}

% Under this choreographing assignment, the string rewriting axioms become rewriting steps that must be derivable:
% \begin{equation*}
%   \atmR{a} \oc \defp{q}
%     \reduces \defp{q}'_a
%   %
%   \atmR{\emp} \oc \defp{q}
%     \reduces \begin{cases*}
%                & if $q \in F$ \\
%                & if $q \notin F$
%              \end{cases*}
% \end{equation*}
% These required rewritings are indeed local: each one contains exactly one recursively defined process in its premise, with all remaining propositions in its premise being input messages for that process.
% For instance, each $\atmR{a} \oc \defp{q} \reduces \defp{q}'_a$

% Solving for each recursively defined proposition, we have one definition, 
% \begin{equation*}
%   \defp{q} \defd \bigwith_{a \in \ialph} \bigwith_{q\smash{'_a}} (\atmR{a} \limp \defp{q}'_a) \with (\atmR{\emp} \limp \nfa{F}(q))
%   \,,
% \end{equation*}
% for each \ac{NFA} state $q \in Q$.
Similarly to one of the binary counter's choreographies, this choreography might be called \enquote*{functional} because the data, an input string, are represented by messages that are acted on in a function-like way by the current state's process, $\defp{q}$.

\begin{proposition}\label{prop:formula-as-process:nfa-functional-chorsig}
  For the above string rewriting specification $(\ialph \dunion \Set{\eow, \symrej}, \srsig)$ and role assignment $\theta$, the judgment $\chorsig{\theta}{\srsig}{\orsig}$ holds (up to focusing equivalence).
\end{proposition}
% %
% \noindent
% The proof of this \lcnamecref{thm:choreographies:nfa-functional-chorsig} is a completely straightforward, if tedious, formalization of the preceding intuition, along the line of \cref{??}.

Recall from \cref{sec:string-rewriting:nfa} the adequacy \lcnamecref{thm:nfa-adequacy-string-rewriting} for the string rewriting specification of \acp{NFA}.
%
\nfaadequacystringrewriting*
%
What we would like now is a \lcnamecref{thm:??} that relates \iac{NFA} not to a string rewriting specification but to the above functional choreography.
In the specific instance of this functional choreography of the \ac{NFA}, \cref{thm:??,thm:??}, the adequacy of the choreographing procedure, can be reduced to the following \lcnamecref{??} -- a result that relates the string rewriting specification to the choreography.
\begin{corollary}
  For the above string rewriting specification $(\ialph \dunion \Set{\eow, \symrej}, \srsig)$ and choreography $(\theta, \orsig)$, the choreography is adequate with respect to the specification:
  \begin{itemize}[nosep]
  \item
    $a \wc q \reduces_{\srsig} q'_a$ only if $\atmR{a} \oc \defp{q} \reduces_{\orsig} \defp{q}'_a$.
    Moreover, if $\atmR{a} \oc \defp{q} \reduces_{\orsig} \octx'$, then $a \wc q \reduces_{\srsig} q'_a$ for some state $q'_a$ such that $\octx' = \defp{q}'_a$.
  \item
    $\eow \wc q \reduces_{\srsig} F(q)$ if, and only if, $\atmR{\eow} \oc \defp{q} \reduces_{\orsig} \atmR{F}(q)$.
  \item
    $\rev{w} \wc q \reduces_{\srsig} q'$ only if $\rev{\atmR{a}} \oc \defp{q} \reduces_{\orsig} \defp{q}'$.
    Moreover, if $\rev{\atmR{w}} \oc \defp{q} \reduces_{\orsig} \octx'$, then $\rev{w} \wc q \reduces_{\srsig} q'$ for some state $q'$ such that $\octx' = \defp{q}'$.
  \end{itemize}
\end{corollary}
%
\noindent By composing this with \cref{??}, the adequacy of the \ac{NFA} string rewriting specification, we arrive at:
%
\begin{marginfigure}
  \begin{equation*}
    \rev{\atmR{w}} =
      \begin{cases*}
        (\octxe) & if $w = \emp$ \\
        \rev{\atmR{w}_0} \oc \atmR{a} & if $w = a \wc w_0$
      \end{cases*}
  \end{equation*}
  \caption{An anti-homomorphism from input words to sequences of right-directed messages.
    Notice that $\rev{\atmR{w}} = \theta(\rev{w})$, where $\rev{}$ is defined in \cref{??}.}\label{fig:formula-as-process:msg-rev}
\end{marginfigure}%
%
\begin{restatable}[
  name=Adequacy of the functional \acs*{NFA} choreography,
  label=cor:formula-as-process:nfa-fnchor-adequacy
]{corollary}{cornfafnchoradequacy}
  Let $\aut{A} = (Q, \nfapow, F)$ be \iac{NFA} over the input alphabet $\ialph$, with choreography $(\theta, \orsig)$ as described above.
  The following hold.
  \begin{itemize}[nosep]
  \item
    $q \nfareduces[a] q'_a$ only if $\atmR{a} \oc \defp{q} \reduces_{\orsig} \defp{q}'_a$.
    Also, if $\atmR{a} \oc \defp{q} \reduces_{\orsig} \octx'$, then $q \nfareduces[a] q'_a$ for some state $q'_a$ such that $\octx' = \defp{q}'_a$.
  \item
    $q \in F$ if, and only if, $\atmR{\eow} \oc \defp{q} \reduces_{\orsig} (\octxe)$.
  \item
    $q \nfareduces[w] q'$ only if $\rev{\atmR{w}} \oc \defp{q} \Reduces_{\orsig} \defp{q}'$.
    Also, if $\rev{\atmR{w}} \oc \defp{q} \Reduces_{\orsig} \octx'$, then $q \nfareduces[w] q'$ for some state $q'$ such that $\octx' = \defp{q}'$.
  \end{itemize}
\end{restatable}
\noindent
This \lcnamecref{cor:formula-as-process:nfa-fnchor-adequacy} gives -- nearly for free -- an end-to-end adequacy result for the functional \ac{NFA} choreography with respect to the mathematical model of \acp{NFA}.
Examining the first part, its first clause captures the completeness of the choreography: each \ac{NFA} transition is simulated by a corresponding rewriting in the choreography.
The second clause captures the soundness of the choreography: each rewriting simulates some \ac{NFA} transition.



\subsection{An object-oriented choreography}\label{sec:formula-as-process:nfa-oo}

Once again, the functional choreography is not the only choreography possible for the \ac{NFA} specification.
As for the binary counter, there is an \enquote*{object-oriented} choreography that treats the states as messages that effect a response from processes that represent symbols of an input word.
In this way, we may use a role assignment that is roughly dual to the assignment used in the preceding functional choreography.

Specifically, let $\theta'$ be%
\marginnote{$
    \theta' =
      \!\begin{lgathered}[t]
        \Set{ a \mapsto \defp{a} \given a \in \ialph } \union \Set{ \eow \mapsto \defp{\eow} } \\
        {} \union \Set{ q \mapsto \atmL{q} \given q \in Q } \\
        {} \union \Set{ \symrej \mapsto \atmR{\symrej} }
      \end{lgathered}
$}
the role assignment that maps each input symbol $a \in \ialph$ and the end-of-word marker, $\eow$, to coinductively defined propositions, $\defp{a}$ and $\defp{\eow}$, respectively; each state $q \in Q$ to a left-directed message, $\atmL{q}$; and the rejection symbol, $\symrej$, to a right-directed message, $\atmR{\symrej}$.

Under the role assignment $\theta'$, the axioms in $\srsig$ that mention $a$ and $\eow$ in their premises induce the rewritings
\begin{equation*}
  {\textstyle \bigunion_{q \in Q} \Set{ \ereduces[\orsig']{\defp{a} \oc \atmL{q}}{\atmL{q}'_a} \given q'_a \in \nfapow(q, a) } }
  \quad\text{and}\quad
  {\textstyle \bigunion_{q \in Q} \Set{ \ereduces[\orsig']{\defp{\eow} \oc \atmL{q}}{\atmR{F}(q)} } }
  \,,
\end{equation*}
respectively,
as constraints on $\orsig'$ that must be satisfied if $(\theta', \orsig')$ is to be a meaningful choreography of the \ac{NFA} specification.
Solving these constraints for $\defp{a}$ and $\defp{\eow}$, respectively, we obtain the definitions
\begin{equation*}
  \defp{a} \defd \bigwith_{q \in Q} \bigl({\textstyle \bigwith_{q'_a \in \nfapow(q, a)} (\atmL{q}'_a \pmir \atmL{q})}\bigr)
  \quad\text{and}\quad
  \defp{\eow} \defd \bigwith_{q \in Q} \bigl(\up \bigfuse \atmR{F}(q) \pmir \atmL{q}\bigr)
  \,.
\end{equation*}
In full, the choreographing signature $\orsig'$ is therefore
\begin{equation*}
  \orsig' =
  \bigl(\defp{\eow} \defd {\textstyle \bigwith_{q \in Q} (\up \bigfuse \atmR{F}(q) \pmir \atmL{q})}\bigr)
  \,,
  \biggl(
    \defp{a} \defd \bigwith_{q \in Q} \bigl(\textstyle \bigwith_{q'_a \in \nfapow(q, a)} (\atmL{q}'_a \pmir \atmL{q})
  \bigr)
  \biggr)_{a \in \ialph}
\end{equation*}
\noindent
Indeed, this is the same choreographing signature that is produced by the formal procedure:
\begin{proposition}
  For the string rewriting specification $(\ialph \dunion \Set{\eow, \symrej}, \srsig)$ and role assignment $\theta'$, the judgment $\chorsig{\theta'}{\srsig}{\orsig'}$ holds.
\end{proposition}

As for the functional choreography, we may then establish a shortcut adequacy theorem for this object-oriented choreography as a \lcnamecref{??} of earlier results.
Composing this \lcnamecref{??} with \cref{??}, the adequacy of formula-as-process choreographies with respect to their underlying string rewriting specifications, we arrive at:
\begin{corollary}\label{cor:formula-as-process:nfa-oochor-adequacy}
  Let $\aut{A} = (Q, \nfapow, F)$ be \iac{NFA} over the input alphabet $\ialph$, with choreography $(\theta', \orsig')$ as described above.
  The following hold.
  \begin{itemize}[nosep]
  \item
    $q \nfareduces[a] q'_a$ only if $\defp{a} \oc \atmL{q} \reduces_{\orsig'} \atmL{q}'_a$.
    Also, if $\defp{a} \oc \atmL{q} \reduces_{\orsig'} \octx'$, then $q \nfareduces[a] q'_a$ for some state $q'_a$ such that $\octx' = \atmL{q}'_a$.
  \item
    $q \in F$ if, and only if, $\defp{\eow} \oc \atmL{q} \reduces_{\orsig'} (\octxe)$.
  \item
    $q \nfareduces[w] q'$ only if $\rev{\defp{w}} \oc \atmL{q} \Reduces_{\orsig'} \atmL{q}'$.
    Also, if $\rev{\defp{w}} \oc \atmL{q} \Reduces_{\orsig'} \octx'$, then $q \nfareduces[w] q'$ for some state $q'$ such that $\octx' = \atmL{q}'$.
  \end{itemize}
\end{corollary}
\noindent
Once again, this gives the adequacy of the object-oriented choreography of \acp{NFA} nearly for free.

However, there is one slight blemish to the soundness clauses.
Its premise, \enquote{If $\defp{a} \oc \atmL{q} \reduces_{\orsig'} \octx'$ \textelp{}} deals only with ideas from the choreography and ordered rewriting.
Its conclusion, however, entangles ideas from the mathematical model of the \ac{NFA} (\enquote{then $q \nfareduces[a] q'_a$ for some state $q'_a$ \textelp{}}) with ideas from the choreography (\enquote{\textelp{} such that $\octx' = \atmL{q}'_a$.}).
In a way, the soundness clause violates a kind of meta-level stratification.
Soundness should relate one model\fixnote{word choice?}, on the one hand, to another model, on the other hand, but that is not what happens here.

Fortunately, it is not difficult to rephrase the soundness clause in a way that adheres to the desired meta-level stratification.
Upon examining the definition of $\defp{a}$, a rewriting $\defp{a} \oc \atmL{q} \reduces_{\orsig'} \octx'$ exists if and, most importantly, only if $\octx' = \atmL{q}'$ for some state $q'$.
The statement of soundness is thus equivalent to:
\begin{itemize}
\item If $\defp{a} \oc \atmL{q} \reduces_{\orsig'} \atmL{q}'$, then $q \nfareduces[a] q'_a$ for some state $q'_a$ such that $\atmL{q}' = \atmL{q}'_a$.
\end{itemize}
But because there is a unique atom for each state, equality of state atoms coincides with equality of the states themselves.
Therefore, we can simplify the statement further and combine it with completeness:
\begin{corollary}[Adequacy of the object-oriented \acs*{NFA} choreography]\label{cor:formula-as-process:nfa-oochor-adequacy-simple}
  Let $\aut{A} = (Q, \nfapow, F)$ be \iac{NFA} over the input alphabet $\ialph$, with choreography $(\theta', \orsig')$ as described above.
  The following hold.
  \begin{itemize}[nosep]
  \item $q \nfareduces[a] q'$ if, and only if, $\defp{a} \oc \atmL{q} \reduces_{\orsig'} \atmL{q}'$.
  \item $q \in F$ if, and only if, $\defp{\eow} \oc \atmL{q} \reduces_{\orsig'} (\octxe)$.
  \item $q \nfareduces[w] q'$ if, and only if, $\rev{\defp{w}} \oc \atmL{q} \Reduces_{\orsig'} \atmL{q}'$.
  \end{itemize}
\end{corollary}

\subsection{Incorporating \acs*{NFA} bisimilarity}\label{sec:formula-as-process:nfa-bisim}

Recall from \cref{??} the nearly free adequacy result for the functional choreography of \acp{NFA}.
%
\cornfafnchoradequacy*
%
The soundness clause has the same blemish as the object-oriented choreography's soundness clause had: its conclusion violates a kind of meta-level stratification, mixing ideas from the mathematical model (\enquote{then $q \nfareduces[a] q'_a$ for some state $q'_a$ \textelp{}}) with ideas from the choreography (\enquote{\textelp{} such that $\octx' = \defp{q}'_a$.}).
It would be much nicer if we could rephrase soundness in a stratified way.

As for the object-oriented choreography, we can begin by noticing that a rewriting $\atmR{a} \oc \defp{q} \reduces_{\orsig} \octx'$ exists if, and only if, $\octx' = \defp{q}'$ for some state $q'$.
The statement of soundness is thus equivalent to:
\begin{itemize}
\item
  If $\atmR{a} \oc \defp{q} \reduces_{\orsig} \defp{q}'$, then $q \nfareduces[a] q'_a$ for some state $q'_a$ such that $\defp{q}' = \defp{q}'_a$.
\end{itemize}
This is still not quite satisfactory because the conclusion brings in choreographic ideas, namely $\defp{q}' = \defp{q}'_a$.
Is it possible to characterize this relationship between $q'$ and $q'_a$ natively in terms of the \ac{NFA}'s mathematical model?

% The second clause captures soundness, but is not phrased exactly as we might have hoped.
% Its premise is stated in terms of the choreography alone, but its conclusion mixes ideas from the \ac{NFA} model (namely $q \nfareduces[a] q'_a$) with ideas from the choreography (namely $\octx' = \defp{q}'_a$).
% This mixing of ideas is philosophically objectionable in the way it violates a kind of meta-level typing property -- 
% it would be much nicer if the conclusion used only ideas native to \acp{NFA}, for then soundness would cleanly relate the choreography, on the one hand, to the \ac{NFA} model, on the other hand.

% Upon examining the definition of $\defp{q}$, we notice that a rewriting $\atmR{a} \oc \defp{q} \reduces_{\orsig} \octx'$ exists if and, most importantly, only if $\octx' = \defp{q}'$ for some state $q'$.
% This allows us to revise the statement of soundness to the equivalent:
% \begin{itemize}
% \item
%   If $\atmR{a} \oc \defp{q} \reduces_{\orsig} \defp{q}'$, then $q \nfareduces[a] q'_a$ for some state $q'_a$ such that $\defp{q}' = \defp{q}'_a$.
% \end{itemize}
% However, this statement is still not quite satisfactory in that its conclusion, with $\defp{q}' = \defp{q}'_a$, still mixes in ideas from the choreography.
% Is it possible to characterize this relationship between $q'$ and $q'_a$ natively on the \ac{NFA}?

Unfortunately, this is not nearly as easy as it was for the object-oriented choreography.
Unlike there, equality of state encodings does not coincide with equality of the states themselves, \ie, $\defp{q}' = \defp{q}'_a$ does \emph{not} imply $q' = q'_a$.
The equirecursive treatment of coinductively defined propositions leads to a quite generous notion of equality on propositions, which in turn makes equality of state encodings a coarser equivalence than equality of the states themselves.
% Because the coinductively defined propositions are interpreted equirecursively, equality of encodings is too generous to imply equality of states.
%
\begin{marginfigure}
    \centering
    % \subfloat[][]{\label{fig:ordered-rewriting:dfa-counterexample:dfa}%
      \begin{equation*}
        \aut{A}'_2 = 
        \begin{tikzpicture}[baseline=(q_0.base)]
          \graph [automaton] {
            q_0
             -> ["a,b", loop above]
            q_0
             -> ["b"]
            q_1 [accepting]
             -> ["a,b"]
            q_2
             -> ["a,b", loop above]
            q_2;
            %
            s_1 [accepting, below = of q_1.south]
             -> ["a,b" sloped]
            q_2;
          };
        \end{tikzpicture}
      \end{equation*}

  \begin{equation*}
    \orsig =
    \begin{lgathered}[t]
      \bigl(\defp{q}_0 \defd (\atmR{a} \limp \up \dn \defp{q}_0) \with (\atmR{b} \limp (\up \dn \defp{q}_0 \with \up \dn \defp{q}_1)) \with (\atmR{\eow} \limp \up \atmR{\symrej})\bigr) \,, \\
      \bigl(\defp{q}_1 \defd (\atmR{a} \limp \up \dn \defp{q}_2) \with (\atmR{b} \limp \up \dn \defp{q}_2) \with (\atmR{\eow} \limp \up \one)\bigr) \,, \\
      \bigl(\defp{q}_2 \defd (\atmR{a} \limp \up \dn \defp{q}_2) \with (\atmR{b} \limp \up \dn \defp{q}_2) \with (\atmR{\eow} \limp \up \atmR{\symrej})\bigr) \,, \\
      \bigl(\defp{s}_1 \defd (\atmR{a} \limp \up \dn \defp{q}_2) \with (\atmR{b} \limp \up \dn \defp{q}_2) \with (\atmR{\eow} \limp \up \one)\bigr)
    \end{lgathered}
  \end{equation*}
    \caption{A slightly modified version of the \ac*{NFA} from \cref{fig:formula-as-process:nfa-example}; and a choreography}\label{fig:formula-as-process:nfa-counterexample}
  \end{marginfigure}%
% \begin{proof}[Counterexample]
  As a concrete counterexample, consider the \ac{NFA} and encoding shown in the adjacent \lcnamecref{fig:formula-as-process:nfa-counterexample}; it is the same \ac{NFA} as shown in \cref{fig:formula-as-process:nfa-example}, but with one added state, $s_1$, that is unreachable from the others.
    %
  % When encoded as an ordered rewriting specification, it corresponds to the following definitions:
  % \begin{equation*}
  %   \begin{lgathered}
  %     \dfa{q}_0 \defd (a \limp \dfa{q}_0) \with (b \limp \dfa{q}_1) \with (\emp \limp \top) \\
  %     \dfa{q}_1 \defd (a \limp \dfa{q}_0) \with (b \limp \dfa{q}_1) \with (\emp \limp \one) \\
  %     \dfa{s}_1 \defd (a \limp \dfa{q}_0) \with (b \limp \dfa{s}_1) \with (\emp \limp \one)
  %   \end{lgathered}
  % \end{equation*}
  In this counterexample, as a coinductive consequence of the equirecursive treatment of definitions, $\defp{q}_1 = \defp{s}_1$ but $q_1 \neq s_1$.
% \end{proof}

% As this counterexample shows, the failure of this claim stems from the fact that the choreography of states is not injective -- here, $q_1 \neq s_1$ even though $\defp{q}_1 = \defp{s}_1$.
% In other words, equality of state encodings is a coarser equivalence than equality of the states themselves.


One possible remedy for this lack of injectivity might be to revise the encoding to have a stronger nominal character.
By tagging each state's encoding with an atom that is unique to that state, we can make the encoding manifestly injective.
For instance, given the pairwise distinct atoms $\Set{\atmR{q} \given q \in F}$ and $\Set{\atmR{\bar{q}} \given q \in Q - F}$ to tag final and non-final states, respectively, we could define an alternative encoding, $\check{q}$:
%
\begin{equation*}
  \check{q} \defd
    (\atmR{\eow} \limp \up \atmR{\check{F}}(q))
    \with
    \bigwith_{a \in \ialph}\bigl(\atmR{a} \limp {\textstyle \bigwith_{q'_a \in \nfapow(q,a)} \up \dn \check{q}'_a}\bigr)
  %
  \enspace\text{where }
  %
  \atmR{\check{F}}(q) =
    \begin{cases*}
      \atmR{q} & if $q \in F$ \\
      \atmR{\bar{q}} & if $q \notin F$%
    \,.
    \end{cases*}
\end{equation*}
%
Under this alternative encoding, the states $q_1$ and $s_1$ of \cref{fig:ordered-rewriting:dfa-counterexample} are no longer a counterexample to injectivity:
Because $q_1$ and $s_1$ are distinct states, they correspond to distinct tags, and so $\check{q}_1 \neq \check{s}_1$.

% One possible remedy
% % for this apparent lack of adequacy
% might be to revise the encoding to have a stronger nominal character % .
% by tagging each state's encoding with an atom that is unique to that state.
% For instance, given the pairwise distinct atoms $\set{q \given q \in F}$ and $\set{\bar{q} \given q \in Q - F}$ to tag final and non-final states, respectively, we could define an alternative encoding, $\check{q}$, that is manifestly injective:
% %
% % \begin{marginfigure}
% \begin{gather*}
%   \check{q} \defd
%     \parens[size=big]{
%       \bigwith_{a \in \ialph}(a \limp \check{q}'_a)}
%     \with
%     \parens[size=big]{\emp \limp \check{F}(q)}
%   %
%   \shortintertext{where}
%   %
%   q \dfareduces[a] q'_a
%   \text{, for all input symbols $a \in \ialph$,\quad and\quad}
%   \check{F}(q) =
%     \begin{cases*}
%       q & if $q \in F$ \\
%       \bar{q} & if $q \notin F$%
%     \,.
%     \end{cases*}
% \end{gather*}
% % \end{marginfigure}%
% % , the encoding can be made to be injective.
% % With this change, the alternative encoding is now injective: $\check{q} = \check{s}$ implies $q = s$.

Although such a solution is certainly possible, it seems unsatisfyingly ad~hoc.
A closer examination of the preceding counterexample reveals that the states $q_1$ and $s_1$, while not equal, are in fact bisimilar~\parencref{??}.
In other words, although the choreographing of states is not, strictly speaking, injective, it is injective \emph{up to bisimilarity}: $\defp{q} = \defp{s}$ implies $q \asim s$.
% This suggests a more elegant solution to the apparent lack of adequacy: the adequacy should be judged up to \ac{NFA} bisimilarity.

% A closer examination of the preceding counterexample reveals that the states $q_1$ and $s_1$, while not equal, are in fact bisimilar~\parencref{??}.
% In other words, although the encoding is not, strictly speaking, injective, it is injective \emph{up to bisimilarity}: $\dfa{q} = \dfa{s}$ implies $q \asim s$.
% This suggests a more elegant solution to the apparent lack of adequacy: the encoding's adequacy should be judged up to \ac{DFA} bisimilarity.

% This \lcnamecref{cor:formula-as-process:nfa-fnchor-adequacy} directly relates the automaton to its choreography, with one exception that is arguably unsatisfactory: an ordered context $\octx'$ is dragged into the second statement.
% Thus, we might like to rephrase this result so that the context $\octx'$ and condition $\octx' = \defp{q}'_a$ are replaced with conditions native to the \ac{NFA} itself.

% The first step is to notice that the definition of $\defp{q}$ is such that there is a rewriting $\atmR{a} \oc \defp{q} \reduces_{\orsig} \octx'$ if, and only if, $\octx' = \defp{s}'$ for some state $s'$.
% \begin{itemize}
% \item
%   If $\atmR{a} \oc \defp{q} \reduces_{\orsig} \defp{s}'$, then $q \nfareduces[a] q'_a$ for some state $q'_a$ such that $\defp{s}' = \defp{q}'_a$.
% \end{itemize}

% Can the relationship $\defp{q}' = \defp{q}'_a$ be stated natively on $q'$ and $a'_a$?
% Notice that $\defp{q} = \defp{s}$ implies that $q$ and $s$ are bisimilar states.
%
\begin{theorem}
  Let $\aut{A} = (Q, \nfapow, F)$ be \iac{NFA} over the input alphabet $\ialph$.
  For all states $q$ and $s$, if $\defp{q} = \defp{s}$, then $q \asim s$.
\end{theorem}
\begin{proof}
  We will show that the relation $\mathord{\simu{R}} = \Set{(q,s) \given \defp{q} = \defp{s}}$ is a bisimulation and is therefore included in \ac{NFA} bisimilarity.
  \begin{description}[parsep=0pt, listparindent=\parindent]
  \item[Input bisimilarity]
    We must show that, for all input symbols $a \in \ialph$, all $\simu{R}$-related states have $\simu{R}$-related $a$-successors. 

    Let $q$ and $s$ be $\simu{R}$-related states, which, being $\simu{R}$-related, have equal encodings, \ie, $\defp{q} = \defp{s}$.
    Because definitions are treated equirecursively, their unrollings are also equal.
    For each state $q'_a$ that $a$-succeeds $q$, there must therefore exist a state $s'_a$ such that $\defp{q}'_a = \defp{s}'_a$.
    In other words, each $a$-successor of state $q$ is $\simu{R}$-related to an $a$-successor of state $s$.

  \item[Finality]
    We must show that all $\simu{R}$-related states have matching finalities, \ie, that $q \simu{R} s$ implies $q \in F$ if, and only if, $s \in F$.

    Let $q$ and $s$ be $\simu{R}$-related states, with $q$ a final state.
    Being $\simu{R}$-related, the states $q$ and $s$ have equal encodings, \ie, $\defp{q} = \defp{s}$.
    Because definitions are treated equirecursively, their unrollings are also equal.
    It follows that $\atmR{F}(q) = \atmR{F}(s)$, and so $s$ is also a final state.
  %
  \qedhere
  \end{description}
\end{proof}


Unfortunately, the converse is not true: bisimilar \ac{NFA} states do not, in general, have equal encodings.
% \begin{falseclaim}
%   Let $\aut{A} = (Q, \nfapow, F)$ be \iac{NFA} over an input alphabet $\ialph$.
%   For all states $q,s \in Q$, if $q \asim s$, then $\defp{q} = \defp{s}$.
% \end{falseclaim}
% %
% \begin{proof}[Counterexample]
  As a concrete counterexample, consider the \ac{NFA} and choreography depicted in the adjacent \lcnamecref{fig:formula-as-process:nfa-bisim-falseclaim}.%
  \begin{marginfigure}
    \begin{equation*}
      \begin{tikzpicture}[baseline=(q_0.base)]
        \graph [automaton] {
          q_0 [accepting]
           -> ["a", loop above]
          q_0
           -> ["a"]
          q_1 [accepting]
           -> ["a", loop above]
          q_1;
        };
      \end{tikzpicture}
    \end{equation*}
    \begin{equation*}
      \begin{lgathered}[t]
        \defp{q}_0 \defd (\atmR{a} \limp \up \dn \defp{q}_0 \with \up \dn \defp{q}_1) \with (\atmR{\eow} \limp \up \one) \\
        \defp{q}_1 \defd (\atmR{a} \limp \up \dn \defp{q}_1) \with (\atmR{\eow} \limp \up \one)
      \end{lgathered}
    \end{equation*}
    \caption{\Iacs*{NFA} that accepts all finite words over the alphabet $\ialph = \Set{a}$}\label{fig:formula-as-process:nfa-bisim-falseclaim}
  \end{marginfigure}
  It is straightforward to verify that $q_0$ and $q_1$ are bisimilar states.
  But their encodings are not equal.
  Unrolling definitions, we have
  \begin{gather*}
    \defp{q}_0 = (\atmR{a} \limp \up \dn \defp{q}_0 \with \up \dn \defp{q}_1) \with (\atmR{\eow} \limp \up \one)
    \neq
    (\atmR{a} \limp \up \dn \defp{q}_1) \with (\atmR{\eow} \limp \up \one) = \defp{q}_1
    \,.
  \end{gather*}
  These propositions are not equal because the former has a first clause with shape $(\atmR{a} \limp \up \dn \mathord{-} \with \up \dn \mathord{-})$, whereas the latter's first clause has shape $(\atmR{a} \limp \up \dn \mathord{-})$.
  In other words, the encodings of states $q_0$ and $q_1$ are not equal precisely because the two states have different numbers of $a$-successors.
  
  In some sense, the generous equality induced by the equirecursive treatment of definitions has outpaced the remaining aspects of propositional equality.
  Because equality of state encodings does not coincide with bisimilarity of \ac{NFA} states, we cannot easily complete our program of simplifying the statement of \ac{NFA} soundness.

However, all is not lost.
First, although for \acp{NFA} bisimilar states do not always have equal encodings, for \emph{deterministic} finite automata bisimilar states do indeed have equal encodings.
\begin{theorem}
  Let $\aut{A} = (Q, \dfanext, F)$ be a \emph{\ac{DFA}} over an input alphabet $\ialph$.
  For all states $q$ and $s$, if $q \asim s$, then $\defp{q} = \defp{s}$.
\end{theorem}
%
\begin{proof}
  Because $\aut{A}$ is deterministic, the states $q$ and $s$ have unique $a$-successors for each input symbol $a \in \ialph$.
  Because $q$ and $s$ are bisimilar, so are their $a$-successors.
  By the coinductive hypothesis, the unique $a$-successors of $q$ and $s$ have equal encodings: $\defp{q}'_a = \defp{s}'_a$
\end{proof}
\noindent
Thus, for \acp{DFA} only, we have $\atmR{a} \oc \defp{q} \reduces_{\orsig} \defp{q}'$ if, and only if, $q \nfareduces[a]\asim q'$, which is properly stratified.

Second, in the following \lcnamecref{ch:??}, we will develop a richer notion of propositional equivalence, \emph{ordered rewriting bisimilarity}.
In applying it to the functional choreography of \acp{NFA}~\parencref{sec:??}, we will see that \ac{NFA} states that are \ac{NFA}-bisimilar have state encodings that, while not necessarily equal, are always rewriting-bisimilar; the converse will also hold.
This will allow us to rephrase soundness (and completeness) of the \ac{NFA} choreography to a form that is properly stratified: $\atmR{a} \oc \defp{q} \reduces_{\orsig}\osim \defp{q}'$ if, and only if, $q \nfareduces[a]\asim q'$.\fixnote{Check}




% % The \lcnamecref{cor:formula-as-process:nfa-fnchor-adequacy} is not phrased exactly how we might have hoped.
% % The first clause, completeness of the choreography with respect to the \ac{NFA}, cleanly relates each \ac{NFA} transition to a rewriting
% % It would be much nicer if the second clause, soundness of the choreography with respect to the \ac{NFA}, could be stated in terms of the \ac{NFA} and its choreography alone.

% % Greedily, we might have hoped for slightly more.
% % % This \lcnamecref{cor:formula-as-process:nfa-fnchor-adequacy} is [...], but perhaps unsatisfactory in one particular detail.
% % The second clause, soundness of the choreography with respect to the \ac{NFA}, drags in an ordered context $\octx'$.
% % It would be much nicer if the choreography's soundness could be stated in terms of the \ac{NFA} and its choreography alone.

% % The first step toward such a rephrasing is to notice that $\atmR{a} \oc \defp{q} \reduces_{\orsig} \octx'$ \emph{only if} $\octx' = \defp{q}'$ for some state $q'$.
% % Thus, the above statement of soundness is equivalent to:
% % \begin{itemize}
% % \item
% %   If $\atmR{a} \oc \defp{q} \reduces_{\orsig} \defp{q}'$, then $q \nfareduces[a] q'_a$ for some state $q'_a$ such that $\defp{q}' = \defp{q}'_a$.
% % \end{itemize}
% % Better


% % This \lcnamecref{cor:formula-as-process:nfa-fnchor-adequacy} is [...], but perhaps unsatisfactory in one particular detail.
% % The second clause, soundness of the choreography with respect to the \ac{NFA}, drags in an ordered context $\octx'$.
% % It would be much nicer if the choreography's soundness were a direct converse of its completeness, as in:
% % \begin{itemize}
% % \item If $\atmR{a} \oc \defp{q} \reduces_{\orsig} \defp{q}'_a$, then $q \nfareduces[a] q'_a$.
% % \end{itemize}
% % Unfortunately, this claim is, in fact, false.
% % %

% We might naively hope that the equality of state encodings would coincide with equality of the states themselves, \ie, that $\defp{q}' = \defp{q}'_a$ if, and only if, $q' = q'_a$.
% That way, soundness would be simply \enquote{If $\atmR{a} \oc \defp{q} \reduces_{\orsig} \defp{q}'$, then $q \nfareduces[a] q'$,} which would satisfy the kind of meta-level typing that we desire.

% Unfortunately, the two equalities do not coincide because the encoding of \ac{NFA} states is simply not injective.
% The equirecursive treatment of coinductively defined propositions leads to a quite generous notion of equality on propositions, which in turn makes equality of state encodings a coarser equivalence than equality of the states themselves.
% % Because the coinductively defined propositions are interpreted equirecursively, equality of encodings is too generous to imply equality of states.
% %
% \begin{marginfigure}
%     \centering
%     % \subfloat[][]{\label{fig:ordered-rewriting:dfa-counterexample:dfa}%
%       \begin{equation*}
%         \aut{A}'_2 = 
%         \begin{tikzpicture}[baseline=(q_0.base)]
%           \graph [automaton] {
%             q_0
%              -> ["a,b", loop above]
%             q_0
%              -> ["b"]
%             q_1 [accepting]
%              -> ["a,b"]
%             q_2
%              -> ["a,b", loop above]
%             q_2;
%             %
%             s_1 [accepting, below = of q_1.south]
%              -> ["a,b" sloped]
%             q_2;
%           };
%         \end{tikzpicture}
%       \end{equation*}

%   \begin{equation*}
%     \orsig =
%     \begin{lgathered}[t]
%       \bigl(\defp{q}_0 \defd (\atmR{a} \limp \up \dn \defp{q}_0) \with (\atmR{b} \limp (\up \dn \defp{q}_0 \with \up \dn \defp{q}_1)) \with (\atmR{\eow} \limp \up \atmR{\symrej})\bigr) \,, \\
%       \bigl(\defp{q}_1 \defd (\atmR{a} \limp \up \dn \defp{q}_2) \with (\atmR{b} \limp \up \dn \defp{q}_2) \with (\atmR{\eow} \limp \up \one)\bigr) \,, \\
%       \bigl(\defp{q}_2 \defd (\atmR{a} \limp \up \dn \defp{q}_2) \with (\atmR{b} \limp \up \dn \defp{q}_2) \with (\atmR{\eow} \limp \up \atmR{\symrej})\bigr) \,, \\
%       \bigl(\defp{s}_1 \defd (\atmR{a} \limp \up \dn \defp{q}_2) \with (\atmR{b} \limp \up \dn \defp{q}_2) \with (\atmR{\eow} \limp \up \one)\bigr)
%     \end{lgathered}
%   \end{equation*}
%     \caption{A slightly modified version of the \ac*{NFA} from \cref{fig:formula-as-process:nfa-example}; and a choreography}\label{fig:formula-as-process:nfa-counterexample}
%   \end{marginfigure}%
% % \begin{proof}[Counterexample]
%   As a concrete counterexample to injectivity, consider the \ac{NFA} and encoding shown in the adjacent \lcnamecref{fig:formula-as-process:nfa-counterexample}; it is the same \ac{NFA} as shown in \cref{fig:formula-as-process:nfa-example}, but with one added state, $s_1$, that is unreachable from the other states.
%     %
%   % When encoded as an ordered rewriting specification, it corresponds to the following definitions:
%   % \begin{equation*}
%   %   \begin{lgathered}
%   %     \dfa{q}_0 \defd (a \limp \dfa{q}_0) \with (b \limp \dfa{q}_1) \with (\emp \limp \top) \\
%   %     \dfa{q}_1 \defd (a \limp \dfa{q}_0) \with (b \limp \dfa{q}_1) \with (\emp \limp \one) \\
%   %     \dfa{s}_1 \defd (a \limp \dfa{q}_0) \with (b \limp \dfa{s}_1) \with (\emp \limp \one)
%   %   \end{lgathered}
%   % \end{equation*}
%   Here, as a coinductive consequence of the equirecursive treatment of definitions, $\defp{q}_1 = \defp{s}_1$ but $q_1 \neq s_1$.
% % \end{proof}

% % As this counterexample shows, the failure of this claim stems from the fact that the choreography of states is not injective -- here, $q_1 \neq s_1$ even though $\defp{q}_1 = \defp{s}_1$.
% % In other words, equality of state encodings is a coarser equivalence than equality of the states themselves.


% One possible remedy for this lack of injectivity might be to revise the encoding to have a stronger nominal character.
% By tagging each state's encoding with an atom that is unique to that state, we can make the encoding manifestly injective.
% For instance, given the pairwise distinct atoms $\Set{\atmR{q} \given q \in F}$ and $\Set{\atmR{\bar{q}} \given q \in Q - F}$ to tag final and non-final states, respectively, we could define an alternative encoding, $\check{q}$:
% %
% \begin{equation*}
%   \check{q} \defd
%     (\atmR{\eow} \limp \up \bigfuse \atmR{\check{F}}(q))
%     \with
%     \bigwith_{a \in \ialph}\bigl(\atmR{a} \limp {\textstyle \bigwith_{q'_a \in \nfapow(q,a)} \up \dn \check{q}'_a}\bigr)
%   %
%   \enspace\text{where }
%   %
%   \atmR{\check{F}}(q) =
%     \begin{cases*}
%       \atmR{q} & if $q \in F$ \\
%       \atmR{\bar{q}} & if $q \notin F$%
%     \,.
%     \end{cases*}
% \end{equation*}
% %
% Under this alternative encoding, the states $q_1$ and $s_1$ of \cref{fig:ordered-rewriting:dfa-counterexample} are no longer a counterexample to injectivity:
% Because $q_1$ and $s_1$ are distinct states, they correspond to distinct tags, and so $\check{q}_1 \neq \check{s}_1$.

% % One possible remedy
% % % for this apparent lack of adequacy
% % might be to revise the encoding to have a stronger nominal character % .
% % by tagging each state's encoding with an atom that is unique to that state.
% % For instance, given the pairwise distinct atoms $\set{q \given q \in F}$ and $\set{\bar{q} \given q \in Q - F}$ to tag final and non-final states, respectively, we could define an alternative encoding, $\check{q}$, that is manifestly injective:
% % %
% % % \begin{marginfigure}
% % \begin{gather*}
% %   \check{q} \defd
% %     \parens[size=big]{
% %       \bigwith_{a \in \ialph}(a \limp \check{q}'_a)}
% %     \with
% %     \parens[size=big]{\emp \limp \check{F}(q)}
% %   %
% %   \shortintertext{where}
% %   %
% %   q \dfareduces[a] q'_a
% %   \text{, for all input symbols $a \in \ialph$,\quad and\quad}
% %   \check{F}(q) =
% %     \begin{cases*}
% %       q & if $q \in F$ \\
% %       \bar{q} & if $q \notin F$%
% %     \,.
% %     \end{cases*}
% % \end{gather*}
% % % \end{marginfigure}%
% % % , the encoding can be made to be injective.
% % % With this change, the alternative encoding is now injective: $\check{q} = \check{s}$ implies $q = s$.

% Although such a solution is certainly possible, it seems unsatisfyingly ad~hoc.

% A closer examination of the preceding counterexample reveals that the states $q_1$ and $s_1$, while not equal, are in fact bisimilar~\parencref{??}.
% In other words, although the choreographing of states is not, strictly speaking, injective, it is injective \emph{up to bisimilarity}: $\defp{q} = \defp{s}$ implies $q \asim s$.
% This suggests a more elegant solution to the apparent lack of adequacy: the adequacy should be judged up to \ac{NFA} bisimilarity.

% % A closer examination of the preceding counterexample reveals that the states $q_1$ and $s_1$, while not equal, are in fact bisimilar~\parencref{??}.
% % In other words, although the encoding is not, strictly speaking, injective, it is injective \emph{up to bisimilarity}: $\dfa{q} = \dfa{s}$ implies $q \asim s$.
% % This suggests a more elegant solution to the apparent lack of adequacy: the encoding's adequacy should be judged up to \ac{DFA} bisimilarity.

% % This \lcnamecref{cor:formula-as-process:nfa-fnchor-adequacy} directly relates the automaton to its choreography, with one exception that is arguably unsatisfactory: an ordered context $\octx'$ is dragged into the second statement.
% % Thus, we might like to rephrase this result so that the context $\octx'$ and condition $\octx' = \defp{q}'_a$ are replaced with conditions native to the \ac{NFA} itself.

% % The first step is to notice that the definition of $\defp{q}$ is such that there is a rewriting $\atmR{a} \oc \defp{q} \reduces_{\orsig} \octx'$ if, and only if, $\octx' = \defp{s}'$ for some state $s'$.
% % \begin{itemize}
% % \item
% %   If $\atmR{a} \oc \defp{q} \reduces_{\orsig} \defp{s}'$, then $q \nfareduces[a] q'_a$ for some state $q'_a$ such that $\defp{s}' = \defp{q}'_a$.
% % \end{itemize}

% % Can the relationship $\defp{q}' = \defp{q}'_a$ be stated natively on $q'$ and $a'_a$?
% % Notice that $\defp{q} = \defp{s}$ implies that $q$ and $s$ are bisimilar states.
% %
% \begin{theorem}
%   Let $\aut{A} = (Q, \nfapow, F)$ be \iac{NFA} over the input alphabet $\ialph$.
%   For all states $q,s \in Q$, if $\defp{q} = \defp{s}$, then $q \asim s$.
% \end{theorem}
% \begin{proof}
%   We will show that the relation $\mathord{\simu{R}} = \Set{(q,s) \given \defp{q} = \defp{s}}$ is a bisimulation and is therefore included in \ac{NFA} bisimilarity.
%   \begin{description}[parsep=0pt, listparindent=\parindent]
%   \item[Input bisimilarity]
%     We must show that, for all input symbols $a \in \ialph$, all $\simu{R}$-related states have $\simu{R}$-related $a$-successors. 

%     Let $q$ and $s$ be $\simu{R}$-related states, which, being $\simu{R}$-related, have equal encodings, \ie, $\defp{q} = \defp{s}$.
%     Because definitions are treated equirecursively, their unrollings are also equal.
%     For each state $q'_a$ that $a$-succeeds $q$, there must therefore exist a state $s'_a$ such that $\defp{q}'_a = \defp{s}'_a$.
%     In other words, each $a$-successor of state $q$ is $\simu{R}$-related to an $a$-successor of state $s$.

%   \item[Finality]
%     We must show that all $\simu{R}$-related states have matching finalities, \ie, that $q \simu{R} s$ implies $q \in F$ if, and only if, $s \in F$.

%     Let $q$ and $s$ be $\simu{R}$-related states, with $q$ a final state.
%     Being $\simu{R}$-related, the states $q$ and $s$ have equal encodings, \ie, $\defp{q} = \defp{s}$.
%     Because definitions are treated equirecursively, their unrollings are also equal.
%     It follows that $\atmR{F}(q) = \atmR{F}(s)$, and so $s$ is also a final state.
%   %
%   \qedhere
%   \end{description}
% \end{proof}


% Unfortunately, the converse is not true: bisimilar \ac{NFA} states do not, in general, have equal encodings.
% % \begin{falseclaim}
% %   Let $\aut{A} = (Q, \nfapow, F)$ be \iac{NFA} over an input alphabet $\ialph$.
% %   For all states $q,s \in Q$, if $q \asim s$, then $\defp{q} = \defp{s}$.
% % \end{falseclaim}
% % %
% % \begin{proof}[Counterexample]
%   As a concrete counterexample, consider the \ac{NFA} and choreography depicted in the adjacent \lcnamecref{fig:formula-as-process:nfa-bisim-falseclaim}.%
%   \begin{marginfigure}
%     \begin{equation*}
%       \begin{tikzpicture}[baseline=(q_0.base)]
%         \graph [automaton] {
%           q_0 [accepting]
%            -> ["a", loop above]
%           q_0
%            -> ["a"]
%           q_1 [accepting]
%            -> ["a", loop above]
%           q_1;
%         };
%       \end{tikzpicture}
%     \end{equation*}
%     \begin{equation*}
%       \begin{lgathered}[t]
%         \defp{q}_0 \defd (\atmR{a} \limp \up \dn \defp{q}_0 \with \up \dn \defp{q}_1) \with (\atmR{\eow} \limp \up \one) \\
%         \defp{q}_1 \defd (\atmR{a} \limp \up \dn \defp{q}_1) \with (\atmR{\eow} \limp \up \one)
%       \end{lgathered}
%     \end{equation*}
%     \caption{\Iacs*{NFA} that accepts all finite words over the alphabet $\ialph = \Set{a}$}\label{fig:formula-as-process:nfa-bisim-falseclaim}
%   \end{marginfigure}
%   It is straightforward to verify that $q_0$ and $q_1$ are bisimilar states.
%   But their encodings are not equal.
%   Unrolling definitions, we have
%   \begin{gather*}
%     \defp{q}_0 = (\atmR{a} \limp \up \dn \defp{q}_0 \with \up \dn \defp{q}_1) \with (\atmR{\eow} \limp \up \one)
%     \neq
%     (\atmR{a} \limp \up \dn \defp{q}_1) \with (\atmR{\eow} \limp \up \one) = \defp{q}_1
%     \,.
%   \end{gather*}
%   These propositions are not equal because the former has a first clause with shape $(\atmR{a} \limp \up \dn \mathord{-} \with \up \dn \mathord{-})$, whereas the latter's first clause has shape $(\atmR{a} \limp \up \dn \mathord{-})$.
%   In other words, the encodings of states $q_0$ and $q_1$ are not equal precisely because the two states have different numbers of $a$-successors.
  
%   In some sense, the generous equality induced by our equirecursive treatment of definitions has outpaced the remaining aspects of propositional equality.
%   Because equality of state encodings does not coincide with bisimilarity of \ac{NFA} states, we cannot complete our program of simplifying the statement of soundness.
%   However, as we will see in \cref{ch:??}, bisimilarity of states does coincide with a notion of bisimilarity for state encodings.

% \begin{theorem}
%   Let $\aut{A} = (Q, ?, F)$ be a \emph{\ac{DFA}} over an input alphabet $\ialph$.
%   For all states $q$ and $s$, if $q \asim s$, then $\defp{q} = \defp{s}$.
% \end{theorem}
% %
% \begin{proof}
%   Being \iac{DFA}, the states $q$ and $s$ have unique $a$-successors for each input symbol $a \in \ialph$.
%   Because $q$ and $s$ are bisimilar, so are their $a$-successors.
%   By the coinductive hypothesis, the unique $a$-successors of $q$ and $s$ have equal encodings: $\defp{q}'_a$
% \end{proof}








%%% Local Variables:
%%% mode: latex
%%% TeX-master: "thesis"
%%% End:

% \chapter{GARBAGE?}

\section{}

In \cref{ch:string-rewriting}, we saw that string rewriting can be used to specify the dynamics of concurrent systems, but that those specifications are quite abstract.
Even the operational semantics is left completely abstract: permitted rewritings just \emph{happen}, as if a central, meta-level actor schedules and otherwise coordinates rewriting.

In the previous \lcnamecref{ch:ordered-rewriting}, we presented a different rewriting framework, this one derived from the ordered sequent calculus and closely related to the \citeauthor{Lambek:AMM58} calculus\autocite{Lambek:AMM58}.
Ordered rewriting, too, leaves the [...] completely abstract

At this high level of abstraction, string rewriting specifications are not amenable to 

In this and the previous \lcnamecrefs{ch:ordered-rewriting,ch:formula-as-process}, we have also seen that the formula-as-process ordered rewriting framework permits only those rewritings that have a sensible interpretation under local, message-passing communication.
Thus far, we have seen that the formula-as-process ordered rewriting framework precludes rewritings, such as $\atmL{a} \oc $, that are not sensible in [...].


Given a string rewriting alphabet $\sralph$, a mapping $\theta\colon \finwds{\sralph} \to {?}$ is a \emph{role assignment for $\sralph$} if it is a monoid homomorphism between finite strings and ordered contexts that uniquely casts each symbol $a \in \sralph$ in the role of either: an atom, $\atmL{a}$ or $\atmR{a}$; or of a recursively defined proposition, $\defp{a}$.

A pair $(\theta, \orsig)$ is a \emph{choreography} of the string rewriting specification $(\sralph, \srsig)$ if:
\begin{itemize}
\item $\theta \colon \finwds{\sralph} \to ?$ is a role assignment for $\sralph$;
\item $\orsig$ is a formula-as-process signature that provides definitions for each of the recursively defined propositions that appear in the image of $\theta$; and
\item 
  $\theta$ is a (strong) bisimulation between $\reduces_{\srsig}$ and $\reduces_{\orsig}$, the string rewriting and (formula-as-process) rewriting relations:
\begin{equation*}
  \begin{tikzcd}
    w \rar[reduces, subscript=\srsig] \dar[relation][swap]{\theta}
     & w\mathrlap{'} \dar[relation, exists]{\theta}
    \\
    \octx \rar[reduces, exists, subscript=\orsig]
     & \octx\mathrlap{'}
  \end{tikzcd}
  \hphantom{'}
  \quad\text{and}\quad
  \begin{tikzcd}
    w \rar[reduces, exists, subscript=\srsig] \dar[relation][swap]{\theta}
     & w\mathrlap{'} \dar[relation, exists]{\theta}
    \\
    \octx \rar[reduces, subscript=\orsig]
     & \octx\mathrlap{' \,.}
  \end{tikzcd}
  \hphantom{' \,.}
\end{equation*}
\end{itemize}


Using the formula-as-process ordered rewriting as a substrate, we would like to choreograph string rewriting specifications $(\sralph, \srsig)$ by mapping them to formula-as-process ordered rewriting.
Specifically, we would like to find a binary relation $\simu{R}$ between strings and ordered contexts and an ordered rewriting signature $\orsig$ such that $\simu{R}$ is a (strong) bisimulation between $\reduces_{\srsig}$ and $\reduces_{\orsig}$, the string rewriting and (formula-as-process) ordered rewriting relations.
\begin{equation*}
  \begin{tikzcd}
    w \rar[reduces, subscript=\srsig] \dar[relation][swap]{\simu{R}}
     & w\mathrlap{'} \dar[relation, exists]{\simu{R}}
    \\
    \octx \rar[reduces, exists, subscript=\orsig]
     & \octx\mathrlap{'}
  \end{tikzcd}
  \hphantom{'}
  \quad\text{and}\quad
  \begin{tikzcd}
    w \rar[reduces, exists, subscript=\srsig] \dar[relation][swap]{\simu{R}}
     & w\mathrlap{'} \dar[relation, exists]{\simu{R}}
    \\
    \octx \rar[reduces, subscript=\orsig]
     & \octx\mathrlap{' \,.}
  \end{tikzcd}
  \hphantom{' \,.}
\end{equation*}
\footnote{A relation $\simu{R}$ such that:
  $\octx \simu{R}^{-1} w \reduces_{\srsig} w'$ implies $\octx \reduces_{\orsig} \octx' \simu{R}^{-1} w'$ for some $\octx'$; and $w \simu{R} \octx \reduces_{\orsig} \octx'$ implies $w \reduces_{\srsig} w' \simu{R} \octx'$ for some $w'$.}



Because the formula-as-process ordered rewriting framework precludes rewritings that [...], this choreographing operationalizes string rewriting.

Given a string rewriting specification $\srsig$, we would like to find an ordered rewriting signature $\orsig$ that mimics 

In other words, More specifically, the relation $\simu{R}$ will be a monoid homomorphism so that 


Not all string rewriting specifications have a valid choreography.
For instance, the string rewriting specification $(\sralph, \srsig)$ where
\begin{equation*}
  \begin{lgathered}
    \sralph = \Set{a, b} \\
    \srsig = (a \wc b \reduces b) \,, (b \reduces \emp) \,, (a \reduces \emp)
  \end{lgathered}
\end{equation*}
has no valid choreography.
Suppose that $\theta$ were a role assignment that led to a valid choreography, $(\theta, \orsig)$.
For the constraints $\theta(b) \reduces_{\orsig} (\octxe)$ and $\theta(a) \reduces_{\orsig} (\octxe)$ to be satisfiable, $\theta$ would have to map $b \mapsto \defp{b}$ and $a \mapsto \defp{a}$.
However, the first axiom would then induce the constraint $\defp{a} \oc \defp{b} \reduces_{\orsig} \defp{b}$, which is not satisfiable -- there are no definitions for \fixnote{Except, there are: $\defp{a} \defd \up \one$.}

Our goal is not to synthesize a choreography from scratch for a given string rewriting specification, $(\sralph, \srsig)$.
Instead, our goal is to synthesize a (formula-as-process) ordered rewriting signature from \emph{a role assignment $\theta$} for a given string rewriting specification.


Given a string rewriting specification $(\sralph, \srsig)$ and a role assignment $\theta\colon \finwds{\sralph} \to \atmL{\sralph} \cup \atmR{\sralph} \cup \defp{\sralph}$\fixnote{fix}, we would like to determine whether $\theta$ gives rise to a meaningful choreography of $(\sralph, \srsig)$.
That is, we would to construct, if possible, an ordered rewriting signature $\orsig$ that makes $\theta$ a (strong) bisimulation between the string rewriting and formula-as-process ordered rewriting relations, $\reduces_{\srsig}$ and $\reduces_{\orsig}$, respectively.
% For this we will define a judgment $\chorsig{\theta}{\srsig}{\orsig}$ such that $\chorsig{\theta}{\srsig}{\orsig}$ implies
\begin{equation*}
  \begin{tikzcd}
    w \rar[reduces, subscript=\srsig] \dar[relation][swap]{\theta}
     & w\mathrlap{'} \dar[relation, exists]{\theta}
    \\
    \octx \rar[reduces, exists, subscript=\orsig]
     & \octx\mathrlap{'}
  \end{tikzcd}
  \hphantom{'}
  \quad\text{and}\quad
  \begin{tikzcd}
    w \rar[reduces, exists, subscript=\srsig] \dar[relation][swap]{\theta}
     & w\mathrlap{'} \dar[relation, exists]{\theta}
    \\
    \octx \rar[reduces, subscript=\orsig]
     & \octx\mathrlap{' \,.}
  \end{tikzcd}
  \hphantom{' \,.}
\end{equation*}
We will first explain by example how such a signature $\orsig$ is constructed, reserving a formal description of the choreographing procedure to \cref{??}.

\newthought{Recall} from \cref{ch:string-rewriting} the string rewriting specification of a system that can rewrite strings over $\sralph = \Set{a,b}$ into the empty string if the initial string ends in $b$;
that specification consists of the axioms
\begin{equation*}
  \srsig = (a \wc b \reduces b) , (b \reduces \emp)
  \,.
\end{equation*}
The monoid homomorphism $\theta$ such that $\theta(a) = \atmR{a}$ and $\theta(b) = \defp{b}$ is a role assignment for this specification.

We can apply the role assignment $\theta$ to the axioms $\srsig$ to see which ordered rewritings must hold of the relation $\reduces_{\orsig}$ if $(\theta, \orsig)$ is to be a meaningful choreography of the specification $(\sralph, \srsig)$.
In this example, the axioms $\srsig$ together with $\theta$ induce the rewritings
\begin{equation*}
  \theta(a \wc b) = \atmR{a} \oc \defp{b} \reduces_{\orsig} \defp{b} = \theta(b)
  \quad\text{and}\quad
  \theta(b) = \defp{b} \reduces_{\orsig} (\octxe) = \theta(\emp)
  \,.
\end{equation*}


\begin{equation*}
  \begin{tikzcd}
    a \wc b \rar[reduces, subscript=\srsig] \dar[relation][swap]{\theta}
     & b \dar[relation]{\theta}
    \\
    \atmR{a} \oc \defp{b} \rar[reduces, subscript=\orsig]
     & \defp{b}
  \end{tikzcd}
  \quad\text{and}\quad
  \begin{tikzcd}
    b \rar[reduces, subscript=\srsig] \dar[relation][swap]{\theta}
     & \emp \dar[relation]{\theta}
    \\
    \defp{b} \rar[reduces, subscript=\orsig]
     & (\octxe)
  \end{tikzcd}
\end{equation*}
So, to find a meaningful choreography $(\theta, \orsig)$ for the string rewriting specification $(\sralph, \srsig)$, it suffices to find a signature $\orsig$ for which the rewritings $\atmR{a} \oc \defp{b} \reduces_{\orsig} \defp{b}$ and $\defp{b} \reduces_{\orsig} (\octxe)$ -- and only those rewritings -- are derivable.
In other words, $\atmR{a} \oc \defp{b} \reduces_{\orsig} \defp{b}$ and $\defp{b} \reduces_{\orsig} (\octxe)$ serve as constraints on $\orsig$ that we must solve.

To solve these constraints, we must find a definition for $\defp{b}$ that makes those -- and only those -- rewritings derivable.
In this instance, such a solution is the definition $\defp{b} \defd (\atmR{a} \limp \up \dn \defp{b}) \with \up \one$ and the corresponding signature, $\orsig \defd \bigl(\defp{b} \defd (\atmR{a} \limp \up \dn \defp{b}) \with \up \one\bigr)$.
Here is how we arrive at that solution:
\begin{itemize}
\item Let's temporarily restrict our attention to the constraint $\atmR{a} \oc \defp{b} \reduces_{\orsig} \defp{b}$.
  Notice that $\atmR{a} \oc (\atmR{a} \limp \up \dn \defp{b}) \reduces \defp{b}$.
  By the universal property of left-handed implication, there must exist an open derivation of $\lfocus{\atmR{a}}{\defp{b}}{}{\p{C}_1}$ from [...].
\item 
  Turning our attention to the constraint $\defp{b} \reduces_{\orsig} (\octxe)$, notice that $\up \one \reduces (\octxe)$.
  By the universal properties of $\up \one$, there must exist an open derivation of $\lfocus{\atmR{\octx}_L}{\defp{b}}{\atmL{\octx}_R}{\p{C}}$ from $\lfocus{\atmR{\octx}_L}{\up \one}{\atmL{\octx}_R}{\p{C}}$.
\end{itemize}
The least proposition $\defp{b}$ that has both of these open derivations is $\defp{b} \defd (\atmR{a} \limp \up \dn \defp{b}) \with \up \one$.

More generally, suppose that we have a constraint $\atmR{\octx}_L \oc \defp{a} \oc \atmL{\octx}_R \reduces_{\orsig} \octx'$.
Then notice that (morally) $\atmR{\octx}_L \oc (\atmR{\octx}_L \limp (\up \bigfuse \octx') \pmir \atmL{\octx}_R) \oc \atmL{\octx}_R \reduces \octx'$.
By the universal propeties of $\atmR{\octx}_L \limp (\up \bigfuse \octx') \pmir \atmL{\octx}_R$, there must exist an open derivation of $\lfocus{\atmR{\octx}_L}{\defp{a}}{\atmL{\octx}_R}{\p{C}}$ from $\lfocus{\atmR{\octx}_L}{\atmR{\octx}_L \limp (\up \bigfuse \octx') \pmir \atmL{\octx}_R}{\atmL{\octx}_R}{\p{C}}$.

Returning to our running example, we need to find a definition for $\defp{b}$ such that both $\atmR{a} \oc \defp{b} \reduces \defp{b}$ and $\defp{b} \reduces (\octxe)$ will be derivable.
By inversion, these induced rewritings will be derivable exactly when both
\begin{gather*}
  \lfocus{\atmR{a}}{\defp{b}}{}{\p{C}_1} \text{ for some $\p{C}_1$ such that } \rfocus{\defp{b}}{\p{C}_1}
\shortintertext{and}
  \lfocus{}{\defp{b}}{}{\p{C}_2} \text{ for some $\p{C}_2$ such that } \rfocus{(\octxe)}{\p{C}_2}
  \,.
\end{gather*}
It is easy to check
\begin{enumerate*}[label=\emph{(\roman*)}]
\item that the first condition would be satisfied if $\defp{b}$ were $\atmR{a} \limp \up \dn \defp{b}$; and
\item that the second condition would be satisfied if $\defp{b}$ were $\up \one$.
\end{enumerate*}
If $\defp{b}$ were somehow simultaneously both $\atmR{a} \limp \up \dn \defp{b}$ and $\up \one$, then both conditions would be satisfied.
Fortunately, additive conjunction allows us to do just that: when $\defp{b} \defd (\atmR{a} \limp \up \dn \defp{b}) \with \up \one$, the induced rewritings, $\atmR{a} \oc \defp{b} \reduces \defp{b}$ and $\defp{b} \reduces (\octxe)$ -- and only those rewritings -- are derivable.
$\octx_L \oc \defp{b} \oc \octx_R \reduces \octx'$ only if either 
\begin{itemize}
\item $\octx_L \reduces \octx'_L$ and $\octx' = \octx'_L \oc \defp{b} \oc \octx_R$ for some $\octx'_L$;
\item $\octx' = \octx_L \oc \octx_R$;
\item $\octx_L = \octx'_L \oc \atmR{a}$ and $\octx' = \octx'_L \oc \defp{b} \oc \octx_R$; or 
\item $\octx_R \reduces \octx'_R$ and $\octx' = \octx_L \oc \defp{b} \oc \octx'_R$ for some $\octx'_R$.
\end{itemize}





To choreograph a string rewriting specification, we would like to assign one, and only one, role to each symbol $a \in \sralph$: in the choreography, each symbol $a$ becomes either a message, $\atmL{a}$ or $\atmR{a}$, or a recursively defined process, $\defp{a}$.
A monoid homomorphism\fixnote{isomorphism?} from strings to ordered contexts that satisfies this condition is called a \vocab{[...] assignment}.


When applied to the specification's axioms, the [...] assignment $\theta$ induces the rewriting steps
\begin{equation*}
  \atmR{a} \oc \defp{b} \reduces \defp{b}
  \quad\text{and}\quad
  \defp{b} \reduces (\octxe)
  \,,
\end{equation*}
which we denote by $\theta(\srsig)$.

For the [...] assignment $\theta$ to yield an actual choreography of the axioms $\srsig$, we must be able to solve these induced rewritings for $\defp{b}$, determining a definition for $\defp{b}$ that makes these -- and only these -- rewriting steps derivable.

More generally, a [...] assignment $\theta$ yields a well-specified choreography for a specification with axioms $\srsig$ if the induced ordered rewriting steps $\theta(\srsig)$ are solvable with definitions for all recursively defined processes that make the induced rewritings $\theta(\srsig)$ -- and only those rewritings -- derivable.
In other words, $\theta$

Returning to our running example, we need to find a definition for $\defp{b}$ such that both $\atmR{a} \oc \defp{b} \reduces \defp{b}$ and $\defp{b} \reduces (\octxe)$ will be derivable.
By inversion, these induced rewritings will be derivable exactly when both
\begin{gather*}
  \lfocus{\atmR{a}}{\defp{b}}{}{\p{C}_1} \text{ for some $\p{C}_1$ such that } \rfocus{\defp{b}}{\p{C}_1}
\shortintertext{and}
  \lfocus{}{\defp{b}}{}{\p{C}_2} \text{ for some $\p{C}_2$ such that } \rfocus{(\octxe)}{\p{C}_2}
  \,.
\end{gather*}
It is easy to check
\begin{enumerate*}[label=\emph{(\roman*)}]
\item that the first condition would be satisfied if $\defp{b}$ were $\atmR{a} \limp \up \dn \defp{b}$; and
\item that the second condition would be satisfied if $\defp{b}$ were $\up \one$.
\end{enumerate*}
If $\defp{b}$ were somehow simultaneously both $\atmR{a} \limp \up \dn \defp{b}$ and $\up \one$, then both conditions would be satisfied.
Fortunately, additive conjunction allows us to do just that: when $\defp{b} \defd (\atmR{a} \limp \up \dn \defp{b}) \with \up \one$, the induced rewritings, $\atmR{a} \oc \defp{b} \reduces \defp{b}$ and $\defp{b} \reduces (\octxe)$ -- and only those rewritings -- are derivable.
$\octx_L \oc \defp{b} \oc \octx_R \reduces \octx'$ only if either 
\begin{itemize}
\item $\octx_L \reduces \octx'_L$ and $\octx' = \octx'_L \oc \defp{b} \oc \octx_R$ for some $\octx'_L$;
\item $\octx' = \octx_L \oc \octx_R$;
\item $\octx_L = \octx'_L \oc \atmR{a}$ and $\octx' = \octx'_L \oc \defp{b} \oc \octx_R$; or 
\item $\octx_R \reduces \octx'_R$ and $\octx' = \octx_L \oc \defp{b} \oc \octx'_R$ for some $\octx'_R$.
\end{itemize}


Not all [...] assignments yield well-specified choreographies.
This happens when there is no solution for the recursively defined propositionsthat makes all of the induced rewritings derivable.
\begin{itemize}
\item
  \emph{Each induced rewriting must have at least one process in its premise\fixnote{wc}.}
  For example, the [...] assignments $\theta'$ such that either $\theta'(b) = \atmL{b}$ or $\theta'(b) = \atmR{b}$ holds do \emph{not} yield well-specified choreographies.
  From the string rewriting axiom $b \reduces \emp$, the [...] assignment $\theta'$ induces either $\atmL{b} \reduces (\octxe)$ or $\atmR{b} \reduces (\octxe)$, and there is no solution that makes either of these induced ordered rewritings derivable.

\item
  \emph{Each induced rewriting must have at most one process in its premise\fixnote{wc}.}
  For example, the [...] assignment $\theta'$ such that $\theta'(a) = \defp{a}$ and $\theta'(b) = \defp{b}$ hold does not yield a well-specified choreography because there is no solution that makes $\defp{a} \oc \defp{b} \reduces \defp{b}$ derivable.

\item
  \emph{Each message in the premises of induced rewritings must be flowing toward that premise's process\fixnote{wc}.}
  For example, the [...] assignment $\theta'$ such that $\theta'(a) = \atmL{a}$ and $\theta'(b) = \defp{b}$ hold does not yield a well-specified choreography because there is no solution that makes $\atmL{a} \oc \defp{b} \reduces \defp{b}$ derivable.
  In \ac{PFOR} there is no process $\defp{b}$ that can receive a message, like $\atmL{a}$, that is flowing away.
\end{itemize}


For the choreography to be well-specified, this [...] assignment must induce from the string rewriting specification's axioms a collection of locally achievable  ordered rewriting steps\fixnote{reductions}.
If the ordered rewriting steps induced by the [...] assignment cannot be achieved by local communication, then the choreography is not well-specified.

For example, recall from \cref{ch:string-rewriting} the string rewriting specification of a system that can rewrite strings over $\sralph = \Set{a,b}$ into the empty string if the initial string ends in $b$;
that specification used axioms
\begin{equation*}
  \srsig = (a \wc b \reduces b) , (b \reduces \emp)
  \,.
\end{equation*}

So, to choreograph this specification, we must choose an assignment of roles -- either message or process -- to symbols $a$ and $b$ --
let's choose $a \mapsto \atmR{a}$ and $b \mapsto \defp{b}$.
From the axioms $\srsig$, this assignment induces the rewritings
\begin{equation*}
  \atmR{a} \oc \defp{b} \reduces \defp{b}
  \quad\text{and}\quad
  \defp{b} \reduces (\octxe)
  \,.
\end{equation*}
Are these reductions achievable by purely local communication?
Because our formula-as-process interpretation of ordered rewriting ensures that all communication is local, we need only verify that there is a solution for $\defp{b}$ [...].

Any solution for $\defp{b}$ must be consistent with $\atmR{a} \limp \up \dn \defp{b}$ so that $\atmR{a} \oc \defp{b} \reduces \defp{b}$ is derivable.
Furthermore, any solution for $\defp{b}$ must be consistent with $\up \one$ so that $\defp{b} \reduces \octxe$ is derivable.
The least such solution is
\begin{equation*}
  \defp{b} \defd (\atmR{a} \limp \up \dn \defp{b}) \with \up \one
  \,,
\end{equation*}
It indeed validates the required reductions,
\begin{gather*}
  \atmR{a} \oc \defp{b} = \atmR{a} \oc \bigl((\atmR{a} \limp \up \dn \defp{b}) \with \up \one\bigr) \reduces \defp{b} \\
  \defp{b} = \atmR{a} \oc \bigl((\atmR{a} \limp \up \dn \defp{b}) \with \up \one\bigr) \reduces (\octxe)
  \,,
\end{gather*}
and only the required reductions:
\begin{quotation}
  If $\octx_L \oc \defp{b} \oc \octx_R \reduces \octx'$, then either:
  \begin{itemize}
  \item $\octx_L = \octx'_L \oc \atmR{a}$ and $\octx' = \octx'_L \oc \defp{b} \oc \octx_R$, for some $\octx'_L$;
  \item $\octx' = \octx_L \oc \octx_R$;
  \item $\octx_L \reduces \octx'_L$ and $\octx' = \octx'_L \oc \defp{b} \oc \octx_R$, for some $\octx'_L$; or
  \item $\octx_R \reduces \octx'_R$ and $\octx' = \octx_L \oc \defp{b} \oc \octx'_R$, for some $\octx'_R$.
  \end{itemize}
\end{quotation}

\begin{gather*}
  \atmL{a} \oc \defp{b} \reduces \defp{b}
  \quad\text{and}\quad
  \defp{b} \reduces (\octxe)
  \\
  \defp{a} \oc \defp{b} \reduces \defp{b}
  \quad\text{and}\quad
  \defp{b} \reduces (\octxe)
  \\
  \defp{a} \oc \atmL{b} \reduces \atmL{b}
  \quad\text{and}\quad
  \atmL{b} \reduces (\octxe)
\end{gather*}



To be well-specified, 

An \emph{[...] assignment} $\theta$ is a monoid homomorphism from strings to ordered contexts that injectively maps each symbol $a \in \sralph$ to either a message, $\atmL{a}$ or $\atmR{a}$, or a recursively defined process, $\defp{a}$.%
\footnote{Injectivity keeps $\theta$ from identifying distinct symbols.}

Given an [...] assignment $\theta$, a string rewriting specification's axioms induce rewriting steps that must hold if the specification is to have a choreography.
For each axiom $w \reduces w' \in \srsig$, the [...] assignment $\theta$ induces a requirement that a faithful choreography must satisfy the rewriting step $\theta(w) \reduces \theta(w')$.

\section{Constructing a choreography from a specification}

For an example of this procedure, let's construct a choreography for the string rewriting specification of the system from \cref{ch:string-rewriting} that can rewrite strings over $\ialph = \Set{a,b}$ into the empty string.
Recall that that specification consisted of the axioms
\begin{equation*}
  \srsig = (a \wc b \reduces b) , (b \reduces \emp)
  \,.
\end{equation*}


The first step in constructing a choreography is to choose a \emph{[...] assignment} that maps each symbol to either an atom or recursively defined proposition, [which represent a message or recursively defined process,respectively.]
For example, $\theta = \Set{a \mapsto \atmR{a} , b \mapsto \defp{b}}$ is an [...] assignment that maps $a$ to a right-directed message and $b$ to a process.
Like $\theta$, all [...] assignments must be injective, to keep distinct symbols from becoming identified in the choreography.

Next, we apply the [...] assignment to each of the string rewriting specification's axioms and simultaneously replace the empty string with the empty ordered context.\footnote{Strictly speaking, the monoid operations are also exchanged, but because both are indicated by juxtaposition, this happens silently.}
This results in a collection of ordered rewriting steps that the choreography must satisfy if it is to be a faithful reflection of the string rewriting specification.
Applying $\theta$ to the axioms of \cref{??} yields 
\begin{equation*}
  \atmR{a} \oc \defp{b} \reduces \defp{b}
  \quad\text{and}\quad
  \defp{b} \reduces \octxe
\end{equation*}
as rewritings required of the choreography.

Finally, we solve for the recursively defined propositions that appear in the required rewritings.
In this example, $\defp{b}$ must be consistent with $\atmR{a} \limp \up \dn \defp{b}$ if $\atmR{a} \oc \defp{b} \reduces \defp{b}$ is to be derivable;
$\defp{b}$ must also be consistent with $\up \one$ if $\defp{b} \reduces \octxe$ is to be derivable.
The least such solution is $\defp{b} \defd (\atmR{a} \limp \up \dn \defp{b}) \with \up \one$.
Indeed, under this definition, 
\begin{equation*}
  \atmR{a} \oc \defp{b} \reduces \defp{b}
  \quad\text{and}\quad
  \defp{b} \reduces \octxe
\end{equation*}
are both derivable.

\clearpage
\subsection{}

Not all [...] assignments yield choreographies.
For instance, suppose we had chosen $\theta' = \Set{a \mapsto \defp{a} , b \mapsto \atmR{b}}$ or any other assignment $\theta'$ that maps $b$ to an atom.
Applying $\theta'$ to the second axiom would yield either $\atmR{b} \reduces \octxe$ or $\atmL{b} \reduces \octxe$ as required rewriting steps.
Neither of these make for a valid choreography both of which require a message to be recognized and acted upon by the ether.


\subsection{}

To construct a choreography, we need to find a \emph{choreographing assignment} that consistently localizes each axiom.

An assignment that maps both $a$ and $b$ to messages, such as $\theta = \Set{a \mapsto \atmR{a} , b \mapsto \atmR{b}}$ which results in $\atmR{a} \oc \atmR{b} \reduces \atmR{b}$  and $\atmR{b} \reduces \octxe$,
 

  
As a string rewriting specification, the axioms are interpreted from a global perspective.
For instance, the first axiom states that when the symbols $a \oc b$ occur in that order, they may be rewritten to $b$.
But the axiom does not describe how that rewriting occurs.

With choreographies, we would like to work at a (slightly) lower level of abstraction to describe 


Suppose that we are given a string rewriting specification that consists of axioms $?$ over the rewriting alphabet $\sralph$.
A \vocab{choreographing assignment} is an injection in which each symbol $a \in \sralph$ is mapped to an ordered proposition: either an atomic proposition, $\atmL{a}$ or $\atmR{a}$, or a recursively defined proposition, $\defp{a}$.

Given a choregraphing assignment $\theta$, we may construct a choregraphy from the string rewriting specification.
Intuitively, each axiom is annotated according to $\theta$, and then the resulting [...] are used to construct a family of recursive definitions, one for each $\defp{a}$ in the image of $\theta$.

A choreography is an ordered rewriting specification that simulates the string rewriting specification [...].

Consider the recurring string rewriting specification with axioms
\begin{equation*}
  \infer{a \oc b \reduces b}{}
  \qquad\text{and}\qquad
  \infer{b \reduces \emp}{}
  \:.
\end{equation*}
We must consistently annotate each symbol as either a left-directed atom, right-directed atom, or recursively defined proposition in sucha way that each axiom's premise $w$ has the form $w_1 \wc a \wc w_2$ with 

\begin{equation*}
  \atmR{a} \oc \defp{b} \reduces \defp{b}
  \qquad\text{and}\qquad
  \defp{b} \reduces \octxe
\end{equation*}

Now we must solve for $\defp{b}$, determining a definition $\defp{b} \defd \n{B}$ such that these two rewriting steps are derivable.
For the first step to be derivable, $\defp{b}$ should have a definition that is consistent with $\atmR{a} \limp \up \dn \defp{b}$, for 
\begin{equation*}
  \atmR{a} \oc (\atmR{a} \limp \up \dn \defp{b}) \reduces \defp{b}
\end{equation*}

Consider the choreographing assignment $\theta$ that maps $a$ to the atom $\atmR{a}$ and $b$ to the recursively defined proposition $\defp{b}$.
Upon annotating the above string rewriting axioms according to $\theta$, we arrive at the [...]
\begin{equation*}
  \atmR{a} \oc \dprop{b} \reduces \dprop{b}
  \qquad\text{and}\qquad
  \dprop{b} \reduces \one
  \,.
\end{equation*}

\begin{equation*}
  \dprop{b} \defd \atmR{a} \limp \up \dn \dprop{b}
  \qquad\text{and}\qquad
  \dprop{b} \defd \up \one
  \,,
\end{equation*}
respectively.
Combining these into a single definition that allows a nondeterministic choice between the two, we have
\begin{equation*}
  \dprop{b} \defd (\atmR{a} \limp \up \dn \dprop{b}) \with \up \one
  \,,
\end{equation*}
or $\dprop{b} \defd (\atmR{a} \limp \up \dprop{b}) \with \one$ if the minimally necessary shifts are elided.

By construction, this choreography is adequate with respect to the specification, in the sense that it can simulate each of the specification's possible steps and vice versa.
\begin{itemize}
\item $w \reduces w'$ only if $\theta(w) \reduces \theta(w')$
  For example, just as the string rewriting specification admits $a \oc b \reduces b$, the ordered rewriting choreography admits
  \begin{equation*}
    \theta(a \oc b) = \atmR{a} \oc \defp{b} = \atmR{a} \oc \bigl((\atmR{a} \limp \up \dn \defp{b}) \with \one\bigr) \reduces \defp{b} = \theta(b)
    \,.
  \end{equation*}

\item $\theta(w) \reduces \octx'$ only if $w \reduces \theta^{-1}(\octx')$
  For example, just as the ordered rewriting choreography admits $\theta(b) = \defp{b} \reduces \octxe$, the string rewriting specification admits $b \reduces \emp = \theta^{-1}(\octxe)$.
\end{itemize}
% The choreography can simulate each of the specification's possible rewriting steps: for example, $\theta(a \oc b) = \atmR{a} \oc \dprop{b} \reduces \dprop{b} = \theta(b)$, just as $a \oc b \reduces b$.
% Conversely, each of the choreography's possible rewriting steps can be simulated by the specification: for example, $\theta^{-1}(\atmR{a} \oc \dprop{b}) = a \oc b \reduces b = \theta^{-1}(\dprop{b})$, just as $\atmR{a} \oc \dprop{b} \reduces \dprop{b}$.

\subsection{Formal description}

In this \lcnamecref{??}, we present a more formal description of the above procedure for choreographing string rewriting specifications.
We define a judgment $\chorsig{\theta}{\srsig}{\orsig}$ that, when given a string rewriting specification $(\sralph, \srsig)$ and [...] assignment $\theta$, yields a formula-as-process ordered rewriting signature $\orsig$ that makes $\theta$ a bisimulation [between $\reduces_{\srsig}$ and $\reduces_{\orsig}$] if such a signature exists:
\begin{equation*}
  \chorsig{\theta}{\srsig}{\orsig}
  \quad\text{implies}\quad
  \begin{tikzcd}
    w \rar[reduces, subscript=\srsig] \dar[relation][swap]{\theta}
     & w\mathrlap{'} \dar[relation, exists]{\theta}
    \\
    \octx \rar[reduces, exists, subscript=\orsig]
     & \octx\mathrlap{'}
  \end{tikzcd}
  \hphantom{'}
  \quad\text{and}\quad
  \begin{tikzcd}
    w \rar[reduces, exists, subscript=\srsig] \dar[relation][swap]{\theta}
     & w\mathrlap{'} \dar[relation, exists]{\theta}
    \\
    \octx \rar[reduces, subscript=\orsig]
     & \octx\mathrlap{' \,.}
  \end{tikzcd}
  \hphantom{' \,.}
\end{equation*}
This principal judgment
% is $\chorsig{\theta}{\srsig}{\orsig}$, and it
also relies on an auxiliary judgment, $\qimp{\atmR{\octx}_L}{\up \p{A}}{\atmL{\octx}_R}{\n{B}}$.
% Before giving the rules for the principal judgment, we will [...].

\newthought{The auxiliary} judgment $\qimp{\atmR{\octx}_L}{\up \p{A}}{\atmL{\octx}_R}{\n{B}}$ elaborates\fixnote{word choice?} the quasi-propo\-si\-tion $\atmR{\octx}_L \limp \up \p{A} \pmir \atmL{\octx}_R$ into a well-formed proposition $\n{B}$ by nondeterministically abstracting one-by-one from either the left or right contexts.%
\footnote{This procedure could be made deterministic by preferring one side over the other, but we refrain from doing so because the choice of side to prefer is completely arbitrary.}
This proposition $\n{B}$ is semantically equivalent to the quasi-proposition $\atmR{\octx}_L \limp \up \p{A} \pmir \atmL{\octx}_R$ in the sense that the two intuitively satisfy the same \enquote{left-focus judgments}:
% $\n{B}$ satisfies $\atmR{\octx}_L \oc \n{B} \oc \atmL{\octx}_R \reduces \octx'$ if, and only if, $\rfocus{\octx'}{\p{A}}$.
$\lfocus{\atmR{\lctx}_L}{\n{B}}{\atmL{\lctx}_R}{\p{C}}$ if, and only if, $\atmR{\lctx}_L = \atmR{\octx}_L$ and $\atmL{\lctx}_R = \atmL{\octx}_R$ and $\p{C} = \p{A}$.
(This is proved below as \cref{lem:qimp-correct}.)

The auxiliary elaboration judgment is defined inductively by the following rules.
\begin{inferences}
  \infer[\jrule{$\up$Q}]{\qimp{(\octxe)}{\up \p{A}}{(\octxe)}{\up \p{A}}}{}
  \\
  \infer[\jrule{$\limp$Q}]{\qimp{(\atmR{\octx}_L \oc \p{\atmR{a}})}{\up \p{A}}{\atmL{\octx}_R}{\p{\atmR{a}} \limp \n{B}}}{
    \qimp{\atmR{\octx}_L}{\up \p{A}}{\atmL{\octx}_R}{\n{B}}}
  \and
  \infer[\jrule{$\pmir$Q}]{\qimp{\atmR{\octx}_L}{\up \p{A}}{(\p{\atmL{a}} \oc \atmL{\octx}_R)}{\n{B} \pmir \p{\atmL{a}}}}{
    \qimp{\atmR{\octx}_L}{\up \p{A}}{\atmL{\octx}_R}{\n{B}}}
\end{inferences}
The $\jrule{$\limp$Q}$ rule states that the quasi-proposition $(\atmR{\octx}_L \oc \p{\atmR{a}}) \limp \up \p{A} \pmir \atmL{\octx}_R$ is equivalent to $\p{\atmR{a}} \limp \n{B}$ if $\atmR{\octx}_L \limp \up \p{A} \pmir \atmL{\octx}_R$ is equivalent to $\n{B}$.
Notice that the $\jrule{$\limp$Q}$ rule moves $\p{\atmR{a}}\!\!$ from the right of $\atmR{\octx}_L$ to the left of $\n{B}$;
this is admittedly counterintuitive, but it is closely related to the equally counterintuitive currying law for left-handed implication in ordered logic:
% Likewise, the quasi-proposition $\atmR{\octx}_L \limp \up \p{A} \pmir (\p{\atmL{a}} \oc \atmL{\octx}_R)$ is equivalent to $(\atmR{\octx}_L \limp \up \p{A} \pmir \atmL{\octx}_R) \pmir \p{\atmL{a}}$.
$(A \fuse B) \limp C \dashv\vdash B \limp (A \limp C)$.
Symmetrically, the $\jrule{$\pmir$Q}$ rule is closely related to the currying law for right-handed implication: $C \pmir (A \fuse B) \dashv\vdash (C \pmir B) \pmir A$.

This intuition is captured in the proof of the following \lcnamecref{lem:qimp-correct}.
\begin{lemma}\label{lem:qimp-correct}
  % If $\chorax{\theta}{w_1}{\p{C}}{w_2}{\n{B}}$, then $\lfocus{\theta(w_1)}{\n{B}}{\theta(w_2)}{\p{C}}$.
  If $\qimp{\atmR{\octx}_L}{\up \p{A}}{\atmL{\octx}_R}{\n{B}}$, then $\lfocus{\atmR{\lctx}_L}{\n{B}}{\atmL{\lctx}_R}{\p{C}}$ if, and only if, $\atmR{\lctx}_L = \atmR{\octx}_L$ and $\atmL{\lctx}_R = \atmL{\octx}_R$ and $\p{C} = \p{A}$.
  %
  % Moreover, if $\qimp{\atmR{\octx}_L}{\up \p{A}}{\atmL{\octx}_R}{\n{B}}$, then $\atmR{\lctx}_L \oc \n{B} \oc \atmL{\lctx}_R \reduces \lctx'$ if, and only if, there exist contexts $\atmR{\lctx}'_L$, $\lctx'_0$, and$\atmL{\lctx}'_R$ such that $\atmR{\lctx}'_L \oc \atmR{\octx}_L = \atmR{\lctx}_L$ and $\atmL{\octx}_R \oc \atmL{\lctx}'_R = \atmL{\lctx}_R$ and $\rfocus{\lctx'_0}{\p{A}}$ and $\lctx' = \atmR{\lctx}'_L \oc \lctx'_0 \oc \atmL{\lctx}'_R$.
\end{lemma}
\begin{proof}
  By induction over the structure of the given elaboration.

  As an example case, consider
  % \begin{itemize}
  % \item Consider the case in which
  %   \begin{equation*}
  %     \infer{\chorax{\theta}{\emp}{\p{A}}{\emp}{\up \p{A}}}{}
  %     \,.
  %   \end{equation*}
  %   We must show that $\lfocus{\atmR{\octx}_L}{\up \p{A}}{\atmL{\octx}_R}{\p{C}}$ if, and only if, $\atmR{\octx}_L = \atmL{\octx}_R = \theta(\emp)$ and $\p{A} = \p{C}$.
  %   Indeed, the $\lrule{\up}$ rule is the unique rule for left-focusing on $\up \p{A}$, and $\octxe = \theta(\emp)$ because $\theta$ is a monoid homomorphism.
  % 
  % \item Consider the case in which
    \begin{equation*}
      \infer{\qimp{(\atmR{\octx}_L \oc \p{\atmR{a}})}{\up \p{A}}{\atmL{\octx}_R}{\p{\atmR{a}} \limp \n{B}}}{
        \qimp{\atmR{\octx}_L}{\up \p{A}}{\atmL{\octx}_R}{\n{B}}}
      \,.
    \end{equation*}
    We must show that $\lfocus{\atmR{\lctx}_L}{\p{\atmR{a}} \limp \n{B}}{\atmL{\lctx}_R}{\p{C}}$ if, and only if, $\atmR{\lctx}_L = \atmR{\octx}_L \oc \p{\atmR{a}}$ and $\atmL{\lctx}_R = \atmL{\octx}_R$ and $\p{C} = \p{A}$.
    Indeed, the $\lrule{\limp}$ rule is the unique rule for left-focusing on the proposition $\p{\atmR{a}} \limp \n{B}$, so $\lfocus{\atmR{\lctx}_L}{\p{\atmR{a}} \limp \n{B}}{\atmL{\lctx}_R}{\p{C}}$ if, and only if, $\atmR{\lctx}_L = \atmR{\lctx}'_L \oc \p{\atmR{a}}\!\!$ and $\lfocus{\atmR{\lctx}'_L}{\n{B}}{\atmL{\lctx}_R}{\p{C}}$ for some $\atmR{\lctx}'_L$.
    By the inductive hypothesis, we have $\lfocus{\atmR{\lctx}'_L}{\n{B}}{\atmL{\lctx}_R}{\p{C}}$ if, and only if, $\atmR{\lctx}'_L = \atmR{\octx}_L$ and $\atmL{\lctx}_R = \atmL{\octx}_R$ and $\p{C} = \p{A}$.
    Putting everything together, $\lfocus{\atmR{\lctx}_L}{\p{\atmR{a}} \limp \n{B}}{\atmL{\lctx}_R}{\p{C}}$ if, and only if, $\atmR{\lctx}_L = \atmR{\octx}_L \oc \p{\atmR{a}}$ and $\atmL{\lctx}_R = \atmL{\octx}_R$ and $\p{C} = \p{A}$, as required.
  % 
  % \item
  %   The case in which
  % \begin{equation*}
  %   \infer{\chorax{\theta}{w_1}{\p{A}}{b \oc w_2}{\n{B} \pmir \atmL{b}}}{
  %     \chorax{\theta}{w_1}{\p{A}}{w_2}{\n{B}} &
  %     \text{($\theta(b) = \atmL{b}$)}}
  % \end{equation*}
  %   is symmtric to the previous one.
  % %
  % \qedhere
  % \end{itemize}
\end{proof}



\newthought{The principal} judgment is $\chorsig{\theta}{\srsig}{\orsig}$.
Given a [...] assignment $\theta$ and a string rewriting signature $\srsig$, this judgment produces a formula-as-process ordered rewriting signature $\orsig$ that, together with $\theta$, constitutes a [well-specified]\fixnote{?} choreography of the string rewriting specification $(\sralph, \srsig)$.

In other words, $\orsig$ is a solution to $\theta(\srsig)$, the rewritings induced by $\theta$ from the string rewriting axioms $\srsig$.
That is, if $\chorsig{\theta}{\srsig}{\orsig}$, then $\theta$ is a (strong) bisimulation between $\reduces_{\srsig}$ and $\reduces_{\orsig}$.
\marginnote{$\!
  \begin{tikzcd}[ampersand replacement=\&]
    w \rar[reduces, subscript=\srsig] \dar[relation][swap]{\theta}
     \& w\mathrlap{'} \dar[relation, exists]{\theta}
    \\
    \theta(w) \rar[reduces, exists, subscript=\orsig]
     \& \theta(w')
  \end{tikzcd}
  $ and $
  \begin{tikzcd}[ampersand replacement=\&]
    w \rar[reduces, exists, subscript=\srsig] \dar[relation][swap]{\theta}
     \& w\mathrlap{'} \dar[relation, exists]{\theta}
    \\
    \theta(w) \rar[reduces, subscript=\orsig]
     \& \octx \mathrlap{' = \theta(w')}
  \end{tikzcd}
  %\hphantom{' = \theta(w')}
  $}
\footnote{%
  Actually, we end up proving a stronger soundness result in \cref{??}.}
% The exact converse -- that $\theta(w) \reduces_{\orsig} \theta(w')$ implies $w \reduces_{\srsig} w'$ -- does hold, but we can prove an even stronger soundness result.
%
If $\chorsig{\theta}{\srsig}{\orsig}$ is not derivable for any $\orsig$, then the [...] assignment $\theta$ does not yield a [well-specified]\fixnote{?} choreography of $\srsig$.

% This judgment relies on an auxiliary judgment, $\qimp{\atmR{\octx}_L}{\up \p{A}}{\atmL{\octx}_R}{\n{B}}$, that transforms the quasi-proposition $\atmR{\octx}_L \limp \up \p{A} \pmir \atmL{\octx}_R$ into a well-formed proposition $\n{B}$ by nondeterministically abstracting atoms one-by-one from either the left or right contexts.
% The proposition $\n{B}$ is semantically equivalent to the quasi-proposition $\atmR{\octx}_L \limp \up \p{A} \pmir \atmL{\octx}_R$ in the sense that $\n{B}$ satisfies $\atmR{\octx}_L \oc \n{B} \oc \atmL{\octx}_R \reduces \octx'$ if, and only if, $\rfocus{\octx'}{\p{A}}$.

This choreographing judgment is defined by just two rules:
\begin{gather*}
  \infer{\chorsig{\theta}{\srsige}{\orsige}}{}
  \\
  \infer{\chorsig{\theta}{\srsig_0, \bigl(w^L_i \wc a \wc w^R_i \reduces w'_i\bigr)_{i \in \mathcal{I}}}{\orsig_0, \bigl(\defp{a} \defd \bigwith_{i \in \mathcal{I}} \n{A}_i\bigr)}}{
    \begin{array}[b]{@{}c@{}}
      \text{($\theta(a) = \defp{a}$)} \quad
      \chorsig{\theta}{\srsig_0}{\orsig_0} \quad
      \text{($\defp{a} \notin \dom{\orsig_0}$)}
      \\
      \multipremise{i \in \mathcal{I}}{
        \text{$\bigl(\theta(w^L_i) = \atmR{\octx}^L_i\bigr)$} \quad
        \text{$\bigl(\theta(w^R_i) = \atmL{\octx}^R_i\bigr)$} \quad
        \qimp{\atmR{\octx}^L_i}{\up \bigfuse \theta(w'_i)}{\atmL{\octx}^R_i}{\n{A}_i}}
    \end{array}}
\end{gather*}
The first of these rules is straightforward: an empty \acl{SR} signature choreographs as an empty ordered rewriting signature.
The second rule is quite a lot to parse and needs to be broken down step by step:
\begin{enumerate}
\item
  Choose a symbol $a$ that is mapped by $\theta$ to a recursively defined proposition, $\defp{a}$.
  Then reorganize the \ac{SR} signature $\srsig$, collecting together all axioms in $\srsig$ that have an $a$ in their premises.
  Let $\bigl(w^L_i \wc a \wc w^R_i \reduces w'_i\bigr)_{i \in \mathcal{I}}$ be those axioms, so that $\srsig = \srsig_0 , \bigl(w^L_i \wc a \wc w^R_i \reduces w'_i\bigr)_{i \in \mathcal{I}}$ for some $\srsig_0$.
\item
  Construct a well-specified choreography $\orsig_0$ from $\srsig_0$ and $\theta$, using the judgment $\chorsig{\theta}{\srsig_0}{\orsig_0}$.
  Check that $\orsig_0$ gives no definition for $\defp{a}$, otherwise there is some axiom in $\srsig_0$ that contains $a$ in its premise and $\bigl(w^L_i \wc a \wc w^R_i \reduces w'_i\bigr)_{i \in \mathcal{I}}$ does not correctly constitute all such axioms.
\item
  Check that each $w^L_i$ contains only those symbols that map to right-directed atoms, using the side condition $\theta(w^L_i) = \atmR{\octx}^L_i$.
  Symmetrically, check that each $w^R_i$ contains only symbols that map to left-directed atoms, using the side condition $\theta(w^R_i) = \atmL{\octx}^R_i$.
\item
  Elaborate each quasi-proposition $\atmR{\octx}^L_i \limp \up \bigfuse \theta(w'_i) \pmir \atmL{\octx}^R_i$ into a semantically equivalent proposition $\n{A}_i$.
  Based on \cref{??}, $\lfocus{\theta(w^L_i)}{\n{A}_i}{\theta(w^R_i)}{\bigfuse \theta(w'_i)}$, and so this proposition acts as the image of the axiom $w^L_i \wc a \wc w^R_i \reduces w'_i$ under $\theta$ -- that is, $\theta(w^L_i) \oc \n{A}_i \oc \theta(w^R_i) \reduces \theta(w'_i)$.
\item
  Collect the $\n{A}_i$s into a single definition, $\defp{a} \defd \bigwith_{i \in \mathcal{I}} \n{A}_i$, which, based on steps 2 and 4, describes all of the axioms from $\srsig$ that contain $a$ in their premises.
\end{enumerate}

If $\chorsig{\theta}{\srsig}{\orsig}$, then $\theta$ is a bisimulation.
That is, $\chorsig{\theta}{\srsig}{\orsig}$ implies
\begin{equation*}
  \begin{tikzcd}
    w \rar[reduces, subscript=\srsig] \dar[relation][swap]{\theta}
      & w\mathrlap{'} \dar[relation, exists]{\theta}
    \\
    \theta(w) \rar[reduces, exists, subscript=\orsig] & \theta(w')
  \end{tikzcd}
  \quad\text{and}\quad
  \begin{tikzcd}
    w \rar[reduces, exists, subscript=\srsig] \dar[relation][swap]{\theta}
      & w\mathrlap{'} \dar[relation, exists]{\theta}
    \\
    \theta(w) \rar[reduces, subscript=\orsig] & \octx \mathrlap{' = \theta(w') \,.}
  \end{tikzcd}
  \hphantom{' = \theta(w') \,.}
\end{equation*}



As stated earlier, when $\chorsig{\theta}{\srsig}{\orsig}$, the string rewriting step $w \reduces_{\srsig} w'$ holds if, and only if, the ordered rewriting step $\theta(w) \reduces_{\orsig} \theta(w')$ holds.
We prove the left-to-right direction as the following completeness \lcnamecref{thm:chor-complete} and then prove a stronger soundness \lcnamecref{thm:chor-sound} that implies the right-to-left direction.
% Each axiom $w \reduces w'$ in the string rewriting signature $\srsig$ is processed in turn.
% \begin{itemize}
% \item First, we verify that the premise $w$ may be expressed as $w = w_1 \wc a \wc w_2$, where:
% $\theta$ assigns a process role to $a$ (\ie, $\theta(a) = \defp{a}$ for some $\defp{a}$);
% right-directed message roles to all symbols in $w_1$ (\ie, $\theta(w_1) = \atmR{\octx}_L$ for some $\atmR{\octx}_L$);
% and left-directed messages roles to all symbols in $w_2$ (\ie, $\theta(w_2) = \atmL{\octx}_R$ for some $\atmL{\octx}_R$).

% \item
%   Next, we construct the quasi-proposition $\theta(w_1) \limp \up \bigfuse \theta(w') \pmir \theta(w_2)$.

% \item
%   Last, we inductively 
% \end{itemize}

% To define this choreographing judgment, we also use an auxiliary judgment that choreographs individual axioms.
% Given a [...] assignment $\theta$, strings $w_1$ and $w_2$, and a positive proposition $\p{A}$, the judgment $\chorax{\theta}{w_1}{\p{A}}{w_2}{\n{B}}$ checks that $\theta(w_1)$ and $\theta(w_2)$ consist of only right- and left-directed atoms, respectively, and then produces a negative proposition $\n{B}$ that is morally $\theta(w_1) \limp \up \p{A} \pmir \theta(w_2)$ -- that is, a proposition $\n{B}$ such that $\lfocus{\theta(w_1)}{\n{B}}{\theta(w_2)}{\p{A}}$.

% Each of the axioms is processed one-by-one.
% From the axiom $w \reduces w'$, a symbol $a$ is nondeterministically selected from the premise $w$;
% the selected symbol must be assigned a process role by $\theta$.

% For each axiom $w \reduces w'$, we verfify that the premise $w$ may be expressed as $w = w_1 \wc a \wc w_2$, where $\theta$ assigns a process role to $a$, right-directed message roles to all symbols in $w_1$, and left-directed messages roles to all symbols in $w_2$.
% Then, for each process $\defp{a}$, all axioms with premises $w_1 \wc a \wc w_2$
% \begin{inferences}
%   \infer{(\octxe) \limp \up \p{A} \pmir (\octxe) \rightsquigarrow \up \p{A}}{}
%   \\
%   \infer{(\atmR{\octx}_L \oc \atmR{a}) \limp \up \p{A} \pmir \atmL{\octx}_R \rightsquigarrow \atmR{a} \limp \n{B}}{
%     \atmR{\octx}_L \limp \up \p{A} \pmir \atmL{\octx}_R \rightsquigarrow \n{B}}
%   \and
%   \infer{\atmR{\octx}_L \limp \up \p{A} \pmir (\atmL{a} \oc \atmL{\octx}_R) \rightsquigarrow \n{B} \pmir \atmL{a}}{
%     \atmR{\octx}_L \limp \up \p{A} \pmir \atmL{\octx}_R \rightsquigarrow \n{B}}
% \end{inferences}
% When $\theta(b) = \atmR{b}$, the quasi-proposition $\theta(w_1 \wc b) \limp \up \p{A} \pmir \theta(w_2)$ is equivalent to $\atmR{b} \limp \bigl(\theta(w_1) \limp \up \p{A} \pmir \theta(w_2)\bigr)$.%
% \footnote{That $b$ moves from the right of $w_1$ to the left of $\theta(w_1)$ as $\atmR{b}$ is somewhat counterintuitive, but the proof of \cref{??} explains.}
% Likewise, when $\theta(b) = \atmL{b}$, the quasi-proposition $\theta(w_1) \limp \up \p{A} \pmir \theta(b \wc w_2)$ is equivalent to $\bigl(\theta(w_1) \limp \up \p{A} \pmir \theta(w_2)\bigr) \pmir \atmL{b}$.
 
% \begin{inferences}
%   \infer{(\octxe) \limp \up \p{A} \pmir (\octxe) \rightsquigarrow \up \p{A}}{}
%   \\
%   \infer{(\atmR{\octx}_L \oc \atmR{a}) \limp \up \p{A} \pmir \atmL{\octx}_R \rightsquigarrow \atmR{a} \limp \n{B}}{
%     \atmR{\octx}_L \limp \up \p{A} \pmir \atmL{\octx}_R \rightsquigarrow \n{B}}
%   \and
%   \infer{\atmR{\octx}_L \limp \up \p{A} \pmir (\atmL{a} \oc \atmL{\octx}_R) \rightsquigarrow \n{B} \pmir \atmL{a}}{
%     \atmR{\octx}_L \limp \up \p{A} \pmir \atmL{\octx}_R \rightsquigarrow \n{B}}
%   \\
%   \infer{\chorsig{\theta}{\srsig, (w_1 \wc a \wc w_2 \reduces w')}{\orsig, (\defp{a} \defd \n{A} \with \n{B})}}{
%     \begin{array}[b]{@{}c@{}}
%       \text{($\theta(w_1) = \atmR{\octx}_L$)} \quad
%       \text{($\theta(a) = \defp{a}$)} \quad
%       \text{($\theta(w_2) = \atmL{\octx}_R$)} \\
%       \atmR{\octx}_L \limp \up \bigfuse \theta(w') \pmir \atmL{\octx}_R \rightsquigarrow \n{B} \quad
%       \chorsig{\theta}{\srsig}{\orsig, (\defp{a} \defd \n{A})}
%     \end{array}}
%   \\
%   \infer{\chorsig{\theta}{\srsig, (w_1 \wc a \wc w_2 \reduces w')}{\orsig, (\defp{a} \defd \n{B})}}{
%     \begin{array}[b]{@{}c@{}}
%       \text{($\theta(w_1) = \atmR{\octx}_L$)} \quad
%       \text{($\theta(a) = \defp{a}$)} \quad
%       \text{($\theta(w_2) = \atmL{\octx}_R$)} \\
%       \atmR{\octx}_L \limp \up \bigfuse \theta(w') \pmir \atmL{\octx}_R \rightsquigarrow \n{B} \quad
%       \chorsig{\theta}{\srsig}{\orsig} \quad
%       \text{($\defp{a} \notin \dom{\orsig}$)}
%     \end{array}}
% \end{inferences}

% Judgments $\chorsig{\theta}{\srsig}{\orsig}$ and $\chorax{\theta}{w_1}{\n{A}}{w_2}{\n{B}}$.
% In both judgments, all terms before the $\chorarrow$ are inputs; all terms after the $\chorarrow$ are outputs.

% \begin{inferences}
%   \infer{\chorsig{\theta}{\srsig, (w_1 \wc a \wc w_2 \reduces w')}{\orsig, (\defp{a} \defd \n{A} \with \n{B})}}{
%     \begin{array}[b]{@{}c@{}}
%       \text{($\theta(w_1) = \atmR{\octx}_L$)} \quad
%       \text{($\theta(a) = \defp{a}$)} \quad
%       \text{($\theta(w_2) = \atmL{\octx}_R$)} \\
%       \atmR{\octx}_L \limp \up \bigfuse \theta(w') \pmir \atmL{\octx}_R \rightsquigarrow \n{B} \quad
%       \chorsig{\theta}{\srsig}{\orsig, (\defp{a} \defd \n{A})}
%     \end{array}}
%   \\
%   \infer{\chorsig{\theta}{\srsig, (w_1 \wc a \wc w_2 \reduces w')}{\orsig, (\defp{a} \defd \n{B})}}{
%     \begin{array}[b]{@{}c@{}}
%       \text{($\theta(w_1) = \atmR{\octx}_L$)} \quad
%       \text{($\theta(a) = \defp{a}$)} \quad
%       \text{($\theta(w_2) = \atmL{\octx}_R$)} \\
%       \atmR{\octx}_L \limp \up \bigfuse \theta(w') \pmir \atmL{\octx}_R \rightsquigarrow \n{B} \quad
%       \chorsig{\theta}{\srsig}{\orsig} \quad
%       \text{($\defp{a} \notin \dom{\orsig}$)}
%     \end{array}}
% \end{inferences}


% \begin{inferences}
%   \infer{\chorsig{\theta}{\sige}{\sige}}{}
%   \and
%   \infer{\chorsig{\theta}{\sig, w \reduces w'}{\sig', \proc{a} \defd \n{A}_1 \with \n{A}_2(\up \p{C})}}{
%     \chorsig{\theta}{\sig}{\sig'} &
%     \chorax{\theta}{w \reduces w'}{\proc{a}}{\n{A}_2(\Box)}{\p{C}} &
%     \text{($\sig'(\proc{a}) = \n{A}_1$)}}
%   \\
%   \infer{\chorsig{\theta}{\sig, w \reduces w'}{\sig', \proc{a} \defd \n{A}(\up \p{C})}}{
%     \chorsig{\theta}{\sig}{\sig'} &
%     \chorax{\thea}{w \reduces w'}{\proc{a}}{\n{A}(\Box)}{\p{C}} &
%     \text{($\proc{a} \notin \dom{\sig'}$)}}
%   \\
%   \infer{\chorax{\theta}{a \reduces w'}{\proc{a}}{\Box}{\bigfuse \octx'}}{
%     \text{($\theta(a) = \proc{a}$)} &
%     \text{($\theta(w') = \octx'$)}}
%   \and
%   \infer{\chorax{\theta}{b \oc w \reduces w'}{\proc{a}}{\n{A}(\atmR{b} \limp \Box)}{\p{C}}}{
%     \chorax{\theta}{w \reduces w'}{\proc{a}}{\n{A}(\Box)}{\p{C}} &
%     \text{($\theta(b) = \atmR{b}$)}}
%   \\
%   \infer{\chorax{\theta}{w \oc b \reduces w'}{\proc{a}}{\n{A}(\Box \pmir \atmL{b})}{C}}{
%     \chorax{\theta}{w \reduces w'}{\proc{a}}{\n{A}(\Box)}{\p{C}} &
%     \text{($\theta(b) = \atmL{b}$)}}
% \end{inferences}




\begin{lemma}[Weakening]
  If $\octx \reduces_{\orsig} \octx'$ and $\dom{\orsig} \cap \dom{\orsig'} = \emptyset$, then $\octx \reduces_{\orsig, \orsig'} \octx'$.
  % Similarly, if $\octx \reduces_{\orsig, (\defp{a} \defd \n{A})} \octx'$ or $\octx \reduces_{\orsig, (\defp{a} \defd \n{B})} \octx'$, then $\octx \reduces_{\orsig, (\defp{a} \defd \n{A} \with \n{B})} \octx'$.
\end{lemma}
\begin{proof}
  By induction over the structure of the given rewriting step.
\end{proof}

% \begin{lemma}
%   If $(w \reduces w') \in \srsig$ and $\chorsig{\theta}{\srsig}{\orsig}$, then $\theta(w) \reduces_{\orsig} \theta(w')$.
% \end{lemma}
% \begin{proof}
%   By induction over the structure of the given choreographing derivation, $\chorsig{\theta}{\srsig}{\orsig}$.
%   \begin{itemize}
%   \item Consider the case in which $w = w_1 \oc a \oc w_2 \reduces w'$ is the axiom in question and
%     \begin{equation*}
%       \infer{\chorsig{\theta}{\srsig, w_1 \oc a \oc w_2 \reduces w'}{\orsig, \defp{a} \defd \n{A} \with \n{B}}}{
%         \text{($\theta(a) = \defp{a}$)} &
%         \chorax{\theta}{w_1}{\bigfuse \theta(w')}{w_2}{\n{B}} &
%         \chorsig{\theta}{\srsig}{\orsig, \defp{a} = \n{A}}}
%     \end{equation*}
%     It follows from \cref{??} that $\lfocus{\theta(w_1)}{\n{B}}{\theta(w_2)}{\bigfuse \theta(w')}$, and hence $\lfocus{\theta(w_1)}{\n{A} \with \n{B}}{\theta(w_2)}{\bigfuse \theta(w')}$.
%     And because $\rfocus{\theta(w')}{\bigfuse \theta(w')}$ and $\defp{a} \defd \n{A} \with \n{B}$, we have $\theta(w) = \theta(w_1) \oc \defp{a} \oc \theta(w_2) \reduces \theta(w')$.
  
%   \item Consider the case in which
%     \begin{equation*}
%       \infer{\chorsig{\theta}{\sig, v_1 \oc a \oc v_2 \reduces v'}{\sig', \hat{a} \defd \n{A} \with \n{B}}}{
%         \text{($\theta(a) = \hat{a}$)} &
%         \chorax{\theta}{v_1}{\bigfuse \theta(v')}{v_2}{\n{B}} &
%         \chorsig{\theta}{\sig}{\sig'} &
%         \text{($\sig'(\hat{a}) = \n{A}$)}}
%     \end{equation*}
%     and the axiom $w \reduces w'$ comes from $\sig$.
%     By the inductive hypothesis, $\theta(w) \reduces_{\sig'} \theta(w')$.
%     By \cref{??}, $\theta(w) \reduces_{\sig', \hat{a} \defd \n{A} \with \n{B}} \theta(w')$.
  
%   \item Consider the case in which
%     \begin{equation*}
%       \infer{\chorsig{\theta}{\sig, v_1 \oc a \oc v_2 \reduces v'}{\sig', \hat{a} \defd \n{B}}}{
%         \text{($\theta(a) = \hat{a}$)} &
%         \chorax{\theta}{v_1}{\bigfuse \theta(v')}{v_2}{\n{B}} &
%         \chorsig{\theta}{\sig}{\sig'} &
%         \text{($\hat{a} \notin \dom{\sig'}$)}}
%     \end{equation*}
%     and the axiom $w \reduces w'$ comes from $\sig$.
%     By the inductive hypothesis, $\theta(w) \reduces_{\sig'} \theta(w')$.
%     By \cref{??}, $\theta(w) \reduces_{\sig', \hat{a} \defd \n{B}} \theta(w')$.
%   %
%   \qedhere
%   \end{itemize}
% \end{proof}

\begin{theorem}[Completeness]\leavevmode
  If $\chorsig{\theta}{\srsig}{\orsig}$, then $w \reduces_{\srsig} w'$ implies $\theta(w) \reduces_{\orsig} \theta(w')$.%
  \marginnote{If $\chorsig{\theta}{\srsig}{\orsig}$, then $
    \begin{tikzcd}[ampersand replacement=\&]
      w \rar[reduces, subscript=\srsig] \dar[relation][swap]{\theta}
       \& w\mathrlap{'} \dar[relation, exists]{\theta}
      \\
      \theta(w) \rar[reduces, exists, subscript=\orsig]
       \& \theta(w')
    \end{tikzcd}$}%
\end{theorem}
\begin{proof}
  By simultaneous structural induction on the given choreographing derivation, $\chorsig{\theta}{\srsig}{\orsig}$, and ordered rewriting step, $w \reduces_{\srsig} w'$.
  \begin{itemize}
  \item
    Consider the case in which
    \begin{equation*}
      \chorsig{\theta}{\srsig}{\orsig}
      \qquad\text{and}\qquad
      w =
      \infer[\jrule{$\reduces$C}]{w_1 \wc w_0 \wc w_2 \reduces_{\srsig} w_1 \wc w'_0 \wc w_2}{
        w_0 \reduces_{\srsig} w'_0}
      = w'
      \,.
    \end{equation*}
    By the inductive hypothesis, $\theta(w_0) \reduces_{\orsig} \theta(w'_0)$.
    It follows from ordered rewriting's $\jrule{$\reduces$C}$ rule that
    \begin{equation*}
      \theta(w) = \theta(w_1) \oc \theta(w_0) \oc \theta(w_2) \reduces_{\orsig} \theta(w_1) \oc \theta(w'_0) \oc \theta(w_2) = \theta(w')
      \,.
    \end{equation*}

  \item
    Consider the case in which
    \begin{gather*}
      \infer{\chorsig{\theta}{\srsig_0, \bigl(w^L_i \wc a \wc w^R_i \reduces w'_i\bigr)_{i \in \mathcal{I}}}{\orsig_0, \bigl(\defp{a} \defd \bigwith_{i \in \mathcal{I}} \n{A}_i\bigr)}}{
        \begin{array}[b]{@{}c@{}}
          \text{($\theta(a) = \defp{a}$)} \quad
          \chorsig{\theta}{\srsig_0}{\orsig_0} \quad
          \text{($\defp{a} \notin \dom{\orsig_0}$)}
          \\
          \multipremise{i \in \mathcal{I}}{
            \text{$\bigl(\theta(w^L_i) = \atmR{\octx}^L_i\bigr)$} \quad
            \text{$\bigl(\theta(w^R_i) = \atmL{\octx}^R_i\bigr)$} \quad
            \qimp{\atmR{\octx}^L_i}{\up \bigfuse \theta(w'_i)}{\atmL{\octx}^R_i}{\n{A}_i}}
        \end{array}}
    %
    \shortintertext{and}
    %
      w = \infer[\jrule{$\reduces$AX}]{w^L_k \wc a \wc w^R_k \reduces_{\srsig} w'_k}{(w^L_k \wc a \wc w^R_k \reduces w'_k) \in \srsig} = w'
    \end{gather*}
    for some $k \in \mathcal{I}$, where $\srsig = \srsig_0, (w^L_i \wc a \wc w^R_i \reduces w'_i)_{i \in \mathcal{I}}$ and $\orsig = \orsig_0 , (\bigwith_{i \in \mathcal{I}} \n{A}_i)$.

    By \cref{??}, $\lfocus{\theta(w^L_k)}{\n{A}_k}{\theta(w^R_k)}{\bigfuse \theta(w'_k)}$.
    $\lfocus{\theta(w^L_k)}{\bigwith_{i \in \mathcal{I}} \n{A}_i}{\theta(w^R_k)}{\bigfuse \theta(w'_k)}$.
    Because $\rfocus{\theta(w'_k)}{\bigfuse \theta(w'_k)}$~\parencref{??}, it follows by the $\jrule{$\reduces$I}$ rule that $\theta(w^L_k) \oc \bigl(\bigwith_{i \in \mathcal{I}} \n{A}_i\bigr) \oc \theta(w^R_k) \reduces_{\orsig} \theta(w'_k)$, and so $\theta(w) = \theta(w^L_k) \oc \defp{a} \oc \theta(w^R_k) \reduces_{\orsig} \theta(w'_k) = \theta(w')$.

  \item
    Consider the case in which
    \begin{gather*}
      \infer{\chorsig{\theta}{\srsig_0, \bigl(v^L_i \wc a \wc v^R_i \reduces v'_i\bigr)_{i \in \mathcal{I}}}{\orsig_0, \bigl(\defp{a} \defd \bigwith_{i \in \mathcal{I}} \n{A}_i\bigr)}}{
        \begin{array}[b]{@{}c@{}}
          \text{($\theta(a) = \defp{a}$)} \quad
          \chorsig{\theta}{\srsig_0}{\orsig_0} \quad
          \text{($\defp{a} \notin \dom{\orsig_0}$)}
          \\
          \multipremise{i \in \mathcal{I}}{
            \text{$\bigl(\theta(v^L_i) = \atmR{\octx}^L_i\bigr)$} \quad
            \text{$\bigl(\theta(v^R_i) = \atmL{\octx}^R_i\bigr)$} \quad
            \qimp{\atmR{\octx}^L_i}{\up \bigfuse \theta(v'_i)}{\atmL{\octx}^R_i}{\n{A}_i}}
        \end{array}}
    %
    \shortintertext{and}
    %
      \infer[\jrule{$\reduces$AX}]{w \reduces_{\srsig} w'}{
        (w \reduces w') \in \srsig_0}
    \end{gather*}
    where $(w \reduces w') \in \srsig_0$ and $\srsig = \srsig_0, (v^L_i \wc a \wc v^R_i \reduces v'_i)_{i \in \mathcal{I}}$ and $\orsig = \orsig_0 , (\bigwith_{i \in \mathcal{I}} \n{A}_i)$.

    By the inductive hypothesis, $\theta(w) \reduces_{\orsig_0} \theta(w')$.
    It follows from weakening~\parencref{??} that $\theta(w) \reduces_{\orsig} \theta(w')$.    

  \item 
    The case in which
    \begin{equation*}
      \infer{\chorsig{\theta}{\srsige}{\orsige}}{}
      \qquad\text{and}\qquad
      \infer[\jrule{$\reduces$AX}]{w \reduces_{\srsig} w'}{
        (w \reduces w') \in \srsig}
    \end{equation*}
    where $\srsig = \srsige$ and $\orsig = \orsige$ is vacuous.
  % \item
  %   Consider the case in which
  %   \begin{gather*}
  %     \infer{\chorsig{\theta}{\srsig_0, (w_1 \wc a \wc w_2 \reduces w')}{\orsig_0, (\defp{a} \defd \n{A} \with \n{B})}}{
  %       \begin{array}[b]{@{}c@{}}
  %         \text{($\theta(w_1) = \atmR{\octx}_L$)} \quad
  %         \text{($\theta(a) = \defp{a}$)} \quad
  %         \text{($\theta(w_2) = \atmL{\octx}_R$)} \\
  %         \atmR{\octx}_L \limp \up \bigfuse \theta(w') \pmir \atmL{\octx}_R \rightsquigarrow \n{B} \quad
  %         \chorsig{\theta}{\srsig_0}{\orsig_0, (\defp{a} \defd \n{A})}
  %       \end{array}}
  %   \shortintertext{and}
  %     w =
  %     \infer[\jrule{$\reduces$AX}]{w_1 \wc a \wc w_2 \reduces_{\srsig} w'}{}
  %   \end{gather*}
  %   where $\srsig = \srsig_0 , (w_1 \wc a \wc w_2 \reduces w')$ and $\orsig = \orsig_0, (\defp{a} \defd \n{A} \with \n{B})$.

  %   By \cref{lem:chorax-sound-complete}, $\lfocus{\theta(w_1)}{\n{B}}{\theta(w_2)}{\bigfuse \theta(w')}$.
  %   Upon adding the $\lrule{\with}_2$ rule, $\lfocus{\theta(w_1)}{\n{A} \with \n{B}}{\theta(w_2)}{\bigfuse \theta(w')}$.
  %   Because $\rfocus{\theta(w')}{\bigfuse \theta(w')}$~\parencref{??}, it follows by the $\jrule{$\reduces$I}$ rule that $\theta(w_1) \oc (\n{A} \with \n{B}) \oc \theta(w_2) \reduces_{\orsig} \theta(w')$, and so $\theta(w) = \theta(w_1) \oc \defp{a} \oc \theta(w_2) = \theta(w_1) \oc (\n{A} \with \n{B}) \oc \theta(w_2) \reduces_{\orsig} \theta(w')$


  % \item
  %   Consider the case in which
  %   \begin{gather*}
  %     \infer{\chorsig{\theta}{\srsig_0, (w_1 \wc a \wc w_2 \reduces w')}{\orsig_0, (\defp{a} \defd \n{B})}}{
  %       \begin{array}[b]{@{}c@{}}
  %         \text{($\theta(w_1) = \atmR{\octx}_L$)} \quad
  %         \text{($\theta(a) = \defp{a}$)} \quad
  %         \text{($\theta(w_2) = \atmL{\octx}_R$)} \\
  %         \atmR{\octx}_L \limp \up \bigfuse \theta(w') \pmir \atmL{\octx}_R \rightsquigarrow \n{B} \quad
  %         \chorsig{\theta}{\srsig_0}{\orsig_0} \quad
  %         \text{($\defp{a} \notin \dom{\orsig_0}$)}
  %       \end{array}}
  %   \shortintertext{and}
  %     w =
  %     \infer[\jrule{$\reduces$AX}]{w_1 \wc a \wc w_2 \reduces_{\srsig} w'}{}
  %   \end{gather*}
  %   where $\srsig = \srsig_0 , (w_1 \wc a \wc w_2 \reduces w')$ and $\orsig = \orsig_0 , (\defp{a} \defd \n{B})$.

  %   By \cref{lem:chorax-sound-complete}, $\lfocus{\theta(w_1)}{\n{B}}{\theta(w_2)}{\bigfuse \theta(w')}$.
  %   Because $\rfocus{\theta(w')}{\bigfuse \theta(w')}$, it follows by the $\jrule{$\reduces$I}$ rule that $\theta(w_1) \oc \n{B} \oc \theta(w_2) \reduces_{\orsig} \theta(w')$, and so $\theta(w) = \theta(w_1) \oc \defp{a} \oc \theta(w_2) = \theta(w_1) \oc \n{B} \oc \theta(w_2) \reduces_{\orsig} \theta(w')$.
    
  % \item
  %   Consider the case in which
  %   \begin{gather*}
  %     \infer{\chorsig{\theta}{\srsig_0, (v_1 \wc b \wc v_2 \reduces v')}{\orsig_0, (\defp{b} \defd \n{A} \with \n{B})}}{
  %       \begin{array}[b]{@{}c@{}}
  %         \text{($\theta(v_1) = \atmR{\octx}_L$)} \quad
  %         \text{($\theta(b) = \defp{b}$)} \quad
  %         \text{($\theta(v_2) = \atmL{\octx}_R$)} \\
  %         \atmR{\octx}_L \limp \up \bigfuse \theta(v') \pmir \atmL{\octx}_R \rightsquigarrow \n{B} \quad
  %         \chorsig{\theta}{\srsig_0}{\orsig_0, (\defp{b} \defd \n{A})}
  %       \end{array}}
  %   \shortintertext{and}
  %     \infer[\jrule{$\reduces$AX}]{w \reduces_{\srsig} w'}{
  %       w \reduces w' \in \srsig_0}
  %   \end{gather*}
  %   where $\srsig = \srsig_0, (v_1 \wc b \wc v_2 \reduces v')$ and $\orsig = \orsig_0 , (\defp{b} \defd \n{A} \with \n{B})$.

  %   By the inductive hypothesis, $\theta(w) \reduces_{\orsig_0, (\defp{b} \defd \n{A})} \theta(w')$.
  %   It follows from the weakening \lcnamecref{??}~\parencref{??} that $\theta(w) \reduces_{\orsig} \theta(w')$, as required.

  % \item
  %   The case in which
  %   \begin{gather*}
  %     \infer{\chorsig{\theta}{\srsig_0, (v_1 \wc b \wc v_2 \reduces v')}{\orsig_0, (\defp{b} \defd \n{B})}}{
  %       \begin{array}[b]{@{}c@{}}
  %         \text{($\theta(v_1) = \atmR{\octx}_L$)} \quad
  %         \text{($\theta(b) = \defp{b}$)} \quad
  %         \text{($\theta(v_2) = \atmL{\octx}_R$)} \\
  %         \atmR{\octx}_L \limp \up \bigfuse \theta(v') \pmir \atmL{\octx}_R \rightsquigarrow \n{B} \quad
  %         \chorsig{\theta}{\srsig_0}{\orsig_0} \quad
  %         \text{($\defp{b} \notin \dom{\orsig_0}$)}
  %       \end{array}}
  %   \shortintertext{and}
  %     \infer[\jrule{$\reduces$AX}]{w \reduces_{\srsig} w'}{
  %       w \reduces w' \in \srsig_0}
  %   \end{gather*}
  %   where $\srsig = \srsig_0, (v_1 \wc b \wc v_2 \reduces v')$ and $\orsig = \orsig_0 , (\defp{b} \defd \n{B})$ is similar to the previous one.
    %
  \qedhere
  \end{itemize}
\end{proof}




% \begin{lemma}
%   If $\chorax{\theta}{w_1}{\p{A}}{w_2}{\n{B}}$ and $\lfocus{\atmR{\octx}_L}{\n{B}}{\atmL{\octx}_R}{\p{C}}$, then $\atmR{\octx}_L = \theta(w_1)$ and $\atmL{\octx}_R = \theta(w_2)$ and $\p{A} = \p{C}$.
% \end{lemma}
% \begin{proof}
%   By induction over the structure of the given choreographing derivation, $\chorax{\theta}{w_1}{\p{A}}{w_2}{\n{B}}$.
%   \begin{itemize}
%   \item
%     Consider the case in which
%   \begin{equation*}
%     \infer{\chorax{\theta}{\emp}{\p{A}}{\emp}{\up \p{A}}}{}
%     \qquad\text{and}\qquad
%     \lfocus{\atmR{\octx}_1}{\up \p{A}}{\atmL{\octx}_2}{\p{C}}
%     \,.
%   \end{equation*}
%   By inversion on the left-focus derivation, $\atmR{\octx}_L = \octxe = \theta(\emp)$ and $\atmL{\octx}_R = \octxe = \theta(\emp)$, as well as $\p{A} = \p{C}$, as required.

%   \item
%     Consider the case in which
%   \begin{equation*}
%     \infer{\chorax{\theta}{w_1 \oc b}{\p{A}}{w_2}{\atmR{b} \limp \n{B}}}{
%       \text{($\theta(b) = \atmR{b}$)} &
%       \chorax{\theta}{w_1}{\p{A}}{w_2}{\n{B}}}
%     \qquad\text{and}\qquad
%     \lfocus{\atmR{\octx}_L}{\atmR{b} \limp \n{B}}{\atmL{\octx}_R}{\p{C}}
%     \,.
%   \end{equation*}
%   By inversion on the left-focus derivation for $\atmR{b} \limp \n{B}$, there exists $\atmR{\octx}'_1$ such that $\atmR{\octx}_L = \atmR{\octx}'_L \oc \atmR{b}$ and $\lfocus{\atmR{\octx}'_L}{\n{B}}{\atmL{\octx}_R}{\p{C}}$.
%   It follows from the inductive hypothesis that $\atmR{\octx}'_L = \theta(w_1)$ and $\atmL{\octx}_R = \theta(w_2)$ and $\p{A} = \p{C}$.
%   So $\atmR{\octx}_L = \theta(w_1) \oc \atmR{b} = \theta(w_1 \oc b)$.

%   \item
%     The case in which
% \begin{equation*}
%     \infer{\chorax{\theta}{w_1 \oc b}{\p{A}}{w_2}{\n{B} \pmir \atmL{b}}}{
%       \text{($\theta(b) = \atmL{b}$)} &
%       \chorax{\theta}{w_1}{\p{A}}{w_2}{\n{B}}}
%     \qquad\text{and}\qquad
%     \lfocus{\atmR{\octx}_L}{\n{B} \pmir \atmL{b}}{\atmL{\octx}_R}{\p{C}}
%   \end{equation*}
%   is symmetric.
%   \qedhere
%   \end{itemize}
% \end{proof}


\begin{lemma}
  If $\chorsig{\theta}{\srsig}{\orsig}$ and $\lfocus{\atmR{\octx}_L}{\defp{a}}{\atmL{\octx}_R}{_{\orsig} \p{C}}$, then there exists an axiom $(w_1 \oc a \oc w_2 \reduces w') \in \srsig$ such that $\atmR{\octx}_L = \theta(w_1)$, $\atmL{\octx}_R = \theta(w_2)$, and $\p{C} = \bigfuse \theta(w')$.
\end{lemma}
\begin{proof}
  By induction over the structure of the given choreographing derivation, $\chorsig{\theta}{\srsig}{\orsig}$.
  \begin{itemize}
  \item
    Consider the case in which
    \begin{gather*}
      \infer{\chorsig{\theta}{\srsig_0, \bigl(w^L_i \wc a \wc w^R_i \reduces w'_i\bigr)_{i \in \mathcal{I}}}{\orsig_0, \bigl(\defp{a} \defd \bigwith_{i \in \mathcal{I}} \n{A}_i\bigr)}}{
        \begin{array}[b]{@{}c@{}}
          \chorsig{\theta}{\srsig_0}{\orsig_0} \quad
          \text{($\theta(a) = \defp{a}$)} \quad
          \text{($\defp{a} \notin \dom{\orsig_0}$)}
          \\
          \multipremise{i \in \mathcal{I}}{
            \text{$\bigl(\theta(w^L_i) = \atmR{\lctx}^L_i\bigr)$} \quad
            \text{$\bigl(\theta(w^R_i) = \atmL{\lctx}^R_i\bigr)$} \quad
            \qimp{\atmR{\lctx}^L_i}{\up \bigfuse \theta(w'_i)}{\atmL{\lctx}^R_i}{\n{A}_i}}
        \end{array}}
    %
    \shortintertext{and}
    %
      \lfocus{\atmR{\octx}_L}{\defp{a} = \textstyle\bigwith_{i \in \mathcal{I}} \n{A}_i}{\atmL{\octx}_R}{_{\orsig} \p{C}}
    \end{gather*}
    where $\srsig = \srsig_0, \bigl(w^L_i \wc a \wc w^R_i \reduces w'_i\bigr)_{i \in \mathcal{I}}$ and $\orsig = \orsig_0, \bigl(\defp{a} \defd \bigwith_{i \in \mathcal{I}} \n{A}_i\bigr)$.

    By inversion on the left-focus derivation, either: $\lfocus{\atmR{\octx}_L}{\n{A}_k}{\atmL{\octx}_R}{\p{C}}$ for some $k \in \mathcal{I}$; or $\mathcal{I}$ is empty.
    \begin{itemize}
    \item
      If $\lfocus{\atmR{\octx}_L}{\n{A}_k}{\atmL{\octx}_R}{\p{C}}$ for some $k \in \mathcal{I}$, then \cref{??} allows us to conclude that $\atmR{\octx}_L = \atmR{\lctx}^L_k = \theta(w^L_k)$ and $\atmL{\octx}_R = \atmL{\lctx}^R_k = \theta(w^R_k)$ and $\p{C} = \bigfuse \theta(w'_k)$.
      Also, the axiom $w^L_k \wc a \wc w^R_k \reduces w'_k$ is contained in $\srsig$.
    \item
      Otherwise, if $\mathcal{I}$ is empty, then $\bigwith_{i \in \mathcal{I}} \n{A}_i = \top$.
      There is no $\lrule{\top}$ rule to derive $\lfocus{\atmR{\octx}_L}{\defp{a} = \top}{\atmL{\octx}_R}{_{\orsig} \p{C}}$, so this case is vacuous.
    \end{itemize}




  \item
    Consider the case in which
    \begin{gather*}
      \infer{\chorsig{\theta}{\srsig_0, \bigl(v^L_i \wc b \wc v^R_i \reduces v'_i\bigr)_{i \in \mathcal{I}}}{\orsig_0, \bigl(\defp{b} \defd \bigwith_{i \in \mathcal{I}} \n{B}_i\bigr)}}{
        \begin{array}[b]{@{}c@{}}
          \chorsig{\theta}{\srsig_0}{\orsig_0} \quad
          \text{($\theta(b) = \defp{b}$)} \quad
          \text{($\defp{b} \notin \dom{\orsig_0}$)}
          \\
          \multipremise{i \in \mathcal{I}}{
            \text{$\bigl(\theta(v^L_i) = \atmR{\lctx}^L_i\bigr)$} \quad
            \text{$\bigl(\theta(v^R_i) = \atmL{\lctx}^R_i\bigr)$} \quad
            \qimp{\atmR{\lctx}^L_i}{\up \bigfuse \theta(v'_i)}{\atmL{\lctx}^R_i}{\n{B}_i}}
        \end{array}}
    %
    \shortintertext{and}
    %
      \lfocus{\atmR{\octx}_L}{\defp{a}}{\atmL{\octx}_R}{_{\orsig} \p{C}}
    \end{gather*}
    where $a \neq b$ and $\srsig = \srsig_0, \bigl(v^L_i \wc b \wc v^R_i \reduces v'_i\bigr)_{i \in \mathcal{I}}$ and $\orsig = \orsig_0, \bigl(\defp{b} \defd \bigwith_{i \in \mathcal{I}} \n{B}_i\bigr)$.

    By the inductive hypthesis, there exists a string rewriting axiom $(w_1 \wc a \wc w_2 \reduces w') \in \srsig_0$ such that $\atmR{\octx}_L = \theta(w_1)$ and $\atmL{\octx}_R = \theta(w_2)$ and $\p{C} = \bigfuse \theta(w')$.
    The same axiom is contained in the signature $\srsig$.


  \item 
    The case in which
    \begin{equation*}
      \infer{\chorsig{\theta}{\srsige}{\orsige}}{}
      \qquad\text{and}\qquad
      \lfocus{\atmR{\octx}_L}{\defp{a}}{\atmL{\octx}_R}{_{\orsig} \p{C}}
    \end{equation*}
    where $\srsig = \srsige$ and $\orsig = \orsige$ is vacuous because there is no definition for $\defp{a}$ in the signature $\orsig$.


  % \item
  %   Consider the case in which
  %   \begin{gather*}
  %     \infer{\chorsig{\theta}{\srsig_0, (v_1 \wc b \wc v_2 \reduces v')}{\orsig_0, (\defp{b} \defd \n{A} \with \n{B})}}{
  %       \begin{array}[b]{@{}c@{}}
  %         \text{($\theta(v_1) = \atmR{\lctx}_L$)} \quad
  %         \text{($\theta(b) = \defp{b}$)} \quad
  %         \text{($\theta(v_2) = \atmL{\lctx}_R$)} \\
  %         \atmR{\lctx}_L \limp \up \bigfuse \theta(v') \pmir \atmL{\lctx}_R \rightsquigarrow \n{B} \quad
  %         \chorsig{\theta}{\srsig_0}{\orsig_0, (\defp{b} \defd \n{A})}
  %       \end{array}}
  %   \shortintertext{and}
  %     \lfocus{\atmR{\octx}_L}{\defp{a}}{\atmL{\octx}_R}{_{\orsig} \p{C}}
  %   \end{gather*}
  %   where $a \neq b$ and $\srsig = \srsig_0, (v_1 \wc b \wc v_2 \reduces v')$ and $\orsig = \orsig_0, (\defp{b} \defd \n{A} \with \n{B})$.

  %   By the inductive hypothesis, there exists a string rewriting axiom $(w_1 \wc a \wc w_2 \reduces w') \in \srsig_0$ such that $\atmR{\octx}_L = \theta(w_1)$, $\atmL{\octx}_R = \theta(w_2)$, and $\p{C} = \bigfuse \theta(w')$.
  % The same axiom is contained in the signature $\srsig$.

  % \item
  %   Consider the case in which
  %   \begin{gather*}
  %     \infer{\chorsig{\theta}{\srsig_0, (v_1 \wc b \wc v_2 \reduces v')}{\orsig_0, (\defp{b} \defd \n{B})}}{
  %       \begin{array}[b]{@{}c@{}}
  %         \text{($\theta(v_1) = \atmR{\lctx}_L$)} \quad
  %         \text{($\theta(b) = \defp{b}$)} \quad
  %         \text{($\theta(v_2) = \atmL{\lctx}_R$)} \\
  %         \atmR{\lctx}_L \limp \up \bigfuse \theta(v') \pmir \atmL{\lctx}_R \rightsquigarrow \n{B} \quad
  %         \chorsig{\theta}{\srsig_0}{\orsig_0} \quad
  %         \text{($\defp{b} \notin \dom{\orsig_0}$)}
  %       \end{array}}
  %   \shortintertext{and}
  %     \lfocus{\atmR{\octx}_L}{\defp{a}}{\atmL{\octx}_R}{_{\orsig} \p{C}}
  %   \end{gather*}
  %   where $a \neq b$ and $\srsig = \srsig_0, (v_1 \wc b \wc v_2 \reduces v')$ and $\orsig = \orsig_0, (\defp{b} \defd \n{B})$.

  %   By the inductive hypothesis, there exists a string rewriting axiom $(w_1 \wc a \wc w_2 \reduces w') \in \srsig_0$ such that $\atmR{\octx}_L = \theta(w_1)$, $\atmL{\octx}_R = \theta(w_2)$, and $\p{C} = \bigfuse \theta(w')$.
  % The same axiom is contained in the signature $\srsig$.

  % \item
  %   Consider the case in which
  %   \begin{gather*}
  %     \infer{\chorsig{\theta}{\srsig_0, (w_1 \wc a \wc w_2 \reduces w')}{\orsig_0, (\defp{a} \defd \n{A} \with \n{B})}}{
  %       \begin{array}[b]{@{}c@{}}
  %         \text{($\theta(w_1) = \atmR{\lctx}_L$)} \quad
  %         \text{($\theta(a) = \defp{a}$)} \quad
  %         \text{($\theta(w_2) = \atmL{\lctx}_R$)} \\
  %         \atmR{\lctx}_L \limp \up \bigfuse \theta(w') \pmir \atmL{\lctx}_R \rightsquigarrow \n{B} \quad
  %         \chorsig{\theta}{\srsig_0}{\orsig_0, (\defp{a} \defd \n{A})}
  %       \end{array}}
  %   \shortintertext{and}
  %     \infer[\lrule{\with}_2]{\lfocus{\atmR{\octx}_L}{\defp{a} = \n{A} \with \n{B}}{\atmL{\octx}_R}{_{\orsig} \p{C}}}{
  %       \lfocus{\atmR{\octx}_L}{\n{B}}{\atmL{\octx}_R}{_{\orsig} \p{C}}}
  %   \end{gather*}
  %   where $\srsig = \srsig_0, (w_1 \wc a \wc w_2 \reduces w')$ and $\orsig = \orsig_0, (\defp{a} \defd \n{A} \with \n{B})$.

  %   By \cref{??}, $\atmR{\octx}_L = \atmR{\lctx}_L = \theta(w_1)$ and $\atmL{\octx}_R = \atmL{\lctx}_R = \theta(w_2)$ and $\p{C} = \bigfuse \theta(w')$.
  %   And the axiom $w_1 \wc a \wc w_2 \reduces w'$ is contained in the signature $\orsig$.

  % \item
  %   Consider the case in which
  %   \begin{gather*}
  %     \infer{\chorsig{\theta}{\srsig_0, (v_1 \wc a \wc v_2 \reduces v')}{\orsig_0, (\defp{a} \defd \n{A} \with \n{B})}}{
  %       \begin{array}[b]{@{}c@{}}
  %         \text{($\theta(v_1) = \atmR{\lctx}_L$)} \quad
  %         \text{($\theta(a) = \defp{a}$)} \quad
  %         \text{($\theta(v_2) = \atmL{\lctx}_R$)} \\
  %         \atmR{\lctx}_L \limp \up \bigfuse \theta(v') \pmir \atmL{\lctx}_R \rightsquigarrow \n{B} \quad
  %         \chorsig{\theta}{\srsig_0}{\orsig_0, (\defp{a} \defd \n{A})}
  %       \end{array}}
  %   \shortintertext{and}
  %     \infer[\lrule{\with}_1]{\lfocus{\atmR{\octx}_L}{\defp{a} = \n{A} \with \n{B}}{\atmL{\octx}_R}{_{\orsig} \p{C}}}{
  %       \lfocus{\atmR{\octx}_L}{\n{A}}{\atmL{\octx}_R}{_{\orsig} \p{C}}}
  %   \end{gather*}
  %   where $\srsig = \srsig_0, (v_1 \wc a \wc v_2 \reduces v')$ and $\orsig = \orsig_0, (\defp{a} \defd \n{A} \with \n{B})$.

  %   Let $\orsig' = \orsig_0 , (\defp{a} \defd \n{A})$.
  %   Then $\lfocus{\atmR{\octx}_L}{\defp{a} = \n{A}}{\atmL{\octx}_R}{_{\orsig'} \p{C}}$.
  %   By inductive hypothesis, there exists an axiom $(w_1 \wc a \wc w_2 \reduces w') \in \srsig_0$ such that $\atmR{\octx}_L = \theta(w_1)$ and $\atmL{\octx}_R = \theta(w_2)$ and $\p{C} = \bigfuse \theta(w')$.
  %   That same axiom is also contained in the signature $\srsig$.

  % \item
  %   Consider the case in which
  %   \begin{gather*}
  %     \infer{\chorsig{\theta}{\srsig_0, (w_1 \wc a \wc w_2 \reduces w')}{\orsig_0, (\defp{a} \defd \n{B})}}{
  %       \begin{array}[b]{@{}c@{}}
  %         \text{($\theta(w_1) = \atmR{\lctx}_L$)} \quad
  %         \text{($\theta(a) = \defp{a}$)} \quad
  %         \text{($\theta(w_2) = \atmL{\lctx}_R$)} \\
  %         \atmR{\lctx}_L \limp \up \bigfuse \theta(w') \pmir \atmL{\lctx}_R \rightsquigarrow \n{B} \quad
  %         \chorsig{\theta}{\srsig_0}{\orsig_0} \quad
  %         \text{($\defp{a} \notin \dom{\orsig_0}$)}
  %       \end{array}}
  %   \shortintertext{and}
  %     \lfocus{\atmR{\octx}_L}{\defp{a} = \n{B}}{\atmL{\octx}_R}{_{\orsig} \p{C}}
  %   \end{gather*}
  %   where $\srsig = \srsig_0, (w_1 \wc a \wc w_2 \reduces w')$ and $\orsig = \orsig_0, (\defp{a} \defd \n{B})$.

  %   By \cref{??}, $\atmR{\octx}_L = \atmR{\lctx}_L = \theta(w_1)$ and $\atmL{\octx}_R = \atmL{\lctx}_R = \theta(w_2)$ and $\p{C} = \bigfuse \theta(w')$.
  %   And the axiom $w_1 \wc a \wc w_2 \reduces w'$ is contained in the signature $\orsig$.


  % % \item
  % %   The case in which
  % % \begin{gather*}
  % %   \infer{\chorsig{\theta}{\srsig, v_1 \oc b \oc v_2 \reduces v'}{\orsig, \defp{a} \defd \n{A}, \defp{b} \defd \n{B}}}{
  % %     \text{($\theta(b) = \defp{b}$)} &
  % %     \chorax{\theta}{v_1}{\bigfuse \theta(v')}{v_2}{\n{B}} &
  % %     \chorsig{\theta}{\srsig}{\orsig, \defp{a} \defd \n{A}} &
  % %     \text{($\defp{b} \notin \dom{\orsig}$)}}
  % %   \\\text{and}\\
  % %   \lfocus{\atmR{\octx}_L}{\defp{a}}{\atmL{\octx}_R}{\p{C}}
  % % \end{gather*}
  % % is similar.

  % % \item
  % % Consider the case in which
  % % \begin{gather*}
  % %   \infer{\chorsig{\theta}{\srsig, v_1 \oc a \oc v_2 \reduces v'}{\orsig, \defp{a} \defd \n{A}_1 \with \n{A}_2}}{
  % %     \text{($\theta(a) = \defp{a}$)} &
  % %     \chorax{\theta}{v_1}{\bigfuse \theta(v')}{v_2}{\n{A}_2} &
  % %     \chorsig{\theta}{\srsig}{\orsig, \defp{a} \defd \n{A}_1}}
  % %   \\\text{and}\\
  % %   \lfocus{\atmR{\octx}_L}{\defp{a}}{\atmL{\octx}_R}{\p{C}}
  % %   \,.
  % % \end{gather*}
  % % There are two cases, according to whether the $\lfocus{\atmR{\octx}_L}{\defp{a}}{\atmL{\octx}_R}{\p{C}}$ derivation ends with the $\lrule{\with}_1$ or $\lrule{\with}_2$ rule.
  % %   \begin{itemize}
  % %   \item If the left-focus derivation ends with the $\lrule{\with}_2$ rule, then $\lfocus{\atmR{\octx}_L}{\n{A}_2}{\atmL{\octx}_R}{\p{C}}$.
  % %     Because $\chorax{\theta}{v_1}{\bigfuse \theta(v')}{v_2}{\n{A}_2}$, it follows from \cref{??} that $\atmR{\octx}_L = \theta(v_1)$ and $\atmL{\octx}_R = \theta(v_2)$ and $\p{C} = \bigfuse \theta(v')$.
  % %     Choose the axiom $w_1 \oc a \oc w_2 \reduces w'$ to be $v_1 \oc a \oc v_2 \reduces v'$.

  % %   \item Otherwise, if the left-focus derivation instead ends with the $\lrule{\with}_1$ rule, then $\lfocus{\atmR{\octx}_L}{\n{A}_1}{\atmL{\octx}_R}{\p{C}}$.
  % %     By the inductive hypothesis, $\atmR{\octx}_L = \theta(w_1)$, $\atmL{\octx}_R = \theta(w_2)$, and $\p{C} = \bigfuse \theta(w')$ for some string rewriting axiom $(w_1 \oc a \oc w_2 \reduces w') \in \srsig$.
  % %     The same axiom is contained in the signuare $\srsig, v_1 \oc a \oc v_2 \reduces v'$.
  % %   \end{itemize}

  % % \item
  % % Consider the case in which 
  % % \begin{equation*}
  % %   \infer{\chorsig{\theta}{\srsig, w_1 \oc a \oc w_2 \reduces w'}{\orsig, \defp{a} \defd \n{A}}}{
  % %     \text{($\theta(a) = \defp{a}$)} &
  % %     \chorax{\theta}{w_1}{\bigfuse \theta(w')}{w_2}{\n{A}} &
  % %     \chorsig{\theta}{\srsig}{\orsig} &
  % %     \text{($\defp{a} \notin \dom{\orsig}$)}}
  % % \end{equation*}
  % % Because $\chorax{\theta}{w_1}{\bigfuse \theta(w')}{w_2}{\n{A}_2}$, it follows from \cref{??} that $\atmR{\octx}_L = \theta(w_1)$ and $\atmL{\octx}_R = \theta(w_2)$ and $\p{C} = \bigfuse \theta(w')$.
  %
  \qedhere
  \end{itemize}
\end{proof}

\begin{theorem}[Soundness]
  If $\chorsig{\theta}{\srsig}{\orsig}$ and $\theta(a) = \defp{a}$ and $\octx_L \oc \defp{a} \oc \octx_R \reduces_{\orsig} \octx'$, then either:
  \begin{itemize}
  \item $\octx_L = \octx'_L \oc \theta(w_1)$ and $\octx_R = \theta(w_2) \oc \octx'_R$ and $\octx' = \octx'_L \oc \theta(w') \oc \octx'_R$ for some contexts $\octx'_L$ and $\octx'_R$ and some strings $w_1$, $w_2$, and $w'$ such that $(w_1 \wc a \wc w_2 \reduces w') \in \srsig$ and $\lfocus{\theta(w_1)}{\defp{a}}{\theta(w_2)}{\bigfuse \theta(w')}$;
  \item $\octx_L \reduces_{\orsig} \octx'_L$ for some context $\octx'_L$ such that $\octx' = \octx'_L \oc \defp{a} \oc \octx_R$; or
  \item $\octx_R \reduces_{\orsig} \octx'_R$ for some context $\octx'_R$ such that $\octx' = \octx_L \oc \defp{a} \oc \octx'_R$.
  \end{itemize}
\end{theorem}
\begin{proof}
  As a negative proposition, $\defp{a}$ serves as a barrier for interactions between $\octx_L$ and $\octx_R$ -- in \ac{PFOR}, implications cannot consume negative propositions.
  Thus, any reduction on $\octx_L \oc \defp{a} \oc \octx_R$ must occur within either $\octx_L$ or $\octx_R$ alone or must arise from $\defp{a}$.

  If the reduction on $\octx_L \oc \defp{a} \oc \octx_R$ arises from $\defp{a}$, then it arises from a bipole that begins by focusing on $\defp{a}$.
  In other words, $\octx_L = \octx'_L \oc \atmR{\lctx}_L$ and $\octx_R = \atmL{\lctx}_R \oc \octx'_R$ and $\octx' = \octx'_L \oc \lctx' \oc \octx'_R$ for some contexts $\atmR{\lctx}_L$, $\atmL{\lctx}_R$, and $\lctx'$ and positive proposition $\p{C}$ such that $\lfocus{\atmR{\lctx}_L}{\defp{a}}{\atmL{\lctx}_R}{\p{C}}$ and $\rfocus{\lctx'}{\p{C}}$.
  By \cref{??}, there exists an axiom $(w_1 \wc a \wc w_2 \reduces w') \in \srsig$ such that $\atmR{\lctx}_L = \theta(w_1)$ and $\atmL{\lctx}_R = \theta(w_2)$ and $\p{C} = \bigfuse \theta(w')$.
  It follows that $\lctx' = \theta(w')$.
\end{proof}

\begin{corollary}[Soundness]
  If $\chorsig{\theta}{\srsig}{\orsig}$ and $\theta(w) \reduces_{\orsig} \octx'$, then $\octx' = \theta(w')$ for some $w'$ such that $w \reduces_{\srsig} w'$.
\end{corollary}

% \begin{theorem}[Soundness]
%   If $\chorsig{\theta}{\srsig}{\orsig}$ and $\theta(w) \reduces_{\orsig} \octx'$, then $\octx' = \theta(w')$ for some $w'$ such that $w \reduces_{\srsig} w'$.
% \end{theorem}
% \begin{proof}
%   By induction over the structure of the given ordered rewriting step, $\theta(w) \reduces_{\orsig} \octx'$.
%   \begin{itemize}
%   \item Consider the case in which
%     \begin{equation*}
%       \chorsig{\theta}{\srsig}{\orsig}
%       \qquad\text{and}\qquad
%       \theta(w) = 
%       \infer[\jrule{$\reduces$C}\smash{_{\jrule{L}}}]{\octx_1 \oc \octx_2 \reduces_{\orsig} \octx'_1 \oc \octx_2}{
%         \octx_1 \reduces_{\orsig} \octx'_1}
%       = \octx'
%       \,.
%     \end{equation*}
%     By inversion, $w = w_1 \oc w_2$ for some $w_1$ and $w_2$ such that $\octx_1 = \theta(w_1)$ and $\octx_2 = \theta(w_2)$.
%     From the inductive hypothesis, it follows that there exists a string $w'_1$ such that $w_1 \reduces_{\srsig} w'_1$ and $\octx'_1 = \theta(w'_1)$.
%     Let $w' = w'_1 \oc w_2$, and notice that $w = w_1 \oc w_2 \reduces_{\srsig} w'_1 \oc w_2 = w'$ and $\octx' = \theta(w'_1) \oc \theta(w_2) = \theta(w')$, as required.

%   \item
%     The case in which
%     \begin{equation*}
%       \chorsig{\theta}{\srsig}{\orsig}
%       \qquad\text{and}\qquad
%       \theta(w) = 
%       \infer[\jrule{$\reduces$C}\smash{_{\jrule{R}}}]{\octx_1 \oc \octx_2 \reduces_{\orsig} \octx_1 \oc \octx'_2}{
%         \octx_2 \reduces_{\orsig} \octx'_2}
%       = \octx'
%     \end{equation*}
%     is symmetric.

%   \item
%     Consider the case in which 
%     \begin{equation*}
%       \chorsig{\theta}{\srsig}{\orsig}
%       \qquad\text{and}\qquad
%       \theta(w) =
%       \infer{\atmR{\octx}_1 \oc \n{A} \oc \atmL{\octx}_2 \reduces_{\orsig} \octx'}{
%         \lfocus{\atmR{\octx}_1}{\n{A}}{\atmL{\octx}_2}{\p{C}} &
%         \rfocus{\octx'}{\p{C}}}
%       \,.
%     \end{equation*}
%     The image $\theta(w)$ can contain a negative proposition $\n{A}$ only if $w = w_1 \oc a \oc w_2$ for some $w_1$, $a$, and $w_2$ such that $\theta(a) = \defp{a}$ with $(\defp{a} \defd \n{A}) \in \orsig$.
%     By inversion, both $\theta(w_1) = \atmR{\octx}_1$ and $\theta(w_2) = \atmL{\octx}_2$ must hold.
%     It then follows from \cref{??} that $\p{C} = \bigfuse \theta(w')$ for some string $w'$ such that $(w_1 \oc a \oc w_2 \reduces w') \in \srsig$.
%     Because $\rfocus{\octx'}{\bigfuse \theta(w')}$ only if $\octx' = \theta(w')$~\parencref{??}, the string $w'$ is such that $w = w_1 \oc a \oc w_2 \reduces_{\srsig} w'$, with $\octx' = \theta(w')$.
%   %
%   \qedhere
%   \end{itemize}
% \end{proof}


\clearpage
\subsection{No choreography}

Not all string rewriting specifications admit a choreography.
For example, the specification
\begin{equation*}
  \infer{a \oc b \reduces b}{}
  \qquad
  \infer{a \reduces \emp}{}
  \qquad\text{and}\qquad
  \infer{b \reduces \emp}{}
\end{equation*}
cannot be given a choreography.
More precisely, there is no choreographing assignment $\theta$ such that $\chorsig{\theta}{\sig}{\sig'}$ is derivable for some signature $\sig'$.
For the sake of contradiction, suppose that $\theta$ were such a choreographing assignment.
Then, for the specification's latter two axioms to be choreographable, both $\theta(a) = \proc{a}$ and $\theta(b) = \proc{b}$ must hold.
In that case, however, the specification's first axiom cannot be choreographed properly because $\theta$ maps more than one of the axiom's symbols to recursively defined propositions.




\chapter{Bisimilarity for ordered rewriting}\label{ch:ordered-bisimilarity}

With the shift from the global, state-transformation view of ordered rewriting put forth in \cref{ch:ordered-rewriting} to the local, formula-as-process view developed in the preceding \lcnamecref{ch:formula-as-process}, we are now in a position to examine how individual propositions and contexts behave and interact.%
\fixnote{Because each proposition has its own \emph{local} thread of control, we can describe a proposition's behavior only to the extent that that behavior can be witnessed by an external observer.}

According to the formula-as-process view, each proposition has its own \emph{local} thread of control.
This locality means that a proposition's\fixnote{or context's?} behavior can be described only to the extent that that behavior can be witnessed by an external observer.
Locality, together with the local interaction semantics of \cref{??}, also suggests\fixnote{dictates?} what may be observed:
A proposition's structure is opaque; only outward-directed atoms, which are viewed as output messages, are observable.
Thus

% (Implicit in the local interaction semantics of \cref{??} is the idea that atoms ought to be observable but that non-atomic propositions are opaque.
% A proposition's neighbors can discover its behavior only by sending it messages and observing the atoms that it sends as replies.)

Intuitively, for example, the contexts $\atmR{a} \oc (\atmR{a} \limp \atmR{b})$ and $\atmR{b}$ should have the same observable behavior: an internal reduction transforms $\atmR{a} \oc (\atmR{a} \limp \atmR{b})$ into $\atmR{b}$, and no other interactions -- reductions, input transitions, nor output transitions -- are possible from $\atmR{a} \oc (\atmR{a} \limp \atmR{b})$.
In particular, the atom $\atmR{a}$ at the left edge of the context $\atmR{a} \oc (\atmR{a} \limp \atmR{b})$ cannot be observed because it is directed inward and therefore acting as an input to, not an output from, $\atmR{a} \limp \atmR{b}$.

As another example, $\atmR{a} \limp (\atmR{c} \pmir \atmL{b})$ and $(\atmR{a} \limp \atmR{c}) \pmir \atmL{b}$ ought to be behaviorally equivalent, intuitively because they are logically equivalent\footnote{That is, $\oseq{\atmR{a} \limp (\atmR{c} \pmir \atmL{b}) \dashv|- (\atmR{a} \limp \atmR{c}) \pmir \atmL{b}}$.}.

\newthought{Following the vast} literature on various forms of bisimilarity\footnote{See \textcite{??} for a survey.}, we will develop a notion of \emph{ordered rewriting bisimilarity} on ordered contexts, $\octx \osim \lctx$.
In \cref{sec:??}, we define rewriting bisimilarity and present a few examples of contexts that are \emph{not} bisimilar under this definition.

However, it will prove to be cumbersome to use rewriting bisimilarity's definition alone to establish that two contexts are bisimilar, a problem that is familiar from process calculi bisimilarities.
Therefore, in \cref{sec:ordered-bisimilarity:labeled-bisim}, we present \emph{labeled bisimilarity}, a sound proof technique for rewriting bisimilarity -- ordered contexts that are labeled bisimilar will also be rewriting bisimilar.
Unlike the $\pi$-calculus's labeled bisimilarity, our labeled bisimilarity is surprisingly complete for rewriting bisimilarity -- contexts that are rewriting bisimilar will also be labeled bisimilar.

This \lcnamecref{ch:ordered-bisimilarity} concludes with several applications of ordered rewriting bisimilarity to our now-familiar running examples, binary counters and \aclp*{NFA}.
\Cref{sec:ordered-bisimilarity:nfa} proves that \ac{NFA}-bisimilar states have rewriting-bisimilar encodings under the functional choreography described in \cref{sec:formula-as-process:nfa-functional}, which allows us to finally rephrase the choreography's adequacy in a clean and stratified form.
And \cref{sec:ordered-bisimilarity:counter} proves that binary counters are rewriting-bisimilar if, and only if, they have the same denotation.

Our rewriting bisimilarity is strongly inspired by the contextual bisimilarity for linear contexts put forth by \textcite{Deng+:MSCS16}.


\section{Toss?}

The previous \lcnamecref{ch:formula-as-process} introduced a formula-as-process view of the focused ordered rewriting framework, and showed how to use it to provide local, message-passing choreographies of the global, string rewriting specifications seen in \cref{ch:string-rewriting}.
The formula-as-process ordered rewriting framework brings (focused) ordered rewriting more in line with process calculi, such as the $\pi$-calculus.

Much more than functional programming, process calculi emphasize observational equivalence.
Owing to their distributed nature, processes are opaque -- only messages are observable.
A client can't peer inside a process to see its internal structure;
a client can only send the process messages and observe any messages that the process sends as a reply.

So far, the formula-as-process ordered rewriting framework does not enjoy a notion of observational equivalence.
In this \lcnamecref{ch:ordered-bisimilarity}, we resolve that deficiency.
In keeping with the large body of work on bisimilarity for message-passing processes\autocite{??}, we develop a notion of bisimilarity for ordered contexts.
Two contexts are judged to be bisimilar if there is no experiment that will eventually distinguish them.

In process calculi, processes are usually treated as opaque, being defined by their behavior rather than their internal structure.
(This emphasis differs from that of functional programming.)
A client can only observe

But process calculi are usually equipped with a notion of observational equivalence, which formula-as-process ordered rewriting does not yet enjoy.
In this \lcnamecref{ch:ordered-bisimilarity}, we develop an observational equivalence for ordered rewriting.
% This \lcnamecref{ch:ordered-bisimilarity} explores the question of when two propositions have equivalent behavior under this process-as-formula view.
In keeping with the large body of work on bisimilarity for message-passing processes\autocite{??}, we develop a notion of bisimilarity for ordered propositions.
This \vocab{ordered rewriting bisimilarity} treats the atomic propositions as the sole observables, in keeping with their interpretation as messages.

Messages are observable, but processes are opaque.



\section{Ordered rewriting bisimilarity}

% An ordered context $\octx$ may be composed when surrounded by ordered contexts $\octx_L$ and $\octx_R$.

% Thus, in $\octx_L \oc \octx \oc \octx_R$, we view $\octx$ as existing within the environment formed by its surrounding contexts, $\octx_L$ and $\octx_R$.
% The context $\octx$ then interacts with that environment along two interfaces: the left end of $\octx$ may interact with the right end of $\octx_L$, and, symmetrically, the right end of $\octx$ may interact with the left end of $\octx_R$.

An atom's location and direction are crucial to its observability.
For an atom to be observable, it must be possible for an external observer to receive that atom as a message.
In $\atmL{a} \oc \octx$ and $\octx \oc \atmR{b}$, the atoms $\atmL{a}$ and $\atmR{b}$, respectively, are observable, because $\octx_O \oc (\p{A} \pmir \atmL{a}) \oc \atmL{a} \oc \octx \reduces \octx_O \oc \p{A} \oc \octx$

But those same atoms are not observable in $\octx \oc \atmL{a}$ and $\atmR{b} \oc \octx$.

\begin{theorem}
  If $\octx = \atmL{a} \oc \octx_0 \reduces \octx'$, then $\octx' = \atmL{a} \oc \octx'_0$ for some $\octx'_0$ such that $\octx_0 \reduces \octx'_0$.
  Symmetrically, if $\octx = \octx_0 \oc \atmR{a} \reduces \octx'$, then $\octx' = \octx'_0 \oc \atmR{a}$ for some $\octx'_0$ such that $\octx_0 \reduces \octx'_0$.
\end{theorem}
\begin{proof}
  By inversion on the given reduction, making use of the fact that $\atmL{a} \limp \n{B}$ and $\n{B} \pmir \atmR{a}$ are not well-formed propositions.
\end{proof}

\newthought{Ordered rewriting} is asynchronous.
As a simple example, it takes two steps to achieve $\octx_L \oc (\up \p{A} \pmir \atmL{a}) \oc (\atmL{a} \fuse \p{B}) \oc \octx_R \Reduces \octx_L \oc \p{A} \oc \p{B} \oc \octx_R$:
first, $\atmL{a} \fuse \p{B}$ is decomposed into $\atmL{a}$ and $\p{B}$, and then that $\atmL{a}$ is used to decompose $\up \p{A} \pmir \atmL{a}$ into $\p{A}$.
The entire rewriting cannot occur in a single, synchronous step:
\begin{equation*}
  \octx_L \oc (\up \p{A} \pmir \atmL{a}) \oc (\atmL{a} \fuse \p{B}) \oc \octx_R
    \nreduces \octx_L \oc \p{A} \oc \p{B} \oc \octx_R
  \,.
\end{equation*}

Because ordered rewriting is asynchronous in this way, we should expect our notion of rewriting bisimilarity to have some analogy to the kinds of bisimilarities developed for asynchronous process calculi, such as the asynchronous $\pi$-calculus\autocite{Amadio+:TCS98} and asynchronous \acs*{CCS}\autocite{Boreale+:IC02}.

% \subsection{}

% Because outgoing atoms are observable at a context's edges, there is a built-in notion of (immediate) output transition: a context $\octx$ outputs $\atmL{a}$ to its left exactly when $\octx = \atmL{a} \oc \octx'$, for some $\octx'$.
% Symmetrically, a context $\octx$ outputs $\atmR{b}$ to its right exactly when $\octx = \octx' \oc \atmR{b}$.
% We could adopt a process-calculus--like labeled transition notation for these output transitions -- such as $\octx = \atmL{a} \oc \octx' \reduces[\atmL{a}] \octx'$ and $\octx = \octx' \oc \atmR{a} \reduces[\atmR{a}] \octx'$ -- but that 

% A weak output transition would then be
% So, in this setting, $\octx$ would have a weak output transition to $\octx'$ if there exists a context $\octx_0$ such that $\octx \Reduces \atmL{a} \oc \octx_0$ and $\octx_0 \Reduces \octx'$ -- or, more simply, if $\octx \Reduces \atmL{a} \oc \octx'$.



% \section{}

% This \lcnamecref{ch:ordered-bisimilarity} marks a change in our perspective on ordered rewriting.
% In the previous \lcnamecref{ch:ordered-rewriting}, we viewed ordered rewriting as an abstract framework for global specifications of concurrent systems, in the vein of previous work on [...].
% The emphasis was placed squarely on state transformation [...].

% Although useful for reasoning about abstract properties of concurrent systems, these global specifications do not immediately suggest [...].
% Therefore, in this [...], we instead refine ordered rewriting into a framework for message-passing concurrency among processes with independent threads of control.

% This message-passing view is obtained through a \vocab{process-as-formula}\autocite{??} reading of ordered propositions and contexts.
% The logical connectives are reinterpreted as process constructors, so that propositions are seen as processes; positive atomic propositions, as messages; and contexts, as process configurations.

% In this \lcnamecref{ch:ordered-bisimilarity}, we would instead like to decrease the level of abstraction and view ordered rewriting as a framework for message-passing concurrency.

%  -- specifically, message-passing among processes arranged in a chain\fixnote{linear?} topology.
% With their independent threads of control, processes bring a more local character to ordered rewriting, bringing it closer to a process calculus such as the $\pi$-calculus.

% This message-passing view is obtained through a \vocab{process-as-formula}\autocite{??} view of ordered propositions and contexts.
% The logical connectives are reinterpreted as process constructors, so that propositions are seen as processes; positive atomic propositions, as messages; and contexts, as process configurations.



% This \lcnamecref{ch:ordered-bisimilarity} marks a change in our perspective on ordered rewriting.
% In the previous \lcnamecref{ch:ordered-rewriting}, we viewed ordered rewriting as an abstract framework for concurrent state transformation, in the vein of previous work on multiset rewriting\autocite{??} [or even Petri nets\autocite{??}].
% With the emphasis on transformation of the entire state, our view of concurrent computation was inherently global.

% In this \lcnamecref{ch:ordered-bisimilarity}, we would instead like to view ordered rewriting as a framework for message-passing concurrency -- specifically, message-passing among processes arranged in a chain\fixnote{linear?} topology.
% With their independent threads of control, processes bring a more local character to ordered rewriting, bringing it closer to a process calculus such as the $\pi$-calculus.

% This message-passing view is obtained through a \vocab{process-as-formula}\autocite{??} view of ordered propositions and contexts.
% The logical connectives are reinterpreted as process constructors, so that propositions are seen as processes; positive atomic propositions, as messages; and contexts, as process configurations.


% % negative propositions can be seen as processes, positive atoms can be seen as messages, and ordered contexts can be seen as process configurations.

% % Accordingly, we take a \vocab{process-as-formula}\autocite{??} view, in which [...].
% % Specifically, uninterpreted positive atoms will act like messages and negative propositions will act like processes.
% % Owing to the conceptual introduction of processes with independent threads of control, this view of concurrency is more local in character and brings ordered rewriting closer to a process calculus, such as the $\pi$-calculus.

% Interestingly, this change in perspective necessitates very few formal changes to the ordered rewriting framework.
% The primary change is that left- and right-handed implications are restricted to positive atoms, corresponding to the common first-order restriction that input processes receive only messages.

% Despite the few formal changes, the new local [, message-passing] perspective does raise a new, important question: when do two processes have equivalent behavior?
% In keeping with the large body of work on bisimilarity\autocite{??}, we develop a notion of bisimilarity between ordered contexts.
% Several examples [...]

% Despite requiring only very few formal changes, the shift from global to local perspective does raise an important question: when do two processes have equivalent behavior?
% We answer this question by developing a notion of bisimilarity for ordered contexts.
% In keeping with the large body of work on bisimilarity\autocite{??}, we develop a notion of bisimilarity between ordered contexts.

% \begin{itemize}
% \item Assign direction to uninterpreted atoms so they act like messagesl
% \item Defined atoms are like processes
% \item Left and right implications are restricted to messages (compare with higher-order $\pi$ calculus)
% \end{itemize}


% \section{}

% \begin{equation*}
%   a \oc \dfa{q} \reduces \dfa{q}'_a
% \end{equation*}
% where $q \dfareduces[a] q'_a$, for each pair $(q, a) \in Q \times \ialph$; and 
% \begin{equation*}
%   \emp \oc \dfa{q} \reduces
%     \begin{cases*}
%       \one & if $q \in F$ \\
%       \top & if $q \notin F$
%     \end{cases*}
% \end{equation*}
% for each $q \in Q$.

% As a specification of \acp{DFA}, this works well.
% But as an implementation, it is significantly lacking.
% The rewriting axioms $a \oc \dfa{q} \reduces \dfa{q}'_a$ presume that a conductor orchestrates the interactions between input symbols and \ac{DFA} states, but a local\fixnote{distributed?} implementation 

% This specification could be choreographed in (at least) two ways.
% One choreography treats the input symbols $a$ as messages that are received by the states $\dfa{q}$, acting as processes.
% \begin{equation*}
%   \dfa{q} \defd (\emp \limp \dfa{F}(q)) \with \bigwith_{a \in \ialph}(\atmR{a} \limp \dfa{q}'_a)
% \end{equation*}
% Because the input word is delivered like data to the state, this choreography has a functional flavor.

% Another choreography of the same specification is dual, treating the input symbols as processes
% \begin{equation*}
%   a \defd \bigwith_{q \in Q}(\atmL{q}'_a \pmir \atmL{q})
%   \quad\text{and}\quad
%   \emp \defd \bigwith_{q \in Q}(\dfa{F}(q) \pmir \atmL{q})
% \end{equation*}

% \section{}

% To interpret polarized ordered propositions as processes, we adapt the \vocab{process-as-formula} view of logical connectives initiated by \textcite{??}.
% The logical connectives are read as process constructors, so that positive atomic propositions may be seen as messages; negative propositions, [may be seen] as processes; ordered contexts, [may be seen] as process configurations with a chain topology; and positive propositions, [may be seen] as processes that reify those configurations.

% To keep the interpretation as simple as possible, we introduce three syntactic restrictions on the ordered propositions.
% Each of these restrictions may be relaxed at the expense of some additional complexity, as we will discuss in \cref{??}.

% First, each positive atom is consistently assigned a direction, either left-directed, $\atmL{a}$, or right-directed, $\atmR{a}$.
% When positive atoms are viewed as messages, these directions indicate the message's sender and intended recipient.
% For example, in the context $\dn \n{C} \oc \atmL{a} \oc \p{B}$, the right-to-left direction of $\atmL{a}$ incicates that $\p{B}$ was the sender and $\dn \n{C}$ is the intended recipient.



% Second, recursively defined \emph{positive} propositions are disallowed.\fixnote{is this necessary?}

% Third, the left- and right-handed implications are restricted to accept only atoms with an incoming direction: $\atmR{a} \limp \n{B}$ and $\n{B} \pmir \atmL{a}$.
% [In conjunction with atoms' directions,] this acts as a mild form of typing -- an input process may receive only intended messages.
% Something like $\atmL{a} \oc (a \limp \up \p{B}) \reduces \p{B}$ should \emph{not} be possible, because its process-as-formula reading


% \begin{syntax*}
%   Ordered contexts &
%     \octx & \octx_1 \oc \octx_2 \mid \octxe \mid \p{A}
% \end{syntax*}
% Concatenation of contexts, $\octx_1 \oc \octx_2$, is viewed as end-to-end composition of process configurations;
% the empty context, $\octxe$, is the empty process configuration;
% and [...].

% To keep the interpretation as simple as possible, we introduce three syntactic restrictions on propositions.
% Each of these restrictions may be relaxed at the expense of some additional complexity, as we will discuss in \cref{??}.

% First, each positive atom is consistently assigned a direction, either left-directed, $\atmL{a}$, or right-directed, $\atmR{a}$.
% Because 
% Second, recursively defined \emph{positive} propositions are disallowed.
% Thus, the positive propositions are generated by the following grammar.
% \begin{syntax*}
%   Positive props. &
%     \p{A} & \atmL{a} \mid \atmR{a} \mid \p{A} \fuse \p{B} \mid \one \mid \dn \n{A}
% \end{syntax*}
% Atoms $\atmL{a}$ and $\atmR{a}$ are viewed as left- and right-directed messages; ordered conjunction, $\p{A} \fuse \p{B}$, denotes end-to-end composition of processes $\p{A}$ and $\p{B}$; $\one$ denotes the terminating 

% \begin{tabular}{@{}rl@{}}
%   $\octx_1 \oc \octx_2$ & end-to-end composition of configurations \\
%   $\octxe$ & empty configuration \\
%   $\p{A}$ & 
% \end{tabular}

% \begin{tabular}{@{}rl@{}}
%   $\atmR{a}$ & right-directed message \\
%   $\atmL{a}$ & left-directed message \\
%   $\p{A} \fuse \p{B}$ & process composition \\
%   $\one$ & terminating process \\
%   $\dn \n{A}$ & 
% \end{tabular}

% \begin{tabular}{@{}rl@{}}
%   $\n{\alpha} \defd \n{A}$ & recursively defined process \\
%   $\atmR{a} \limp \n{B}$ & receive $\atmR{a}$ from the left, then continue as $\n{B}$ \\
%   $\n{B} \pmir \atmL{a}$ & receive $\atmL{a}$ from the right, then continue as $\n{B}$ \\
%   $\n{A} \with \n{B}$ & nondeterministically choose to continue as $\n{A}$ or $\n{B}$ \\
%   $\top$ & \\
%   $\up \p{A}$ &
% \end{tabular}

% \begin{tabular}{@{}rl@{}}
%   $\octx_1 \oc \octx_2$ & composition of configurations $\octx_1$ and $\octx_2$ \\
%   $\octxe$ & empty configuration \\
%   $\p{A}$ & single process configuration
% \end{tabular}

% \begin{syntax*}
%   Positive props. &
%     \p{A} & \atmR{a} \mid \atmL{a} \mid \p{A} \fuse \p{B} \mid \one \mid \dn \n{A}
% \end{syntax*}

% Atoms' directions act as a very mild form of typing.
% The left- and right-handed implications are restricted to accept only atoms with an incoming direction: $\atmR{a} \limp \n{B}$ and $\n{B} \pmir \atmL{a}$.
% The full syntax of negative propositions is thus:
% \begin{syntax*}
%   Negative props. &
%     \n{A} & \n{\alpha} \mid \atmR{a} \limp \n{B} \mid \n{B} \pmir \atmL{a} \mid \n{A} \with \n{B} \mid \top \mid \up \p{A}
%   \,,
% \end{syntax*}
% with equirecursively defined negative propositions $\n{\alpha} \defd \n{A}$.


%  acting like recursively defined processes.

% In addition to fully general ordered contexts of positive propositions, it will also be useful to characterize two refinements: contexts that contain only atoms of one direction or the other.
% We use an arrow decoration to indicate the direction.
% \begin{syntax*}
%   Ordered contexts &
%     \octx & \octx_1 \oc \octx_2 \mid \octxe \mid \p{A}
%   \\[-2\jot]
%   Right-directed &
%     \atmR{\octx} & \atmR{\octx}_1 \oc \atmR{\octx}_2 \mid \octxe \mid \atmR{a}
%   \\[-2\jot]
%   Left-directed &
%     \atmL{\octx} & \atmL{\octx}_1 \oc \atmL{\octx}_2 \mid \octxe \mid \atmL{a}
% \end{syntax*}

% Having restricted the premises of left- and right-handed implications to incoming atoms, $\atmR{a}$ and $\atmL{a}$, respectively, the left focus judgment and its rules may be refined.
% The judgment is now $\lfocus{\atmR{\octx}_L}{\n{A}}{\atmL{\octx}_R}{\p{C}}$, because [...inputs can only be incoming messages...].
% Other than this refinement, the inference rules remain essentially the same as in \cref{??}.
% The revised 


% \begin{figure}
%   \begin{syntax*}
%     Positive props. &
%       \p{A} & \atmR{a} \mid \atmL{a} \mid \p{A} \fuse \p{B} \mid \one \mid \dn \n{A}
%     \\
%     Negative props. &
%       \n{A} & \n{\alpha} \mid
%                 \atmR{a} \limp \n{B} \mid \n{B} \pmir \atmL{a} \mid
%                 \n{A} \with \n{B} \mid \top \mid \up \p{A}
%     \\
%     Ordered contexts &
%         \octx & \octx_1 \oc \octx_2 \mid \octxe \mid \p{A} \\[-2\jot]
%       & \atmR{\octx} & \atmR{\octx}_1 \oc \atmR{\octx}_2 \mid \octxe \mid \atmR{a} \\[-2\jot]
%       & \atmL{\octx} & \atmL{\octx}_1 \oc \atmL{\octx}_2 \mid \octxe \mid \atmL{a}
%   \end{syntax*}
%   \begin{inferences}[Rewriting: $\octx \reduces \octx'$ and $\octx \Reduces \octx'$]
%     \infer[\jrule{$\dn$D}]{\atmR{\octx}_L \oc \dn \n{A} \oc \atmL{\octx}_R \reduces \p{C}}{
%       \lfocus{\atmR{\octx}_L}{\n{A}}{\atmL{\octx}_R}{\p{C}}}
%     \and
%     \infer[\jrule{$\fuse$D}]{\p{A} \fuse \p{B} \reduces \p{A} \oc \p{B}}{}
%     \and
%     \infer[\jrule{$\one$D}]{\one \reduces \octxe}{}
%     \\
%     \text{(no $\jrule{$\plus$D}$ and $\jrule{$\zero$D}$ rules)}
%     \\
%     \infer[\jrule{$\reduces$C}_{\jrule{L}}]{\octx_1 \oc \octx_2 \reduces \octx'_1 \oc \octx_2}{
%       \octx_1 \reduces \octx'_1}
%     \and
%     \infer[\jrule{$\reduces$C}_{\jrule{R}}]{\octx_1 \oc \octx_2 \reduces \octx'_1 \oc \octx_2}{
%       \octx_1 \reduces \octx'_1}
%   \end{inferences}
%   \begin{inferences}
%     \infer[\jrule{$\Reduces$R}]{\octx \Reduces \octx}{}
%     \and
%     \infer[\jrule{$\Reduces$T}]{\octx \Reduces \octx''}{
%       \octx \reduces \octx' & \octx' \Reduces \octx''}
%   \end{inferences}

%   \begin{inferences}[Left focus: $\lfocus{\atmR{\octx}_L}{\n{A}}{\atmL{\octx}_R}{\p{C}}$]
%     \infer[\lrule{\limp}']{\lfocus{\atmR{\octx}_L}{\atmR{a} \limp \n{B}}{\atmL{\octx}_R}{\p{C}}}{
%       \lfocus{\atmR{\octx}_L \oc \atmR{a}}{\n{B}}{\atmL{\octx}_R}{\p{C}}}
%     \and
%     \infer[\lrule{\pmir}']{\lfocus{\atmR{\octx}_L}{\n{B} \pmir \atmL{a}}{\atmL{\octx}_R}{\p{C}}}{
%       \lfocus{\atmR{\octx}_L}{\n{B}}{\atmL{a} \oc \atmL{\octx}_R}{\p{C}}}
%     \\
%     \infer[\lrule{\with}_1]{\lfocus{\atmR{\octx}_L}{\n{A} \with \n{B}}{\atmL{\octx}_R}{\p{C}}}{
%       \lfocus{\atmR{\octx}_L}{\n{A}}{\atmL{\octx}_R}{\p{C}}}
%     \and
%     \infer[\lrule{\with}_2]{\lfocus{\atmR{\octx}_L}{\n{A} \with \n{B}}{\atmL{\octx}_R}{\p{C}}}{
%       \lfocus{\atmR{\octx}_L}{\n{B}}{\atmL{\octx}_R}{\p{C}}}
%     \and
%     \text{(no $\lrule{\top}$ rule)}
%     \\
%     \infer[\lrule{\up}]{\lfocus{}{\up \p{A}}{}{\p{A}}}{}
%   \end{inferences}
%   \caption{A weakly focused ordered rewriting framework}
% \end{figure}

% \begin{figure}
%  \begin{inferences}[Input transition: $\ireduces{\atmR{\octx}_L \oc ##1 \oc \atmL{\octx}_R}{\octx}{\octx'}$]
%     \infer{\ireduces{\atmR{\octx}_L \oc #1 \oc \atmL{\octx}_R}{\dn \n{A}}{\p{C}}}{
%       \lfocus{\atmR{\octx}_L}{\n{A}}{\atmL{\octx}_R}{\p{C}}}
%     \and
%     \infer{\ireduces{\atmR{\octx}_L \oc #1 \oc \atmL{\octx}_R}{\atmR{a} \oc \octx}{\octx'}}{
%       \ireduces{\atmR{\octx}_L \oc \atmR{a} \oc #1 \oc \atmL{\octx}_R}{\octx}{\octx'}}
%     \and
%     \infer{\ireduces{\atmR{\octx}_L \oc #1 \oc \atmL{\octx}_R}{\octx \oc \atmL{a}}{\octx'}}{
%       \ireduces{\atmR{\octx}_L \oc #1 \oc \atmL{a} \oc \atmL{\octx}_R}{\octx}{\octx'}}
%     \\
%     \infer{\ireduces{#1 \oc \atmL{\octx}_R}{\p{A} \oc \octx}{\p{A} \oc \octx'}}{
%       \ireduces{#1 \oc \atmL{\octx}_R}{\octx}{\octx'}}
%     \and
%     \infer{\ireduces{\atmR{\octx}_L \oc #1}{\octx \oc \p{A}}{\octx' \oc \p{A}}}{
%       \ireduces{\atmR{\octx}_L \oc #1}{\octx}{\octx'}}
%   \end{inferences}
%   \caption{A weakly focused ordered rewriting framework}
% \end{figure}


% \section{Input transitions}

% With the above restriction of left- and right-handed implications to atomic premises of hte 

% \begin{inferences}
%   \infer{\ireduces{\atmR{\octx}_L \oc #1 \oc \atmL{\octx}_R}{\dn \n{A}}{\p{C}}}{
%     \lfocus{\atmR{\octx}_L}{\n{A}}{\atmL{\octx}_R}{\p{C}}}
%   \\
%   \infer{\ireduces{\atmR{\octx}_L \oc #1 \oc \atmL{\octx}_R}{\atmR{a} \oc \octx}{\octx'}}{
%     \ireduces{\atmR{\octx}_L \oc \atmR{a} \oc #1 \oc \atmL{\octx}_R}{\octx}{\octx'}}
%   \and
%   \infer{\ireduces{\atmR{\octx}_L \oc #1 \oc \atmL{\octx}_R}{\octx \oc \atmL{a}}{\octx'}}{
%     \ireduces{\atmR{\octx}_L \oc #1 \oc \atmL{a} \oc \atmL{\octx}_R}{\octx}{\octx'}}
%   \\
%   \infer{\ireduces{#1 \oc \atmL{\octx}_R}{\p{A} \oc \octx}{\p{A} \oc \octx'}}{
%     \ireduces{#1 \oc \atmL{\octx}_R}{\octx}{\octx'}}
%   \and
%   \infer{\ireduces{\atmR{\octx}_L \oc #1}{\octx \oc \p{A}}{\octx' \oc \p{A}}}{
%     \ireduces{\atmR{\octx}_L \oc #1}{\octx}{\octx'}}
% \end{inferences}

% In its most basic form, an input transition derives from the inputs required by [...].

% The following \lcnamecref{thm:input-transition-reduction} relates input transitions to reductions.
% \begin{theorem}\label{thm:input-transition-reduction}
%   If $\ireduces{\atmR{\octx}_L \oc #1 \oc \atmL{\octx}_R}{\octx}{\octx'}$, then $\atmR{\octx}_L \oc \octx \oc \atmL{\octx}_R \reduces \octx'$.
%   Conversely, if $\octx \reduces \octx'$, then there exist $\octx_L$ and $\octx_R$ such that either:
%   \begin{itemize}
%   \item $\octx = \octx_L \oc \atmR{\lctx}_L \oc \octx_0 \oc \atmL{\lctx}_R \oc \octx_R$ and $\ireduces{\atmR{\lctx}_L \oc #1 \oc \atmL{\lctx}_R}{\octx_0}{\octx'_0}$ and $\octx' = \octx_L \oc \octx'_0 \oc \octx_R$, for some $\atmR{\lctx}_L$, $\octx_0$, $\atmL{\lctx}_R$, and $\octx'_0$;
%   \item $\octx = \octx_L \oc (\p{A} \fuse \p{B}) \oc \octx_R$ and $\octx' = \octx_L \oc \p{A} \oc \p{B} \oc \octx_R$, for some $\p{A}$ and $\p{B}$; or
%   \item $\octx = \octx_L \oc \one \oc \octx_R$ and $\octx' = \octx_L \oc \octx_R$.
%   \end{itemize}
% \end{theorem}
% \begin{proof}
%   By structural induction on the given input transition or reduction, respectively.
% \end{proof}

% \begin{lemma}\label{lem:input-framing}
%   If $\ireduces{\atmR{\lctx}_L \oc #1 \oc \atmL{\lctx}_R}{\atmR{a} \oc \octx}{\octx'}$, then either:
%   \begin{itemize}[nosep]
%   \item $\atmR{a}$ satisfies an input demand -- \ie, $\ireduces{\atmR{\lctx}_L \oc \atmR{a} \oc #1 \oc \atmL{\lctx}_R}{\octx}{\octx'}$; or
%   \item $\atmR{a}$ does not participate in the input transition -- \ie, $\atmR{\lctx}_L = \octxe$ and $\octx' = \atmR{a} \oc \octx'_a$ for some $\octx'_a$ such that $\ireduces{#1 \oc \atmL{\lctx}_R}{\octx}{\octx'_a}$.
%   \end{itemize}
%   Symmetrically, if $\ireduces{\atmR{\lctx}_L \oc #1 \oc \atmL{\lctx}_R}{\octx \oc \atmL{a}}{\octx'}$, then either:
%   \begin{itemize}[nosep]
%   \item $\atmL{a}$ satisfies an input demand -- \ie, $\ireduces{\atmR{\lctx}_L \oc #1 \oc \atmL{a} \oc \atmL{\lctx}_R}{\octx}{\octx'}$; or
%   \item $\atmL{a}$ does not participate in the input transition -- \ie, $\atmL{\lctx}_R = \octxe$ and $\octx' = \octx'_a \oc \atmL{a}$ for some $\octx'_a$ such that $\ireduces{\atmR{\lctx}_L \oc #1}{\octx}{\octx'_a}$.
%   \end{itemize}
% \end{lemma}
% \begin{proof}
%   By structural induction on the given input transition.
% \end{proof}

\section{Rewriting bisimilarity}\label{sec:ordered-bisimilarity:rewriting-bisimilarity}

With the shift from a global, state transformation view of ordered rewriting to a local, \enquote{formula-as-process} view, it is now possible to consider how the individual \enquote{formula-as-process} -- or, more generally, \enquote{context-as-configuration} -- components behave and how they interact with each other.
Because each component has its own, local thread of control, we can describe its behavior only to the extent that its behavior is observable.
To the extent that its behavior can be witnessed by an external observer

Now that we can decompose concurrent systems into individual \enquote{formula-as-process} -- or \enquote{context-as-configuration} -- components, 

Intuitively, for example, the contexts\fixnote{configurations?} $\atmR{a} \oc (\atmR{a} \limp \atmR{b})$ and $\atmR{b}$ should be behaviorally equivalent: an internal reduction transforms $\atmR{a} \oc (\atmR{a} \limp \atmR{b})$ into $\atmR{b}$, and no other interactions -- reductions or input or output transitions -- are possible from $\atmR{a} \oc (\atmR{a} \limp \atmR{b})$.
As another example, $\atmR{a} \limp (\atmR{c} \pmir \atmL{b})$ and $(\atmR{a} \limp \atmR{c}) \pmir \atmL{b}$ should also be behaviorally equivalent, intuitively because they are logically equivalent.

Following the vast literature on various forms of bisimilarity\footnote{See \textcite{??} for a survey.}, we will develop a notion of \vocab{rewriting bisimilarity} on ordered contexts.
First, we need a few auxiliary definitions.

related to Deng et al.



In this \lcnamecref{sec:ordered-bisimilarity:rewriting-bisimilarity}, we define \emph{ordered rewriting bisimilarity}, a notion of observational equivalence for ordered contexts.
Before that, however, we must define a few auxiliary relations on contexts.
%
\begin{definition}[Framed binary relations]\label{def:ordered-bisimilarity:framed-relation}
  Let $\simu{R}$ be a binary relation on ordered contexts.
  Given ordered contexts $\lctx_L$ and $\lctx_R$, let $\lrframe{\lctx_L}{\simu{R}}{\lctx_R}$ be the least binary relation such that:
  \begin{inferences}
    \infer{\octx \lrframe{\lctx_L}{\simu{R}}{\lctx_R} \octx'}{
      \text{($\octx = \lctx_L \oc \octx_0 \oc \lctx_R$)} & 
      \octx_0 \simu{R} \octx'_0 &
      \text{($\lctx_L \oc \octx'_0 \oc \lctx_R = \octx'$)}}
  \end{inferences}
  In other words, $\lrframe{\lctx_L}{\simu{R}}{\lctx_R}$ relates contexts consisting of $\simu{R}$-related middles that are each surrounded by $\lctx_L$ and $\lctx_R$.

  Furthermore, let $\ctxc{\simu{R}}$ be the input contextual closure of $\simu{R}$ -- that is, $\ctxc{\simu{R}}$ is the least binary relation such that:
  \begin{inferences}
    \infer{\octx \ctxc{\simu{R}} \lctx}{
      \octx \simu{R} \lctx}
    \and
    \infer{\atmR{a} \oc \octx \ctxc{\simu{R}} \atmR{a} \oc \lctx}{
      \octx \ctxc{\simu{R}} \lctx}
    \and
    \infer{\octx \oc \atmL{a} \ctxc{\simu{R}} \lctx \oc \atmL{a}}{
      \octx \ctxc{\simu{R}} \lctx}
  \end{inferences}
  Equivalently, $\octx \ctxc{\simu{R}} \lctx$ if, and only if, there exist input contexts $\atmR{\lctx}_L$ and $\atmL{\lctx}_R$ such that $\octx \lrframe{\atmR{\lctx}_L}{\simu{R}}{\atmL{\lctx}_R} \lctx$.
\end{definition}
\noindent
With these auxiliary relations in hand, we can now turn to defining ordered rewriting bisimilarity.

\newthought{Contexts should be} equivalent only if they are not observably distinguishable by a single type of experiment: provide the contexts with some incoming messages (or none at all) and observe any outgoing messages that are eventually produced.%
% We take the position that outgoing messages are observable.
If two contexts may eventually produce different outgoing messages, then observing those messages will allow us to distinguish the two contexts.
% We may also perform simple experiments on contexts: provide the contexts with some incoming messages (or none at all) and observe any outgoing messages that are eventually produced.%
In particular, we do not allow the time or number of computational steps to be observed -- all that matters is whether the same outputs are eventually produced from the same inputs.
The resulting bisimilarity will therefore be a weak bisimilarity.
\footnote{We could equivalently combine these into one monolithic condition: If $\atmR{\lctx}_L \oc \octx \oc \atmL{\lctx}_R \lrframe{\atmR{\lctx}_L}{\simu{R}}{\atmL{\lctx}_R}\Reduces \atmL{\lctx}'_L \oc \lctx' \oc \atmR{\lctx}'_R$, then $\atmR{\lctx}_L \oc \octx \oc \atmL{\lctx}_R \Reduces\lrframe{\atmL{\lctx}'_L}{\simu{R}}{\atmR{\lctx}'_R} \atmL{\lctx}'_L \oc \lctx' \oc \atmR{\lctx}'_R$.
  \begin{equation*}
    \begin{tikzcd}[ampersand replacement=\&, sep=large]
      \atmR{\lctx}_L \oc \octx \oc \atmL{\lctx}_R
        \rar[relation, "\lrframe{\atmR{\lctx}_L}{\simu{R}}{\atmL{\lctx}_R}"]
        \dar[Reduces, exists]
      \&
      \atmR{\lctx}_L \oc \lctx \oc \atmL{\lctx}_R
        \dar[Reduces]
      \\
      \atmL{\lctx}'_L \oc \octx' \oc \atmL{\lctx}'_R
        \rar[relation, exists, "\lrframe{\atmL{\lctx}'_L}{\simu{R}}{\atmL{\lctx}'_R}"]
      \&
      \atmL{\lctx}'_L \oc \lctx' \oc \atmL{\lctx}'_R
    \end{tikzcd}
  \end{equation*}
  However, this formulation is unwieldy, so we prefer the separate output and input bisimulation conditions.}

 they have the same input and output behavior.
Based on this principle, we arrive at the following notion of rewriting bisimilarity.
We will state its definition first and then justify that definition on the basis of observable indistinguishability.
%
\begin{definition}\label{def:ordered-bisimilarity:bisim}
  A \vocab{rewriting bisimulation}, $\simu{R}$, is a symmetric binary relation among contexts that satisfies the following conditions.
  \begin{thmdescription}[nosep]
  \item[Output bisimulation]
    If $\octx \simu{R}\Reduces \atmL{\lctx}'_L \oc \lctx' \oc \atmR{\lctx}'_R$, then $\octx \Reduces\lrframe{\atmL{\lctx}'_L}{\simu{R}}{\atmR{\lctx}'_R} \atmL{\lctx}'_L \oc \lctx' \oc \atmR{\lctx}'_R$.
  \item[Input bisimulation]
    If $\atmR{\lctx}_L \oc \octx \oc \atmL{\lctx}_R \lrframe{\atmR{\lctx}_L}{\simu{R}}{\atmL{\lctx}_R}\Reduces \lctx'$, then $\atmR{\lctx}_L \oc \octx \oc \atmL{\lctx}_R \Reduces\simu{R} \lctx'$.
  \end{thmdescription}
  \begin{marginfigure}
    \begin{center}
      \begin{tabular}{@{}c@{}}
        \begin{tikzcd}[sep=large]
          \octx
            \rar[relation, "\simu{R}"]
            \dar[Reduces, exists]
          &
          \lctx
            \dar[Reduces]
          \\
          \atmL{\lctx}'_L \oc \octx' \oc \atmR{\lctx}'_R
            \rar[relation, exists, "\lrframe{\atmL{\lctx}'_L}{\simu{R}}{\atmR{\lctx}'_R}"]
          &
          \atmL{\lctx}'_L \oc \lctx' \oc \atmR{\lctx}'_R
        \end{tikzcd}
        \\
        \emph{Output bisimulation}
        \\[2ex]
        \begin{tikzcd}[sep=large]
          \atmR{\lctx}_L \oc \octx \oc \atmL{\lctx}_R
            \rar[relation, "\lrframe{\atmR{\lctx}_L}{\simu{R}}{\atmL{\lctx}_R}"]
            \dar[Reduces, exists]
          &
          \atmR{\lctx}_L \oc \lctx \oc \atmL{\lctx}_R
            \dar[Reduces]
          \\
          \octx\mathrlap{'}
            \rar[relation, exists, "\simu{R}"]
          &
          \lctx\mathrlap{'}
        \end{tikzcd}
        \\
        \emph{Input bisimulation}
      \end{tabular}
    \end{center}
    \caption{Rewriting bisimulation conditions, in diagrams}
  \end{marginfigure}
  Then \vocab{rewriting bisimilarity}, $\osim$, is the largest rewriting bisimulation.
\end{definition}
\noindent
Notice that a third, reduction bisimulation property is a trivial instance of the output and input bisimulation conditions -- namely when the output and input contexts, $\atmL{\lctx}'_L$ and $\atmR{\lctx}'_R$ and $\atmR{\lctx}_L$ and $\atmL{\lctx}_R$, respectively, are empty:
\begin{theorem}\label{thm:bisim-reduction-closure}
  If $\simu{R}$ is a rewriting bisimulation, then $\simu{R}$ satisifies
  \begin{thmdescription}[nosep]
  \item[Reduction bisimulation]
    If $\octx \simu{R}\Reduces \lctx'$, then $\octx \Reduces\simu{R} \lctx'$.
  \end{thmdescription}
\end{theorem}

% Both the output and input bisimulation conditions serve to ensure that related contexts are not observably distinguishable.

The clauses of \cref{def:ordered-bisimilarity:bisim} could do with some explanation.
Let's begin with the output bisimulation condition.

Expanding slightly, we are given that there exists a context $\lctx$ such that $\octx \simu{R} \lctx \Reduces \atmL{\lctx}'_L \oc \lctx' \oc \atmR{\lctx}'_R$.
Based on the local interaction semantics~\parencref{sec:formula-as-process:transition-semantics}, this means
\begin{enumerate*}[label=\emph{(\roman*)}]
\item that $\lctx$ can eventually output $\atmL{\lctx}'_L$ and $\atmR{\lctx}'_R$ and then continue as $\lctx'$; and
\item that $\octx$ is $\simu{R}$-related to $\lctx$
\end{enumerate*}.
For $\simu{R}$ to be a rewriting bisimulation, the context $\octx$ ought to be able to simulate $\lctx$'s eventual output of $\atmL{\lctx}'_L$ and $\atmR{\lctx}'_R$, otherwise the $\simu{R}$-related contexts $\octx$ and $\lctx$ could be distinguished based on their (eventual) output behavior.
Moreover, the continuations ought to be $\simu{R}$-related as well.
Formally, we ought to have $\octx \Reduces \atmL{\lctx}'_L \oc \octx' \oc \atmR{\lctx}'_R$ (\ie, $\octx$ eventually outputs $\atmL{\lctx}'_L$ and $\atmR{\lctx}'_R$) and $\octx' \simu{R} \lctx'$, for some context $\octx'$, which is all neatly packaged up as $\octx \Reduces\lrframe{\atmL{\lctx}'_L}{\simu{R}}{\atmR{\lctx}'_R} \atmL{\lctx}'_L \oc \lctx' \oc \atmR{\lctx}'_R$.

Input bisimulation is dual to output bisimulation.
For input bisimulation, we are given that $\atmR{\lctx}_L \oc \octx \oc \atmL{\lctx}_R \lrframe{\atmR{\lctx}_L}{\simu{R}}{\atmL{\lctx}_R}\Reduces \lctx'$.
Expanding slightly, there exists a context $\lctx$ such that $\octx \simu{R} \lctx$ and $\atmR{\lctx}_L \oc \lctx \oc \atmL{\lctx}_R \Reduces \lctx'$.
In other words, once provided with the incoming messages $\atmR{\lctx}_L$ and $\atmL{\lctx}_R$, the context $\lctx$ can eventually evolve to $\lctx'$.
Being $\simu{R}$-related to $\lctx$, the context $\octx$, when provided with the same incoming messages, must be able to evolve to a context that is $\simu{R}$-related to $\lctx'$, otherwise $\octx$ and $\lctx$ could be distinguished by how they react to $\atmR{\lctx}_L$ and $\atmL{\lctx}_R$.
That is, we must have $\atmR{\lctx}_L \oc \octx \oc \atmL{\lctx}_R \Reduces \octx' \simu{R} \lctx'$ for some $\octx'$, which is neatly packaged as $\atmR{\lctx}_L \oc \octx \oc \atmL{\lctx}_R \Reduces\simu{R} \lctx'$.


\newthought{The other way} to understand rewriting bisimilarity is by analogy with the asynchronous \ac{CCS}'s notion of weak bisimilarity.\autocites{Amadio+:TCS98}{Boreale+:IC02}
There, a weak bisimulation can be described as a symmetric relation $\simu{R}$ on processes that satisfies three conditions:%
\footnote{%
  The premises of these conditions are usually stated with strong transitions, but we prefer this phrasing for its similarity to rewriting bisimilarity.}
\begin{itemize}[noitemsep]% [label=\emph{(\roman*)}]
\item If $P \simu{R}\overset{\smash{\bar{c}}}{\Reduces} Q'$, then $P \overset{\smash{\bar{c}}}{\Reduces}\simu{R} Q'$.
\item If $P \simu{R}\overset{\tau}{\Reduces} Q'$, then $P \overset{\tau}{\Reduces}\simu{R} Q'$.
\item If $P \simu{R}\overset{c}{\Reduces} Q'$, then either $P \overset{c}{\Reduces}\simu{R} Q'$ or there exists a process $P'$ such that $P \overset{\tau}{\Reduces} P'$ and $\bar{c} \mid P' \simu{R} Q'$.
\end{itemize}
Weak bisimilarity for the asynchronous \ac{CCS} is then the largest such bisimulation.

The first of these three conditions corresponds to rewriting bisimilarity's output bisimulation condition with nonempty output contexts $\atmL{\lctx}'_L$ and $\atmR{\lctx}'_R$.
As mentioned in \cref{sec:formula-as-process:transition-semantics}, $\lctx \Reduces \atmL{\lctx}'_L \oc \lctx' \oc \atmR{\lctx}'_R$ functions as an implicit weak output transition from $\lctx$ to $\lctx'$, with outputs $\atmL{\lctx}'_L$ and $\atmR{\lctx}'_R$.
Similarly, $\octx \Reduces\lrframe{\atmL{\lctx}'_L}{\simu{R}}{\atmR{\lctx}'_R} \atmL{\lctx}'_L \oc \lctx' \oc \atmR{\lctx}'_R$ is analogous to $P \overset{\smash{\bar{c}}}{\Reduces}\simu{R} Q'$.

The second of the three conditions imposed by asynchronous \ac{CCS} weak bisimilarity corresponds to rewriting bisimilarity's reduction bisimulation property~\parencref{thm:bisim-reduction-closure}, which is really just either output or input bisimulation with empty output contexts $\atmL{\lctx}'_L$ and $\atmR{\lctx}'_R$ or input contexts $\atmR{\lctx}_L$ and $\atmL{\lctx}_R$.

The third of the three conditions imposed by asynchronous \ac{CCS} weak bisimilarity corresponds to rewriting bisimilarity's input bisimulation condition.
% Rewriting bisimilarity's input bisimulation condition corresponds to the third of the asynchronous \ac{CCS} weak bisimilarity's conditions.
The \ac{CCS} condition is equivalent to \enquote{If $P \simu{R}\overset{c}{\reduces} Q'$, then $\bar{c} \mid P \overset{\tau}{\Reduces}\simu{R} Q'$.}
(In fact, it is typical to use this phrasing in the definition of asynchronous \ac{CCS} bisimilarity.)
Because asynchronous \ac{CCS} weak bisimilarity is a congruence, that condition can be rephrased as \enquote{If $P \simu{R} Q$ and $\bar{c} \mid Q \overset{\tau}{\reduces} Q'$, then $\bar{c} \mid P \overset{\tau}{\Reduces}\simu{R} Q'$} without affecting the resulting bisimilarity.%
\footnote{See \cref{app:ccs} for a proof sketch.}
And, in that form, the correspondence with rewriting bisimilarity's input bisimulation condition becomes apparent.




% As described in \cref{sec:??}, our formula-as-process rewriting framework does not use an explicit internal transition judgment, but instead relies on reduction, so that $\reduces$ and $\Reduces$ are analogous to the \ac{CCS} $\overset{\tau}{\reduces}$ and $\overset{\tau}{\Reduces}$ strong and weak internal transition relations, respectively.
% Thus, the above \cref{thm:bisim-reduction-closure} roughly plays the role of the condition imposed on an asynchronous \ac{CCS} bisimulation that $P \simu{R}\overset{\tau}{\reduces} Q'$ implies $P \overset{\tau}{\Reduces}\simu{R} Q'$.

% \Cref{sec:??} also describes how the formula-as-process rewriting framework uses context equality, like $\lctx = \atmL{\lctx}'_L \oc \lctx' \oc \atmR{\lctx}'_R$


% In the asynchronous CCS, a weak bisimulation is a symmetric binary relation on processes that, among other conditions, satisfies:
% \begin{itemize}
% \item If $P \simu{R}\overset{\bar{c}}{\reduces} Q'$, then $P \overset{\bar{c}}{\Reduces}\simu{R} Q'$.
% \end{itemize}
% where $\overset{\bar{c}}{\reduces}$ and $\overset{\bar{c}}{\Reduces}$ are the strong and weak output transition relations on channel $c$.



% As described in \cref{sec:??}, our formula-as-process rewriting framework does not use an explicit output transition judgment, but instead handles output transitions implicitly with context equalities like $\lctx = \atmL{\lctx}'_L \oc \lctx'_0 \oc \atmR{\lctx}'_R$.
% In this implicit setting, there would be a weak output transition from $\lctx$ to $\atmL{\lctx}'_L \oc \lctx' \oc \atmR{\lctx}'_R$ if $\lctx \Reduces \atmL{\lctx}'_L \oc \lctx'_0 \oc \atmR{\lctx}'_R$ and $\lctx'_0 \Reduces \lctx'$

% In the asynchronous CCS, a weak bisimulation is a symmetric binary relation on processes that satisfies:
% \begin{itemize}
% \item If $P \simu{R}\overset{\bar{c}}{\reduces} Q'$, then $P \overset{\bar{c}}{\Reduces}\simu{R} Q'$.
% \item If $P \simu{R}\overset{\tau}{\reduces} Q'$, then $P \overset{\tau}{\Reduces}\simu{R} Q'$.
% \item If $P \simu{R}\overset{c}{\reduces} Q'$, then $\bar{c} \mid P \overset{\tau}{\Reduces}\simu{R} Q'$.%
%   \footnote{Sometimes this is presented as the equivalent \enquote{If $P \simu{R}\overset{c}{\reduces} Q'$, then $P \overset{c}{\Reduces}\simu{R} Q'$ or there exists a process $P'$ such that $P \overset{\tau}{\Reduces} P'$ and $\bar{c} \mid P' \simu{R} Q'$.}}
% \end{itemize}
% In the presence of contextuality, the third condition is equivalent to 
% \begin{itemize}
% \item If $P \simu{R} Q$ and $\bar{c} \mid Q \overset{\tau}{\reduces} Q'$, then $\bar{c} \mid P \overset{\tau}{\Reduces}\simu{R} Q'$.
% \end{itemize}

% These correspond roughly to three conditions found in asynchronous CCS bisimilarity.
% \begin{itemize}
% \item If $P \simu{R}\overset{\bar{c}}{\reduces} Q'$, then $P \overset{\bar{c}}{\reduces}\simu{R} Q'$.
% \item If $P \simu{R}\overset{\bar{c}}{\reduces} Q'$, then $P \overset{\bar{c}}{\reduces}\simu{R} Q'$.
% \end{itemize}

% Allowing the local interaction semantics of \cref{sec:??}, an output of $\atmL{\lctx}'_L$ and $\atmR{\lctx}'_R$ on the left and right respectively occurs at context $\lctx$ if $\lctx = \atmL{\lctx}'_L \oc \lctx' \oc \atmR{\lctx}'_R$.
% In $\pi$-calculus terms, $\lctx = \atmL{\lctx}'_L \oc \lctx' \oc \atmR{\lctx}'_R$ describes a strong output barb on the context $\lctx$.
% But because we are interested in a weak bisimilarity, we generalize this to a weak output barb: $\lctx \Reduces \atmL{\lctx}'_L \oc \lctx' \oc \atmR{\lctx}'_R$, \ie, after finitely many internal reductions, an output of $\atmL{\lctx}'_L$ and $\atmR{\lctx}'_R$ is eventually reached.
% Thus the above output bisimulation condition -- that $\octx \simu{R}\Reduces \atmL{\lctx}'_L \oc \lctx' \oc \atmR{\lctx}'_R$ implies $\octx \Reduces\lrframe{\atmL{\lctx}'_L}{\simu{R}}{\atmR{\lctx}'_R} \atmL{\lctx}'_L \oc \lctx' \oc \atmR{\lctx}'_R$ -- corresponds to the kind of (weak) output barb preservation familiar from asynchronous $\pi$-calculus bisimilarity.

% The above input bisimulation condition can be understood as merely the dual of the output bisimulation condition.
% But it can also be understood by analogy with asynchronous $\pi$-calculus bisimilarity.
% In the asynchronous $\pi$-calculus, 
% \begin{itemize}
% \item If $P \simu{R}\overset{\bar{c}}{\reduces} Q'$, then $P \overset{\bar{c}}{\reduces}\simu{R} Q'$.
% \item If $P \simu{R}\overset{c}{\reduces} Q'$, then either $P \overset{c}{\reduces} \simu{R} Q'$ or there exists $P'$ such that $P \overset{\tau}{\reduces} P'$ and $\bar{c} \mid P' \simu{R} Q'$.
% \item If $P \simu{R}\overset{c}{\reduces} Q'$, then $\bar{c} \mid P \overset{\tau}{\reduces}\simu{R} Q'$.
% \item If $P \simu{R} Q$ and $\bar{c} \mid Q \overset{\tau}{\reduces} Q'$, then $\bar{c} \mid P \overset{\tau}{\reduces}\simu{R} Q'$.
% \end{itemize}



% where $\overset{x(y)}{\Reduces}$ and $\overset{\tau}{\Reduces}$ are weak input and internal transitions, respectively.
% Because the judgment $\ireduces{\atmR{\lctx}_L \oc #1 \oc \atmL{\lctx}_R}{\lctx}{\lctx'}$ is a kind of input transition, we might adopt the condition that 
% \begin{itemize}
% \item If $\octx \simu{R} \lctx$ and $\ireduces{\atmR{\lctx}_L \#1 \oc \atmL{\lctx}_2}{\lctx}{\lctx'}$, then either: $\octx \Reduces \octx^*$ and $\ireduces{\atmR{\lctx}_L \#1 \oc \atmL{\lctx}_2}{\octx^'}{\octx'} \simu{R} \lctx'$; or there exists a process $P'$ such that $P \overset{\tau}{\Reduces} P'$ and $\bar{x}\langle y\rangle \mid P' \simu{R} Q'$.
% \end{itemize}


% Input barbs can also be described using

% $x(y).P' \simu{R} Q$ implies $\bar{x}\langle y\rangle \mid Q \Reduces Q' \simu{R}^{-1} P'$

% If $\octx \simu{R} \lctx$ and $\ireduces{\atmR{\lctx}_L \oc #1 \oc \atmL{\lctx}_R}{\lctx}{\lctx'}$, then either: $\octx \Reduces \octx^*$ and $\ireduces{\atmR{\lctx}_L \oc #1 \oc \atmL{\lctx}_R}{\octx^*}{\octx'} \Reduces\simu{R} \lctx'$; or $\octx \Reduces \octx'$ and $\atmR{\lctx}_L \oc \octx' \oc \atmL{\lctx}_R \simu{R} \lctx'$.

% If $\octx \simu{R} \lctx$ and $\ireduces{\atmR{\lctx}_L \oc #1 \oc \atmL{\lctx}_R}{\lctx}{\lctx'}$, then $\atmR{\lctx}_L \oc \octx \oc \atmL{\lctx}_R \Reduces\simu{R} \lctx'$.



\newthought{Rewriting bisimilarity imposes} very strong conditions upon bisimilar contexts, quantifying over all traces % (not just individual rewriting steps) and also over
and
all output and input contexts.
Combined with the coinductive nature of bisimilarity, this results in a rather fine-grained equivalence.
Some contexts that might, at first glance, seem like they ought to be equivalent are, in fact, not bisimilar.
\begin{itemize}
\item The contexts $\atmL{a} \pmir \atmL{a}$ and $(\octxe)$ are \emph{not} bisimilar.
  Suppose, for the sake of contradiction, that they are bisimilar.
  Framing $\atmL{b}$ onto the right, we have $(\atmL{a} \pmir \atmL{a}) \oc \atmL{b} \rframe{\osim}{\atmL{b}} \atmL{b}$.
  Composing the input and ouput bisimulation conditions, $(\atmL{a} \pmir \atmL{a}) \oc \atmL{b} \Reduces\lframe{\atmL{b}}{\osim} \atmL{b}$ must follow.
  However, this is impossible: $(\atmL{a} \pmir \atmL{a}) \oc \atmL{b}$ is irreducible and does not expose $\atmL{b}$ at its left end.
  Therefore, $\atmL{a} \pmir \atmL{a}$ and $(\octxe)$ \emph{cannot} be bisimilar.

\item The contexts $\atmR{a}$ and $\atmR{a} \with \atmR{b}$ are not bisimilar.
  The context $\atmR{a} \with \atmR{b}$ can output $\atmR{b}$ at its right end: $\atmR{a} \with \atmR{b} \reduces \atmR{b}$.
  But $\atmR{a}$ cannot simulate that output -- the output bisimulation condition demands $\atmR{a} \Reduces\rframe{\osim}{\atmR{b}} \atmR{b}$, which is impossible.

\item If we were working in an unfocused framework, the contexts $\atmR{a}$ and $\atmR{a} \with \top$ would not be bisimilar.
  The context $\atmR{a} \with \top$ would reduce (\ie, $\atmR{a} \with \top \reduces \top$), and so the input bisimulation condition and the irreducibility of $\atmR{a}$ would imply $\atmR{a} \osim \top$.
  The output bisimulation condition would then demand that $\top$ expose $\atmL{a}$ at its left end, which is impossible.
  As we will see later, $\atmR{a}$ and $\atmR{a} \with \top$ are bisimilar in a focused framework.
\end{itemize}

Now we would like to confirm our earlier intuition about the equivalence of $\atmR{a} \oc (\atmR{a} \pmir \atmR{b})$ and $\atmR{b}$ by proving that $\atmR{a} \oc (\atmR{a} \pmir \atmR{b}) \osim \atmR{b}$.
Unfortunately, the definition of rewriting bisimilarity is not immediately suitable for establishing that two contexts are bisimilar.
The output and input bisimulation conditions are so strong that they become difficult to prove directly.
For instance, to establish $\atmR{a} \oc (\atmR{a} \limp \atmR{b}) \osim \atmR{b}$, we would need to prove that:
\begin{description}[noitemsep]% [leftmargin=0pt]
\item[Input bisimulation]% \leavevmode
 $\atmR{\lctx}_L \oc \atmR{a} \oc (\atmR{a} \limp \atmR{b}) \oc \atmL{\lctx}_R \Reduces \lctx'$ implies $\atmR{\lctx}_L \oc \atmR{b} \oc \atmL{\lctx}_R \osim \lctx'$; and % , for all $\atmR{\lctx}_L$ and $\atmL{\lctx}_R$; and
 symmetrically, $\atmR{\lctx}_L \oc \atmR{b} \oc \atmL{\lctx}_R \Reduces \lctx'$ implies $\atmR{\lctx}_L \oc \atmR{a} \oc (\atmR{a} \limp \atmR{b}) \oc \atmL{\lctx}_R \osim \lctx'$; % , for all $\atmR{\lctx}_L$ and $\atmL{\lctx}_R$; and
%  \end{itemize}
  
\item[Output bisimulation]% \leavevmode
 % \begin{itemize}[nosep]
   $\atmR{a} \oc (\atmR{a} \limp \atmR{b}) \Reduces \atmL{\lctx}'_L \oc \lctx' \oc \atmR{\lctx}'_R$ implies $\atmR{b} \Reduces\lrframe{\atmL{\lctx}'_L}{\osim}{\atmR{\lctx}'_R} \atmL{\lctx}'_L \oc \lctx' \oc \atmR{\lctx}'_R$; and
   symmetrically, $\atmR{b} \Reduces \atmL{\lctx}'_L \oc \lctx' \oc \atmR{\lctx}'_R$ implies $\atmR{a} \oc (\atmR{a} \limp \atmR{b}) \Reduces\lrframe{\atmL{\lctx}'_L}{\osim}{\atmR{\lctx}'_R} \atmL{\lctx}'_L \oc \lctx' \oc \atmR{\lctx}'_R$.
  % \end{itemize}
\end{description}
In this small example, it is possible to imagine tediously proving these statements -- after all, there are not that many traces involving $\atmR{a} \oc (\atmR{a} \limp \atmR{b})$.
However, in general, a proof technique for rewriting bisimilarity is sorely needed.


% Simple examples of bisimilar (or non-bisimilar) contexts
% \begin{itemize}
% \item $\atmL{a} \pmir \atmL{a} \nosim \octxe$ because input bisimulation followed by output bisimulation demands that $\atmR{b} \oc (\atmL{a} \pmir \atmL{a}) \Reduces\lrframe{}{\osim}{\atmR{b}} \atmR{b}$, which is impossible because $\atmR{b} \oc (\atmL{a} \pmir \atmL{a})$ has no nontrivial reductions and does not expose $\atmR{b}$ at its right.
% \item $\atmR{a} \with \atmR{b} \nosim \atmR{a}$ because $\atmR{a} \Reduces\lrframe{}{\osim}{\atmR{b}} \atmR{b}$ is impossible.
% \item {[$\atmR{a} \with \top \osim \atmR{a}$, but only because rewriting is (weakly) focused.]}
% \item $\atmR{a} \oc (\atmR{a} \limp \atmR{b}) \osim \atmR{b}$ intuitively because $\atmR{a} \oc (\atmR{a} \limp \atmR{b})$ has no input transitions and reduces to $\atmR{b}$.
%   Need a proof technique to establish this.
% \item $\atmR{a} \limp (\atmR{c} \pmir \atmL{b}) \osim (\atmR{a} \limp \atmR{c}) \pmir \atmL{b}$ intuitively because the two propositions are logically equivalent.
%   Both have the same input transitions.
%   Also, $\atmR{a} \limp \up \dn (\atmR{c} \pmir \atmL{b}) \osim \up \dn (\atmR{a} \limp \atmR{c}) \pmir \atmL{b}$.
% \end{itemize}


\subsection{Labeled bisimilarity: A proof technique for rewriting bisimilarity}\label{sec:ordered-bisimilarity:labeled-bisim}

In \ac{CCS} and the $\pi$-calculus, bisimilarity is similarly too strong to be used directly in proving the equivalence of processes.
There, a sound proof technique for bisimilarity is built around a labeled transition system and a notion of labeled bisimulation.\autocite{Sangiorgi+Walker:CUP03}
Because the labeled transition system is image-finite, proving that two processes are labeled bisimilar is more tractable than directly proving them to be bisimilar.

In this \lcnamecref{sec:ordered-bisimilarity:labeled-bisim}, we follow that strategy and develop \vocab{labeled bisimilarity} as a sound -- and, surprisingly, also complete -- proof technique for rewriting bisimilarity.
Like its \ac{CCS} and $\pi$-calculus analogues, labeled bisimilarity is more tractable than rewriting bisimilarity because it uses individual input transitions in place of full rewriting sequences.

Instead of defining labeled bisimulations directly, we use a refactorization, standard in the study of up-to techniques\autocite{??}, in which we first define a notion of \emph{progression} and then characterize labeled bisimulations in terms of progression.
\begin{definition}
  A binary relation $\simu{R}$ on contexts \vocab{progresses} to binary relation $\simu{S}$ if $\simu{R}$ is symmetric and the two relations satisfy the following conditions.
  \begin{thmdescription}[nosep]
  \item[Immediate output bisim.]
    If $\octx \simu{R} \lctx = \atmL{\lctx}'_L \oc \lctx' \oc \atmR{\lctx}'_R$, then $\octx \Reduces\lrframe{\atmL{\lctx}'_L}{\simu{S}}{\atmR{\lctx}'_R} \lctx$.
  \item[Immediate input bisimulation]
    If $\octx \simu{R} \lctx$ and $\ireduces{\atmR{\lctx}_L \oc #1 \oc \atmL{\lctx}_R}{\lctx}{\lctx'}$, then\\$\atmR{\lctx}_L \oc \octx \oc \atmL{\lctx}_R \Reduces\simu{S} \lctx'$.
  \item[Reduction bisimulation]
    If $\octx \simu{R}\reduces \lctx'$, then $\octx \Reduces\simu{S} \lctx'$.
  \item[Emptiness bisimulation]
    If $\octx \simu{R} (\octxe)$, then:
    \begin{itemize*}[label=, afterlabel=]
    \item $\atmR{\lctx} \oc \octx \Reduces\rframe{\simu{S}}{\atmR{\lctx}} \atmR{\lctx}$ for all $\atmR{\lctx}$; and
    \item $\octx \oc \atmL{\lctx} \Reduces\lframe{\atmL{\lctx}}{\simu{S}} \atmL{\lctx}$ for all $\atmL{\lctx}$.
    \end{itemize*}
  \end{thmdescription}
  A \vocab{labeled bisimulation} is a relation that progresses to itself, and \vocab{labeled bisimilarity} is the largest labeled bisimulation.
  \begin{marginfigure}
    \begin{center}
      \begin{tabular}{@{}c@{}}
        \begin{tikzcd}[sep=large]
          \octx
            \rar[relation, "\simu{R}"]
            \dar[Reduces, exists]
          &
          \lctx \mathrlap{{} = \atmL{\lctx}'_L \oc \lctx' \oc \atmR{\lctx}'_R}
          \\
          \atmL{\lctx}'_L \oc \octx' \oc \atmR{\lctx}'_R
            \urar[relation, exists, "\lrframe{\atmL{\lctx}'_L}{\simu{S}}{\atmR{\lctx}'_R}" {sloped, below}]
        \end{tikzcd}%
        \phantom{${} = \atmL{\lctx}'_L \oc \lctx' \oc \atmR{\lctx}'_R$}
        \\
        \emph{Immediate output bisimulation}
        \\[2ex]
        \begin{tikzcd}[sep=large]
          \atmR{\lctx}_L \oc \octx \oc \atmL{\lctx}_R
            \rar[relation, "\lrframe{\atmR{\lctx}_L}{\simu{R}}{\atmL{\lctx}_R}"]
            \arrow[Reduces, exists]{dd}
          &
          \atmR{\lctx}_L \oc \lctx \oc \atmL{\lctx}_R
          \\[-6ex]
          &
          \atmR{\lctx}_L \oc [\lctx] \oc \atmL{\lctx}_R
            \dar[reduces]
          \\
          \octx\mathrlap{'}
            \rar[relation, exists]{\simu{S}}
          &
          \lctx\mathrlap{'}
        \end{tikzcd}
        \\
        \emph{Immediate input bisimulation}
        \\[2ex]
        \begin{tikzcd}[sep=large]
          \octx
            \rar[relation, "\simu{R}"]
            \dar[Reduces, exists]
          &
          \lctx
            \dar[reduces]
          \\
          \octx\mathrlap{'}
            \rar[relation, exists, "\simu{S}"]
          &
          \lctx\mathrlap{'}
        \end{tikzcd}
        \\
        \emph{Reduction bisimulation}
        \\[2ex]
        \begin{tikzcd}[sep=large]
          \atmR{\lctx} \oc \octx
            \rar[relation, "\lrframe{\atmR{\lctx}}{\simu{R}}{}"]
            \dar[Reduces, exists]
          &
          \atmR{\lctx}
          \\
          \octx' \oc \atmR{\lctx}
            \urar[relation, exists, "\lrframe{}{\simu{S}}{\atmR{\lctx}}" {sloped, below}]
        \end{tikzcd}
        \quad
        \begin{tikzcd}[sep=large]
          \octx \oc \atmL{\lctx}
            \rar[relation, "\lrframe{}{\simu{R}}{\atmL{\lctx}}"]
            \dar[Reduces, exists]
          &
          \atmL{\lctx}
          \\
          \atmL{\lctx} \oc \octx'
            \urar[relation, exists, "\lrframe{\atmL{\lctx}}{\simu{S}}{}" {sloped, below}]
        \end{tikzcd}
        \\
        \emph{Emptiness bisimulation}
      \end{tabular}
    \end{center}
    \caption{Labeled bisimulation conditions, in diagrams}
  \end{marginfigure}
\end{definition}

The immediate output, immediate input, and reduction bisimulation conditions are all single-step forms of rewriting bisimilarity's output and input bisimulation conditions~\parencref{def:ordered-bisimilarity:bisim} and reduction bisimulation property~\parencref{thm:bisim-reduction-closure}.
The emptiness bisimulation condition, on the other hand, is necessary for labeled bisimilarity to be complete.
A similar condition appears in \textcite{Deng+:MSCS16}'s simulation preorder.

Emptiness bisimulation is equivalent to requiring that $\octx \simu{R} (\octxe)$ implies both $\octx \Reduces (\octxe)$ and $(\octxe) \simu{S} (\octxe)$.%
\footnote{See \cref{thm:emptiness-bisim-equiv} for a proof sketch.}
In this way, it can be seen that being $\simu{R}$-related to $(\octxe)$ is possible only if $\octx$ is morally equivalent to $(\octxe)$, in the sense that $\octx$ must be able to spontaneously evolve to $(\octxe)$.


% O R (.) implies a O ==>(Ra) a
% a O ==> O' a and O' R (.)
% Case: a O = O' a
%   O = (.) because it can't end with a 
% Case: a O -->  ==> O' a
%   O --> O' and a O'' ==> O' a
%   O'' ==> (.)
%   O ==> (.)


% If D O ==>(Sa) D, then O ==> (.) and (.) S (.).
% If O ==> (.) S (.), then D O ==>(SD) D.

% a O ==> O' a then O ==> (.) and O' = (.)
% Case: a O = O' a
%   O = O' = . because a can't apear in O
% Case: a O --> ==> O' a
%   O --> O1 and a O1 ==> O' a
%   O1 ==> (.) and O' = (.)
%   O ==> (.)

\newthought{It is relatively} straightforward to show that labeled bisimilarity is complete with respect to rewriting bisimilarity: every rewriting bisimulation is itself a labeled bisimulation.
%
\begin{theorem}[Completeness of labeled bisimilarity]\label{thm:ordered-bisimilarity:labeled-complete}
  Every rewriting bisimulation is also a labeled bisimulation, and labeled bisimilarity consequently contains rewriting bisimilarity.
\end{theorem}
\begin{proof}
  Let $\simu{R}$ be a rewriting bisimulation.
  The immediate output, immediate input, and reduction bisimulation conditions are trivial instances of the output and input bisimulation conditions.
  For instance, to prove that $\simu{R}$ is an immediate input bisimulation, assume that $\octx \simu{R} \lctx$ and $\ireduces{\atmR{\lctx}_L \oc #1 \oc \atmL{\lctx}_R}{\lctx}{\lctx'}$; then $\atmR{\lctx}_L \oc \octx \oc \atmL{\lctx}_R \lrframe{\atmR{\lctx}_L}{\simu{R}}{\atmL{\lctx}_R} \reduces \lctx'$.
  Because $\simu{R}$ is a rewriting bisimulation, it follows from the input bisimulation property that $\atmR{\lctx}_L \oc \octx \oc \atmL{\lctx}_R \Reduces\simu{R} \lctx'$.

  The emptiness bisimulation condition follows from the composition of the input bisimulation property with the output bisimulation property.%
  \begin{marginfigure}
    \begin{center}
      \begin{tabular}{@{}c@{\quad}c@{}}
        \begin{tikzcd}
          \atmR{\lctx} \oc \octx
            \rar[relation, "\lrframe{\atmR{\lctx}}{\simu{R}}{}"]
            \dar[Reduces]
          &
          \atmR{\lctx}
            \arrow[Reduces, loop right]{}
          \\
          \octx\mathrlap{'}
            \urar[relation, "\simu{R}" sloped]
            \dar[Reduces]
          \\
          \octx'' \oc \atmR{\lctx}
            \arrow[relation, "\lrframe{}{\simu{R}}{\atmR{\lctx}}" {sloped, below}]{uur}
        \end{tikzcd}
        &
        \begin{tikzcd}
          \octx \oc \atmL{\lctx}
            \rar[relation, "\lrframe{}{\simu{R}}{\atmL{\lctx}}"]
            \dar[Reduces]
          &
          \atmL{\lctx}
            \arrow[Reduces, loop right]{}
          \\
          \octx\mathrlap{'}
            \urar[relation, "\simu{R}" sloped]
            \dar[Reduces]
          \\
          \atmL{\lctx} \oc \octx''
            \arrow[relation, "\lrframe{\atmL{\lctx}}{\simu{R}}{}" {sloped, below}]{uur}
        \end{tikzcd}
      \end{tabular}
    \end{center}
    \caption{Emptiness bisimulation property as a consequence of input and output bisimulation properties}
  \end{marginfigure}
\end{proof}



Unfortunately, the direct converse is not true: a labeled bisimulation is not necessarily itself a rewriting bisimulation.
For example, consider the least symmetric binary relation $\simu{R}$ such that
\begin{equation*}
  \atmR{a} \limp (\atmR{c} \pmir \atmL{b}) \simu{R} (\atmR{a} \limp \atmR{c}) \pmir \atmL{b}
  \quad\text{and}\quad
  \atmR{c} \simu{R} \atmR{c}
  \quad\text{and}\quad
  (\octxe) \simu{R} (\octxe)
  \,.
\end{equation*}
The relation $\simu{R}$ is a labeled bisimulation, but it does not qualify as a rewriting bisimulation because it does not satisfy the more general input bisimulation condition: for instance, $\atmR{a} \oc (\atmR{a} \limp (\atmR{c} \pmir \atmL{b})) \lframe{\atmR{a}}{\simu{R}} \atmR{a} \oc ((\atmR{a} \limp \atmR{c}) \pmir \atmL{b})$ does not imply $\atmR{a} \oc (\atmR{a} \limp (\atmR{c} \pmir \atmL{b})) \Reduces\simu{R} \atmR{a} \oc ((\atmR{a} \limp \atmR{c}) \pmir \atmR{b})$.
That would be possible only if $\atmR{a} \oc (\atmR{a} \limp (\atmR{c} \pmir \atmL{b}))$ and $\atmR{a} \oc ((\atmR{a} \limp \atmR{c}) \pmir \atmR{b})$ were $\simu{R}$-related.

Even though a labeled bisimulation itself is not a rewriting bisimulation, a slightly weaker statement is nevertheless true: each labeled bisimulation is contained within \emph{some} rewriting bisimulation.
Specifically, if $\simu{R}$ is a labeled bisimulation, then its input contextual closure, $\ctxc{\simu{R}}$, as described in \cref{def:ordered-bisimilarity:framed-relation} is such a rewriting bisimulation.
Fortunately, this will be enough to prove that labeled bisimilarity is sound.



% \begin{definition}
%   A symmetric binary relation $\simu{R}$ \vocab{(labeled-)progresses} to binary relation $\simu{S}$ if the two relations satisfy the following conditions.
%   \begin{thmdescription}
%   \item[Immediate output bisimulation]
%     If $\octx \simu{R} \lctx = \atmL{\lctx}'_L \oc \lctx' \oc \atmR{\lctx}'_R$, then $\octx \Reduces\lrframe{\atmL{\lctx}'_L}{\simu{S}}{\atmR{\lctx}'_R} \lctx$.
%   \item[Immediate input bisimulation]
%     If $\octx \simu{R} \lctx$ and $\ireduces{\atmR{\lctx}_L \oc #1 \oc \atmL{\lctx}_R}{\lctx}{\lctx'}$, then $\atmR{\lctx}_L \oc \octx \oc \atmL{\lctx}_R \Reduces\simu{S} \lctx'$.
%   \item[Reduction bisimulation]
%     If $\octx \simu{R}\reduces \lctx'$, then $\octx \Reduces\simu{S} \lctx'$.
%   \item[Emptiness bisimulation]
%     If $\octx \simu{R} \octxe$, then:
%     \begin{itemize*}[label=, afterlabel=]
%     \item $\atmR{\lctx} \oc \octx \Reduces\lrframe{}{\simu{S}}{\atmR{\lctx}} \atmR{\lctx}$ for all $\atmR{\lctx}$; and
%     \item $\octx \oc \atmL{\lctx} \Reduces\lrframe{\atmL{\lctx}}{\simu{S}}{} \atmL{\lctx}$ for all $\atmL{\lctx}$.
%     \end{itemize*}
%   \end{thmdescription}
% \end{definition}
% %
% \noindent
% Notice that the labeled bisimulations are exactly those relations that progress to themselves.



% \begin{lemma}
%   If $\simu{R}$ is a labeled bisimulation, then $\ctxc{\simu{R}}$ satisfies the following properties.
%   \begin{thmdescription}
%   \item[Immediate output bisimulation]
%     If $\octx \ctxc{\simu{R}} \lctx = \atmL{\lctx}'_L \oc \lctx' \oc \atmR{\lctx}'_R$, then $\octx \Reduces\lrframe{\atmL{\lctx}'_L}{\ctxc{\simu{R}}}{\atmR{\lctx}'_R} \lctx$.
%   \item[Reduction bisimulation]
%     If $\octx \ctxc{\simu{R}}\reduces \lctx'$, then $\octx \Reduces\ctxc{\simu{R}} \lctx'$.
%   \end{thmdescription}
% \end{lemma}
% \begin{proof}
%   The two properties are established separately. 
%   \begin{description}
%   \item[Immediate output bisimulation]
%     Assume that $\octx \ctxc{\simu{R}} \lctx = \atmL{\lctx}'_L \oc \lctx' \oc \atmR{\lctx}'_R$.
%     There are four cases, according to whether $\atmL{\lctx}'_L$ and $\atmR{\lctx}'_R$ are empty.
%     \begin{itemize}[parsep=0pt, listparindent=\parindent]
%     \item Consider the case in which both $\atmL{\lctx}'_L$ and $\atmR{\lctx}'_R$ are empty; in this case, we must show that $\octx \Reduces\ctxc{\simu{R}} \lctx$.
%       Using a trivial trace, that follows directly from the assumption $\octx \ctxc{\simu{R}} \lctx$.

%     \item Consider the case in which both $\atmL{\lctx}'_L$ and $\atmR{\lctx}'_R$ are nonempty.
%       The context $\lctx$ therefore exposes output atoms\fixnote{messages?} at its left and right ends.
%       And so the $\ctxc{\simu{R}}$-related contexts $\octx$ and $\lctx$ must, in fact, be $\simu{R}$-related, for otherwise at least one end of $\lctx$ would expose an input, not output, atom.
%       In other words, $\octx \simu{R} \lctx = \atmL{\lctx}'_L \oc \lctx' \oc \atmR{\lctx}'_R$.
%       Because $\simu{R}$ is a labeled bisimulation and satisfies the immediate output bisimulation property, it follows that $\octx \Reduces\lrframe{\atmL{\lctx}'_L}{\simu{R}}{\atmR{\lctx}'_R} \lctx$.
%       Since $\ctxc{\simu{R}}$ trivially contains $\simu{R}$, we conclude that $\octx \Reduces\lrframe{\atmL{\lctx}'_L}{\ctxc{\simu{R}}}{\atmR{\lctx}'_R} \lctx$, as required.

%     \item Consider the case in which $\atmL{\lctx}'_L$ is nonempty and $\atmR{\lctx}'_R$ is empty; in this case, we must show that $\octx \Reduces\lrframe{\atmL{\lctx}'_L}{\ctxc{\simu{R}}}{} \lctx$.
%       Similar to the previous case, $\lctx$ exposes output atoms at its left end because $\atmL{\lctx}'_L$ is nonempty.
%       And so the $\ctxc{\simu{R}}$-related contexts $\octx$ and $\lctx$ must, in fact, be $\lrframe{}{\simu{R}}{\atmL{\octx}_R}$-related, for some $\atmL{\octx}_R$, for otherwise the left end of $\lctx$ would expose an input, not output, atom.
%       In other words, $\octx \lrframe{}{\simu{R}}{\atmL{\octx}_R} \lctx = \atmL{\lctx}'_L \oc \lctx'$.

%       There are two subcases, according to how far the left edge of $\atmL{\octx}_R$ extends into $\lctx$.
%       \begin{itemize}
%       \item Suppose that the left edge of $\atmL{\octx}_R$ does not extend into $\atmL{\lctx}'_L$.
%         Because $\simu{R}$ is a labeled bisimulation, we may appeal to the immediate output bisimulation property after framing off $\atmL{\octx}_R$ -- we deduce $\octx \mathrel{\bigl((\Reduces\lrframe{\atmL{\lctx}'_L}{\simu{R}}{})\atmL{\octx}_R\bigr)} \lctx$.
%         Reduction is closed under framing, so we conclude $\octx \Reduces\lrframe{\atmL{\lctx}'_L}{\ctxc{\simu{R}}}{} \lctx$.
%       \item Otherwise, suppose that the left edge of $\atmL{\octx}_R$ does indeed extend into $\atmL{\lctx}'_L$.
%         In this case, there exist contexts $\atmL{\lctx}''_L$ and $\atmL{\octx}'_L$ such that $\atmL{\lctx}'_L = \atmL{\lctx}''_L \oc \atmL{\octx}'_L$ and $\atmL{\octx}_R = \atmL{\octx}'_L \oc \lctx'$.
%         Because $\simu{R}$ is a labeled bisimulation, we will compose the immediate output and emptiness bisimulation properties to establish $\octx \Reduces\lrframe{\atmL{\lctx}'_L}{\ctxc{\simu{R}}}{} \lctx$.

%         After framing off $\atmL{\octx}_R$, we may appeal to the immediate output bisimulation property and deduce $\octx \mathrel{\bigl((\Reduces\lrframe{\atmL{\lctx}''_L}{\simu{R}}{})\atmL{\octx}_R\bigr)} \lctx = \atmL{\lctx}''_L \oc \atmL{\octx}_R$.
%         Reduction is closed under framing, so $\octx \Reduces\lrframe{\atmL{\lctx}''_L}{\simu{R}}{\atmL{\octx}_R} \lctx$.
%         Upon respelling $\atmL{\octx}_R$ as $\atmL{\octx}'_L \oc \lctx'$ and framing off $\atmL{\lctx}''_L$ and $\lctx'$, we may appeal to the emptiness bisimulation property and deduce $\octx \Reduces\mathrel{\bigl(\atmL{\lctx}''_L\mathord{(\Reduces\lrframe{\atmL{\octx}'_L}{\simu{R}}{})}\lctx'\bigr)} \lctx = \atmL{\lctx}''_L \oc \atmL{\octx}'_L \oc \lctx'$.
%         Once again, reduction is closed under framing and we may respell $\atmL{\lctx}''_L \oc \atmL{\octx}'_L$ as $\atmL{\lctx}'_L$, so $\octx \Reduces\lrframe{\atmL{\lctx}'_L}{\simu{R}}{\lctx'} \lctx$.
%         Finally, because $\lctx'$ is an input context, we may conclude that $\octx \Reduces\lrframe{\atmL{\lctx}'_L}{\ctxc{\simu{R}}}{} \lctx$.
%       \end{itemize}

%     \item The case in which $\atmR{\lctx}'_R$ is nonempty and $\atmL{\lctx}'_L$ is empty is symmetric to the previous case.
%     \end{itemize}

%   \item[Reduction bisimulation]
%     Assume that $\octx \ctxc{\simu{R}}\reduces \lctx'$.
%     There are two cases: either the reduction arises from the $\simu{R}$-related component alone, or it arises from an input transition of the $\simu{R}$-related component that has its input demands met by the framing environment.
%     \begin{itemize}
%     \item Consider the case in which the reduction arises from the $\simu{R}$-related component alone -- that is, the case in which $\octx \ctxc{\simu{R}\reduces} \lctx'$.
%       Because $\simu{R}$ is a labeled bisimulation and therefore satisfies reduction bisimulation, it follows that $\octx \ctxc{\Reduces\simu{R}} \lctx'$.
%       Reduction is closed under framing, so $\octx \Reduces\ctxc{\simu{R}} \lctx'$.
%     \item Consider the case in which the reduction arises from an input transition of the $\simu{R}$-related component that has its input demands met by the framing environment -- that is, the case in which $\octx \ctxc{\lrframe{\atmR{\lctx}_L}{\simu{R}}{\atmL{\lctx}_R}\mathrel{(\prescript{\atmR{\lctx}_L}{}{\simu{I}}^{\atmL{\lctx}_R})}} \lctx'$ for some $\atmR{\lctx}_L$ and $\atmL{\lctx}_R$, where $\atmR{\lctx}_L \oc \lctx_0 \oc \atmL{\lctx}_R \mathrel{(\prescript{\atmR{\lctx}_L}{}{\simu{I}}^{\atmL{\lctx}_R})} \lctx'_0$ if $\ireduces{\atmR{\lctx}_L \oc #1 \oc \atmL{\lctx}_R}{\lctx_0}{\lctx'_0}$.
%       Because $\simu{R}$ is a labeled bisimulation and therefore satisfies immediate input bisimulation, it follows that $\octx \ctxc{\Reduces\simu{R}} \lctx'$.
%       Once again, reduction is closed under framing, so $\octx \Reduces\ctxc{\simu{R}} \lctx'$.
%     \end{itemize}
%     Note that it is impossible for the reduction to arise from the framing environment alone, because $\ctxc{\simu{R}}$ surrounds the $\simu{R}$-related components with only input messages, which are passive.
%   \qedhere
%   \end{description}
% \end{proof}

% \begin{theorem}
%   If $\simu{R}$ is a labeled bisimulation, then $\ctxc{\simu{R}}$ is a rewriting bisimulation.
% \end{theorem}
% \begin{proof}
%   To prove that $\ctxc{\simu{R}}$ is a rewriting bisimulation whenever $\simu{R}$ is a labeled bisimulation, we shall establish output and input bisimulation properties for $\ctxc{\simu{R}}$.
%   \begin{description}
%   \item[Output bisimulation]
%     follows by composing the reduction bisimulation and immediate output bisimulation properties of $\ctxc{\simu{R}}$, as proved in \cref{??}.%
%     \fixnote{Diagram?}
%   \item[Input bisimulation]
%     for $\ctxc{\simu{R}}$ is simply an instance of its reduction bisimulation property, as proved in \cref{??}, because $\ctxc{\simu{R}}$ is closed under framing of input message contexts.%
%     \fixnote{Diagram?}
%   \qedhere
%   \end{description}
% \end{proof}

% \begin{corollary}
%   Rewriting bisimilarity contains labeled bisimilarity.
% \end{corollary}


% \begin{theorem}
%   Given a binary relation $\simu{R}$, let $\ctxc{\simu{R}}$ be the \emph{input context[ual] closure} of $\simu{R}$ -- the least relation such that $\lctx \ctxc{\simu{R}} \octx$ if $\lctx \lrframe{\atmR{\lctx}_L}{\simu{R}}{\atmL{\lctx}_R} \octx$ for some $\atmR{\lctx}_L$ and $\atmL{\lctx}_R$.
%   If $\simu{R}$ is a labeled bisimulation, then $\ctxc{\simu{R}}$ is a rewriting bisimulation.
% \end{theorem}
% \begin{proof}
%   \begin{itemize}
%   \item
%     Suppose that $\octx \simu{R}^{-1}\reduces\Reduces \atmL{\lctx}'_L \oc \lctx' \oc \atmR{\lctx}'_R$.
%     $\octx \Reduces\simu{R}^{-1}\Reduces \atmL{\lctx}'_L \oc \lctx' \oc \atmR{\lctx}'_R$
%     $\octx \Reduces\lrframe{\atmL{\lctx}'_L}{\simu{R}}{\atmR{\lctx}'_R}^{-1} \atmL{\lctx}'_L \oc \lctx' \oc \atmR{\lctx}'_R$
%   \item
%     Suppose that $\atmR{\lctx}_L \oc \octx \oc \atmL{\lctx}_R \lrframe{\atmR{\lctx}_L}{\simu{R}}{\atmL{\lctx}_R}^{-1} \lctx'$.
%     So $\octx \simu{R}^{-1} \lctx'_0$ and $\lctx' = \atmR{\lctx}_L \oc \lctx'_0 \oc \atmL{\lctx}_R$.
%   \end{itemize}
% \end{proof}


% \begin{theorem}
%   A symmetric relation $\simu{R}$ is contained in bisimilarity if it satisfies the following conditions.
%   \begin{thmdescription}
%   \item[Immediate output]
%     If $\octx \simu{R}^{-1} \lctx = \atmL{\lctx}'_L \oc \lctx' \oc \atmR{\lctx}'_R$, then $\octx \Reduces\lrframe{\atmL{\lctx}'_L}{\simu{R}}{\atmR{\lctx}'_R}^{-1} \lctx$.
%   \item[Immediate input]
%     If $\octx \simu{R}^{-1} \lctx$ and $\ireduces{\atmR{\lctx}_L \oc #1 \oc \atmL{\lctx}_R}{\lctx}{\lctx'}$, then $\atmR{\lctx}_L \oc \octx \oc \atmL{\lctx}_R \Reduces\simu{R}^{-1} \lctx'$.
%   \item[Reduction closure]
%     If $\octx \simu{R}^{-1}\reduces \lctx'$, then $\octx \Reduces\simu{R}^{-1} \lctx'$.
%   % \item[Emptiness]
%   %   If $\octx \simu{R}^{-1} \octxe$, then:
%   %   \begin{itemize*}[label=, afterlabel=]
%   %   \item $\atmR{\lctx} \oc \octx \Reduces\lrframe{}{\simu{R}}{\atmR{\lctx}}^{-1} \atmR{\lctx}$ for all $\atmR{\lctx}$; and
%   %   \item $\octx \oc \atmL{\lctx} \Reduces\lrframe{\atmL{\lctx}}{\simu{R}}{}^{-1} \atmL{\lctx}$ for all $\atmL{\lctx}$.
%   %   \end{itemize*}
%   \item[Input contextuality]
%     If $\lctx \simu{R} \octx$, then $\atmR{\lctx}_L \oc \lctx \oc \atmL{\lctx}_R \simu{R} \atmR{\lctx}_L \oc \octx \oc \atmL{\lctx}_R$ for all $\atmR{\lctx}_L$ and $\atmL{\lctx}_R$.
%   \end{thmdescription}
% \end{theorem}
% \begin{proof}
%   \begin{description}
%   \item[Output bisimulation]
%     Suppose that $\octx \simu{R}^{-1} \lctx \Reduces \atmL{\lctx}'_L \oc \lctx' \oc \atmR{\lctx}'_R$.
%     By induction on the structure of the given trace, we can show $\octx \Reduces\lrframe{\atmL{\lctx}'_L}{\simu{R}}{\atmR{\lctx}'_R}^{-1} \atmL{\lctx}'_L \oc \lctx' \oc \atmR{\lctx}'_R$.
%     The base case follows from the immediate output bisimulation property of $\simu{R}$;
%     the inductive case follows from reduction closure of $\simu{R}$ and the inductive hypothesis.
%   \item[Input bisimulation]
%     Suppose that $\atmR{\lctx}_L \oc \octx \oc \atmL{\lctx}_R \lrframe{\atmR{\lctx}_L}{\simu{R}}{\atmL{\lctx}_R}^{-1} \atmR{\lctx}_L \oc \lctx \oc \atmL{\lctx}_R \Reduces \lctx'$.
%     We can show, by induction on the structure of the given trace, that $\atmR{\lctx}_L \oc \octx \oc \atmL{\lctx}_R \Reduces\simu{R}^{-1} \lctx'$.
%     The base case follows from the input context closure property of $\simu{R}$.
%     \begin{itemize}
%     \item Suppose that $\atmR{\lctx}_L \oc \octx \oc \atmL{\lctx}_R \lrframe{\atmR{\lctx}_L}{\simu{R}}{\atmL{\lctx}_R}^{-1} \atmR{\lctx}_L \oc \lctx \oc \atmL{\lctx}_R \reduces \lctx_1 \Reduces \lctx'$.
%       By Lemma?, there are three possible cases:
%       \begin{itemize}
%       \item If $\lctx \reduces \lctx'_1$ and $\lctx_1 = \atmR{\lctx}_L \oc \lctx'_1 \oc \atmL{\lctx}_R$, then it follows from the reduction closure property of $\simu{R}$ that $\atmR{\lctx}_L \oc \octx \oc \atmL{\lctx}_R \Reduces\lrframe{\atmR{\lctx}_L}{\simu{R}}{\atmL{\lctx}_R}^{-1}\Reduces \lctx'$.
%       \end{itemize}
%       In any case, $\atmR{\lctx}_L \oc \octx \oc \atmL{\lctx}_R \Reduces\lrframe{\atmR{\lctx}^1_L}{\simu{R}}{\atmL{\lctx}^1_R}^{-1} \lctx_1 \Reduces \lctx'$.
%       It follows from the inductive hypothesis that $\atmR{\lctx}_L \oc \octx \oc \atmL{\lctx}_R \Reduces\simu{R}^{-1} \lctx'$.
%     \end{itemize}
%     the inductive case follows from reduction closure of $\simu{R}$ and the inductive hypothesis.
%   \end{description}
% \end{proof}


% \begin{theorem}
%   Let $\simu{R}$ be a symmetric relation that satisfies the following conditions;
%   then $\ctxc{\simu{R}}$ is a rewriting bisimulation.
%   \begin{thmdescription}
%   \item[Immediate output]
%     If $\octx \simu{R}^{-1} \lctx = \atmL{\lctx}'_L \oc \lctx' \oc \atmR{\lctx}'_R$, then $\octx \Reduces\lrframe{\atmL{\lctx}'_L}{\ctxc{\simu{R}}}{\atmR{\lctx}'_R}^{-1} \lctx$.
%   \item[Immediate input]
%     If $\octx \simu{R}^{-1} \lctx$ and $\ireduces{\atmR{\lctx}_L \oc #1 \oc \atmL{\lctx}_R}{\lctx}{\lctx'}$, then $\atmR{\lctx}_L \oc \octx \oc \atmL{\lctx}_R \Reduces\ctxc{\simu{R}}^{-1} \lctx'$.
%   \item[Reduction closure]
%     If $\octx \simu{R}^{-1}\reduces \lctx'$, then $\octx \Reduces\ctxc{\simu{R}}^{-1} \lctx'$.
%   \item[Emptiness]
%     If $\octx \simu{R}^{-1} \octxe$, then:
%     \begin{itemize*}[label=, afterlabel=]
%     \item $\atmR{\lctx} \oc \octx \Reduces\lrframe{}{\ctxc{\simu{R}}}{\atmR{\lctx}}^{-1} \atmR{\lctx}$ for all $\atmR{\lctx}$; and
%     \item $\octx \oc \atmL{\lctx} \Reduces\lrframe{\atmL{\lctx}}{\ctxc{\simu{R}}}{}^{-1} \atmL{\lctx}$ for all $\atmL{\lctx}$.
%     \end{itemize*}
%   % \item[Input contextuality]
%   %   If $\lctx \simu{R} \octx$, then $\atmR{\lctx}_L \oc \lctx \oc \atmL{\lctx}_R \simu{R} \atmR{\lctx}_L \oc \octx \oc \atmL{\lctx}_R$ for all $\atmR{\lctx}_L$ and $\atmL{\lctx}_R$.
%   \end{thmdescription}
% \end{theorem}
% \begin{proof}
%   \begin{description}
%   \item[Immediate output]
%     We must show that $\octx \ctxc{\simu{R}}^{-1} \lctx = \atmL{\lctx}'_L \oc \lctx' \oc \atmR{\lctx}'_R$ implies $\octx \Reduces\lrframe{\atmL{\lctx}'_L}{\ctxc{\simu{R}}}{\atmR{\lctx}'_R}^{-1} \lctx$.
%     \begin{itemize}
%     \item 
%     \end{itemize}
%   \end{description}
% \end{proof}

% \subsection{}

% \begin{definition}
%   A symmetric binary relation $\simu{R}$ is a labeled bisimulation up to reflexivity and context if it satisfies the following conditions.
%   \begin{thmdescription}
%   \item[Immediate output bisimulation]
%     If $\octx \simu{R} \lctx = \atmL{\lctx}'_L \oc \lctx' \oc \atmR{\lctx}'_R$, then $\octx \Reduces\lrframe{\atmL{\lctx}'_L}{(\osim\ctxc{\reflc{\simu{R}}})}{\atmR{\lctx}'_R} \lctx$.
%   \item[Immediate input bisimulation]
%     If $\octx \simu{R} \lctx$ and $\ireduces{\atmR{\lctx}_L \oc #1 \oc \atmL{\lctx}_R}{\lctx}{\lctx'}$, then $\atmR{\lctx}_L \oc \octx \oc \atmL{\lctx}_R \Reduces\osim\ctxc{\reflc{\simu{R}}} \lctx'$.
%   \item[Reduction bisimulation]
%     If $\octx \simu{R}\reduces \lctx'$, then $\octx \Reduces\osim\ctxc{\reflc{\simu{R}}} \lctx'$.
%   \item[Emptiness bisimulation]
%     If $\octx \simu{R} \octxe$, then:
%     \begin{itemize*}[label=, afterlabel=]
%     \item $\atmR{\lctx} \oc \octx \Reduces\lrframe{}{(\osim\ctxc{\reflc{\simu{R}}})}{\atmR{\lctx}} \atmR{\lctx}$ for all $\atmR{\lctx}$; and 
%     \item $\octx \oc \atmL{\lctx} \Reduces\lrframe{\atmL{\lctx}}{(\osim\ctxc{\reflc{\simu{R}}})}{} \atmL{\lctx}$ for all $\atmL{\lctx}$
%     \end{itemize*}%
%     .
%   \end{thmdescription}
% \end{definition}

% \begin{lemma}
%   If $\simu{R}$ is a labeled bisimulation up to reflexivity and context, then $\osim\ctxc{\reflc{\simu{R}}}$ satisfies the following conditions.
%   \begin{thmdescription}
%   \item[Immediate output bisimulation]
%     If $\octx \osim\ctxc{\reflc{\simu{R}}} \lctx = \atmL{\lctx}'_L \oc \lctx' \oc \atmR{\lctx}'_R$, then $\octx \Reduces\lrframe{\atmL{\lctx}'_L}{(\osim\ctxc{\reflc{\simu{R}}})}{\atmR{\lctx}'_R} \lctx$.
%   \item[Reduction bisimulation]
%     If $\octx \osim\ctxc{\reflc{\simu{R}}}\reduces \lctx'$, then $\octx \Reduces\osim\ctxc{\reflc{\simu{R}}} \lctx'$.
%   \end{thmdescription}
% \end{lemma}
% \begin{proof}
%   \begin{description}
%   \item[Immediate output bisimulation]
%     Assume that $\octx \ctxc{\reflc{\simu{R}}} \lctx = \atmL{\lctx}'_L \oc \lctx' \oc \atmR{\lctx}'_R$; we must show that $\octx \Reduces\lrframe{\atmL{\lctx}'_L}{\ctxc{\reflc{\simu{R}}}}{\atmR{\lctx}'_R} \lctx$.
%     There are two cases: either the $\reflc{\simu{R}}$-related components of $\octx$ and $\lctx$ are equal or merely $\simu{R}$-related.
%     \begin{itemize}
%     \item If the $\reflc{\simu{R}}$-related components of $\octx$ and $\lctx$ are in fact equal, then so are $\octx$ and $\lctx$.
%       $\octx \ctxc{\reflc{\simu{R}}} \lctx$
%     \item
%       Otherwise, the $\reflc{\simu{R}}$-related components of $\octx$ and $\lctx$ are merely $\simu{R}$-related, with $\octx \ctxc{\simu{R}} \lctx = \atmL{\lctx}'_L \oc \lctx' \oc \atmR{\lctx}'_R$.
%       There are four subcases according to whether $\atmL{\lctx}'_L$ and $\atmR{\lctx}'_R$ are empty.
%       \begin{itemize}
%       \item Consider the subcase in which both $\atmL{\lctx}'_L$ and $\atmR{\lctx}'_R$ are empty;
%         in this subcase, we must show that $\octx \Reduces\ctxc{\reflc{\simu{R}}} \lctx$.
%         Because $\ctxc{\reflc{\simu{R}}}$ includes $\ctxc{\simu{R}}$, that follows directly.

%       \item Consider the subcase in which both $\atmL{\lctx}'_L$ and $\atmR{\lctx}'_R$ are nonempty.
%         The context $\lctx$ therefore exposes output atoms at its left and right ends, and so the $\ctxc{\simu{R}}$-related contexts $\octx$ and $\lctx$ must, in fact, be $\simu{R}$-related, for otherwise at least one end of $\lctx$ would expose an input, not output, atom.
%         In other words, $\octx \simu{R} \lctx = \atmL{\lctx}'_L \oc \lctx' \oc \atmR{\lctx}'_R$.
%         Because $\simu{R}$ is a labeled bisimulation up to reflexivity and contect, $\octx \Reduces\lrframe{\atmL{\lctx}'_L}{\ctxc{\reflc{\simu{R}}}}{\atmR{\lctx}'_R} \lctx$ follows from the immediate output property.

%       \item Consider the subcase in which $\atmL{\lctx}'_L$ is nonempty and $\atmR{\lctx}'_R$ is empty;
%         in this subcase, we must show that $\octx \Reduces\lrframe{\atmL{\lctx}'_L}{\ctxc{\reflc{\simu{R}}}}{} \lctx$.
%         Similarly to the previous case, $\lctx$ exposes output atoms at its left end because $\atmL{\lctx}'_L$ is nonempty.
%         And so the $\ctxc{\simu{R}}$-related contexts $\octx$ and $\lctx$ must, in fact, be $\lrframe{}{\simu{R}}{\atmL{\octx}_R}$-related, for some input context $\atmL{\octx}_R$.
%         In other words, $\octx \lrframe{}{\simu{R}}{\atmL{\octx}_R} \lctx = \atmL{\lctx}'_L \oc \lctx'$.

%         \begin{itemize}
%         \item Suppose that the left edge of $\atmL{\octx}_R$ does not extend into $\atmL{\lctx}'_L$.
%           Because $\simu{R}$ is a labeled bisimulation up to reflexivity and context, we may appeal to the immediate output property after framing off $\atmL{\octx}_R$ -- we deduce $\octx \lrframe{}{(\Reduces\lrframe{\atmL{\lctx}'_L}{\ctxc{\reflc{\simu{R}}}}{})}{\atmL{\octx}_R} \lctx$.
%           Reduction is closed under framing, so $\octx \Reduces\lrframe{\atmL{\lctx}'_L}{\ctxc{\reflc{\simu{R}}}}{} \lctx$.

%         \item Otherwise, suppose that the left edge of $\atmL{\octx}_R$ does extend into $\atmL{\lctx}'_L$ -- in other words, that $\atmL{\lctx}'_L = \atmL{\lctx}''_L \oc \atmL{\octx}'_L$ and $\atmL{\octx}_R = \atmL{\octx}'_L \oc \lctx'$ for some $\atmL{\octx},_L$.
%           We therefore have $\octx \lrframe{}{\simu{R}}{\atmL{\octx}_R} \lctx = \atmL{\lctx}''_L \oc \atmL{\octx}_R$.
%           Because $\simu{R}$ is a labeled bisimulation up to reflexivity and context, we may appeal to the immediate output property after framing off $\atmL{\octx}_R$ -- we deduce $\octx \lrframe{}{(\Reduces\lrframe{\atmL{\lctx}''_L}{\ctxc{\reflc{\simu{R}}}}{})}{\atmL{\octx}_R} \lctx$.
%           Reduction is closed under framing, so $\octx \Reduces\lrframe{\atmL{\lctx}''_L}{\ctxc{\reflc{\simu{R}}}}{\atmL{\octx}_R} \lctx$.

%           Respelling $\atmL{\octx}_R$ as $\atmL{\octx}'_L \oc \lctx'$, we have $\octx \Reduces\lrframe{\atmL{\lctx}''_L}{\lrframe{}{\ctxc{\reflc{\simu{R}}}}{\atmL{\octx}'_L}}{\lctx'} \lctx = \atmL{\lctx}''_L \oc \atmL{\octx}'_L \oc \lctx'$.
%           Once again, because $\simu{R}$ is a labeled bisimulation up to reflexivity and context, we may appeal to the emptiness property after framing off $\atmL{\lctx}''_L$ and $\lctx'$ -- we deduce $\octx \Reduces\lrframe{\atmL{\lctx}''_L}{(\Reduces\lrframe{\atmL{\octx}'_L}{\ctxc{\reflc{\simu{R}}}}{})}{\lctx'} \atmL{\lctx}''_L \oc \atmL{\octx}'_L \oc \lctx' = \lctx$.
%           Again, reduction is closed under framing, so $\octx \Reduces\lrframe{\atmL{\lctx}'_L}{\ctxc{\reflc{\simu{R}}}}{} \lctx$.
%         \end{itemize}
%       \item The subcase in which $\atmL{\lctx}'_L$ is nonempty and $\atmR{\lctx}'_R$ is empty.
%       \end{itemize}
%     \end{itemize}

%   \item[Reduction bisimulation]
%     Assume that $\octx \ctxc{\reflc{\simu{R}}}\reduces \lctx'$.
%     The $\reflc{\simu{R}}$-related components are either equal or $\simu{R}$-related.
%     In the latter case, either the reduction arises from the $\simu{R}$-related component alone, or it arises from an input transition of the $\simu{R}$-component that has its input demands met by the framing environment. 
%     There are thus three cases:
%     \begin{itemize}
%     \item Consider the case in which the $\reflc{\simu{R}}$-related components are in fact equal -- that is, the case in which $\octx \ctxc{=}\reduces \lctx'$.
%       $\octx \reduces \lctx'$.
%       Because $\ctxc{\reflc{\simu{R}}}$ is reflexive, $\octx \Reduces\ctxc{\reflc{\simu{R}}} \lctx'$.
      
%     \item Consider the case in which the reduction arises from the $\simu{R}$-related component alone -- that is, the case in which $\octx \ctxc{\simu{R}\reduces} \lctx'$.
%       Because $\simu{R}$ satisfies reduction bisimulation up to reflexivity and context, $\octx \ctxc{\Reduces\ctxc{\reflc{\simu{R}}}} \lctx'$.
%       Reduction is closed under framing and $\ctxc{}$ is an idempotent operation, so $\octx \Reduces\ctxc{\reflc{\simu{R}}} \lctx'$.
  
%     \item Consider the case in which the reduction arises from an input transition of the $\simu{R}$-related component that has had its input demands met by the framing environment -- that is, the case in which [...].
%       Because $\simu{R}$ satisfies immediate input bisimulation up to reflexivity and context, it follows that $\octx \ctxc{\Reduces\ctxc{\reflc{\simu{R}}}} \lctx'$.
%       Once again, reduction is closed under framing and $\ctxc{}$ is an idempotent operation, so $\octx \Reduces\ctxc{\reflc{\simu{R}}} \lctx'$.
%     \end{itemize}
%   \end{description}
% \end{proof}


% \section{}

% Given a binary relation $\simu{R}$, let $\simu{R}^*$ be the least \emph{reflexive} relation containing $\lrframe{\atmR{a}}{\simu{R}}{}$ for all $\atmR{a}$.
% If $\simu{R}$ is a blabeled bisimulation, then so is $\simu{R}^*$.
% \begin{description}
% \item[Immediate output bisimulation]
%   Assume that $\octx \lrframe{\atmR{a}}{\simu{R}}{} \lctx = \atmL{\lctx}'_L \oc \lctx' \oc \atmR{\lctx}'_R$ for some $\atmR{a}$; we must show that $\octx \Reduces\lrframe{\atmL{\lctx}'_L}{\simu{R}^*}{\atmR{\lctx}'_R} \lctx$.
%   \begin{itemize}
%   \item $\lctx'$ is nonempty.
%   \item $\lctx'$ is empty.
%   \end{itemize}

% \item[Reduction bisimulation]
%   Assume that $\octx \lrframe{\atmR{a}}{\simu{R}}{}\reduces \lctx'$.
%   \begin{itemize}
%   \item $\octx \lrframe{\atmR{a}}{(\simu{R}\reduces)}{} \lctx'$
%   \item $\octx \lrframe{\atmR{a}}{\simu{R}}{} \atmR{a} \oc \lctx$ where $\ireduces{\atmR{a} \oc #1}{\lctx}{\lctx'}$
%   \end{itemize}

% \item[Immediate input bisimulation]
%   Assume that $\octx \lrframe{\atmR{a}}{\simu{R}}{} \lctx$ and $\ireduces{\atmR{\lctx}_L \oc #1 \oc \atmL{\lctx}_R}{\lctx}{\lctx'}$; we must show that $\atmR{\lctx}_L \oc \octx \oc \atmL{\lctx}_R \Reduces\simu{R}^* \lctx'$.
%   \begin{itemize}
%   \item $\lctx = \atmR{a} \oc \lctx_0$ and $\ireduces{\atmR{\lctx}_L \oc \atmR{a} \oc #1 \oc \atmL{\lctx}_R}{\lctx_0}{\lctx'}$
%   \item $\lctx = \atmR{a} \oc \lctx_0$ and $\ireduces{#1 \oc \atmL{\lctx}_R}{\lctx_0}{\lctx'_0}$
%   \end{itemize}
% \end{description}

% If $\simu{R}$ is a labeled bisimulation, then $\ctxc{\simu{R}}$ satisfies immediate output and reduction properties.
% \begin{description}
% \item[Immediate output]
%   Assume that $\octx \lrframe{\atmR{a}}{\ctxc{\simu{R}}}{} \lctx = \atmL{\lctx}'_L \oc \lctx' \oc \atmR{\lctx}'_R$.
%   Inductive hypothesis.
%   Lemma: $\octx \Reduces\reflc{\lrframe{\atmR{a}}{\ctxc{\simu{R}}}{}} \lctx$.
% \item[Reduction]
%   Assume that $\octx \lrframe{\atmR{a}}{\ctxc{\simu{R}}}{}\reduces \lctx'$.
%   By the inductive hypothesis, $\ctxc{\simu{R}}$ satisfies the reduction property.
%   So, the above lemma yields $\octx \Reduces\reflc{\lrframe{\atmR{a}}{\ctxc{\simu{R}}}{}} \lctx'$.
%   \begin{itemize}
%   \item $\octx \ctxc{\Reduces\simu{R}^*} \lctx'$
%   \end{itemize}
% \end{description}


% \section{}

The proof of soundness is clean, but it does wind through a few \lcnamecrefs{lem:labeled-bisim-union}.
The first of these describes a condition under which the union of two relations is a labeled bisimulation.
%
\begin{lemma}\label{lem:labeled-bisim-union}
  Let $\simu{S}$ be a labeled bisimulation.
  If $\simu{R}$ progresses to $\simu{R} \union \simu{S}$, then $\simu{R} \union \simu{S}$ is also a labeled bisimulation.
\end{lemma}
\begin{proof}
  When $\simu{S}$ is a labeled bisimulation and $\simu{R}$ progresses to $\simu{R} \union \simu{S}$, then the relation $\simu{R} \union \simu{S}$ progresses to itself, \ie, $\simu{R} \union \simu{S}$ is 
  % satisfies the conditions required of
  a labeled bisimulation.
  If $\octx \mathrel{(\simu{R} \union \simu{S})} \lctx$ because $\octx$ and $\lctx$ are $\simu{R}$-related, then the conditions for progressing to $\simu{R} \union \simu{S}$ are satisfied by $\simu{R}$ progressing to $\simu{R} \union \simu{S}$.
  If, on the other hand, $\octx \mathrel{(\simu{R} \union \simu{S})} \lctx$ because $\octx$ and $\lctx$ are $\simu{S}$-related, then the conditions for progressing to $\simu{R} \union \simu{S}$ are satisfied by the fact that $\simu{S}$ is a labeled bisimulation.
\end{proof}

Next, we use this result to prove that framing a single input atom onto a labeled bisimulation results in a binary relation that does not stray too far from a labeled bisimulation.
\begin{lemma}\label{lem:single-input-atom}
  If $\simu{R}$ is a labeled bisimulation, then so are $\lframe{\atmR{a}}{\simu{R}} \union \simu{R}$ and $\rframe{\simu{R}}{\atmL{a}} \union \simu{R}$, for all $\atmR{a}$ and $\atmL{a}$, respectively.
\end{lemma}
\begin{proof}
  Let $\simu{R}$ be a labeled bisimulation.
  We shall prove that $\lframe{\atmR{a}}{\simu{R}} \union \simu{R}$ is a labeled bisimulation; the proof for $\rframe{\simu{R}}{\atmL{a}} \union \simu{R}$ is symmetric.

  According to \cref{lem:labeled-bisim-union}, because $\simu{R}$ is a labeled bisimulation, it suffices to show that $\lframe{\atmR{a}}{\simu{R}}$ progresses to $\lframe{\atmR{a}}{\simu{R}} \union \simu{R}$.
  We prove each property in turn.
  \begin{description}
  \item[Immediate output bisimulation]
    Assume that $\octx \lframe{\atmR{a}}{\simu{R}} \lctx = \atmL{\lctx}'_L \oc \lctx' \oc \atmR{\lctx}'_R$; we must show that $\octx \Reduces\lrframe[\big]{\atmL{\lctx}'_L}{(\lframe{\atmR{a}}{\simu{R}} \union \simu{R})}{\atmR{\lctx}'_R} \lctx$.
    Because the input atom $\atmR{a}$ cannot be unified with the output atoms $\atmL{\lctx}'_L$, the context $\atmL{\lctx}'_L$ must be empty.
    We distinguish cases on the size of $\lctx'$.
    \begin{itemize}
    \item
      Consider the case in which $\lctx'$ is nonempty.
      Because $\simu{R}$ is a labeled bisimulation, we may appeal to its immediate output bisimulation property after framing off $\atmR{a}$ and deduce that $\octx \lframe[\big]{\atmR{a}}{(\Reduces\rframe{\simu{R}}{\atmR{\lctx}'_R})} \lctx$.
      Reduction is closed under framing, so we conclude that $\octx \Reduces\rframe{\lframe{\atmR{a}}{\simu{R}}}{\atmR{\lctx}'_R} \lctx$, as required.%
      \begin{marginfigure}[-8\baselineskip]
        $
        \phantom{\octx = {}}
        \begin{tikzcd}[%
          /tikz/column 1/.append style={anchor=base east},
          /tikz/column 2/.append style={anchor=base east}
        ]
          \mathllap{\octx = {}} \atmR{a} \oc \octx_0
            \rar[relation, "\lframe{\atmR{a}}{\simu{R}}"]
            \dar[Reduces]
          &
          \atmR{a} \oc \lctx'_0 \oc \atmR{\lctx}'_R \mathrlap{{} = \lctx}
          \\
          \atmR{a} \oc \octx'_0
            \urar[relation, "\rframe{\lframe{\atmR{a}}{\simu{R}}}{\atmR{\lctx}'_R}" {sloped, below}]
          % \\
          % \octx_0
          %   \rar[relation, "\simu{R}"]
          %   \dar[Reduces]
          % &
          % \lctx'_0 \oc \atmR{\lctx}'_R
          % \\
          % \octx'_0
          %   \urar[relation, "\rframe{\simu{R}}{\atmR{\lctx}'_R}" {sloped, below}]
        \end{tikzcd}
        \phantom{{} = \lctx}
        $
      \end{marginfigure}%
    %
    \item
      Consider the case in which $\lctx'$ is empty -- that is, the case in which $\octx \lframe{\atmR{a}}{\simu{R}} \lctx = \atmR{\lctx}'_R = \atmR{a} \oc \atmR{\lctx}''_R$ for some $\atmR{\lctx}''_R$.
      Because $\simu{R}$ is a labeled bisimulation, we may appeal to its immediate output bisimulation property after framing off $\atmR{a}$ and deduce that $\octx \lframe{\atmR{a}}{(\Reduces\rframe{\simu{R}}{\atmR{\lctx}''_R})} \atmR{\lctx}'_R$.
      Reduction is closed under framing, so $\octx \Reduces\rframe{\lframe{\atmR{a}}{\simu{R}}}{\atmR{\lctx}''_R} \atmR{\lctx}'_R$.
      After framing off $\atmR{\lctx}''_R$, we may subsequently appeal to the emptiness bisimulation property of $\simu{R}$ and deduce that $\octx \Reduces\rframe{(\Reduces\rframe{\simu{R}}{\atmR{a}})}{\atmR{\lctx}''_R} \atmR{\lctx}'_R$.
      Once again, reduction is closed under framing, so we conclude that $\octx \Reduces\rframe{\simu{R}}{\atmR{\lctx}'_R} \lctx$, as required.%
      \begin{marginfigure}[-11\baselineskip]
        $
        \phantom{\octx = {}}
        \begin{tikzcd}[%
          /tikz/column 1/.append style={anchor=base east},
          /tikz/column 2/.append style={anchor=base east}
        ]
          \mathllap{\octx = {}} \atmR{a} \oc \octx_0
            \rar[relation, "\lframe{\atmR{a}}{\simu{R}}"]
            \dar[Reduces]
          &
          \atmR{a} \oc \atmR{\lctx}''_R \mathrlap{{} = \atmR{\lctx}'_R = \lctx}
          \\
          \atmR{a} \oc \octx'_0
            \urar[relation, "\rframe{\lframe{\atmR{a}}{\simu{R}}}{\atmR{\lctx}''_R}" {sloped, below}]
            \dar[Reduces]
          \\
          \octx''_0 \oc \atmR{a}
            \arrow[relation, "\rframe{\rframe{\simu{R}}{\atmR{a}}}{\atmR{\lctx}''_R}" {sloped, below}, bend right]{uur}
        \end{tikzcd}
        \phantom{{} = \atmR{\lctx}'_R \oc \lctx}
        $
      \end{marginfigure}%
    \end{itemize}

  \item[Immediate input bisimulation]
    Assume that $\octx \lrframe{\atmR{a}}{\simu{R}}{} \lctx$ and $\ireduces{\atmR{\lctx}_L \oc #1 \oc \atmL{\lctx}_R}{\lctx}{\lctx'}$; we must show that $\atmR{\lctx}_L \oc \octx \oc \atmL{\lctx}_R \Reduces\mathrel{(\lframe{\atmR{a}}{\simu{R}} \union \simu{R})} \lctx'$.
    According to \cref{lem:input-framing}, there are two cases: either $\atmR{a}$ satisfies an input demand, or it does not participate in the given input transition.
    \begin{itemize}
    \item
      Consider the case in which $\atmR{a}$ does participate in the input transition -- that is, the case in which $\octx \lframe{\atmR{a}}{\simu{R}} \atmR{a} \oc \lctx_0 = \lctx$ and $\ireduces{\atmR{\lctx}_L \oc \atmR{a} \oc #1 \oc \atmL{\lctx}_R}{\lctx_0}{\lctx'}$, for some $\lctx_0$.
      Because $\simu{R}$ is a labeled bisimulation, we may appeal to its immediate input bisimulation property and deduce $\atmR{\lctx}_L \oc \octx \oc \atmL{\lctx}_R \Reduces\simu{R} \lctx'$, as required.
      \begin{marginfigure}[-8\baselineskip]
        $
        \begin{tikzcd}
          \atmR{\lctx}_L \oc \octx \oc \atmL{\lctx}_R
            \dar[phantom]{=}
          &[2em]
          \atmR{\lctx}_L \oc \lctx \oc \atmL{\lctx}_R
            \dar[phantom]{=}
          \\[-3ex]
          \atmR{\lctx}_L \oc \atmR{a} \oc \octx_0 \oc \atmL{\lctx}_R
            \rar[relation, "\lrframe{(\atmR{\lctx}_L \atmR{a})}{\simu{R}}{\atmL{\lctx}_R}"]
            \arrow[Reduces]{dd}
          &
          \atmR{\lctx}_L \oc \atmR{a} \oc \lctx_0 \oc \atmL{\lctx}_R
          \\[-4ex]
          &
          \atmR{\lctx}_L \oc \atmR{a} \oc [\lctx_0] \oc \atmL{\lctx}_R
            \dar[reduces]
          \\
          \octx\mathrlap{'}
            \rar[relation, "\simu{R}" {sloped, below}]
          &
          \lctx\mathrlap{'}
        \end{tikzcd}
        $
      \end{marginfigure}%
    %
    \item
      Consider the case in which $\atmR{a}$ does not participate in the input transition -- that is, the case in which $\atmR{\lctx}_L$ is empty and $\octx \lframe{\atmR{a}}{\simu{R}} \atmR{a} \oc \lctx_0 = \lctx$ and $\ireduces{#1 \oc \atmL{\lctx}_R}{\lctx_0}{\lctx'_0}$ and $\lctx' = \atmR{a} \oc \lctx'_0$, for some $\lctx_0$ and $\lctx'_0$.
      Because $\simu{R}$ is a labeled bisimulation, we may appeal to its immediate input bisimulation property after framing off $\atmR{a}$ and deduce that $\octx \oc \atmL{\lctx}_R \lframe{\atmR{a}}{(\Reduces\simu{R})} \lctx'$.
      Reduction is closed under framing, so we conclude that $\octx \oc \atmL{\lctx}_R \Reduces\lframe{\atmR{a}}{\simu{R}} \lctx'$, as required.%
      \begin{marginfigure}
        $
        \phantom{\octx \oc \atmL{\lctx}_R = {}}
        \begin{tikzcd}
          \mathllap{\octx \oc \atmL{\lctx}_R = {}}
          \atmR{a} \oc \octx_0 \oc \atmL{\lctx}_R
            \rar[relation, "\lframe{\atmR{a}}{\rframe{\simu{R}}{\atmL{\lctx}_R}}"]
            \arrow[Reduces]{ddd}
          &[1em]
          \atmR{a} \oc \lctx_0 \oc \atmL{\lctx}_R
          \mathrlap{{} = \lctx \oc \atmL{\lctx}_R}
          \\[-4ex]
          &
          \hphantom{\atmR{a} \oc} [\lctx_0] \oc \atmL{\lctx}_R
            \dar[reduces]
          \\
          &
          \lctx\mathrlap{'_0}
          \\[-4ex]
          \atmR{a} \oc \octx'_0
            \rar[relation, "\lframe{\atmR{a}}{\simu{R}}" {sloped, below}, shorten >=0.7em]
          &
          \mathllap{\atmR{a} \oc {}} \lctx\mathrlap{'_0 = \lctx'}
        \end{tikzcd}
        \phantom{{} = \lctx \oc \atmL{\lctx}\vphantom{_R}}
        $
      \end{marginfigure}%
    \end{itemize}

  \item[Reduction bisimulation]
    Assume that $\octx \lframe{\atmR{a}}{\simu{R}}\reduces \lctx'$ holds; we must show that $\octx \Reduces\mathrel{(\lframe{\atmR{a}}{\simu{R}} \union \simu{R})} \lctx'$.
    We distinguish cases on the origin of the given reduction.
    \begin{itemize}
    \item
      Consider the case in which the reduction arises from the $\simu{R}$-related component alone -- that is, the case in which $\octx \lframe{\atmR{a}}{(\simu{R}\reduces)} \lctx'$.
      Because $\simu{R}$ is a labeled bisimulation, we may appeal to its reduction bisimulation property after framing off $\atmR{a}$ and deduce that $\octx \lframe{\atmR{a}}{(\Reduces\simu{R})} \lctx'$.
      Reduction is closed under framing, so we conclude that $\octx \Reduces\lframe{\atmR{a}}{\simu{R}} \lctx'$, as required.%
      \begin{marginfigure}[-9\baselineskip]
        $
        \phantom{\octx = {}}
        \begin{tikzcd}
          \mathllap{\octx = {}}
          \atmR{a} \oc \octx_0
            \rar[relation, "\lframe{\atmR{a}}{\simu{R}}"]
            \dar[Reduces]
          &
          \atmR{a} \oc \lctx_0
            \dar[reduces]
          \\
          \atmR{a} \oc \octx'_0
            \rar[relation, "\lframe{\atmR{a}}{\simu{R}}" {sloped, below}]
          &
          \atmR{a} \oc \lctx'_0 \mathrlap{{} = \lctx'}
        \end{tikzcd}
        \phantom{{} = \lctx'}
        $
      \end{marginfigure}%
    \item
      Consider the case in which the reduction arises from an input transition on the $\simu{R}$-related component -- that is, the case in which $\octx \lframe{\atmR{a}}{\simu{R}} \atmR{a} \oc \lctx_0 = \lctx$ and $\ireduces{\atmR{a} \oc #1}{\lctx_0}{\lctx'}$, for some $\lctx_0$.
      Because $\simu{R}$ is a labeled bisimulation, we may appeal to its immediate input bisimulation property and deduce that $\octx \Reduces\simu{R} \lctx'$, as required.%
      \begin{marginfigure}[-8\baselineskip]
        $
        \phantom{\octx = {}}
        \begin{tikzcd}
          \mathllap{\octx = {}}
          \atmR{a} \oc \octx_0
            \rar[relation, "\lframe{\atmR{a}}{\simu{R}}"]
            \arrow[Reduces]{dd}
          &
          \atmR{a} \oc \lctx_0 \mathrlap{{} = \lctx}
          \\[-4ex]
          &
          \atmR{a} \oc [\lctx_0]
            \dar[reduces]
          \\
          \octx\mathrlap{'}
            \rar[relation, "\simu{R}" {sloped, below}]
          &
          \lctx\mathrlap{'}
        \end{tikzcd}
        \phantom{{} = \lctx}
        $
      \end{marginfigure}%
    \end{itemize}

  \item[Emptiness bisimulation]
    Assume that $\octx \lframe{\atmR{a}}{\simu{R}} (\octxe)$.
    This is, in fact, impossible because the empty context does not contain $\atmR{a}$.
  %
  \qedhere
  \end{description}
\end{proof}

Having proved the preceding \lcnamecref{lem:single-input-atom} about framing a single input atom, we can apply it inductively to prove that framing input contexts preserves labeled bisimulations. 
\begin{lemma}\label{lem:ctxc-labeled-bisim}
  If $\simu{R}$ is a labeled bisimulation, then so is $\ctxc{\simu{R}}$.
\end{lemma}
\begin{proof}
  Let $(\simu{S}_n)_{n \in \nats}$ be the indexed family of relations given by
  \begin{align*}
    \mathord{\simu{S}_0} &= \mathord{\simu{R}} \\
    \mathord{\simu{S}_{n+1}} &= \textstyle
                                  \parens[size=Big]{\bigunion_{\atmR{a}} \mathord{\lframe{\atmR{a}}{\simu{S}_n}}}
                                  \union \parens[size=Big]{\bigunion_{\atmL{a}} \mathord{\rframe{\simu{S}_n}{\atmL{a}}}}
                                  \union \mathord{S}_n
    \,.
  \end{align*}
  % We shall prove that $\mathord{\ctxc{\simu{R}}} = \bigunion_{n=0}^{\infty}{\mathord{\simu{S}_n}}$.
  %
  It is easy to prove by structural induction that each $\ctxc{\simu{R}}$-related pair of contexts is also $\simu{S}_n$-related for some natural number $n$; and so $\ctxc{\simu{R}}$ is contained within $\bigunion_{n=0}^{\infty}{\mathord{\simu{S}_n}}$.
  % 
  Conversely, using \cref{lem:single-input-atom}, it is equally easy to prove by induction on $n$ that each $\simu{S}_n$ is contained within $\ctxc{\simu{R}}$ and, moreover, that each $\simu{S}_n$ is a labeled bisimulation.

  Because each $\simu{S}_n$ is a labeled bisimulation, so is their least upper bound, namely $\bigunion_{n=0}^{\infty}{\mathord{\simu{S}_n}} = \mathord{\ctxc{\simu{R}}}$.
  %
  % For each $\ctxc{\simu{R}}$-related pair of contexts, $\octx$ and $\lctx$, there exists a least natural number $n$ for which those contexts are $\simu{S}_n$-related.
  % Specifically, if $\octx$ and $\lctx$ are $\ctxc{\simu{R}}$-related by virtue of $\octx \lrframe{\atmR{\lctx}_L}{\simu{R}}{\atmL{\lctx}_R} \lctx$, then $\card{\atmR{\lctx}_L} + \card{\atmL{\lctx}_R}$ is the least natural number $n$ for which $\octx$ and $\lctx$ are $\simu{S}_n$-related.
  %
  % Using \cref{lem:single-input-atom}, it is easy to prove by induction on $n$ that each $\simu{S}_n$ is contained within $\ctxc{\simu{R}}$ and, moreover, that each $\simu{S}_n$ is a labeled bisimulation.
  % Therefore, $\mathord{\ctxc{\simu{R}}} = \bigunion_{n=0}^{\infty}{\mathord{\simu{S}_n}}$.
  % Because each $\simu{S}_n$ is a labeled bisimulation, so is their least upper bound, $\ctxc{\simu{R}}$.
\end{proof}

Now we use this \lcnamecref{lem:ctxc-labeled-bisim}
% fact that $\ctxc{\simu{R}}$ is a labeled bisimulation whenever $\simu{R}$ is also one 
to prove that $\ctxc{\simu{R}}$ is a rewriting bisimulation if $\simu{R}$ is a labeled bisimulation.
\begin{theorem}\label{thm:labeled-proof-technique}
  If $\simu{R}$ is a labeled bisimulation, then rewriting bisimilarity contains $\simu{R}$.
\end{theorem}
\begin{proof}
  Let $\simu{R}$ be a labeled bisimulation.
  By \cref{lem:ctxc-labeled-bisim}, so is $\ctxc{\simu{R}}$.
  The relation $\ctxc{\simu{R}}$ is also a rewriting bisimulation, as we will show by proving each property in turn.
  (Notice, too, that $\ctxc{\simu{R}}$ is symmetric because $\simu{R}$ is.)
  \begin{description}[itemsep=\dimexpr\itemsep+\parsep\relax, parsep=0pt, listparindent=\parindent]
  \item[Output bisimulation]
    Assume that $\octx \ctxc{\simu{R}}\Reduces \atmL{\lctx}'_L \oc \lctx' \oc \atmR{\lctx}'_R$; we must show that $\octx \Reduces\lrframe{\atmL{\lctx}'_L}{\ctxc{\simu{R}}}{\atmR{\lctx}'_R} \atmL{\lctx}'_L \oc \lctx' \oc \atmR{\lctx}'_R$.

    As a labeled bisimulation, $\ctxc{\simu{R}}$ satisfies the reduction bisimulation property, so we deduce that $\octx \Reduces\ctxc{\simu{R}} \atmL{\lctx}'_L \oc \lctx' \oc \atmR{\lctx}'_R$.
    The relation $\ctxc{\simu{R}}$ also satisfies the immediate output bisimulation property, so we conclude that $\octx \Reduces\lrframe{\atmL{\lctx}'_L}{\ctxc{\simu{R}}}{\atmR{\lctx}'_R} \atmL{\lctx}'_L \oc \lctx' \oc \atmR{\lctx}'_R$, as required.%
      \begin{marginfigure}[-9\baselineskip]
        $
        \begin{tikzcd}
          \octx
            \rar[relation, "\ctxc{\simu{R}}"]
            \dar[Reduces]
          &[0.8em]
          \lctx
            \dar[Reduces]
          \\
          \octx\mathrlap{'}
            \rar[relation, "\ctxc{\simu{R}}" {sloped}]
            \dar[Reduces]
          &
          \atmL{\lctx}'_L \oc \lctx' \oc \atmR{\lctx}'_R
          \\
          \octx\mathrlap{''}
            \urar[relation, "\lrframe{\atmL{\lctx}'_L}{\ctxc{\simu{R}}}{\atmR{\lctx}'_R}" {sloped, below}, shorten <=0.2em]
        \end{tikzcd}
        \phantom{{} = \lctx'}
        $
      \end{marginfigure}%
  %
  \item[Input bisimulation]
    Assume that $\atmR{\lctx}_L \oc \octx \oc \atmL{\lctx}_R \lrframe{\atmR{\lctx}_L}{\ctxc{\simu{R}}}{\atmL{\lctx}_R}\Reduces \lctx'$; we must show that $\atmR{\lctx}_L \oc \octx \oc \atmL{\lctx}_R \Reduces\ctxc{\simu{R}} \lctx'$.

    Because $\ctxc{\simu{R}}$ is input contextual, we deduce that $\atmR{\lctx}_L \oc \octx \oc \atmL{\lctx}_R \ctxc{\simu{R}}\Reduces \lctx'$.
    As a labeled bisimulation, $\ctxc{\simu{R}}$ satisfies the reduction bisimulation property, so we conclude that $\atmR{\lctx}_L \oc \octx \oc \atmL{\lctx}_R \Reduces\ctxc{\simu{R}} \lctx'$, as required.%
    % Assume that $\octx \ctxc{\simu{R}} \lctx$ and $\ireduces{\atmR{\lctx}_L \oc #1 \oc \atmL{\lctx}_R}{\lctx}{\lctx'}$; we must show that $\atmR{\lctx}_L \oc \octx \oc \atmL{\lctx}_R \Reduces\ctxc{\simu{R}} \lctx'$.
    % 
    % By \cref{??}, the given input transition gives rise to a reduction: $\atmR{\lctx}_L \oc \octx \oc \atmL{\lctx}_R \lrframe{\atmR{\lctx}_L}{\ctxc{\simu{R}}}{\atmL{\lctx}_R}\reduces \lctx'$.
    % Because $\ctxc{\simu{R}}$ is input contextual, we deduce that $\atmR{\lctx}_L \oc \octx \oc \atmL{\lctx}_R \ctxc{\simu{R}}\reduces \lctx'$.
    % As a labeled bisimulation, $\ctxc{\simu{R}}$ satisfies the reduction bisimulation property, so we conclude that $\atmR{\lctx}_L \oc \octx \oc \atmL{\lctx}_R \Reduces\ctxc{\simu{R}} \lctx'$, as required.%
    \begin{marginfigure}[-7\baselineskip]
        $
        \begin{tikzcd}
          \atmR{\lctx}_L \oc \octx \oc \atmL{\lctx}_R
            \rar[relation, "\lrframe{\atmR{\lctx}_L}{\ctxc{\simu{R}}}{\atmL{\lctx}_R}"]
            \rar[relation, "\ctxc{\simu{R}}" {sloped, below}, bend right=20]
            \dar[Reduces]
          &[1.75em]
          \atmR{\lctx}_L \oc \lctx \oc \atmL{\lctx}_R
            \dar[Reduces]
          \\[2ex]
          \octx\mathrlap{'}
            \rar[relation, "\ctxc{\simu{R}}" {sloped, below}]
          &
          \lctx\mathrlap{'}
        \end{tikzcd}
        $
      \end{marginfigure}%
  %
  \qedhere
  \end{description}
\end{proof}

Rewriting bisimilarity therefore contains every labeled bisimulation and, in particular, the largest labeled bisimulation, namely labeled bisimilarity.
\begin{corollary}
  Labeled bisimilarity is sound and complete with respect to rewriting bisimilarity.
\end{corollary}

\newthought{As a simple example} of this labeled bisimilarity proof technique for rewriting bisimilarity, we shall now establish that $\atmR{a} \oc (\atmR{a} \limp \atmR{b})$ and $\atmR{b}$ are rewriting-bisimilar contexts.
Let $\simu{R}$ be the least symmetric binary relation for which $\atmR{a} \oc (\atmR{a} \limp \atmR{b}) \simu{R} \atmR{b}$ and $\atmR{b} \simu{R} \atmR{b}$ and $(\octxe) \simu{R} (\octxe)$ hold.
The relation $\simu{R}$ is a labeled bisimulation:
\begin{itemize}
\item The immediate output bisimulation condition holds because $\atmR{a} \oc (\atmR{a} \limp \atmR{b})$ can simulate $\atmR{b}$'s output of $\atmR{b}$ (with $\atmR{a} \oc (\atmR{a} \limp \atmR{b}) \reduces\rframe{\simu{R}}{\atmR{b}} \atmR{b}$) and the former makes no immediate outputs of its own.
  Moreover, $\atmR{b}$ and $\atmR{b}$ can simulate each other's output of $\atmR{b}$.
\item The immediate input bisimulation condition holds vacuously for the relation $\simu{R}$ because neither $\atmR{a} \oc (\atmR{a} \limp \atmR{b})$ nor $\atmR{b}$ accept any inputs on either side.
\item The reduction bisimulation condition holds because $\atmR{b}$ can simulate the reduction $\atmR{a} \oc (\atmR{a} \limp \atmR{b}) \reduces \atmR{b}$ trivially (with $\atmR{b} \Reduces\simu{R} \atmR{b}$).
\item The emptiness bisimulation condition holds trivially: $\atmR{\lctx} \Reduces\rframe{\simu{R}}{\atmR{\lctx}} \atmR{\lctx}$ for all $\atmR{\lctx}$ because $(\octxe) \simu{R} (\octxe)$, and symmetrically for all $\atmL{\lctx}$.
\end{itemize}
We may conclude from the above proof technique~\parencref{thm:labeled-proof-technique} that $\simu{R}$ is contained within rewriting bisimilarity and that $\atmR{a} \oc (\atmR{a} \limp \atmR{b})$ and $\atmR{b}$ are indeed bisimilar.

We can similarly prove that $\atmR{a} \limp (\atmR{c} \pmir \atmL{b})$ and $(\atmR{a} \limp \atmR{c}) \pmir \atmL{b}$ are rewriting-bisimilar by showing that the least symmetric relation $\simu{R}$ such that $\atmR{a} \limp (\atmR{c} \pmir \atmL{b}) \simu{R} (\atmR{a} \limp \atmR{c}) \pmir \atmL{b}$ and $\atmR{c} \simu{R} \atmR{c}$ and $(\octxe) \simu{R} (\octxe)$ is a labeled bisimulation.

Somewhat surprisingly, even $\atmR{a} \limp \up \dn (\atmR{c} \pmir \atmL{b})$ and $\up \dn (\atmR{a} \limp \atmR{c}) \pmir \atmL{b}$ are bisimilar.
This one is rather surprising because the $\up \dn$ shift is placed in two different locations: over $\mathord{-} \pmir \atmL{b}$ in the former, and over $\atmR{a} \limp \mathord{-}$ in the latter.
One might expect that the placement of $\up \dn$ and the different intermediate contexts that it induces would make it possible to distinguish $\atmR{a} \limp \up \dn (\atmR{c} \pmir \atmL{b})$from $\up \dn (\atmR{a} \limp \atmR{c}) \pmir \atmL{b}$.

But by using least symmetric relation $\simu{R}$ such that
$\atmR{a} \limp \up \dn (\atmR{c} \pmir \atmL{b}) \simu{R} \up \dn (\atmR{a} \limp \atmR{c}) \pmir \atmL{b}$ and
$\atmR{c} \pmir \atmL{b} \simu{R} \atmR{a} \oc \bigl(\up \dn (\atmR{a} \limp \atmR{c}) \pmir \atmL{b}\bigr)$ and
$\bigl(\atmR{a} \limp \up \dn (\atmR{c} \pmir \atmL{b})\bigr) \oc \atmL{b} \simu{R} \atmR{a} \limp \atmR{c}$ and
$\atmR{c} \simu{R} \atmR{c}$ and
$(\octxe) \simu{R} (\octxe)$,
we can prove that the two propositions are indistinguishable.
The labeled bisimulation $\simu{R}$ shows how the inputs protected by the $\up \dn$ double shifts are treated lazily in establishing the equivalence: the proposition $\atmR{c} \pmir \atmL{b}$ is $\simu{R}$-related to the context $\atmR{a} \oc \bigl(\up \dn (\atmR{a} \limp \atmR{c}) \pmir \atmL{b}\bigr)$, for example.



% $\atmR{a} \limp \up \dn (\atmR{c} \pmir \atmL{b}) \simu{R} \up \dn (\atmR{a} \limp \atmR{c}) \pmir \atmL{b}$
% and
% $\atmR{c} \pmir \atmL{b} \simu{R} \atmR{a} \oc \bigl(\up \dn (\atmR{a} \limp \atmR{c}) \pmir \atmL{b}\bigr)$
% and
% $\bigl(\atmR{a} \limp \up \dn (\atmR{c} \pmir \atmL{b})\bigr) \oc \atmL{b} \simu{R} \atmR{a} \limp \atmR{c}$
% and
% $(\atmR{c} \pmir \atmL{b}) \oc \atmL{b} \simu{R} \atmR{c}$
% and
% $\atmR{a} \oc (\atmR{a} \limp \atmR{c}) \simu{R} \atmR{c}$
% and
% $\atmR{c} \simu{R} \atmR{c}$
% and
% $\octxe \simu{R} \octxe$.
% The relation is a labeled bisimulation because:
% \begin{itemize}
% \item Notice that $\ireduces{#1 \oc \atmL{b}}{\up \dn (\atmR{a} \limp \atmR{c}) \pmir \atmL{b}}{\atmR{a} \limp \atmR{c}}$.
%   And indeed $\bigl(\atmR{a} \limp \up \dn (\atmR{c} \pmir \atmL{b})\bigr) \oc \atmL{b} \Reduces\simu{R} \atmR{a} \limp \atmR{c}$.
%   Similarly, notice that $\ireduces{\atmR{a} \oc #1}{\atmR{a} \limp \up \dn (\atmR{c} \pmir \atmL{b})}{\atmR{c} \pmir \atmL{b}}$.
%   And indeed $\atmR{a} \oc \bigl(\up \dn (\atmR{a} \limp \atmR{c}) \pmir \atmL{b}\bigr) \Reduces\simu{R} \atmR{c} \pmir \atmL{b}$.
% \item Notice that $\ireduces{\atmR{a} \oc #1}{\atmR{a} \limp \atmR{c}}{\atmR{c}}$.
%   And indeed $\atmR{a} \oc \bigl(\up \dn (\atmR{a} \limp \atmR{c}) \pmir \atmL{b}\bigr) \oc \atmL{b} \Reduces\simu{R} \atmR{c}$.
%   Notice that $\ireduces{\atmR{a} \oc #1}{\bigl(\atmR{a} \limp \up \dn (\atmR{c} \pmir \atmL{b})\bigr) \oc \atmL{b}}{(\atmR{c} \pmir \atmL{b}) \oc \atmL{b}}$.
%   And indeed $\atmR{a} \oc \bigl(\up \dn (\atmR{a} \limp \atmR{c}) \pmir \atmL{b}\bigr) \oc \atmL{b} \Reduces\simu{R} (\atmR{c} \pmir \atmL{b}) \oc \atmL{b}$.
% \end{itemize}

% \clearpage
\subsection{A simple up-to proof technique: Reflexivity}

As a slight enhancement of the above proof technique, we can consider a simple up-to technique: bisimilarity up to reflexivity.
Let us call a relation $\simu{R}$ a labeled bisimulation \vocab{up to reflexivity} if $\simu{R}$ progresses to its reflexive closure, which we write as $\reflc{\simu{R}}$.
%
% \begin{lemma}\label{lem:identity-labeled-bisim}
%   The identity relation is a labeled bisimulation.
% \end{lemma}
% \begin{proof}
%
% \end{proof}
%
\begin{theorem}\label{thm:bisim-technique-up-to-refl}
  If $\simu{R}$ is a labeled bisimulation up to reflexivity, then rewriting bisimilarity contains $\simu{R}$.
\end{theorem}
\begin{proof}
  Let $\simu{R}$ be a labeled bisimulation up to reflexivity.
  First, notice that the identity relation is a labeled bisimulation -- each of the labeled bisimulation conditions is trivially true of the identity relation.
  Then, it follows from \cref{lem:labeled-bisim-union} that $\reflc{\simu{R}}$, the reflexive closure of $\simu{R}$, is a labeled bisimulation.
  By \cref{thm:labeled-proof-technique}, we may conclude that rewriting bisimilarity contains $\reflc{\simu{R}}$ and hence $\simu{R}$.
\end{proof}

% \begin{lemma}
%   If $\simu{R}$ is a labeled bisimulation up to reflexivity, then $\lframe{\atmR{a}}{\reflc{\simu{R}}} \union \reflc{\simu{R}}$ and $\rframe{\reflc{\simu{R}}}{\atmL{a}} \union \reflc{\simu{R}}$ are labeled bisimulations, for all $\atmR{a}$.
% \end{lemma}
% \begin{proof}
%   Let $\simu{R}$ be a labeled bisimulation up to reflexivity.
%   We shall prove that $\lframe{\atmR{a}}{\reflc{\simu{R}}} \union \reflc{\simu{R}}$ is a labeled bisimulation; the proof for $\rframe{\reflc{\simu{R}}}{\atmL{a}} \union \reflc{\simu{R}}$ is symmetric.

%   First, notice that $\mathord{\lframe{\atmR{a}}{\reflc{\simu{R}}}} \union \mathord{\reflc{\simu{R}}} = \mathord{\lframe{\atmR{a}}{\simu{R}}} \union \mathord{\reflc{\simu{R}}}$.
%   Because $\simu{R}$ is a labeled bisimulation up to reflexivity, $\reflc{\simu{R}}$ is a labeled bisimulation \parencref{??}.
%   According to \cref{??}, to show that $\lframe{\atmR{a}}{\reflc{\simu{R}}} \union \reflc{\simu{R}}$ is a labeled bisimulation, it therefore suffices to show that $\lframe{\atmR{a}}{\simu{R}}$ progresses to $\lframe{\atmR{a}}{\reflc{\simu{R}}} \union \reflc{\simu{R}}$.
%   We prove each property in turn.
%   \begin{description}
%   \item[Immediate output bisimulation]
%     Assume that $\octx \lframe{\atmR{a}}{\simu{R}} \lctx = \atmL{\lctx}'_L \oc \lctx' \oc \atmR{\lctx}'_R$; we must show that $\octx \Reduces\lrframe{\atmL{\lctx}'_L}{(\lframe{\atmR{a}}{\reflc{\simu{R}}} \union \reflc{\simu{R}})}{\atmR{\lctx}'_R} \lctx$.
%     Because the input atom $\atmR{a}$ cannot be unified with the output atoms $\atmL{\lctx}'_L$, the context $\atmL{\lctx}'_L$ must be empty.
%     We distinguish cases on the size of $\lctx'$.
%     \begin{itemize}
%     \item
%       Consider the case in which $\lctx'$ is nonempty.
%       Because $\simu{R}$ is a labeled bisimulation up to reflexivity, we may appeal to its immediate output bisimulation property after framing off $\atmR{a}$ and deduce that $\octx \lframe{\atmR{a}}{(\Reduces\rframe{\reflc{\simu{R}}}{\atmR{\lctx}'_R})} \lctx$.
%       Reduction is closed under framing, so we conclude that $\octx \Reduces\rframe{\lframe{\atmR{a}}{\reflc{\simu{R}}}}{\atmR{\lctx}'_R} \lctx$, as required.
%     %
%     \item
%       Consider the case in which $\lctx'$ is empty -- that is, the case in which $\octx \lframe{\atmR{a}}{\simu{R}} \lctx = \atmR{\lctx}'_R = \atmR{a} \oc \atmR{\lctx}''_R$ for some $\atmR{\lctx}''_R$.
%       Because $\simu{R}$ is a labeled bisimulation up to reflexivity, we may appeal to its immediate output bisimulation property after framing off $\atmR{a}$ and deduce that $\octx \lframe{\atmR{a}}{(\Reduces\rframe{\reflc{\simu{R}}}{\atmR{\lctx}''_R})} \atmR{\lctx}'_R$.
%       Reduction is closed under framing, so $\octx \Reduces\rframe{\lframe{\atmR{a}}{\reflc{\simu{R}}}}{\atmR{\lctx}''_R} \atmR{\lctx}'_R$.
%       After framing off $\atmR{\lctx}''_R$, we may subsequently appeal to the emptiness bisimulation property of $\reflc{\simu{R}}$ and deduce that $\octx \Reduces\rframe{(\Reduces\rframe{\reflc{\simu{R}}}{\atmR{a}})}{\atmR{\lctx}''_R} \atmR{\lctx}'_R$.
%       Once again, reduction is closed under framing, so we conclude that $\octx \Reduces\rframe{\reflc{\simu{R}}}{\atmR{\lctx}'_R} \lctx$, as required.
%     \end{itemize}

%   \item[Immediate input bisimulation]
%     Assume that $\octx \lrframe{\atmR{a}}{\simu{R}}{} \lctx$ and $\ireduces{\atmR{\lctx}_L \oc #1 \oc \atmL{\lctx}_R}{\lctx}{\lctx'}$; we must show that $\atmR{\lctx}_L \oc \octx \oc \atmL{\lctx}_R \Reduces\mathrel{(\lframe{\atmR{a}}{\reflc{\simu{R}}} \union \reflc{\simu{R}})} \lctx'$.
%     According to \cref{??}, there are two cases: either $\atmR{a}$ does not participate in the given input transition, or it is an input demand that is already present.
%     \begin{itemize}
%     \item
%       Consider the case in which $\atmR{a}$ does not participate in the transition -- that is, the case in which $\atmR{\lctx}_L$ is empty and $\octx \lframe{\atmR{a}}{\simu{R}} \atmR{a} \oc \lctx_0 = \lctx$ and $\ireduces{#1 \oc \atmL{\lctx}_R}{\lctx_0}{\lctx'_0}$ and $\lctx' = \atmR{a} \oc \lctx'_0$, for some $\lctx_0$ and $\lctx'_0$.
%       Because $\simu{R}$ is a labeled bisimulation up to reflexivity, we may appeal to its immediate input bisimulation property after framing off $\atmR{a}$ and deduce that $\octx \oc \atmL{\lctx}_R \lframe{\atmR{a}}{(\Reduces\reflc{\simu{R}})} \lctx'$.
%       Reduction is closed under framing, so we conclude that $\octx \oc \atmL{\lctx}_R \Reduces\lframe{\atmR{a}}{\reflc{\simu{R}}} \lctx'$, as required.
%     %
%     \item
%       Consider the case in which $\atmR{a}$ does participate in the input transition -- that is, the case in which $\octx \lframe{\atmR{a}}{\simu{R}} \atmR{a} \oc \lctx_0 = \lctx$ and $\ireduces{\atmR{\lctx}_L \oc \atmR{a} \oc #1 \oc \atmL{\lctx}_R}{\lctx_0}{\lctx'}$, for some $\lctx_0$.
%       Because $\simu{R}$ is a labeled bisimulation up to reflexivity, we may appeal to its immediate input bisimulation property and deduce that $\atmR{\lctx}_L \oc \octx \oc \atmL{\lctx}_R \Reduces\reflc{\simu{R}} \lctx'$, as required.
%     \end{itemize}

%   \item[Reduction bisimulation]
%     Assume that $\octx \lframe{\atmR{a}}{\simu{R}}\reduces \lctx'$; we must show that $\octx \Reduces\mathrel{(\lframe{\atmR{a}}{\reflc{\simu{R}}} \union \reflc{\simu{R}})} \lctx'$.
%     We distinguish cases on the origin of the given reduction.
%     \begin{itemize}
%     \item
%       Consider the case in which the reduction arises from the $\simu{R}$-related component alone -- that is, the case in which $\octx \lframe{\atmR{a}}{(\simu{R}\reduces)} \lctx'$.
%       Because $\simu{R}$ is a labeled bisimulation up to reflexivity, we may appeal to its reduction bisimulation property after framing off $\atmR{a}$ and deduce that $\octx \lframe{\atmR{a}}{(\Reduces\reflc{\simu{R}})} \lctx'$.
%       Reduction is closed under framing, so we conclude that $\octx \Reduces\lframe{\atmR{a}}{\reflc{\simu{R}}} \lctx'$, as required.
%     \item
%       Consider the case in which the reduction arises from an input transition on the $\simu{R}$-related component -- that is, the case in which $\octx \lframe{\atmR{a}}{\simu{R}} \atmR{a} \oc \lctx_0 = \lctx$ and $\ireduces{\atmR{a} \oc #1}{\lctx_0}{\lctx'}$, for some $\lctx_0$.
%       Because $\simu{R}$ is a labeled bisimulation up to reflexivity, we may appeal to its immediate input bisimulation property and deduce that $\octx \Reduces\reflc{\simu{R}} \lctx'$, as required.
%     \end{itemize}

%   \item[Emptiness bisimulation]
%     Assume that $\octx \lframe{\atmR{a}}{\simu{R}} \octxe$.
%     This is, in fact, impossible because the empty context does not contain $\atmR{a}$.
%   %
%   \qedhere
%   \end{description}
% \end{proof}


% \begin{lemma}
%   If $\simu{R}$ is a labeled bisimulation up to reflexivity and context, then $\lframe{\atmR{a}}{\ctxc{\reflc{\simu{R}}}} \union \ctxc{\reflc{\simu{R}}}$ is a labeled bisimulation.
% \end{lemma}
% \begin{proof}
%   \begin{description}
%   \item[Output] Assume that $\octx \lframe{\atmR{a}}{\ctxc{\simu{R}}} \lctx = \atmL{\lctx}'_L \oc \lctx' \oc \atmR{\lctx}'_R$.
%     $\octx_0 \ctxc{\simu{R}} \lctx'_0 \oc \atmR{\lctx}'_R$.
%     \begin{itemize}
%     \item $\octx \lframe{\atmR{a}}{(\Reduces\rframe{\reflc{\simu{R}}}{\atmR{\lctx}'_R})} \lctx$.
%     \item $\octx \lframe{\atmR{a}}{(\Reduces\rframe{\reflc{\simu{R}}}{\atmR{\lctx}''_R})} \lctx$.
%       $\octx \Reduces\rframe{\lframe{\atmR{a}}{\reflc{\simu{R}}}}{\atmR{\lctx}''_R} \lctx$.
%     \end{itemize}
%     Assume that $\octx \simu{R} \lctx = ...$.
%     $\octx \Reduces\lrframe{\atmL{\lctx}'_L}{\reflc{\simu{R}}}{\atmR{\lctx}'_R} \lctx$.
%   \end{description}
% \end{proof}

% \begin{theorem}
%   If $\simu{R}$ is a labeled bisimulation up to reflexivity, then $\ctxc{\reflc{\simu{R}}}$ is a labeled bisimulation.
% \end{theorem}
% \begin{proof}
%   \begin{align*}
%     \mathord{\simu{S}_0} &= \mathord{\reflc{\simu{R}}} \\
%     \mathord{\simu{S}_{n+1}} &= \mathord{\reflc{\simu{S}_n}} \union \parens[size=Big]{\bigunion_{\atmR{a}} \lframe{\atmR{a}}{\reflc{\simu{S}_n}}} \union \parens[size=Big]{\bigunion_{\atmL{a}} \rframe{\reflc{\simu{S}_n}}{\atmL{a}}} 
%   \end{align*}
% \end{proof}



\subsection{Other properties of rewriting bisimilarity}

In addition to soundness and completeness of labeled bisimilarity with respect to rewriting bisimilarity, we also expect rewriting bisimilarity to be a (monoidal) congruence relation.

Rewriting bisimilarity is, indeed, an equivalence relation.
%
\begin{theorem}\label{thm:ordered-bisimilarity:equivalence}
  Rewriting bisimilarity is reflexive, symmetric, and transitive.
\end{theorem}
\begin{proof}
  The identity relation on contexts can be shown to be a bisimulation, so rewriting bisimilarity is reflexive.
  Rewriting bisimilarity is symmetric by definition.
  The relation $\osim\osim$ can be shown to be a bisimulation, so rewriting bisimilarity is also transitive.
\end{proof}


% \begin{equation*}
%   \begin{lgathered}
%     \congsimu*{\octxe}{\simu{R}} = \mathord{\simu{R}} \\
%     \congsimu*{\lctx_L \oc \atmR{a}}{\simu{R}} = \mathord{\simu{S}}(\lctx_L, \lframe{\atmR{a}}{\simu{R}} \union \simu{R}) \\
%     \mathord{\simu{S}}(\lctx_L \oc \atmL{a}, \simu{R}) = \mathord{\simu{S}}(\lctx_L, \lframe{\atmL{a}}{\simu{R}} \union \simu{R}) \\
%     \congsimu*{\lctx_L \oc \n{A}}{\simu{R}} = \congsimu*{\lctx_L}{\lframe{\n{A}}{\simu{R}} \union \bigcup_{\lctx'} \congsimu*{\lctx'}{\simu{R}}}
%   \end{lgathered}
% \end{equation*}
% Notice that $\congsimu*{\lctx_L}{\simu{R}}$ contains $\lframe{\lctx_L}{\simu{R}}$.

% \begin{theorem}
%   If $\simu{R}$ is a labeled bisimulation, then so is $\congsimu*{\lctx_L}{\simu{R}}$, for all contexts $\lctx_L$.
% \end{theorem}
% \begin{proof}
%   By structural induction on the context $\lctx_L$.

%   Assume that $\simu{R}$ is a labeled bisimulation;
%   we must show that so is $\congsimu*{\lctx_L}{\simu{R}}$.
%   \begin{itemize}[itemsep=\dimexpr\itemsep+\parsep\relax, parsep=0pt, listparindent=\parindent]
%   \item Consider the case in which $\lctx_L = (\octxe)$.
%     Because $\congsimu*{\octxe}{\simu{R}} = \mathord{\simu{R}}$, it is immediate that $\congsimu*{\octxe}{\simu{R}}$ is a labeled bisimulation.

%   \item Consider the case in which $\lctx_L = \lctx'_L \oc \atmR{a}$.
%     According to \cref{lem:single-input-atom}, $\lframe{\atmR{a}}{\simu{R}} \union \simu{R}$ is also a labeled bisimulation.
%     By the inductive hypothesis, we conclude that $\congsimu*{\lctx'_L}{\lframe{\atmR{a}}{\simu{R}} \union \simu{R}} = \congsimu*{\lctx'_L \oc \atmR{a}}{\simu{R}}$ is a labeled bisimulation.

%   \item Consider the case in which $\lctx_L = \lctx'_L \oc \atmL{a}$.
%     We will show that $\lframe{\atmL{a}}{\simu{R}} \union \simu{R}$ is a labeled bisimulation.
%     It will then follow from the inductive hypothesis that $\congsimu*{\lctx'_L}{\lframe{\atmL{a}}{\simu{R}} \union \simu{R}} = \congsimu*{\lctx'_L \oc \atmL{a}}{\simu{R}}$ is a labeled bisimulation.

%     To show that $\lframe{\atmL{a}}{\simu{R}} \union \simu{R}$ is a labeled bisimulation, it suffices, according to \cref{??}, to show that $\lframe{\atmL{a}}{\simu{R}}$ progresses to $\lframe{\atmL{a}}{\simu{R}} \union \simu{R}$.
%     \begin{description}
%     \item[Immediate output bisimulation]
%       Assume that $\octx \lframe{\atmL{a}}{\simu{R}} \lctx = \atmL{\lctx}'_L \oc \lctx' \oc \atmR{\lctx}'_R$;
%       we must show that $\octx \Reduces\lrframe[\bigl]{\atmL{\lctx}'_L}{\mathrel{(\lframe{\atmL{a}}{\simu{R}} \union \simu{R})}}{\atmR{\lctx}'_R} \lctx$.
%       We distinguish cases on whether $\atmL{\lctx}'_L$ is empty.
%       \begin{itemize}
%       \item Consider the subcase in which $\atmL{\lctx}'_L$ is empty.
%         Because $\atmL{a}$ cannot be unified with the right-directed atoms $\atmR{\lctx}'_R$, it must be that $\lctx'$ is nonempty.
%         Because $\simu{R}$ is a labeled bisimulation, we can appeal to its immediate output bisimulation property after framing off $\atmL{a}$ and deduce that $\octx \lframe[\bigl]{\atmL{a}}{\mathrel{(\Reduces\rframe{\simu{R}}{\atmR{\lctx}'_R})}} \lctx$.
%         Reduction is closed under framing, so we conclude that $\octx \Reduces\rframe{\lframe{\atmL{a}}{\simu{R}}}{\atmR{\lctx}'_R} \lctx$, as required.

%       \item Consider the subcase in which $\atmL{\lctx}'_L$ is nonempty -- that is, $\atmL{\lctx}'_L = \atmL{a} \oc \atmL{\lctx}''_L$ for some $\atmL{\lctx}''_L$.
%         Because $\simu{R}$ is a labeled bisimulation, we can appeal to its immediate output bisimulation property after framing off $\atmL{a}$ and deduce that $\octx \lframe[\bigl]{\atmL{a}}{\mathrel{(\Reduces\lrframe{\atmL{\lctx}''_L}{\simu{R}}{\atmR{\lctx}'_R})}} \lctx$.
%         Reduction is closed under framing, so we conclude that $\octx \Reduces\lrframe[\bigl]{(\atmL{a} \oc \atmL{\lctx}''_L)}{\simu{R}}{\atmR{\lctx}'_R} \lctx$, as required.
%       \end{itemize}
%     \end{description}

%   \item Consider the case in which $\lctx_L = \lctx'_L \oc \n{A}$.
%     In this case, $\congsimu*{\lctx_L}{\simu{R}} = \mathord{\lframe{\n{A}}{\simu{R}} \union \simu{X}}$.
%     \begin{description}
%     \item[Reduction bisimulation]
%       Assume that $\octx \lframe{\n{A}}{\simu{R}} \lctx \reduces \lctx'$;
%       we must show that $\octx \Reduces\mathrel{(\lframe{\n{A}}{\simu{R}} \union \simu{X})} \lctx'$.
%       \begin{itemize}
%       \item Consider the case in which the given reduction arises from an input transition on $\n{A}$ -- that is, $\lctx = \n{A} \oc \atmL{\lctx}_R \oc \lctx_0$ and $\ireduces{#1 \oc \atmL{\lctx}_R}{\n{A}}{\lctx'_A}$ and $\lctx'_A \oc \lctx_0 = \lctx'$, for some $\atmL{\lctx}_R$, $\lctx_0$, and $\lctx'_A$.
%         Because $\simu{R}$ is a labeled bisimulation, we can appeal to its immediate output bisimulation property and deduce that $\octx \Reduces\lframe{(\n{A} \oc \atmL{\lctx}_R)}{\simu{R}} \lctx$.
%         After carrying out the reduction involving $\n{A}$, we have $\octx \Reduces\reduces\lframe{\lctx'_A}{\simu{R}} \lctx'$.
%         Because $\congsimu*{\lctx'_A}{\simu{R}}$ contains $\lframe{\lctx'_A}{\simu{R}}$, it follows that $\octx \Reduces\congsimu{\lctx'_A}{\simu{R}} \lctx'$.

%         I would like to say that $\congsimu*{\lctx'_A}{\simu{R}}$ is a labeled bisimulation.
%         But I don't see how to use the inductive hypothesis -- $\lctx'_A$ doesn't seem to be smaller than $\lctx_L$.
%       \end{itemize}
%     \end{description}
%   \end{itemize}
% \end{proof}


At this point, we would like to prove, as a \lcnamecref{lem:ordered-bisimilarity:append-left}, that $\lframe{\lctx_L}{\simu{R}}$ is contained in some labeled bisimulation, for all contexts $\lctx_L$ and all labeled bisimulations $\simu{R}$.
Ideally, the proof that proceeds by induction (or possibly coinduction), decomposing the context $\lctx_L$ and framing each antecedent of $\lctx_L$ onto the relation, one at a time.
This would allow \cref{lem:single-input-atom} to be reused, and would also streamline other cases.

Unfortunately, such a proof has been elusive so far.
So, instead, we will prove the following \lcnamecref{lem:ordered-bisimilarity:append-left} by handling the context $\lctx_L$ all at once.
The proof rehashes cases from \cref{lem:single-input-atom} and is not particularly enlightening beyond what was already presented there.
For that reason, the majority of the following \lcnamecref{lem:ordered-bisimilarity:append-left}'s proof is postponed to \cref{app:bisim}.
%
\begin{lemma}\label{lem:ordered-bisimilarity:append-left}
  If $\simu{R}$ is a labeled bisimulation, then, for each context $\lctx_L$, there exists a labeled bisimulation that contains $\lframe{\lctx_L}{\simu{R}}$.
\end{lemma}
\begin{proof}
  Let $\mathord{\simu{S}}$ be the relation such that $\octx \simu{S} \lctx$ if, and only if, there exists a context $\lctx_L$ such that $\octx \lframe{\lctx_L}{\simu{R}} \lctx$.
  We will show that $\simu{S}$ is a labeled bisimulation.
  \begin{description}
  \item[Immediate output bisimulation]
    Assume that $\octx \lframe{\lctx_L}{\simu{R}} \lctx = \atmL{\lctx}'_L \oc \lctx' \oc \atmR{\lctx}'_R$;
    we must show that $\octx \Reduces\lrframe{\atmL{\lctx}'_L}{\simu{S}}{\atmR{\lctx}'_R} \lctx$.
    We distinguish three cases according to where the right edge of $\lctx_L$ occurs in $\lctx$.
    \begin{itemize}
    \item Consider the case in which the right edge of $\lctx_L$ occurs in either $\atmL{\lctx}'_L$ or $\lctx'$ -- that is, the case in which $\atmL{\lctx}'_L = \lctx_L \oc \atmL{\lctx}''_L$ for some $\atmL{\lctx}''_L$.
      Because $\simu{R}$ is a labeled bisimulation, we can appeal to its immediate output bisimulation property after framing off $\lctx_L$ and deduce that $\octx \lframe[\big]{\lctx_L}{\mathrel{(\Reduces\lrframe{\atmL{\lctx}''_L}{\simu{R}}{\atmR{\lctx}'_R})}} \lctx$.
      Reduction is closed under framing, so we may conclude that $\octx \Reduces\lrframe[\big]{(\lctx_L \atmL{\lctx}''_L)}{\simu{R}}{\atmR{\lctx}'_R} \lctx$, as required.

    \item Consider the case in which the right edge of $\lctx_L$ occurs in $\lctx'$ -- that is, the case in which $\lctx' = \lctx'_1 \oc \lctx'_2$ for some $\lctx'_1$ and $\lctx'_2$ such that $\lctx_L = \atmL{\lctx}'_L \oc \lctx'_1$.
      Because $\simu{R}$ is a labeled bisimulation, we can appeal to its immediate output bisimulation property after framing off $\lctx_L$ and deduce that $\octx \lframe[\big]{\lctx_L}{\mathrel{(\Reduces\rframe{\simu{R}}{\atmR{\lctx}'_R})}} \lctx$.
      Reduction is closed under framing, so $\octx \Reduces\lrframe[\big]{\atmL{\lctx}'_L}{\lframe{\lctx'_1}{\simu{R}}}{\atmR{\lctx}'_R} \lctx$, as required.

    \item Consider the case in which the right edge of $\lctx_L$ occurs in $\atmR{\lctx}'_R$ -- that is, the case in which $\atmR{\lctx}'_R = \atmR{\lctx}'_{R1} \oc \atmR{\lctx}'_{R2}$ for some $\atmR{\lctx}'_{R1}$ and $\atmR{\lctx}'_{R2}$ such that $\lctx_L = \atmL{\lctx}'_L \oc \lctx' \oc \atmR{\lctx}'_{R1}$.
      Because $\simu{R}$ is a labeled bisimulation, we can appeal to its immediate output bisimulation properties after framing off $\lctx_L$ and deduce that $\octx \lframe[\big]{\lctx_L}{\mathrel{(\Reduces\rframe{\simu{R}}{\atmR{\lctx}'_{R2}})}} \lctx$.
      Reduction is closed under framing, so $\octx \Reduces\lrframe{\lctx_L}{\simu{R}}{\atmR{\lctx}'_{R2}} \lctx = \lctx_L \oc \atmR{\lctx}'_{R2}$.
      We can also appeal to $\simu{R}$'s emptiness bisimulation property after framing off $(\atmL{\lctx}'_L \oc \lctx')$ and $\atmR{\lctx}'_{R2}$ and deduce that $\octx \Reduces\lrframe[\big]{(\atmL{\lctx}'_L \lctx')}{\mathrel{(\Reduces\rframe{\simu{R}}{\atmR{\lctx}'_{R1}})}}{\atmR{\lctx}'_{R2}} \lctx$.
      Once again, reduction is closed under framing, so we conclude that $\octx \Reduces\lrframe[\big]{\atmL{\lctx}'_L}{\lframe{\lctx'}{\simu{R}}}{\atmR{\lctx}'_R} \lctx$, as required.
    \end{itemize}

  \item[Immediate input bisimulation]
    Assume that $\octx \lframe{\lctx_L}{\simu{R}} \lctx$ and $\ireduces{\atmR{\lctx}_L \oc #1 \oc \atmL{\lctx}_R}{\lctx}{\lctx'}$;
    we must show that $\atmR{\lctx}_L \oc \octx \oc \atmL{\lctx}_R \Reduces\simu{S} \lctx'$.

    According to 
    \begin{itemize}
    \item 
      Consider the case in which $\lctx_L$ only consists of (possibly zero) right-directed messages.
      In this case, an input transition $\ireduces{\atmR{\lctx}_L \oc \lctx_L \oc #1 \oc \atmL{\lctx}_R}{\lctx_0}{\lctx'}$ exists, for some $\lctx_0$ such that $\octx \lframe{\lctx_L}{\simu{R}} \lctx_L \oc \lctx_0 = \lctx$.
      Because $\simu{R}$ is a labeled bisimulation, we may appeal to its immediate input bisimulation property to deduce that $\atmR{\lctx}_L \oc \octx \oc \atmL{\lctx}_R \Reduces\simu{R} \lctx'$, as required.

    \item
      Consider the case in which some part of $\lctx_L$ does not participate in the input transition -- that is, the case in which $\lctx_L = \lctx'_L \oc \atmR{\lctx}'_L$ and $\atmR{\lctx}_L$ is empty and $\ireduces{\atmR{\lctx}'_L \oc #1 \oc \atmL{\lctx}_R}{\lctx_0}{\lctx'_0}$ and $\lctx'_L \oc \lctx'_0 = \lctx'$.
      Because $\simu{R}$ is a labeled bisimulation, we may appeal to its immediate input bisimulation property after framing off $\lctx'_L$ and deduce that $\octx \oc \atmL{\lctx}_R \lframe{\lctx'_L}{\mathrel{(\Reduces\simu{R})}} \lctx'$.
      Reduction is closed under framing, so we conclude that $\octx \oc \atmL{\lctx}_R \Reduces\lframe{\lctx'_L}{\simu{R}} \lctx'$, as required.
    \end{itemize}

  \item[Reduction bisimulation]
    Assume that $\octx \lframe{\lctx_L}{\simu{R}} \lctx \reduces \lctx'$;
    we must show that $\octx \Reduces\simu{S} \lctx'$.
    We distinguish cases on the origin of the given reduction.
    \begin{itemize}
    \item
      Consider the case in which the reduction arises from the $\simu{R}$-related component alone -- that is, the case in which $\octx \lframe[\big]{\lctx_L}{\mathrel{(\simu{R}\reduces)}} \lctx'$.
      Because $\simu{R}$ is a labeled bisimulation, we may appeal to its reduction bisimulation property after framing off $\lctx_L$ and deduce that $\octx \lframe[\big]{\lctx_L}{\mathrel{(\Reduces\simu{R})}} \lctx'$.
      Reduction is closed under framing, so we may conclude that $\octx \Reduces\lframe{\lctx_L}{\simu{R}} \lctx'$, as required.

    \item
      Consider the case in which the reduction arises from an input transition on the $\simu{R}$-related component -- that is, the case in which $\lctx_L = \lctx'_L \oc \atmR{\lctx}_L$ and $\ireduces{\atmR{\lctx}_L \oc #1}{\lctx}{\lctx'_0}$ and $\lctx'_L \oc \lctx'_0 = \lctx'$ for some contexts $\lctx'_L$, $\atmR{\lctx}_L$, and $\lctx'_0$.
      Because $\simu{R}$ is a labeled bisimulation, we may appeal to its reduction bisimulation property after framing off $\lctx'_L$ and deduce that $\octx \lframe[\big]{\lctx'_L}{\mathrel{(\Reduces\simu{R})}} \lctx'$.
      Reduction is closed under framing, so we may conclude that $\octx \Reduces\lframe{\lctx'_L}{\simu{R}} \lctx'$, as required.
    \end{itemize}

  \item[Emptiness bisimulation]
    Assume that $\octx \lframe{\lctx_L}{\simu{R}} (\octxe)$;
    we must show that:
    $\atmR{\lctx} \oc \octx \Reduces\rframe{\simu{S}}{\atmR{\lctx}} \atmR{\lctx}$ for all $\atmR{\lctx}$; and $\octx \oc \atmL{\lctx} \lframe{\atmL{\lctx}}{\simu{S}} \atmL{\lctx}$ for all $\atmL{\lctx}$.

    It can only be that $\lctx_L$ is empty, and so $\octx \simu{R} (\octxe)$.
    Because $\simu{R}$ is a labeled bisimulation, it follows that: $\atmR{\lctx} \oc \octx \Reduces\rframe{\simu{R}}{\atmR{\lctx}} \atmR{\lctx}$ for all $\atmR{\lctx}$; and $\octx \oc \atmL{\lctx} \lframe{\atmL{\lctx}}{\simu{R}} \atmL{\lctx}$ for all $\atmL{\lctx}$.
  %
  \qedhere
  \end{description}
\end{proof}



% \begin{theorem} 
  
% \end{theorem}
% \begin{proof}
%   \begin{equation*}
%     \mathord{\simu{R}} = \Set{ (\lctx_L \oc \octx, \lctx_L \oc \lctx) \given \octx \osim \lctx }
%   \end{equation*}
%   \begin{description}
%   \item[Immediate output bisimulation]
%     Assume that $\lctx_L \oc \octx \simu{R} \lctx_L \oc \lctx = \atmL{\lctx}'_L \oc \lctx' \oc \atmR{\lctx}'_R$;
%     we must show that $\lctx_L \oc \octx \Reduces\lrframe{\atmL{\lctx}'_L}{\simu{R}}{\atmR{\lctx}'_R} \lctx_L \oc \lctx$.
    
%   \end{description}
% \end{proof}

% \begin{theorem}
%   \begin{itemize}
%   \item If $\simu{R}$ is a labeled bisimulation, then, for each context $\lctx_L$, there exists a labeled bisimulation that contains $\lframe{\lctx_L}{\simu{R}}$.
%   \item If $\simu{R}$ is a labeled bisimulation, then for each $\oante$ there exists a labeled bisimulation that contains $\lframe{\oante}{\simu{R}}$.
%   \end{itemize}
% \end{theorem}
% \begin{proof}
%   If $\lctx$ is empty, we are done.
%   Otherwise, $\lctx$ ends with some $\oante$.
%   \begin{itemize}
%   \item If $\oante$ is $\atmR{a}$, then $\lframe{\atmR{a}}{\simu{R}} \union \simu{R}$ is a labeled bisimulation.
%     By the inductive hypothesis, there exists a labeled bisimulation that contains $\lframe{\lctx}{\simu{R}} \union \lframe{\lctx_0}{\simu{R}}$.
%   \item If $\oante$ is $\atmL{a}$, then $\lframe{\atmL{a}}{\simu{R}}$
%   \end{itemize}

%   Let $\simu{S}$ be the relation 
%   \begin{equation*}
%     \mathord{\simu{S}}(\oante) =
%       \begin{cases*}
%         \lframe{\atmR{a}}{\simu{R}} \union \simu{R} & if $\oante = \atmR{a}$ \\
%         \lframe{\atmL{a}}{\simu{R}} \union \simu{R} & if $\oante = \atmL{a}$ \\
%         \lframe{\n{A}}{\simu{R}} \union \simu{R} & if $\oante = \n{A}$
%       \end{cases*}
%   \end{equation*}
% \end{proof}

% $\osim$ is a labeled bisimulation.
% So are $\lframe{\atmR{a}}{\osim} \union \osim$ and $\rframe{\osim}{\atmL{a}} \union \osim$.
% So $\osim$ contains them.
% In particular, $\osim$ contains $\lframe{\atmR{a}}{\osim}$ and symmetrically.

% It suffices to show that $\lframe{\n{A}}{\osim} \union \osim$ and $\rframe{\osim}{\n{A}} \union \osim$.

% $\octx \lframe{\n{A}}{\simu{R}} \lctx = \atmL{\lctx}'_L \oc \lctx' \oc \atmR{\lctx}'_R$


% \begin{theorem}
%   \begin{itemize}
%   \item If $\simu{R}$ is a labeled bisimulation, then so are $\lframe{\oante}{\simu{R}} \union \simu{R}$ and $\rframe{\simu{R}}{\oante} \union \simu{R}$, for all $\oante$.
%   \item If $\simu{R}$ is a labeled bisimulation, then so are $\lframe{\lctx_L}{\simu{R}} \union \simu{R}$ and $\rframe{\simu{R}}{\lctx_R} \union \simu{R}$, for all $\lctx_L$ and $\lctx_R$, respectively.
%   \end{itemize}
% \end{theorem}
% \begin{proof}
%   $\lctx = \oante_1 \oc \oante_2 \dotsm \oante_n$
%   and $\lctx_0^n = (\octxe)$ and $\lctx_{i+1}^n = \oante_{n-i} \lctx_i$
%   and $\lctx = \lctx_n^n$.

%   If $\simu{R}$ is a labeled bisimulation, then so is $\bigcup_{i=0}^n \lframe{\lctx_i^n}{\simu{R}}$.

%   By the inductive hypothesis, $\bigcup_{i=0}^n \lframe{\lctx_i}{\simu{R}}$ is a labeled bisimulation.

  
%   $\bigcup_{i=2}^n \lframe{\lctx_i}{\simu{R}}$ is a labeled bisimulation.
  

%   $\lctx_L = \oante \oc \lctx'_L$.
%   By the inductive hypothesis, $\lframe{\lctx'_L}{\simu{R}} \union \simu{R}$ is a labeled bisimulation.
%   $\mathord{\lframe{\lctx_L}{\simu{R}}} \union \mathord{\lframe{\oante}{\simu{R}}} \union \mathord{\simu{R}}$ is a labeled bisimulation.
  

%   Let $\simu{R}$ be a labeled bisimulation.
%   We shall prove that $\lframe{\n{A}}{\simu{R}} \union \simu{R}$ is a labeled bisimulation;
%   the proof for $\rframe{\simu{R}}{\n{A}} \union \simu{R}$ is symmetric.

%   If $\oante = $

%   \begin{description}
%   \item[Immediate output bisimulation]
%     Assume that $\octx \lframe{\oante}{\simu{R}} \lctx = \atmL{\lctx}'_L \oc \lctx' \oc \atmR{\lctx}'_R$;
%     we must show that $\octx \Reduces\lrframe[\big]{\atmL{\lctx}'_L}{\mathrel{(\lframe{\oante}{\simu{R}} \union \simu{R})}}{\atmR{\lctx}'_R} \lctx$.
%     \begin{itemize}
%     \item Consider the case in which $\oante$ is a negative proposition -- \ie, $\oante = \n{A}$, for some $\n{A}$.
%       Because the proposition $\n{A}$ cannot be unified with the output atoms $\atmL{\lctx}'_L$, the context $\atmL{\lctx}'_L$ must be empty.
%       Moreover, because $\n{A}$ cannot be unified with the output atoms $\atmR{\lctx}'_R$, the context $\lctx'$ must be \emph{non}empty -- that is, the proposition $\n{A}$ must occur at the left edge of $\lctx'$.
%       Because $\simu{R}$ is a labeled bisimulation, we may appeal to its immediate output bisimulation property after framing off $\n{A}$ and deduce that $\octx \lframe{\n{A}}{\mathrel{(\Reduces\rframe{\simu{R}}{\atmR{\lctx}'_R})}} \lctx$.
%       Reduction is closed under framing, so we conclude that $\octx \Reduces\rframe{\lframe{\n{A}}{\simu{R}}}{\atmR{\lctx}'_R} \lctx$, as required.

%     \item Consider the case in which $\oante$ is a left-directed atom -- \ie, $\oante = \atmL{a}$ for some $\atmL{a}$.
%       Unlike negative propositions, the atom $\atmL{a}$ can be unified with the output atoms $\atmL{\lctx}_L$.
%       By the same reasoning as 
%     \end{itemize}

%   \item[Immediate input bisimulation]
%     Assume that $\octx \lframe{\oante}{\simu{R}} \lctx$ and $\ireduces{\atmR{\lctx}_L \oc #1 \oc \atmL{\lctx}_R}{\lctx}{\lctx'}$;
%     we must show that $\atmR{\lctx}_L \oc \octx \oc \atmL{\lctx}_R \Reduces\mathrel{(\lframe{\oante}{\simu{R}} \union \simu{R})} \lctx'$.

%     According to \cref{??}, $\oante$ cannot participate in the given input transition.
%     That is, $\atmR{\lctx}_L$ is empty and $\octx \lframe{\oante}{\simu{R}} \oante \oc \lctx_0 = \lctx$ and $\ireduces{#1 \oc \atmL{\lctx}_R}{\lctx_0}{\lctx'_0}$ and $\lctx' = \oante \oc \lctx'_0$, for some contexts $\lctx_0$ and $\lctx'_0$.
%     Because $\simu{R}$ is a labeled bisimulation, we may appeal to its immediate input bisimulation property after framing off $\oante$ and deduce that $\octx \oc \atmL{\lctx}_R \lframe{\oante}{\mathrel{(\Reduces\simu{R})}} \lctx'$.
%     Reduction is closed under framing, so we conclude that $\octx \oc \atmL{\lctx}_R \Reduces\lframe{\oante}{\simu{R}} \lctx'$, as required.

%   \item[Reduction bisimulation]
%     Assume that $\octx \lframe{\oante}{\simu{R}}\reduces \lctx'$ holds;
%     we must show that $\octx \Reduces\mathrel{(\lframe{\oante}{\simu{R}} \union \simu{R})} \lctx'$.
%     We distinguish cases on the origin of the given reduction.
%     \begin{itemize}
%     \item Consider the case in which the reduction arises from the $\simu{R}$-related component alone -- that is, the case in which $\octx \lframe{\oante}{\mathrel{(\simu{R}\reduces)}} \lctx'$.
%       Because $\simu{R}$ is a labeled bisimulation, we may appeal to its reduction bisimulation property after framing off $\oante$ and deduce that $\octx \lframe{\oante}{\mathrel{(\Reduces\simu{R})}} \lctx'$.
%       Reduction is closed under framing, so we conclude that $\octx \Reduces\lframe{\oante}{\simu{R}} \lctx'$, as required.

%     \item Consider the case in which the reduction arises from an input transition on the $\simu{R}$-related conponent -- $\octx \lframe{\oante}{\simu{R}} \oante \oc \lctx_0$

%     \item Consider the case in which $\oante = \n{A}$ for some negative proposition $\n{A}$ and the reduction arises from $\n{A}$ -- that is, the case in which $\octx \lframe{\n{A}}{\simu{R}} \n{A} \oc \atmL{\lctx}_L \oc \lctx_0 = \lctx$ and $\ireduces{#1 \oc \atmL{\lctx}_L}{\n{A}}{\lctx'_L}$ and $\lctx'_L \oc \lctx_0 = \lctx'$, for some contexts $\atmL{\lctx}_L$ and $\lctx'_L$.
%       Because $\simu{R}$ is a labeled bisimulation, we may appeal to its immediate output bisimulation property after framing off $\n{A}$ and deduce that $\octx \lframe{\n{A}}{\mathrel{(\Reduces\lframe{\atmL{\lctx}_L}{\simu{R}})}} \lctx$.

%       $\octx_0 \Reduces\lframe{\atmL{\lctx}_L}{\simu{R}} \atmL{\lctx}_L \oc \lctx_0$.
%       $\n{A} \oc \octx_0 \Reduces\reduces \lctx'_L \oc \octx'_0 \lframe{\lctx'_L}{\simu{R}} \lctx'$

%       Reduction is closed under framing, so $\octx \Reduces\lframe{(\n{A} \oc \atmL{\lctx}_L)}{\simu{R}} \lctx$.
%       Owing to the input transition on $\n{A}$, we can $\octx \Reduces\reduces\lframe{\lctx'_L}{\simu{R}} \lctx'$.

%     \item Consider the case in which $\octx = \n{A} \oc \octx_R \lframe{\n{A}}{\simu{R}} \n{A} \oc (\atmL{\lctx}_L \oc \lctx_0) = \lctx$ and $\ireduces{#1 \oc \atmL{\lctx}_L}{\n{A}}{\lctx'_L}$.
%       We must show that $\octx \Reduces\mathrel{(\lframe{\n{A}}{\simu{R}} \union \simu{R})} \lctx'_L \oc \lctx_0 = \lctx'$.

%       By output bisimulation, $\octx_R \Reduces\lframe{\atmL{\lctx}_L}{\simu{R}} \atmL{\lctx}_L \oc \lctx_0$.
%       Framing $\n{A}$ on, $\n{A} \oc \octx_R \Reduces\lframe{(\n{A} \oc \atmL{\lctx}_L)}{\simu{R}} \n{A} \oc \atmL{\lctx}_L \oc \lctx_0$.
%       We can append a reduction: $\n{A} \oc \octx_R \Reduces\lframe{(\n{A} \oc \atmL{\lctx}_L)}{\simu{R}} \n{A} \oc \atmL{\lctx}_L \oc \lctx_0 \reduces \lctx'_L \oc \lctx_0$.
%       Therefore, $\n{A} \oc \octx_R \Reduces\lframe{\lctx'_L}{\simu{R}} \lctx'_L \oc \lctx_0 = \lctx'$.

%     \item The case in which $\oante = \atmL{a}$ for some left-directed atom $\atmL{a}$ and the reduction arises from $\oante$ is 
%     \end{itemize}


%   \item[Emptiness bisimulation]
%     Assume that $\octx \lframe{\oante}{\simu{R}} (\octxe)$.
%     This is, in fact, impossible because the empty context does not contain $\oante$.
%   \end{description}


%   \begin{description}
%   \item[Immediate output bisimulation]
%     Assume that $\octx \lframe{\n{A}}{\simu{R}} \lctx = \atmL{\lctx}'_L \oc \lctx' \oc \atmR{\lctx}'_R$;
%     we must show that $\octx \Reduces\lrframe[\big]{\atmL{\lctx}'_L}{\mathrel{(\lframe{\n{A}}{\simu{R}} \union \simu{R})}}{\atmR{\lctx}'_R} \lctx$.

%     Because the proposition $\n{A}$ cannot be unified with the output atoms $\atmL{\lctx}'_L$, the context $\atmL{\lctx}'_L$ must be empty.
%     Moreover, because $\n{A}$ cannot be unified with the output atoms $\atmR{\lctx}'_R$, the context $\lctx'$ must be \emph{non}empty -- that is, the proposition $\n{A}$ must occur at the left edge of $\lctx'$.
%     Because $\simu{R}$ is a labeled bisimulation, we may appeal to its immediate output bisimulation property after framing off $\n{A}$ and deduce that $\octx \lframe{\n{A}}{\mathrel{(\Reduces\rframe{\simu{R}}{\atmR{\lctx}'_R})}} \lctx$.
%     Reduction is closed under framing, so we conclude that $\octx \Reduces\rframe{\lframe{\n{A}}{\simu{R}}}{\atmR{\lctx}'_R} \lctx$, as required.

%   \item[Immediate input bisimulation]
%     Assume that $\octx \lframe{\n{A}}{\simu{R}} \lctx$ and $\ireduces{\atmR{\lctx}_L \oc #1 \oc \atmL{\lctx}_R}{\lctx}{\lctx'}$;
%     we must show that $\atmR{\lctx}_L \oc \octx \oc \atmL{\lctx}_R \Reduces\mathrel{(\lframe{\n{A}}{\simu{R}} \union \simu{R})} \lctx'$.

%     According to \cref{??}, $\n{A}$ does not participate in the given input transition.
%     That is, $\atmR{\lctx}_L$ is empty and $\octx \lframe{\n{A}}{\simu{R}} \n{A} \oc \lctx_0 = \lctx$ and $\ireduces{#1 \oc \atmL{\lctx}_R}{\lctx_0}{\lctx'_0}$ and $\lctx' = \n{A} \oc \lctx'_0$, for some contexts $\lctx_0$ and $\lctx'_0$.
%     Because $\simu{R}$ is a labeled bisimulation, we may appeal to its immediate input bisimulation property after framing off $\n{A}$ and deduce that $\octx \oc \atmL{\lctx}_R \lframe{\n{A}}{\mathrel{(\Reduces\simu{R})}} \lctx'$.
%     Reduction is closed under framing, so we conclude that $\octx \oc \atmL{\lctx}_R \Reduces\lframe{\n{A}}{\simu{R}} \lctx'$, as required.

%   \item[Reduction bisimulation]
%     Assume that $\octx \lframe{\n{A}}{\simu{R}}\reduces \lctx'$ holds;
%     we must show that $\octx \Reduces\mathrel{(\lframe{\n{A}}{\simu{R}} \union \simu{R})} \lctx'$.

%   \item[Emptiness bisimulation]
%     Assume that $\octx \lframe{\n{A}}{\simu{R}} (\octxe)$.
%     This is, in fact, impossible because the empty context does not contain $\n{A}$.
%   \end{description}
% \end{proof}

\begin{theorem}
  If $\octx_1 \osim \lctx_1$ and $\octx_2 \osim \lctx_2$, then $\octx_1 \oc \octx_2 \osim \lctx_1 \oc \lctx_2$.
\end{theorem}
\begin{proof}
  Assume that $\octx_1 \osim \lctx_1$ and $\octx_2 \osim \lctx_2$.
  Notice that $\octx_1 \oc \octx_2 \lframe{\octx_1}{\osim} \octx_1 \oc \lctx_2 \rframe{\osim}{\lctx_2} \lctx_1 \oc \lctx_2$.

  Rewriting bisimilarity is a labeled bisimulation~\parencref{thm:ordered-bisimilarity:labeled-complete}.
  By \cref{lem:ordered-bisimilarity:append-left}, there exists a labeled bisimulation that contains $\lframe{\octx_1}{\osim}$.
  According to \cref{thm:labeled-proof-technique}, rewriting bisimilarity therefore contains $\lframe{\octx_1}{\osim}$.
  By symmetric reasoning, rewriting bisimilarity must also contain $\rframe{\osim}{\lctx_2}$.

  Applying these to the previous observation, $\octx_1 \oc \octx_2 \osim \octx_1 \oc \lctx_2 \osim \lctx_1 \oc \lctx_2$.
  Because rewriting bisimilarity is transitive~\parencref{thm:ordered-bisimilarity:equivalence}, we conclude that $\octx_1 \oc \octx_2 \osim \lctx_1 \oc \lctx_2$.
\end{proof}

\begin{corollary}
  Rewriting bisimilarity is a congruence.
\end{corollary}

% \begin{lemma}
%   Rewriting bisimilarity contains $\lframe{\p{A}}{\osim}$ and $\rframe{\osim}{\p{A}}$, for all propositions $\p{A}$.
% \end{lemma}
% \begin{proof}
%   By induction over the structure of $\p{A}$.
%   We show only the proof for $\lframe{\p{A}}{\osim}$; the proof for $\rframe{\osim}{\p{A}}$ is symmetric.

%   According to \cref{??}, it suffices to show that $\lframe{\p{A}}{\osim} \union \osim$ is a labeled bisimulation.
%   By \cref{??}, we need only show that $\lframe{\p{A}}{\osim}$ progresses to $\lframe{\p{A}}{\osim} \union \osim$.
%   Many of the cases follow the pattern laid out in the proof of \cref{??}, substituting $\p{A}$ for $\atmR{a}$; we show only the new cases.
%   \begin{description}
%   \item[Immediate output bisimulation]
%     Assume that $\octx \lframe{\p{A}}{\osim} \lctx = \atmL{\lctx}'_L \oc \lctx' \oc \atmR{\lctx}'_R$;
%     we must show that $\octx \Reduces\lrframe[\big]{\atmL{\lctx}'_L}{(\lframe{\p{A}}{\osim} \union \osim)}{\atmR{\lctx}'_R} \lctx$.

%     Unlike in the proof of \cref{??}, here it is possible that $\p{A} = \atmL{a}$ with $\atmL{\lctx}'_L$ nonempty: $\atmL{\lctx}'_L = \atmL{a} \oc \atmL{\lctx}''_L$, for some $\atmL{\lctx}''_L$.
%     Because $\osim$ is a labeled bisimulation~\parencref{??}, we may appeal to its immediate output bisimulation property after framing $\atmL{a}$ and deduce that $\octx \lframe[\big]{\atmL{a}}{(\Reduces\lrframe{\atmL{\lctx}''_L}{\osim}{\atmR{\lctx}'_R})} \lctx$.
%     Reduction is closed under framing, so $\octx \Reduces\lrframe{\atmL{\lctx}'_L}{\osim}{\atmR{\lctx}'_R} \lctx$, as required.

%     The other cases follow the pattern laid out in the proof of \cref{??}.

%   \item[Immediate input bisimulation]
%     Assume that $\octx \lframe{\p{A}}{\osim} \lctx$ and $\ireduces{\atmR{\lctx}_L \oc #1 \oc \atmL{\lctx}_R}{\lctx}{\lctx'}$;
%     we must show that $\atmR{\lctx}_L \oc \octx \oc \atmL{\lctx}_R \Reduces\mathrel{\bigl(\lframe{\p{A}}{\osim} \union \osim\bigr)} \lctx'$.

%     All cases here follow the pattern laid out in the proof of \cref{??}.
    
%   \item[Reduction bisimulation]
%     Assume that $\octx \lframe{\p{A}}{\osim}\reduces \lctx'$;
%     we must show that $\octx \Reduces\mathrel{(\lframe{\p{A}}{\osim} \union \osim)} \lctx'$.

%     As in the proof of \cref{??}, we distinguish cases based on the origin of the reduction.
%     Here, there are three new cases because the reduction might originate from $\p{A}$; the other cases follow the pattern laid out in the proof of \cref{??}.
%     \begin{itemize}
%     \item
%       Consider the case in which $\p{A} = \p{A}_1 \fuse \p{A}_2$ and $\octx \lframe{\p{A}}{\osim} (\p{A}_1 \fuse \p{A}_2) \oc \lctx_0 \reduces \p{A}_1 \oc \p{A}_2 \oc \lctx_0 = \lctx'$ for some $\lctx_0$.
%       Notice that $\octx \reduces\lframe{\p{A}_1}{\lframe{\p{A}_2}{\osim}} \lctx'$.
%       By appealing to the inductive hypothesis for $\p{A}_2$, we deduce that rewriting bisimilarity contains $\lframe{\p{A}_2}{\osim}$, and so $\octx \reduces\lframe{\p{A}_1}{\osim} \lctx'$.
%       Similar reasoning for $\p{A}_1$ allows us to conclude that $\octx \reduces\osim \lctx'$, as required.

%     \item
%       Consider the case in which $\p{A} = \one$ and $\octx \lframe{\p{A}}{\osim} \one \oc \lctx' \reduces \lctx'$.
%       Notice that $\octx \reduces\osim \lctx'$, as required.

%     \item
%       Consider the case in which $\p{A} = \n{A}_0$ and $\octx \lframe[\big]{\p{A}}{\osim} \dn \n{A}_0 \oc \atmL{\lctx}_L \oc \lctx'_0 \reduces \p{C} \oc \lctx'_0 = \lctx'$ because $\lfocus{}{\n{A}_0}{\atmL{\lctx}_L}{\p{C}}$.
%       Because $\osim$ is a labeled bisimulation~\parencref{??}, we may appeal to immediate output bisimulation property after framing off $\dn \n{A}_0$ and deduce that $\octx \lframe[\big]{\p{A}}{(\Reduces\lframe{\atmL{\lctx}_L}{\osim})} \dn \n{A}_0 \oc \atmL{\lctx}_L \oc \lctx'_0$.
%       Reduction is closed under framing, so $\octx \Reduces\lframe[\big]{(\dn \n{A} \oc \atmL{\lctx}_L)}{\osim} \dn \n{A} \oc \atmL{\lctx}_L \oc \lctx'_0$.
%       We can insert the reduction $\dn \n{A} \oc \atmL{\lctx}_L \reduces \p{C}$ and arrive at $\octx \Reduces\lframe{\p{C}}{\osim} \p{C} \oc \lctx'_0 = \lctx'$.
%       The proposition $\p{C}$ is a subformula of $\dn \n{A}$, so, by the inductive hypothesis, $\lframe{\p{C}}{\osim}$ is contained within $\osim$.
%       It follows that $\octx \Reduces\osim \lctx'$, as required. 
%     \end{itemize}

%   \item[Emptiness bisimulation]
%     Assume that $\octx \lframe{\p{A}}{\osim} \octxe$.
%     As before, this is impossible.
%   %
%   \qedhere
%   \end{description}
% \end{proof}

% \begin{itemize}
% \item Assume that $\octx \lframe{\p{A}}{\osim} \lctx = \atmL{\lctx}'_L \oc \lctx' \oc \atmR{\lctx}'_R$;
%   we must show that $\octx \Reduces\lrframe[\big]{\atmL{\lctx}'_L}{(\lframe{\p{A}}{\osim} \union \osim)}{\atmR{\lctx}'_R} \lctx$.
% \item 
%   Assume that $\octx \lframe{(\p{A}_1 \fuse \p{A}_2)}{\osim} \lctx \reduces \p{A}_1 \oc \p{A}_2 \oc \lctx'_0 = \lctx'$;
%   we must show that $\octx \Reduces\mathrel{\bigl(\lframe{(\p{A}_1 \fuse \p{A}_2)}{\osim} \union \osim\bigr)} \lctx'$.
%   Notice that $\octx \reduces\lframe{\p{A}_1}{\lframe{\p{A}_2}{\osim}} \lctx'$.
%   By the inductive hypothesis on $\p{A}_1$ and $\p{A}_2$, respectively, $\octx \reduces\lframe{\p{A}_1}{\osim} \lctx'$, and then $\octx \reduces\osim \lctx'$.
% \item
%   Assume that $\octx \lframe{\one}{\osim} \lctx \reduces \lctx'$.
%   Notice that $\octx \reduces\osim \lctx'$.
% \item
%   Assume that $\octx \lframe{(\dn \n{A})}{\osim} (\dn \n{A}) \oc \atmL{\lctx}'_L \oc \lctx_0 = \lctx$ and $\lfocus{}{\n{A}}{\atmL{\lctx}'_L}{\p{C}}$ and $\lctx' = \p{C} \oc \lctx_9$.
%   Notice that $\octx \lframe[\big]{\dn \n{A}}{(\Reduces\lframe{\atmL{\lctx}'_L}{\osim})} \lctx$, and so $\octx \Reduces\reduces\lframe{\p{C}}{\osim} \lctx'$.
%   By the inductive hypothesis, $\octx \Reduces\reduces\osim \lctx'$.
% \end{itemize}


% \begin{theorem}
%   If $\simu{R}$ is a labeled bisimulation, then so is $\lframe{\lctx_L}{\simu{R}}$ for all $\lctx_L$.
% \end{theorem}
% %
% \begin{proof}
%   \begin{description}
%   \item[Immediate output bisimulation]
%     Assume that $\octx \lframe{\octx_L}{\simu{R}} \lctx = \atmL{\lctx}'_L \oc \lctx' \oc \atmR{\lctx}'_R$;
%     we must show that $\octx \Reduces\lrframe[\big]{\atmL{\lctx}'_L}{(\lframe{\p{A}}{\simu{R}} \union \simu{R})}{\atmR{\lctx}'_R} \lctx$.
%     \begin{itemize}
%     \item
%       Consider the case in which the context $\atmL{\lctx}'_L$ is nonempty -- that is, the case in which $\octx \lframe{\p{A}}{\simu{R}} \lctx = \p{A} \oc \atmL{\lctx}''_L \oc \lctx' \oc \atmR{\lctx}'_R$ and $\atmL{\lctx}'_L = \p{A} \oc \atmL{\lctx}''_L$, for some $\atmL{\lctx}''_L$.
%       Because $\simu{R}$ is a labeled bisimulation, we may appeal to its immediate output bisimulation property after framing off $\p{A}$ and deduce that $\octx \lframe[\big]{\p{A}}{(\Reduces\lrframe{\atmL{\lctx}''_L}{\simu{R}}{\atmR{\lctx}'_R})} \lctx$.
%       Reduction is closed under framing, 
      
%     \item
%       Consider the case in which the context $\atmL{\lctx}'_L$ is empty and $\lctx'$ is nonempty.
%       Because $\simu{R}$ is a labeled bisimulation, we may appeal to its immediate output bisimulation property after framing off $\p{A}$ and deduce that $\octx \lframe[\big]{\p{A}}{(\Reduces\rframe{\simu{R}}{\atmR{\lctx}'_R})} \lctx$.
%       Reduction is closed under framing, so we conclude that $\octx \Reduces\rframe{\lframe{\p{A}}{\simu{R}}}{\atmR{\lctx}'_R} \lctx$, as required.

%     \item
%       Consider the case in which the contexts $\atmL{\lctx}'_L$ and $\lctx'$ are empty. -- that is, the case in whihc $\p{A} = \atmR{a}$ and $\octx \lframe{\p{A}}{\simu{R}} \lctx = \atmR{\lctx}'_R = \atmR{a} \oc \atmR{\lctx}''_R$, for some $\atmR{\lctx}''_R$.
%       Because $\simu{R}$ is a labeled bisimulation, we may appeal to its immediate output bisimulation property after framing off $\p{A}$ and deduce that $\octx \lframe[\big]{\p{A}}{(\Reduces\rframe{\simu{R}}{\atmR{\lctx}''_R})} \lctx$.
%       Reduction is closed under framing, so $\octx \Reduces\rframe{\lframe{\p{A}}{\simu{R}}}{\atmR{\lctx}''_R} \lctx$.
%       After framing off $\atmR{\lctx}''_R$, we may appeal to the emptiness bisimulation property of $\simu{R}$ and deduce that $\octx \Reduces\rframe[\big]{(\Reduces\rframe{\simu{R}}{\atmR{a}})}{\atmR{\lctx}''_R} \lctx$.
%       Once again, reduction is closed under framing, so we conclude that $\octx \Reduces\rframe{\simu{R}}{\atmR{\lctx}'_R} \lctx$, as required.
%     \end{itemize}

%   \item[Immediate input bisimulation]
%     Assume that $\octx \lframe{\p{A}}{\simu{R}} \lctx$ and $\ireduces{\atmR{\lctx}_L \oc #1 \oc \atmL{\lctx}_R}{\lctx}{\lctx'}$;
%     we must show that $\atmR{\lctx}_L \oc \octx \oc \atmL{\lctx}_R \Reduces\mathrel{(\lframe{\p{A}}{\simu{R}} \union \simu{R})} \lctx'$.
%     \begin{itemize}
%     \item
%       Consider the case in which $\p{A} = \atmR{a}$ and does participate in the input transition -- that is, the case in which $\octx \lframe{\atmR{a}}{\simu{R}} \atmR{a} \oc \lctx_0 = \lctx$ and $\ireduces{\atmR{\lctx}_L \oc \atmR{a} \oc #1 \oc \atmL{\lctx}_R}{\lctx_0}{\lctx'}$, for some $\lctx_0$.
%       Because $\simu{R}$ is a labeled bisimulation, we may appeal to its immediate input bisimulation property and deduce that $\atmR{\lctx}_L \oc \octx \oc \atmL{\lctx}_R \Reduces\simu{R} \lctx'$, as required.
%     \item
%       Consider the case in which $\p{A}$ does not participate in the input transition -- that is, the case in which $\atmR{\lctx}_L$ is empty and $\octx \lframe{\p{A}}{\simu{R}} \p} \oc \lctx_0 = \lctx$ and $\ireduces{#1 \oc \atmL{\lctx}_R}{\lctx_0}{\lctx'_0}$ and $\lctx' = \p{A} \oc \lctx'_0$, for some $\lctx_0$ and $\lctx'_0$.
%       Because $\simu{R}$ is a labeled bisimulation, we may appeal to its immediate input bisimulation property after framing off $\p{A}$ and deduce that $\octx \oc \atmL{\lctx}_R \lframe{\p{A}}{(\Reduces\simu{R})} \lctx'$.
%       Reduction is closed under framing, so we conclude that $\octx \oc \atmL{\lctx}_R \Reduces\lframe{\p{A}}{\simu{R}} \lctx'$, as required.
%     \end{itemize}

%   \item[Reduction bisimulation]
%     Assume that $\octx \lframe{\p{A}}{\simu{R}}\reduces \lctx'$;
%     we must show that $\octx \Reduces\mathrel{(\lframe{\p{A}}{\simu{R}} \union \simu{R})} \lctx'$.
%     \begin{itemize}
%     \item
%       Consider the case in which the reduction arises from the $\simu{R}$-related component alone -- that is, the case in which $\octx \lframe{\p{A}}{(\simu{R}\reduces)} \lctx'$.
%       Because $\simu{R}$ is a labeled bisimulation, we may appeal to its reduction bisimulation property after framing off $\p{A}$ and deduce that $\octx \lframe{\p{A}}{(\Reduces\simu{R})} \lctx'$.
%       Reduction is closed under framing, so we conclude that $\octx \Reduces\lframe{\p{A}}{\simu{R}} \lctx'$.

%     \item
%       Consider the case in which the reduction arises from an input transition on the $\simu{R}$-related component -- that is, the case in which $\p{A} = \atmR{a}$ and $\octx \lframe{\atmR{a}}{\simu{R}} \atmR{a} \oc \lctx_0 = \lctx$ and $\ireduces{\atmR{a} \oc #1}{\lctx_0}{\lctx'}$, for some $\lctx_0$.
%       Because $\simu{R}$ is a labeled bisimulation, we may appeal to its immediate input bisimulation property and deduce that $\octx \Reduces\simu{R} \lctx'$, as required.

%     \item 
%       Consider the case in which the reduction arises from $\p{A}$ alone -- that is, the case in which $\octx = \p{A} \oc \octx_0$ and
%       \begin{itemize}
%       \item $\octx \lframe{(\p{A}_1 \fuse \p{A}_2)}{\simu{R}} \lctx = (\p{A}_1 \fuse \p{A}_2) \oc \lctx_0 \reduces \p{A}_1 \oc \p{A}_2 \oc \lctx_0 = \lctx'$, for some $\lctx_0$.
%         Then $\octx \reduces\lframe{\p{A}_1}{\lframe{\p{A}_2}{\simu{R}}} \lctx'$.
%       \end{itemize}
%     \end{itemize}

%   \item[Emptiness bisimulation]
%     Assume that $\octx \lframe{\p{A}}{\simu{R}} \octxe$.
%     This is, in fact, impossible because the empty context does not contain $\p{A}$.
%   \end{description}
% \end{proof}


% \begin{theorem}
%   If $\octx \osim \lctx$, then $\octx_L \oc \octx \oc \octx_R \osim \octx_L \oc \lctx \oc \octx_R$ for all $\octx_L$ and $\octx_R$.
% \end{theorem}
% \begin{proof}
%   By induction on the structures of $\octx_L$ and $\octx_R$, appealing to \cref{??}.
% \end{proof}


% \begin{theorem}[Reduction closure]
%   Let $\simu{R}$ be a rewriting bisimulation.
%   Then $\simu{R}$ is reduction-closed: $\octx \simu{R}\Reduces \lctx'$ implies $\octx \Reduces\simu{R} \lctx'$.
% \end{theorem}
% %
% \begin{proof}
%   Reduction closure follows immediately as the trivial instance of either the output or input bisimulation properties.
% \end{proof}




\clearpage
\section{Example of rewriting bisimilarity: \Aclp*{NFA}}\label{sec:ordered-bisimilarity:nfa}

Recall from \cref{sec:formula-as-process:nfa-bisim} the conjecture that bisimilar \ac{NFA} states have bisimilar encodings and vice versa.
Once established, this will allow us to rephrase soundness and completeness of the \ac{NFA} choreography in a properly stratified way: $\atmR{a} \oc \defp{q} \reduces_{\orsig}\osim \defp{q}'$ if, and only if, $q \nfareduces[a]\asim q'$.
More precisely, we will prove the following \lcnamecref{thm:nfa-bisim-osim}.
\begin{restatable*}[
  label=thm:nfa-bisim-osim
]{theorem}{thmnfabisim}
  Let $\aut{A} = (Q, \nfapow, F)$ be \iac{NFA} over the input alphabet $\ialph$.
  Then $q \asim s$ if, and only if, $\nfa{q} \osim \nfa{s}$ for all states $q$ and $s$.
\end{restatable*}
\noindent
Before proving this statement, we need a few \lcnamecrefs{lem:a-succ-bisim}.
%
% \begin{lemma}\label{lem:nfa-reduces}
%   For all states $q$:
%   \begin{enumerate}[nosep]
%   \item\label{enum:nfa-reduces-1} $\defp{q} \nreduces$.
%   \item\label{enum:nfa-reduces-2} If $\atmR{a} \oc \defp{q} \Reduces \octx'$, then either $\atmR{a} \oc \defp{q} = \octx'$ or $\atmR{a} \oc \defp{q} \reduces \defp{q}'_a = \octx'$ for some state $q'_a$ that $a$-succeeds $q$.
%   \item\label{enum:nfa-reduces-3} If $\atmR{\emp} \oc \defp{q} \Reduces \octx'$, then either: $\atmR{\emp} \oc \defp{q} = \octx'$; $\atmR{\emp} \oc \defp{q} \reduces \defp{F}(q) = \octx'$; or $q$ is a final state and $\atmR{\emp} \oc \defp{q} \reduces \defp{F}(q) \reduces \octxe = \octx'$.
%   \end{enumerate}
% \end{lemma}
% \begin{proof}
%   Part~\ref{enum:nfa-reduces-1} is proved by inversion of a hypothetical rewriting of $\defp{q}$.

%   Part~\ref{enum:nfa-reduces-2} is proved by inversion of the given rewriting sequence:
%   If the rewriting sequence is nontrivial, it must be $\atmR{a} \oc \defp{q} \reduces \defp{q}'_a \Reduces \octx'$ for some state $q'_a$ that $a$-succeeds $q$; by part~\ref{enum:nfa-reduces-1}, we deduce that $\octx' = \defp{q}'_a$.
%   Otherwise, if the rewriting sequence is trivial, the desired result is immediate.
% \end{proof}
%

% \noindent
These results hold only because the formula-as-process ordered rewriting framework is focused;
under an unfocused rewriting framework, $\defp{q}$ would admit rewritings, such as $\defp{q} \Reduces \atmR{\eow} \limp \atmR{F}(q)$, and $\atmR{a} \oc \defp{q}$ would admit rewritings to contexts other than encodings of $a$-successors.


\begin{lemma}\label{lem:a-succ-bisim}\label{lem:nfa-latent}\label{lem:final-bisim}
  Let $\aut{A} = (Q, \nfapow, F)$ be \iac{NFA} over the alphabet $\ialph$.
  Then:
  \begin{itemize}[nosep]
  \item $\defp{q} \nreduces$ for all states $q$.
  \item If $\atmR{a} \oc \defp{q} \Reduces\osim \defp{q}'$, then $\defp{q}'_a \osim \defp{q}'$ for some state $q'_a$ that $a$-succeeds $q$.
  \item If $\atmR{\eow} \oc \defp{q} \Reduces\osim \atmR{F}(s)$, then $q \in F$ if, and only if, $s \in F$.
  \end{itemize}
\end{lemma}
\begin{proof}
  The first part can be proved by examining the encoding of an arbitrary state $q$.

  To prove the second part, assume that $\atmR{a} \oc \defp{q} \Reduces\osim \defp{q}'$.
  By inversion on the given trace, there are two cases: either
  \begin{enumerate*}[label=\emph{(\roman*)}]
  \item $\atmR{a} \oc \defp{q} \osim \defp{q}'$ or
  \item $\atmR{a} \oc \defp{q} \reduces \defp{q}'_a \Reduces\osim \defp{q}'$ for some state $q'_a$ that $a$-succeeds $q$.
  \end{enumerate*}
  \begin{itemize}
  \item
    Consider the case in which $\atmR{a} \oc \defp{q} \osim \defp{q}'$.
    Because the underlying \ac{NFA} is well-formed~\parencref{def:finite-automata:nfa}, $q$ has at least one $a$-successor;
    let $q'_a$ be one such successor.
    By definition of the encoding, $\atmR{a} \oc \defp{q} \reduces \defp{q}'_a$.
    Because rewriting bisimilarity is reduction-closed (\cref{thm:bisim-reduction-closure}), $\defp{q}'_a \osim\secudeR \defp{q}'$.
    The first part of this \lcnamecref{lem:a-succ-bisim} shows that states are encoded by propositions that do not reduce, and so we may conclude that, in fact, $\defp{q}'_a \osim \defp{q}'$.

  \item 
    Consider the case in which $\atmR{a} \oc \defp{q} \reduces \defp{q}'_a \Reduces\osim \defp{q}'$ for some state $q'_a$ that $a$-succeeds $q$.
    Because states are encoded by propositions that do not reduce, $\defp{q}'_a \osim \defp{q}'$ for some state $q'_a$ that $a$-succeeds $q$, as required.
  \end{itemize}

  To prove the third part, we reason as in the preceding part and deduce that $\atmR{F}(q) \osim \atmR{F}(s)$;
  we conclude that $q \in F$ if, and only if, $s \in F$.
  % 
  % \begin{itemize}
  % \item Consider the case in which the trace is trivial: $\atmR{a} \oc \defp{q} \osim \defp{q}'$.
  %   Because the underlying \ac{NFA} is well-formed~\parencref{??}, $q$ has at least one $a$-successor;
  %   let $q'_a$ be one such successor.
  %   By definition of the encoding, $\atmR{a} \oc \defp{q} \reduces \defp{q}'_a$.
  %   Because rewriting bisimilarity is reduction-closed (\cref{thm:bisim-reduction-closure}), $\defp{q}'_a \osim\secudeR \defp{q}'$.
  %   States are encoded by latent\autocite{??} propositions (\cref{??}), and so we may conclude that, in fact, $\defp{q}'_a \osim \defp{q}'$.
  %
  % \item Consider the case in which the trace contains at least one step.
  %   By inversion, that step corresponds to \iac{NFA} transition: $\atmR{a} \oc \defp{q} \reduces \defp{q}'_a \Reduces\osim \defp{q}'$, for some state $q'_a$ that $a$-succeeds $q$.
  %   Once again, because states are encoded by latent propositions, the trace from $\defp{q}'_a$ must be trivial.
  % \qedhere
  % \end{itemize}
\end{proof}
%
% \begin{lemma}\label{lem:final-bisim}
%   If $\atmR{\emp} \oc \nfa{q} \Reduces\osim \nfa{F}(s)$, then $q \in F$ if, and only if, $s \in F$.
% \end{lemma}
% %
% \begin{proof}
%   \begin{itemize}
%   \item Consider the case in which the trace is trivial -- \ie, $\atmR{\emp} \oc \nfa{q} \osim \nfa{F}(s)$.
%     By definition of the encoding, $\nfa{F}(q) \secuder \atmR{\emp} \oc \nfa{q} \osim \nfa{F}(s)$.
%     $\nfa{F}(q) \osim\secudeR \nfa{F}(s)$
%   \end{itemize}
% \end{proof}


% \begin{lemma}
%   If $\nfa{F}(q) \Reduces\osim \nfa{F}(s)$, then $q \in F$ if, and only if, $s \in F$.
% \end{lemma}
% %
% \begin{proof}
%   Assume that $\nfa{F}(q) \Reduces\osim \nfa{F}(s)$ and $q \notin F$.
%   By inversion, The trace can only be the trivial one, so $\nfa{F}(q) = \top$ and $\nfa{F}(s)$ are bisimilar.
%   Suppose, for the sake of contradiction, that $s \in F$ and so $\nfa{F}(s) = \one$.
%   Then $\top \osim \one$; hence, $\atmR{a} \oc \top \Reduces\osim \atmR{a}$ follows from the input bisimilarity property.
%   But output bisimilarity implies $\atmR{a} \oc \top \Reduces\rframe{\osim}{\atmR{a}} \atmR{a}$, which is impossible because $\atmR{a} \oc \top$ cannot produce $\atmR{a}$ at its right end.
% \end{proof}

\thmnfabisim
%
\begin{proof}
  We shall show that \ac{NFA} bisimilarity coincides with rewriting bisimilarity of encodings, proving each direction separately.
  \begin{itemize}[itemsep=\dimexpr\itemsep+\parsep\relax, parsep=0em, listparindent=\parindent]
    \item
      To prove that bisimilar \ac{NFA} states have bisimilar encodings -- \ie, that $q \asim s$ implies $\defp{q} \osim \defp{s}$ -- we shall now show that the relation $\mathord{\simu{R}} = \Set{(\defp{q}, \defp{s}) \given q \asim s}$ is a labeled bisimulation up to reflexivity and, by \cref{thm:bisim-technique-up-to-refl}, is included in rewriting bisimilarity.
      Notice that $\simu{R}$ is, by definition, symmetric because \ac{NFA} bisimilarity is symmetric~\parencref{thm:ordered-bisimilarity:equivalence}.
      \begin{description}
      \item[Immediate output bisimulation]
        Assume that $\defp{q} \simu{R} \defp{s} = \atmL{\lctx}'_L \oc \lctx' \oc \atmR{\lctx}'_R$; we must show that $\defp{q} \Reduces\lrframe{\atmL{\lctx}'_L}{\reflc{\simu{R}}}{\atmR{\lctx}'_R} \defp{s}$.
        By definition of the encoding, $\defp{s}$ is a negative propostion and does not expose outputs.
        Therefore, $\atmL{\lctx}'_L$ and $\atmR{\lctx}'_R$ are empty and $\lctx'$ is $\defp{s}$.
        The required $\defp{q} \Reduces\lrframe{\atmL{\lctx}'_L}{\reflc{\simu{R}}}{\atmR{\lctx}'_R} \defp{s}$ follows trivially.

      \item[Immediate input bisimulation]
        Assume that $\defp{q} \simu{R} \defp{s}$ and $\ireduces{\atmR{\lctx}_L \oc #1 \oc \atmL{\lctx}_R}{\defp{s}}{\lctx'}$; we must show that $\atmR{\lctx}_L \oc \defp{q} \oc \atmL{\lctx}_R \Reduces\reflc{\simu{R}} \lctx'$.
        Inversion of the input transition yields two cases.
        \begin{itemize}
        \item
          Consider the case in which the input transition is $\ireduces{\atmR{a} \oc #1}{\defp{s}}{\defp{s}'_a}$, where state $s$ is $a$-succeeded by $s'_a$.
          Because $q$ and $s$ are bisimilar, there must exist an $a$-successor of $q$, say $q'_a$, that is bisimilar to $s'_a$.
          By definition of the encoding, we thus have $\atmR{a} \oc \defp{q} \reduces \defp{q}'_a$.
          So indeed, because $q'_a$ and $s'_a$ are bisimilar states, $\atmR{a} \oc \defp{q} \Reduces\reflc{\simu{R}} \defp{s}'_a$, as required.

        \item
          Consider the case in which the input transition is $\ireduces{\atmR{\eow} \oc #1}{\defp{s}}{\atmR{F}(s)}$.
          Because $q$ and $s$ are bisimilar states, $\atmR{F}(q) = \atmR{F}(s)$.
          By definition of the encoding, $\atmR{\eow} \oc \defp{q} \reduces \atmR{F}(q)$, and so, indeed, $\atmR{\eow} \oc \defp{q} \Reduces\reflc{\simu{R}} \atmR{F}(s)$, as required.
        \end{itemize}

      \item[Reduction bisimulation]
        Assume that $\defp{q} \simu{R} \defp{s} \reduces \lctx'$.
        The reduction bisimulation property holds vacuously because states are encoded as propositions that do not reduce~\parencref{lem:nfa-latent} -- there is no $\lctx'$ such that $\defp{s} \reduces \lctx'$.

      \item[Emptiness bisimulation]
        Assume that $\defp{q} \simu{R} \defp{s} = (\octxe)$.
        The emptiness bisimulation property also holds vacuously because states are encoded as propositions, not empty contexts.
      \end{description}

    \item
      To prove the converse -- that states with bisimilar encodings are themselves bisimilar -- we shall now show that the relation $\mathord{\simu{R}} = \Set{(q,s) \given \defp{q} \osim \defp{s}}$, which relates states if they have rewriting-bisimilar encodings, is \iac{NFA} bisimulation and is therefore included in \ac{NFA} bisimilarity.

      Because rewriting bisimilarity is symmetric~\parencref{thm:ordered-bisimilarity:equivalence}, so too is the relation $\simu{R}$.
      We must also prove that $\simu{R}$ satisfies the conditions of \ac{NFA} bisimilarity.
      \begin{description}[itemsep=\dimexpr\itemsep+\parsep\relax, parsep=0em, listparindent=\parindent]
      \item[Input bisimulation] Let $q$ and $s$ be states with bisimilar encodings, and let $q'_a$ be an $a$-successor of $q$;
        we must exhibit a state $s'_a$ that $a$-succeeds $s$ and has an encoding that is bisimilar to that of $q'_a$.

        By definition of the encoding, $\atmR{a} \oc \defp{q} \reduces \defp{q}'_a$.
        Because $q$ and $s$ have bisimilar encodings, the input bisimulation property allows us to deduce that $\atmR{a} \oc \defp{s} \Reduces\osim \defp{q}'_a$.
        An appeal to \cref{lem:a-succ-bisim} provides exactly what is needed: a state $s'_a$ that $a$-succeeds $s$ and has an encoding bisimilar to that of $q'_a$.
        
      \item[Finality bisimulation] Let $q$ and $s$ be states with bisimilar encodings, and assume that $q$ is a final state;
        we must show that $s$ is also a final state.

        By definition of the encoding, $\atmR{\eow} \oc \defp{q} \reduces \atmR{F}(q) = \atmR{\symacc}$.
        Because $q$ and $s$ have bisimilar encodings, it follows from the input bisimulation property that $\atmR{\eow} \oc \defp{s} \Reduces\osim \atmR{F}(q)$.
        An appeal to \cref{lem:final-bisim} allows us to conclude that $s$, like $q$, is a final state.
      %
      \qedhere
      \end{description}
    \end{itemize}
\end{proof}


% \begin{proof}
%   The parts are proved in order, with parts [...] depending on part [...].
%   \begin{enumerate}
%   \item To prove that bisimilar states are exactly those states that have bisimilar encodings, we take each direction in turn.
%     \begin{itemize}
%     \item First, we will prove that bisimilar states have bisimilar encodings.
%       Let $\simu{R}$ be the binary relation that relates two states' encodings if their underlying states are \ac{NFA}-bisimilar -- that is, $\mathord{\simu{R}} = \set{(\nfa{q}, \nfa{s}) \given q \asim s}$; we shall show that $\simu{R}$ satisfies the conditions of \cref{thm:??} and is therefore included in rewriting bisimilarity.


%     \item Conversely, we will now prove that states that have bisimilar encodings are themselves bisimilar.
%       Let $\simu{R}$ be the binary relation that relates two states if their encodings are rewriting-bisimilar -- that is, $\mathord{\simu{R}} = \set{(q,s) \given \nfa{q} \osim \nfa{s}}$;
%       we shall show that $\simu{R}$ is \iac{NFA} bisimulation and therefore included in \ac{NFA} bisimilarity.
%       \begin{itemize}[listparindent=\parindent]
%       \item Let $q$ and $s$ be states with bisimilar encodings, and let $q'_a$ be an $a$-successor of $q$;
%         we must exhibit a state $s'_a$ that $a$-succeeds $s$ and has an encoding that is bisimilar to that of $q'_a$.

%         By definition of the encoding, $\atmR{a} \oc \nfa{q} \reduces \nfa{q}'_a$.
%         Because $q$ and $s$ have bisimilar encodings, the input bisimilarity property allows us to deduce that $\atmR{a} \oc \nfa{s} \Reduces\miso \nfa{q}'_a$.
%         An appeal to \cref{lem:??} provides exactly what is needed: a state $s'_a$ that $a$-succeeds $s$ and has an encoding bisimilar to that of $q'_a$.
        
%       \item Let $q$ and $s$ be states with bisimilar encodings, and assume that $q$ is a final state;
%         we must show that $s$ is also a final state.

%         By definition of the encoding, $\atmR{\emp} \oc \nfa{q} \reduces \nfa{F}(q) = \one$.
%         Because $q$ and $s$ have bisimilar encodings, it follows from input bisimilarity that $\atmR{\emp} \oc \nfa{s} \Reduces\miso \nfa{F}(q)$.
%         \begin{itemize}
%         \item $\nfa{F}(s) \secuder \atmR{\emp} \oc \nfa{s} \miso \nfa{F}(q)$.
%           Then $\nfa{F}(s) \miso\secudeR \nfa{F}(q)$.
%         \item $\nfa{F}(s) \Reduces\miso \nfa{F}(q)$.
          
%         \end{itemize}
%       \end{itemize}
      
%     \end{itemize}

%   \item 
%     \begin{itemize}
%     \item Assume that, up to bisimilarity, $q'$ is an $a$-successor of $q$ -- that is, that $q \misa s \nfareduces[a] s'_a \asim q'$ for some states $s$ and $s'_a$.
%       By definition of the encoding, $\atmR{a} \oc \nfa{s} \reduces \nfa{s}'_a$.
%       Because bisimilar states have bisimilar encodings (part~\ref{??}), $\nfa{q} \miso \nfa{s}$ and $\nfa{s}'_a \osim \nfa{q}'$.
%       Moreover, because rewriting bisimilarity is an atomic congruence (\cref{??}), $\atmR{a} \oc \nffa{q} \miso \atmR{a} \oc \nfa{s}$.
%       and putting everything together, we have $\atmR{a} \oc \nfa{q} \miso\reduces\osim \nfa{s}'_a$.

%     \item Assume that, up to bisimilarity, $\atmR{a} \oc \nfa{q}$ rewrites to $\nfa{q}'$ -- that is, assume that $\atmR{a} \oc \nfa{q} \miso\reduces\osim \nfa{q}'$.
%       An appeal to the reduction bisimilarity property yields $\atmR{a} \oc \nfa{q} \Reduces\miso\osim \nfa{q}'$.
%       Because bisimilarity is a symmetric relation, it follows from \cref{lem:??} that there exists a state $q'_a$ that $a$-succeeds $q$ and has an encoding that is bisimilar to that of $q'$.
%       Moreover, because states with bisimilar encodings are themselves bisimilar (part~\ref{??}) and because bisimilarity is reflexive, $q \misa\nfareduces[a]\asim q'$.
%     \end{itemize}

%   \item
%   \item 
%     \begin{itemize}
%     \item We must show that $q \misa\asim q'$ if, and only if, $\nfa{q} \miso\Reduces\osim \nfa{q}'$.
%     \item We must show that $q \misa\nfareduces[a]\nfareduces[w]\asim q'$ if, and only if, $\atmR{w} \oc \atmR{a} \oc \nfa{q} \miso\Reduces\osim \nfa{q}'$.
      
%     \end{itemize}
%   \end{enumerate}

%   Let $\simu{R}$ be the binary relation on ordered contexts such that $\octx$ and $\lctx$ are $\simu{R}$-related if they are equal to the encodings of a pair of bisimilar states -- that is, $\mathord{\simu{R}} = \set{(\octx, \lctx) \given \exists q,s \in Q.\, (\octx = \nfa{q}) \land (q \asim s) \land (\nfa{s} = \lctx)}$.
%   \begin{itemize}
%   \item Let $q$ and $s$ be bisimilar states, and assume that $\ireduces{\atmR{\lctx}_L \oc #1 \oc \atmL{\lctx}_R}{\nfa{q}}{\lctx'}$.
%     By inversion, there are two cases; in either case, the context $\atmL{\lctx}_R$ must be empty.
%     \begin{itemize}
%     \item Consider the case in which the context $\atmR{\lctx}_L$ is a single atom $\atmR{\emp}$ and $\lctx' = \nfa{F}(q)$.
%       By the encoding's construction, $\ireduces{\atmR{\emp} \oc #1}{\nfa{s}}{\nfa{F}(s)}$.
%       And, because states $q$ and $s$ are bisimilar, the two are both final or both nonfinal states.
%       $\nfa{F}(s)$
%     \item  or $\atmR{a}$ for some input symbol $a$.
%     \end{itemize}
%   \end{itemize}
  

%   Let $\simu{R}$ be the binary relation on states such that $q$ and $s$ are $\simu{R}$-related if their encodings are ordered bisimilar -- that is, $\mathord{\simu{R}} = \set{(q, s) \given \nfa{q} \osim \nfa{s}}$.
%   \begin{itemize}
%   \item Assume that $s \simu{R}^{-1} q \nfareduces[a] q'_a$.
%     Because $q'_a$ is an $a$-successor of $q$, there exists a trace $\atmR{a} \oc \nfa{q} \Reduces \nfa{q}'_a$.
%     Because $q$ and $s$ have bisimilar encodings, it then follows from the input bisimilarity property that $\atmR{a} \oc \nfa{s} \Reduces\miso \nfa{q}'_a$.
%     By inversion, there are two cases
%     \begin{itemize}
%     \item $\nfa{s}'_a \secuder \atmR{a} \oc \nfa{s} \miso \nfa{q}'_a$, so $\nfa{q}'_a \Reduces\osim \nfa{s}'_a$
%     \item $\atmR{a} \oc \nfa{s} \reduces \nfa{s}'_a \miso \nfa{q}'_a$
%     \end{itemize}
%     By inversion of this trace, there must exist a state $s'_a$ that is an $a$-successor of $s$ and has an encoding that is bisimilar to the encoding of $q'_a$ -- in other words, $s \nfareduces[a] s'_a \simu{R}^{-1} q'_a$.

%   \item Assume that $s \simu{R}^{-1} q \in F$.
%     With $q$ being a final state, there exists a trace $\atmR{\emp} \oc \nfa{q} \Reduces \octxe$.
%     Because $q$ and $s$ have bisimilar encodings, $\atmR{\emp} \oc \nfa{s} \Reduces\miso \octxe$.
%     By inversion of this trace, $\nfa{F}(s)$ is bisimilar to the empty context.
%     That is impossible if $s \notin F$, so $s$ must be a final state, like $q$.
%   \end{itemize}

%   To establish the completness of our \ac{NFA} encoding with respect to bisimularity, it then suffices to show that ordered bisimularity contains the relation $\simu{R}$.
%   Appealing to the preceding proof technique for ordered bisimilarity\parencref{thm:ord-bisim-technique}, we need only establish that $\simu{R}$ has immediate output bisimulation, immediate input bisimulation, reduction bisimulation, and emptiness bisimulation properties.

%   Only the immediate input bisimulation and reduction bisimulation conditions apply to the relation $\simu{R}$.
%   \begin{description}
%   \item[Immediate input bisimulation]
%     Assume that $\lctx \simu{R} \octx$ and $\ireduces{\atmR{\lctx}_L \oc #1 \oc \atmL{\lctx}_R}{\lctx}{\lctx'}$;
%     we must show that $\atmR{\lctx}_L \oc \octx \oc \atmL{\lctx}_R \Reduces\refl*{\simu{R}}^{-1} \lctx'$.

%     Inversion allows us to deduce $\lctx = \nfa{q}$ and $\octx = \nfa{s}$ for some states $q$ and $s$ such that $q \asim s$.
%     Examining the encoding, we see that there are two possible input transitions from $\nfa{q}$.
%     \begin{itemize}
%     \item Consider the input transition $\ireduces{\atmR{a} \oc #1}{\nfa{q}}{\nfa{q}'_a}$, with $a \in \ialph$ and $q \nfareduces[a] q'_a$ -- that is, $\atmR{\lctx}_L = \atmR{a}$; $\atmL{\lctx}_R = \octxe$; and $\lctx' = \nfa{q}'_a$.
%       We must show that $\atmR{a} \oc \nfa{s} \Reduces\refl*{\simu{R}}^{-1} \nfa{q}'_a$.

%       Because $q$ and $s$ are bisimilar states, $s \nfareduces[a] s'_a \misa q'_a$ for some state $s'_a$.
%       Recall from \cref{thm:nfa-encoding-reduces} that the encoding of \acp{NFA} is complete with respect to input transitions; so, $\atmR{a} \oc \nfa{s} \reduces \nfa{s}'_a$.
%       As $q'_a$ and $s'_a$ are bisimilar states, we conclude that $\atmR{a} \oc \nfa{s} \reduces\refl*{\simu{R}}^{-1} \nfa{q}'_a$, as required.

%     \item Consider the input transition $\ireduces{\atmR{\emp} \oc #1}{\nfa{q}}{\octxe}$ when $q$ is a final state -- that is, $\atmR{\lctx}_L = \atmR{\emp}$ and $\atmL{\lctx}_R = \lctx' = \octxe$.
%       We must show that $\atmR{\emp} \oc \nfa{s} \Reduces\refl*{\simu{R}}^{-1} \octxe$.

%       Because $q$ and $s$ are bisimilar states, $s$ must also be a final state.
%       Recall from \cref{thm:nfa-encoding-reduces} that the encoding of \acp{NFA} is complete with respect to input transitions; so, $\atmR{\emp} \oc \nfa{s} \reduces \octxe$.
%       We conclude that $\atmR{\emp} \oc \nfa{s} \reduces\refl*{\simu{R}}^{-1} \octxe$, as required.
%     \end{itemize}
%   %
%   \item[Reduction bisimulation]
%   \end{description}
% \end{proof}

% \begin{theorem}
%   If $\nfa{q} \osim \nfa{s}$, then $q \asim s$.
% \end{theorem}
% %
% \begin{proof}
%   Let $\simu{R}$ be the binary relation on states such that $q \simu{R} s$ exactly when $\nfa{q} \osim \nfa{s}$.
%   We will show that $\simu{R}$ is \iacs{NFA} bisimulation.

%   Among other properties, we must show that $\simu{R}$ simulates inputs.
%   Assume that $\nfa{s} \miso \nfa{q}$ and $q \nfareduces[a] q'$; we must show that $s \nfareduces[a] s'$ for some $s'$ such that $\nfa{s}' \miso \nfa{q}'$.
%   Because $\atmR{a} \oc \nfa{q} \Reduces \nfa{q}'$, it follows by input bisimilarity that $\atmR{a} \oc \nfa{s} \Reduces\miso \nfa{q}'$.
%   There are two cases, according to the structure of the reduction sequence from $\atmR{a} \oc \nfa{s}$.
%   \begin{itemize}
%   \item If the reduction sequence is trivial, then $\atmR{a} \oc \nfa{s} \miso \nfa{q}'$.
%     Because the transition relation is left-total, $s \nfareduces[a] s'$ for some state $s'$.
%     It follows that $\atmR{a} \oc \nfa{s} \Reduces \nfa{s}'$, and so, by input bisimilarity, $\nfa{q}' \Reduces\osim \nfa{s}'$.
%     However, $\nfa{q}' \longarrownot\reduces$, allowing us to conclude that $\nfa{q}' \osim \nfa{s}'$.
%   \item If the reduction sequence is nontrivial, then $\atmR{a} \oc \nfa{s} \reduces\Reduces\miso \nfa{q}'$.
%     Then
%     \begin{equation*}
%       \with_{s^* \mid s \nfareduces[a] s^*} \nfa{s}^* \Reduces\miso \nfa{q}'
%     \end{equation*}
%     It follows that $\with_{s^* \in S} \nfa{s}^* \miso \nfa{q}$ where $S$ is a subset of the $a$-successors of state $s$.
%     Because bisimilarity is reduction-closed, $\nfa{s}^* \miso \nfa{q}'$ for each $s^* \in S$.

%     How do we know that the subset $S$ is nonempty?
%     In other words, what happens if $\top \miso \nfa{q}'$?
%   \end{itemize}

%   Assume that $\nfa{s} \miso \nfa{q}$ and $q$ is a final state;
%   we must show that $s$ is also a final state.
%   Because $q$ is final, $\atmR{\emp} \oc \nfa{q} \Reduces \octxe$.
%   By input bisimilarity, $\atmR{\emp} \oc \nfa{s} \Reduces\miso \octxe$.
%   Choose a fresh atom $\atmR{x}$.
%   It follows by emptiness bisimilarity that $\atmR{x} \oc \atmR{\emp} \oc \nfa{s} \Reduces\rframe{\osim}{\atmR{x}}^{-1} \atmR{x}$.
%   However, $\atmR{x} \oc \atmR{\emp} \oc \nfa{s}$ exposes $\atmR{x}$ on the right only if $s$ is also a final state.
% \end{proof}

% \clearpage
\section{Example of rewriting bisimilarity: Binary counter}\label{sec:ordered-bisimilarity:counter}

For a further application of rewriting bisimilarity, we can revisit the binary counter.
Recall from \cref{sec:formula-as-process:counters-oo} its object-oriented choreography:
%d of the binary counter:
\begin{equation*}
  \orsig =
  \begin{lgathered}[t]
    \defp{e} \defd (\defp{e} \fuse \defp{b}_1 \pmir \atmL{i}) \with (\atmR{z} \pmir \atmL{d}) \\
    \defp{b}_0 \defd (\up \dn \defp{b}_1 \pmir \atmL{i}) \with (\atmL{d} \fuse \defp{b}'_0 \pmir \atmL{d}) \\
    \defp{b}_1 \defd (\atmL{i} \fuse \defp{b}_0 \pmir \atmL{i}) \with (\defp{b}_0 \fuse \atmR{s} \pmir \atmL{d}) \\
    \defp{b}'_0 \defd (\atmR{z} \limp \atmR{z}) \with (\atmR{s} \limp \defp{b}_1 \fuse \atmR{s})
  \end{lgathered}
\end{equation*}
Also, recall that denotations were assigned directly to choreographed counters using $\aval{\octx}{n}$, $\ainc{\octx}{n}$, and $\adec{\octx}{n}$ judgments.

Intuitively, any two counters that have the same denotation ought to be indistinguishable.
After all, the only two operations that we have on counters are increment and head-unary normalization (also known as decrement), and both of these are reflected in the denotation.
For instance, both $\defp{e} \oc \atmL{i} \oc \defp{b}_1$ and $\defp{e} \oc \defp{b}_1 \oc \defp{b}_0 \oc \atmL{i}$ denote the natural number $3$, so any sequence of increments and decrements that we apply to these counters ought not to distinguish them.

% Strictly speaking, denotations were only defined for the string rewriting specification of binary counters~\parencref{sec:??}: string $w$ denotes natural number $n$ exactly when $\ainc{w}{n}$.
% We can, of course, lift the denotations to choreographed contexts:
% $\octx$ denotes $n$ exactly when $\ainc{\theta^{-1}(\octx)}{n}$.
% For instance, $\defp{e} \oc \atmL{i} \oc \defp{b}_1$ denotes $3$ because $\theta^{-1}(\defp{e} \oc \atmL{i} \oc \defp{b}_1) = \ainc{e \wc i \wc b_1}{3}$.
% We could even assign denotations directly to choreographed contexts by defining new $\aval{}{}$, $\ainc{}{}$, and $\adec{}{}$ relations on choreographed contexts.
% \begin{inferences}
%   \infer[\jrule{$\defp{e}$-V}]{\aval{\defp{e}}{0}}{}
%   \and
%   \infer[\jrule{$\defp{b}_0$-V}]{\aval{\octx \oc \defp{b}_0}{2n}}{
%     \aval{\octx}{n}}
%   \and
%   \infer[\jrule{$\defp{b}_1$-V}]{\aval{\octx \oc \defp{b}_1}{2n+1}}{
%     \aval{\octx}{n}}
%   \\
%   \infer[\jrule{$\defp{e}$-I}]{\ainc{\defp{e}}{0}}{}
%   \and
%   \infer[\jrule{$\defp{b}_0$-I}]{\ainc{\octx \oc \defp{b}_0}{2n}}{
%     \ainc{\octx}{n}}
%   \and
%   \infer[\jrule{$\defp{b}_1$-I}]{\ainc{\octx \oc \defp{b}_1}{2n+1}}{
%     \ainc{\octx}{n}}
%   \and
%   \infer[\jrule{$\atmL{i}$-I}]{\ainc{\octx \oc \atmL{i}}{n+1}}{
%     \ainc{\octx}{n}}
%   \\
%   \infer[\jrule{$\atmL{d}$-D}]{\adec{\octx \oc \atmL{d}}{n}}{
%     \ainc{\octx}{n}}
%   \and
%   \infer[\jrule{$\atmR{z}$-D}]{\adec{\atmR{z}}{0}}{}
%   \and
%   \infer[\jrule{$\atmR{s}$-D}]{\adec{\octx \oc \atmR{s}}{n+1}}{
%     \ainc{\octx}{n}}
%   \and
%   \infer[\jrule{$\defp{b}'_0$-D}]{\adec{\octx \oc \defp{b}'_0}{2n}}{
%     \adec{\octx}{n}}
% \end{inferences}
% A context $\octx$ is an \vocab{increment counter} or \vocab{increment context} if $\ainc{\octx}{n}$ for some natural number $n$;
% likewise, we will say that a context $\octx$ is an \vocab{decrement counter} or \vocab{decrement context} if $\adec{\octx}{n}$ for some natural number $n$.

% With these relations, 
We can state and prove that counters with equal denotations are bisimilar, and conversely, that bisimilar counters have equal denotations.
\begin{restatable}[
  label=thm:oo-counter-bisim
]{theorem}{thmoocounterbisim}
  If either
  \begin{enumerate*}[label=\emph{(\roman*)}]
  \item $\ainc{\octx}{n}$ and $\ainc{\lctx}{n'}$ or
  \item $\adec{\octx}{n}$ and $\adec{\lctx}{n'}$
  \end{enumerate*}, then $\octx \osim \lctx$ if, and only if, $n = n'$.
\end{restatable}

Before proving this \lcnamecref{thm:oo-counter-bisim}, recall the big-step adequacy of decrements for the choreography.
\coroocounteradequacy*

We also need to prove an easy lemma that characterizes the  output and input transitions possible from binary counters.
\begin{lemma}\label{lem:ordered-bisimilarity:oo-counter-bisim-easy}\leavevmode
  \begin{itemize}[nosep]
  \item
    If $\ainc{\lctx}{n}$, then:
    \begin{itemize}[nosep]
    \item $\lctx = \atmL{\lctx}'_L \oc \lctx' \oc \atmR{\lctx}'_R$ only if $\atmL{\lctx}'_L$ and $\atmR{\lctx}'_R$ are empty; and
    \item $\ireduces{\atmR{\lctx}_L \oc #1 \oc \atmL{\lctx}_R}{\lctx}{\lctx'}$ only if $\atmR{\lctx}_L$ is empty and either $\atmL{\lctx}_R = \atmL{i}$ or $\atmL{\lctx}_R = \atmL{d}$.
    \end{itemize}

  \item
    If $\adec{\lctx}{n}$, then:
    \begin{itemize}[nosep]
    \item $\lctx = \atmL{\lctx}'_L \oc \lctx' \oc \atmR{\lctx}'_R$ only if
      $\atmL{\lctx}'_L$ is empty and either:
      \begin{itemize}[nosep]
      \item $n=0$ and $\atmR{\lctx}'_R = \atmR{z}$ and $\lctx'$ is empty;
      \item $n > 0$ and $\atmR{\lctx}'_R = \atmR{s}$ and $\ainc{\lctx'}{n-1}$; or
      \item $\atmR{\lctx}'_R$ is empty; and
      \end{itemize}

    \item
      $\ireduces{\atmR{\lctx}_L \oc #1 \oc \atmL{\lctx}_R}{\lctx}{\lctx'}$ is impossible.
    \end{itemize}
  \end{itemize}
\end{lemma}
\begin{proof}
  By structural induction on the derivation of the given denotation.
  The second part relies on the first part in one case.
\end{proof}

% \begin{lemma}\leavevmode
%   \begin{itemize}
%   \item
%     If $\lctx = \atmL{\lctx}'_L \oc \lctx' \oc \atmR{\lctx}'_R$, then:
%     \begin{itemize}[nosep]
%     \item $\ainc{\lctx}{n}$ only if $\atmL{\lctx}'_L$ and $\atmR{\lctx}'_R$ are empty; and
%     \item $\adec{\lctx}{n}$ only if $\atmL{\lctx}'_L$ is empty and either:
%       \begin{itemize}[nosep]
%       \item $n=0$ and $\atmR{\lctx}'_R = \atmR{z}$ and $\lctx'$ is empty;
%       \item $n > 0$ and $\atmR{\lctx}'_R = \atmR{s}$ and $\ainc{\lctx'}{n-1}$; or
%       \item $\atmR{\lctx}'_R$ is empty.
%       \end{itemize}
%     \end{itemize}

%   \item
%     If $\ireduces{\atmR{\octx}_L \oc #1 \oc \atmL{\octx}_R}{\octx}{\octx'}$, then:
%     \begin{itemize}[nosep]
%     \item $\ainc{\octx}{n}$ only if $\atmR{\octx}_L$ is empty and either $\atmL{\octx}_R = \atmL{i}$ or $\atmL{\octx}_R = \atmL{d}$ or $\atmL{\octx}_R$ is empty; and
%     \item $\adec{\octx}{n}$ only if $\atmR{\octx}_L$ and $\atmL{\octx}_R$ are empty.
%     \end{itemize}
%   \end{itemize}
% \end{lemma}
% \begin{proof}
  
% \end{proof}

% \begin{lemma}\label{lem:ordered-bisimilarity:counter-ireduces}
%   If $\ireduces{\atmR{\octx}_L \oc #1 \oc \atmL{\octx}_R}{\octx}{\octx'}$, then:
%   \begin{itemize}[nosep]
%   \item $\ainc{\octx}{n}$ only if $\atmR{\octx}_L$ is empty and either $\atmL{\octx}_R = \atmL{i}$ or $\atmL{\octx}_R = \atmL{d}$ or $\atmL{\octx}_R$ is empty; and
%   \item $\adec{\octx}{n}$ only if $\atmR{\octx}_L$ and $\atmL{\octx}_R$ are empty.
%   \end{itemize}
% \end{lemma}
% \begin{proof}
%   Each part is proved by structural induction on the denotation's derivation, with the second part depending on the first.
% \end{proof}

Using this \lcnamecref{lem:ordered-bisimilarity:oo-counter-bisim-easy}, we may prove the correspondence between denotation and bisimilarity.
It is especially interesting that the proof of this \lcnamecref{thm:oo-counter-bisim} is quite modular, relying heavily on the choreography's adequacy.
We conjecture that this proof pattern will be useful in proving \lcnamecrefs{thm:oo-counter-bisim} about bisimilarities for other choreographies.
%
\thmoocounterbisim*
\begin{proof}
  We prove each direction separately.
  \begin{itemize}
  \item
    To prove that states with equal denotations are bisimilar, consider the relation $\simu{R}$ given by 
    % Let $\simu{R}$ be a relation on states with equal denotations: 
  \begin{equation*}
    \mathord{\simu{R}}
    =
    \begin{lgathered}[t]
      \Set{(\octx, \lctx) \given \exists n \in \nats.\,(\ainc{\octx}{n}) \land (\ainc{\lctx}{n})} \\
      {} \union
      \Set{(\octx, \lctx) \given \exists n \in \nats.\,(\adec{\octx}{n}) \land (\adec{\lctx}{n})}
      \,.
    \end{lgathered}
  \end{equation*}
  We shall show that $\simu{R}$ progresses to its reflexive closure and then conclude, by \cref{thm:bisim-technique-up-to-refl}, that $\simu{R}$ is contained within rewriting bisimilarity.
  \begin{description}
  \item[Immediate output bisimulation]
    Assume that $\octx \simu{R} \lctx = \atmL{\lctx}'_L \oc \lctx' \oc \atmR{\lctx}'_R$; we must show that $\octx \Reduces\lrframe{\atmL{\lctx}'_L}{\reflc{\simu{R}}}{\atmR{\lctx}'_R} \lctx$.
    Because $\octx$ and $\lctx$ are $\simu{R}$-related, either:
      $\ainc{\octx}{n}$ and $\ainc{\lctx}{n}$ for some natural number $n$; or
      $\adec{\octx}{n}$ and $\adec{\lctx}{n}$ for some natural number $n$.
    \begin{itemize}
    \item
      Consider the case in which $\ainc{\octx}{n}$ and $\ainc{\lctx}{n}$.
      % A straightforward induction on the derivation of $\ainc{\lctx}{n}$
      \Cref{lem:ordered-bisimilarity:oo-counter-bisim-easy} shows that $\atmL{\lctx}'_L$ and $\atmR{\lctx}'_R$ must both be empty.
      The required $\octx \Reduces\lrframe{\atmL{\lctx}'_L}{\reflc{\simu{R}}}{\atmR{\lctx}'_R} \lctx$ is then trivial.

    \item
      Consider the case in which $\adec{\octx}{n}$ and $\adec{\lctx}{n}$.
    According to big-step adequacy of decrements for the choreography~\parencref{thm:msg-dec-big-adequacy}, there are three cases that derive from $\lctx$.
    \begin{itemize}
    \item 
      Consider the case in which $\adec{\octx}{0}$ and $\lctx = \adec{\atmR{z}}{0}$.
      According to big-step adequacy of decrements for the choreography, $\octx \Reduces \atmR{z}$ and so, indeed, $\octx \Reduces\rframe{\reflc{\simu{R}}}{\atmR{z}} \atmR{z} = \lctx$.

    \item 
      Consider the case in which $\adec{\octx}{n}$ and $n > 0$ and $\lctx = \lctx' \oc \atmR{s}$ and $\ainc{\lctx'}{n-1}$.
      According to big-step adequacy of decrements for the choreography, $\octx \Reduces \octx' \oc \atmR{s}$ for some $\octx'$ such that $\ainc{\octx'}{n-1}$.
      It immediately follows that $\octx \Reduces\rframe{\simu{R}}{\atmR{s}} \lctx' \oc \atmR{s} = \lctx$.

    \item
      Consider the case in which $\adec{\octx}{n}$ and $\adec{\lctx}{n}$, with both $\atmL{\lctx}'_L$ and $\atmR{\lctx}'_R$ empty.
      The required $\octx \Reduces\lrframe{\atmL{\lctx}'_L}{\reflc{\simu{R}}}{\atmR{\lctx}'_R} \lctx$ is trivial.
    \end{itemize}
    \end{itemize}

  % \item[Immediate output bisimulation]
  %   Assume that $\octx \simu{R} \lctx = \atmL{\lctx}'_L \oc \lctx' \oc \atmR{\lctx}'_R$.
  %   According to big-step adequacy of decrements~\parencref{thm:msg-dec-big-adequacy}, there are three cases that derive from $\lctx$.
  %   \begin{itemize}
  %   \item 
  %     Consider the case in which $\adec{\octx}{0}$ and $\lctx = \adec{\atmR{z}}{0}$.
  %     According to big-step adequacy of decrements~\parencref{thm:msg-dec-big-adequacy}, $\octx \Reduces \atmR{z}$ and so, indeed, $\octx \Reduces\rframe{\reflc{\simu{R}}}{\atmR{z}} \atmR{z}$.

  %   \item 
  %     Consider the case in which $\adec{\octx}{n}$ and $n > 0$ and $\lctx = \lctx' \oc \atmR{s}$ and $\ainc{\lctx'}{n-1}$.
  %     According to big-step adequacy of decrements~\parencref{thm:msg-dec-big-adequacy}, $\octx \Reduces \octx' \oc \atmR{s}$ for some $\octx'$ such that $\ainc{\octx'}{n-1}$.
  %     It immediately follows that $\octx \Reduces\rframe{\simu{R}}{\atmR{s}} \lctx$.

  %   \item
  %     Consider the case in which $\adec{\octx}{n}$ and $\adec{\lctx}{n}$, with both $\atmL{\lctx}'_L$ and $\atmR{\lctx}'_R$ empty.
  %     The required $\octx \Reduces\lrframe{\atmL{\lctx}'_L}{\reflc{\simu{R}}}{\atmR{\lctx}'_R} \lctx$ is trivial.
  %   \end{itemize}


    % \begin{itemize}
    % \item 
    %   Consider the case in which $\octx \simu{R} \atmR{z}$ and $\adec{\octx}{0}$.
    %   According to \cref{??}, $\octx \Reduces \atmR{z}$ and so, indeed, $\octx \Reduces\rframe{\simu{R}}{\atmR{z}} \atmR{z}$.

    % \item 
    %   Consider the case in which $\octx \simu{R} \lctx = \lctx' \oc \atmR{s}$ and $\adec{\octx}{n'+1} = n$ and $\ainc{\lctx'}{n'}$.
    %   According to \cref{??}, $\octx \Reduces \octx' \oc \atmR{s}$ for some $\octx'$ such that $\ainc{\octx'}{n'}$.
    %   It immediately follows that $\octx \Reduces\rframe{\simu{R}}{\atmR{s}} \lctx$.

    % \item
    %   Consider the case in which $\octx \simu{R} \lctx = \lctx'$ and both $\atmL{\lctx}'_L$ and $\atmR{\lctx}'_R$ are empty.
    %   The required $\octx \Reduces\lrframe{\atmL{\lctx}'_L}{\simu{R}}{\atmR{\lctx}'_R} \lctx$ is trivial.
    % \end{itemize}

  \item[Immediate input bisimulation]
    Assume that $\octx \simu{R} \lctx$ and $\ireduces{\atmR{\lctx}_L \oc #1 \oc \atmL{\lctx}_R}{\lctx}{\lctx'}$.
    There are several cases.
    \begin{itemize}[listparindent=\parindent]
    \item
      Consider the case in which $\ainc{\octx}{n}$ and $\ainc{\lctx}{n}$, for some natural number $n$.
      According to \cref{lem:ordered-bisimilarity:oo-counter-bisim-easy}, the input transition is either $\ireduces{#1 \oc \atmL{i}}{\lctx}{\lctx'}$ or $\ireduces{#1 \oc \atmL{d}}{\lctx}{\lctx'}$.
      \begin{itemize}
      \item
        If the input transition $\ireduces{\atmR{\lctx}_L \oc #1 \oc \atmL{\lctx}_R}{\lctx}{\lctx'}$ is $\ireduces{#1 \oc \atmL{i}}{\lctx}{\lctx'}$, then  apply the $\jrule{$\atmL{i}$-I}$ rule to deduce that $\ainc{\octx \oc \atmL{i}}{n+1}$ and $\ainc{\lctx \oc \atmL{i}}{n+1}$.
        Because $\lctx \oc \atmL{i} \reduces \lctx'$, it follows from $\ainc{}{}$-preservation~\parencref{??} that $\ainc{\lctx'}{n+1}$.
        We conclude that $\octx \oc \atmL{i} \Reduces\simu{R} \lctx'$, as required.

      \item
        If the input transition $\ireduces{\atmR{\lctx}_L \oc #1 \oc \atmL{\lctx}_R}{\lctx}{\lctx'}$ is $\ireduces{#1 \oc \atmL{d}}{\lctx}{\lctx'}$, then similar reasoning applies.
      \end{itemize}

    \item 
      Consider the case in which $\adec{\octx}{n}$ and $\adec{\lctx}{n}$, for some $n$.
      By \cref{lem:ordered-bisimilarity:oo-counter-bisim-easy}, this input transition is impossible.
    \end{itemize}

  \item[Reduction bisimulation]
    Assume that $\octx \simu{R} \lctx \reduces \lctx'$.
    If $\ainc{\octx}{n}$ and $\ainc{\lctx}{n}$ for some natural number $n$, then $\ainc{}{}$-preservation~\parencref{??} yields $\ainc{\lctx'}{n}$;
    it immediately follows that $\octx \Reduces\simu{R} \lctx'$.
    Otherwise, if $\adec{\octx}{n}$ and $\adec{\lctx}{n}$ for some natural number $n$, then $\adec{}{}$-preservation similarly allows us to conclude that $\octx \Reduces\simu{R} \lctx'$.

  \item[Emptiness bisimulation]
    This is vacuously true because $(\octxe)$ has no denotation under $\ainc{}{}$ and $\adec{}{}$.
  \end{description}

  \item
    To prove the converse, that bisimilar counters have equal denotations, we shall prove that 
    \begin{enumerate}[noitemsep]
    \item If $\adec{\octx}{n}$ and $\adec{\lctx}{n'}$ and $\octx \osim \lctx$, then $n = n'$.
    \item If $\ainc{\octx}{n}$ and $\ainc{\lctx}{n'}$ and $\octx \osim \lctx$, then $n = n'$.
    \end{enumerate}
    using a lexicographic induction, first on the denotation $n$ and then on the inductive hypothesis used, with $\text{1} < \text{2}$.
  \begin{enumerate}
  \item Assume that $\adec{\octx}{n}$ and $\adec{\lctx}{n'}$ and $\octx \osim \lctx$.
    \begin{itemize}
    \item
      Consider the case in which $n = 0$.
      By big-step adequacy of decrements~\parencref{cor:oo-counter-adequacy}, $\octx \Reduces \atmR{z}$.
      Because $\octx$ and $\lctx$ are bisimilar, $\lctx \Reduces\rframe{\osim}{\atmR{z}} \atmR{z}$.
      According to big-step adequacy of decrements again, $\lctx$ eventually emits $\atmR{z}$ only if its denotation is $n' = 0$, and so $n = 0 = n'$.

    \item 
      Consider the case in which $n > 0$.
      By big-step adequacy of decrements~\parencref{cor:oo-counter-adequacy}, $\octx \Reduces \octx' \oc \atmR{s}$ for some $\octx'$ such that $\ainc{\octx'}{n-1}$.
      Because $\octx$ and $\lctx$ are bisimilar, $\lctx \Reduces\rframe{\osim}{\atmR{s}} \octx' \oc \atmR{s}$; in other words, $\lctx \Reduces \lctx' \oc \atmR{s}$ for some $\lctx'$ such that $\octx' \osim \lctx'$.
      According to big-step adequacy of decrements again, $\lctx$ eventually emits $\atmR{s}$ only if $n' > 0$ and $\ainc{\lctx'}{n'-1}$.
      By the inductive hypothesis, it follows that $n-1 = n'-1$, and so $n = n'$ as required.
    \end{itemize}
    
  \item Assume that $\ainc{\octx}{n}$ and $\ainc{\lctx}{n'}$ and $\octx \osim \lctx$.
    By applying the $\jrule{$\atmL{d}$-D}$ rule, we may deduce that $\adec{\octx \oc \atmL{d}}{n}$ and $\adec{\lctx \oc \atmL{d}}{n'}$.
    Moreover, because rewriting bisimilarity is a congruence~\parencref{??}, $\octx \oc \atmL{d} \osim \lctx \oc \atmL{d}$.
    By part~\ref{item:??} of the inductive hypothesis, we conclude that $n = n'$, as required.
  %
  \qedhere
  \end{enumerate}
  \end{itemize}
\end{proof}



% \begin{theorem}[Adequacy of binary counter decrements]
%   If $\adec{\octx}{n}$, then:
%   \begin{itemize}[nosep]
%   \item $\octx \Reduces \atmR{z}$ if, and only if, $n = 0$;
%   \item $\octx \Reduces \octx' \oc \atmR{s}$ for some $\octx'$ such that $\ainc{\octx'}{n-1}$, if $n > 0$; and
%   \item $\octx \Reduces \octx' \oc \atmR{s}$ only if $n > 0$ and $\ainc{\octx'}{n-1}$.
%   \end{itemize}
% \end{theorem}

% \begin{theorem}[Small-step adequacy of binary counter decrements]\leavevmode
%   \begin{thmdescription}
%   \item[Preservation]
%     If $\adec{\octx}{n}$ and $\octx \reduces \octx'$, then $\adec{\octx'}{n}$.
%   \item[Progress]
%     If $\adec{\octx}{n}$, then either:
%     \begin{itemize}[nosep]
%     \item $\octx \reduces \octx'$, for some $\octx'$;
%     \item $n = 0$ and $\octx = \atmR{z}$; or
%     \item $n > 0$ and $\octx = \octx' \oc \atmR{s}$, for some $\octx'$ such that $\ainc{\octx'}{n-1}$.\fixnote{Value instead of increment relation?}
%     \end{itemize}
%   \item[Termination]
%     If $\adec{\octx}{n}$, then every rewriting sequence from $\octx$ is finite.
%   \end{thmdescription}
% \end{theorem}


\subsection{A comment on atom directions and bisimilarity}

This \lcnamecref{thm:oo-counter-bisim} gives us the opportunity to remark on the interplay between atoms' directionality and rewriting bisimilarity.

Suppose that the formula-as-process ordered rewriting framework was designed without assigning direction to atoms.
Instead of the directional atoms $\atmL{a}$ and $\atmR{a}$, the framework would have only un(i)directional\fixnote{bidirectional?} atoms $\atm{a}$.
The definition of rewriting bisimilarity would also be ever so slightly revised to use the un(i)directed atoms.
Rewriting bisimilarity would be the largest symmetric relation $\simu{R}$ to satisfy:
\begin{description}[noitemsep]
\item[Output bisimulation]
  If $\octx \simu{R}\Reduces \atm{\lctx}'_L \oc \lctx' \oc \atm{\lctx}'_R$, then $\octx \Reduces\lrframe{\atm{\lctx}'_L}{\simu{R}}{\atm{\lctx}'_R} \atm{\lctx}'_L \oc \lctx' \oc \atm{\lctx}'_R$.
\item[Input bisimulation]
  If $\atm{\lctx}_L \oc \octx \oc \atm{\lctx}_R \lrframe{\atm{\lctx}_L}{\simu{R}}{\atm{\lctx}_R}\Reduces \lctx'$, then $\atm{\lctx}_L \oc \octx \oc \atm{\lctx}_R \Reduces\simu{R} \lctx'$.
\end{description}

Unfortunately, this un(i)directed notion of bisimilarity would be too fine.
Without atom directions to distinguish messages intended as inputs from those intended as outputs, input messages could be incorrectly observed as outputs.
In practice, this means that bisimilarity would rule out desirable equivalences.

Consider, for example, $\defp{e}$ and $\defp{e} \oc \defp{b}_0$.
These two counters have the same denotation -- both represent $0$.
Nevertheless, in opposition to \cref{thm:oo-counter-bisim} for directed atoms, $\defp{e}$ and $\defp{e} \oc \defp{b}_0$ would \emph{not} be bisimilar when using un(i)directed atoms.

To see why, suppose, for the sake of deriving a contradiction, that $\defp{e} \osim \defp{e} \oc \defp{b}_0$.
Because $\defp{e} \oc \defp{b}_0 \oc \atm{d} \Reduces \atm{z} \oc \defp{b}'_0$, it follows from the input bisimulation property that $\defp{e} \oc \atm{d} \Reduces\osim \atm{z} \oc \defp{b}'_0$.
There are only two contexts that can arise from $\defp{e} \oc \atm{d}$, so either $\defp{e} \oc \atm{d} \osim \atm{z} \oc \defp{b}'_0$ or $\atm{z} \osim \atm{z} \oc \defp{b}'_0$.
\begin{itemize}
\item The former is impossible because $\atm{z} \oc \defp{b}'_0$ cannot produce $\atm{d}$ on the right (nor on the left) and so violates output bisimilarity.

\item The latter is also impossible.
It has an output of $\atm{z}$ on the left of $\atm{z} \oc \defp{b}'_0$, from which the output bisimulation property yields $(\octxe) \osim \defp{b}'_0$.
From the input bisimulation property, $\atm{a} \osim \defp{b}'_0 \oc \atm{a}$ follows, for any atom $\atm{a}$.
And that violates output bisimulation because $\defp{b}'_0 \oc \atm{a}$, which does not reduce, cannot match the left output that $\atm{a}$ makes.
\end{itemize}

The key feature of this counterexample is that atoms' lack of direction means that the output bisimilarity condition also applies to atoms intended to act as inputs ($\atm{d}$ and $\atm{a}$, for instance).
Just as atom directions were used in \cref{??} to prevent a process from capturing a message it just sent, so do atom directions prevent \emph{input} messages from being observed.


% \subsection{An alternative specification of a binary counter}

% The above description of a binary counter, repeated here%
% \begin{marginfigure}
%   \begin{equation*}
%     \begin{lgathered}
%       e \defd (e \fuse b_1 \pmir \atmL{i}) \with (\atmR{z} \pmir \atmL{d}) \\
%       b_0 \defd (\up \dn b_1 \pmir \atmL{i}) \with (\atmL{d} \fuse b'_0 \pmir \atmL{d}) \\
%       b_1 \defd (\atmL{i} \fuse b_0 \pmir \atmL{i}) \with (b_0 \fuse \atmL{s} \pmir \atmL{d}) \\
%       b'_0 \defd (\atmR{z} \limp \atmR{z}) \with (\atmR{s} \limp b_1 \fuse \atmR{s})
%     \end{lgathered}
%   \end{equation*}
%   \caption{An object-oriented specification of a binary counter}
% \end{marginfigure}
% for convenience, could be described as object-oriented.
% Like objects, the processes $e$, $b_0$, and $b_1$ dispatch on incoming messages $\atmL{i}$ and $\atmL{d}$, and the process $b'_0$ dispatches on incoming messages $\atmR{z}$ and $\atmR{s}$.%
% \footnote{For a study of the relationship between (session-typed) processes and objects, see \textcite{Balzer+Pfenning:AGERE15}.}

% Alternatively, we could specify the binary counter in a dual way: like functions are applied to data, the processes $i$ and $d$ act on incoming messages $\atmR{e}$, $\atmR{b}_0$, and $\atmR{b}_1$, and the processes $z$ and $s$ act on incoming $\atmL{b}'_0$ messages.
% \begin{equation*}
%   \begin{lgathered}
%     i \defd (\atmR{e} \limp \atmR{e} \fuse \atmR{b}_1) \with (\atmR{b}_0 \limp \atmR{b}_1) \with (\atmR{b}_1 \limp i \fuse \atmR{b}_0) \\
%     d \defd (\atmR{e} \limp \up \dn z) \with (\atmR{b}_0 \limp d \fuse \atmL{b}'_0) \with (\atmR{b}_1 \limp \atmR{b}_0 \fuse s) \\
%     z \defd \up \dn z \pmir \atmL{b}'_0 \\
%     s \defd \atmR{b}_1 \fuse s \pmir \atmL{b}'_0
%   \end{lgathered}
% \end{equation*}
% In contrast with the earlier object-oriented specification, this specification could be described as functional in style.
% \begin{equation*}
%   \atmR{e} \oc \atmR{b}_1 \oc i \Reduces \atmR{e} \oc i \oc \atmR{b}_0 \Reduces \atmR{e} \oc \atmR{b}_1 \oc \atmR{b}_1
% \end{equation*}

% Intuitively, we should expect these two specifications to be equivalent descriptions of a binary counter.
% To make this equivalence concrete, we might imagine defining a binary relation $\simu{D}$ on binary counters that makes the duality precise;
% for example, $e \oc b_1 \oc \atmL{i} \simu{D} \atmR{e} \oc \atmR{b}_1 \oc i$.

% However, in defining the duality relation, we implicitly observe and compare the counters' internal structures.
% Although certainly possible at the meta-level, this is somewhat unsatisfying because it doesn't compare the counters' \emph{behaviors}.
% There ought to be a way to characterize the counters' equivalence using bisimilarity.

% Doing so requires a few small changes to 



% \section{Unary counter}

% \begin{equation*}
%   \begin{lgathered}
%     z \defd (z \fuse s \pmir \atmL{i}) \with (\atmR{z} \pmir \atmL{d}) \\
%     s \defd (s \fuse s \pmir \atmL{i}) \with (\atmR{s} \pmir \atmL{d})
%   \end{lgathered}
% \end{equation*}

% \begin{inferences}
%   \infer{\ainc{z}{0}}{}
%   \and
%   \infer{\ainc{\lctx \oc s}{n+1}}{
%     \ainc{\lctx}{n}}
%   \and
%   \infer{\ainc{\lctx \oc \atmL{i}}{n+1}}{
%     \ainc{\lctx}{n}}
%   \\
%   \infer{\ainc{z \fuse s}{1}}{}
%   \and
%   \infer{\ainc{\lctx \oc (s \fuse s)}{n+2}}{
%     \ainc{\lctx}{n}}
%   \\
%   \infer{\adec{\lctx \oc \atmL{d}}{n}}{
%     \ainc{\lctx}{n}}
%   \and
%   \infer{\adec{\atmR{z}}{0}}{}
%   \and
%   \infer{\adec{\lctx \oc \atmR{s}}{n+1}}{
%     \ainc{\lctx}{n}}
% \end{inferences}

% \begin{theorem}[Adequacy of unary counter increments]
%   If $\ainc{\lctx}{n}$, then $\lctx \Reduces\aval{}{n'}$ if, and only if, $n' = n+1$.
% \end{theorem}

% \begin{theorem}[Small-step adequacy of unary counter increments]\leavevmode
%   \begin{thmdescription}
%   \item[Preservation]
%     If $\ainc{\lctx}{n}$ and $\lctx \reduces \lctx'$, then $\ainc{\lctx'}{n}$.
%   \item[Progress]
%     If $\ainc{\lctx}{n}$, then either:
%     \begin{itemize}[nosep]
%     \item $\lctx \reduces \lctx'$, for some $\lctx'$; or
%     \item $\lctx \nreduces$ and $\aval{\lctx}{n}$.
%     \end{itemize}
%   \item[Termination]
%     If $\ainc{\lctx}{n}$, then every rewriting sequence from $\lctx$ is finite.
%   \end{thmdescription}
% \end{theorem}

% \begin{theorem}[Adequacy of unary counter decrements]
%   If $\adec{\lctx}{n}$, then:
%   \begin{itemize}[nosep]
%   \item $\lctx \Reduces \atmR{z}$ if, and only if, $n = 0$;
%   \item $\lctx \Reduces \lctx' \oc \atmR{s}$ for some $\lctx'$ such that $\ainc{\lctx'}{n-1}$, if $n > 0$; and
%   \item $\lctx \Reduces \lctx' \oc \atmR{s}$ only if $n > 0$ and $\ainc{\lctx'}{n-1}$.
%   \end{itemize}
% \end{theorem}

% \begin{theorem}[Small-step adequacy of unary counter decrements]\leavevmode
%   \begin{thmdescription}
%   \item[Preservation]
%     If $\adec{\lctx}{n}$ and $\lctx \reduces \lctx'$, then $\adec{\lctx'}{n}$.
%   \item[Progress]
%     If $\adec{\lctx}{n}$, then either:
%     \begin{itemize}[nosep]
%     \item $\lctx \reduces \lctx'$, for some $\lctx'$;
%     \item $n = 0$ and $\lctx = \atmR{z}$; or
%     \item $n > 0$ and $\lctx = \lctx' \oc \atmR{s}$, for some $\lctx'$ such that $\ainc{\lctx'}{n-1}$.\fixnote{Value instead of increment relation?}
%     \end{itemize}
%   \item[Termination]
%     If $\adec{\lctx}{n}$, then every rewriting sequence from $\lctx$ is finite.
%   \end{thmdescription}
% \end{theorem}


% % \begin{lemma}\leavevmode
% %   \begin{itemize}[nosep]
% % %  \item If $\ainc{\lctx}{0}$, then $\lctx = z$.
% % %  \item If $\ainc{\lctx}{n+1}$, then $\lctx \Reduces \lctx' \oc s$ for some $\ainc{\lctx'}{n}$.
% %   \item If $\adec{\lctx}{0}$, then $\lctx \Reduces \atmR{z}$.
% %   \item If $\adec{\lctx}{n+1}$, then $\lctx \Reduces \lctx' \oc \atmR{s}$ for some $\ainc{\lctx'}{n}$.
% % %   \end{itemize}
% % % \end{lemma}

% % % \begin{lemma}\leavevmode
% % %   \begin{itemize}[nosep]
% % %  \item If $\ainc{\octx}{0}$, then either $\octx = e$ or $\octx = \octx' \oc b_0$ for some $\ainc{\octx'}{0}$.
% % %  \item If $\ainc{\octx}{2n} > 0$, then $\octx \Reduces \octx' \oc b_0$ for some $\ainc{\octx'}{n}$.
% % %  \item If $\ainc{\octx}{2n+1}$, then $\octx \Reduces \octx' \oc b_1$ for some $\ainc{\octx'}{n}$.
% %   % \item If $\adec{\octx}{0}$, then $\octx \Reduces \atmR{z}$.
% %   % \item If $\adec{\octx}{n+1}$, then $\octx \Reduces \octx' \oc \atmR{s}$ for some $\ainc{\octx'}{n}$.
% %   \end{itemize}
% % \end{lemma}

% \begin{theorem}[Bisimilarity of counters]\leavevmode
%   \begin{itemize}[nosep]
%   \item If $\ainc{\octx}{n}$ and $\ainc{\lctx}{n}$, then $\octx \osim \lctx$.
%   \item If $\adec{\octx}{n}$ and $\adec{\lctx}{n}$, then $\octx \osim \lctx$.
%   \end{itemize}
% \end{theorem}
% \begin{proof}
%   Let $\simu{R}$ be the symmetric closure of $\Set{(\octx, \lctx) \given \exists n.\, (\ainc{\octx}{n}) \land (\ainc{\lctx}{n})} \union \Set{(\octx, \lctx) \given \exists n.\, (\adec{\octx}{n}) \land (\adec{\lctx}{n})} \union \Set{(\octxe, \octxe)}$.
%   We will show that $\simu{R}$ is a labeled bisimulation.
%   \begin{description}
%   \item[Immediate output bisimulation]
%     \begin{itemize}
%     \item Consider the case in which $\adec{\octx}{0}$ and $\adec{\atmR{z}}{0}$.
%       By \cref{??}, $\octx \Reduces \atmR{z}$.
%       It follows that $\octx \Reduces\rframe{\simu{R}}{\atmR{z}} \atmR{z}$, as required.
%     \item Consider the case in which $\adec{\octx}{n+1}$ and $\adec{\lctx \oc \atmR{s}}{n+1}$ because $\ainc{\lctx}{n}$.
%       By \cref{??}, $\octx \Reduces \octx' \oc \atmR{s}$ for some $\ainc{\octx'}{n}$.
%       It follows that $\octx \Reduces\rframe{\simu{R}}{\atmR{s}} \lctx \oc \atmR{s}$, as required.
%     \item Consider the case in which $\adec{\atmR{z}}{0}$ and $\adec{\lctx}{0}$.
%       By \cref{??}, $\lctx \Reduces \atmR{z}$.
%       It follows that $\lctx \Reduces\rframe{\simu{R}}{\atmR{z}} \atmR{z}$, as required.
%     \item Consider the case in which $\adec{\octx \oc \atmR{s}}{n+1}$ and $\adec{\lctx}{n+1}$ because $\ainc{\octx}{n}$.
%       By \cref{??}, $\lctx \Reduces \lctx' \oc \atmR{s}$ for some $\ainc{\lctx'}{n}$.
%       It follows that $\lctx \Reduces\rframe{\simu{R}}{\atmR{s}} \octx \oc \atmR{s}$, as required.
%     \end{itemize}

%   \item[Immediate input bisimulation]
%     \begin{itemize}
%     \item Consider the case in which $\ainc{\octx}{n}$ and $\ainc{\lctx}{n}$ and $\ireduces{#1 \oc \atmL{i}}{\lctx}{\lctx'}$.
%       Notice that $\ainc{\lctx \oc \atmL{i}}{n+1}$ and $\lctx \oc \atmL{i} \reduces \lctx'$; by preservation \parencref{??}, $\ainc{\lctx'}{n+1}$.
%       It is then trivial that $\octx \oc \atmL{i} \Reduces\simu{R} \lctx'$, as required.
%     \item Consider the case in which $\ainc{\octx}{n}$ and $\ainc{\lctx}{n}$ and $\ireduces{#1 \oc \atmL{d}}{\lctx}{\lctx'}$.
%       Notice that $\adec{\lctx \oc \atmL{d}}{n}$ and $\lctx \oc \atmL{d} \reduces \lctx'$; by preservation \parencref{??}, $\adec{\lctx'}{n}$.
%       It is then trivial that $\octx \oc \atmL{d} \Reduces\simu{R} \lctx'$, as required.
%     \end{itemize}

%   \item[Reduction bisimulation]
%     \begin{itemize}
%     \item Consider the case in which $\ainc{\octx}{n}$ and $\ainc{\lctx}{n}$ and $\lctx \reduces \lctx'$.
%       By preservation \parencref{??}, $\ainc{\lctx'}{n}$.
%       It is then trivial that $\octx \Reduces\simu{R} \lctx'$, as required.
%     \item Consider the case in which $\adec{\octx}{n}$ and $\adec{\lctx}{n}$ and $\lctx \reduces \lctx'$.
%       In this case, the proof is similar to the above increment case.
%     \item Consider the cases in which a binary counter $\octx$ reduces; these are analogous to the previous cases involving a unary counter that reduces.
%     \end{itemize}

%   \item[Emptiness bisimulation]
%     The only case involving empty contexts is that of $\octxe \simu{R} \octxe$.
%     In this case, indeed $\atmR{\lctx} \Reduces\rframe{\simu{R}}{\atmR{\lctx}} \atmR{\lctx}$ for all $\atmR{\lctx}$, and, symmetrically, $\atmL{\lctx} \Reduces\lframe{\atmL{\lctx}}{\simu{R}} \atmL{\lctx}$ for all $\atmL{\lctx}$.
%   \end{description}

%   % \begin{description}
%   % \item[Immediate output]
%   %   Vacuous
%   % \item[Immediate input] 
%   %   \begin{itemize}
%   %   \item Suppose that $\ainc{e}{0}$ and $\ainc{z}{0}$.
%   %     \begin{itemize}
%   %     \item Suppose that $\ireduces{#1 \oc \atmL{i}}{e}{e \fuse b_1}$.
%   %       $z \oc \atmL{i} \Reduces \ainc{z \oc s}{1}$.
%   %       Symmetrically involving transition on $z$.
%   %     \item Suppose that $\ireduces{#1 \oc \atmL{d}}{e}{\atmR{z}}$.
%   %       $z \oc \atmL{d} \Reduces \adec{\atmR{z}}{0}$.
%   %       Symmetrically.
%   %     \end{itemize}
%   %   \item $\ainc{\octx \oc b_0}{0}$ and$\ainc{z}{0}$
%   %     \begin{itemize}
%   %     \item $\octx \oc b_1$ and $z \fuse s$
%   %     \item $\octx \oc (d \fuse b'_0)$ and $\atmR{z}$
%   %     \end{itemize}
%   %   \end{itemize}
%   % \item[Reduction] 
%   %   Preservation
%   % \item[Emptiness] 
%   %   Vacuous
%   % \end{description}

%   % \begin{description}
%   % \item[Immediate output]
%   %   \begin{itemize}
%   %   \item $\atmR{z}$ and $\atmR{z}$
%   %   \item $\octx \oc \atmR{s}$ and $\lctx \oc \atmR{s}$
%   %   \end{itemize}
%   % \item[Immediate input] 
%   %   Vacuous
%   % \item[Reduction] 
%   %   Preservation
%   % \item[Emptiness] 
%   %   Vacuous
%   % \end{description}
% \end{proof}


% \section{Without using the adequacy relations}

% \begin{inferences}
%   \infer{e \simu{R}_v z}{}
%   \and
%   \infer{\octx \oc b_0 \simu{R}_v \lctx}{
%     \octx \simu{R}_v\simu{D} \lctx}
%   \and
%   \infer{\octx \oc b_1 \simu{R}_v \lctx \oc s}{
%     \octx \simu{R}_v\simu{D} \lctx}
%   \\
%   \infer{z \simu{D} z}{}
%   \and
%   \infer{\lctx \oc s \simu{D} \lctx' \oc s \oc s}{
%     \lctx \simu{D} \lctx'}
% \end{inferences}

% \begin{inferences}
%   \infer{e \simu{R}_i z}{}
%   \and
%   \infer{\octx \oc b_0 \simu{R}_i \lctx}{
%     \octx \simu{R}_i\simu{D} \lctx}
%   \and
%   \infer{\octx \oc b_1 \simu{R}_i \lctx \oc s}{
%     \octx \simu{R}_i\simu{D} \lctx}
%   \and
%   \infer{\octx \oc \atmL{i} \simu{R}_i \lctx \oc s}{
%     \octx \simu{R}_i \lctx}
%   \\
%   \infer{e \fuse b_1 \simu{R}_i \lctx}{
%     e \oc b_1 \simu{R}_i \lctx}
%   \and
%   \infer{\octx \oc (\atmL{i} \fuse b_0) \simu{R}_i \lctx}{
%     \octx \oc \atmL{i} \oc b_0 \simu{R}_i \lctx}
%   \\
%   \infer{\octx \simu{R}_i z \fuse s}{
%     \octx \simu{R}_i z \oc s}
%   \and
%   \infer{\octx \simu{R}_i \lctx \oc (s \fuse s)}{
%     \octx \simu{R}_i \lctx \oc s \oc s}
% \end{inferences}

% \begin{inferences}
%   \infer{\octx \oc \atmL{d} \simu{R}_d \lctx \oc \atmL{d}}{
%     \octx \simu{R}_i \lctx}
%   \and
%   \infer{\atmR{z} \simu{R}_d \atmR{z}}{}
%   \and
%   \infer{\octx \oc \atmR{s} \simu{R}_d \lctx \oc \atmR{s}}{
%     \octx \simu{R}_i \lctx}
%   \and
%   \infer{\octx \oc b'_0 \simu{R}_d \lctx}{
%     \octx \simu{R}_d\simu{D}_d \lctx}
%   \\
%   \infer{\octx \oc (\atmL{d} \fuse b'_0) \simu{R}_d \lctx}{
%     \octx \oc \atmL{d} \oc b'_0 \simu{R}_d \lctx}
%   \and
%   \infer{\octx \oc (b_1 \fuse \atmR{s}) \simu{R}_d \lctx}{
%     \octx \oc b_1 \oc \atmR{s} \simu{R}_d \lctx}
%   \and
%   \infer{\octx \oc (b_0 \fuse \atmR{s}) \simu{R}_d \lctx}{
%     \octx \oc b_0 \oc \atmR{s} \simu{R}_d \lctx}
% \end{inferences}

% \begin{inferences}
%   \infer{\atmR{z} \simu{D}_d \atmR{z}}{}
%   \and
%   \infer{\lctx \oc \atmR{s} \simu{D}_d \lctx' \oc s \oc \atmR{s}}{
%     \lctx \simu{D} \lctx'}
% \end{inferences}

% \begin{description}
% \item[Immediate output bisimulation]
%   Consider the case in which 
% \end{description}


% \begin{description}
% \item[Immediate input]
%   \begin{itemize}
%   \item $z \oc \atmL{i} \Reduces\simu{R} e \fuse b_1$
%   \item $\lctx \oc \atmL{i} \Reduces\simu{R} \octx \oc b_1$ when $\octx \simu{R}\simu{D} \lctx$
%   \item $\lctx \oc s \oc \atmL{i} \Reduces\simu{R} \octx \oc (\atmL{i} \fuse b_0)$ when $\octx \simu{R}\simu{D} \lctx$
%   \item $e \oc \atmL{i} \Reduces\simu{R} z \fuse s$
%   \item $\octx \oc b_0 \oc \atmL{i} \Reduces\simu{R} z \fuse s$ when $\octx \simu{R} z$
%   \item $\octx \oc b_0 \oc \atmL{i} \Reduces\simu{R} \lctx \oc s \oc (s \fuse s)$ when $\octx \simu{R}\simu{D} \lctx \oc s \oc s$
%   \item $\octx \oc b_1 \oc \atmL{i} \Reduces\simu{R} \lctx \oc (s \fuse s)$ when $\octx \simu{R}\simu{D} \lctx$.
%     In the case, $\octx \oc \atmL{i} \oc b_0 \simu{R} \lctx \oc s \oc s$.
%   \end{itemize}
% \item[Reduction] 
%   \begin{itemize}
%   \item $\lctx \Reduces\simu{R} \octx' \oc b_0$ when $\lctx \simu{D}^{-1}\Reduces\simu{R} \octx'$
%   \item $\lctx \oc s \Reduces\simu{R} \octx' \oc b_1$ when $\lctx \simu{D}^{-1}\Reduces\simu{R} \octx'$
%   \item $\lctx \oc s \Reduces\simu{R} \octx' \oc \atmL{i}$ when $\lctx \Reduces\simu{R} \octx'$
%   \item $\lctx \Reduces\simu{R} e \oc b_1$ when $\lctx \simu{R} e \oc b_1$
%   \item $z \fuse s \Reduces\simu{R} \octx'$ when $z \oc s \Reduces\simu{R} \octx'$
%   \item $\octx \oc b_0 \Reduces\simu{R} \lctx'$ when $\octx \simu{R}\simu{D}\reduces \lctx'$
%   \item $\octx \oc b_1 \Reduces\simu{R} \lctx'$ when $\octx \simu{R}\simu{D}\reduces \lctx'$
%   \item $\octx \oc \atmL{i} \Reduces\simu{R} \lctx' \oc s$ when $\octx \Reduces\simu{R} \lctx'$
%   \item $e \fuse b_1 \Reduces\simu{R} \lctx'$ when $e \oc b_1 \Reduces\simu{R} \lctx'$
%   \item $\octx \oc (\atmL{i} \fuse b_0) \Reduces\simu{R} \lctx'$ when $\octx \oc \atmL{i} \oc b_0 \Reduces\simu{R} \lctx'$
%   \item $\octx \Reduces\simu{R} z \oc s$ when $\octx \simu{R} z \oc s$
%   \item $\octx \Reduces\simu{R} \lctx' \oc (s \fuse s)$ when $\octx \Reduces\simu{R} \lctx' \oc s \oc s$
%   \item $\octx \Reduces\simu{R} \lctx \oc s \oc s$ when $\octx \simu{R} \lctx \oc s \oc s$
%   \end{itemize}
% \end{description}


% \begin{equation*}
%   \begin{lgathered}
%     c \defd (i \fuse c \pmir \atmL{i}) \with (d \fuse \atmL{u} \pmir \atmL{d}) \\
%     z \defd (\up \dn z \pmir \atmL{b}'_0) \with (\atmR{z} \pmir \atmL{u}) \\
%     s \defd (\atmR{b}_1 \fuse s \pmir \atmL{b}'_0) \with (c \fuse \atmR{s} \pmir \atmL{u})
%   \end{lgathered}
% \end{equation*}

% \begin{inferences}
%   \infer{\iainc{\atmR{e}}{0}}{}
%   \and
%   \infer{\iainc{\lctx \oc \atmR{b}_0}{2n}}{
%     \iainc{\lctx}{n}}
%   \and
%   \infer{\iainc{\lctx \oc \atmR{b}_1}{2n+1}}{
%     \iainc{\lctx}{n}}
%   \and
%   \infer{\iainc{\lctx \oc i}{n+1}}{
%     \iainc{\lctx}{n}}
%   \\
%   \infer{\iainc{\atmR{e} \fuse \atmR{b}_1}{1}}{}
%   \and
%   \infer{\iainc{\lctx \oc (i \fuse \atmR{b}_0)}{2(n+1)}}{
%     \iainc{\lctx}{n}}
% \end{inferences}

% \begin{inferences}
%   \infer{\eainc{\lctx \oc c}{n}}{
%     \iainc{\lctx}{n}}
%   \and
%   \infer{\eainc{\lctx \oc \atmL{i}}{n+1}}{
%     \eainc{\lctx}{n}}
%   \and
%   \infer{\eainc{\lctx \oc (i \fuse c)}{n+1}}{
%     \iainc{\lctx}{n}}
% \end{inferences}

% \begin{theorem}[small-step adequacy of increments]
%   \leavevmode
%   \begin{thmdescription}
%   \item[Value soundness]

%   \item[Preservation]
%     If $\eainc{\lctx}{n}$ and $\lctx \reduces \lctx'$, then $\eainc{\lctx'}{n}$.

%   \item[Progress]
%     If $\eainc{\lctx}{n}$, then either:
%     \begin{itemize*}
%     \item $\lctx \reduces \lctx'$, for some $\lctx'$; or
%     \item $\eaval{\lctx}{n}$.
%     \end{itemize*}

%   \item[Termination]
%     If $\eainc{\lctx}{n}$, then every rewriting sequence from $\lctx$ is finite.
%   \end{thmdescription}
% \end{theorem}

% \begin{inferences}
%   \infer{\iadec{\lctx \oc d}{n}}{
%     \iainc{\lctx}{n}}
%   \and
%   \infer{\iadec{\lctx \oc \atmL{b}'_0}{2n}}{
%     \iadec{\lctx}{n}}
%   \and
%   \infer{\iadec{z}{0}}{}
%   \and
%   \infer{\iadec{\lctx \oc s}{n+1}}{
%     \iainc{\lctx}{n}}
%   \\
%   \infer{\iadec{\lctx \oc (d \fuse \atmL{b}'_0)}{2n}}{
%     \iainc{\lctx}{n}}
%   \and
%   \infer{\iadec{\lctx \oc (\atmR{b}_0 \fuse s)}{2n+1}}{
%     \iainc{\lctx}{n}}
%   \and
%   \infer{\iadec{\lctx \oc (\atmR{b}_1 \fuse s)}{2n+2}}{
%     \iainc{\lctx}{n}}
% \end{inferences}

% \begin{inferences}
%   \infer{\eadec{\lctx \oc \atmL{d}}{n}}{
%     \eainc{\lctx}{n}}
%   \and
%   \infer{\eadec{\lctx \oc \atmL{u}}{n}}{
%     \iadec{\lctx}{n}}
%   \and
%   \infer{\eadec{\atmR{z}}{0}}{}
%   \and
%   \infer{\eadec{\lctx \oc c \oc \atmR{s}}{n+1}}{
%     \iainc{\lctx}{n}}
%   \\
%   \infer{\eadec{\lctx \oc (d \fuse \atmL{u})}{n}}{
%     \iainc{\lctx}{n}}
%   \and
%   \infer{\eadec{\lctx \oc (c \fuse \atmR{s})}{n+1}}{
%     \iainc{\lctx}{n}}
% \end{inferences}


% \begin{theorem}[Small-step adequacy of decrements]
%   \leavevmode
%   \begin{thmdescription}
%   \item[Preservation]
%     If $\eadec{\lctx}{n}$ and $\lctx \reduces \lctx'$, then $\eadec{\lctx'}{n}$.

%   \item[Progress]
%     If $\eadec{\lctx}{n}$, then either:
%     \begin{itemize}[nosep]
%     \item $\lctx \reduces \lctx'$ for some $\lctx'$;
%     \item $n = 0$ and $\lctx = \atmR{z}$;
%     \item $n > 0$ and $\lctx = \lctx' \oc c \oc \atmR{s}$, for some $\lctx'$ such that $\iainc{\lctx'}{n-1}$.
%     \end{itemize}

%   \item[Termination]
%     If $\eadec{\lctx}{n}$, then every rewriting sequence from $\lctx$ is finite.
%   \end{thmdescription}
% \end{theorem}

% \begin{theorem}[Big-step adequacy of decrements]
%   If $\eadec{\lctx}{n}$, then:
%   \begin{itemize}[nosep]
%   \item $\lctx \Reduces \atmL{\lctx}'_L \oc \lctx' \oc \atmR{\lctx}'_R$ only if either: $\atmL{\lctx}'_L = \atmR{\lctx}'_R = \octxe$; or $n = 0$ and $\atmL{\lctx}'_L = \octxe$ and $\atmR{\lctx}'_R = \atmR{z}$; or $n > 0$ and $\atmL{\lctx}'_L = \octxe$ and $\atmR{\lctx}'_R = \atmR{s}$;
%   \item $\lctx \Reduces \atmR{z}$ if $n = 0$;
%   \item $\lctx \Reduces \lctx' \oc c \oc \atmR{s}$ for some $\lctx'$ such that $\iainc{\lctx'}{n-1}$, if $n > 0$; and 
%   \item $\lctx \Reduces \lctx' \oc \atmR{s}$ only if $n > 0$ and $\lctx' = \lctx' \oc c$ for some $\lctx'$ such that $\iainc{\lctx'}{n-1}$.
%   \end{itemize}
% \end{theorem}

% \begin{theorem}[Big-step adequacy of decrements]
%   If $\eadec{\lctx}{n}$, then:
%   \begin{itemize}[nosep]
%   \item $\lctx \Reduces \atmR{z}$ if, and only if, $n = 0$;
%   \item $\lctx \Reduces \lctx' \oc c \oc \atmR{s}$ for some $\lctx'$ such that $\iainc{\lctx'}{n-1}$, if $n > 0$; and 
%   \item $\lctx \Reduces \lctx' \oc \atmR{s}$ only if $n > 0$ and $\lctx' = \lctx' \oc c$ for some $\lctx'$ such that $\iainc{\lctx'}{n-1}$.
%   \end{itemize}
% \end{theorem}

% \begin{theorem}
%   If $\ainc{\octx}{n}$ and $\eainc{\lctx}{n'}$, then $\octx \osim \lctx$ if, and only if, $n = n'$.
%   Similarly, if $\adec{\octx}{n}$ and $\eadec{\lctx}{n'}$, then $\octx \osim \lctx$ if, and only if, $n = n'$.
% \end{theorem}

% \begin{proof}
%   \begin{itemize}
%   \item
%     $\adec{\octx}{0}$ and $\eadec{\atmR{z}}{0}$.
%     $\octx \Reduces \atmR{z}$
%   \item 
%     $\adec{\octx}{n+1}$ and $\eadec{\lctx \oc c \oc \atmR{s}}{n+1}$.
%     $\octx \Reduces \octx' \oc \atmR{s}$ and $\ainc{\octx'}{n}$.
%     $\octx' \simu{R} \lctx \oc c$
%   \end{itemize}
% \end{proof}


% \begin{theorem}\leavevmode
% \begin{itemize}[nosep]
% \item If $\ainc{\octx}{n}$ and $\eainc{\lctx}{n}$, then $\octx \osim \lctx$.
% \item If $\adec{\octx}{n}$ and $\eadec{\lctx}{n}$, then $\octx \osim \lctx$.
% \end{itemize}
% \end{theorem}
% %
% \begin{proof}
%   \begin{description}
%   \item[Immediate output bisimulation]
%     \begin{itemize}
%     \item Consider the case in which $\adec{\octx}{0}$ and $\eadec{\atmR{z}}{0}$.
%       By decrement adequacy, $\octx \Reduces \atmR{z}$.
%       It is trivial that $\octx \Reduces\rframe{\simu{R}}{\atmR{z}} \atmR{z}$.

%     \item Consider the case in which $\adec{\octx}{n+1}$ and $\eadec{\lctx \oc p \oc \atmR{s}}{n+1}$ where $\iainc{\lctx}{n}$.
%       By decrement adequacy, $\octx \Reduces \octx' \oc \atmR{s}$ for some $\ainc{\octx'}{n}$.
%       Notice that $\eainc{\lctx \oc p}{n}$.
%       It follows that $\octx \Reduces\rframe{\simu{R}}{\atmR{s}} \lctx \oc p \oc \atmR{s}$.
      
%     \item Consider the case in which $\adec{\atmR{z}}{0}$ and $\eadec{\lctx}{0}$.
%       By decrement adequacy, $\lctx \Reduces \atmR{z}$.
%       It is trivial that $\lctx \Reduces\rframe{\simu{R}}{\atmR{z}} \atmR{z}$.

%     \item
%       Consider the case in which $\adec{\octx' \oc \atmR{s}}{n+1}$ because $\ainc{\octx'}{n}$ and $\eadec{\lctx}{n+1}$.
%       By decrement adequacy, $\lctx \Reduces \lctx' \oc p \oc \atmR{s}$ for some $\iainc{\lctx'}{n}$.
%       Notice that $\eainc{\lctx' \oc p}{n}$.
%       It follows that $\lctx \Reduces\rframe{\simu{R}}{\atmR{s}} \octx' \oc \atmR{s}$.
%     \end{itemize}
%   \item[Immediate input bisimulation]
%     \begin{itemize}
%     \item Consider the case in which $\ainc{\octx}{n}$ and $\eainc{\lctx}{n}$ and $\ireduces{#1 \oc \atmL{i}}{\lctx}{\lctx'}$.
%       Notice that $\eainc{\lctx \oc \atmL{i}}{n+1}$ and $\lctx \oc \atmL{i} \reduces \lctx'$;
%       by preservation\parencref{??}, $\eainc{\lctx'}{n+1}$.
%       Also, notice that $\ainc{\octx \oc \atmL{i}}{n+1}$, so it is trivial that $\octx \oc \atmL{i} \Reduces\simu{R} \lctx'$, as required.
%       %
%     \item Consider the case in which $\ainc{\octx}{n}$ and $\eainc{\lctx}{n}$ and $\ireduces{#1 \oc \atmL{i}}{\octx}{\octx'}$.
%       Notice that $\ainc{\octx \oc \atmL{i}}{n+1}$ and $\octx \oc \atmL{i} \reduces \octx'$;
%       by preservation\parencref{??}, $\ainc{\octx'}{n+1}$.
%       Also, notice that $\ainc{\lctx \oc \atmL{i}}{n+1}$, so it is trivial that $\lctx \oc \atmL{i} \Reduces\simu{R} \octx'$, as required.
%     \end{itemize}

%   \item[Reduction bisimulation]
%     \begin{itemize}
%     \item Consider the case in which $\ainc{\octx}{n}$ and $\eainc{\lctx}{n}$ and $\lctx \reduces \lctx'$.
%       By preservation\parencref{??}, $\eainc{\lctx'}{n}$.
%       So it is trivial that $\octx \Reduces\simu{R} \lctx'$, as required.
%       %
%     \item Consider the case in which $\ainc{\octx}{n}$ and $\eainc{\lctx}{n}$ and $\octx \reduces \octx'$.
%       By preservation\parencref{??}, $\eainc{\octx'}{n}$.
%       So it is trivial that $\lctx \Reduces\simu{R} \octx'$, as required.
%       %
%     \end{itemize}
%   \end{description}
% \end{proof}


%%% Local Variables:
%%% mode: latex
%%% TeX-master: "thesis"
%%% End:


\part[Concurrency as proof reduction]{Concurrency as\\proof reduction}\label{part:proof-reduction}

\chapter{Singleton logic}\label{ch:singleton-logic}

Intuitionistic sequents are typically asymmetric:
in an intuitionistic sequent $\Gamma \vdash A$, there are finitely many antecedents, all collected into the context $\Gamma$, yet there is only a single consequent, $A$.%
\footnote{Or, at most one consequent if multiplicative falsehood is included~\parencref[see, for example,]{sec:ordered-logic:extensions}.}
We might naturally wonder if a greater degree of symmetry can be brought to sequents.
% Of course, classical sequents in calculi such as Gentzen's LK\autocite{Gentzen:??} do enjoy a pleasant symmetry, but does there exist an \emph{intuitionistic} logic whose sequent calculus presentation uses symmetric sequents?
Of course, classical sequents in calculi such as Gentzen's LK\autocite{Gentzen:MZ35} are symmetric, but does there exist an \emph{intuitionistic} logic whose sequent calculus presentation enjoys a similarly pleasant symmetry?

One approach might be to permit finitely many consequents, as in multiple-conclusion sequent calculi for intuitionistic logic\autocites{Maehara:NMJ54}{Kleene:NH52}, but \citeauthor{Steinberger:JPL11}\autocite{Steinberger:JPL11} raises troubling concerns about the validity of meaning-theoretic explanations of such calculi, primarily that the meanings of the logical connectives are not properly independent but inextricably linked with the meaning of disjunction.

So, in this \lcnamecref{ch:singleton-logic}, we will instead follow a dual path to symmetry and examine a restriction in which sequents have exactly one antecedent -- no more and no less.
We call this requirement the \emph{single-antecedent restriction}; the sequent calculus to which it leads, the \emph{singleton sequent calculus}; and the underlying logic, \emph{singleton logic}.
That such a severe restriction on the structure of sequents yields a well-defined, computationally useful logic is quite surprising.

\newthought{Aside from} motivations of symmetry, the single-antecedent restriction is sensible within each branch of the computational trinity\autocite{Harper:ET11} -- proof theory, category theory, and type theory -- as we sketch in \cref{sec:singleton-logic:restriction}.
This \lcnamecref{ch:singleton-logic} will thereafter focus on the proof-theoretic consequences of the single-antecedent restriction.
% [This \lcnamecref{ch:singleton-logic} is primarily devoted to the proof-theoretic consequences of the single-antecedent restriction, so the operational aspects are mostly postponed to \cref{ch:singleton-processes}.]

Having fully motivated the single-antecedent restriction, we then proceed to \cref{sec:singleton-logic:seq-calc} where we derive the singleton sequent calculus by systematically applying the restriction to the intuitionistic ordered sequent calculus of \cref{ch:ordered-logic}.
(Not all of the ordered logical connectives will be able to survive the restriction, however.
As we will explain, it is precisely the multiplicative connectives that are absent from singleton logic.)

To ensure that the resulting calculus properly defines the meaning of each connective by its inference rules, \cref{sec:singleton-logic:seq-calc:metatheory} establishes the calculus's basic metatheory.
Together, the cut elimination and identity elimination metatheorems identify the cut-free, identity-long proofs as verifications that form the foundation by exhibiting a subformula property.

% To ensure that the resulting calculus is indeed a sequent calculus with the characteristic verificationist meaning-explanation, \cref{sec:singleton-logic:seq-calc:metatheory} then establishes its basic metatheory.
% Together, the cut elimination and identity expansion metatheorems identify the cut-free, $\eta$-long proofs as analytic verifications that exhibit a subformula property.

Yet there are certainly other presentations of logics besides sequent calculi, so, in \cref{sec:singleton-logic:hilbert}, we develop a Hilbert-style axiomatization of singleton logic.
This Hilbert system can also be viewed as a variant of the sequent calculus, so we dub it the \emph{semi-axiomatic sequent calculus}\autocite{DeYoung+:FSCD20}.
% Our interest in a Hilbert system is not, however, zoological;
An analysis of its basic metatheory~\parencref{sec:singleton-logic:sax:cutelim} begins to suggest the basis of a Curry--Howard interpretation of semi-axiomatic sequent proofs as chains of well-typed, asynchronously communicating processes.
\Cref{ch:process-chains} will be devoted to developing that observation more fully.

% Although the single-antecedent restriction precludes a true Hilbert system, we are able to consruct a Hilbert-like system that can also be viewed as a variant of the singleton sequent calculus.


% \Cref{sec:singleton-logic:hilbert:metatheory} establishes the admissibility of a modus ponens-like rule, and contrasts the proof with that of admissibility of cut from \cref{sec:singleton-logic:metatheory}.

Finally, \cref{sec:singleton-logic:extensions} briefly overviews several possible extensions to singleton logic, including a \emph{sub}\-singleton extension that relaxes the single-an\-tecedent restriction and permits an empty context.



% Intuitionistic sequents often contain
% \begin{description}
% \item[Proof-theoretical]
%   Under the single-antecedent restriction, sequents enjoy an elegant symmetry -- exactly one consequent and now exactly one antecedent -- $\slseq{A |- B}$ rather than $\oseq{\octx |- B}$.
% \item[Categorical]
% \item[Operational] 
% \end{description}
% That such a drastic restriction on the structure of sequents yields a well-defined, computationally useful logic is rather surprising.

% Recall from \cref{ch:ordered-logic} that the sequents of intuitionistic ordered logic are inherently asymmetric.
% In an ordered sequent such as $\oseq{\octx |- A}$, there are finitely many antecedents, all collected into the context $\octx$, yet there is only a single consequent.%
% \footnote{Or, at most one consequent if multiplicative falsehood is included\parencref[see]{sec:ordered-logic:mult-falsehood}.}
% We might naturally wonder if a greater degree of symmetry can be brought to these ordered sequents.
% Does there exist a related logic whose sequent calculus presentation uses symmetric sequents?

% To achieve greater symmetry, intuitionism could, of course, be abandoned in favor of a classical noncommutative logic~\autocite{Abrusci:JSL91}.
% One alternative that maintains an intuitionistic character might be to permit finitely many consequents, in a vein similar to multiple-conclusion sequent calculi for intuitionistic logic\autocite{??}.%
% \footnote{However, \textcite{Steinberger:JPL11} raises troubling concerns about meaning-theoretic explanations of such calculi.}

% However, in this \lcnamecref{ch:singleton-logic}, we will instead follow a dual path to symmetry and examine a fragment of intuitionistic ordered logic that is obtained by requiring sequents to have exactly one antecedent -- no more and no less.
% We call this requirement the \emph{single-antecedent restriction} and the logical fragment to which it leads \emph{singleton logic}.
% That such a drastic restriction on the structure of sequents yields a well-defined, computationally useful logic is rather surprising.

% We will begin our examination of singleton logic in \cref{sec:singleton-logic:derive} by deriving its sequent calculus from that of intuitionistic ordered logic by a systematic application of the single-antecedent restriction.
% \Cref{sec:singleton-logic:seq-calc} then establishes the basic metatheory of singleton logic's sequent calculus: identity expansion and, more importantly, admissibility of cut.

% By analogy with a Curry--Howard interpretation of intuitionistic linear logic\autocite{Toninho:?}, we will argue in \cref{sec:singleton-logic:?} that the proof of admissibility of cut describes inherently synchronous interactions between proofs.
% Such a calculus is not well-suited to the remainder of ...

% So, in \cref{sec:singleton-logic:async-seq-calc}, we develop a novel form of sequent calculus and accompanying proof of admissibility of cut that describes inherently asynchronous interactions between proofs.

% A discussion of singleton logic's computational model\alertinline{interpretation?} is postponed to the following \lcnamecref{ch:singleton-processes}.
% In this \lcnamecref{ch:singleton-logic}, we confine our attention to the proof-theoretic aspects of singleton logic.


%  and require exactly one antecedent.

%  we present one such logic, \emph{singleton logic}.
% As its name suggests, singleton logic is the fragment of ordered logic obtained by requiring sequents to have exactly one antecedent -- no more and no less.
% That such a drastic restriction on the structure of contexts yields a well-defined, computationally useful logic is somewhat surprising.

% A discussion of singleton logic's isomorphic\alertnote{Is it really isomorphic?} computational model is postponed to the following \lcnamecref{ch:singleton-processes}.
% In this \lcnamecref{ch:singleton-logic}, we confine our attention to the proof-theoretic aspects of singleton logic.


\section{The single-antecedent restriction}\label{sec:singleton-logic:restriction}

As sketched above, the \emph{single-antecedent restriction} demands that each sequent contain exactly one antecedent, so that sequents are $\slseq{A |- B}$ instead of $\Gamma \vdash B$.

In addition to providing sequents with an elegant symmetry between antecedents and consequents, the single-antecedent restriction is a worthwhile object of investigation when viewed from the perspective of each branch of the computational trinity\autocite{Harper:ET11} -- proof theory, category theory, and type theory:
%
\begin{description}[parsep=0pt, listparindent=\parindent]
\item[Proof theory]
  In sequent calculi, antecedents are subject, either implicitly or explicitly, to structural properties, such as weakening, contraction, and exchange.
  For instance, antecedents in linear logic are subject to exchange, but neither weakening nor contraction; linear contexts thus form a commutative monoid over antecedents.
  Ordered logic goes further and rejects exchange; ordered contexts thus form a \emph{non}\-commutative monoid.

  % The single-antecedent restriction goes still further and rejects the very idea that contexts have any structure whatsoever.
  % Under this restriction, there can be no binary operation to join contexts, so singleton contexts form only the degenerate algebraic structure of a set.

  Singleton logic is a natural object of investigation, precisely because it takes the idea of rejecting structural properties to its extreme.
  In adopting the single-antecedent restriction, singleton logic rejects the very idea that contexts have any structure whatsoever; there can be no binary operation to join contexts.

\item[Category theory]
  Each morphism in a category, $f\colon X \rightarrow Y$, has exactly one object -- no more and no less -- as its domain.
  Because sequents represent a kind of function, single antecedents are just as natural as single-object domains.

  More specifically, in categorical semantics of sequent calculi, proofs are represented by the morphisms of a monoidal category, and so contexts of several antecedents are packaged into a single domain object using the monoidal product:
  % Because proofs sequents should represent a kind of function, categorical semantics for sequent calculi package contexts of several antecedents into a single domain object using a categorical product
  \begin{equation*}
    \ulcorner \DD :: (A_1, A_2, \dotsc, A_n \vdash B) \urcorner :
      \ulcorner\mkern-1mu A_1 \mkern-\thinmuskip\urcorner \otimes \ulcorner\mkern-1mu A_2 \mkern-\thinmuskip\urcorner \otimes \dotsb \otimes \ulcorner\mkern-1mu A_n \mkern-\thinmuskip\urcorner \to \ulcorner B \urcorner
  \end{equation*}
  Because working in a monoidal category complicates matters, it is worthwhile to investigate whether there exists a sequent calculus whose categorical semantics uses no monoidal product.
  % Because the tensor product complicates the category, it is worthwhile to consider whether there exists a sequent calculus whose categorical semantics uses no tensor product.
  The single-antecedent restriction is exactly what results from these considerations, and the singleton sequent calculus will have a cleaner, more direct categorical semantics because of it.

\item[Type theory]
  In \citeauthor{Caires+:MSCS16}'s \acs{SILL} type theory\autocites{Caires+Pfenning:CONCUR10}{Caires+:TLDI12}{Toninho+:ESOP13}{Caires+:MSCS16} based on intuitionistic linear logic, each well-typed process $P$ acts as a client of multiple services $(A_i)_{i=1}^n$ along channels $(x_i)_{i=1}^n$, while simultaneously offering a service $A$ of its own along a single channel $x$.
  Thus, networks of well-typed processes have a tree topology, as depicted in the neighboring display.%
  \marginnote[-1.25\baselineskip]{\centering
    \begin{tikzcd}[ampersand replacement=\&, row sep=small]
      {} \drar[dash][sloped,above]{x_1{:}A_1}
      \\
      {} \rar[phantom]{\smash{\vdots}\vphantom{t}} \& |[circle,draw]| P \& {} \lar[arrow style=tikz,>={Circle[open]}][swap]{x{:}A}
      \\
      {} \urar[dash][sloped,below]{x_n{:}A_n}
    \end{tikzcd}
    \\%
    $\downsquigarrow$%
    \\[\baselineskip]
    \begin{tikzcd}[ampersand replacement=\&, row sep=small]
      {} \rar[dash]{A} \& |[circle,draw]| P \& {} \lar[arrow style=tikz, >={Circle[open]}][swap]{C}
    \end{tikzcd}
    % \begin{tikzpicture}
    %   \graph [nodes={draw}, math nodes] {
    %     / [coordinate] -- P [circle] -> / [coordinate];
    %   };
    % \end{tikzpicture}
  }

  In data pipelines, the computational processes are arranged in a chain topology, with each process having exactly one upstream provider -- no more and no less.
  To study pipelines, a \enquote{single-provider restriction} is needed -- a type-theoretic analogue of the single-antecedent restriction.
\end{description}


\section{A sequent calculus for propositional singleton logic}\label{sec:singleton-logic:seq-calc}

Having sketched proof-theoretic, category-theoretic, and type-theoretic reasons to investigate the single-antecedent restriction, we now turn to identifying a sequent calculus that satisfies that restriction.

% \subsection{Deriving the sequent calculus rules}\label{sec:singleton-logic:seq-calc:derive}

% The single-antecedent restriction described above
% The sequents of singleton logic have exactly one consequent and, more characteristically, exactly one antecedent: $\slseq{A |- B}$ instead of $\oseq{\octx |- B}$.

% In addition to providing sequents with this elegant symmetry, the single-antecedent restriction is also natural from a category-theoretic perspective.
% In a category, each morphism $f\colon X \rightarrow Y$ has exactly one object -- no more and no less -- as its domain.
% Because sequents should represent a kind of function, single antecedents are just as natural as single object domains.

% Once we impose the single-antecedent restriction upon sequents, all of the rules from the sequent calculus for intuitionistic ordered logic must be reconsidered.

\newthought{One approach} to constructing a singleton sequent calculus is to take the intuitionistic ordered sequent calculus of \cref{ch:ordered-logic}, apply the single-antecedent restriction to each rule's sequents, and solve the constraints that that restriction imposes.
% upon the ordered sequent calculus's rules.

For instance, consider the ordered cut rule (see neighboring display).
%
\begin{marginfigure}[8.5\baselineskip]
  \normalsize
  \vspace*{-\abovedisplayskip}
  \begin{gather*}
    \infer[\jrule{CUT}^B]{\oseq{\octx'_L \oc \octx \oc \octx'_R |- C}}{
      \oseq{\octx |- B} & \oseq{\octx'_L \oc B \oc \octx'_R |- C}}
    \\
    \mathord{\downsquigarrow}\hspace{5pt}\hphantom{\jrule{CUT}^B}
    \\
    \infer[\jrule{CUT}^B]{\slseq{A |- C}}{
      \slseq{A |- B} & \slseq{B |- C}}
  \end{gather*}
  \caption{Deriving the singleton sequent calculus's cut rule from the corresponding ordered sequent calculus rule}\label{fig:singleton-logic:seq-calc:derive-cut}
\end{marginfigure}
%
% \begin{equation*}
%   \infer[\jrule{CUT}^B]{\oseq{\octx'_L \oc \octx \oc \octx'_R |- C}}{
%     \oseq{\octx |- B} & \oseq{\octx'_L \oc B \oc \octx'_R |- C}}
% \end{equation*}
For the first premise to satisfy the single-antecedent restriction, the finitary context $\octx$ must be exactly a single antecedent, $A$.
Because the second premise already contains the antecedent $B$, the contexts $\octx'_L$ and $\octx'_R$ must also be empty.
After these revisions, the rule contains only well-formed singleton sequents and is a candidate for inclusion in the singleton sequent calculus.

% With these revisions, all sequents of the resulting rule are well-formed singleton sequents, 


% Because the second premise already contains the antecedent $B$, the contexts $\octx'_L$ and $\octx'_R$ must be empty if the premise is to satisfy the single-antecedent restriction.
% % Under this constraint, the cut principle becomes:
% % \begin{equation*}
% %   \infer[\jrule{CUT}^B]{\oseq{\octx'_L \oc \octx \oc \octx'_R |- C}}{
% %     \oseq{\octx |- B} & \oseq{\octx'_L \oc B \oc \octx'_R |- C}}
% %   \rightsquigarrow
% %   \infer{\oseq{\octx |- C}}{
% %     \oseq{\octx |- B} & \oseq{B |- C}}
% % \end{equation*}
% We then replace the finitary context $\octx$ with a single antecedent, $A$, so that the rule's first premise and conclusion also satisfy the [characteristic] restriction.
% All sequents of the resulting rule are well-formed, making it a candidate for inclusion in the singleton sequent calculus.
% % \begin{equation*}
% %   \begin{array}{@{}ccl@{}}
% %     \text{\scshape ordered logic} && \text{\scshape singleton logic}
% %     \\[3\jot]
% %     \infer[\jrule{CUT}^B]{\oseq{\octx'_L \oc \octx \oc \octx'_R |- C}}{
% %       \oseq{\octx |- B} & \oseq{\octx'_L \oc B \oc \octx'_R |- C}}
% %   % \rightsquigarrow
% %   % \infer{\oseq{\octx |- C}}{
% %   %   \oseq{\octx |- B} & \oseq{B |- C}}
% %     & \rightsquigarrow &
% %     \infer[\jrule{CUT}^B]{\slseq{A |- C}}{
% %       \slseq{A |- B} & \slseq{B |- C}}
% %   \end{array}
% % \end{equation*}
% % We could equally well justify this cut rule for singleton logic by first principles, as it expresses the composition of two well-formed sequents\alertinline{proofs?} in singleton logic.

We could equally well justify this new cut rule by first principles, as it expresses the composition of two well-formed singleton proofs.
But the above method of considering the constraints imposed by the single-antecedent restriction is a straightforward, mechanical way ahead for the other inference rules.
For example, singleton sequent calculus rules for additive disjunction may also be constructed in this way~\parencref[see]{fig:singleton-logic:seq-calc:derive-plus}.
%
\begin{figure*}[tbp]
  % \captionsetup{captionskip=0pt,farskip=0pt,nearskip=0pt}
  \vspace*{-\abovecaptionskip}
  
  $\begin{array}{@{}l@{}ccc@{}}
    & \text{\itshape Ordered sequent calculus} && \text{\itshape Singleton sequent calculus}
    \\[1ex]
%    \subfloat[\label{fig:singleton-logic:seq-calc:derive-cut}]{\quad}
    % &
    % \infer[\mathrlap{\jrule{CUT}^B}]{\oseq{\octx'_L \oc \octx \oc \octx'_R |- C}}{
    %   \oseq{\octx |- B} & \oseq{\octx'_L \oc B \oc \octx'_R |- C}}
    % &
    % \mathrel{\phantom{\rrule{\plus}_2}\mathord{\rightsquigarrow}}
    % &
    % \infer[\mathrlap{\jrule{CUT}^B}]{\slseq{A |- C}}{
    %   \slseq{A |- B} & \slseq{B |- C}}      
    % \\
%    \subfloat[\label{fig:singleton-logic:seq-calc:derive-plus}]{\quad}
    &\!
    \begin{gathered}
      \infer[\rrule{\plus}_1]{\oseq{\octx |- B_1 \plus B_2}}{
        \oseq{\octx |- B_1}}
      \quad
      \infer[\mathrlap{\rrule{\plus}_2}]{\oseq{\octx |- B_1 \plus B_2}}{
        \oseq{\octx |- B_2}}
      \\[\baselineskip]
      \infer[\mathrlap{\lrule{\plus}}]{\oseq{\octx'_L \oc (B_1 \plus B_2) \oc \octx'_R |- C}}{
        \oseq{\octx'_L \oc B_1 \oc \octx'_R |- C} &
        \oseq{\octx'_L \oc B_2 \oc \octx'_R |- C}}
    \end{gathered}
    &
    \mathrel{\phantom{\rrule{\plus}_2}\mathord{\rightsquigarrow}}
    &\!
    \begin{gathered}
      \infer[\rrule{\plus}_1]{\slseq{A |- B_1 \plus B_2}}{
        \slseq{A |- B_1}}
      \quad
      \infer[\mathrlap{\rrule{\plus}_2}]{\slseq{A |- B_1 \plus B_2}}{
        \slseq{A |- B_2}}
      \\[\baselineskip]
      \infer[\mathrlap{\lrule{\plus}}]{\slseq{B_1 \plus B_2 |- C}}{
        \slseq{B_1 |- C} & \slseq{B_2 |- C}}
    \end{gathered}
  \end{array}$
  \caption{Deriving the singleton sequent calculus rules for additive disjunction from the corresponding ordered sequent calculus rules}\label{fig:singleton-logic:seq-calc:derive-plus}
\end{figure*}
%
Rules for the other additive connectives ($\with$, $\top$, and $\zero$) can be constructed, too, but we will momentarily postpone displaying them.
% Similarly, the other additive connectives ($\with$, $\top$, and $\zero$) can be given singleton sequent calculus rules.

\newthought{%
However, not all} ordered logical connectives fare as well under the single-antecedent restriction as the additive connectives do.
In particular, the multiplicative connectives do not have analogues in singleton logic.
% [, precisely because their multiplicative nature involves splitting antecedents among several premises and, in other rules, extending the context with additional antecedents.]
% 
\begin{marginfigure}[7\baselineskip]
  \normalsize
  \begin{gather*}
    \infer[\rrule{\limp}]{\oseq{\octx |- B_1 \limp B_2}}{
      \oseq{B_1 \oc \octx |- B_2}}
    \\
    \mathord{\downsquigarrow}\hspace{5pt}\hphantom{\rrule{\limp}}
    \\
    \infer[\rrule{\limp}\mathrlap{?}]{\slseq{A |- B_1 \limp B_2}}{
      \slseq{B_1 \oc A |- B_2}}
    \\
    \mathord{\downsquigarrow}\hspace{5pt}\hphantom{\rrule{\limp}}
    \\
    \infer[\rrule{\limp}\mathrlap{?}]{\slseq{A |- B_1 \limp B_2}}{
      \slseq{B_1 \fuse A |- B_2}}
  \end{gather*}
  \caption{A failed attempt at constructing a right rule for left-handed implication}\label{fig:singleton-logic:seq-calc:derive-limp}
\end{marginfigure}%
%
Consider, for example, left-handed implication and its right rule (see neighboring \lcnamecref{fig:singleton-logic:seq-calc:derive-limp}).
% \begin{equation*}
%   \infer[\rrule{\limp}]{\oseq{\octx |- B_1 \limp B_2}}{
%     \oseq{B_1 \oc \octx |- B_2}}
% \end{equation*}
The finitary context $\octx$ must be replaced with a single antecedent, $A$, if the rule's conclusion is to be a well-formed singleton sequent.
% \begin{equation*}
%   \infer[\rrule{\limp}]{\oseq{\octx |- B_1 \limp B_2}}{
%     \oseq{B_1 \oc \octx |- B_2}}
%   \rightsquigarrow
%   \infer[\rrule{\limp}?]{\oseq{A |- B_1 \limp B_2}}{
%     \oseq{B_1 \oc A |- B_2}}
% \end{equation*}
Now the revised rule's conclusion is well-formed, but its premise is not.

From a category-theoretic perspective, it would be quite natural to rewrite the premise using ordered conjunction so that the two antecedents are packaged together as one.
% \begin{equation*}
%   \infer[\rrule{\limp}?]{\slseq{A |- B_1 \limp B_2}}{
%     \slseq{B_1 \fuse A |- B_2}}
% \end{equation*}
% 
% Viewed through a category-theoretic lens, this rule is quite innocuous, even natural.
However, from a proof-theoretic perspective, this rule is not suitable -- with this rule, the meaning of left-handed implication depends on the meaning of another connective, namely multiplicative conjunction.
% From a proof-theoretic perspective, however, the meaning of a logical connective should be independent from other connectives, and this rule creates an objectionable dependence of left-handed implication upon ordered conjunction.
As a practical consequence, the subformula property and related cut elimination theorem would fail to hold if the singleton sequent calculus adopted this rule.

In attempting to construct singleton sequent calculus rules for left-handed implication, the fundamental problem is that the $\rrule{\limp}$ rule introduces an additional antecedent to a context that is, and must remain, a singleton.
Changing the size of the context by introducing, or sometimes removing, antecedents is an essential characteristic of multiplicative connectives, and so the multiplicative connectives, by their very nature, cannot appear in singleton logic.

% The left rule for left-handed implication is equally problematic:
% \begin{equation*}
%   \infer[\lrule{\limp}]{\oseq{\octx_L \oc \octx \oc (B_1 \limp B_2) \oc  \octx_R |- C}}{
%     \oseq{\octx |- B_1} & \oseq{\octx_L \oc B_2 \oc \octx_R |- C}}
%   \rightsquigarrow
%   \infer[\lrule{\limp}?]{\slseq{A \oc (B_1 \limp B_2) |- C}}{
%     \slseq{A |- B_1} & \slseq{B_2 |- C}}
% \end{equation*}

% Attempting to give singleton calculus rules for the other multiplicative connectives fails similarly.







% \begin{marginfigure}
%   \normalsize
%   \begin{inferences}
%     \infer[\lrule{\limp}?]{\slseq{A \oc (B_1 \limp B_2) |- C}}{
%       \slseq{A |- B_1} & \slseq{B_2 |- C}}
%   \end{inferences}
%   \caption{A left rule for left-handed implication is equally problematic.}
% \end{marginfigure}

% \begin{inferences}
%   \infer[\rrule{\fuse}?]{\slseq{A_1 \fuse A_2 |- B_1 \fuse B_2}}{
%     \slseq{A_1 |- B_1} & \slseq{A_2 |- B_2}}
%   \and
%   \infer[\lrule{\fuse}?]{\slseq{B_1 \fuse B_2 |- C}}{
%     \slseq{B_1 \fuse B_2 |- C}}
% \end{inferences}

% \subsection{A sequent calculus for propositional singleton logic}\label{sec:singleton-logic:seq-calc:full}

\newthought{%
\Cref{fig:singleton-logic:seq-calc}} presents the complete set of rules for propositional singleton logic's sequent calculus.
%
%
% The well-formed propositions are exactly the additive propositions of ordered logic, and the complete set of inference rules has been derived from those of the ordered sequent calculus by 

% There is an important observation to be made here.
Although the propositions of singleton logic are exactly the additive propositions of ordered logic, singleton logic is \emph{not} the additive fragment of ordered logic.
For instance, the sequent $\oseq{A \oc B |- \top}$ is provable in the additive fragment of ordered logic, but it
% $\slseq{A \oc B |- \top}$
is not even a well-formed sequent in the singleton sequent calculus, for the simple reason that it violates the single-antecedent restriction.
%
\begin{figure}[tbp]
  \vspace*{\dimexpr-\abovedisplayskip-\abovecaptionskip\relax}
  \begin{syntax*}
    Propositions &
      A,B,C & a \mid A \plus B \mid \zero \mid A \with B \mid \top
  \end{syntax*}
  \begin{inferences}
    \infer[\jrule{CUT}^B]{\slseq{A |- C}}{
      \slseq{A |- B} & \slseq{B |- C}}
    \and
    \infer[\jrule{ID}^A]{\slseq{A |- A}}{}
    \\
    \infer[\rrule{\plus}_1]{\slseq{A |- B_1 \plus B_2}}{
      \slseq{A |- B_1}}
    \and
    \infer[\rrule{\plus}_2]{\slseq{A |- B_1 \plus B_2}}{
      \slseq{A |- B_2}}
    \and
    \infer[\lrule{\plus}]{\slseq{B_1 \plus B_2 |- C}}{
      \slseq{B_1 |- C} & \slseq{B_2 |- C}}
    \\
    \text{(no $\rrule{\zero}$ rule)}
    \and
    \infer[\lrule{\zero}]{\slseq{\zero |- C}}{}
    \\
    \infer[\rrule{\with}]{\slseq{A |- B_1 \with B_2}}{
      \slseq{A |- B_1} & \slseq{A |- B_2}}
    \and
    \infer[\lrule{\with}_1]{\slseq{B_1 \with B_2 |- C}}{
      \slseq{B_1 |- C}}
    \and
    \infer[\lrule{\with}_2]{\slseq{B_1 \with B_2 |- C}}{
      \slseq{B_2 |- C}}
    \\
    \infer[\rrule{\top}]{\slseq{A |- \top}}{}
    \and
    \text{(no $\lrule{\top}$ rule)}
  \end{inferences}
  \vspace*{-\belowdisplayskip}
  \caption{A sequent calculus for propositional singleton logic\label{fig:singleton-logic:seq-calc}}
\end{figure}

That said, singleton logic only differs from the additive fragment of ordered logic in its treatment of $\zero$ and $\top$ -- the $\zero$,$\top$-free fragment of singleton logic coincides exactly with the $\zero$,$\top$-free, additive fragment (that is, the $\plus$,$\with$-fragment) of ordered logic.
% 
% That said, the two logics are related once $\zero$ and $\top$ are excluded -- the $\zero,\top$-free fragment of singleton logic is exactly the $\zero,\top$-free additive fragment of ordered logic.
% Stated differently, the $\plus,\with$-fragment of singleton sequent calculus coincides with the $\plus,
% 
% That said, if $\top$ and $\zero$ are removed, the remainder of singleton logic is indeed exactly the $\plus,\with$-fragment of ordered logic.
A simple structural induction proves this:
\begin{theorem}\label{thm:singleton-logic:additive-fragment}
  If\/ $\oseq{\octx |- B}$ in the $\plus$,$\with$-fragment of the ordered sequent calculus, then there exists a proposition $A$ such that $\octx = A$ and $\slseq{A |- B}$ in the $\plus$,$\with$-frag\-ment of the singleton sequent calculus.
\end{theorem}

In substructural logics and systems built upon them, the logical constants $\zero$ and $\top$ are often problematic because they indiscriminately consume any and all resources placed in front of them.\autocites{Cervesato+:TCS00}{Schack-Nielsen+Schuermann:IJCAR08}
Interestingly, because the single-antecedent restriction exactly constrains the resources that those logical constants may consume, working in singleton logic is one possible way to sanitize $\zero$ and $\top$.

% Second, it is worth reiterating that singleton logic, peculiarly, has no form of implication.
% % as mentioned previously, singleton logic contains no multipicative connectives and, most peculiarly, no form of implication.
% It is odd to contemplate that a logic without an implication connective to internalize the logic's underlying hypothetical judgment could possibly be well-defined.
% But as the metatheoretic results of the following \lcnamecref{sec:singleton-logic:seq-calc:metatheory} verify, singleton sequent caluculus is indeed well-defined, resting on the solid foundation of a verificationist meaning-theory.


\newthought{Symmetry was} one of the motivations behind examining the single-ante\-cedent restriction and the singleton logic that results from it.
With a sequent calculus for singleton logic, we can now make that symmetry precise.
For instance, the $\rrule{\with}$ rule can be exactly obtained from the $\lrule{\plus}$ rule -- and \emph{vice versa} -- by reversing the turnstiles and replacing the $\plus$ connective with $\with$:
\begin{equation*}
  \infer[\lrule{\plus}]{\slseq{A_1 \plus A_2 |- B}}{
    \slseq{A_1 |- B} & \slseq{A_2 |- B}}
  \quad\raisebox{1.25ex}{$\leftrightsquigarrow$}\quad
  \infer[\rrule{\with}]{A_1 \with A_2 \dashv B}{
    A_1 \dashv B & A_2 \dashv B}
\end{equation*}
Applying this transformation to the other rules results in similar symmetric pairs.
The $\jrule{CUT}$ even maps to itself.
\begin{equation*}
  \infer[\jrule{CUT}\smash{^B}]{\slseq{A |- C}}{
    \slseq{A |- B} & \slseq{B |- C}}
  \quad\raisebox{1.25ex}{$\leftrightsquigarrow$}\quad
  \infer[\jrule{CUT}\smash{^B}]{A \dashv C}{
    A \dashv B & B \dashv C}
\end{equation*}

To make this automorphism formal, we define an involution, $\sym*{}$, on propositions (see the adjacent \lcnamecref{fig:singleton-logic:involution})%
\begin{marginfigure}[-4\baselineskip]
  \begin{gather*}
    \sym*{a} = \sym{a} \qquad \sym*{\sym{a}} = a
    \\
    \!\begin{aligned}
      \sym*{A \plus B} &= \sym{A} \with \sym{B} \\
      \sym*{\zero} &= \top \\
      \sym*{A \with B} &= \sym{A} \plus \sym{B} \\
      \sym*{\top} &= \zero
    \end{aligned}
  \end{gather*}
  \caption{An involution on propositions}\label{fig:singleton-logic:involution}
\end{marginfigure}%
, presuming that the involution of an atom is again an atom.
With this involution, it is relatively straightforward to state and prove symmetry:
\begin{theorem}\label{thm:singleton-logic:symmetry}
  $\slseq{A |- B}$ if and only if $\slseq{\sym{B} |- \sym{A}}$.
\end{theorem}
\begin{proof}
  The left-to-right direction can be proved by structural induction on the derivation of $\slseq{A |- B}$.
  The converse follows immediately, because $\sym*{}$ is an involution.
\end{proof}

Notice that the $\sym*{}$ involution is the additive fragment of the involution commonly used in one-sided sequent calculi for classical linear logic.
In this sense, singleton logic exhibits the same symmetries as classical logic, but in an intuitionistic setting.

\subsection{Metatheory: Cut elimination and identity expansion}\label{sec:singleton-logic:seq-calc:metatheory}

The rules shown in \cref{fig:singleton-logic:seq-calc} certainly have the appearance of sequent calculus rules, but do they truly constitute a well-defined sequent calculus?
Most peculiarly, the singleton sequent calculus has no implication connective that internalizes the underlying hypothetical judgment.
Can such a calculus possibly be well-defined?

Because it coincides exactly with a fragment of the ordered sequent calculus~\parencref{thm:singleton-logic:additive-fragment}, the singleton sequent calculus is indeed well-defined.
However, for our subsequent development, it will prove useful to examine the singleton sequent calculus's metatheory, especially cut elimination, natively.

\newthought{In the tradition} of \citeauthor{Gentzen:MZ35}, \citeauthor{Dummett:WJ76}, and \citeauthor{Martin-Lof:Siena83}\autocites{Gentzen:MZ35}{Dummett:WJ76}{Martin-Lof:Siena83}, a sequent calculus is well-defined if it rests on the solid foundation of a verificationist meaning-explanation.
That is, the meaning of each logical connective must be given entirely by its right (and left) inference rules, and those rules must exist in harmony with the left rules.

A \emph{verification}, then, is a proof that relies only on the right and left inference rules and the $\jrule{ID}^{p}$ rule for propositional variables $p$ -- stated differently, verifications may not contain instances of the $\jrule{CUT}$ or general $\jrule{ID}^A$ rules.

If every proof has a corresponding verification, then we can be sure that neither the $\jrule{CUT}$ nor $\jrule{ID}$ rules play any role in defining the logical connectives.

%
% In the tradition of \citeauthor{Gentzen:??}, \citeauthor{Dummett:??}, and \citeauthor{Martin-Lof:??}\autocites{Gentzen:??}{Dummett:??}{Martin-Lof:??}, a sequent calculus is well-defined if it rests on the solid foundation of a verificationist meaning-theory.
% That is, the meaning of each logical connective must be given entirely by its right [and left inference] rules, and those rules must exist in harmony [with the left rules].%
% \alertnote{Right rules only, because it is verificationist?}
%
% In \citeauthor{Martin-Lof:??}'s words, the meaning of a logical connective must be given by what counts as a verification of it.
% A \emph{verification}, then, is a proof that relies only on the right and left inference rules and the $\jrule{ID}^{\alpha}$ rule for propositional variables $\alpha$ -- stated differently, verifications may not contain instances of the $\jrule{CUT}$ or general $\jrule{ID}^A$ rules.
%
%
For this program to succeed, we need to be sure that for every proof there is a corresponding verification -- 
%
in this sense, the usual cut elimination metatheorem states a weak normalization result.
%
\begin{restatable*}[
  name=Cut elimination,
  label=thm:singleton-logic:seq-calc:cut-elimination
]{theorem}{thmsingletoncutelim}
  If a proof of $\slseq{A |- C}$ exists, then there exists a \emph{cut-free} proof of $\slseq{A |- C}$.
\end{restatable*}
%
As usual, the cut elimination \lcnamecref{thm:singleton-logic:seq-calc:cut-elimination} may be proved by a straightforward induction on the structure of the given proof, provided that a cut principle for cut-free proofs is admissible:
% 
\begin{restatable*}[
  name=Admissibility of cut,
  label=lem:singleton-logic:cut-admissibility
]{lemma}{lemsingletoncutadmissible}
  If cut-free proofs of $\slseq{A |- B}$ and $\slseq{B |- C}$ exist, then there exists a \emph{cut-free} proof of $\slseq{A |- C}$.
\end{restatable*}

\newthought{Before proceeding} to this \lcnamecref{lem:singleton-logic:cut-admissibility}'s proof, it is worth emphasizing a subtle distinction between the singleton sequent calculus's primitive $\jrule{CUT}$ rule and the admissible cut principle that this \lcnamecref{lem:singleton-logic:cut-admissibility} establishes.

To be completely formal, we could treat cut-freeness as an extrinsic, Curry-style property of proofs%
\footnote[][-.5\baselineskip]{Contrast this with a separate, intrinsically cut-free sequent calculus in the style of Church~\parencite{Pfenning:Andrews08}.}
and indicate cut-freeness by decorating the turnstile, so that $\cfslseq{A |- C}$ denotes a cut-free proof of $\slseq{A |- C}$.
The admissible cut principle stated in \cref{lem:singleton-logic:cut-admissibility} could then be expressed as the rule
\begin{equation*}
  \infer-[\jrule{A-CUT}\smash{^B}]{\cfslseq{A |- C}}{
    \cfslseq{A |- B} & \cfslseq{B |- C}}
  ,
\end{equation*}
with the dotted line indicating that it is an admissible, not primitive, rule.
Writing it in this way emphasizes that proving \cref{lem:singleton-logic:cut-admissibility} amounts to defining a meta-level function that takes cut-free proofs of $\slseq{A |- B}$ and $\slseq{B |- C}$ and produces a \emph{cut-free} proof of $\slseq{A |- C}$.
% Moreover, the decorated turnstile makes it clear that all three proofs are cut-free by construction.
% 
Contrast this with the primitive $\jrule{CUT}$ rule of the singleton sequent calculus%
\marginnote[-.5\baselineskip]{
  \normalsize
  $\infer[\jrule{CUT}\smash{^B}]{\slseq{A |- C}}{
     \slseq{A |- B} & \slseq{B |- C}}$%
}%
, which forms a (cut-full) proof of $\slseq{A |- C}$ from (potentially cut-full) proofs of $\slseq{A |- B}$ and $\slseq{B |- C}$.
% \footnote{Which are not necessarily cut-free.}.

From here on, however, we won't bother to be quite so pedantic, instead often omitting the turnstile decoration on cut-free proofs with the understanding that the admissible $\jrule{A-CUT}$ rule may only be applied to cut-free proofs.

%  writing the admissible cut principle as
% \begin{equation*}
%   \infer-[\jrule{A-CUT}^B]{\slseq{A |- C}}{
%     \slseq{A |- B} & \slseq{B |- C}}
% \end{equation*}
% with the understanding that this admissible rule may only be applied to cut-free proofs.


\newthought{With that clarification} out of the way, we are finally ready to prove the admissibility of cut \lcnamecref{lem:singleton-logic:cut-admissibility}.
%
\lemsingletoncutadmissible
%
\begin{proof}
  Just as in the proof of admissibility of cut for the ordered sequent calculus~\parencref{lem:ordered-logic:cut-admissibility}, we use a standard lexicographic structural induction, first on the structure of the cut formula, and then on the structures of the given proofs.

  As usual, the cases can be classified into three categories: principal cases, identity cases, and commutative cases.
  % We show a few sample cases.
  \begin{description}[listparindent=\parindent, parsep=0pt]
  \item[Principal cases]
    As usual, the principal cases pair a proof ending in a right rule together with a proof ending in a corresponding left rule.
    % , so that the last inference of each proof introduces the cut formula.
    % For example, one of the principal cases pairs a proof ending in the $\rrule{\plus}_1$ rule with one ending in the $\lrule{\plus}$ rule; it is resolved as follows.
    One such principal case is:
    \begin{gather*}
      \infer-[\jrule{A-CUT}\smash{^{B_1 \plus B_2}}]{\slseq{A |- C}}{
        \infer[\rrule{\plus}_1]{\slseq{A |- B_1 \plus B_2}}{
          \deduce{\slseq{A |- B_1}}{\DD_1}} &
        \infer[\lrule{\plus}]{\slseq{B_1 \plus B_2 |- C}}{
          \deduce{\slseq{B_1 |- C}}{\EE_1} &
          \deduce{\slseq{B_2 |- C}}{\EE_2}}}
      % 
      \\=\hphantom{\jrule{A-CUT}^{B_1}}\\
      % 
      \infer-[\jrule{A-CUT}\smash{^{B_1}}]{\slseq{A |- C}}{
        \deduce{\slseq{A |- B_1}}{\DD_1} &
        \deduce{\slseq{B_1 |- C}}{\EE_1}}
      \qquad
    \end{gather*}
    Notice that the interaction between proofs here is synchronous -- the case is resolved by appealing to the inductive hypothesis at a smaller cut formula but also smaller proofs.
  
  \item[Identity cases]
    In the identity cases, one of the proofs is the $\jrule{ID}$ rule alone.
    For example:
    % One of the identity cases pairs the $\jrule{ID}$ rule with a proof of $\slseq{A |- C}$:
    \begin{equation*}
      \infer-[\jrule{A-CUT}\smash{^A}]{\slseq{A |- C}}{
        \infer[\jrule{ID}\smash{^A}]{\slseq{A |- A}}{} &
        \deduce{\slseq{A |- C}}{\EE}}
      % 
      \quad=\quad
      % 
      \deduce{\slseq{A |- C}}{\EE}
      \,.
    \end{equation*}
    % That $\jrule{CUT}$ and $\jrule{ID}$ are inverses here is consistent with the idea that the cut and identity principles are dual.
    
  \item[Commutative cases]
    As in the proof of ordered logic's admissible cut principle~\parencref{lem:ordered-logic:cut-admissibility}, the commutative cases are those in which one of the proofs ends by introducing a side formula.
    % the last inference in one of the proofs introduces a formula other than the cut formula;
    % the cases are subcategorized as left- or right-commutative according to the proof involved.
    % In both scenarios, the cut and involved inference rule commute, with an appeal to the inductive hypothesis at the same cut formula but smaller proofs.

    % % The left commutative cases pair a proof of $\slseq{A |- B}$ ending in a left rule together with a proof of $\slseq{B |- C}$.
    % For example, one left-commutative case
    % % of the left commutative cases
    % pairs a proof of $\slseq{A_1 \plus A_2 |- B}$ ending in the $\lrule{\plus}$ rule together with a proof of $\slseq{B |- C}$:
    % \begin{gather*}
    %   \infer-[\mathrlap{\jrule{A-CUT}\smash{^B}}]{\slseq{A_1 \plus A_2 |- C}}{
    %     \infer[\lrule{\plus}]{\slseq{A_1 \plus A_2 |- B}}{
    %       \deduce{\slseq{A_1 |- B}}{\DD_1} &
    %       \deduce{\slseq{A_2 |- B}}{\DD_2}} &
    %     \deduce{\slseq{B |- C}}{\EE}}
    %   % 
    %   \\=\\
    %   % 
    %   \infer[\lrule{\plus}]{\slseq{A_1 \plus A_2 |- C}}{
    %     \infer-[\jrule{A-CUT}\smash{^B}]{\slseq{A_1 |- C}}{
    %       \deduce{\slseq{A_1 |- B}}{\DD_1} &
    %       \deduce{\slseq{B |- C}}{\EE}} &
    %     \infer-[\mathrlap{\jrule{A-CUT}\smash{^B}}]{\slseq{A_2 |- C}}{
    %       \deduce{\slseq{A_2 |- B}}{\DD_2} &
    %       \deduce{\slseq{B |- C}}{\EE}}}
    % \end{gather*}

  % \item[Right commutative cases]
    % The right commutative cases pair a proof of $\slseq{A |- B}$ together with a proof of $\slseq{B |- C}$ ending in a right rule.
    % For example, one such case
    As an example, one right-commutative case
    % of the right commutative cases
    pairs a proof of $\slseq{A |- B}$ with a proof of $\slseq{B |- C_1 \plus C_2}$ ending in the $\rrule{\plus}_1$ rule:
    % ; it is resolved as follows.
    \begin{equation*}
      \infer-[\jrule{A-CUT}\smash{^B}]{\slseq{A |- C_1 \plus C_2}}{
        \deduce{\slseq{A |- B}}{\DD} &
        \infer[\rrule{\plus}_1]{\slseq{B |- C_1 \plus C_2}}{
          \deduce{\slseq{B |- C_1}}{\EE_1}}}
      % 
      \quad=\quad
      % 
      \infer[\rrule{\plus}_1]{\slseq{A |- C_1 \plus C_2}}{
        \infer-[\jrule{A-CUT}\smash{^B}]{\slseq{A |- C_1}}{
          \deduce{\slseq{A |- B}}{\DD} &
          \deduce{\slseq{B |- C_1}}{\EE_1}}}
    \end{equation*}
    Unlike in ordered logic, there can be no right-commutative cases involving left rules because the cut formula is the only antecedent in the sequent $\slseq{B |- C}$.
    % In this way, the proof of admissibility of cut for the singleton sequent calculus is more symmetric than that of the ordered sequent calculus.
    In this way, the symmetry of singleton sequents is manifest even in proving the admissibility of cut.
    \qedhere
  \end{description}
\end{proof}


\newthought{With the admissibility} of cut established, we can finally prove cut elimination for the singleton sequent calculus.
%
\thmsingletoncutelim
%
\begin{proof}
  By structural induction on the proof of $\slseq{A |- C}$, appealing to the admissibility of cut~\parencref{lem:singleton-logic:cut-admissibility} when encountering a $\jrule{CUT}$ rule.

  If we display the inductive hypothesis as an admissible rule, then the crucial case in the proof of cut elimination is resolved as follows.
  \begin{equation*}
    \infer-[\jrule{CE}]{\cfslseq{A |- C}}{
      \infer[\jrule{CUT}\smash{^B}]{\slseq{A |- C}}{
        \deduce{\slseq{A |- B}}{\DD_1} & \deduce{\slseq{B |- C}}{\DD_2}}}
    \quad=\quad
    \infer-[\jrule{A-CUT}\smash{^B}]{\cfslseq{A |- C}}{
      \infer-[\jrule{CE}]{\cfslseq{A |- B}}{
        \deduce{\slseq{A |- B}}{\DD_1}} &
      \infer-[\jrule{CE}]{\cfslseq{B |- C}}{
        \deduce{\slseq{B |- C}}{\DD_2}}}
  \end{equation*}
  All other cases are handled compositionally.

  This cut elimination proof amounts to defining a meta-level function for normalizing proofs to cut-free form.
\end{proof}


In addition to cut elimination, we can also prove identity elimination.
An identity-long proof is one in which all applications of the $\jrule{ID}$ occur at propositional variables.
Identity elimination -- a slight misnomer -- transforms a proof into an identity-long proof of the same sequent by replacing instances of the $\jrule{ID}^A$ rule with an identity-long proof of $\slseq{A |- A}$.%
\footnote{Identity elimination is a slight misnomer because instances of the $\jrule{ID}$ rule at propositional variables will remain.}

\begin{lemma}[Admissibility of identity]
  For all propositions $A$, an identity-long proof of $\slseq{A |- A}$ exists.
  Moreover, this proof is cut-free.
\end{lemma}
\begin{proof}
  As usual, by induction on the structure of the proposition $A$.
\end{proof}

\begin{theorem}[Identity elimination]
  If a proof of $\slseq{A |- C}$ exists, then there exists an \emph{identity-long} proof of $\slseq{A |- C}$.
  Moreover, if the given proof is cut-free, so is the identity-long proof.
\end{theorem}
\begin{proof}
  As usual, by structural induction on the proof of $\slseq{A |- C}$.
\end{proof}


\section{A semi-axiomatic sequent calculus for singleton logic}\label{sec:singleton-logic:hilbert}\label{sec:singleton-logic:sax}

Sequent calculi are not the only way to present logics, so
% natural deduction calculi and Hilbert systems are also commonly used, for instance.
in this \lcnamecref{sec:singleton-logic:hilbert} we also consider a Hilbert-style axiomatization of singleton logic, which can be viewed as a sequent calculus variant that we dub the \emph{semi-axiomatic sequent calculus}\autocite{DeYoung+:FSCD20}.
Our interest in a semi-axiomatic sequent calculus for singleton logic is not taxonomic, however.
Rather, over the course of the next \lcnamecref{ch:process-chains} and a half, we shall see that normalization of semi-axiomatic sequent proofs serves as the basis of a Curry--Howard isomorphism with chains of asynchronously communicating processes.

% \subsection{A Hilbert-style axiomatization of linear logic}

\newthought{In a seq\-uent calculus}, the meaning of a connective is given by its right and left inference rules.
Hilbert-style axiomatizations, on the other hand, strive to use as few rules of inference as possible, with the meaning of a connective instead given by a small collection of axiom schemas.

The term \enquote*{axiom schema} is often interpreted narrowly to mean only categorical judgments like $\vdash A \imp B \imp A \land B$, not hypothetical judgments like $\ctx , A , B \vdash A \land B$ adopted as zero-premise rules of inference.
Consequently, Hilbert-style axiomatizations usually rely heavily on implication and a \emph{modus ponens} rule
%
\begin{marginfigure}
  \begin{equation*}
    \infer[\jrule{MP}]{\vdash B}{
      \vdash A \imp B & \vdash A}
  \end{equation*}
  \caption{\emph{Modus ponens} for a Hilbert-style axiomatization of intuitionistic logic}
\end{marginfigure}%
%
to effect the meanings of the logical connectives.

However, as explained in \cref{sec:singleton-logic:seq-calc}, singleton logic does not enjoy the luxury of an implication connective.
So a true Hilbert-style axiomatization that relies on a \emph{modus ponens} rule as its only inference rule is not possible.
Instead, for a semi-axiomatic sequent calculus of singleton logic, 
% So in a Hilbert-style axiomatization of singleton logic,
we must content ourselves with a broad interpretation of the term \enquote*{axiom schema} that encompasses zero-premise rules that derive hypothetical judgments.

\newthought{To construct} a semi-axiomatic sequent calculus for singleton logic, we will ask, in turn, whether each sequent calculus rule can be reduced to a schema.

First, consider the judgmental rules, $\jrule{ID}$ and $\jrule{CUT}$, for the identity and cut principles (see neighboring display).%
\marginnote[-\baselineskip]{%
  \begin{inferences}
    \infer[\mathrlap{\jrule{ID}^A}]{\slseq{A |- A}}{}
    \hphantom{\jrule{CUT}^B}
    \\
    \infer[\mathrlap{\jrule{CUT}^B}]{\slseq{A |- C}}{
      \slseq{A |- B} & \slseq{B |- C}}
    \hphantom{\jrule{CUT}^B}
  \end{inferences}
}
With zero premises, the $\jrule{ID}$ rule itself is already an axiom schema and can be adopted directly in singleton logic's semi-axiomatic sequent calculus.

The $\jrule{CUT}$ rule is not so accommodating.
As a rule for composing proofs, the $\jrule{CUT}$ rule serves a similar purpose to the traditional \emph{modus ponens} rule.
Just as \emph{modus ponens} cannot be reduced to an axiom schema, so must $\jrule{CUT}$ remain a rule of inference.
Moreover, because singleton logic has no implication connective, the rule's hypothetical judgments cannot even be simplified to categorical judgments.
Therefore, the $\jrule{CUT}$ rule is adopted wholesale in the semi-axiomatic sequent calculus.

% Because singleton logic does not have an implication connective to internalize the hypothetical judgment, there is no obvious way to turn this rule into an axiom.
% Both rules are therefore adopted wholesale in the Hilbert-style axiomatization.

Next, consider the sequent calculus's $\rrule{\plus}_1$ inference rule.
% ; how much can we push this rule toward an axiom?
Using the $\jrule{ID}$ axiom schema, we can obtain a zero-premise derived rule from $\rrule{\plus}_1$:
\begin{equation*}
  \infer[\rrule{\plus}_1]{\slseq{A_1 |- A_1 \plus A_2}}{
    \infer[\jrule{ID}]{\slseq{A_1 |- A_1}}{}}
  %
  \quad\leftrightsquigarrow\quad
  %
  \infer[\rrule{\plus}'_1]{\slseq{A_1 |- A_1 \plus A_2}}{}
\end{equation*}
Moreover, by combining this new $\rrule{\plus}'_1$ axiom schema with $\jrule{CUT}$, we can recover the original $\rrule{\plus}_1$ rule as a derived rule:
\begin{equation*}
  \infer[\jrule{CUT}]{\slseq{A |- B_1 \plus B_2}}{
    \slseq{A |- B_1} &
    \infer[\rrule{\plus}'_1]{\slseq{B_1 |- B_1 \plus B_2}}{}}
  %
  \quad\leftrightsquigarrow\quad
  %
  \infer[\rrule{\plus}_1]{\slseq{A |- B_1 \plus B_2}}{
    \slseq{A |- B_1}}
\end{equation*}
Together, these two observations suggest that $\rrule{\plus}'_1$ be adopted as an axiom schema in the semi-axiomatic sequent calculus for singleton logic.
A symmetric $\rrule{\plus}'_2$ axiom schema should be adopted, too.

What about the sequent calculus's $\lrule{\plus}$ rule (see neighboring display)?%
\marginnote{%
  \begin{equation*}
    \infer[\lrule{\plus}]{\slseq{B_1 \plus B_2 |- C}}{
      \slseq{B_1 |- C} &
      \slseq{B_2 |- C}}
  \end{equation*}%
}
Can it also be reduced to an axiom schema?
Once again, singleton logic's lack of an implication connective prevents us from even simplifying the $\lrule{\plus}$ rule's hypothetical judgments to categorical judgments.
Like $\jrule{CUT}$, the sequent calculus's $\lrule{\plus}$ rule is thus adopted wholesale in singleton logic's semi-axiomatic sequent calculus.
Including the additive $\lrule{\plus}$ rule as a primitive rule of inference is perhaps not unexpected.
It is consistent with Hilbert-style axiomatizations of linear logic\autocite{Avron:TCS88}, which include an adjunction rule -- essentially the linear sequent calculus's $\rrule{\with}$ rule -- to effect the additive behavior that linear implication and its multiplicative \emph{modus ponens} rule cannot.

% means that we will just
% Because singleton logic does not have an implication connective to internalize the hypothetical judgment, there is really no way to turn this rule into an axiom.
% Instead, we will
% carry the $\lrule{\plus}$ sequent calculus rule over to our Hilbert-style axiomatization [of singleton logic] as a primitive rule of inference.

The treatment of additive conjunction is dual to that of $A_1 \plus A_2$:
The sequent calculus's $\rrule{\with}$ rule will be adopted wholesale, and $\lrule{\with}'_1$ and $\lrule{\with}'_2$ axiom schemas will be derived from the sequent calculus's $\lrule{\with}_1$, $\lrule{\with}_2$, and $\jrule{ID}$ rules.
%
\begin{marginfigure}
  \begin{inferences}
    \infer[\rrule{\with}]{\slseq{A |- C_1 \with C_2}}{
      \slseq{A |- C_1} &
      \slseq{A |- C_2}}
    \\
    \infer[\lrule{\with}'_1]{\slseq{C_1 \with C_2 |- C_1}}{}
    \and
    \infer[\lrule{\with}'_2]{\slseq{C_1 \with C_2 |- C_2}}{}
  \end{inferences}
  \caption{Semi-axiomatic sequent calculus rules for additive conjunction from singleton logic}
\end{marginfigure}%
%
And finally, $\zero$ and $\top$ are treated as the nullary analogues of those of the binary $\plus$ and $\with$ connectives, respectively.

\Cref{fig:singleton-logic:hilbert} summarizes this semi-axiomatic sequent calculus for singleton logic.
%
\begin{figure}[tbp]
  \vspace*{\dimexpr-\abovedisplayskip-\abovecaptionskip\relax}
  \begin{syntax*}
    Propositions &
      A,B,C & a \mid A \plus B \mid \zero \mid A \with B \mid \top
  \end{syntax*}
  \begin{inferences}
    \infer[\jrule{CUT}^B]{\slseq{A |- C}}{
      \slseq{A |- B} & \slseq{B |- C}}
    \and
    \infer[\jrule{ID}^A]{\slseq{A |- A}}{}
    \\
    \infer[\rrule{\plus}_1']{\slseq{A_1 |- A_1 \plus A_2}}{}
    \and
    \infer[\rrule{\plus}_2']{\slseq{A_2 |- A_1 \plus A_2}}{}
    \and
    \infer[\lrule{\plus}]{\slseq{A_1 \plus A_2 |- C}}{
      \slseq{A_1 |- C} & \slseq{A_2 |- C}}
    \\
    \text{(no $\rrule{\zero}$ rule)}
    \and
    \infer[\lrule{\zero}]{\slseq{\zero |- C}}{}
    \\
    \infer[\rrule{\with}]{\slseq{A |- C_1 \with C_2}}{
      \slseq{A |- C_1} & \slseq{A |- C_2}}
    \and
    \infer[\lrule{\with}_1']{\slseq{C_1 \with C_2 |- C_1}}{}
    \and
    \infer[\lrule{\with}_2']{\slseq{C_1 \with C_2 |- C_2}}{}
    \\
    \infer[\rrule{\top}]{\slseq{A |- \top}}{}
    \and
    \text{(no $\lrule{\top}$ rule)}
  \end{inferences}
  \vspace{-\belowdisplayskip}
  \caption{A semi-axiomatic for singleton logic}%
  \label{fig:singleton-logic:hilbert}
\end{figure}
%



% \newthought{In a sequent calculus}, the meaning of a connective is given by its right and left inference rules.
% Hilbert-style axiomatizations, on the other hand, strive to use as few rules of inference as possible, with the meaning of a connective instead given by a small collection of axioms.
% % For example, the neighboring display shows a possible axiomatization of intuitionistic conjunction.%
% % \marginnote{%
% %   $\begin{lgathered}
% %     \vdash A \imp B \imp A \land B \\
% %     \vdash (A \land B \imp A \imp C) \imp (A \land B \imp C) \\
% %     \vdash (A \land B \imp B \imp C) \imp (A \land B \imp C)
% %   \end{lgathered}$
% % }

% The term \enquote*{axiom} is often interpreted narrowly to mean only categorical judgments [logical tautologies?] like $\vdash A \imp B \imp A \land B$, not hypothetical judgments like $\ctx , A , B \vdash A \land B$ adopted as zero-premise rules of inference.
% Consequently, Hilbert-style axiomatizations rely heavily on implication and a \emph{modus ponens} rule to effect the maximum ...

% Although categorical axioms are usually preferred, they are equivalent to the hypothetical axioms.
% The deduction theorem, used to bridge ..., captures this equivalence.
% For example, the following deduction theorem for intuitionistic logic, together with weakening, shows that the categorical axiom $\vdash A \imp B \imp A \land B$ is equivalent to the hypothetical axiom $\ctx , A , B \vdash A \land B$.


% As explained in \cref{??}, singleton logic does not have the luxury of an implication connective, so in a Hilbert-style axiomatization, we will have to satisfy ourselves with hypothetical axioms.


% Next, take the sequent calculus's $\rrule{\plus}_1$ inference rule; how much can we push this rule toward an axiom?
% Consider the derived rule obtained from $\rrule{\plus}_1$ and $\jrule{ID}$:
% \begin{equation*}
%   \infer[\rrule{\plus}_1]{\slseq{A_1 |- A_1 \plus A_2}}{
%     \infer[\jrule{ID}]{\slseq{A_1 |- A_1}}{}}
%   %
%   \infer[\rrule{\plus}'_1]{\slseq{A_1 |- A_1 \plus A_2}}{}
% \end{equation*}

% Given a sequent calculus and Hilbert system, we would like to be sure that the sequent- and Hilbert-style meanings of the logical connectives coincide.
% To bridge the gap between the categorical judgment preferred in a Hilbert-style axiomatization and the hypothetical judgment at the heart of a sequent calculus, a hypothetical Hilbert system and accompanying deduction theorem are used.
% The deduction theorem captures the idea that implication embodies a hypothetical judgment as a logical connective.
% For intuitionistic logic, that theorem would be:
% \begin{theorem*}[Deduction theorem]
%   $\ctx , A \vdash B$ if and only if\/ $\ctx \vdash A \imp B$.
% \end{theorem*}


% Although categorical judgments are preferred, a hypothetical Hilbert system 

% Central to this endeavor is often a deduction theorem which shows that implication embodies the logic's underlying hypothetical judgment as a logical connective.
% For example, a Hilbert-style axiomatization of intuitionistic logic would enjoy the following theorem.
% \begin{theorem*}[Deduction theorem]
%   $\ctx , A \vdash B$ if and only if $\ctx \vdash A \imp B$.
% \end{theorem*}
% On the basis of this theorem, 


% As explained in \cref{??}, singleton logic does not have the luxury of an implication connective and consequently no deduction theorem, so the final  in a Hilbert-style axiomatization, we will have to 


% Often, the term \enquote*{axiom} is interpreted narrowly to mean only 
% \begin{equation*}
%   \begin{lgathered}
%     \vdash A \imp B \imp A \land B \\
%     \vdash (A \imp C) \imp (A \land B \imp C) \\
%     \vdash (B \imp C) \imp (A \land B \imp C)
%   \end{lgathered}
% \end{equation*}


% To avoid unnecessarily introducing rules of inference,
% % Most of the time,
% these axioms rely heavily on implication and a \textit{modus ponens} rule to effect ... 

% To avoid unnecessarily introducing rules of inference,
% % Most of the time,
% these axioms rely heavily on implication and a \textit{modus ponens} rule to effect ... 
% For example, in a Hilbert-style axiomatization of intuitionistic ordered logic\autocite{Avron:??}, ordered conjunction is described by the following axioms.
% \begin{equation*}
%   \begin{lgathered}
%     \vdash B \limp (A \limp A \fuse B) \\
%     \vdash (B \limp (A \limp C)) \limp (A \fuse B \limp C)
%   \end{lgathered}
% \end{equation*}
% The axioms are suggestive of the right and left sequent calculus rules for 
% \begin{equation*}
%   \begin{lgathered}
%     \vdash A \lolli B \lolli A \tensor B \\
%     \vdash (A \lolli B \lolli C) \lolli (A \tensor B \lolli C)
%   \end{lgathered}
%   \quad\rightsquigarrow\quad
%   \begin{gathered}
%     \infer{A , B \vdash A \tensor B}{} \\
%     \infer{A \tensor B \vdash C}{
%       A, B \vdash C}
%   \end{gathered}
% \end{equation*}



% % Whereas sequent calculi use many inference rules but few axioms, Hilbert systems shift the balance far in favor of axioms.


% % In a Hilbert system, each logical connective is defined by a collection of axioms.
% % For example, in a Hilbert system for intuitionistic ordered logic, additive disjunction would be defined by three axioms:
% \begin{equation*}
%   \begin{lgathered}
%     \infer[\with]{\vdash A \with B}{
%       \vdash A & \vdash B}
%     \\
%     \vdash \bigl((A \limp B_1) \with (A \limp B_2)\bigr) \limp (A \limp B_1 \with B_2)
%   \end{lgathered}
%   \qquad
%   \begin{lgathered}[b]
%     \vdash A_1 \with A_2 \limp A_1 \\
%     \vdash A_1 \with A_2 \limp A_2
%   \end{lgathered}
% \end{equation*}


% \newcommand*{\approxident}{%
%   \mathrel{\vcenter{\offinterlineskip
%   \hbox{$\sim$}\vskip-.35ex\hbox{$\sim$}\vskip-.35ex\hbox{$\sim$}}}}

% % For reasons better explained in future \lcnamecrefs{ch:singleton-logic}, we need a Curry--Howard interpretation of singleton logic as a session-type system for asynchronous processes.

% Notice that there is more than one way to extend a proof of $\slseq{A |- B_1}$ to a proof of $\slseq{A |- B_1 \plus B_2}$.
% Of course, the most obvious way is to simply apply the $\rrule{\plus}_1$ rule to the given proof.
% Another, less direct way is to cut the given proof against an identity embellished with the $\rrule{\plus}_1$ rule.
% \begin{equation*}
%   \infer[\rrule{\plus}_1]{\slseq{A |- B_1 \plus B_2}}{
%     \slseq{A |- B_1}}
%   \qquad\text{and}\qquad
%   \infer[\jrule{CUT}^{B_1}]{\slseq{A |- B_1 \plus B_2}}{
%     \slseq{A |- B_1} &
%     \infer[\rrule{\plus}_1]{\slseq{B_1 |- B_1 \plus }}{
%       \infer[\jrule{ID}]{\slseq{B_1 |- B_1}}{}}}
% \end{equation*}
% In fact, these proofs do not merely conclude with the same sequent -- these proofs are related by cut reduction:
% \begin{equation*}
%   \infer[\jrule{CUT}^{B_1}]{\slseq{A |- B_1 \plus B_2}}{
%     \slseq{A |- B_1} &
%     \infer[\rrule{\plus}_1]{\slseq{B_1 |- B_1 \plus B_2}}{
%       \infer[\jrule{ID}]{\slseq{B_1 |- B_1}}{}}}
%   \longrightarrow\longrightarrow\enspace
%   \infer[\rrule{\plus}_1]{\slseq{A |- B_1 \plus B_2}}{
%     \slseq{A |- B_1}}
% \end{equation*}
% This suggests that the $\rrule{\plus}_1$-embellished identity may be worthy of special consideration.

% Suppose that we replace the usual $\rrule{\plus}_1$ rule with a new primitive rule that collapses the embellished identity.
% \begin{equation*}
%   \infer[\rrule{\plus}_1]{\slseq{B_1 |- B_1 \plus B_2}}{
%     \infer[\jrule{ID}]{\slseq{B_1 |- B_1}}{}}
%   \rightsquigarrow
%   \infer[\rrule{\plus}_1']{\slseq{B_1 |- B_1 \plus B_2}}{}
% \end{equation*}


% \begin{gather*}
%   \infer[\jrule{CUT}^{B_1 \plus B_2}]{\slseq{A |- C}}{
%     \infer[\jrule{CUT}^{B_1}]{\slseq{A |- B_1 \plus B_2}}{
%       \slseq{A |- B_1} &
%       \infer[\rrule{\plus}_1]{\slseq{B_1 |- B_1 \plus B_2}}{
%         \infer[\jrule{ID}]{\slseq{B_1 |- B_1}}{}}} &
%     \infer[\lrule{\plus}]{\slseq{B_1 \plus B_2 |- C}}{
%       \slseq{B_1 |- C} & \slseq{B_2 |- C}}}
%   \\\equiv\\
%   \infer[\jrule{CUT}^{B_1}]{\slseq{A |- C}}{
%     \slseq{A |- B_1} &
%     \infer[\jrule{CUT}^{B_1 \plus B_2}]{\slseq{B_1 |- C}}{
%       \infer[\rrule{\plus}_1]{\slseq{B_1 |- B_1 \plus B_2}}{
%         \infer[\jrule{ID}]{\slseq{B_1 |- B_1}}{}} &
%       \infer[\lrule{\plus}]{\slseq{B_1 \plus B_2 |- C}}{
%         \slseq{B_1 |- C} & \slseq{B_2 |- C}}}}
%   \\\longrightarrow\longrightarrow\\
%   \infer[\jrule{CUT}^{B_1}]{\slseq{A |- C}}{
%     \slseq{A |- B_1} & \slseq{B_1 |- C}}
% \end{gather*}

% \begin{gather*}
%   \infer[\jrule{CUT}^{B_1 \plus B_2}]{\slseq{A |- C}}{
%     \infer[\rrule{\plus}_1]{\slseq{A |- B_1 \plus B_2}}{
%       \slseq{A |- B_1}} &
%     \infer[\lrule{\plus}]{\slseq{B_1 \plus B_2 |- C}}{
%       \slseq{B_1 |- C} & \slseq{B_2 |- C}}}
%   \\\longrightarrow\\
%   \infer[\jrule{CUT}^{B_1}]{\slseq{A |- C}}{
%     \slseq{A |- B_1} & \slseq{B_1 |- C}}
% \end{gather*}


% \begin{equation*}
%   \selectR{\kay}[P]
%   \approxident
%   \spawn{P}{(\selectR{\kay}[\fwd])}
% \end{equation*}

% \begin{marginfigure}
%   \begin{gather*}
%     \infer[\rrule{\plus}_1]{\slof{A |- \selectR{\kay}[P] : B_1 \plus B_2}}{
%       \slseq{A |- P : B_1}}
%     \\\approxident\\
%     \infer[\jrule{CUT}^{B_1}]{\slseq{A |- \spawn{P}{(\selectR{\kay}[\fwd])} : B_1 \plus B_2}}{
%       \slseq{A |- P : B_1} &
%       \infer[\rrule{\plus}_1]{\slseq{B_1 |- \selectR{\kay}[\fwd] : B_1 \plus B_2}}{
%         \infer[\jrule{ID}]{\slseq{B_1 |- \fwd : B_1}}{}}}
%   \end{gather*}
%   \caption{}
% \end{marginfigure}


  
% \begin{gather*}
%   \begin{lgathered}
%     \spawn{(\selectR{\kay}[P])}{\caseL[\ell \in L]{\ell => Q_{\ell}}}
%       \longrightarrow \spawn{P}{Q_{\kay}}
%     \\
%     \begin{aligned}
%       \MoveEqLeft[.5]
%       \spawn{\bigl(\spawn{P}{(\selectR{\kay}[\fwd])}\bigr)}{\caseL[\ell \in L]{\ell => Q_{\ell}}} \\[-.75\jot]
%         &\equiv \spawn{P}{\bigl(\spawn{(\selectR{\kay}[\fwd])}{\caseL[\ell \in L]{\ell => Q_{\ell}}}\bigr)} \\[-.75\jot]
%         % &\longrightarrow \spawn{P}{(\spawn{\fwd}{Q_{\kay}})} \\[-.75\jot]
%         &\longrightarrow\longrightarrow \spawn{P}{Q_{\kay}}
%     \end{aligned}
%   \end{lgathered}
% \end{gather*}

% % \begin{gather*}
% %   \infer-[\jrule{CUT}^{B_1 \plus B_2}]{\slseq{A |- C}}{
% %     \infer[\rrule{\plus}_1]{\slseq{A |- B_1 \plus B_2}}{
% %       \slseq{A |- B_1}} &
% %     \infer[\lrule{\plus}]{\slseq{B_1 \plus B_2 |- C}}{
% %       \slseq{B_1 |- C} & \slseq{B_2 |- C}}}
% %   =
% %   \infer-[\jrule{CUT}^{B_1}]{\slseq{A |- B_1 \plus B_2}}{
% %     \slseq{A |- B_1} & \slseq{B_1 |- C}}
% %   \\=\\
% %   \infer-[\jrule{CUT}^{B_1 \plus B_2}]{\slseq{A |- C}}{
% %     \infer[\jrule{CUT}^{B_1}]{\slseq{A |- B_1 \plus B_2}}{
% %       \slseq{A |- B_1} &
% %       \infer[\rrule{\plus}_1]{\slseq{B_1 |- B_1 \plus B_2}}{
% %         \infer[\jrule{ID}]{\slseq{B_1 |- B_1}}{}}} &
% %     \infer[\lrule{\plus}]{\slseq{B_1 \plus B_2 |- C}}{
% %       \slseq{B_1 |- C} & \slseq{B_2 |- C}}}
% %   \equiv
% %   \infer-[\jrule{CUT}^{B_1}]{\slseq{A |- C}}{
% %     \slseq{A |- B_1} &
% %     \infer-[\jrule{CUT}^{B_1 \plus B_2}]{\slseq{B_1 |- C}}{
% %       \infer[\rrule{\plus}_1]{\slseq{B_1 |- B_1 \plus B_2}}{
% %         \infer[\jrule{ID}]{\slseq{B_1 |- B_1}}{}} &
% %       \infer[\lrule{\plus}]{\slseq{B_1 \plus B_2 |- C}}{
% %         \slseq{B_1 |- C} & \slseq{B_2 |- C}}}}
% %   \\=\\
% %   \infer-[\jrule{CUT}^{B_1}]{\slseq{A |- C}}{
% %     \slseq{A |- B_1} &
% %     \infer-[\jrule{CUT}^{B_1 \plus B_2}]{\slseq{B_1 |- C}}{
% %       \infer[\jrule{ID}]{\slseq{B_1 |- B_1}}{} &
% %       \slseq{B_1 |- C}}}
% %   \\=\\
% %   \infer-[\jrule{CUT}^{B_1}]{\slseq{A |- C}}{
% %     \slseq{A |- B_1} & \slseq{B_1 |- C}}
% % \end{gather*}


% \begin{equation*}
%   \infer[\rrule{\plus}_1]{\slseq{B_1 |- B_1 \plus B_2}}{
%     \infer[\jrule{ID}]{\slseq{B_1 |- B_1}}{}}
%   \leftrightsquigarrow
%   \infer[\rrule{\plus}_1']{\slseq{B_1 |- B_1 \plus B_2}}{}
% \end{equation*}


% \begin{equation*}
%   \infer[\rrule{\plus}_1]{\slseq{A |- B_1 \plus B_2}}{
%     \slseq{A |- B_1}}
%   \leftrightsquigarrow
%   \infer[\jrule{CUT}^{B_1}]{\slseq{A |- B_1 \plus B_2}}{
%     \slseq{A |- B_1} &
%     \infer[\rrule{\plus}_1']{\slseq{B_1 |- B_1 \plus B_2}}{}}
% \end{equation*}


% \begin{equation*}
%   \infer-[\jrule{CUT}^{A_1}]{\slseq{A_1 |- C}}{
%     \infer[\rrule{\plus}_1']{\slseq{A_1 |- A_1 \plus A_2}}{} &
%     \infer[\lrule{\plus}]{\slseq{A_1 \plus A_2 |- C}}{
%       \deduce{\slseq{A_1 |- C}}{\EE_1} &
%       \deduce{\slseq{A_2 |- C}}{\EE_2}}}
%   \rightsquigarrow
%   \deduce{\slseq{A_1 |- C}}{\EE_1}
% \end{equation*}



% \begin{inferences}
%   \infer[\jrule{MP}]{\slseq{|- C}}{
%     \slseq{|- B} & \slseq{|- B \limp C}}
%   \\
%   \begin{lgathered}
%   \vdash A \limp A
%   \\
%   \vdash A_1 \limp A_1 \plus A_2 \\
%   \vdash A_2 \limp A_1 \plus A_2 \\
%   \vdash (A_2 \limp C) \limp (A_1 \limp C) \limp (A_1 \plus A_2 \limp C)
% \end{lgathered}
% \end{inferences}





\newthought{The semi-axiomatic} sequent calculus of \cref{fig:singleton-logic:hilbert} shares many rules with the singleton sequent calculus~\parencref{fig:singleton-logic:seq-calc}.
In fact, it is a variant in which each connective's non-invertible rules have been replaced with zero-premise rules.
From this observation, we conjecture that every intuitionistic sequent calculus has a corresponding semi-axiomatic sequent calculus.
But, in this document, we are only interested in the semi-axiomatic sequent calculus for singleton logic, so we do not pursue this conjecture further.

This semi-axiomatic variant being so closely related to the sequent calculus, we should seek to prove that it enjoys the usual sequent calculus metatheorems -- cut elimination and identity expansion.
Strictly speaking, however, cut elimination does not hold for the semi-axiomatic sequent calculus.
As a concrete counterexample, there is no cut-free semi-axiomatic proof of the sequent $\slseq{a_2 |- a_1 \plus (a_2 \plus a_3)}$, even though the same sequent is provable using cut:
% \begin{equation*}
%   \infer[\jrule{CUT}]{\slseq{p |- \top \plus \top}}{
%     \infer[\rrule{\top}]{\slseq{p |- \top}}{} &
%     \infer[\rrule{\plus}'_1]{\slseq{\top |- \top \plus \top}}{}}
% \end{equation*}
\begin{equation*}
  \infer[\jrule{CUT}]{\slseq{a_2 |- a_1 \plus (a_2 \plus a_3)}}{
    \infer[\rrule{\plus}'_1]{\slseq{a_2 |- a_2 \plus a_3}}{} &
    \infer[\rrule{\plus}'_2]{\slseq{a_2 \plus a_3 |- a_1 \plus (a_2 \plus a_3)}}{}}
\end{equation*}
Although cut elimination does not hold, normal forms nevertheless exist.
% , we will still be able to prove a weak normalization result.
Normal semi-axiomatic proofs will contain cuts, but those cuts will have a particular, analytic form.
In other words, although full cut elimination does not hold,
% of all cuts is not possible, but
elimination of \emph{non-analytic} cuts does.

\subsection{A proof term assignment for the semi-axiomatic sequent calculus}

Before presenting a proof of non-analytic cut elimination, we will take a moment to introduce a proof term assignment for the semi-axiomatic sequent calculus.
These proof terms will be a convenient, succinct notation with which to describe the elimination procedure.
To keep the proof terms compact, we will also take this opportunity to introduce labeled, $n$-ary forms of additive disjunction and conjunction.

\begin{figure}[tbp]
  \vspace*{\dimexpr-\abovedisplayskip-\abovecaptionskip\relax}
  \begin{syntax*}
    Propositions &
      A & a \mid \plus*[sub=_{\ell \in L}]{\ell:A_{\ell}}
            \mid \with*[sub=_{\ell \in L}]{\ell:A_{\ell}}
    \\
    Proof terms &
      P & \spawn{P_1}{P_2} \mid \fwd
          \begin{array}[t]{@{{} \mid {}}l@{}}
            \selectR{\kay} \mid \caseL[\ell \in L]{\ell => P_{\ell}} \\
            \caseR[\ell \in L]{\ell => P_{\ell}} \mid \selectL{\kay}
          \end{array}
  \end{syntax*}
  \begin{inferences}
    \infer[\jrule{CUT}^B]{\slseq{A |- \spawn{P_1}{P_2} : C}}{
      \slseq{A |- P_1 : B} & \slseq{B |- P_2 : C}}
    \and
    \infer[\jrule{ID}^A]{\slseq{A |- \fwd : A}}{}
    \\
    \infer[\rrule{\plus}']{\slseq{A_{\kay} |- \selectR{\kay} : \plus*[sub=_{\ell \in L}]{\ell:A_{\ell}}}}{
      \text{($\kay \in L$)}}
    \and
    \infer[\lrule{\plus}]{\slseq{\plus*[sub=_{\ell \in L}]{\ell:A_{\ell}} |- \caseL[\ell \in L]{\ell => P_{\ell}} : C}}{
      \multipremise{\ell \in L}{\slseq{A_{\ell} |- P_{\ell} : C}}}
    \\
    \infer[\rrule{\with}]{\slseq{A |- \caseR[\ell \in L]{\ell => P_{\ell}} : \with*[sub=_{\ell \in L}]{\ell:C_{\ell}}}}{
      \multipremise{\ell \in L}{\slseq{A |- P_{\ell} : C_{\ell}}}}
    \and
    \infer[\lrule{\with}']{\slseq{\with*[sub=_{\ell \in L}]{\ell:C_{\ell}} |- \selectL{\kay} : C_{\kay}}}{
      \text{($\kay \in L$)}}
  \end{inferences}
  \vspace{-\belowdisplayskip}%
  \caption{Proof terms for a labeled, $n$-ary variant of the semi-axiomatic sequent calculus of \cref{fig:singleton-logic:hilbert}}%
  \label{fig:singleton-logic:hilbert-terms}
\end{figure}
%
\Cref{fig:singleton-logic:hilbert-terms} presents a labeled, $n$-ary generalization of singleton logic's semi-axiomatic sequent calculus, equipped with proof terms.
Individual labels $\ell$ and $\kay$ are drawn from an unspecified universe of labels, and the metavariable $L$ is used for index sets of labels.
The labeled, $n$-ary proposition $\plus*[sub=_{\ell \in L}]{\ell:A_{\ell}}$ generalizes binary additive disjunction, $A \plus B$, and, because the label set $L$ may even be empty, it also generalizes additive falsehood, $\zero$.
Likewise, $\with*[sub=_{\ell \in L}]{\ell:A_{\ell}}$ generalizes both $A \with B$ and $\top$.

Because the $\jrule{CUT}$ rule serves to compose two proofs of compatible sequents, the proof term $\spawn{P_1}{P_2}$ was chosen for its suggestion of function composition, $f_2 \circ f_1$.%
\footnote{Notice that the order of composition in the $\spawn{P_1}{P_2}$ term matches the order of premises in the $\jrule{CUT}$ rule, but is opposite the order traditionally used for function composition.}
The proof term $\fwd$ is used for the $\jrule{ID}$ rule.
%
Because of their similar structure, the $\rrule{\plus}'$ and $\lrule{\with}'$ rules are assigned the similar proof terms $\selectR{\kay}$ and $\selectL{\kay}$; the direction of the underlying arrow distinguishes them.
Similarly, the $\lrule{\plus}$ and $\rrule{\with}$ rules are assigned the proof terms $\caseL[\ell \in L]{\ell => P_{\ell}}$ and $\caseR[\ell \in L]{\ell => P_{\ell}}$.
%
All of these terms serve as variable-free combinators.%
%
\subsection{Non-analytic cut elimination for the semi-axiomatic sequent calculus}\label{sec:singleton-logic:sax:cutelim}

With proof terms in hand, we can now return to our goal of establishing a \emph{non-analytic} cut elimination \lcnamecref{thm:singleton:hilbert:cutelim} for singleton logic's semi-axiomatic sequent calculus.

The cut elimination procedure will normalize a semi-axiomatic proof so that any remaining cuts are analytic, specifically of the forms $\spawn{\selectL{\kay}}{P}$ or $\spawn{P}{\selectR{\kay}}$.
As shown in the neighboring display,%
%
\marginnote[-3\baselineskip]{
  \begin{gather*}
    \infer[\jrule{CUT}^{A_{\kay}}]{\slof{\with*[sub=_{\ell \in L}]{\ell:A_{\ell}} |- \spawn{\selectL{\kay}}{P} : C}}{
      \infer[\lrule{\with}']{\slof{\with*[sub=_{\ell \in L}]{\ell:A_{\ell}} |- \selectL{\kay} : A_{\kay}}}{
        \text{($\kay \in L$)}} &
      \slof{A_{\kay} |- P : C}}
    \\[2\jot]
    \text{and}
    \\[2\jot]
    \infer[\jrule{CUT}^{C_{\kay}}]{\slof{A |- \spawn{P}{\selectR{\kay}} : \plus*[sub=_{\ell \in L}]{\ell:C_{\ell}}}}{
      \slof{A |- P : C_{\kay}} &
      \infer[\rrule{\plus}']{\slof{C_{\kay} |- \selectR{\kay} : \plus*[sub=_{\ell \in L}]{\ell:C_{\ell}}}}{
        \text{($\kay \in L$)}}}
  \end{gather*}
}
%
cuts of these forms are analytic because the cut formula is a subformula of the conclusion sequent.

We say that a term is \emph{normal} if it contains only cuts of these analytic forms;
the normal terms are generated by the following grammar.
\begin{equation*}
  % Normal terms &
    N,M % &
    \Coloneqq
    \begin{array}[t]{@{}l@{}}
      \fwd \mid \spawn{N}{\selectR{\kay}} \mid \spawn{\selectL{\kay}}{N} \\
      \begin{array}[t]{@{\mathllap{\mid {}}}l@{}}
        \selectR{\kay} \mid \caseL[\ell \in L]{\ell => N_{\ell}} \\
        \caseR[\ell \in L]{\ell => N_{\ell}} \mid \selectL{\kay}
      \end{array}
    \end{array}
\end{equation*}
In other words, normality is an extrinsic property of terms that is judged by membership in the above grammar.

Non-analytic cut elimination then amounts to proof term normalization:
%
\begin{restatable*}[
  name = Non-analytic cut elimination,
  label = thm:singleton:hilbert:cutelim
]{theorem}{thmsingletonhilbertcutelim}
  If $\slof{A |- P : C}$, then $\slof{A |- N : C}$ for some normal term $N$.
\end{restatable*}
%
Just like the sequent calculus's cut elimination result~\parencref{thm:singleton-logic:seq-calc:cut-elimination}, this \lcnamecref{thm:singleton:hilbert:cutelim} can be proved by a straightforward structural induction, this time on the given term, $P$.
First, however, we need the admissibility of non-analytic cut as a \lcnamecref{lem:singleton-logic:hilbert:cut-admissibility}:
%
\begin{lemma}[Admissibility of non-analytic cut]\label{lem:singleton-logic:hilbert:cut-admissibility}
  If $\slof{A |- N : B}$ and $\slof{B |- M : C}$, then $\slof{A |- N' : C}$ for some normal term $N'$.
\end{lemma}
%
\begin{proof}
  As with \cref{lem:singleton-logic:cut-admissibility} for the sequent calculus, this \lcnamecref{lem:singleton-logic:hilbert:cut-admissibility} states the admissibility of a cut principle, and its proof amounts to the definition of a meta-level function on proofs.
  However, with proof terms, we can now make that function definition more apparent.

  Let $\nspawn{}{}$ be a nondeterministic binary function on normal terms $N$ and $M$ of compatible types such that $\nspawn{N}{M}$ is a normal term of the corresponding type:
  % We will define $\nspawn{}{}$ as a meta-level function from pairs of normal terms of compatible types to normal terms of the corresponding type.
  \begin{equation*}
    \infer-[\jrule{A-CUT}\smash{^B}]{\slof{A |- \nspawn{N}{M} : C}}{
      \slof{A |- N : B} & \slof{B |- M : C}}
    \,.
  \end{equation*}

  Once again, we will prove the cut principle by a lexicographic induction, first on the cut formula, $B$, and then on the structures of the given terms, $N$ and $M$.
  However, because the semi-axiomatic formulation uses different rules than the sequent calculus, the proof's cases are organized a bit differently.
  In addition to the usual classes of principal, identity, and commutative cases, a new class of associative cases is introduced.
  \begin{description}[listparindent=\parindent, parsep=0pt]
  \item[Associative cases]
    Consider, for example, the case $\nspawn{(\spawn{N_0}{\selectR{\kay}})}{M}$.
    Because the term $\selectR{\kay}$ is itself normal, the above term can be reassociated, suggesting that we adopt 
    \begin{equation*}
      \nspawn{(\spawn{N_0}{\selectR{\kay}})}{M}
        = \nspawn{N_0}{(\nspawn{\selectR{\kay}}{M})}
    \end{equation*}
    as a clause in the definition of $\nspawn{}{}$.
    But is this clause terminating?

    Yes, indeed it is.
    In $\nspawn{N_0}{(\nspawn{\selectR{\kay}}{M})}$, the inner $\nspawn{\selectR{\kay}}{M}$ terminates because the terms have together become smaller -- $\selectR{\kay}$ is a proper subterm of $\spawn{N_0}{\selectR{\kay}}$ -- while the cut formula
% and other term
 remains unchanged.
    The outer $\nspawn{N_0}{(\nspawn{\selectR{\kay}}{M})}$ also terminates, despite $\nspawn{\selectR{\kay}}{M}$ possibly being larger than $M$, because the cut formula has become smaller.
    % \footnote{
    To aid the reader in tracking the types, \cref{fig:singleton-logic:hilbert:associative-cut} shows the full typing derivations.%}%
    %
    \begin{figure}[tbp]
      \vspace*{\dimexpr-\abovedisplayskip-\abovecaptionskip\relax}
      \begin{gather*}
        \infer-[\jrule{A-CUT}\smash{^B}]{\slof{A |- \nspawn{(\spawn{N_0}{\selectR{\kay}})}{M} : C}}{
          \infer[\jrule{CUT}\smash{^{B_{\kay}}}]{\slof{A |- \spawn{N_0}{\selectR{\kay}} : \plus*[sub=_{\ell \in L}]{\ell:B_{\ell}}}}{
            \slof{A |- N_0 : B_{\kay}} &
            \infer[\rrule{\plus}']{\slof{B_{\kay} |- \selectR{\kay} : \plus*[sub=_{\ell \in L}]{\ell:B_{\ell}}}}{
              \text{($\kay \in L$)}}} &
          \slof{\plus*[sub=_{\ell \in L}]{\ell:B_{\ell}} |- M : C}}
        \\=\\
        \infer-[\jrule{A-CUT}\smash{^{B_{\kay}}}]{\slof{A |- \nspawn{N_0}{(\nspawn{\selectR{\kay}}{M})} : C}}{
          \slof{A |- N_0 : B_{\kay}} &
          \infer-[\jrule{A-CUT}\smash{^B}]{\slof{B_{\kay} |- \nspawn{\selectR{\kay}}{M} : C}}{
            \infer[\rrule{\plus}']{\slof{B_{\kay} |- \selectR{\kay} : \plus*[sub=_{\ell \in L}]{\ell:B_{\ell}}}}{
              \text{($\kay \in L$)}} &
            \slof{\plus*[sub=_{\ell \in L}]{\ell:B_{\ell}} |- M : C}}}
      \end{gather*}
      \vspace{-\belowdisplayskip}
      % \bottomrule
      \caption{One of the associative cases in the proof of non-analytic cut admissibility~\parencref{lem:singleton-logic:hilbert:cut-admissibility}}%
      \label{fig:singleton-logic:hilbert:associative-cut}
    \end{figure}

    The symmetric case, $\nspawn{N}{(\spawn{\selectL{\kay}}{M_0})}$, is also an associative case and is handled similarly.
    The complete set of associative clauses is therefore:
    \begin{align*}
      \nspawn{(\spawn{N_0}{\selectR{\kay}})}{M}
        &= \nspawn{N_0}{(\nspawn{\selectR{\kay}}{M})}
      \\
      \nspawn{N}{(\spawn{\selectL{\kay}}{M_0})}
        &= \nspawn{(\nspawn{N}{\selectL{\kay}})}{M_0}
      \,.
    \end{align*}
    Both of these associative cases detach a label and group it together with the neighboring term, thereby enabling interactions between the label and term.

  \item[Principal cases]
    Because the above associative cases decompose the analytic cuts $\spawn{N_0}{\selectR{\kay}}$ and $\spawn{\selectL{\kay}}{M_0}$, the principal cases need only cover those pairings of the $\rrule{\plus}'$ rule with a proof ending in the $\lrule{\plus}$ rule and the symmetric pairings involving the $\rrule{\with}$ and $\lrule{\with}'$ rules:
    \begin{align*}
      \nspawn{\selectR{\kay}}{\caseL[\ell \in L]{\ell => M_{\ell}}}
        &= M_{\kay}
      \\
      \nspawn{\caseR[\ell \in L]{\ell => N_{\ell}}}{\selectL{\kay}}
        &= N_{\kay}
    \end{align*}
    If $\selectR{\kay}$ and $\selectL{\kay}$ are viewed as directed messages, then these principal clauses in $\nspawn{}{}$'s definition look much like rules for asynchronous message-passing communication.
    This observation is at the heart of the Curry--Howard interpretation of singleton logic's semi-axiomatic sequent calculus that we develop in the following \lcnamecref{ch:process-chains}.

  \item[Identity cases]
    As in the proof of admissibility of cut for the sequent calculus~\parencref{lem:singleton-logic:cut-admissibility}, the identity cases cover pairings involving the $\jrule{ID}$ rule and yield the following clauses.
    \begin{align*}
      \nspawn{\fwd}{M} &= M \\
      \nspawn{N}{\fwd} &= N
    \end{align*}


\item[Commutative cases]
  In the remaining cases, one of the two terms has a top-level constructor that introduces a side formula.
  For instance, in the cut $\nspawn{\caseL[\ell \in L]{\ell => N_{\ell}}}{M}$, the constructor $\caseL[\ell \in L]{\ell => \mathord{-}}$ introduces the side formula $\plus*[sub=_{\ell \in L}]{\ell:A_{\ell}}$.
  The left-commutative cases yield the following clauses for the definition of $\nspawn{}{}$.
  \begin{align*}
    \nspawn{(\spawn{\selectL{\kay}}{N_0})}{M}
      &= \spawn{\selectL{\kay}}{(\nspawn{N_0}{M})}
    \\
    \nspawn{\selectL{\kay}}{M} &= \spawn{\selectL{\kay}}{M}
    \\
    \nspawn{\caseL[\ell \in L]{\ell => N_{\ell}}}{M}
      &= \caseL[\ell \in L]{\ell => \nspawn{N_{\ell}}{M}}
  \end{align*}
  In all of these clauses, the $\nspawn{}{}$ is permuted with a normal term's top-level constructor.
  The first clause reassociates cuts, but it has a much different flavor than the above associative cases because $\selectL{\kay}$ is directed outward here.

  There are also several right-commutative cases that are symmetric to the preceding left-commutative cases:
  \begin{align*}
    \nspawn{N}{(\spawn{M_0}{\selectR{\kay}})}
      &= \spawn{(\nspawn{N}{M_0})}{\selectR{\kay}}
    \\
    \nspawn{N}{\selectR{\kay}} &= \spawn{N}{\selectR{\kay}}
    \\
    \nspawn{N}{\caseR[\ell \in L]{\ell => M_{\ell}}}
      &= \caseR[\ell \in L]{\ell => \nspawn{N}{M_{\ell}}}
  \mathrlap{\,.}
  \qedhere
  \end{align*}
\end{description}
\end{proof}

Notice that the function $\nspawn{}{}$ defined by this \lcnamecref{lem:singleton-logic:hilbert:cut-admissibility} is, in fact, nondeterministic.
Many nontrivial critical pairs exist, due to overlapping clauses in the function's definition.
For instance, both
\begin{align*}
  \MoveEqLeft[.75]
  \nspawn{\caseL[\ell \in L]{\ell => N_{\ell}}}
         {\caseR[\kay \in K]{\kay => M_{\kay}}} \\
    &= \caseL[\ell \in L]{\ell =>
         \caseR[\kay \in K]{\kay => \nspawn{N_{\ell}}{M_{\kay}}}}
%
\shortintertext{and}
%
  \MoveEqLeft[.75]
  \nspawn{\caseL[\ell \in L]{\ell => N_{\ell}}}
         {\caseR[\kay \in K]{\kay => M_{\kay}}} \\
    &= \caseR[\kay \in K]{\kay =>
         \caseL[\ell \in L]{\ell => \nspawn{N_{\ell}}{M_{\kay}}}}
\end{align*}
hold.
We conjecture that $\nspawn{}{}$ is deterministic up to commuting conversions, but will not attempt to prove that result here.

Of course, with the addition of enough side conditions, the function $\nspawn{}{}$ could be refined into one that is also deterministic at a purely syntactic level.
But many of the choices that would be made in breaking ties, such as between the above two terms, seem rather arbitrary, so we prefer to have $\nspawn{}{}$ remain nondeterministic.

\newthought{With this \lcnamecref{lem:singleton-logic:hilbert:cut-admissibility}} in hand, we may finally proceed to proving non-analytic cut elimination.
%
\thmsingletonhilbertcutelim
%
\begin{proof}
  By structural induction on the proof term $P$.

  This theorem can be phrased as the following admissible rule.
  \begin{equation*}
    \infer-[\jrule{CE}]{\slof{A |- \ce{P} : C}}{
      \slof{A |- P : C}}
  \end{equation*}
  The principal case in proving the admissibility of this rule is:
  \begin{equation*}
    \infer-[\jrule{CE}]{\slof{A |- \ce{\spawn{P_1}{P_2}} : C}}{
      \infer[\jrule{CUT}^B]{\slof{A |- \spawn{P_1}{P_2} : C}}{
        \slof{A |- P_1 : B} & \slof{B |- P_2 : C}}}
    =\:
    \infer-[\jrule{A-CUT}^B]{\slof{A |- \nspawn{\ce{P_1}}{\ce{P_2}} : C}}{
      \infer-[\jrule{CE}]{\slof{A |- \ce{P_1} : C}}{
        \slof{A |- P_1 : B}} &
      \infer-[\jrule{CE}]{\slof{B |- \ce{P_2} : C}}{
        \slof{B |- P_2 : C}}}
  \end{equation*}
  In this case, a cut is replaced with an appeal to the admissibility of cut.
  The remaining cases are handled compositionally:
  \begin{equation*}
    \begin{lgathered}[b]
      \ce{\spawn{P_1}{P_2}} = \nspawn{\ce{P_1}}{\ce{P_2}} \\
      \ce{\fwd} = \fwd \\
      \ce{\selectR{\kay}} = \selectR{\kay} \\
      \ce{\caseL[\ell \in L]{\ell => P_{\ell}}} = \caseL[\ell \in L]{\ell => \ce{P_{\ell}}} \\
      \ce{\caseR[\ell \in L]{\ell => P_{\ell}}} = \caseR[\ell \in L]{\ell => \ce{P_{\ell}}} \\
      \ce{\selectL{\kay}} = \selectL{\kay}
    \end{lgathered}
    \qedhere
  \end{equation*}
\end{proof}




% \subsection{Operational semantics}

% \begin{syntax*}
%   Configurations & \cnf & \cnf_1 \cc \cnf_2 \mid \cnfe \mid P
% \end{syntax*}

% \begin{equation*}
%   \begin{gathered}
%     \!\begin{aligned}
%       \fwd &\reduces \cdot \\
%       \spawn{P_1}{P_2} &\reduces P_1 \cc P_2
%     \end{aligned} \\
%     \selectR{\kay} \cc \caseL[\ell \in L]{\ell => P_{\ell}}
%       \reduces P_{\kay} \\
%     \caseR[\ell \in L]{\ell => P_{\ell}} \cc \selectL{\kay}
%       \reduces P_{\kay}
%     \\[2\jot]
%     \cnf \cc \cnfe = \cnf = \cnfe \cc \cnf \\
%     (\cnf_1 \cc \cnf_2) \cc \cnf_3 = \cnf_1 \cc (\cnf_2 \cc \cnf_3)
%   \end{gathered}
% \end{equation*}

% \begin{equation*}
%   \begin{aligned}
%     (\cnf_1 \cc \cnf_2)^\sharp &= \spawn{\cnf_1^\sharp}{\cnf_2^\sharp} \\
%     (\cnfe)^\sharp &= \fwd \\
%     P^\sharp &= P
%   \end{aligned}
% \end{equation*}


% \section{}

% \begin{equation*}
%   \begin{lgathered}
%     \eta(\alpha) = \fwd \\
%     \eta(\plus*[sub=_{\ell \in L}]{\ell:A_{\ell}}) = \caseL[\ell \in L]{\ell => \selectR{\ell}} \\
%     \eta(\with*[sub=_{\ell \in L}]{\ell:A_{\ell}}) = \caseR[\ell \in L]{\ell => \selectL{\ell}}
%   \end{lgathered}
% \end{equation*}

% \begin{equation*}
%   \begin{lgathered}
%     \eta(\alpha) = \fwd \\
%     \eta(\plus*[sub=_{\ell \in L}]{\ell:A_{\ell}}) = \caseL[\ell \in L]{\ell => \spawn{\eta(A_{\ell})}{\selectR{\ell}}} \\
%     \eta(\with*[sub=_{\ell \in L}]{\ell:A_{\ell}}) = \caseR[\ell \in L]{\ell => \spawn{\selectL{\ell}}{\eta(A_{\ell})}}
%   \end{lgathered}
% \end{equation*}

% \begin{equation*}
%   \begin{lgathered}
%     n(A, \spawn{P}{^B Q}, C) = \nspawn{n(A, P, B)}{n(B, Q, C)} \\
%     n(A, \fwd, A) = \eta(A) \\
%     n(\plus*[sub=_{\ell \in L}]{\ell:A_{\ell}}, \caseL[\ell \in L]{\ell => P_{\ell}}, C) = \caseL[\ell \in L]{\ell => n(A_{\ell}, P_{\ell}, C)}
%   \end{lgathered}
% \end{equation*}





\section{Extensions of singleton logic}\label{sec:singleton-logic:extensions}

The usual and semi-axiomatic sequent calculi for singleton logic support various extensions.
One simple but useful extension would be to introduce first-order universal and existential quantifiers, $\forall x{:}\tau.A$ and $\exists x{:}\tau.A$, over well-sorted data.
Variable typing assumptions, $x{:}\tau$, would not be subject to the single-antecedent restriction -- the usual weakening, contraction, and exchange properties apply to variable typing assumptions.

Another direction for extension is to slightly relax the single-antecedent restriction.
Instead of demanding that sequents have exactly one antecedent and exactly one consequent, we could allow each sequent to have zero or one antecedents and zero or one consequents.
% So now, instead of sequents $\slseq{A |- C}$, we have sequents $\slseq{\sctx |- \cseq}$, where $\sctx$ and $\cseq$ adhere to the following grammars.
% \begin{syntax*}
%   Subsingleton antecedents & \sctx & A \mid \sctxe \\
%   Subsingleton conseq{}uents & \cseq & C \mid \cseqe
% \end{syntax*}
With this relaxation, we arrive at \emph{sub}singleton logic, which now includes the multiplicative units $\one$ and $\bot$.
%  can now be characterized by right and left rules:
% \begin{inferences}
%   \infer[\rrule{\one}]{\slseq{\sctxe |- \one}}{}
%   \and
%   \infer[\lrule{\one}]{\slseq{\one |- \cseq}}{
%     \slseq{\sctxe |- \cseq}}
%   \\
%   \infer[\rrule{\bot}]{\slseq{\sctx |- \bot}}{
%     \slseq{\sctx |- \cseqe}}
%   \and
%   \infer[\lrule{\bot}]{\slseq{\bot |- \cseqe}}{}
% \end{inferences}
% These rules apply to both the sequent calculus and Hilbert system presentations of subsingleton logic.

It is even possible to modify subsingleton logic to include the \enquote*{of course} modality, $\bang A$, from linear logic.
To make this work, singleton sequents are extended with a structural zone.
Both the right and left rules for $\bang A$ require that the context be allowed to be empty, so $\bang A$ is not possible for singleton logic, only subsingleton logic.

Interestingly, the proposition $A \mathbin{\bang\mathord{\tensor}} B$ with right and left rules
\begin{inferences}
  \infer[\rrule{\bang\tensor}]{\slseq{\uctx ; A |- B_1 \mathbin{\bang\tensor} B_2}}{
    \slseq{\uctx ; \sctxe |- B_1} & \slseq{\uctx ; A |- B_2}}
  \and
  \infer[\lrule{\bang\tensor}]{\slseq{\uctx ; B_1 \mathbin{\bang\tensor} B_2 |- C}}{
    \slseq{\uctx, B_1 ; B_2 |- C}}
\end{inferences}
would, strictly speaking, not be possible for singleton logic's sequent calculus: the $\rrule{\mathord{\bang\mathord{\tensor}}}$ rule uses an empty context.
But, interestingly, such a proposition would be possible in singleton logic's \emph{semi-axiomatic} sequent calculus:
the right rule would be replaced by
\begin{equation*}
  \infer[\rrule{\bang\tensor}]{\slseq{\uctx, A_1 ; A_2 |- A_1 \mathbin{\bang\tensor} A_2}}{}
  \,,
\end{equation*}
which does not violate the single-antecedent restriction.



% \section{Hilbert}

% \subsection{Hypothetical Hilbert system}

% \NewDocumentCommand{\hil}{}{\:\mathit{hil}}

% \begin{inferences}
%   \infer[\jrule{MP}]{\Gamma \vdash B \hil}{
%     \Gamma \vdash A \to B \hil & \Gamma \vdash A \hil}
%   \and
%   \infer[\jrule{HYP}]{\Gamma, A \hil \vdash A \hil}{}
%   \\
%   \begin{array}{l}
%     \Gamma \vdash A \to A \hil \\
%     \Gamma \vdash A \to (B \to A) \hil \\
%     \Gamma \vdash (A \to B \to C) \to (A \to B) \to (A \to C) \hil
%   \end{array}
% \end{inferences}

% \begin{theorem}
%   If $\Gamma \vdash A$, then $\hat{\Gamma} \vdash A \hil$.
% \end{theorem}
% \begin{proof}
%   \begin{equation*}
%     \infer[\rrule{\to}]{\Gamma \vdash A \to B}{
%       \Gamma, A \vdash B}
%   \end{equation*}
%   We need a deduction theorem.

%   \begin{equation*}
%     \infer[\lrule{\to}]{\Gamma, A \to B \vdash C}{
%       \Gamma, A \to B \vdash A & \Gamma, A \to B, B \vdash C}
%   \end{equation*}
%   We have $\Gamma, A \to B \vdash A \hil$ by induction.
%   We also have $\Gamma, A \to B \vdash B \to C \hil$ by induction and the deduction theorem.
%   Prove $\Gamma, A \to B \vdash B \hil$ by $\jrule{HYP}$ and $\jrule{MP}$.
%   Conclude $\Gamma, A \to B \vdash C \hil$ by $\jrule{MP}$.

%   \begin{equation*}
%     \infer[\jrule{CUT}]{\Gamma \vdash C}{
%       \Gamma \vdash A & \Gamma, A \vdash C}
%   \end{equation*}
%   Apply the inductive hypothesis and deduction theorem, then $\jrule{MP}$.
% \end{proof}

% \subsection{}

% \NewDocumentCommand{\sem}{m}{\llbracket#1\rrbracket}

% \begin{equation*}
%   \begin{lgathered}
%     \sem{x}\,\rho = \rho(x) \\
%     \sem{\lambda x.M}\,\rho\,v = \sem{M}\,(\rho[x \mapsto v]) \\
%     \sem{M\,N}\,\rho = \sem{M}\,\rho\,(\sem{N}\,\rho)
%   \end{lgathered}
% \end{equation*}

% \begin{equation*}
%   \begin{lgathered}
%     \sem{0}\,(\rho, v_0) = v_0 \qquad \sem{n+1}\,(\rho, v_0) = \sem{n}\,\rho \\
%     \sem{\lambda x.M}\,\rho\,v_0 = \sem{M}\,(\rho, v_0) \\
%     \sem{M\,N}\,\rho = \sem{M}\,\rho\,(\sem{N}\,\rho)
%   \end{lgathered}
% \end{equation*}

% \begin{equation*}
%   \begin{lgathered}
%     \sem{n} = \pi\,n \\
%     \sem{\lambda x.M} = \Lambda\,\sem{M} \\
%     \sem{M\,N} = S\,\sem{M}\,\sem{N}
%   \end{lgathered}
% \end{equation*}

% \begin{equation*}
%   \begin{lgathered}
%     \pi\,0\,(x,y) = y \qquad \pi\,(n+1)\,(x,y) = \pi\,n\,x \\
%     \Lambda\,f\,x\,y = f\,(x,y) \\
%     S\,x\,y\,z = x\,z\,(y\,z)
%   \end{lgathered}
% \end{equation*}


% \begin{equation*}
%   \begin{lgathered}
%     \vdash \circ : (B \to C) \to (A \to B) \to (A \to C) \\
%     \vdash \iota_1 : A \to A \lor B \\
%     \vdash \iota_2 : B \to A \lor B \\
%     f{:}A \to C, g{:}B \to C \vdash [f,g] : A \lor B \to C
%   \end{lgathered}
% \end{equation*}

% \begin{equation*}
%   \begin{lgathered}
%     [f_1,f_2]\,(\iota_i\,x) = f_i\,x \\
%     [f_1,f_2] \circ \iota_i = f_i
%   \end{lgathered}
% \end{equation*}


% \begin{equation*}
%   \begin{lgathered}
%     \sem{\selectR{i_1}[P]} = S_1\,\sem{P} \\
%     \sem{\caseL{i_1 => Q_1 | i_2 => Q_2}} = [\sem{Q_1}, \sem{Q_2}] \\
%     \sem{\spawn{P}{Q}} = \sem{Q} \circ \sem{P}
%   \end{lgathered}
% \end{equation*}

% \begin{equation*}
%   [g_1, g_2](S_1\,f\,x) = g_1(f\,x) and S_1\,f\,x = \iota_1\,(f\,x)
% \end{equation*}

% \section{Natural deduction}

% \begin{inferences}
%   \infer[\jrule{HYP}]{A \vdash A}{}
%   \and
%   \infer-[\jrule{SUBST}]{A \vdash C}{
%     A \vdash B & B \vdash C}
%   \\
%   \infer[{\plus}\text{\scshape i}]{A \vdash \plus*[sub=_{\ell \in L}]{\ell:B_{\ell}}}{
%     A \vdash B_{\kay} & \text{($\kay \in L$)}}
%   \and
%   \infer[{\plus}\text{\scshape e}]{A \vdash C}{
%     A \vdash \plus*[sub=_{\ell \in L}]{\ell:B_{\ell}} &
%     \multipremise{\ell \in L}{\!\!\!B_{\ell} \vdash C}}
%   \\
%   \infer[{\with}\text{\scshape i}]{A \vdash \with*[sub=_{\ell \in L}]{\ell:B_{\ell}}}{
%     \multipremise{\ell \in L}{\!\!\!A \vdash B_{\ell}}}
%   \and
%   \infer[{\with}\text{\scshape e}]{A \vdash B_{\kay}}{
%     A \vdash \with*[sub=_{\ell \in L}]{\ell:B_{\ell}} & \text{($\kay \in L$)}}
% \end{inferences}

% \begin{theorem}
%   If $A \vdash B$ in natural deduction, then $\slseq{A |- B}$ in the sequent calculus.
% \end{theorem}
% \begin{proof}
%   \begin{equation*}
%     \infer[{\with}\text{\scshape e}]{A \vdash B_{\kay}}{
%       A \vdash \with*[sub=_{\ell \in L}]{\ell:B_{\ell}} & \text{($\kay \in L$)}}  
%     \rightsquigarrow
%     \infer[\jrule{CUT}]{\slseq{A |- B_{\kay}}}{
%       \slseq{A |- \with*[sub=_{\ell \in L}]{\ell:B_{\ell}}} &
%       \infer[\lrule{\with}]{\slseq{\with*[sub=_{\ell \in L}]{\ell:B_{\ell}} |- B_{\kay}}}{
%         \text{($\kay \in L$)}}}
%   \end{equation*}
% \end{proof}

% \begin{theorem}
%   If $A \vdash B$ in sequent calculus, then $\slseq{A |- B}$ in the natural deduction.
% \end{theorem}
% \begin{proof}
%   \begin{gather*}
%     \infer[\lrule{\with}]{\with*[sub=_{\ell \in L}]{\ell:A_{\ell}} \vdash C}{
%       A_{\kay} \vdash C & \text{($\kay \in L$)}}
%     \\\rightsquigarrow\\
%     \infer-[\jrule{SUBST}]{\with*[sub=_{\ell \in L}]{\ell:A_{\ell}} \vdash C}{
%       \infer[{\with}\text{\scshape e}]{\with*[sub=_{\ell \in L}]{\ell:A_{\ell}} \vdash A_{\kay}}{
%         \infer[\jrule{HYP}]{\with*[sub=_{\ell \in L}]{\ell:A_{\ell}} \vdash \with*[sub=_{\ell \in L}]{\ell:A_{\ell}}}{} &
%         \text{($\kay \in L$)}} &
%       A_{\kay} \vdash C}
%   \end{gather*}

%   \begin{gather*}
%     \infer[\lrule{\plus}]{\plus*[sub=_{\ell \in L}]{\ell:A_{\ell}} \vdash C}{
%       \multipremise{\ell \in L}{\!\!\!A_{\ell} \vdash C}}
%     \\\rightsquigarrow\\
%     \infer[{\plus}\text{\scshape e}]{\plus*[sub=_{\ell \in L}]{\ell:A_{\ell}} \vdash C}{
%       \infer[\jrule{HYP}]{\plus*[sub=_{\ell \in L}]{\ell:A_{\ell}} \vdash \plus*[sub=_{\ell \in L}]{\ell:A_{\ell}}}{} &
%       \multipremise{\ell \in L}{\!\!\!A_{\ell} \vdash C}}
%   \end{gather*}
% \end{proof}


% \begin{inferences}
%   \infer[\jrule{HYP}]{A \dn \vdash A \dn}{}
%   \and
%   \infer-[\jrule{SUBST}]{A \dn \vdash C \up}{
%     A \dn \vdash B \dn & B \dn \vdash C \up}
%   \and
%   \infer-[\jrule{SUBST}]{A \dn \vdash C \dn}{
%     A \dn \vdash B \dn & B \dn \vdash C \dn}
%   \\
%   \infer[{\plus}\text{\scshape i}]{A \dn \vdash \plus*[sub=_{\ell \in L}]{\ell:B_{\ell}} \up}{
%     A \dn \vdash B_{\kay} \up & \text{($\kay \in L$)}}
%   \and
%   \infer[{\plus}\text{\scshape e}]{A \dn \vdash C \up}{
%     A \dn \vdash \plus*[sub=_{\ell \in L}]{\ell:B_{\ell}} \up &
%     \multipremise{\ell \in L}{\!\!\!B_{\ell} \dn \vdash C \up}}
%   \\
%   \infer[{\with}\text{\scshape i}]{A \dn \vdash \with*[sub=_{\ell \in L}]{\ell:B_{\ell}} \up}{
%     \multipremise{\ell \in L}{\!\!\!A \dn \vdash B_{\ell} \up}}
%   \and
%   \infer[{\with}\text{\scshape e}]{A \dn \vdash B_{\kay} \dn}{
%     A \dn \vdash \with*[sub=_{\ell \in L}]{\ell:B_{\ell}} \dn & \text{($\kay \in L$)}}
% \end{inferences}

\section{Other related work}\label{sec:singleton-logic:related-work}

The singleton sequent calculus (in its infinitary variant) appears independently in work by \citeauthor{Fortier+Santocanale:CSL13} on cut elimination for circular proofs of inductive and coinductive types.\autocites{Santocanale:FOSSACS02}{Fortier+Santocanale:CSL13}
They seem to have arrived at the single-antecedent restriction from category-theoretic semantic considerations.
\Citeauthor{Fortier+Santocanale:CSL13} do not develop a semi-axiomatic sequent calculus for singleton logic; that is a contribution of this work.


% CSL '12 paper on asynchronous Curry--Howard for linear logic\fixnote{Belongs in next chapter}



\subsection{Connections to Basic Logic}

\Textcite{Sambin+:JSL00} propose a system called Basic Logic in which connectives are defined by a single \emph{definitional equation}.
If we apply their ideas to ordered logic, for example, the definitional equation for alternative conjunction would be
\begin{equation*}
  \infer=[\with]{\oseq{\octx |- A \with B}}{
    \oseq{\octx |- A} & \oseq{\octx |- B}}
\end{equation*}
Read top-down, the rule is a \emph{formation} rule; read bottom-up, the rule is two \emph{implicit reflection} rules:
\begin{equation*}
  \infer[\jrule{$\with$F}]{\oseq{\octx |- A \with B}}{
    \oseq{\octx |- A} & \oseq{\octx |- B}}
  \qquad
  \infer[\jrule{$\with$IR}_1]{\oseq{\octx |- A}}{
    \oseq{\octx |- A \with B}}
  \qquad
  \infer[\jrule{$\with$IR}_2]{\oseq{\octx |- B}}{
    \oseq{\octx |- A \with B}}
\end{equation*}
The formation rule for alternative conjunction is the same as its usual sequent calculus right rule, and the implicit reflection rules correspond to natural deduction elimination rules for $\with$.

To arrive at the usual sequent calculus left rules for alternative conjunction, \citeauthor{Sambin+:JSL00} proceed by way of what they call \emph{axioms of reflection}.
They obtain these axioms by trivializing the implicit reflection rules:
\begin{equation*}
  \infer[\jrule{$\with$A}_1]{\oseq{A \with B |- A}}{}
  \qquad
  \infer[\jrule{$\with$A}_2]{\oseq{A \with B |- B}}{}
\end{equation*}
Combining these axioms with $\jrule{CUT}$, they arrive at \emph{explicit reflection} rules:%
\marginnote[0.25\baselineskip]{%
  \begin{equation*}
    \infer[\jrule{CUT}^A]{\oseq{\octx'_L \oc (A \with B) \oc \octx'_R |- C}}{
      \infer[\jrule{$\with$A}_1]{\oseq{A \with B |- A}}{} &
      \oseq{\octx'_L \oc A \oc \octx'_R |- C}}
  \end{equation*}
}
\begin{equation*}
  \infer[\jrule{$\with$ER}_1]{\oseq{\octx'_L \oc (A \with B) \oc \octx'_R |- C}}{
    \oseq{\octx'_L \oc A \oc \octx'_R |- C}}
  \qquad
  \infer[\jrule{$\with$ER}_2]{\oseq{\octx'_L \oc (A \with B) \oc \octx'_R |- C}}{
    \oseq{\octx'_L \oc B \oc \octx'_R |- C}}
\end{equation*}
Alternatively, the implicit reflection rules could be obtained from the explicit reflection rules by trivializing the explicit rules and then combining the resulting axioms with $\jrule{CUT}$.%
\marginnote[-1.25\baselineskip]{%
  \begin{equation*}
    \infer[\jrule{CUT}^{A \with B}]{\oseq{\octx |- A}}{
      \oseq{\octx |- A \with B} &
      \infer[\jrule{$\with$A}_1]{\oseq{A \with B |- A}}{}}
  \end{equation*}
}
So, in fact, the implicit and explicit reflection rules and axioms are all equivalent in the presence of $\jrule{CUT}$.

These axioms of reflection are exactly the axioms that we use in our semi-axiomatic sequent calculus in place of the usual left rules.
(They are also the same as the decomposition rules for $\with$ from the refactored ordered sequent calculus~\parencref{fig:ordered-rewriting:decompose-seq-calc} of \cref{ch:ordered-rewriting}.)
Their process of obtaining the explicit reflection rules by combining the axioms of reflection with $\jrule{CUT}$ matches the way that we derive the usual left rules in our semi-axiomatic sequent calculus.
However, \citeauthor{Sambin+:JSL00} appear to have considered the axioms of reflection only as the means to an end, and do not appear to have considered a calculus with these axioms as primitive rules, nor a cut elimination procedure involving the axioms.

% \newthought{As another example} of this process, consider the definitional equation for ordered conjunction:
% \begin{equation*}
%   \infer=[\fuse]{\oseq{\octx'_L \oc (A \fuse B) \oc \octx'_R |- C}}{
%     \oseq{\octx'_L \oc A \oc B \oc \octx'_R |- C}}
% \end{equation*}
% The formation and implicit reflection rules are:
% \begin{equation*}
%   \infer[\jrule{$\fuse$F}]{\oseq{\octx'_L \oc (A \fuse B) \oc \octx'_R |- C}}{
%     \oseq{\octx'_L \oc A \oc B \oc \octx'_R |- C}}
%   \qquad
%   \infer[\jrule{$\fuse$IR}]{\oseq{\octx'_L \oc A \oc B \oc \octx'_R |- C}}{
%     \oseq{\octx'_L \oc (A \fuse B) \oc \octx'_R |- C}}
% \end{equation*}
% By trivializing the implicit reflection rule, we obtain the axiom
% \begin{equation*}
%   \infer[\jrule{$\fuse$A}]{\oseq{A \oc B |- A \fuse B}}{}
% \end{equation*}
% Combining this axiom with $\jrule{CUT}$, we arrive at the explicit reflection rule%
% \marginnote{%
%   \begin{equation*}
%       \infer[\jrule{CUT}^A]{\oseq{\octx_1 \oc \octx_2 |- A \fuse B}}{
%     \oseq{\octx_1 |- A} &
%     \infer[\jrule{CUT}^B]{\oseq{A \oc \octx_2 |- A \fuse B}}{
%       \oseq{\octx_2 |- B} & \infer[\jrule{$\fuse$A}]{\oseq{A \oc B |- A \fuse B}}{}}}
%   \end{equation*}
% }
% \begin{equation*}
%   \infer[\jrule{$\fuse$ER}]{\oseq{\octx_1 \oc \octx_2 |- A \fuse B}}{
%     \oseq{\octx_1 |- A} & \oseq{\octx_2 |- B}}
% \end{equation*}
% In the other direction, the axiom can also be obtained by trivializing the explicit reflection rule; the implicit reflection rules is then obtained from $\jrule{CUT}$.%
% \marginnote{%
%   \begin{equation*}
%     \infer[\jrule{CUT}^{A \fuse B}]{\oseq{\octx'_L \oc A \oc B \oc \octx'_R |- C}}{
%       \infer[\jrule{$\fuse$A}]{\oseq{A \oc B |- A \fuse B}}{} &
%       \oseq{\octx'_L \oc (A \fuse B) \oc \octx'_R |- C}}
%   \end{equation*}
% }

% \newthought{As a final example} of this process, consider the definitional equation for left-handed implication:
% \begin{equation*}
%   \infer=[\limp]{\oseq{\octx |- A \limp B}}{
%     \oseq{A \oc \octx |- B}}
% \end{equation*}
% The formation and implicit reflection rules are:
% \begin{equation*}
%   \infer[\jrule{$\limp$F}]{\oseq{\octx |- A \limp B}}{
%     \oseq{A \oc \octx |- B}}
%   \qquad
%   \infer[\jrule{$\limp$IR}]{\oseq{A \oc \octx |- B}}{
%     \oseq{\octx |- A \limp B}}
% \end{equation*}
% By trivializing the implicit reflection rule, we obtain the axiom
% \begin{equation*}
%   \infer[\jrule{$\limp$A}]{\oseq{A \oc (A \limp B) |- B}}{}
% \end{equation*}
% Combining this axiom with $\jrule{CUT}$, we arrive at the explicit reflection rule%
% \marginnote{%
%   \begin{equation*}
%     \infer[\jrule{CUT}^A]{\oseq{\octx'_L \oc \octx \oc (A \limp B) \oc \octx'_R |- C}}{
%       \oseq{\octx |- A} &
%       \infer[\jrule{CUT}^B]{\oseq{\octx'_L \oc A \oc (A \limp B) \oc \octx'_R |- C}}{
%         \infer[\jrule{$\limp$A}]{\oseq{A \oc (A \limp B) |- B}}{} &
%         \oseq{\octx'_L \oc B \oc \octx'_R |- C}}}
%   \end{equation*}
% }
% \begin{equation*}
%   \infer[\jrule{$\limp$ER}]{\oseq{\octx'_L \oc \octx \oc (A \limp B) \oc \octx'_R |- C}}{
%     \oseq{\octx |- A} &
%     \oseq{\octx'_L \oc B \oc \octx'_R |- C}}
% \end{equation*}
% In the other direction, the axiom can also be obtained by trivializing the explicit reflection rule; the implicit reflection rules is then obtained from $\jrule{CUT}$.%
% \marginnote{%
%   \begin{equation*}
%     \infer[\jrule{CUT}^{A \limp B}]{\oseq{A \oc \octx |- B}}{
%       \oseq{\octx |- A \limp B} &
%       \infer[\jrule{$\limp$A}]{\oseq{A \oc (A \limp B) |- B}}{}}
%   \end{equation*}
% }
  
%%% Local Variables:
%%% mode: latex
%%% TeX-master: "thesis"
%%% End:

\chapter{A computational interpretation of the semi-axiomatic\\singleton sequent calculus as session-typed communicating chains}\label{ch:process-chains}

In the previous \lcnamecref{ch:singleton-logic}, we took a purely proof-theoretic view of singleton logic and its semi-axiomatic sequent calculus.
The proof terms assigned to semi-axiomatic sequent proofs were simply syntactic objects,
% and the proof of the admissibility of non-analytic cuts\parencref{lem:singleton-logic:hilbert:cut-admissible} described a meta-level function for manipulating these syntactic objects.
and the proof of non-analytic cut elimination~\parencref{thm:singleton:hilbert:cutelim} described a meta-level function for normalizing these syntactic objects.

Even in a purely proof-theoretic setting, however, the computational suggestions of these syntactic manipulations were too strong to ignore:
% In proving the admissibility of non-analytic cuts\parencref{lem:singleton-logic:hilbert:cut-admissible},
We saw that the principal cases in the proof of admissibility of non-analytic cuts~\parencref{lem:singleton-logic:hilbert:cut-admissibility}%
\marginnote{%
  \vspace*{-\abovedisplayskip}
  \begin{align*}
    \nspawn{\selectR{\kay}}{\caseL[\ell \in L]{\ell => M_{\ell}}}
      &= M_{\kay}
    \\
    \nspawn{\caseR[\ell \in L]{\ell => N_{\ell}}}{\selectL{\kay}}
      &= N_{\kay}
  \end{align*}
% \end{marginfigure}
}
are reminiscent of asynchronous message-passing communication.
Following the rich tradition of Curry--Howard isomorphisms between logics and computational systems, this \lcnamecref{ch:process-chains} therefore pursues a concurrent computational interpretation of the semi-axiomatic sequent calculus for singleton logic.

In particular, we will see that singleton propositions can be interpreted as session types that describe patterns of interprocess communication~\parencref{sec:process-chains:typed-processes}; semi-axiomatic sequent proofs can be interpreted as chains of session-typed processes~\parencref{sec:process-chains:typed-chains}; and cut reduction can be interpreted as asynchronous message-passing communication~\parencref{sec:process-chains:reductions}.
% \footnote{As we will see in \cref{??}, Hilbert-style proofs may also be viewed as well-behaved chains of the communicating automata familiar from \cref{??}.}
For instance, a proof of $\plus*[sub=_{\ell \in L}]{\ell:A_{\ell}}$ corresponds to a process that sends a message carrying some label $\kay \in L$ and then continues communicating according to pattern $A_{\kay}$.

This roughly parallels a recent line of research into a Curry--Howard isomorphism, dubbed \acs{SILL}, between the intuitionistic linear sequent calculus and session-typed concurrent computation\autocites{Caires+:MSCS16}{Caires+:TLDI12} -- with two key differences.
First, unlike \ac{SILL}, we use singleton logic, not intuitionistic linear logic.
This restricts the process topology to chains, rather than the tree topology that \ac{SILL} permits.
But this restriciton should not be seen as a weakness;
by leveraging this restriction, computation pipelines can be expressed as a logically motivated fragment.
% However, as sketched in \cref{??}, 

Second and most importantly, we use the \emph{semi-axiomatic} sequent calculus, rather than a standard sequent calculus like \ac{SILL} does.
% Third, and most importantly,
The use of semi-axiomatic sequent proofs enables a clean and direct interpretation of cut reductions as asynchronous communication, unlike the cut-reductions-as-syn\-chro\-nous-communication view espoused by \ac{SILL}.%
\footnote{It is possible to give a rather ad hoc treatment of asynchronous communication using \ac{SILL}'s sequent proofs~\parencite{DeYoung+:CSL12}, but, in our view, the treatment of asynchronous communication using semi-axiomatic sequent proofs is far more elegant.}


% In particular, this parallels a recent line of research into a Curry--Howard isomorphism between the intuitionistic linear sequent calculus and session-typed concurrent computation\autocites{Caires+:MSCS16}{Caires+:TLDI12}.
% In that work, linear propositions are interpreted as session types that describe patterns of interprocess communication; sequent proofs, as session-typed processes; and cut reduction, as synchronous message-passing communication.
% For instance, a proof of $A \lolli B$ corresponds to a process that inputs a channel that communicates according to the pattern $A$, and then continues connumicating according to pattern $B$.


% We will see that 
% propositions can be interpreted as session types that specify patterns of communication; Hilbert-style proofs, as chains of session-typed processes; and cut reduction, as asynchronous message-passing communication.

% Alternatively, Hilbert-style proofs can be seen as well-behaved chains of communicating automata from \cref{??}.


\newthought{We begin}, in \cref{sec:process-chains:untyped-chains}, by introducing process chains in their untyped form as finite sequences of processes arranged in a linear topoloogy.
Then, in \cref{sec:process-chains:typed-processes}, we describe the structure of well-typed process expressions that may be used in these chains, and show that the session-typing rules for process expressions correspond to the rules of the semi-axiomatic sequent calculus;
\cref{sec:process-chains:typed-chains} presents session-typing rules for the process chains themselves.
In \cref{sec:process-chains:reductions}, we assign an operational semantics to process chains; this operational semantics arises naturally from the semi-axiomatic proof normalization procedure given in the previous \lcnamecref{ch:singleton-logic}.
Lastly, \cref{sec:process-chains:coinductive} describes coinductively defined type and process definitions.


\section{Process chains and process expressions}\label{sec:process-chains:interpretation}

\subsection{Untyped process chains}\label{sec:process-chains:untyped-chains}

We envision a process chain, $\chn$, as a (possibly empty) finite sequence of processes $(P_i)_{i=1}^{n}$, each with its own independent thread of control and arranged in a linear topology.
As depicted in the adjacent \lcnamecref{fig:singleton-processes:chain-topology},%
%
\begin{marginfigure}
  \centering
  \begin{tikzcd}[
    ampersand replacement = \&,
    execute at end picture = {
      \node [fit = (P_1) (P_i) (P_n), inner xsep = .5em,
             draw = gray,
             label distance = 2em, label = {[gray]$\chn$}]
        {};
    },
  ]
    {} \rar[dash] \&
    |[circle, draw, alias=P_1]| P_1 \rar[dash][description, text height=1ex]{\displaystyle\dotsb} \&[1.2em]
    |[circle, draw, alias=P_i]| P_i \rar[dash][description, text height=1ex]{\displaystyle\dotsb} \&[1.2em]
    |[circle, draw, alias=P_n]| P_{\mathrlap{n}\phantom{i}} \rar[dash] \&
    {}
  \end{tikzcd}
  % \begin{tikzpicture}
  %   \graph [math nodes, nodes={circle, draw}] {
  %     P_0 / [coordinate]
  %      --
  %     P_1
  %      --
  %     / \dotsb [rectangle, text height=1ex, draw=none]
  %      --
  %     P_i
  %      --
  %     / \dotsb [rectangle, text height=1ex, draw=none]
  %      --
  %     P_n
  %      --
  %     / [coordinate];
  %   };
  %   \node [fit=(P_1) (P_i) (P_n), inner xsep=.5em,
  %          draw=gray,
  %          label distance=2em, label={[gray]$\chn$}] {};
  % \end{tikzpicture}
  \caption{A prototypical process chain, $\chn$}\label{fig:singleton-processes:chain-topology}
\end{marginfigure}
%
% each process $P_i$ shares a unique channel with its left-hand neighbor and a unique channel with its right-hand neighbor.
each process $P_i$ shares unique channels with its left- and right-hand neighbors.
Along these channels, neighboring processes may interact and react, changing their own internal state.
Because process chains always maintain a linear topology, the channels need not be named -- they can instead be referred to as simply the left- and right-hand channels of $P_i$.

A chain $\chn$ does not compute in isolation, however.
The left-hand channel of $P_1$ and the right-hand channel of $P_n$ enable the chain to interact with its surroundings.
Because these two channels are the only ones exposed to the external environment, they may be referred to as the left- and right-hand channels of $\chn$.

Chains may even be composed end to end by conjoining the right-hand channel of one chain with the left-hand channel of another.

As finite sequences of processes $P_i$, chains form a free monoid:
\begin{equation*}
  \chn,\chn* \Coloneqq (\chn_1 \cc \chn_2) \mid (\chne) \mid P
  \,,
\end{equation*}
where $\cc$ denotes end-to-end composition of chains and $(\chne)$ denotes the empty chain.
As a monoid, chains are subject to associativity and unit laws (see adjacent \lcnamecref{fig:chains:monoid-laws}).
\begin{marginfigure}
\begin{gather*}
  (\chn_1 \cc \chn_2) \cc \chn_3 = \chn_1 \cc (\chn_2 \cc \chn_3) \\
  (\chne) \cc \chn = \chn = \chn \cc (\chne)
\end{gather*}
\caption{Monoid laws for process chains}\label{fig:chains:monoid-laws}
\end{marginfigure}
We do not distinguish chains that are equivalent up to these laws, instead treating such chains as syntactically identical.

The notation for a composition $\chn_1 \cc \chn_2$ is intended to recall parallel composition of $\pi$-calculus processes, $P_1 \mid P_2$.
However, unlike $\pi$-calculus composition, parallel composition of chains is \emph{not} commutative because the sequential order of processes within a chain matters.


\subsection{Session-typed process expressions}\label{sec:process-chains:typed-processes}

Thus far, we have examined the overall structure of (untyped) process chains without detailing the internal structure of individual processes.
% Thus far, we have diligently avoided describing the specific internals of processes.
We now turn to the specifics of well-typed processes.

As previously alluded, each of a chain's processes constitutes its own, independent thread of control dedicated to executing the instructions described by a process expression $P$.
Processes are thus dynamic realizations of the static process expressions, in the same way that executables run source code.\footnote{It is occasionally convenient to blur this distinction, so we sometimes abuse terminology and refer to a process expression $P$ as a \enquote{process}.}

% To follow a Curry--Howard isomorphism, we will adopt the propositions as session types; Hilbert-style proof terms as process expressions; and Hilbert system's inference rules as session-typing rules.

% To describe the patterns by which a process $P$ communicates with its left- and right-hand neighbors, % we will use a session-typing judgment
\newthought{Reinterpreting the proof-term judgment} of singleton logic's semi-axiomatic sequent calculus, we arrive at a session-typing judgment for process expressions.
The judgment
\begin{equation*}
  \slof{A |- P : B} % \,,
\end{equation*}
now means that $P$ is the expression for a process that
% that the process $P$
offers, along its right-hand channel, the service described by the session type $B$, while concurrently using, along its left-hand channel, the service described by the type $A$.
In other words, the right-hand neighbor acts as a client of service $B$ from $P$, while the left-hand neighbor of process $P$ acts as a provider of service $A$ to $P$.

Session types describe the patterns by which a process is permitted to communicate with its left- and right-hand neighbors.

Under this reinterpretation of the basic judgment, the proof rules of the singleton Hilbert system become session-typing rules for process expressions.
Specifically, the right rules define what it means for a provider to offer a particular service, while the left rules show how a client may use that service.


\begin{table}[tbp]
  \aboverulesep=1.25ex
  \belowrulesep=1.25ex
  \renewcommand{\arraystretch}{1.2}
  \centering
  \begin{tabular}{@{}lll@{}}
    \toprule
    % \quad & \emph{Process expressions} & \emph{Description}
    % \\ \midrule
    \rlap{\emph{Judgmental rules}} \\
      & $\spawn{P_1}{P_2}$
          & \renewcommand{\arraystretch}{0.95}%
            \begin{tabular}[t]{@{}l@{}}
              Spawn new, neighboring threads of control for $P_1$ and\\
              \quad $P_2$, then terminate the current thread of control
            \end{tabular} \\
      & $\fwd$ & Terminate the current thread of control
    \\[.5ex] % \midrule
    \rlap{\emph{Internal choice, $\plus*[sub=_{\ell \in L}]{\ell:A_{\ell}}$}} \\
      & $\selectR{\kay}$, with $\kay \in L$
          & A message to the right-hand client, carrying label $\kay$ \\
      & $\caseL[\ell \in L]{\ell => P_{\ell}}$
          & \renewcommand{\arraystretch}{0.95}%
            \begin{tabular}[t]{@{}l@{}}
              Await a message $\selectR{\kay}$ from the left-hand provider,\\
              \quad then continue as $P_{\kay}$
            \end{tabular}
    \\[5ex] % \midrule
    \rlap{\emph{External choice, $\with*[sub=_{\ell \in L}]{\ell:A_{\ell}}$}} \\
      & $\caseR[\ell \in L]{\ell => P_{\ell}}$
          & \renewcommand{\arraystretch}{0.95}%
            \begin{tabular}[t]{@{}l@{}}
              Await a message $\selectL{\kay}$ from the right-hand client,\\
              \quad then continue as $P_{\kay}$
            \end{tabular} \\
      & $\selectL{\kay}$, with $\kay \in L$ & A message to the left-hand provider, carrying label $\kay$
    \\ \bottomrule    
  \end{tabular}
  \caption{Singleton session types}\label{tab:singleton-processes:types}%
\end{table}

% \begin{table*}[tbp]
%   \centering
%   \begin{tabular}{lll}
%     \toprule
%     \emph{Type} & \emph{Process expressions} & \emph{Description}
%     \\ \midrule
%     -- & $\spawn{P_1}{P_2}$ & Spawn new, neighboring threads of control for $P_1$ and $P_2$
%     \\
%     -- & $\fwd$ & Terminate the current thread of control
%     \\
%     $\plus*[sub=_{\ell \in L}]{\ell:A_{\ell}}$
%       & $\selectR{\kay}$ & message \\
%       & $\caseL[\ell \in L]{\ell => P_{\ell}}$ & Await a message $\selectR{\kay}$ from the left
%     \\
%     $\with*[sub=_{\ell \in L}]{\ell:A_{\ell}}$
%       & $\caseR[\ell \in L]{\ell => P_{\ell}}$ & Await a message $\selectL{\kay}$ from the right \\
%       & $\selectL{\kay}$ & message
%     \\ \bottomrule    
%   \end{tabular}
%   \caption{Singleton session types}\label{tab:singleton-processes:types}
% \end{table*}


% As an example, consider additive disjunction and its proof rules.
% From a computational perspective, additive disjunction is interpreted as \emph{internal choice}.
% The service ... is 
% The internal choice $\plus*[sub=_{\ell \in L}]{\ell:A_{\ell}}$ is the service in which the provider chooses which one of the services $(A_{\ell})_{\ell \in L}$ it will offer its client.

% type of a process that sends some label $\kay \in L$ to its right-hand client and then behaves like $A_{\kay}$.

As an example, consider additive disjunction and its proof rules.
% Consider the rules involving additive disjunction, for example.
From a computational perspective, an additive disjunction $\plus*[sub=_{\ell \in L}]{\ell:A_{\ell}}$ is interpreted as an internal choice, the type of a process that sends some label $\kay \in L$ to its right-hand client and then behaves like $A_{\kay}$.
Recall the $\rrule{\plus}'$ and $\lrule{\plus}$ rules:
% The $\rrule{\plus}'$ and $\lrule{\plus}$ proof rules become session-typing rules for process expressions.
\begin{inferences}
  \infer[\rrule{\plus}']{\slof{A_{\kay} |- \selectR{\kay} : \plus*[sub=_{\ell \in L}]{\ell:A_{\ell}}}}{
    \text{($\kay \in L$)}}
  \and
  \infer[\lrule{\plus}]{\slof{\plus*[sub=_{\ell \in L}]{\ell:A_{\ell}} |- \caseL[\ell \in L]{\ell => P_{\ell}} : C}}{
    \multipremise{\ell \in L}{\slof{A_{\ell} |- P_{\ell} : C}}}
\end{inferences}
The proof term $\selectR{\kay}$ is now reinterpreted as the expression for a message, sent to the right-hand client (as the arrow suggests), that carries the label $\kay$ as its payload.
The proof term $\caseL[\ell \in L]{\ell => P_{\ell}}$ is reinterpreted as the expression for a client process that awaits a message $\selectR{\kay}$ from its left-hand provider and then continues the thread of control with the corresponding branch, $P_{\kay}$.

% Dually, additive conjunction becomes a form of external choice.
% A provider of service $\with*[sub=_{\ell \in L}]{\ell:A_{\ell}}$ offers its client its choice of services $(A_{\ell})_{\ell \in L}$.
% \begin{inferences}
%   \infer[\rrule{\with}]{\slof{A |- \caseR[\ell \in L]{\ell => P_{\ell}} : \with*[sub=_{\ell \in L}]{\ell: C_{\ell}}}}{
%     \multipremise{\ell \in L}{\slof{A |- P_{\ell} : C_{\ell}}}}
%   \and
%   \infer[\lrule{\with}']{\slof{\with*[sub=_{\ell \in L}]{\ell:C_{\ell}} |- \selectL{\kay} : C_{\kay}}}{
%     \text{($\kay \in L$)}}
% \end{inferences}
% The client is a message $\selectL{\kay}$ that uses its payload [of label $k \in L$] to indicate the client's choice.
% The provider, $\caseR[\ell \in L]{\ell => P_{\ell}}$, is an input process that awaits a message indicating the client's choice and then continues along the chosen branch.

Additive conjunction, $\with*[sub=_{\ell \in L}]{\ell:A_{\ell}}$, is interpreted dually as external choice, the type of a process that awaits a label $\kay \in L$ from its client and then behaves like $A_{\kay}$.
\begin{inferences}
  \infer[\rrule{\with}]{\slof{A |- \caseR[\ell \in L]{\ell => P_{\ell}} : \with*[sub=_{\ell \in L}]{\ell: C_{\ell}}}}{
    \multipremise{\ell \in L}{\slof{A |- P_{\ell} : C_{\ell}}}}
  \and
  \infer[\lrule{\with}']{\slof{\with*[sub=_{\ell \in L}]{\ell:C_{\ell}} |- \selectL{\kay} : C_{\kay}}}{
    \text{($\kay \in L$)}}
\end{inferences}
As might be expected, the proof terms $\caseR[\ell \in L]{\ell => P_{\ell}}$ and $\selectL{\kay}$ are interpreted symmetrically to internal choice's $\selectR{\kay}$ and $\caseL[\ell \in L]{\ell => P_{\ell}}$ expressions: $\selectL{\kay}$ is a message to the left-hand provider, and $\caseR[\ell \in L]{\ell => P_{\ell}}$ is an input process that awaits a message from the right-hand client.

The proof term $\spawn{P_1}{P_2}$ for composition of proofs is now reinterpreted as the expression for a process that will spawn new, neighboring threads of control for $P_1$ and $P_2$ and then terminate the original thread of control.
In effect, $\spawn{P_1}{P_2}$ now composes process expressions.
\begin{equation*}
  \infer[\jrule{CUT}^B]{\slof{A |- \spawn{P_1}{P_2} : C}}{
    \slof{A |- P_1 : B} & \slof{B |- P_2 : C}}
\end{equation*}
For $\spawn{P_1}{P_2}$ to be a well-typed composition, $P_1$ must offer the same service that $P_2$ uses.

% Reflecting the intuition 
% \begin{equation*}
%   \infer{\spawn{P_1}{P_2} \reduces P_1 \cc P_2}{}
% \end{equation*}

Proof-theoretically, the identity and cut rules are inverses, so we should expect their process interpretations to be similarly inverse.
The process expression $\spawn{P_1}{P_2}$ spawns threads of control, so $\fwd$, as its inverse, terminates the thread of control.
\begin{equation*}
  \infer[\jrule{ID}^A]{\slof{A |- \fwd : A}}{}
\end{equation*}
% \begin{equation*}
%   \infer{\fwd \reduces \chne}{}
% \end{equation*}


% Additive conjunction, $\with*[sub=_{\ell \in L}]{\ell:A_{\ell}}$, is interpreted dually as external choice.
% \begin{inferences}
%   \infer[\rrule{\with}]{\slof{A |- \caseR[\ell \in L]{\ell => P_{\ell}} : \with*[sub=_{\ell \in L}]{\ell:C_{\ell}}}}{
%     \multipremise{\ell \in L}{\slof{A |- P_{\ell} : C_{\ell}}}}
%   \and
%   \infer[\lrule{\with}']{\slof{\with*[sub=_{\ell \in L}]{\ell:C_{\ell}} |- \selectL{\kay} : C_{\kay}}}{
%     \text{($\kay \in L$)}}
% \end{inferences}
% % \begin{equation*}
% %   \infer{\caseR[\ell \in L]{\ell => P_{\ell}} \cc \selectL{\kay} \reduces P_{\kay}}{
% %     \text{($\kay \in L$)}}
% % \end{equation*}



% Discussion of operational semantics here?

% Interpret additive disjunction as internal choice, and $\selectR{\kay}$ as a message, and $\caseL[\ell \in L]{\ell => P_{\ell}}$ as a process that waits for a message...

% Interpret additive conjunction dually...



% \begin{marginfigure}
%   \begin{tikzpicture}
%     \graph [math nodes, nodes={draw, circle}] {
%       / [coordinate]
%        --
%       k / {\,\smash[b]{\selectR{\kay}}\!}
%        --
%       P / {\caseL[\ell \in L]{\ell => P_{\ell}}} [rounded rectangle]
%        --
%       / [coordinate];
%     };
%   \end{tikzpicture}
% \end{marginfigure}

% Process expressions, $P$, and their session-typing rules are isomorphic to the Hilbert-style proof terms and inference rules of \cref{??}.
% [Propositions are reinterpreted as session types.]


% Additive disjunction, $\plus*[sub=_{\ell \in L}]{\ell:A_{\ell}}$, is interpreted as internal choice, the type of a process that sends a label.
% \begin{inferences}
%   \infer[\rrule{\plus}']{\slof{A_{\kay} |- \selectR{\kay} : \plus*[sub=_{\ell \in L}]{\ell:A_{\ell}}}}{
%     \text{($\kay \in L$)}}
%   \and
%   \infer[\lrule{\plus}]{\slof{\plus*[sub=_{\ell \in L}]{\ell:A_{\ell}} |- \caseL[\ell \in L]{\ell => P_{\ell}} : C}}{
%     \multipremise{\ell \in L}{\slof{A_{\ell} |- P_{\ell} : C}}}
% \end{inferences}
% The proof term $\selectR{\kay}$ is now viewed as a message, sent to the right-hand neighbor (as the arrow suggests), that carries the label $\kay$ as its payload.
% % \begin{equation*}
% %   \infer{\selectR{\kay} \cc \caseL[\ell \in L]{\ell => P_{\ell}} \reduces P_{\kay}}{
% %     \text{($\kay \in L$)}}
% % \end{equation*}
% % Clearly reminiscent of the principal cut reduction $\nspawn{\selectR{\kay}}{\caseL[\ell \in L]{\ell => N_{\ell}}} = N_{\kay}$.

\subsection{Session-typed process chains}\label{sec:process-chains:typed-chains}

With the session-typing system for process expressions in hand, session types can be assigned to process chains, too.
We use a session-typing judgment
\begin{equation*}
  \slcof{A |- \chn : B} \,,
\end{equation*}
%
\begin{marginfigure}[-4\baselineskip]
  \centering
  \begin{tikzpicture}
    \graph [math nodes, nodes={draw}] {
      / [coordinate]
       -- ["$A$"]
      P_1 / \phantom{P} [circle] 
       --
      / \dotsb [text height=1ex, text width=1.375em, draw=none]
       --
      P_n / \phantom{P} [circle]
       -- ["$B$"]
      / [coordinate];
    };

    \node [fit=(P_1) (P_n), draw=gray, label=$\chn$] {};
  \end{tikzpicture}
  % \caption{A well-typed process chain that offers service $B$ to its right-hand client, while using service $A$ from its left-hand provider}
  \caption{A well-typed process chain that uses service $A$ to offer service $B$}\label{fig:singleton-processes:well-typed-chain}
\end{marginfigure}%
%
meaning that the chain $\chn$ offers, along its right-hand channel, the service $B$, while concurrently using, along its left-hand channel, the service $A$.
Similar to individual processes, a chain $\chn$ thus enjoys client and provider relationships with its left- and right-hand environments, respectively.

The simplest session-typing rule for chains is the one that types a chain consisting of a single running process $P$:
\begin{equation*}
  \infer[\jrule{C-PROC}]{\slcof{A |- P : B}}{
    \slof{A |- P : B}}
\end{equation*}
%
\begin{marginfigure}[-3\baselineskip]
  \centering
  \begin{tikzpicture}
    \graph [math nodes, nodes={draw}] {
      / [coordinate]
       -- ["$A$"]
      P [circle]
       -- ["$B$"]
      / [coordinate];
    };
 
    \node [fit=(P), draw=gray] {};
  \end{tikzpicture}
  \caption{A chain made of one well-typed process that uses service $A$ to offer service $B$}\label{fig:singleton-processes:single-process-chain}
\end{marginfigure}%
%
In other words, a running process has the same type as its underlying process expression.


The session-typing rule for the empty chain, $(\chne)$, is also fairly direct.
The empty chain offers a service $A$ to its right-hand client by using the service of its left-hand provider:
\begin{equation*}
  \infer[\jrule{C-ID}\smash{^A}]{\slcof{A |- \chne : A}}{}
\end{equation*}
%
\begin{marginfigure}
  \centering
  \begin{tikzpicture}
    \graph [math nodes, nodes={draw}] {
      / [coordinate]
       -- ["$A$"]
      P_1 / \phantom{P} [circle, draw=none]
       -- ["$A$"]
      / [coordinate];

      (P_1.west) -- (P_1.east);
    };
 
    \node [fit=(P_1), draw=gray] {};
  \end{tikzpicture}
  \caption{A well-typed empty chain that uses service $A$ to offer service $A$}\label{fig:singleton-processes:empty-chain}
\end{marginfigure}%
%
Save for the contrasting $\slcof{|-}$ turnstile and the empty chain in place of a forwarding process expression, this mirrors the $\jrule{ID}\smash{^A}$, the identity rule for process expressions.
We will see shortly that this is not a coincidence.

Finally, a parallel composition of chains, $\chn_1 \cc \chn_2$, is well-typed only if $\chn_1$ offers the same service that $\chn_2$ uses, otherwise communication between $\chn_1$ and $\chn_2$ would be mismatched.
This condition is reflected in a cut principle for the session-typing judgment:
\begin{equation*}
  \infer[\jrule{C-CUT}\smash{^B}]{\slcof{A |- \chn_1 \cc \chn_2 : C}}{
    \slcof{A |- \chn_1 : B} & \slcof{B |- \chn_2 : C}}
\end{equation*}
%
\begin{marginfigure}[-4\baselineskip]
  \centering
  % \begin{tikzpicture}
  %   \graph [math nodes, nodes={draw}] {
  %     / [coordinate]
  %      -- ["$A$"]
  %     P_1 / [circle] 
  %      --
  %     / \dotsb [text height=1ex, text width=1.375em, draw=none]
  %      --
  %     P_n / [circle]
  %      -- ["$B$"]
  %     Q_1 / [circle] 
  %      --
  %     / \dotsb [text height=1ex, text width=1.375em, draw=none]
  %      --
  %     Q_n / [circle]
  %      -- ["$C$"]
  %     / [coordinate];
  %   };
  %
  %   \node [fit=(P_1) (P_n), draw, label=$\chn_1$] {};
  %   \node [fit=(Q_1) (Q_n), draw, label=$\chn_2$] {};
  % \end{tikzpicture}
  % \vspace{\baselineskip}
  %
  \begin{tikzpicture}
    \graph [math nodes, nodes={draw}] {
      / [coordinate]
       -- ["$A$"]
      C_1 / \chn_1
       -- ["$B$" {name=B}]
      C_2 / \chn_2
       -- ["$C$"]
      / [coordinate];
    };

    \node [fit=(C_1) (C_2) (C_1.north), draw=gray, label={[gray]above:$\chn_1 \cc \chn_2$}] {};
  \end{tikzpicture}
  \caption{A well-typed process chain that uses service $A$ to offer service $B$}\label{fig:singleton-processes:well-typed-cut}
\end{marginfigure}%
%
% The notation $\chn_1 \cc \chn_2$ is intended to recall parallel composition of processes, $P_1 \mid P_2$, in the $\pi$-calculus.
% However, unlike $\pi$-calculus composition, parallel composition of chains is \emph{not} commutative because the sequential order of processes within a chain matters.
%
Once again, there are strong similarities to a process expression -- $\spawn{P_1}{P_2}$ and its $\jrule{CUT}^B$ session-typing rule, in this case.
We can make these similarities explicit by defining a homomorphism, $\pf*{-}$, from chains to process expressions:
\begin{marginfigure}
  \begin{align*}
    \pf*{\chn_1 \cc \chn_2} &= \spawn{\pf{\chn_1}}{\pf{\chn_2}} \\
    \pf*{\chne} &= \fwd \\
    \pf{P} &= P
  \end{align*}
  \caption{A homomorphism from chains to process expressions}
\end{marginfigure}%
%
This homomorphism is type-preserving:
\begin{theorem}
  If $\slcof{A |- \chn : B}$, then $\slof{A |- \pf{\chn} : B}$.
\end{theorem}
\begin{proof}
  By structural induction on the session-typing derivation.
\end{proof}


% The empty chain, $\chne$, offers a service $A$ to its right-hand client by directly using the service of its left-hand provider.
% This is reflected in an identity principle:
% \begin{equation*}
%   \infer[\jrule{C-ID}\smash{^A}]{\slcof{A |- \chne : A}}{}
% \end{equation*}
% %
% \begin{marginfigure}
%   \centering
%   \begin{tikzpicture}
%     \graph [math nodes, nodes={draw}] {
%       / [coordinate]
%        -- ["$A$"]
%       P_1 / \phantom{P} [circle, draw=none]
%        -- ["$A$"]
%       / [coordinate];

%       (P_1.west) -- (P_1.east);
%     };
 
%     \node [fit=(P_1), draw=gray] {};
%   \end{tikzpicture}
%   \caption{A well-typed empty chain that uses service $A$ to offer service $A$}\label{fig:singleton-processes:empty-chain}
% \end{marginfigure}%
% %

% Lastly, a chain consisting of a single process, $P$, offers
% \begin{equation*}
%   \infer[\jrule{C-PROC}]{\slcof{A |- P : B}}{
%     \slof{A |- P : B}}
% \end{equation*}
% %
% \begin{marginfigure}
%   \centering
%   \begin{tikzpicture}
%     \graph [math nodes, nodes={draw}] {
%       / [coordinate]
%        -- ["$A$"]
%       P [circle]
%        -- ["$B$"]
%       / [coordinate];
%     };
 
%     \node [fit=(P), draw=gray] {};
%   \end{tikzpicture}
%   \caption{A chain made of one well-typed process that uses service $A$ to offer service $B$}\label{fig:singleton-processes:single-process-chain}
% \end{marginfigure}%


% This judgment is similar in structure to the singleton sequent $\slseq{A |- B}$, but the difference in turnstile


% \begin{marginfigure}
%   \centering
%   \begin{tikzpicture}
%     \graph [math nodes, nodes={draw}] {
%       / [coordinate]
%        -- ["$A$"]
%       C_1 / \chn_1
%        -- ["$B$" name=B]
%       C_2 / \chn_2
%        -- ["$C$"]
%       / [coordinate];
%     };

%     \node [fit=(C_1) (B) (C_2), draw, dashed,
%            label=$\chn_1 \cc \chn_2$]
%       {};
%   \end{tikzpicture}
%   \caption{A prototypical process chain, $\chn$}\label{fig:singleton-processes:chain-topology}
% \end{marginfigure}

At first, the distinction between offering and using a service may seem a bit odd, given that we placed so much emphasis on the symmetry of singleton sequents $\slseq{A |- B}$.
Singleton sequents are indeed symmetric, as \cref{thm:singleton-logic:symmetry} showed.
But imposing a provider--client, offer--use distinction is useful for placing our process chains and expressions within existing conceptual frameworks for session-typed concurrency.
In particular, the distinction helps to relate this process interpretation of singleton logic back to the \ac{SILL} interpretation of linear logic\autocites{Caires+Pfenning:CONCUR10}{Caires+:TLDI12}{Caires+:MSCS16}{Toninho+:ESOP13}.

% but because we view them as (binary) hypothetical judgments, a judgmental asymmetry between the antecedent and consequent remains.
% It is this judgmental asymmetry that is reflected in the provider--client, offer--use asymmetry.


% If a chain $\chn_1$ offers a service $B$ along its right-hand channel and a chain $\chn_2$ uses the same service $B$ along its left-hand channel, then the two chains may be composed end to end as $\chn_1 \cc \chn_2$:
% \begin{equation*}
%   \infer[\jrule{C-CUT}\smash{^B}]{\slcof{A |- \chn_1 \cc \chn_2 : C}}{
%     \slcof{A |- \chn_1 : B} & \slcof{B |- \chn_2 : C}}
% \end{equation*}
% The notation $\chn_1 \cc \chn_2$ for chain composition is intended to suggest that $\chn_1$ and $\chn_2$ execute in parallel.

% If the left-hand surroundings offer service $A$
% \begin{equation*}
%   \infer[\jrule{C-ID}\smash{^A}]{\slcof{A |- \chne : A}}{}
% \end{equation*}


\begin{figure}[tbp]
  \begin{syntax*}
    Session types & A,B,C &
      \alpha \mid \plus*[sub=_{\ell \in L}]{\ell:A_{\ell}}
             \mid \with*[sub=_{\ell \in L}]{\ell:A_{\ell}}
    \\
    Process chains & \chn,\chn* &
      (\chn_1 \cc \chn_2) \mid \chne \mid P
  \end{syntax*}
  \begin{gather*}
    (\chn_1 \cc \chn_2) \cc \chn_3 = \chn_1 \cc (\chn_2 \cc \chn_3) \\
    (\chne) \cc \chn = \chn = \chn \cc (\chne)
  \end{gather*}
  \begin{inferences}
    \infer[\jrule{C-CUT}\smash{^B}]{\slcof{A |- \chn_1 \cc \chn_2 : C}}{
      \slcof{A |- \chn_1 : B} & \slcof{B |- \chn_2 : C}}
    \and
    \infer[\jrule{C-ID}\smash{^A}]{\slof{A |- \chne : A}}{}
    \and
    \infer[\jrule{C-PROC}]{\slcof{A |- P : B}}{
      \slof{A |- P : B}}
  \end{inferences}

  \begin{syntax*}
    Process expressions & P,Q &
      \spawn{P_1}{P_2} \mid \fwd
      \begin{array}[t]{@{{} \mid {}}l@{}}
        \selectR{\kay} \mid \caseL[\ell \in L]{\ell => P_{\ell}} \\
        \caseR[\ell \in L]{\ell => P_{\ell}} \mid \selectL{\kay}
      \end{array}
  \end{syntax*}
  \begin{inferences}
    \infer[\jrule{CUT}\smash{^B}]{\slof{A |- \spawn{P_1}{P_2} : C}}{
      \slof{A |- P_1 : B} & \slof{B |- P_2 : C}}
    \and
    \infer[\jrule{ID}\smash{^A}]{\slof{A |- \fwd : A}}{}
    \\
    \infer[\rrule{\plus}']{\slof{A_{\kay} |- \selectR{\kay} : \plus*[sub=_{\ell \in L}]{\ell:A_{\ell}}}}{
      \text{($\kay \in L$)}}
    \and
    \infer[\lrule{\plus}]{\slof{\plus*[sub=_{\ell \in L}]{\ell:A_{\ell}} |- \caseL[\ell \in L]{\ell => P_{\ell}} : C}}{
      \multipremise{\ell \in L}{\slof{A_{\ell} |- P_{\ell} : C}}}
    \\
    \infer[\rrule{\with}]{\slof{A |- \caseR[\ell \in L]{\ell => P_{\ell}} : \with*[sub=_{\ell \in L}]{\ell:C_{\ell}}}}{
      \multipremise{\ell \in L}{\slof{A |- P_{\ell} : C_{\ell}}}}
    \and
    \infer[\lrule{\with}']{\slof{\with*[sub=_{\ell \in L}]{\ell:C_{\ell}} |- \selectL{\kay} : C_{\kay}}}{
      \text{($\kay \in L$)}}
  \end{inferences}
  \caption{Well-typed process expressions and process chains}
  \label{fig:process-chains:session-typed-summary}
\end{figure}

% \clearpage
\subsection{From admissibility of non-analytic cuts to an operational semantics}\label{sec:process-chains:reductions}

In the previous \lcnamecref{ch:singleton-logic}, we presented a procedure for normalizing semi-axio\-matic sequent proofs in singleton logic.
Proof normalization was important to establish

In this \lcnamecref{ch:process-chains}, however, our perspective has shifted from proof theory to concurrent computation, from proofs to processes.
And so normalization is no longer appropriate -- we now want to expose the concurrent computational behavior, not just normal forms.
The situation is analogous to that of intuitionistic natural deduction and simply-typed functional computation: there, proof normalization occurs in the premise of the implication introduction rule but the usual operational semantics for functional computation does not reduce under function abstractions.

In fact, the difference is even starker here because, once recursive process definitions are introduced~\parencref{sec:process-chains:coinductive}, many useful processes will be nonterminating.
Thus, there is no clear notion of value, as exists in functional computation.
Nevertheless, in good Curry--Howard fashion, the principal cases of semi-axiomatic proof normalization will still directly inform the operational semantics of processes.

\newthought{In the previous \lcnamecref{sec:process-chains:typed-processes},} the description of how proof terms are reinterpreted as process expressions already hinted at a computational strategy.
Here we present that operational semantics in its full detail.

The operational semantics for process chains is centered around \emph{reduction}, a binary relation on chains which we write as $\reduces$;
we will use $\Reduces$ for the reflexive, transitive closure of reduction.
Reductions may occur among any of the chain's processes, and thus the relation is compatible with the monoid operation, $\cc$:
\begin{inferences}
  \infer{\chn_1 \cc \chn_2 \reduces \chn'_1 \cc \chn_2}{
    \chn_1 \reduces \chn'_1}
  \and
  \infer{\chn_1 \cc \chn_2 \reduces \chn_1 \cc \chn'_2}{
    \chn_2 \reduces \chn'_2}
\end{inferences}

At the heart of reduction are two symmetric rules that describe how messages are received:
\begin{inferences}
  \infer{\selectR{\kay} \cc \caseL[\ell \in L]{\ell => P_{\ell}} \reduces P_{\kay}}{
    \text{($\kay \in L$)}}
  \and
  \infer{\caseR[\ell \in L]{\ell => P_{\ell}} \cc \selectL{\kay} \reduces P_{\kay}}{
    \text{($\kay \in L$)}}
\end{inferences}
%
% \begin{marginfigure}
%   \centering
%   \begin{gather*}
%     \begin{tikzpicture}
%       \graph [math nodes, nodes={draw, circle}] {
%         / [coordinate]
%          --
%         k / \selectR{\kay}
%          --
%         P / {\caseL[\ell  \in L]{\ell => P_{\ell}}} [rounded rectangle, inner ysep=.5em]
%          --
%         / [coordinate];
%       };
%     \end{tikzpicture}
%     \\\reduces\\
%     \begin{tikzpicture}
%       \graph [math nodes, nodes={draw, circle, inner sep=.2em}] {
%         / [coordinate]
%          --
%         P_{\kay}
%          --
%         / [coordinate];
%       };
%     \end{tikzpicture}
%   \end{gather*}
%   \\\text{\emph{and}}\\
%   \begin{gather*}
%     \begin{tikzpicture}
%       \graph [math nodes, nodes={draw, circle}] {
%         / [coordinate]
%          --
%         P / {\caseR[\ell  \in L]{\ell => P_{\ell}}} [rounded rectangle, inner ysep=.5em]
%          --
%         k / \selectL{\kay}
%          --
%         / [coordinate];
%       };
%     \end{tikzpicture}
%     \\\reduces\\
%     \begin{tikzpicture}
%       \graph [math nodes, nodes={draw, circle, inner sep=.2em}] {
%         / [coordinate]
%          --
%         P_{\kay}
%          --
%         / [coordinate];
%       };
%     \end{tikzpicture}
%   \end{gather*}

%   \caption{Pictorial representations of the principal reductions}
% \end{marginfigure}%
As suggested earlier, when a process $\caseL[\ell \in L]{\ell => P_{\ell}}$ receives a message from its left-hand provider, it continues the thread of control with the indicated branch, $P_{\kay}$; the rule involving $\caseR[\ell \in L]{\ell => P_{\ell}}$ is symmetric.
These two reduction rules mimic the principal proof normalization steps for singleton logic's semi-axiomatic sequent proofs: $\nspawn{\selectR{\kay}}{\caseL[\ell \in L]{\ell => M_{\ell}}} = M_{\kay}$ and $\nspawn{\caseR[\ell \in L]{\ell => N_{\ell}}}{\selectL{\kay}} = N_{\kay}$.

As suggested earlier, a process $\spawn{P_1}{P_2}$ spawns, in place, new neighboring threads of control for $P_1$ and $P_2$, respectively, while the original thread of control terminates; and a process $\fwd$ terminates its thread of control.
The operational semantics formalizes these notions in rules that decompose $\spawn{P_1}{P_2}$ and $\fwd$:
\begin{inferences}
  \infer{\spawn{P_1}{P_2} \reduces P_1 \cc P_2}{}
  \and
  \infer{\fwd \reduces (\chne)}{}
\end{inferences}
%
% \begin{marginfigure}
%   \begin{gather*}
%     \begin{tikzpicture}
%       \graph [math nodes, nodes={draw, circle}] {
%         / [coordinate]
%          --
%         P / \spawn{P_1}{P_2} [rounded rectangle, inner ysep=.5em]
%          --
%         / [coordinate];
%       };
%     \end{tikzpicture}
%     \\\reduces\\
%     \begin{tikzpicture}
%       \graph [math nodes, nodes={draw, circle, inner sep=.2em}] {
%         / [coordinate]
%          --
%         P_1
%          --
%         P_2
%          --
%         / [coordinate];
%       };
%     \end{tikzpicture}
%   \\\text{\emph{and}}\\
%     \begin{tikzpicture}
%       \graph [math nodes, nodes={draw, circle, inner sep=.2em}] {
%         / [coordinate]
%          --
%         P / \fwd
%          --
%         / [coordinate];
%       };
%     \end{tikzpicture}
%     \\\reduces\\
%     \begin{tikzpicture}
%       \graph [math nodes, nodes={draw, circle, inner sep=.2em}] {
%         / [coordinate]
%          --
%         P / \phantom{\fwd} [draw=none]
%          --
%         / [coordinate];
%         (P.west) -- (P.east);
%       };
%     \end{tikzpicture}
%   \end{gather*}
%   \caption{Pictorial representations of the reductions for $\spawn{P_1}{P_2}$ and $\fwd$}
% \end{marginfigure}%
Because process chains are always considered up to associativity and unit laws, these reduction rules (along with the above $\cc$-compatibility rules) reflect the associative and identity normalization steps in the proof of admissibility of non-analytic cuts\parencref{lem:singleton-logic:hilbert:cut-admissibility}.
For example, just as
\begin{gather*}
  \nspawn{(\spawn{N_0}{\selectR{\kay}})}{M} = \nspawn{N_0}{(\nspawn{\selectR{\kay}}{M})} \\
%
\intertext{is an associative normalization step,}
%
  (\spawn{P_0}{\selectR{\kay}}) \cc P_1
    \reduces % (P_0 \cc \selectR{\kay}) \cc P_1
    = P_0 \cc (\selectR{\kay} \cc P_1)
\end{gather*}
is a reduction.
Similarly, $\fwd \cc P \reduces= P$ is a reduction that reflects the normalization step $\nspawn{\fwd}{M} = M$.

% For example:
% \begin{align*}
%   (\spawn{P_0}{\selectR{\kay}}) \cc P_1 \reduces= P_0 \cc (\selectR{\kay} \cc P_1)
%   &\quad\text{just as}\quad
%   \nspawn{(\spawn{N_0}{\selectR{\kay}})}{M} = \nspawn{N_0}{(\nspawn{\selectR{\kay}}{M})}
% \shortintertext{and}
%   \fwd \cc P \reduces= % \chne \cc P
%     P
%   &\quad\text{just as}\quad
%   \nspawn{\fwd}{M} = M
%     \,.
% \end{align*}


These rules witness the close connection between proof normalization and the operational semantics of processes%
% and non-analytic cut elimination
, but one class of normalization steps does not have a direct analogue in the operational semantics: the class of commutative normalization steps.
As a prototypical example, recall the step involving $\caseL{}$:
\begin{equation*}
  \nspawn{\caseL[\ell \in L]{\ell => N_{\ell}}}{M} = \caseL[\ell \in L]{\ell => \nspawn{N_{\ell}}{M}}
  \,.
\end{equation*}
As part of proof normalization, this step is quite natural because it progresses toward a normal form by pushing the admissible cut, represented by the $\nspawn{}{}$ constructor, further in and pulling the $\caseL[\ell \in L]{\ell => {-}}$ construction out.
In the operational semantics, however, it would be wrong to have the corresponding
\begin{equation*}
  \caseL[\ell \in L]{\ell => P_{\ell}} \cc \chn \reduces \caseL[\ell \in L]{\ell => P_{\ell} \cc \chn}
\end{equation*}
as a reduction -- it inappropriately mixes dynamic and static objects by bringing the chain $\chn$ within the process expression $\caseL[\ell \in L]{\ell => {-}}$.

% Recall 
% \begin{gather*}
%   \begin{aligned}
%     \nspawn{(\spawn{N_0}{\selectR{\kay}})}{M}
%       &= \nspawn{N_0}{(\nspawn{\selectR{\kay}}{M})}
%     \\
%     \nspawn{N}{(\spawn{\selectL{\kay}}{M_0})}
%       &= \nspawn{(\nspawn{N}{\selectL{\kay}})}{M_0}
%   \end{aligned}
%   \\[2\jot]
%   \begin{aligned}
%     \nspawn{\fwd}{M}
%       &= M
%     \\
%     \nspawn{N}{\fwd}
%       &= N
%   \end{aligned}
%   \\[2\jot]
%   \begin{aligned}
%     \nspawn{\selectR{\kay}}{\caseL[\ell \in L]{\ell => M_{\ell}}}
%       &= M_{\kay}
%     \\
%     \nspawn{\caseR[\ell \in L]{\ell => N_{\ell}}}{\selectL{\kay}}
%       &= N_{\kay}
%   \end{aligned}
%   \\[2\jot]
%   \begin{aligned}
%     \nspawn{(\spawn{\selectL{\kay}}{N_0})}{M}
%       &= \spawn{\selectL{\kay}}{(\nspawn{N_0}{M})}
%     \\
%     \nspawn{N}{(\spawn{M_0}{\selectR{\kay}})}
%       &= \spawn{(\nspawn{N}{M_0})}{\selectR{\kay}}
%     \\
%     \nspawn{\selectL{\kay}}{M}
%       &= \spawn{\selectL{\kay}}{M}
%     \\
%     \nspawn{N}{\selectR{\kay}}
%       &= \spawn{N}{\selectR{\kay}}
%     \\
%     \nspawn{\caseL[\ell \in L]{\ell => N_{\ell}}}{M}
%       &= \caseL[\ell \in L]{\ell => \nspawn{N_{\ell}}{M}}
%     \\
%     \nspawn{N}{\caseR[\ell \in L]{\ell => M_{\ell}}}
%       &= \caseR[\ell \in L]{\ell => \nspawn{N}{M_{\ell}}}
%   \end{aligned}
% \end{gather*}
% etc.

The session-typing rules and operational semantics enjoy preservation and progress theorems.

\begin{theorem}[Preservation]
  If $\slcof{A |- \chn : B}$ and $\chn \reduces \chn'$, then $\slcof{A |- \chn' : B}$.
\end{theorem}
\begin{proof}
  By structural induction on the given reduction, $\chn \reduces \chn'$.
\end{proof}

\begin{theorem}[Progress]
  If $\slcof{A |- \chn : B}$, then either:
  \begin{itemize}[nosep]
  \item chain $\chn$ can reduce, that is, $\chn \reduces \chn'$;
  \item chain $\chn$ is waiting to interact with its right-hand client, that is, $\chn = \chn' \cc \selectR{\kay}$ or $\chn = \chn' \cc \caseR[\ell \in L]{\ell => P_{\ell}}$;
  \item chain $\chn$ is waiting to interact with its left-hand provider, that is, $\chn = \selectL{\kay} \cc \chn'$ or $\chn = \caseL[\ell \in L]{\ell => P_{\ell}} \cc \chn'$; or
  \item chain $\chn$ is empty, that is, $\chn = (\chne)$.
  \end{itemize}
\end{theorem}
\begin{proof}
  By structural induction on the typing derivation, $\slcof{A |- \chn : B}$.
\end{proof}

\clearpage
\section{Coinductively defined types and process expressions}\label{sec:process-chains:coinductive}

Unfortunately, there are many relatively simple computational behaviors that cannot be described by the finitary session types thus far.
For instance, a transducer process that receives, one-by-one, a stream of input symbols and forms an output stream by replacing each $b$ with an $a$ cannot be represented.

The solution is to introduce coinductively defined types, in a manner reminiscent of the coinductively defined propositions, $\n{\defp{p}} \defd \n{A}$, seen in \cref{sec:formula-as-process:coinductive}.
We will often write coinductively defined types with Greek letters, such as $\alpha \defd A$.
Coinductively defined types are not particularly useful if process expressions remain finitary, so we also introduce mutually coinductively defined process expressions: $\slof{A |- \defp{p} : C} \defd P$.

That these definitions are coinductive is guaranteed by requiring that along every cycle among defined types and processes there is a type constructor or process expression constructor.\footnote{This generalizes the local contractivity condition described by \textcite{Gay+Hole:AI05}.}
This justifies an \emph{equi}\-recursive treatment of types in which type definitions may be silently unfolded (or re-folded) at will.

As with coinductively defined propositions, these type and process expression definitions are coinductive in only a syntactic sense.
In particular, the coinductively defined process expressions are not necessarily behaviorally coinductive, \ie, productive.
For example, $\defp{p} \defd \caseL{a => \spawn{\selectR{a}}{p}}$ is not a productive process -- after receiving an initial message $\selectR{a}$, $p$ diverges without sending or receiving any further messages.

Once coinductively defined types and process expressions are added, there is, strictly speaking, no longer a Curry--Howard isomorphism between session-typed process chains and the semi-axiomatic proofs of singleton logic.
% Extending the session-type system for process chains with recursive type and process definitions means that, strictly speaking, there is no longer a Curry--Howard isomorphism with singleton logic's Hilbert system.
Importantly, however, the core system remains unchanged and still enjoys the isomorphism because the unbounded behavior is added modularly.
The situation is once again analogous to the Curry--Howard isomorphism between intuitionistic natural deduction and the simply-typed $\lambda$-calculus:
When the $\lambda$-calculus is extended with recursive types and functions, the meaning of the type constuctor for simple function types remains unchanged and still isomorphic with intuitionistic implication.

Just as a Curry--Howard isomorphism can be recovered in the $\lambda$-calculus when general recursive types are restricted to inductive and coinductive types, work by \textcite{Derakhshan+Pfenning:LMCS20} and \textcite{Somayyajula+Pfenning:20} shows that a Curry--Howard isomorphism between session-typed process chains and (sub-)singleton sequent calculus proofs can be recovered if types are inductive or behaviorally coinductive.


\section{Examples}\label{sec:process-chains:examples}

\subsection{Binary counter}\label{sec:process-chains:binary-counter}

We can once again revisit binary counters, this time as an example of session-typed process chains.

\paragraph*{Session types}

A natural number in binary will be represented by a process chain of type $\slcof{\ctre |- \ctr}$, where $\ctr$ is a session type that 
describes the service that a counter offers
% offers the client a choice between incrementing and decrementing the counter
and $\ctre$ is a type parameter that represents a terminated counter.

The type $\ctr$ of counters is given by
\begin{equation*}
  \ctr \defd \with*{ i: \ctr , d: \plus*{ z: \ctre , s: \ctr } }
  \,
\end{equation*}
and describes a service that offers the client a choice between incrementing ($i$ branch) and decrementing ($d$ branch) the counter:
\begin{itemize}
\item
  If the client chooses to increment the counter, then the incremented counter again offers the same service, $\ctr$, to the client.
\item
  Otherwise, if the client chooses to decrement the counter, then the provider replies with either $\selectR{z}$ or $\selectR{s}$ depending on the counter's value.
  \begin{itemize}
  \item
    If the counter's value is $0$, it cannot be decremented further and so it emits $\selectR{z}$ and then terminates at type $\ctre$.
  \item
    If the counter's value is some strictly positive natural number $n$, then the provider signals that by emitting $\selectR{s}$ and then continues as a counter of value $n-1$ that offers service $\ctr$.
  \end{itemize}
\end{itemize}
As a shorthand, we can introduce a type $\ctrd$ of decrement responses:
\begin{equation*}
  \begin{lgathered}
    \ctr \defd \with*{ i: \ctr , d: \ctrd } \\
    \ctrd \defd \plus*{ z: \ctre , s: \ctr }
    \,.
  \end{lgathered}
\end{equation*}

\paragraph*{Process expressions}

A counter will be a big-endian chain of processes $\defp{b}_0$ and $\defp{b}_1$, prefixed by a process $\defp{e}$.
For example, the chain $\defp{e} \cc \defp{b}_1 \cc \defp{b}_0$ would represent a counter of value $2$.
For these chains to have type $\slcof{\ctre |- \ctr}$, the process expressions $\defp{b}_0$ and $\defp{b}_1$ must have type $\slof{\ctr |- \ctr}$, while $\defp{e}$ must have type $\slof{\ctre |- \ctr}$.
%
In addition, to implement decrements involving $\defp{b}_0$, it is also convenient to have a coinductively defined process expression $\defp{b}'_0$ of type $\slof{\ctrd |- \ctrd}$.

These process expressions are defined by 
\begin{equation*}
  \begin{lgathered}
    \slof{ \ctre |- \defp{e} : \ctr } \defd
      \caseR{ i => \spawn{\defp{e}}{\defp{b}_1}
            | d => \selectR{z} }
    \\
    \slof{ \ctr |- \defp{b}_0 : \ctr } \defd
      \caseR{ i => \defp{b}_1
            | d => \spawn{\selectL{d}}{\defp{b}'_0} }
    \\
    \slof{ \ctr |- \defp{b}_1 : \ctr } \defd
      \caseR{ i => \spawn{\selectL{i}}{\defp{b}_0}
            | d => \spawn{\defp{b}_0}{\selectR{s}} }
    \\
    \slof{ \ctrd |- \defp{b}'_0 : \ctrd } \defd
      \caseL{ z => \selectR{z}
            | s => \spawn{\defp{b}_1}{\selectR{s}} }
  \end{lgathered}
\end{equation*}
For instance, a $\defp{b}_1$ process that receives the $\selectL{i}$ increment message will spawn, in place, neighboring threads of control for $\selectL{i}$ and $\defp{b}_0$ and then terminate the original thread of control.
In effect, this sends the $\selectL{i}$ increment message to the more significant bits as a carry and flips this bit to $\defp{b}_1$.

As a second example, a $\defp{b}_0$ process that receives the $\selectL{d}$ decrement message should decrement the counter formed by the more significant bits and then analyze the response with a $\defp{b}'_0$ process.
% We need to introduce one more coinductively defined process expression, $\defp{b}'_0$, that serves to analyze the response and then produce its own response; $\defp{b}'_0$ therefore must have type $\slof{\ctrd |- \ctrd}$.
% \begin{equation*}
%   \slof{ \ctrd |- \defp{b}'_0 : \ctrd } \defd
%     \caseL{ z => \selectR{z}
%           | s => \spawn{\defp{b}_1}{\selectR{s}} }
% \end{equation*}
\begin{itemize}
\item If that $\defp{b}'_0$ process receives a response of $\selectR{z}$, then the more significant bits had value $0$ and so must the counter as a whole.
  According to the type $\ctrd$, the current thread of control must produce a response of type $\slof{\ctre |- \ctrd}$, which is easily done by sending $\selectR{z}$.
\item Otherwise, if the $\defp{b}'_0$ process receives a response of $\selectR{s}$, then the more significant bits had value $n+1$, for some $n \geq 0$, and the counter as a whole must have value $2n+1$ after the decrement.
  This is accomplished by emitting $\selectR{s}$ and replacing $\defp{b}'_0$ with a recursive call to $\defp{b}_1$.
\end{itemize}

As an example of these processes in action, observe that $\defp{e} \cc \defp{b}_1 \cc \selectL{i} \cc \selectL{d}$ has the following trace, among others.
\begin{align*}
  \MoveEqLeft[.5]
  \defp{e} \cc \defp{b}_1 \cc \selectL{i} \cc \selectL{d} \\
    &\reduces \defp{e} \cc (\spawn{\selectL{i}}{\defp{b}_0}) \cc \selectL{d}
     \reduces \defp{e} \cc \selectL{i} \cc \defp{b}_0 \cc \selectL{d} \\
    &\reduces \defp{e} \cc \selectL{i} \cc (\spawn{\selectL{d}}{\defp{b}'_0})
     \reduces \defp{e} \cc \selectL{i} \cc \selectL{d} \cc \defp{b}'_0 \\
    &\reduces (\spawn{\defp{e}}{\defp{b}_1}) \cc \selectL{d} \cc \defp{b}'_0
     \reduces \defp{e} \cc \defp{b}_1 \cc \selectL{d} \cc \defp{b}'_0 \\
    &\reduces \defp{e} \cc (\spawn{\defp{b}_0}{\selectR{s}}) \cc \defp{b}'_0
     \reduces \defp{e} \cc \defp{b}_0 \cc \selectR{s} \cc \defp{b}'_0 \\
    &\reduces \defp{e} \cc \defp{b}_0 \cc (\spawn{\defp{b}_1}{\selectR{s}})
     \reduces \defp{e} \cc \defp{b}_0 \cc \defp{b}_1 \cc \selectR{s}
\end{align*}



\begin{equation*}
  \begin{lgathered}
    \defp{\imath} \defd \caseL{e => \spawn{\selectR{e}}{\selectR{b}_1}
                             | b_0 => \selectR{b}_1
                             | b_1 => \spawn{\defp{\imath}}{\selectR{b}_0}}
    \\
    \defp{d} \defd \caseL{e => \selectR{z}
                        | b_0 => \spawn{\defp{d}}{\defp{b}'_0}
                        | b_1 => \spawn{\selectR{b}_0}{\selectR{s}}}
    \\
    \defp{b}'_0 \defd \caseL{z => \selectR{z}
                           | s => \spawn{\defp{d}}{\defp{b}'_0}}
  \end{lgathered}
\end{equation*}



\subsection{Sequential transducers}\label{sec:process-chains:transducer}

By this point in this document, the reader will likely expect either \acp{DFA} or \acp{NFA} as our next example of session-typed process chains.
Instead, we will use sequential transducers, as introduced in \cref{ch:finite-automata}.

\paragraph*{Session types}

We should first construct a type that describes the words over an alphabet $\ialph$, respectively.
However, because the language formulated in this \lcnamecref{ch:process-chains} does not have inductive types, we cannot describe the finite words $\finwds{\ialph}$ alone.
With the inductive session types studied by \textcite{Derakhshan+Pfenning:LMCS20} and \textcite{Somayyajula+Pfenning:20}, that would be possible, but we do not pursue that extension here.

Instead, we will construct a type that describes the \emph{infinite} words over alphabet $\ialph$:
\begin{equation*}
  % \begin{lgathered}
    \infinwds{\ialph} \defd \plus*[sub=_{a \in \ialph}]{ a: \infinwds{\ialph} } % \\
    % \infinwds{\oalph} \defd \plus*[sub=_{b \in \oalph}]{ b: \infinwds{\oalph} }
  \,.
  % \end{lgathered}
\end{equation*}
A process that offers type $\infinwds{\ialph}$ is one that emits a sequence of messages that correspond to an infinite word over $\ialph$.
For instance, when given by the following definition, $\slof{\epsilon |- \defp{w} : \infinwds{\ialph}}$ is a process that corresponds to the infinite word $w = abbabb\dotsm$:
\begin{equation*}
  \slof{\epsilon |- \defp{w} : \infinwds{\ialph}} \defd
    \spawn{\defp{w}}{(\spawn{\spawn{\selectR{b}}{\selectR{b}}}{\selectR{a}})}
  \,.
\end{equation*}
Notice that the type parameter $\epsilon$ is never used directly and could be replaced with any type $A$.

Using this type, we can now implement \emph{infinite}-word sequential transducers with well-typed process expressions.



% Nevertheless, we can use the following types to describe \emph{all} words, finite or infinite, over $\ialph$ and $\oalph$, respectively:
% \begin{equation*}
%   \begin{lgathered}
%     \str \defd \plus*[sub=_{a \in \ialph}]{ a: \str , \eow: \stre } \\
%     \str[\oalph] \defd \plus*[sub=_{b \in \oalph}]{ b: \str[\oalph] , \eow: \stre }
%   \,.
%   \end{lgathered}
% \end{equation*}
% A process of type $\slof{\stre |- \str[\ialph]}$ is then one that emits a sequence of messages that corresponds to a word over $\ialph$.

\paragraph*{Process expressions}

Let $\aut{T} = (Q, \sftnext, \sftout)$ be an infinite-word sequential transducer over the input and output alphabets $\ialph$ and $\oalph$, respectively.
Each state $q \in Q$ maps words from $\infinwds{\ialph}$ to $\infinwds{\oalph}$, and therefore ought to correspond to a process expression $\defp{q}$ of type $\slof{\infinwds{\ialph} |- \infinwds{\oalph}}$:
\begin{equation*}
  \slof{\infinwds{\ialph} |- \defp{q} : \infinwds{\oalph}} \defd
    \caseL[a \in \ialph]{ a => \spawn{\defp{q}'}{\rev{\selectR{w}}} }  \text{\enspace where $\sftnext(q, a) = q'$ and $\sftout(q, a) = w$.}
\end{equation*}
(The anti-homomorphism $\rev{\selectR{w}}$ is defined in the adjacent \lcnamecref{fig:process-chains:transducer-rev}.
It is notationally identical to, but formally distinct from, the anti-homomorphism from words to contexts of right-directed atoms~(\cref{fig:formula-as-process:msg-rev}).
However, as we will see in \cref{ch:correspond}, the two anti-homomorphisms are conceptually related.)%
\begin{marginfigure}[-5\baselineskip]
  \begin{equation*}
    \rev{\selectR{w}} =
    \begin{cases*}
      \fwd & if $w = \emp$ \\
      \spawn{\rev{\selectR{w}_0}}{\selectR{a}} & if $w = a \wc w_0$
    \end{cases*}
  \end{equation*}
  \caption{An anti-homomorphism from $\finwds{\oalph}$ to processes of type $\slof{\infinwds{\oalph} |- \infinwds{\oalph}}$}\label{fig:process-chains:transducer-rev}
\end{marginfigure}

For a concrete example, recall from \cref{ch:finite-automata} the infinite-word sequential transducer over $\ialph = \oalph = \Set{ a , b }$ (repeated in the adjacent \lcnamecref{fig:process-chains:sft-example}) that compresses each run of $b$s within the input word into a single $b$.%
\begin{marginfigure}[-1\baselineskip]
  \begin{equation*}
    \mathllap{\aut{T} = {}}
    \begin{tikzpicture}[baseline=(q_0.base)]
      \graph [automaton] {
        q_0
         -> [loop above, "$\tio{a | a}$"]
        q_0
         -> [bend left, "$\tio{b | b}$"]
        q_1 [right=-0.5em of q_0]
         -> [loop above, "$\tio{b | \emp}$"]
        q_1
         -> [bend left, "$\tio{a | a}$"]
        q_0 ;
        % e_0 [coordinate, below=-1.75em of q_0.south];
        % e_1 [coordinate, below=-5.25em of q_1.south];
        % (q_0.south) -> ["$\emp$", swap] e_0 ;
        % (q_1.south) -> ["$\vphantom{\emp}\smash{b}$"] e_1 ;
      };
    \end{tikzpicture}
  \end{equation*}
  \caption{An infinite-word sequential transducer that compresses runs of consecutive $b$s.  (Repeated from \cref{fig:finite-automata:sft-example}.)}\label{fig:process-chains:sft-example}
\end{marginfigure}
Because $\ialph = \oalph = \Set{ a , b }$, the types for input and output words are defined as:
\begin{equation*}
  \begin{lgathered}
    \infinwds{\ialph} \defd
      \plus*{ a: \infinwds{\ialph} , b: \infinwds{\ialph} }
    \\
    \infinwds{\oalph} \defd
      \plus*{ a: \infinwds{\oalph} , b: \infinwds{\oalph} }
    \,.
  \end{lgathered}
\end{equation*}
Following the encoding laid out above, the states $q_0$ and $q_1$ become process expressions $\defp{q}_0$ and $\defp{q}_1$ of type $\slof{ \infinwds{\ialph} |- \infinwds{\oalph} }$ defined by:
\begin{equation*}
  \begin{lgathered}
    \slof{ \infinwds{\ialph} |- \defp{q}_0 : \infinwds{\oalph} } \defd
      \caseL{ a => \spawn{\defp{q}_0}{\selectR{a}}
            | b => \spawn{\defp{q}_1}{\selectR{b}} }
    \\
    \slof{ \infinwds{\ialph} |- \defp{q}_1 : \infinwds{\oalph} } \defd
      \caseL{ a => \spawn{\defp{q}_0}{\selectR{a}}
            | b => \spawn{\defp{q}_1}{\fwd} }
    \,.
  \end{lgathered}
\end{equation*}

Sequential transducers are closed under composition.
To represent the composition of two infinite-word sequential transducers $\aut{T}_1$ and $\aut{T}_2$ as a well-typed process, we could simply construct their composition, $\aut{T} = \aut{T}_2 \circ \aut{T}_1$, as a sequential transducer in its own right and then represent the transducer $\aut{T}$ as a well-typed process.

Even easier, however, is to directly compose the processes that represent the transducers $\aut{T}_1$ and $\aut{T}_2$.
If $\aut{T}_1$ and $\aut{T}_2$ are in states $q$ and $s$, respectively, then the processes $\slof{\infinwds{\ialph} |- \defp{q} : \infinwds{\oalph}}$ and $\slof{\infinwds{\oalph} |- \defp{s} : \infinwds{\Delta}}$ are well-typed and the process $\slof{\infinwds{\ialph} |- \spawn{\defp{q}}{\defp{s}} : \infinwds{\Delta}}$  describes the current state of the composition, $\aut{T} = \aut{T}_2 \circ \aut{T}_1$.
In fact, \textfootcite{DeYoung+Pfenning:APLAS16} prove -- in a very closely related, if slightly different, framework -- that cut elimination actually constructs a normal-form process for the transducer $\aut{T}$.

\subsection{Turing machines}\label{sec:process-chains:turing-machines}

\paragraph*{Two-way infinite tape Turing machine}

Let $\aut{M} = (Q, \delta)$ be a two-way infinite tape Turing machine over alphabet $\ialph$.
We imagine the two-way infinite tape as divided into two one-way infinite halves with the machine's finite control, or head, sitting between them.
Each of these halves will be represented as a stream of symbols from $\ialph$, directed inward toward $\aut{M}$'s head.
As such, these two one-way infinite halves are described by the following dual types:%
\footnote{The involution $\sym*{}$ was defined in \cref{fig:singleton-logic:involution}, but not for coinductively defined propositions.
  Here we use $\sym*{\infinwds{\ialph}}$ merely as the name of a coinductively defined type, though the choice of name is intended to evoke the duality with the type $\infinwds{\ialph}$.}%
\begin{equation*}
  \begin{lgathered}
    \infinwds{\ialph} \defd \plus*[sub=_{a \in \ialph}]{ a: \infinwds{\ialph} } \\
    \sym*{\infinwds{\ialph}} \defd \with*[sub=_{a \in \ialph}]{ a : \sym*{\infinwds{\ialph}} }
  \,.
  \end{lgathered}
\end{equation*}
% The type $\infinwds{\ialph}$ describes the left half of the two-way infinite tape: a process that offers $\infinwds{\ialph}$ will emit to its right-hand client an infinite stream of tape symbols.
% Dually, the type $\sym*{\infinwds{\ialph}}$ describes the right half of the two-way infinite tape: a process that uses $\sym*{\infinwds{\ialph}}$ will emit to its left-hand provider an infinite stream of tape symbols.
That is, a process of type $\slof{ A |- \infinwds{\ialph} }$, for some $A$, acts as the left-hand one-way infinite half of the tape, whereas a process of the dual type, $\slof{ \sym*{\infinwds{\ialph}} |- B }$ for some $B$, acts as the right-hand one-way infinite half of the tape.

Because the machine's head sits between these two halves of the two-way infinite tape,
% As the finite control sits between the two halves of the two-way infinite tape,
it ought to correspond to a process of type $\slof{\infinwds{\ialph} |- \sym*{\infinwds{\ialph}}}$.
Indeed, for each state $q \in Q$, we will define two process expressions, $\slof{\infinwds{\ialph} |- \tlhead{\defp{q}} : \sym*{\infinwds{\ialph}}}$ and $\slof{\infinwds{\ialph} |- \trhead{\defp{q}} : \sym*{\infinwds{\ialph}}}$, for the two heads possible in state $q$:
\begin{equation*}
  \begin{lgathered}
    \slof{ \infinwds{\ialph} |- \tlhead{\defp{q}} : \sym*{\infinwds{\ialph}} } \defd
      \mathsf{caseL}_{a \in \ialph}\left(
        a \Rightarrow
          \begin{cases*}
            \spawn{\tlhead{\defp{q}'}}{\selectL{b}} & if $\delta(q, a) = (q', b, \mathsf{L})$ \\
            \spawn{\selectR{b}}{\trhead{\defp{q}'}} & if $\delta(q, a) = (q', b, \mathsf{R})$
          \end{cases*} \right)
  \\
    \slof{ \infinwds{\ialph} |- \trhead{\defp{q}} : \sym*{\infinwds{\ialph}} } \defd
      \mathsf{caseR}_{a \in \ialph}\left(
        a \Rightarrow
          \begin{cases*}
            \spawn{\tlhead{\defp{q}'}}{\selectL{b}} & if $\delta(q, a) = (q', b, \mathsf{L})$ \\
            \spawn{\selectR{b}}{\trhead{\defp{q}'}} & if $\delta(q, a) = (q', b, \mathsf{R})$
          \end{cases*} \right)
  \end{lgathered}
\end{equation*}
That is, a left-facing head begins by reading a symbol from its left.
Depending on the symbol that is read, the machine writes a new symbol in its place and then either advances the head to the left or otherwise turns the head to face the right.
Writing a new symbol is accomplished by spawning a message directed toward the head.
Right-facing heads are symmetric.

Notice that the tape in this process implementation is truly two-way infinite.
This is consistent with the model presented in \cref{ch:finite-automata}, but differs from the traditional model of a Turing machine.
In the traditional model, the tape might more accurately be described as two-way \emph{unbounded} -- at any given moment, the tape is finite, but can be extended.
If we wanted to prove the adequacy of our process implementation adequate with respect to traditional Turing machines, we would have to prove adequacy for infinite tapes that behave as merely unbounded, using a distinguished blank symbol.


% Notice a subtle difference between this implementation and the mathematical model described in \cref{ch:??}.
% In the mathematical model, configurations used finite strings to represent the two one-way halves of the two-way infinite tape.
% This was possible because of an invariant on the tape's contents: after a finite number of moves, the tape contained only finitely many non-blank symbols.%
% \footnote{In this sense, the tape might more accurately be described as two-way \emph{unbounded}.}
% However, in the implementation presented here, the types $\infinwds{\ialph}$ and $\sym*{\infinwds{\ialph}}$ do not make the same guarantee -- as previously explained, our language's types do not include the inductive types that would be necessary to enforce the finiteness invariant.
% Instead, we are the mercy of the processes of types $\slof{A |- \infinwds{\ialph}}$ and $\slof{\sym*{\infinwds{\ialph}} |- B}$ that represent the two tape halves to behave properly and eventually emit only blank symbols.

\paragraph*{One-way infinite tape Turing machine}

Because the head of a two-way infinite tape machine corresponds to a process of type $\slof{ \infinwds{\ialph} |- \sym*{\infinwds{\ialph}} }$, it is not possible to easily compose two such machines using cut, as we had done for sequential transducers.
The types simply do not match:
\begin{equation*}
  \slof{ \infinwds{\ialph} |- \tlhead{\defp{q}} : \sym*{\infinwds{\ialph}} } \neq \slof{ \infinwds{\ialph} |- \trhead{\defp{s}} : \sym*{\infinwds{\ialph}} }
  \,.
\end{equation*}
However, if we use one-way infinite tape Turing machines, composition of machines is possible.

We will use the following types.
The type $\infinwds{\ialph}$ is the same as above, but we introduce a type $\sym*{\finwds{\ialph}}$ that will be used in describing the one-way infinite tape's finite end.
\begin{equation*}
  \begin{lgathered}
    \infinwds{\ialph} \defd \plus*[sub=_{a \in \ialph}]{ a : \infinwds{\ialph} } \\
    \sym*{\finwds{\ialph}} \defd \with*[sub=_{a \in \ialph}]{ a : \sym*{\finwds{\ialph}} , \eow : \infinwds{\ialph} }
  \end{lgathered}
\end{equation*}
In particular, the label $\eow$ will be used to mark the tape's finite end.

The type name $\sym*{\finwds{\ialph}}$ was chosen to suggest a left-directed finite string, but should not be taken too literally -- as previously mentioned, the language presented in this \lcnamecref{ch:process-chains} does not have the inductive types that would be necessary to enforce finiteness.
Notice, too, that any process that offers the type $\sym*{\finwds{\ialph}}$ continues by offering $\infinwds{\ialph}$ when $\selectL{\eow}$ is received, which will be crucial in composing machines (although not exactly what would be expected of a strict dual of $\finwds{\ialph}$).

The machine's head ought to correspond to a process of type $\slof{ \infinwds{\ialph} |- \sym*{\finwds{\ialph}} }$.
For each state $q \in Q$, we will define a process $\slof{ \infinwds{\ialph} |- \tlhead{\defp{q}} : \sym*{\finwds{\ialph}} }$ just like we did for the two-way infinite tape machines:
\begin{equation*}
    \slof{ \infinwds{\ialph} |- \tlhead{\defp{q}} : \sym*{\finwds{\ialph}} } \defd
      \mathsf{caseL}_{a \in \ialph}\left(
        a \Rightarrow
          \begin{cases*}
            \spawn{\tlhead{\defp{q}'}}{\selectL{b}} & if $\delta(q, a) = (q', b, \mathsf{L})$ \\
            \spawn{\selectR{b}}{\trhead{\defp{q}'}} & if $\delta(q, a) = (q', b, \mathsf{R})$
          \end{cases*} \right)
\end{equation*}

For each state $q \in Q$, we will also define a process $\slof{ \infinwds{\ialph} |- \trhead{\defp{q}} : \sym*{\finwds{\ialph}} }$ for the right-facing head in state $q$.
This process is mainly like the process definition for a right-facing head in the two-way infinite tape Turing machine.
The difference is that here we have a $\eow$ branch, owing to the presence of label $\eow$ in the type $\sym*{\finwds{\ialph}}$:
\begin{equation*}
    \slof{ \infinwds{\ialph} |- \trhead{\defp{q}} : \sym*{\finwds{\ialph}} } \defd
      \mathsf{caseR}_{a \in \ialph}\left(
        \begin{array}{@{}r@{}l@{}}
          a \Rightarrow {} &
            \begin{cases*}
              \spawn{\tlhead{\defp{q}'}}{\selectL{b}} & if $\delta(q, a) = (q', b, \mathsf{L})$ \\
              \spawn{\selectR{b}}{\trhead{\defp{q}'}} & if $\delta(q, a) = (q', b, \mathsf{R})$
            \end{cases*}
          \\[4\jot]
          \mid \eow \Rightarrow {} &
            \begin{cases*}
              \fwd & if $q \in F$ \\
              \mathrlap{\spawn{\tlhead{\defp{q}}}{\selectL{\eow}}} \hphantom{\spawn{\tlhead{\defp{q}'}}{\selectL{b}}} & otherwise
            \end{cases*}
        \end{array} \right)
\end{equation*}
Just like the mathematical model of a one-way infinite tape Turing machine, the behavior of a right-facing head that reaches the finite end of the tape depends on whether the current state is final.
If it is final, then the forwarding process causes the machine's head to terminate, leaving the tape.
Otherwise, if the state is not final, then the head is turned to the left using $\spawn{\tlhead{\defp{q}}}{\selectL{\eow}}$, where the message $\selectL{\eow}$ is recreated to preserve our marking of the tape's end.

Thus, to start a machine, we would use the process
\begin{equation*}
  \slof{ \infinwds{\ialph} |- \spawn{\tlhead{\defp{q}}}{\selectL{\eow}} : \infinwds{\ialph} }
  \,.
\end{equation*}
Notice that this type can be readily composed.
States $q$ and $s$ from different machines could be easily composed with cut:
\begin{equation*}
  \slof{ \infinwds{\ialph} |- \spawn{(\spawn{\tlhead{\defp{q}}}{\selectL{\eow}})}{(\spawn{\tlhead{\defp{s}}}{\selectL{\eow}})} : \infinwds{\ialph} }
  \,.
\end{equation*}
This form of composition is sequential, not parallel, because the left-facing head $\tlhead{\defp{s}}$ must block until the preceding machine terminates and forwards its tape on to $\tlhead{\defp{s}}$.

% Rather than situating the machine's head below a cell on a two-way infinite tape, it will be convenient to think of the head as \begin{enumerate*}[label=\emph{(\roman*)}] \item occuring between the two one-way halves of the tape, and \item facing either the left half or right half of the tape. \end{enumerate*}
% Thus, each state in the Turing machine will correspond to two processes, $\tlhead{\defp{q}}$ and $\trhead{\defp{q}}$, for the state's left- and right-facing instances, respectively.

% These processes will be typed as $\slof{\tape |- \tlhead{\defp{q}} : \epat}$ and $\slof{\tape |- \trhead{\defp{q}} : \epat}$

% \begin{equation*}
%   \begin{lgathered}
%     \tape \defd \plus*[sub=_{a \in \ialph}]{ a: \tape , \eow: \tapee } \\
%     \epat \defd \with*[sub=_{a \in \ialph}]{ a: \epat , \eow: \tape }
%   \end{lgathered}
% \end{equation*}

% \begin{equation*}
%   \begin{lgathered}
%     \infinwds{\ialph} \defd \plus*[sub=_{a \in \ialph}]{ a: \infinwds{\ialph} } \\
%     \finwds{\ialph}_{\with} \defd \with*[sub=_{a \in \ialph}]{ a: \finwds{\ialph}_{\with} , \eow: \infinwds{\ialph} }
%   \end{lgathered}
% \end{equation*}


% \begin{equation*}
%   \slof{ \infinwds{\ialph} |- \tlhead{\defp{q}} : \finwds{\ialph}_{\with} } \defd
%     \mathsf{caseL}_{a \in \ialph}
%     \left\lparen
%         a \Rightarrow
%           \begin{cases*}
%              \spawn{\tlhead{\defp{q}'_a}}{\selectL{a}'}
%                & if $\delta(q, a) = (q'_a, a', \mathsf{L})$
%              \\
%              \spawn{\selectR{a}'}{\trhead{\defp{q}'_a}}
%                & if $\delta(q, a) = (q'_a, a', \mathsf{R})$
%            \end{cases*}
%     \right\rparen
% \end{equation*}

% \begin{equation*}
%   \slof{ \infinwds{\ialph} |- \trhead{\defp{q}} : \finwds{\ialph}_{\with} } \defd
%     \mathsf{caseR}_{a \in \ialph}
%     \left\lparen
%       \begin{array}{@{}l@{}}
%         \hphantom{\mid {}}
%         a \Rightarrow
%           \begin{cases*}
%             \spawn{\tlhead{\defp{q}'_a}}{\selectL{a}'}
%               & if $\delta(q, a) = (q'_a, a', \mathsf{L})$
%             \\
%             \spawn{\selectR{a}'}{\trhead{\defp{q}'_a}}
%               & if $\delta(q, a) = (q'_a, a', \mathsf{R})$
%           \end{cases*}
%       \\[3ex]
%         \mid \eow \Rightarrow \spawn{\trhead{\defp{q}}}{\selectL{\eow}}
%       \end{array}
%     \right\rparen
% \end{equation*}


% \begin{equation*}
%   \slof{ \tape |- \tlhead{\defp{q}} : \epat } \defd
%     \mathsf{caseL}_{a \in \ialph}
%     \left\lparen
%       \begin{array}{@{}l@{}}
%         \hphantom{\mid {}}
%         a \Rightarrow
%           \begin{cases*}
%              \spawn{\tlhead{\defp{q}'_a}}{\selectL{a}'}
%                & if $\delta(q, a) = (q'_a, a', \mathsf{L})$
%              \\
%              \spawn{\selectR{a}'}{\trhead{\defp{q}'_a}}
%                & if $\delta(q, a) = (q'_a, a', \mathsf{R})$
%            \end{cases*}
%       \\[3ex]
%         \mid \eow \Rightarrow \spawn{\selectR{\eow}}
%                                     {\spawn{\selectR{\tblank}}{\tlhead{\defp{q}}}}
%       \end{array}
%     \right\rparen
% \end{equation*}

% \begin{equation*}
%   \slof{ \tape |- \trhead{\defp{q}} : \epat } \defd
%     \mathsf{caseR}_{a \in \ialph}
%     \left\lparen
%       \begin{array}{@{}l@{}}
%         \hphantom{\mid {}}
%         a \Rightarrow
%           \begin{cases*}
%             \spawn{\tlhead{\defp{q}'_a}}{\selectL{a}'}
%               & if $\delta(q, a) = (q'_a, a', \mathsf{L})$
%             \\
%             \spawn{\selectR{a}'}{\trhead{\defp{q}'_a}}
%               & if $\delta(q, a) = (q'_a, a', \mathsf{R})$
%           \end{cases*}
%       \\[3ex]
%         \mid \eow \Rightarrow \spawn{\spawn{\trhead{\defp{q}}}{\selectL{\tblank}}}
%                                     {\selectL{\eow}}
%       \end{array}
%     \right\rparen
% \end{equation*}


% Tape |- q1 : epaT    q1 reads from either side
% epaT |- q2 : Tape    q2 writes to either side
% Tape |- q3 : Tape    q3 reads from the left and writes to the right
% epaT |- q4 : epaT    q4 writes to the left and reads from the right



% \section{Automata and transducers}

% \begin{equation*}
%   \slof{\infinwds{\Sigma} |- q : \finwds{\Gamma}}
%   \defd
%   \caseL[a \in \Sigma]{a => \spawn{q'_a}{\selectR{w}_{q,a}}}
% \end{equation*}


% \clearpage

% \section{Toward asynchronous SILL}

% Most work on SILL uses a synchronous interpretation of cut reductions as communication.
% \begin{gather*}
%   \infer[\jrule{CUT}]{\lctx'_1, \lctx'_2, \lctx \vdash (\nu x)(x(y).P \mid (\nu y)\overline{x}\langle y\rangle.(Q_1 \mid Q_2)) :: z{:}C}{
%     \infer[\rrule{\lolli}]{\lctx \vdash x(y).P :: x{:}A \lolli B}{
%       \lctx, y{:}A \vdash P :: x{:}B} &
%     \infer[\lrule{\lolli}]{\lctx'_1, \lctx'_2, x{:}A \lolli B \vdash (\nu y)\overline{x}\langle y\rangle.(Q_1 \mid Q_2) :: z{:}C}{
%       \lctx'_1 \vdash Q_1 :: y{:}A &
%       \lctx'_2, x{:}B \vdash Q_2 :: z{:}C}}
%   \\\reduces\\
%   \infer[\jrule{CUT}]{\lctx'_2, \lctx, \lctx'_1 \vdash (\nu x)((\nu y)(Q_1 \mid P) \mid Q_2) :: z{:}C}{
%     \infer[\jrule{CUT}]{\lctx, \lctx'_1 \vdash (\nu y)(Q_1 \mid P) :: x{:}B}{
%       \lctx'_1 \vdash Q_1 :: y{:}A &
%       \lctx, y{:}A \vdash P :: x{:}B} &
%     \lctx'_2, x{:}B \vdash Q_2 :: z{:}C}
% \end{gather*}

% \begin{inferences}
%   \infer[\rrule{\lolli}]{\lctx \vdash x(y,x').P :: x{:}A \lolli B}{
%     \lctx, y{:}A \vdash P :: x'{:}B}
%   \and
%   \infer[\lrule{\lolli}']{x{:}A \lolli B, y{:}A \vdash \overline{x}\langle y,x'\rangle :: x'{:}B}{}
% \end{inferences}

% \begin{inferences}
%   \infer[\rrule{\bang}']{\uctx, u{:}A ; \lctxe \vdash \overline{x}\langle u\rangle :: x{:}\bang A}{}
%   \and
%   \infer[\lrule{\bang}]{\uctx ; \lctx, x{:}\bang A \vdash x(u).P :: z{:}C}{
%     \uctx, u{:}A ; \lctx \vdash P :: z{:}C}
% \end{inferences}

% \begin{inferences}
% %  \infer[\jrule{CUT}
% \infer[\rrule{\bang}']{\uctx, u{:}A ; \lctxe \vdash \overline{x}\langle u\rangle :: x{:}\bang A}{}
%   \and
%   \infer[\lrule{\bang}]{\uctx ; \lctx, x{:}\bang A \vdash x(u).P :: z{:}C}{
%     \uctx, u{:}A ; \lctx \vdash P :: z{:}C}
% \end{inferences}





% \section{}

% \subsection{Process chains}

% % A computational process represents a single thread of control that interacts with its environment.
% %
% % Then,
% By analogy with chains of communicating automata, we envision a process chain, $\chn$, as a (possibly empty) finite sequence of processes $(P_i)_{i=1}^{n}$, each with its own independent thread of control and arranged in a linear topology.
% As depicted in the adjacent \lcnamecref{fig:singleton-processes:chain-topology},%
% %
% \begin{marginfigure}
%   \centering
%   \begin{tikzpicture}
%     \graph [math nodes, nodes={circle, draw}] {
%       P_0 / [coordinate]
%        --
%       P_1
%        --
%       / \dotsb [rectangle, text height=1ex, draw=none]
%        --
%       P_i
%        --
%       / \dotsb [rectangle, text height=1ex, draw=none]
%        --
%       P_n
%        --
%       / [coordinate];
%     };
%     \node [fit=(P_1) (P_i) (P_n), inner xsep=.5em,
%            draw,
%            label distance=2em, label=$\chn$] {};
%   \end{tikzpicture}
%   \caption{A prototypical process chain, $\chn$}\label{fig:singleton-processes:chain-topology}
% \end{marginfigure}
% %
% % each process $P_i$ shares a unique channel with its left-hand neighbor and a unique channel with its right-hand neighbor.
% each process $P_i$ shares unique channels with its left- and right-hand neighbors. %, along which it communicates with those neighbors.
% Along these channels, neighboring processes may interact -- and react, changing their internal state.
% Because process chains always maintain a linear topology, 
% % these
% channels need not be named -- they can instead be referred to as simply the left- and right-hand channels of $P_i$.

% A chain $\chn$ does not compute in isolation, however.
% The left-hand channel of $P_1$ and the right-hand channel of $P_n$ enable the chain to interact with its surroundings.
% Because these two channels are the only ones exposed to the external environment [surroundings], they may be referred to as the left- and right-hand channels of the chain.

% Chains may be composed end to end by conjoining the right-hand channel of one chain with the left-hand channel of another chain.

% \paragraph{Chains as a free monoid}
% % \newthought
% {Moving from} this informal intuition to a more formal characterization, process chains $\chn$ form a free monoid over processes $P$:
% \begin{equation*}
%   \chn \Coloneqq \chne \mid (\chn_1 \cc \chn_2) \mid P
%   \,,
% \end{equation*}
% where $\chne$ denotes the empty chain and $\cc$ denotes the monoid operation, chain composition.
% As the monoid operation, composition is subject to the usual associativity and unit laws%
% \footnote{Unlike composition in most process calculi, chain composition is not commutative.}%
% :
% \begin{gather*}
%   (\chn_1 \cc \chn_2) \cc \chn_3 = \chn_1 \cc (\chn_2 \cc \chn_3) \\
%   \chne \cc \chn = \chn = \chn \cc \chne
% \end{gather*}
% Because these monoid laws may be freely applied, we switch between two alternative views of process chains whenever convenient: the view that a chain $\chn$ is either empty ($\chn = \chne$), a composition ($\chn = \chn_1 \cc \chn_2$), or a single process ($\chn = P$); and 
% % has one of the forms $\chne$, $\chn_1 \cc \chn_2$, or $P$; or
% the view that a chain $\chn$ is a finite sequence of processes ($\chn = P_1 \cc \dotsb \cc P_n$).

% \newthought{To ...}, a session-type system for process chains can be developed.
% to describe how the process chain $\chn$ interacts with its environment, we use a judgment
% \begin{equation*}
%   \slcof{A |- \chn : B}
%   \,,
% \end{equation*}
% meaning that the chain $\chn$ offers service $B$ along its right-hand channel, while concurrently using service $A$ along its left-hand channel.

% For a chain composition $\chn_1 \cc \chn_2$ to be well-typed, the service offered by $\chn_1$ along its right-hand channel must be the same service that $\chn_2$ expects to use along its left-hand channel.
% Otherwise, communication between $\chn_1$ and $\chn_2$
% \begin{equation*}
%   \infer[\jrule{C-CUT}^B]{\slcof{A |- \chn_1 \cc \chn_2 : C}}{
%     \slcof{A |- \chn_1 : B} & \slcof{B |- \chn_2 : C}}
% \end{equation*}

% The empty chain, $\chne$, offers a service $A$ to its right by directly using the same service from its left:
% \begin{equation*}
%   \infer[\jrule{C-ID}^A]{\slcof{A |- \chne : A}}{}
% \end{equation*}

% Lastly, a chain that consists of a single process $P$ is well-typed if its process expression $P$ is well-typed:
% \begin{equation*}
%   \infer[\jrule{C-PROC}]{\slcof{A |- P : C}}{
%     \slof{A |- P : C}}
% \end{equation*}

% \begin{figure}[tbp]
%   \begin{inferences}
%   \infer[\jrule{C-CUT}^B]{\slcof{A |- \chn_1 \cc \chn_2 : C}}{
%     \slcof{A |- \chn_1 : B} & \slcof{B |- \chn_2 : C}}
%   \and
%   \infer[\jrule{C-ID}^A]{\slcof{A |- \chne : A}}{}
%   \and
%   \infer[\jrule{C-PROC}]{\slcof{A |- P : C}}{
%       \slof{A |- P : C}}  
%   \end{inferences}
%   \caption{Process chains and their session-type system}%
%   \label{fig:process-chains:chains}
% \end{figure}

% Offers/uses distinction: retained for consistency with the hypothetical judgement asymmetry and SILL.
% Judgmental asymmetry between antecedents and consequents of a sequent.


% A chain $\chn$ does not compute in isolation, but instead interacts with its environment along two channels:
% % A chain $\chn$ may interact with its environment along two channels:
% to its left along the left-hand channel of $P_1$, and to its right along the right-hand channel of $P_n$.
% Chains can be composed end to end by
% %
% The left-hand channel of $P_1$ and the right-hand channel of 

% More formally, as ordered lists of processes, process chains form a free monoid.

% Alternatively, process chains may be characterized algebraically as forming a free monoid over processes.


% Process chains form a free monoid...

% Process chains communicate with their environment...

% The judgment $\slcof{A |- \chn : B}$ describes the pattern of communication...

% Chain composition
% \begin{equation*}
%   \infer[\jrule{C-CUT}^B]{\slcof{A |- \chn_1 \cc \chn_2 : C}}{
%     \slcof{A |- \chn_1 : B} & \slcof{B |- \chn_2 : C}}
% \end{equation*}

% Empty chain
% \begin{equation*}
%   \infer[\jrule{C-ID}^A]{\slcof{A |- \chne : A}}{}
% \end{equation*}

% Monoid laws applies silently  ...

% Chain consisting of one process
% \begin{equation*}
%   \infer[\jrule{C-PROC}]{\slcof{A |- P : B}}{
%     \slof{A |- P : B}}
% \end{equation*}

% \section{Session-typed asynchronous process chains}

% \begin{itemize}
% \item Foreshadow theorem about relationship with communicating automata
% \end{itemize}

% By analogy with chains of communicating automata, we envision a process chain, $\chn$, as a finite sequence of processes $(P_i)_{i=1}^{n}$, each with its own independent thread of control and arranged in a linear topology.
% As depicted in the adjacent \lcnamecref{fig:singleton-processes:chain-topology},%
% %
% \begin{marginfigure}
%   \centering
%   \begin{tikzpicture}
%     \graph [math nodes, nodes={circle, draw}] {
%       P_0 / [coordinate]
%        --
%       P_1
%        --
%       / \dotsb [rectangle, text height=1ex, draw=none]
%        --
%       P_i
%        --
%       / \dotsb [rectangle, text height=1ex, draw=none]
%        --
%       P_n
%        --
%       / [coordinate];
%     };
%     \node [fit=(P_1) (P_i) (P_n), inner xsep=.5em,
%            draw,
%            label distance=2em, label=$\chn$] {};
%   \end{tikzpicture}
%   \caption{A process chain, $\chn$}\label{fig:singleton-processes:chain-topology}
% \end{marginfigure}
% %
% each process $P_i$ shares a unique channel with its left-hand neighbor and a unique channel with its right-hand neighbor.
% % Because process chains always have a linear topology, 
% These channels need not be named -- they can instead be referred to as simply the left- and right-hand channels of $P_i$.

% Process chains are never isolated from the surrounding environment.
% Both the left-hand channel of $P_1$ and the right-hand channel of $P_n$ continue to allow external communication, even as communication among neighboring processes changes the chain's internal state.


% As a string of processes, 

% Formally, then, process chains $\chn$ form a free monoid over processes $P$:
% % As a free monoidSyntactically, chains are generated by the following grammar.
% \begin{equation*}
%   \chn \Coloneqq \chne \mid (\chn_1 \cc \chn_2) \mid P
%   \,,
% \end{equation*}
% where we write $\chne$ for the empty chain and $\cc$ for the monoid operation, which is subject to the usual associativity and unit laws:
% \begin{gather*}
%   (\chn_1 \cc \chn_2) \cc \chn_3 = \chn_1 \cc (\chn_2 \cc \chn_3) \\
%   \chne \cc \chn = \chn = \chn \cc \chne
% \end{gather*}
% \Cref{fig:process-chains:chain-shapes} gives a graphical depiction of the three basic shapes that chains may take.
% %
% \begin{figure}[tbp]
%   \centering
%   \begin{tikzpicture}
%     \graph [math nodes, nodes={circle}] {
%       / [coordinate]
%        --
%       P_1 / \phantom{P}
%        --
%       / [coordinate];
%       (P_1.west) -- (P_1.east);
%     };
%     \node [fit=(P_1), inner xsep=.5em, draw,
%            label distance=2em, label={$\chn = \chne$}] {};
%   \end{tikzpicture}
%   \qquad
%   \begin{tikzpicture}
%     \graph [math nodes, nodes={circle, draw}] {
%       / [coordinate]
%        --
%       P_1 / P
%        --
%       / [coordinate];
%     };
%     \node [fit=(P_1), inner xsep=.5em, draw,
%            label distance=2em, label={$\chn = P$}] {};
%   \end{tikzpicture}

%   \begin{tikzpicture}
%     \graph [math nodes, nodes={circle, draw}] {
%       / [coordinate]
%        --
%       P_1 /
%        --
%       / \dotsb [rectangle, text height=1ex, draw=none]
%        --
%       P_n /
%        --
%       P_{n+1} /
%        --
%       / \dotsb [rectangle, text height=1ex, draw=none]
%        --
%       P_{n+m} /
%        --
%       / [coordinate];
%     };
%     \node (C1)
%           [fit=(P_1) (P_n), inner xsep=.5em, draw, dashed,
%            label distance=2em, label=$\chn_1$]
%           {};
%     \node (C2)
%           [fit=(P_{n+1}) (P_{n+m}), inner xsep=.5em, draw, dashed,
%            label distance=2em, label=$\chn_2$]
%           {};
%     \node [fit=(C1) (C2), inner xsep=.5em, draw,
%            label distance=2em, label={$\chn = \chn_1 \cc \chn_2$}] {};
%   \end{tikzpicture}
%   \caption{A graphical depiction of process chain constructors}%
%   \label{fig:process-chains:chain-shapes}
% \end{figure}

% \newthought{So far,} this definition of process chains has intentionally abstracted from what exactly a process is, and how exactly communication occurs over channels.


% \begin{figure}[tbp]
%   \vspace*{\dimexpr-\abovedisplayskip-\abovecaptionskip\relax}
%   \begin{syntax*}
%     Session types &
%       A & \alpha \mid \plus*[sub=_{\ell \in L}]{\ell:A_{\ell}} \mid \with*[sub=_{\ell \in L}]{\ell:A_{\ell}}
%     \\
%     Process expressions &
%       P & \spawn{P_1}{P_2} \mid \fwd
%         \begin{array}[t]{@{{} \mid {}}l@{}}
%           \selectR{\kay} \mid \caseL[\ell \in L]{\ell => P_{\ell}} \\
%           \caseR[\ell \in L]{\ell => P_{\ell}} \mid \selectL{\kay}
%         \end{array}
%   \end{syntax*}

%   \begin{inferences}
%     \infer[\jrule{CUT}^B]{\slof{A |- \spawn{P_1}{P_2} : C}}{
%       \slof{A |- P_1 : B} & \slof{B |- P_2 : C}}
%     \and
%     \infer[\jrule{ID}^A]{\slof{A |- \fwd : A}}{}
%     \\
%     \infer[\rrule{\plus}']{\slof{A_{\kay} |- \selectR{\kay} : \plus*[sub=_{\ell \in L}]{\ell:A_{\ell}}}}{
%       \text{($\kay \in L$)}}
%     \and
%     \infer[\lrule{\plus}]{\slof{\plus*[sub=_{\ell \in L}]{\ell:A_{\ell}} |- \caseL[\ell \in L]{\ell => P_{\ell}} : C}}{
%       \multipremise{\ell \in L}{\slof{A_{\ell} |- P_{\ell} : C}}}
%     \\
%     \infer[\rrule{\with}]{\slof{A |- \caseR[\ell \in L]{\ell => P_{\ell}} : \with*[sub=_{\ell \in L}]{\ell:C_{\ell}}}}{
%       \multipremise{\ell \in L}{\slof{A |- P_{\ell} : C_{\ell}}}}
%     \and
%     \infer[\lrule{\with}']{\slof{\with*[sub=_{\ell \in L}]{\ell:C_{\ell}} |- \selectL{\kay} : C_{\kay}}}{
%       \text{($\kay \in L$)}}
%   \end{inferences}
%   \vspace*{-\belowdisplayskip}
%   \caption{Asynchronous process chains and their session-type system}\label{fig:singleton-processes:typing-rules}
% \end{figure}
% %
% \Cref{fig:singleton-processes:typing-rules} presents the syntax of process chains and their session-type system.
% Formally, the session types are identical to the propositions of singleton logic; the process terms, identical to the Hilbert-style proof terms; and the session-typing rules, identical to the Hilbert-style inference rules.
% In fact, the whole of this \lcnamecref{fig:singleton-processes:typing-rules} is identical to \cref{fig:singleton-logic:hilbert}, save for the small difference in terminology.

% This size of this difference, however, belies its significance.
% \begin{itemize}
% \item The proof term $\spawn{P_1}{P_2}$ for composition of proofs is now reinterpreted as the expression for a process that will spawn, to the immediate left, a new thread of control for $P_1$, while the original thread of control continues with $P_2$.
%   In effect, $\spawn{P_1}{P_2}$ now composes process behaviors.

% \item The proof term $\fwd$ is reinterpreted as the expression for a process that terminates its thread of control, excising the process from the chain.

% \item The proof terms $\selectL{\kay}$ and $\selectR{\kay}$ are now viewed as messages carrying the label $\kay$ as their payloads.
%   The direction of the underlying arrow indicates the message's intended recipient: $\selectL{\kay}$ is being sent to the left-hand neighbor; $\selectR{\kay}$, to the right-hand neighbor.

% \item The proof term $\caseL[\ell \in L]{\ell => P_{\ell}}$ is reinterpreted as the expression for a process that waits to receive a message $\selectR{\kay}$ from its left-hand neighbor and then branches on the received label, so that the thread of control continues with $P_{\kay}$.
%   The proof term $\caseR[\ell \in L]{\ell => P_{\ell}}$ is interpreted dually as the expression for a process that branches on a message from its right-hand neighbor.
% \end{itemize}

% Just as session types characterize the communication behavior of individual process expressions, the same types can be used to describe the behavior of entire process chains.
% The judgment $\slcof{A |- \chn : B}$ indicates that the process chain $\chn$ is well-typed, with the left-hand channel of $\chn$ having type $A$ and the right-hand channel having type $B$.
% % Using the session-type system for process expressions, process chains can also be typed.
% % The judgment $\slcof{A |- \chn : C}$ indicates that the left-hand channel of chain $\chn$ has type $A$ and the right-hand channel of $\chn$ has type $C$.

% The simplest chain is the one that consists of a single process $P$; the chain inherits the process's type:
% \begin{equation*}
%   \infer[\jrule{C-PROC}]{\slcof{A |- P : C}}{
%     \slof{A |- P : C}}
% \end{equation*}

% The composition $\chn_1 \cc \chn_2$ is typable if the two chains assign the same type to their shared channel.
% \begin{equation*}
%   \infer[\jrule{C-CUT}^B]{\slcof{A |- \chn_1 \cc \chn_2 : C}}{
%     \slcof{A |- \chn_1 : B} & \slcof{B |- \chn_2 : C}}
% \end{equation*}
% \begin{equation*}
%   \infer[\jrule{C-ID}^A]{\slcof{A |- \chne : A}}{}
% \end{equation*}

% \begin{figure}[tbp]
%   \begin{syntax*}
%     Process chains &
%       \chn & \chne \mid (\chn_1 \cc \chn_2) \mid P
%   \end{syntax*}

%   \begin{inferences}
%     \infer[\jrule{CUT}^B]{\slcof{A |- \chn_1 \cc \chn_2 : C}}{
%       \slcof{A |- \chn_1 : B} & \slcof{B |- \chn_2 : C}}
%     \and
%     \infer[\jrule{ID}^A]{\slcof{A |- \chne : A}}{}
%     \and
%     \infer[\jrule{PROC}]{\slcof{A |- P : C}}{
%       \slof{A |- P : C}}
%   \end{inferences}

%   \begin{gather*}
%     (\chn_1 \cc \chn_2) \cc \chn_3 = \chn_1 \cc (\chn_2 \cc \chn_3) \\
%     \chne \cc \chn = \chn = \chn \cc \chne
%   \end{gather*}
%   \vspace*{-\abovedisplayskip}
%   \caption{Syntax and session-typing rules for process chains}%
%   \label{fig:process-chains:session-types}
% \end{figure}

% Chains can be reified as process expressions.
% Let $\pf*{-}$ be a function from chains to process expressions given by 
% \begin{align*}
%     \pf*{\chne} &= \fwd \\
%     \pf*{\chn_1 \cc \chn_2} &= \spawn{\pf{\chn_1}}{\pf{\chn_2}} \\
%     \pf{P} &= P
% \end{align*}

% \begin{theorem}
%   If $\slcof{A |- \chn : B}$, then $\slof{A |- \pf{\chn} : B}$.
% \end{theorem}

% \subsection{From admissibility of non-analytic cuts to an operational semantics}

% In the previous \lcnamecref{ch:singleton-logic}, we presented a procedure for normalizing Hilbert-style [singleton?] proofs.
% Full proof normalization was important to ...

% In this \lcnamecref{ch:process-chains}, however, our perspective has shifted from proof theory to concurrent computation, from proofs to processes.
% And so full normalization is no longer appropriate -- we now want to expose the concurrent computational behavior, not just ...
% The situation is analogous to that of intuitionistic natural deduction and simply-typed functional computation:

% In fact, the difference is even starker here because, once recursive process definitions are introduced\parencref{sec:??}, many useful processes will be nonterminating.
% Thus, there is no clear notion of value, as exists in functional computation.
% Nevertheless, in good Curry--Howard fashion, the principal cases of Hilbert-style proof normalization will still directly inform the operational semantics of processes.

% \begin{itemize}
% \item Operational semantics does not observe processes, observes only messages
% \end{itemize}

% \newthought{In the previous \lcnamecref{sec:??},} the description of how proof terms are reinterpreted as process expressions already hinted at a computational strategy.
% Here we present that operational semantics in its full detail.


% At the heart of the operational semantics for process chains is \emph{reduction}, a binary relation on chains which we write as $\reduces$.
% Reductions may occur among any of the chain's processes, and thus the relation is compatible with the monoid operation, $\cc$:
% \begin{inferences}
%   \infer{\chn_1 \cc \chn_2 \reduces \chn'_1 \cc \chn_2}{
%     \chn_1 \reduces \chn'_1}
%   \and
%   \infer{\chn_1 \cc \chn_2 \reduces \chn_1 \cc \chn'_2}{
%     \chn_2 \reduces \chn'_2}
% \end{inferences}

% A process $\spawn{P_1}{P_2}$ spawns, to its immediate left, a new thread of control for $P_1$, while the original thread of control continues with $P_2$.
% \begin{equation*}
%   \infer{\spawn{P_1}{P_2} \reduces P_1 \cc P_2}{}
% \end{equation*}
% Because process chains are always ... up to associativity and unit laws, these reductions 

% Recall 
% \begin{gather*}
%   \begin{aligned}
%     \nspawn{(\spawn{N_0}{\selectR{\kay}})}{M}
%       &= \nspawn{N_0}{(\nspawn{\selectR{\kay}}{M})}
%     \\
%     \nspawn{N}{(\spawn{\selectL{\kay}}{M_0})}
%       &= \nspawn{(\nspawn{N}{\selectL{\kay}})}{M_0}
%   \end{aligned}
%   \\[2\jot]
%   \begin{aligned}
%     \nspawn{\fwd}{M}
%       &= M
%     \\
%     \nspawn{N}{\fwd}
%       &= N
%   \end{aligned}
%   \\[2\jot]
%   \begin{aligned}
%     \nspawn{\selectR{\kay}}{\caseL[\ell \in L]{\ell => M_{\ell}}}
%       &= M_{\kay}
%     \\
%     \nspawn{\caseR[\ell \in L]{\ell => N_{\ell}}}{\selectL{\kay}}
%       &= N_{\kay}
%   \end{aligned}
%   \\[2\jot]
%   \begin{aligned}
%     \nspawn{(\spawn{\selectL{\kay}}{N_0})}{M}
%       &= \spawn{\selectL{\kay}}{(\nspawn{N_0}{M})}
%     \\
%     \nspawn{N}{(\spawn{M_0}{\selectR{\kay}})}
%       &= \spawn{(\nspawn{N}{M_0})}{\selectR{\kay}}
%     \\
%     \nspawn{\selectL{\kay}}{M}
%       &= \spawn{\selectL{\kay}}{M}
%     \\
%     \nspawn{N}{\selectR{\kay}}
%       &= \spawn{N}{\selectR{\kay}}
%     \\
%     \nspawn{\caseL[\ell \in L]{\ell => N_{\ell}}}{M}
%       &= \caseL[\ell \in L]{\ell => \nspawn{N_{\ell}}{M}}
%     \\
%     \nspawn{N}{\caseR[\ell \in L]{\ell => M_{\ell}}}
%       &= \caseR[\ell \in L]{\ell => \nspawn{N}{M_{\ell}}}
%   \end{aligned}
% \end{gather*}
% etc.

% \begin{itemize}
% \item The operational semantics uses a particular strategy: $\reduces$ is the least compatible relation that satisfies the following.
%   \begin{gather*}
%     \spawn{P_1}{P_2} \reduces P_1 \cc P_2 \\
%     \fwd \reduces \cnfe \\
%     \selectR{\kay} \cc \caseL[\ell \in L]{\ell => P_{\ell}} \reduces P_{\kay} \\
%     \caseR[\ell \in L]{\ell => P_{\ell}} \cc \selectL{\kay} \reduces P_{\kay}
%   \end{gather*}
%   We denote the reflexive, transitive closure of $\reduces$ by $\Reduces$.

%   \begin{theorem}[Type preservation]
%     If $\slcof{A |- \chn : B}$ and $\chn \reduces \chn'$, then $\slcof{A |- \chn' : B}$.
%   \end{theorem}
%   %
%   \begin{proof}
%     By structural induction on the given chain.
%   \end{proof}


%   \begin{lemma}
%     If $\slcof{A |- \chn : B}$ and $\chn = \chn'$, then $\slcof{A |- \chn' : B}$.
%   \end{lemma}

%   \begin{theorem}[Progress]
%     If $\slcof{A |- \chn : B}$, then either:
%     \begin{itemize}
%     \item $\chn \reduces \chn'$ for some $\chn'$;
%     \item $\chn$ is empty: $\chn = \chne$;
%     \item $\chn$ is ready to communicate along its left-hand channel: $\chn = \selectL{\kay} \cc \chn_0$ or $\chn = \caseL[\ell \in L]{\ell => P_{\ell}} \cc \chn_0$ for some $\chn_0$; or
%     \item $\chn$ is ready to communicate along its right-hand channel: $\chn = \chn_0 \cc \selectR{\kay}$ or $\chn = \chn_0 \cc \caseR[\ell \in L]{\ell => P_{\ell}}$ for some $\chn_0$.
%     \end{itemize}
%   \end{theorem}
%   %
%   \begin{proof}
%     By structural induction on the given process chain.
%   \end{proof}

%   \begin{theorem}
%     $\chn^\sharp \Reduces \chn$ for all $\chn$.
%   \end{theorem}
%   \begin{proof}
%     By structural induction on the given chain.
%   \end{proof}

%   % \begin{conjecture}
%   %   If $\cnf_0 \Reduces \cnf \longarrownot\reduces$, then $\wn{\cnf_0^\sharp} = \cnf^\sharp$.
%   %   \begin{itemize}
%   %   \item If $\cnf_0 \longarrownot\reduces$, then $\wn{\cnf_0^\sharp} = \cnf_0^\sharp$.
%   %   \item If $\cnf_0 \reduces \cnf_1$ and $\wn{\cnf_1^\sharp} = P$, then $\wn{\cnf_0^\sharp} = P$, 
%   %   \end{itemize}
%   % \end{conjecture}

%   % \begin{corollary}
%   %   If $P \Reduces \cnf \longarrownot\reduces$, then $\wn{P} = \cnf^\sharp$.
%   % \end{corollary}
% \end{itemize}




% \begin{example}
%   An expression for a process that will wait for an $a$- or $b$-message to arrive from its left-hand neighbor and then send to its right-hand neighbor either two consecutive $a$-messages or a single $b$-message, respectively, is:
%   \begin{equation*}
%     \slof{
%       \plus*{a:\epsilon, b:\epsilon}
%       |-
%       \caseL{a => \spawn{\selectR{a}}{\selectR{a}}
%            | b => \selectR{b}}
%       :
%       \plus*{a:\plus*{a:\epsilon}, b:\epsilon}
%     }
%     \mathrlap{\,.}
%   \end{equation*}
%   Indeed, the process chain in which that process is sent an $a$-message computes as follows.
%   \begin{equation*}
%     \selectR{a} \cc \caseL{a => \spawn{\selectR{a}}{\selectR{a}}
%                          | b => \selectR{b}}
%       \reduces \spawn{\selectR{a}}{\selectR{a}}
%       \reduces \selectR{a} \cc \selectR{a}
%   \end{equation*}
% \end{example}



%%% Local Variables:
%%% mode: latex
%%% TeX-master: "thesis"
%%% End:

%% \chapter{Session-typed processes}\label{ch:singleton-processes}

\begin{itemize}
\item Connections to SILL
\end{itemize}

\section{Session-typed processes: A Curry--Howard interpretation of singleton logic}

\begin{syntax*}
  Session types &
    A & \alpha \mid \plus*[sub=_{\ell \in L}]{\ell:A_{\ell}} \mid \with*[sub=_{\ell \in L}]{\ell:A_{\ell}}
  \\
  Process terms &
    P,Q & \spawn{P}{Q} \mid \fwd
            \begin{array}[t]{@{{} \mid {}}l@{}}
              \selectR{\kay} \mid \caseL[\ell \in L]{\ell => Q_{\ell}} \\
              \caseR[\ell \in L]{\ell => P_{\ell}} \mid \selectL{\kay}
            \end{array}
\end{syntax*}

\begin{inferences}
  \infer[\jrule{CUT}^A]{\slof{\sctx |- \spawn{P}{Q} : C}}{
    \slof{\sctx |- P : A} & \slof{A |- Q : C}}
  \and
  \infer[\jrule{ID}^A]{\slof{A |- \fwd : A}}{}
  \\
  \infer[\rrule{\plus}]{\slof{A_{\kay} |- \selectR{\kay} : \plus*[sub=_{\ell \in L}]{\ell:A_{\ell}}}}{
    \text{($\kay \in L$)}}
  \and
  \infer[\lrule{\plus}]{\slof{\plus*[sub=_{\ell \in L}]{\ell:A_{\ell}} |- \caseL[\ell \in L]{\ell => Q_{\ell}} : C}}{
    \multipremise{\ell \in L}{\slof{A_{\ell} |- Q_{\ell} : C}}}
  \\
  \infer[\rrule{\with}]{\slof{\sctx |- \caseR[\ell \in L]{\ell => P_{\ell}} : \with*[sub=_{\ell \in L}]{\ell:A_{\ell}}}}{
    \multipremise{\ell \in L}{\slof{\sctx |- P_{\ell} : A_{\ell}}}}
  \and
  \infer[\lrule{\with}]{\slof{\with*[sub=_{\ell \in L}]{\ell:A_{\ell}} |- \selectL{\kay} : A_{\kay}}}{
    \text{($\kay \in L$)}}
\end{inferences}

\subsection{Cut reduction}

\begin{gather*}
  \infer[\jrule{CUT}]{\slof{A_{\kay} |- \spawn{\selectR{\kay}}{\caseL[\ell \in L]{\ell => Q_{\ell}}} : C}}{
  \infer[\rrule{\plus}]{\slof{A_{\kay} |- \selectR{\kay} : \plus*[sub=_{\ell \in L}]{\ell:A_{\ell}}}}{
    \text{($\kay \in L$)}} &
  \infer[\lrule{\plus}]{\slof{\plus*[sub=_{\ell \in L}]{\ell:A_{\ell}} |- \caseL[\ell \in L]{\ell => Q_{\ell}} : C}}{
    \multipremise{\ell \in L}{\slof{A_{\ell} |- Q_{\ell} : C}}}}
  \\
  \cutreduces
  \\
  \slof{A_{\kay} |- Q_{\kay} : C}
\end{gather*}

Discussion of cut reduction vs.\ admissibility

\subsection{An operational semantics}

\begin{syntax*}
  Configurations &
    \cnf & \cnfe \mid (\cnf_1 \cc \cnf_2) \mid P
\end{syntax*}

\begin{inferences}
  \infer[\jrule{CUT}^{\sctx'}]{\slcof{\sctx |- \cnf_1 \cc \cnf_2 : \sctx''}}{
    \slcof{\sctx |- \cnf_1 : \sctx'} & \slcof{\sctx' |- \cnf_2 : \sctx''}}
  \and
  \infer[\jrule{ID}^{\sctx}]{\slcof{\sctx |- \cnfe : \sctx}}{}
  \and
  \infer[\jrule{PROC}]{\slcof{\sctx |- P : A}}{
    \slof{\sctx |- P : A}}
\end{inferences}

\begin{inferences}
  \infer{\spawn{P}{Q} \reduces P \cc Q}{}
  \and
  \infer{\fwd \reduces \cnfe}{}
  \\
  \infer{\selectR{\kay} \cc \caseL[\ell \in L]{\ell => Q_{\ell}} \reduces Q_{\kay}}{}
  \and
  \infer{\caseR[\ell \in L]{\ell => P_{\ell}} \cc \selectL{\kay} \reduces P_{\kay}}{}
\end{inferences}

\begin{itemize}
\item SSOS is an ordered rewriting specification.  (How does this work with definitions?)
\end{itemize}

\begin{equation*}
  \begin{lgathered}
    \proc{\spawn{P}{Q}} \defd \proc{P} \fuse \proc{Q} \\
    \proc{\fwd} \defd \one \\
    \proc{\caseL[\ell \in L]{\ell => Q_{\ell}}} \defd \bigwith_{\ell \in L}\bigl(\proc{\selectR{\ell}} \limp \proc{Q_{\ell}}\bigr) \\
    \proc{\caseR[\ell \in L]{\ell => P_{\ell}}} \defd \bigwith_{\ell \in L}\bigl(\proc{P_{\ell}} \pmir \proc{\selectL{\ell}}\bigr)
  \end{lgathered}
\end{equation*}

\paragraph{Hypersequent}



\subsection{Example: Binary counter}

\begin{equation*}
  \proc{b_1} \defd (\msgL{i} \fuse \proc{b_0} \pmir \msgL{i}) \with (\proc{b_0} \fuse \msgR{s} \pmir \msgL{d})
\end{equation*}

\subsection{Example: \Aclp*{DFA}}

Contrast with inability to express \acp{NFA} (languages vs.\ operational semantics)

%%% Local Variables:
%%% mode: latex
%%% TeX-master: "thesis"
%%% End:

%% \chapter{Typed bisimilarity}\label{ch:typed-bisim}

\begin{definition}
  \emph{Typed bisimilarity} is the largest \emph{symmetric} typed binary relation that satisfies the following conditions.
  \begin{thmdescription}
  \item[Output bisimilarity]
    If $\slcof{\sctx |- \cnf' \cc \selectR{\kay} \secudeR\tsim \dnf : \plus*[sub=_{\ell \in L}]{\ell:A_{\ell}}}$, then $\dnf \Reduces \dnf' \cc \selectR{\kay}$ for some $\dnf'$ such that $\slcof{\sctx |- \cnf' \tsim \dnf' : A_{\kay}}$.
    Symmetrically, if $\slcof{\with*[sub=_{\ell \in L}]{\ell:A_{\ell}} |- \selectL{\kay} \cc \cnf' \secudeR\tsim \dnf : \sctx'}$, then $\dnf \Reduces \selectL{\kay} \cc \dnf'$ for some $\dnf'$ such that $\slcof{A_{\kay} |- \cnf' \tsim \dnf' : \sctx'}$.

  \item[Input bisimilarity]
    If $\slcof{\sctx |- \cnf \tsim \dnf : \with*[sub=_{\ell \in L}]{\ell:A_{\ell}}}$ and $\cnf' \secudeR \cnf \cc \selectL{\kay}$, then $\slcof{\sctx |- \dnf \cc \selectL{\kay} \Reduces\mist \cnf' : A_{\kay}}$.
    Symmetrically, if $\slcof{\plus*[sub=_{\ell \in L}]{\ell:A_{\ell}} |- \cnf \tsim \dnf : \sctx'}$ and $\cnf' \secudeR \selectR{\kay} \cc \cnf$, then $\slcof{A_{\kay} |- \selectR{\kay} \cc \dnf \Reduces\mist \cnf' : \sctx'}$.

  \item[Reduction bisimilarity]
    If $\slcof{\sctx |- \dnf \mist\Reduces \cnf' : \sctx'}$, then $\slcof{\sctx |- \dnf \Reduces\mist \cnf' : \sctx'}$.
  \end{thmdescription}
\end{definition}

What about reduction closure?
Compare with ordered bisimilarity.


\begin{lemma}
  Let $\simu{R}$ be a symmetric typed binary relation for which $\slcof{\sctx |- \dnf \simu{R}^{-1}\reduces \cnf' : \sctx'}$ implies $\slcof{\sctx |- \dnf \Reduces\simu{R}^{-1} \cnf' : \sctx'}$.
  Then $\slcof{\sctx |- \dnf \congrel*{\refl{\simu{R}}}^{-1}\reduces \cnf' : \sctx'}$ implies $\slcof{\sctx |- \dnf \Reduces\congrel*{\refl{\simu{R}}}^{-1} \cnf' : \sctx'}$.
\end{lemma}


\begin{conjecture}
  Let $\simu{R}$ be a typed binary relation that satisfies the following conditions.
  Then $\simu{R}$ is included in typed bisimilarity.
  \begin{thmdescription}
  \item[Immediate output bisimulation]
    If $\slcof{\sctx |- (\cnf' \cc \selectR{\kay}) \simu{R} \dnf : \plus*[sub=_{\ell \in L}]{\ell:A_{\ell}}}$, then $\dnf \Reduces \dnf' \cc \selectR{\kay}$ for some $\dnf'$ such that $\slcof{\sctx |- \cnf' \refl{\simu{R}} \dnf' : A_{\kay}}$.
    Symmetrically, if $\slcof{\with*[sub=_{\ell \in L}]{\ell:A_{\ell}} |- (\selectL{\kay} \cc \cnf') \simu{R} \dnf : \sctx'}$, then $\dnf \Reduces \selectL{\kay} \cc \dnf'$ for some $\dnf'$ such that $\slcof{A_{\kay} |- \cnf' \refl{\simu{R}} \dnf' : \sctx'}$.

  \item[Immediate input bisimulation]
    If $\slcof{\sctx |- (\cnf' \cc \caseR[\ell \in L]{\ell => P_{\ell}}) \simu{R} \dnf : \with*[sub=_{\ell \in L}]{\ell:A_{\ell}}}$, then $\slcof{\sctx |- \dnf \cc \selectL{\kay} \Reduces\refl*{\simu{R}}^{-1} \cnf' \cc P_{\kay} : A_{\kay}}$, for all $\kay \in L$.
    Symmetrically, if $\slcof{\plus*[sub=_{\ell \in L}]{\ell:A_{\ell}} |- (\caseL[\ell \in L]{\ell => Q_{\ell}} \cc \cnf') \simu{R} \dnf : \sctx'}$, then $\slcof{A_{\kay} |- \selectR{\kay} \cc \dnf \Reduces\refl*{\simu{R}}^{-1} Q_{\kay} \cc \cnf' : \sctx'}$, for all $\kay \in L$.

  \item[Reduction bisimulation]
    If $\slcof{\sctx |- \dnf \simu{R}^{-1}\reduces \cnf' : \sctx'}$, then $\slcof{\sctx |- \dnf \Reduces\refl*{\simu{R}}^{-1} \cnf' : \sctx'}$.

  \item[Emptiness bisimulation]
  \end{thmdescription}
\end{conjecture}
%
\begin{proof}
  We must establish the output, input, and reduction bisimilarity properties:
  \begin{description}
  \item[Output bisimilarity]
    Assume that $\slcof{\sctx |- \dnf \congrel*{\refl{\simu{R}}}^{-1}\Reduces \cnf' \cc \selectR{\kay} : \plus*[sub=_{\ell \in L}]{\ell:A_{\ell}}}$;
    we must exhibit $\dnf'$ such that $\dnf \Reduces \dnf' \cc \selectR{\kay}$ and $\slcof{\sctx |- \cnf' \congrel{\refl{\simu{R}}} \dnf' : A_{\kay}}$.

    Because $\congrel{\refl{\simu{R}}}$ is a reduction bisimulation\pcref{lem:congrel-reduction-bisim}, $\slcof{\sctx |- \dnf \Reduces\congrel*{\refl{\simu{R}}}^{-1} \cnf' \cc \selectR{\kay} : \plus*[sub=_{\ell \in L}]{\ell:A_{\ell}}}$.
    The relation $\congrel{\refl{\simu{R}}}$ is also an imediate output bisimulation, so, indeed, $\dnf \Reduces \dnf' \cc \selectR{\kay}$ for some $\dnf'$ such that $\slcof{\sctx |- \cnf' \congrel{\refl{\simu{R}}} \dnf' : A_{\kay}}$.
    \begin{itemize}
    \item In case $\slcof{\sctx |- \dnf \Reduces= \cnf' \cc \selectR{\kay} : \plus*[sub=_{\ell \in L}]{\ell:A_{\ell}}}$, then choose $\dnf'$ to be $\cnf'$ and note that $\slcof{\sctx |- \cnf' \refl{\simu{R}} \dnf' : A_{\kay}}$.
    \item In case $\slcof{\sctx |- \dnf \Reduces\simu{R}^{-1} \cnf' \cc \selectR{\kay} : \plus*[sub=_{\ell \in L}]{\ell:A_{\ell}}}$, then appeal to immediate output bisimilarity: there must exist a configuration $\dnf'$ such that $\dnf \Reduces\Reduces \dnf' \cc \selectR{\kay}$ and $\slcof{\sctx |- \cnf' \refl{\simu{R}} \dnf' : A_{\kay}}$.
    \end{itemize}
    In either case, $\dnf \Reduces \dnf' \cc \selectR{\kay}$ for some $\dnf'$ such that $\slcof{\sctx |- \cnf' \refl{\simu{R}} \dnf' : A_{\kay}}$.

  \item[Input bisimilarity]
    Assume that $\slcof{\sctx |- \cnf \congrel{\refl{\simu{R}}} \dnf : \with*[sub=_{\ell \in L}]{\ell:A_{\ell}}}$ and $\cnf \cc \selectL{\kay} \Reduces \cnf'$;
    we must show that $\slcof{\sctx |- \dnf \cc \selectL{\kay} \Reduces\congrel*{\refl{\simu{R}}}^{-1} \cnf' : A_{\kay}}$.

    By the definition of $\congrel{\refl{\simu{R}}}$, we have $\slcof{\sctx |- \dnf \cc \selectL{\kay} \congrel*{\refl{\simu{R}}}^{-1}\Reduces \cnf' : A_{\kay}}$.
    Because $\congrel{\refl{\simu{R}}}$ is a reduction bisimulation\pcref{lem:congrel-reduction-bisim}, $\slcof{\sctx |- \dnf \cc \selectL{\kay} \Reduces\congrel*{\refl{\simu{R}}}^{-1} \cnf' : A_{\kay}}$, as required.

    
    \begin{itemize}
    \item In case $\cnf \cc \selectL{\kay} = \cnf'$, then...
    \item In case $\cnf \cc \selectL{\kay} \reduces\Reduces \cnf'$, there are two subcases according to whether the message is read during the first reduction step.
      \begin{itemize}
      \item Consider the subcase in which the message is read during the first step -- that is, $\cnf = \cnf'_0 \cc \caseR[\ell \in L]{\ell => P_{\ell}}$ and $\cnf'_0 \cc P_{\kay} \Reduces \cnf'$.
        By immediate input bisimilarity, $\slcof{\sctx |- \dnf \cc \selectL{\kay} \Reduces\refl*{\simu{R}}^{-1} \cnf'_0 \cc P_{\kay} : A_{\kay}}$.
        Upon appealing to reduction bisimilarity, we obtain the required $\slcof{\sctx |- \dnf \cc \selectL{\kay} \Reduces\refl*{\simu{R}}^{-1} \cnf' : A_{\kay}}$.
      \item Consider the subcase in which the message is \emph{not} read during the first step -- that is, $\cnf \reduces \cnf'_0$ and $\cnf'_0 \cc \selectL{\kay} \Reduces \cnf'$ for some $\cnf'_0$.
        By reduction bisimilarity, $\slcof{\sctx |- \dnf \Reduces\refl*{\simu{R}}^{-1} \cnf'_0 : \with*[sub=_{\ell \in L}]{\ell:A_{\ell}}}$.
        \begin{itemize}
        \item In case $\slcof{\sctx |- \dnf \Reduces= \cnf'_0 : \with*[sub=_{\ell \in L}]{\ell:A_{\ell}}}$, then the frame property for reduction yields $\slcof{\sctx |- \dnf \cc \selectL{\kay} \Reduces= \cnf' : A_{\kay}}$.
        \item In case $\slcof{\sctx |- \dnf \Reduces\simu{R}^{-1} \cnf'_0 : \with*[sub=_{\ell \in L}]{\ell:A_{\ell}}}$, then we may obtain $\slcof{\sctx |- \dnf \cc \selectL{\kay} \Reduces\refl*{\simu{R}}^{-1} \cnf' : A_{\kay}}$ by appealing to the inductive hypothesis (and the frame property for reduction).
        \end{itemize}
      \end{itemize}
    \end{itemize}
    In all cases, $\slcof{\sctx |- \dnf \cc \selectL{\kay} \Reduces\refl*{\simu{R}}^{-1} \cnf' : A_{\kay}}$.

  \item[Reduction bisimilarity]
  \end{description}
\end{proof}


What about closed bisimilarity?
Transducers from the example would be closed bisimilar, I think, but they are not (open) bisimilar.


\begin{definition}
  \emph{Typed closed bisimilarity} is the largest \emph{symmetric} typed binary relation that satisfies the following conditions.
  \begin{thmdescription}
  \item[Closure]
    If $\sctx$ is a compound type, then $\slcof{\sctx |- \cnf \tsim \dnf : \sctx'}$ if and only if $\slcof{\alpha |- \cnf_0 \tsim \dnf_0 : \sctx}$ implies $\slcof{\alpha |- \cnf_0 \cc \cnf \tsim \dnf_0 \cc \dnf : \sctx'}$.
  \item[Output bisimilarity]
    If $\slcof{\alpha |- \cnf \tsim \dnf : \plus*[sub=_{\ell \in L}]{\ell:A_{\ell}}}$ and $\cnf \Reduces \cnf' \cc \selectR{\kay}$, then $\dnf \Reduces \dnf' \cc \selectR{\kay}$ for some $\dnf'$ such that $\slcof{\alpha |- \cnf' \tsim \dnf' : A_{\kay}}$.
  \item[Input bisimilarity]
    If $\slcof{\alpha |- \cnf \tsim \dnf : \with*[sub=_{\ell \in L}]{\ell:A_{\ell}}}$ and $\cnf \cc \selectL{\kay} \Reduces \cnf'$, then $\slcof{\alpha |- \dnf \cc \selectL{\kay} \Reduces\mist \cnf' : A_{\kay}}$.
  \end{thmdescription}
\end{definition}


\begin{conjecture}
  Let $\simu{R}$ be a typed binary relation that satisfies the following conditions.
  Then $\simu{R}$ is included in closed typed bisimilarity.
  \begin{thmdescription}
  \item[Closure]
    If $\sctx$ is a compound type, then $\slcof{\sctx |- \cnf \tsim \dnf : \sctx'}$ if and only if $\slcof{\alpha |- \cnf_0 \tsim \dnf_0 : \sctx}$ implies $\slcof{\alpha |- \cnf_0 \cc \cnf \tsim \dnf_0 \cc \dnf : \sctx'}$.
  \item[Immediate output bisimulation]
    If $\slcof{\alpha |- \cnf' \cc \selectR{\kay} \simu{R} \dnf : \plus*[sub=_{\ell \in L}]{\ell:A_{\ell}}}$, then $\dnf \Reduces \dnf' \cc \selectR{\kay}$ for some $\dnf'$ such that $\slcof{\alpha |- \cnf' \simu{R} \dnf' : A_{\kay}}$.
  \item[Immediate input bisimulation]
    If $\slcof{\alpha |- \cnf' \cc \caseR[\ell \in L]{\ell => P_{\ell}} \simu{R} \dnf : \with*[sub=_{\ell \in L}]{\ell:A_{\ell}}}$, then $\slcof{\alpha |- \dnf \cc \selectL{\kay} \Reduces\simu{R}^{-1} \cnf' \cc P_{\kay} : A_{\kay}}$.
  \item[Reduction bisimulation]
    If $\slcof{\alpha |- \dnf \simu{R}^{-1}\reduces \cnf' : A}$, then $\slcof{\alpha |- \dnf \Reduces\simu{R}^{-1} \cnf' : A}$.
  \end{thmdescription}
\end{conjecture}


\subsection{Example}

Given the type $\alpha \defd \plus*{a:\alpha}$, the following are well-typed at $\slseq{\alpha |- \alpha}$.
\begin{equation*}
  q \defd \caseL{a => \spawn{q}{\selectR{a}}}
  \qquad
  \begin{lgathered}[t]
    s_0 \defd \caseL{a => s_1} \\
    s_1 \defd \caseL{a => \spawn{\spawn{s_0}{\selectR{a}}}{\selectR{a}}}
  \end{lgathered}
\end{equation*}
We would like to prove that $q$ and $s_0$ are typed-bisimilar configurations, but that is false for open typed bisimilarity.
Suppose, for the sake of deriving a contradiction, that $q \tsim s_0$.
By input bisimilarity, $\slcof{\alpha |- \selectR{a} \cc s_0 \Reduces\mist q \cc \selectR{a} : \alpha}$.
By output bisimilarity, $\selectR{a} \cc s_0 \Reduces \dnf' \cc \selectR{a}$ for some $\dnf'$ such that $\slcof{\alpha |- q \tsim \dnf' : \alpha}$.
This is impossible because $\selectR{a} \cc s_0$ reduces only to $s_1$ (and itself).


\begin{equation*}
  g \defd \spawn{g}{\selectR{a}}
\end{equation*}

\begin{description}
\item{}
\item[Immediate input bisimulation]
  Assume that $\slcof{\epsilon |- g \cc q \simu{R} g \cc s_0 : \alpha}$;
  we must show that $\slcof{\epsilon |- g \cc s_0 \ \tsim \dnf' : \alpha}$.
\end{description}

%%% Local Variables:
%%% mode: latex
%%% TeX-master: "thesis"
%%% End:


\part[Relationship between proof construction and reduction]{Relationship between\\proof construction\\and proof reduction}\label{part:comparison}

\chapter{From processes to rewriting}\label{ch:correspond}

\fixnote{Introduction}

\section{Embedding process configurations in formula-as-process ordered rewriting}

Here we give an embedding, $\trconf{}$, of process configurations into ordered contexts from the formula-as-process rewriting framework of \cref{ch:formula-as-process}.
% More specifically, we first define two mutually inductive translations, $\trproc[+]{}$ and $\trproc[-]{}$, that map process expressions to positive and negative propositions, respectively, and then define a homomorphism, $\trconf{}$, that maps process configurations to ordered contexts.
For the embedding to be adequate, it must preserve the dynamics of processes -- $\trconf{}$ must serve as a bisimulation between the reduction semantics for process configurations and the formula-as-process rewriting semantics for ordered contexts.%
\footnote{Because they relate process configuration reductions to ordered rewritings, these bisimulations are not rewriting bisimulations of the kind introduced in \cref{??}.
  Instead, they are ordinary, unlabeled \enquote{reduction bisimulations}, using the terminology of \textcite{Sangiorgi+Walker:??}.}

Ideally, the embedding $\trconf{}$ will be a \emph{strong} bisimulation, so that the correspondence is lockstep.
Because the embedding will be a total function, it will be a strong bisimulation if the diagrams
\begin{equation*}
  \begin{tikzcd}
    \cnf \rar[reduces] \dar[relation][swap]{\trconf{}} & \cnf\mathrlap{'} \dar[relation]{\trconf{}}
    \\
    \trconf{\cnf} \rar[reduces, exists] & \trconf{\cnf'}
  \end{tikzcd}
  \qquad\text{and}\qquad
  \begin{tikzcd}
    \cnf \rar[reduces, exists] \dar[relation][swap]{\trconf{}} & \cnf\mathrlap{'} \dar[relation, exists]{\trconf{}}
    \\
    \trconf{\cnf} \rar[reduces] & \octx\mathrlap{'}% \,.}
  \end{tikzcd}% \hphantom{'\,.}
\end{equation*}
hold.
But strong bisimulations are often elusive, so we will be satisfied to settle for a weak bisimulation if necessary.

In addition, the embedding $\trconf{}$ ought to be a monoid homomorphism from process configurations to ordered contexts, so we define
\begin{equation*}
  \begin{aligned}
    \trconf{\cnf_1 \cc \cnf_2} &= \trconf{\cnf_1} \oc \trconf{\cnf_2} \\
    \trconf{\cnfe} &= (\octxe)
  \end{aligned}
\end{equation*}
with a clause for $\trconf{P}$ that remains to be filled in.

By varying how that $\trconf{P}$ clause is filled in and whether we are working with weak focusing or full focusing (hereafter \emph{strong} focusing), we will arrive at three closely related embeddings: a weakly focused, strongly bisimilar embedding~\parencref{sec:embed:weak-focused-strong-bisim}; a strongly focused, strongly bisimilar embedding~\parencref{sec:embed:strong-focused-strong-bisim}; and a strongly focused, weakly bisimilar embedding~\parencref{sec:embed:strong-focused-weak-bisim}.
Of these three embeddings, we prefer the first and last because they most directly expose an appealing correspondence between process constructors and logical connectives.


\subsection{A weakly focused, strongly bisimilar embedding}\label{sec:embed:weak-focused-strong-bisim}


% Ideally, $\trconf{}$ ought to be a strong bisimulation, so that the operational correspondence is as tight as possible.
% Because the translation will be a total function, the strong bisimulation requirement amounts to the diagrams
% \begin{equation*}
%   \begin{tikzcd}
%     \cnf \rar[reduces] \dar[relation][swap]{\trconf{}} & \cnf\mathrlap{'} \dar[relation]{\trconf{}}
%     \\
%     \trconf{\cnf} \rar[reduces, exists] & \trconf{\cnf'}
%   \end{tikzcd}
%   \qquad\text{and}\qquad
%   \begin{tikzcd}
%     \cnf \rar[reduces, exists] \dar[relation][swap]{\trconf{}} & \cnf\mathrlap{'} \dar[relation, exists]{\trconf{}}
%     \\
%     \trconf{\cnf} \rar[reduces] & \octx\mathrlap{'\,.}
%   \end{tikzcd}\hphantom{'\,.}
% \end{equation*}

% \begin{equation*}
%   \begin{tikzcd}
%     \cnf \rar[reduces] \dar[relation][swap]{\trconf{}} & \cnf\mathrlap{'} \dar[relation]{\trconf{}}
%     \\
%     \trconf{\cnf} \rar[Reduces, exists] & \trconf{\cnf'}
%   \end{tikzcd}
%   \qquad\text{and}\qquad
%   \begin{tikzcd}
%     \cnf \rar[Reduces, exists] \dar[relation][swap]{\trconf{}} & \cnf\mathrlap{'} \dar[relation, exists]{\trconf{}}
%     \\
%     \trconf{\cnf} \rar[reduces] & \octx\mathrlap{'\,.}
%   \end{tikzcd}\hphantom{'\,.}
% \end{equation*}

For the first embedding, we will fill in the $\trconf{P}$ clause as
\begin{equation*}
  \trconf{P} = \trproc{P}
\end{equation*}
where $\trproc{}$ is an auxiliary function that embeds process expressions.
(For the moment, we will ignore coinductively defined process expressions because they introduce a few complications.)
% So that $\trconf{\cnf}$ is a well-formed ordered context in the formula-as-process focused ordered rewriting framework, $\trproc{}$ should map process expressions to either atoms or negative propositions.%
% \footnote{Recall from \cref{??} that formula-as-process ordered contexts may not contain non-atomic positive propositions.}

With this definition for $\trconf{}$ in hand, we can then run each process reduction axiom through the first bisimulation diagram to generate constraints on $\trproc{}$ that must be satisfied if $\trconf{}$ is to be a strong bisimulation.
These axioms and induced constraints are summarized in \cref{tbl:trconf-constraints}.%
%
\begin{table}[tb]
  \renewcommand{\arraystretch}{1.2}
  \begin{tabular}{@{}c@{\quad\ }c@{}}
    \toprule
    \emph{Process reduction} & \emph{Formula-as-process rewriting constraint}
    \\ \midrule
    $\spawn{P}{Q} \reduces P \cc Q$ & $\trproc{\spawn{P}{Q}} \dreduces \trproc{P} \oc \trproc{Q}$\hphantom{\quad($\kay \in L$)}
    \\
    $\fwd \reduces (\cnfe)$ & $\trproc{\fwd} \dreduces (\octxe)$\hphantom{\quad($\kay \in L$)}
    \\
    $\selectR{\kay} \cc \caseL[\ell \in L]{\ell => Q_{\ell}} \reduces Q_{\kay}$ & $\trproc{\selectR{\kay}} \oc \trproc{\caseL[\ell \in L]{\ell => Q_{\ell}}} \dreduces \trproc{Q_{\kay}}$\quad($\kay \in L$)
    \\
    $\caseR[\ell \in L]{\ell => P_{\ell}} \cc \selectL{\kay} \reduces P_{\kay}$ & $\trproc{\caseR[\ell \in L]{\ell => P_{\ell}}} \oc \trproc{\selectL{\kay}} \dreduces \trproc{P_{\kay}}$\quad($\kay \in L$)
    \\ \addlinespace \bottomrule
  \end{tabular}
  \caption{Constraints on $\trproc{}$ that must be satisfied if $\trconf{}$ is to be a strong bisimulation}\label{tbl:trconf-constraints}
\end{table}

For example, the process reduction axiom $\spawn{P}{Q} \reduces P \cc Q$ induces the constraint $\trproc{\spawn{P}{Q}} \dreduces \trproc{P} \oc \trproc{Q}$.
In other words, $\trproc{\spawn{P}{Q}}$ ought to decompose compositionally in a single step.


% \subsection{A weakly focused, strongly bisimilar embedding}

We usually presume that the formula-as-process ordered rewriting framework is strongly focused, as described in \cref{ch:formula-as-process}.
But suppose that we instead move to a \emph{weakly focused} framework, as sketched in \cref{??}.
Then we can define 
\begin{equation*}
  \trproc{\spawn{P}{Q}} = \trproc{P} \fuse \trproc{Q}
  \quad\text{which satisfies}\quad
  \trproc{\spawn{P}{Q}} \dreduces \trproc{P} \oc \trproc{Q}
\end{equation*}
because the weak focusing discipline does not invert positive propositions eagerly.
\begin{marginfigure}
  \begin{equation*}
    \begin{tikzcd}
      \spawn{P}{Q} \arrow[reduces]{rrr} \dar[relation][swap]{\trconf{}}
        &[-1.85em] &&[-1.85em] P \cc Q \dar[relation]{\trconf{}}
      \\
      \trconf{\spawn{P}{Q}} \rar[phantom]{=}
        & \trproc{P} \fuse \trproc{Q} \rar[reduces]
        & \trproc{P} \oc \trproc{Q} \rar[phantom]{=}
        & \trconf{P \cc Q}
    \end{tikzcd}
  \end{equation*}
\end{marginfigure}%
% By similar reasoning, we might define
We can also define 
$  \trproc{\fwd} = \one$
%  \quad\text{
which similarly satisfies %}\quad
$  \trproc{\fwd} \dreduces (\octxe)$.
%  \,,
%\end{equation*}
% again because of weak focusing's treatment of positive propositions.
% \begin{marginfigure}
%   \begin{equation*}
%     \begin{tikzcd}
%       \fwd \arrow[reduces]{rrr} \dar[relation][swap]{\trconf{}}
%         &&& (\cnfe) \dar[relation]{\trconf{}}
%       \\
%       \trconf{\fwd} \rar[phantom]{=}
%         & \one \rar[reduces]
%         & (\octxe) \rar[phantom]{=}
%         & \trconf{\cnfe}
%     \end{tikzcd}
%   \end{equation*}
% \end{marginfigure}
% 
Notice that there is a clean and direct correspondence between the process constructor $\spawn{}{}$ and the logical connective $\fuse$ here; moreover, just as $\fwd$ is the unit of $\spawn{}{}$, so is its embedding the unit of the embedding of $\spawn{}{}$.

The constraints on $\trproc{\caseL[\ell \in L]{\ell => Q_{\ell}}}$ can be satisfied by defining
\begin{gather*}
  \trproc{\caseL[\ell \in L]{\ell => Q_{\ell}}}
    = \bigwith_{\ell \in L}\bigl(\trproc{\selectR{\ell}} \limp \up \trproc{Q_{\ell}}\bigr)
  % \intertext{which satisfies}
  % \trproc{\selectR{\kay}} \oc \trproc{\caseL[\ell \in L]{\ell => Q_{\ell}}}
  %   \reduces \trproc{Q_{\kay}}
  \,.
\end{gather*}
% for all $\kay \in L$.
Of course, because the formula-as-process framework restricts left- and right-handed implications to have atomic premises of complementary direction\footnote{As described in \cref{??}.}, this suggests that we also define
\begin{equation*}
  \trproc{\selectR{\kay}} = \atmR{\kay}
  \,,
\end{equation*}
for otherwise the above left-handed implication will not be a well-formed proposition.
% Indeed, 
% \begin{equation*}
%   \trproc{\selectR{\kay}} \oc \trproc{\caseL[\ell \in L]{\ell => Q_{\ell}}} = \atmR{\kay} \oc {\textstyle \bigl(\bigwith_{\ell \in L}(\atmR{\ell} \limp \up \trproc{Q_{\ell}})\bigr)}
%     \reduces \trproc{Q_{\kay}}
% \end{equation*}
% for all $\kay \in L$.
Symmetrically, we also define 
\begin{align*}
  \trproc{\caseR[\ell \in L]{\ell => P_{\ell}}} &= \bigwith_{\ell \in L}\bigl(\up \trproc{P_{\ell}} \pmir \atmL{\ell}\bigr) \\
  \trproc{\selectL{\kay}} &= \atmL{\kay}
  \,.
\end{align*}

\newthought{Coinductively defined processes} require a bit of extra care.
As described in \cref{??}, the weakly focused ordered rewriting framework did not include coinductively defined propositions.
Here we assume that coinductively defined negative propositions, like those of the strongly focused formula-as-process ordered rewriting framework~\parencref{??}, also exist for the weakly focused framework.%
\footnote{An argument can be made that the weakly focused framework should include coinductively defined propositions that are only \emph{positive}, not negative, in keeping with its partiality toward positive propositions in stable contexts.
  That is, however, at odds with the formula-as-process interpretation of coinductively defined propositions as coinductively defined processes, so we choose to have only negative defined propositions.
  Perhaps this small wrinkle could be ironed out if definitions were treated iso\-re\-cur\-sive\-ly, rather than equi\-re\-cur\-sive\-ly, but we choose not to pursue that here.}

To embed a process call, $\defp{p}$, in the weakly focused formula-as-process ordered rewriting framework, we will translate that process's definition to the definition of a coinductively defined negative proposition, $\n{\defp{p}}$.
Then the process call is simply embedded as 
\begin{equation*}
  \trproc{\defp{p}} = \dn \n{\defp{p}}
  \,.
\end{equation*}
But how should process \emph{definitions} be translated?

A first, natural attempt would be to map each process definition $\defp{p} \defd P$ to a negative proposition definition $\n{\defp{p}} \defd \up \trproc{P}$.
Unfortunately, this doesn't quite work.
It introduces a stutter in the unfolding and rewriting of defined processes like $\defp{p} \defd \caseL[\ell \in L]{\ell => P_{\ell}}$.
On the process side, we have the single-step
\begin{equation*}
  \selectR{\kay} \cc \defp{p} = \selectR{\kay} \cc \caseL[\ell \in L]{\ell => P_{\ell}} \reduces P_{\kay}
  \,.
\end{equation*}
But on the ordered rewriting side, it would take two steps to reach the same point:
\begin{equation*}
  \trconf{\selectR{\kay} \cc \defp{p}}
    = \atmR{\kay} \oc \dn \n{\defp{p}}
    = \atmR{\kay} \oc \dn \up \dn {\textstyle \bigwith_{\ell \in L}(\atmR{\ell} \limp \up \trproc{P_{\kay}})}
    \reduces \atmR{\kay} \oc \dn {\textstyle \bigwith_{\ell \in L}(\atmR{\ell} \limp \up \trproc{P_{\kay}})}
    \reduces \trproc{P_{\kay}}
  \,,
\end{equation*}
for all $\kay \in L$.
The extra step is caused by the $\up \dn$ double shift, with the $\up$ shift arising from the way we are trying to translate definitions, and with the $\dn$ shift arising from $\trproc{\caseL[\ell \in L]{\ell => P_{\ell}}}$.

Fortunately, this stuttering can be eliminated with a more careful translation of process definitions.
Instead of blindly inserting an $\up$ shift in front of $\trproc{P}$, let's construct a negative proposition by stripping off the outermost $\dn$ shift if $\trproc{P}$ has the form $\dn \n{A}$, and otherwise inserting an $\up$ shift:
\begin{align*}
  \trsig{\sig, (\defp{p} \defd P)}
    &= \trsig{\sig} , \left(\n{\defp{p}} \defd
         \begin{cases*}
           \n{A} & if $\trproc{P} = \dn \n{A}$ \\
           \up \trproc{P} & otherwise
         \end{cases*}
                            \right)
  \\
  \trconf{\sige} &= (\orsige)
\end{align*}
Now a defined process like $\defp{p} \defd \caseL[\ell \in L]{\ell => P_{\ell}}$ is translated as
\begin{equation*}
  \trsig{\defp{p} \defd \caseL[\ell \in L]{\ell => P_{\ell}}}
    = \bigl(\n{\defp{p}} \defd {\textstyle \bigwith_{\ell \in L}(\atmR{\ell} \limp \up \trproc{P_{\ell}})}\bigr)
  \,,
\end{equation*}
which does not induce the stutter:
\begin{equation*}
  \trconf{\selectR{\kay} \cc \defp{p}}
    = \atmR{\kay} \oc \dn \n{\defp{p}}
    = \atmR{\kay} \oc \dn {\textstyle \bigwith_{\ell \in L}(\atmR{\ell} \limp \up \trproc{P_{\kay}})}
    \reduces \trproc{P_{\kay}}
  \,,
\end{equation*}
for all $\kay \in L$.

\newthought{%
In total, the embedding} of process configurations and process expressions into a weakly focused formula-as-process ordered rewriting framework is summarized in \cref{fig:process-embedding:weak-focus-strong-bisim}.
\begin{marginfigure}
  \begin{align*}
    \trconf{\cnf_1 \cc \cnf_2} &= \trconf{\cnf_1} \oc \trconf{\cnf_2} \\
    \trconf{\cnfe} &= (\octxe) \\
    \trconf{P} &= \trproc{P}
  %
  \shortintertext{and}
  %
    \trsig{\sig, (\defp{p} \defd P)}
      &= \trsig{\sig} , \left(\n{\defp{p}} \defd
           \begin{cases*}
             \n{A} & if $\trproc{P} = \dn \n{A}$ \\
             \up \trproc{P} & otherwise
           \end{cases*}
         \right)
    \\
    \trconf{\sige} &= (\orsige)
  %
  \shortintertext{where}
  %
    \trproc{\spawn{P_1}{P_2}}
      &= \trproc{P_1} \fuse \trproc{P_2} \\
    \trproc{\fwd} &= \one
    \\
    \trproc{\selectR{\kay}} &= \atmR{\kay} \\
    \trproc{\caseL[\ell \in L]{\ell => P_{\ell}}}
      &= \dn {\textstyle \bigwith_{\ell \in L}(\atmR{\ell} \limp \up \trproc{P_{\ell}})}
    \\
    \trproc{\caseR[\ell \in L]{\ell => P_{\ell}}}
      &= \dn {\textstyle \bigwith_{\ell \in L}(\up \trproc{P_{\ell}} \pmir \atmL{\ell})} \\
    \trproc{\selectL{\kay}} &= \atmL{\kay}
    \\
    \trproc{\defp{p}} &= \dn \n{\defp{p}}
  \end{align*}
  \caption{A \emph{strongly} bisimilar embedding of process configurations within a \emph{weakly} focused formula-as-process ordered rewriting framework}\label{fig:process-embedding:weak-focus-strong-bisim}
\end{marginfigure}

Notice that, with the exception of a small wrinkle in the translation of process definitions, this embedding is quite clean and direct.
Especially appealing is the close correspondence between process constructors and logical connectives noted previously.
% The process constructor $\spawn{}{}$ for spawning a new process cleanly corresponds to the logical connective $\fuse$ for ordered conjunction.
% And, just as the forwarding process, $\fwd$, can be seen as the unit of $\spawn{}{}$, so is the embedding $\trproc{\fwd} = \one$ the unit of the $\fuse$ connective.
%
Moreover, the embedding is a \emph{strong} bisimulation:
%
\begin{theorem}[Adequacy of $\trconf{}$]
  Under the weakly focused formula-as-process ordered rewriting framework, $\trconf{}$ constitutes a strong bisimulation.
  That is:
  \begin{itemize}[nosep]
  \item If\/ $\cnf \reduces \cnf'$, then $\trconf{\cnf} \reduces \trconf{\cnf'}$.
  \item If\/ $\trconf{\cnf} \reduces \p{\octx'{}}$, then there exists a configuration $\cnf'$ such that $\cnf \reduces \cnf'$ and $\trconf{\cnf'} = \p{\octx'{}}$.
  \end{itemize}
\end{theorem}
\begin{proof}
  The first part is by induction on the process configuration reduction, $\cnf \reduces \cnf'$; the second part is by induction on the weakly focused formula-as-process ordered rewriting, $\trconf{\cnf} \reduces \p{\octx'{}}$.
\end{proof}

\subsection{A strongly focused, strongly bisimilar embedding}\label{sec:embed:strong-focused-strong-bisim}

One might object to the preceding embedding's reliance on weak focusing to achieve strong bisimilarity.
Needing to switch from strong focusing to the more exotic weak focusing is admittedly a blemish, but one that is hidden by the appealing correspondence of the process constructors with logical connectives. 
% $\spawn{}{}$ with $\fuse$ and $\fwd$ with $\one$.

If one insists upon strong focusing, there is nevertheless
% two avenues to a bisimilar embedding of process configurations.
a way forward.
% 
% First,
Recall from \cref{??} that weak focusing can be embedded within strong focusing by the careful addition of shifts.
The $\embed*{}$ embedding described there could be composed with the $\trproc{}$ embedding of processes described above.
For example, the composed embedding of $\spawn{P_1}{P_2}$ would be
\begin{equation*}
  \embed{\trproc{\spawn{P_1}{P_2}}} = \up \bigl(\bigfuse (\embed{\trproc{P_1}}) \fuse \bigfuse (\embed{\trproc{P_2}})\bigr)
  \,,
\end{equation*}
which satisfies $\embed{\trproc{\spawn{P_1}{P_2}}} \reduces \embed{\trproc{P_1}} \oc \embed{\trproc{P_2}}$ in the strong focusing framework.
% because $\rfocus{\embed{\trproc{P_1}} \oc \embed{\trproc{P_2}}}{\bigfuse (\embed{\trproc{P_1}}) \fuse \bigfuse (\embed{\trproc{P_2}})}$ holds.
Using such a composition via $\trconf{P} = \embed{\trproc{P}}$ would turn $\trconf{}$ into a strongly bisimilar embedding of process configurations within \emph{strongly} focused formula-as-process ordered rewriting framework.
(For a definition of the embedding for process definitions, see \cref{fig:process-embedding:full-focus-strong-bisim}.)
%
\begin{marginfigure}
  \begin{align*}
    \trconf{\cnfe} &= (\octxe) \\
    \trconf{\cnf_1 \cc \cnf_2} &= \trconf{\cnf_1} \oc \trconf{\cnf_2} \\
    \trconf{P} &= \embed{\trproc{P}}
  %
  \shortintertext{and}
  %
    \trsig{\sig, (\defp{p} \defd P)}
      &= \trsig{\sig} , \left(\n{\defp{p}} \defd
           \begin{cases*}
             \embed{\trproc{P}} & if $P \neq \selectL{\kay}$ and $P \neq \selectR{\kay}$ \\
             \up \embed{\trproc{P}} & otherwise
           \end{cases*}
         \right)
    \\
    \trconf{\sige} &= (\orsige)
  % %
  % \shortintertext{where}
  % %
  %   \embed{\trproc{\spawn{P_1}{P_2}}}
  %     &= \up (\bigfuse \embed{\trproc{P_1}} \fuse \bigfuse \embed{\trproc{P_2}}) \\
  %   \embed{\trproc{\fwd}} &= \up \one
  %   \\
  %   \embed{\trproc{\selectR{\kay}}} &= \atmR{\kay} \\
  %   \embed{\trproc{\caseL[\ell \in L]{\ell => P_{\ell}}}}
  %     &= {\textstyle \bigwith_{\ell \in L}(\atmR{\ell} \limp \up \bigfuse \embed{\trproc{P_{\ell}}})}
  %   \\
  %   \embed{\trproc{\caseR[\ell \in L]{\ell => P_{\ell}}}}
  %     &= {\textstyle \bigwith_{\ell \in L}(\up \bigfuse \embed{\trproc{P_{\ell}}} \pmir \atmL{\ell})} \\
  %   \embed{\trproc{\selectL{\kay}}} &= \atmL{\kay}
  %   \\
  %   \embed{\trproc{\defp{p}}} &= \n{\defp{p}}
  \end{align*}
  where $\trproc{P}$ is defined exactly as in \cref{fig:process-embedding:weak-focus-strong-bisim}.
  %
  \vspace{.75\baselineskip}
  %
  \caption{A \emph{strongly} bisimilar embedding of process configurations within the \emph{strongly} focused formula-as-process ordered rewriting framework}\label{fig:process-embedding:full-focus-strong-bisim}
\end{marginfigure}%
%

Explicit weak focusing is not needed, as long as we are satisfied with a somewhat more complex embedding that obscures the correspondence between process constructors and logical connectives.
However, we do not dwell further on this embedding, instead viewing it as of secondary value because we do prefer the more direct correspondence.
Moreover, as we will now show, strong focusing does not exclude a more direct correspondence.

\subsection{A strongly focused, weakly bisimilar embedding}\label{sec:embed:strong-focused-weak-bisim}

If we abandon our desire for a strong bisimulation and content ourselves with a weak bisimulation that operates on the strongly focused rewriting framework, then we can, in fact, retain the direct and appealing correspondence with logical connectives.
Without changing the definition of $\trproc{}$ at all, let us define $\trconf{}$ by 
\begin{align*}
  \trconf{\cnf_1 \cc \cnf_2} &= \trconf{\cnf_1} \oc \trconf{\cnf_2} \\
  \trconf{\cnfe} &= (\octxe) \\
  \trconf{P} &= \octx \text{ where $\rfocus{\octx}{\trproc{P}}$}
  \,.
\end{align*}
%
\begin{marginfigure}
  \begin{align*}
    \trconf{\cnf_1 \cc \cnf_2} &= \trconf{\cnf_1} \oc \trconf{\cnf_2} \\
    \trconf{\cnfe} &= (\octxe) \\
    \trconf{P} &= \octx \text{ where $\rfocus{\octx}{\trproc{P}}$}
  \end{align*}
  and where $\trsig{\sig}$ and $\trproc{P}$ are defined exactly as in \cref{fig:process-embedding:weak-focus-strong-bisim}.
  %
  \vspace{.75\baselineskip}
  %
  \caption{A \emph{weakly} bisimilar embedding of process configurations within the \emph{strongly} focused formula-as-process ordered rewriting framework}\label{fig:process-embedding:strong-focus-weak-bisim}
\end{marginfigure}
%
Decomposing the process expression $\spawn{P}{Q}$ now takes no time at all in its embedded form:
\begin{equation*}
  \begin{tikzcd}[row sep=tiny]
    \spawn{P}{Q} \arrow[reduces]{rr}
      &[-2em] &[-2em] P \cc Q
    \\ & \text{and yet} & \\
    \trconf{\spawn{P}{Q}} \rar[phantom]{=}
      & \trconf{P} \oc \trconf{Q} \rar[phantom]{=}
      & \trconf{P \cc Q}
  \mathrlap{\,.}
  \end{tikzcd}
\end{equation*}
It is this mismatch that makes this version of $\trconf{}$ a weak, not strong, bisimulation.
%
\begin{theorem}
  Under the strongly focused formula-as-process ordered rewriting framework, $\trconf{}$ constitutes only a \emph{weak} bisimulation.
  That is:
  \begin{itemize}[nosep]
  \item
    If\/ $\cnf \reduces \cnf'$, then either $\trconf{\cnf} \reduces \trconf{\cnf'}$ or $\trconf{\cnf} = \trconf{\cnf'}$.
    More specifically, if $\cnf \reduces \cnf'$ arises from the receipt of a message, then $\trconf{\cnf} \reduces \trconf{\cnf'}$; otherwise, $\trconf{\cnf} = \trconf{\cnf'}$.
  \item
    If\/ $\trconf{\cnf} \reduces \octx'$, then there exists a configuration $\cnf'$ such that $\cnf \reduces^{+} \cnf'$ and $\trconf{\cnf'} = \octx'$.
  \end{itemize}
\end{theorem}
\begin{proof}
  The first part is by induction on the process configuration reduction, $\cnf \reduces \cnf'$; the second part is by induction on the strongly focused formula-as-process ordered rewriting, $\trconf{\cnf} \reduces \octx'$.
\end{proof}

By using the $\trproc{}$ function unchanged and only substituting the clause $\trconf{P} = \trproc{P}$ with $\trconf{P} = \octx$ where $\rfocus{\octx}{\trproc{P}}$, it is clear that this new version of the $\trconf{}$ bisimulation is weak only because positive propositions are now eagerly inverted.
That is to say, the weakness of the bisimulation is a purely administrative artifact that is unrelated to the main computational aspects.

% Also, by using the same $\trproc{}$ function here, we can preserve the appealing close correspondence of process constructors with logical connectives, in spite of the move to strong focusing.



% In the formula-as-process ordered rewriting framework with the eager inversion of positive propositions that its fully focused nature implies, we can define
% \begin{equation*}
%   \trproc{\spawn{P}{Q}} = \up (\bigfuse \trproc{P} \fuse \bigfuse \trproc{Q})
%     \reduces \trproc{P} \oc \trproc{Q}
% \end{equation*}
% because $\rfocus{\trproc{P}}{\bigfuse \trproc{P}}$ and, similarly, $\rfocus{\trproc{Q}}{\bigfuse \trproc{Q}}$.

% But this $\up (\bigfuse \octx_1 \fuse \bigfuse \octx_2)$ pattern looks quite familiar.
% In \cref{??}, it was used in the embedding of weakly focused ordered rewriting into fully focused ordered rewriting.
% What if, instead of using the fully focused formula-as-process rewriting framework, we were to move to a \emph{weakly} focused framework?
% Then we could simplify the definition of $\trproc{\spawn{P}{Q}}$ to merely
% \begin{equation*}
%   \trproc{\spawn{P}{Q}} = \trproc{P} \fuse \trproc{Q}
%   \,,
% \end{equation*}
% which indeed takes a complete step to decompose the conjunction: $\trproc{P} \fuse \trproc{Q} \reduces \trproc{P} \oc \trproc{Q}$.
% This As an embedding, weakly focused rewriting 

% \begin{equation*}
%   \trproc{\fwd} = \one
% \end{equation*}

% Observe that $\trproc{\selectR{\kay}} \oc \dn {\textstyle \bigwith_{\ell \in L}(\trproc{\selectR{\ell}} \limp \up \trproc{Q_{\ell}})} \reduces \trproc{Q_{\kay}}$ is a valid weakly focused formula-as-process rewriting, for all $\kay \in L$, if $\trproc{\selectR{\kay}}$ is a right-directed atom.
% \begin{align*}
%   \trproc{\selectR{\kay}} &= \atmR{\kay} \\
%   \trproc{\caseL[\ell \in L]{\ell => Q_{\ell}}} &= \dn {\textstyle \bigwith_{\ell \in L}(\atmR{\ell} \limp \up \trproc{Q_{\ell}})}
% %
% \intertext{and, symmetrically,}
% %
%   \trproc{\caseR[\ell \in L]{\ell => P_{\ell}}} &= \dn {\textstyle \bigwith_{\ell \in L}(\up \trproc{P_{\ell}} \pmir \atmL{\ell})} \\
%   \trproc{\selectL{\kay}} &= \atmL{\kay}
% \end{align*}

% The obvious candidate is to define
% \begin{equation*}
%   \trproc{\spawn{P}{Q}} = \trproc{P} \fuse \trproc{Q}
%   \,,
% \end{equation*}
% but there are two obstacles to such a definition.
% First, $\trproc{P} \fuse \trproc{Q}$ is not a non-atomic positive proposition.
% But, second and more troublingly, positive propositions are decomposed all at once during the focusing bipole that constitutes a rewriting step in the formula-as-process \emph{focused} ordered rewriting framework.
% This leads to a mismatch between the small-step process reductions and the big-step formula-as-process focused ordered rewriting, as shown in the adjacent \lcnamecref{fig:translation:focused-mismatch}.%
% %
% \begin{marginfigure}
%   \begin{equation*}
%     \begin{tikzcd}
%       \spawn{(\spawn{P}{Q})}{R} \rar[reduces] \arrow[relation]{dd}[swap]{\trconf{}}
%         & (\spawn{P}{Q}) \cc R \dar[relation]{\trconf{}}
%       \\
%         & (\trproc{P} \fuse \trproc{Q}) \oc \trproc{R} \dar[phantom]{\neq}
%       \\[-2ex]
%       (\trproc{P} \fuse \trproc{Q}) \fuse \trproc{R} \rar[reduces]
%         & (\trproc{P} \oc \trproc{Q}) \oc \trproc{R}
%     \end{tikzcd}
%   \end{equation*}
%   \caption{Mismatch between process reduction and big-step decomposition of positive propositions}\label{fig:translation:focused-mismatch}
% \end{marginfigure}%

% There are (at least) two solutions.
% We could abandon our ideal strong bisimulation and instead settle for a weak bisimulation.

% Another possible solution, and the one we pursue here, is to shift the rewriting framework from a fully focused framework to the \emph{weakly focused} formula-as-process ordered rewriting framework described in \cref{??}.
% With weak focusing, positive propositions are not decomposed eagerly in a big-step manner -- 
% \begin{gather*}
%   (\trproc{P} \fuse \trproc{Q}) \fuse \trproc{R} \nreduces (\trproc{P} \oc \trproc{Q}) \oc \trproc{R}
% \intertext{but rather}
%   (\trproc{P} \fuse \trproc{Q}) \fuse \trproc{R} \reduces (\trproc{P} \fuse \trproc{Q}) \oc \trproc{R}
% \end{gather*}
% %
% \begin{marginfigure}
%   \begin{equation*}
%     \begin{tikzcd}
%       \spawn{(\spawn{P}{Q})}{R} \rar[reduces] \dar[relation][swap]{\trconf{}}
%         & (\spawn{P}{Q}) \cc R \dar[relation]{\trconf{}}
%       \\
%       (\trproc{P} \fuse \trproc{Q}) \fuse \trproc{R} \rar[reduces]
%         & (\trproc{P} \fuse \trproc{Q}) \oc \trproc{R}
%     \end{tikzcd}
%   \end{equation*}
%   \caption{Mismatch between process reduction and big-step decomposition of positive propositions}\label{fig:translation:focused-mismatch}
% \end{marginfigure}%




% \begin{marginfigure}
%   \begin{align*}
%     \trconf{\cnfe} &= (\octxe) \\
%     \trconf{\cnf_1 \cc \cnf_2} &= \trconf{\cnf_1} \oc \trconf{\cnf_2} \\
%     \trconf{P} &= \trproc{P}
%   %
%   \shortintertext{where}
%   %
%     \trproc{\spawn{P_1}{P_2}}
%       &= \trproc{P_1} \oc \trproc{P_2} \\
%     \trproc{\fwd} &= \octxe
%     \\
%     \trproc{\selectR{\kay}} &= \atmR{\kay} \\
%     \trproc{\caseL[\ell \in L]{\ell => P_{\ell}}}
%       &= {\textstyle \bigwith_{\ell \in L}(\atmR{\ell} \limp \up \bigfuse \trproc{P_{\ell}})}
%     \\
%     \trproc{\caseR[\ell \in L]{\ell => P_{\ell}}}
%       &= {\textstyle \bigwith_{\ell \in L}(\up \bigfuse \trproc{P_{\ell}} \pmir \atmL{\ell})} \\
%     \trproc{\selectL{\kay}} &= \atmL{\kay}
%     \\
%     \trproc{\defp{p}} &= \n{\defp{p}}
%   \end{align*}
%   \caption{A \emph{weakly} bisimilar embedding of process configurations within \emph{fully} focused formula-as-process ordered rewriting}\label{fig:process-embedding:full-focus-weak-bisim}
% \end{marginfigure}



% \begin{equation*}
%   \begin{aligned}
%     \trconf{\cnf_1 \cc \cnf_2} &= \trconf{\cnf_1} \oc \trconf{\cnf_2} \\
%     \trconf{\cnfe} &= (\octxe) \\
%     \trconf{P} &= \octx \text{, where $\rfocus{\octx}{\trproc{P}}$}
%   \end{aligned}
% \end{equation*}





% % \begin{align*}
% %   \embed*{\bigfuse \trproc{\spawn{P_1}{P_2}}}
% %     &= \up (\bigfuse \embed*{\bigfuse \trproc{P_1}} \fuse \bigfuse \embed*{\bigfuse \trproc{P_2}}) \\
% %   \embed*{\bigfuse \trproc{\fwd}} &= \up \one
% %   \\
% %   \embed*{\bigfuse \trproc{\selectR{\kay}}} &= \atmR{\kay} \\
% %   \embed*{\bigfuse \trproc{\caseL[\ell \in L]{\ell => P_{\ell}}}} &= {\textstyle \bigwith_{\ell \in L}(\atmR{\ell} \limp \up \bigfuse \embed*{\bigfuse \trproc{P_{\ell}}})}
% %   \\
% %   \embed*{\bigfuse \trproc{\caseR[\ell \in L]{\ell => P_{\ell}}}} &= {\textstyle \bigwith_{\ell \in L}(\up \bigfuse \embed*{\bigfuse \trproc{P_{\ell}}} \pmir \atmL{\ell})} \\
% %   \embed*{\bigfuse \trproc{\selectL{\kay}}} &= \atmL{\kay}
% % \end{align*}

% \begin{equation*}
%   \begin{aligned}
%     \trconf{\octxe} &= (\octxe) \\
%     \trconf{\cnf_1 \cc \cnf_2} &= \trconf{\cnf_1} \oc \trconf{\cnf_2} \\
%     \trconf{P} &= \trproc[\fuse]{P}
%   \end{aligned}
%   \quad\text{and}\quad
%   \begin{aligned}
%     \trproc[\fuse]{\spawn{P_1}{P_2}}
%       &= \up (\bigfuse \trproc[\fuse]{P_1} \fuse \bigfuse \trproc[\fuse]{P_2}) \\
%     \trproc[\fuse]{\fwd} &= \up \one
%     \\
%     \trproc[\fuse]{\selectR{\kay}} &= \atmR{\kay} \\
%     \trproc[\fuse]{\caseL[\ell \in L]{\ell => P_{\ell}}} &= {\textstyle \bigwith_{\ell \in L}(\atmR{\ell} \limp \up \bigfuse \trproc[\fuse]{P_{\ell}})}
%     \\
%     \trproc[\fuse]{\caseR[\ell \in L]{\ell => P_{\ell}}} &= {\textstyle \bigwith_{\ell \in L}(\up \bigfuse \trproc[\fuse]{P_{\ell}} \pmir \atmL{\ell})} \\
%     \trproc[\fuse]{\selectL{\kay}} &= \atmL{\kay} \\
%     \trproc[\fuse]{\defp{p}} &= \n{\defp{p}}
%   \end{aligned}
% \end{equation*}

% \begin{theorem}
%   $\trproc[\fuse]{P} = \embed*{\trproc{P}}$
% \end{theorem}



\clearpage
\subsection{Examples and other comments}


In summary, we have three distinct embeddings of process expressions and configurations within focused formula-as-process ordered rewriting.
\begin{table}[tbp]
  \begin{tabular}{@{}lll@{}}
    \toprule
    \emph{Focusing} & \emph{Bisimilarity} & \emph{Key clause}
    \\ \midrule
    weakly focused & strongly bisimilar & $\trconf{P} = \trproc{P}$ \\
    strongly focused & strongly bisimilar & $\trconf{P} = \embed{\trproc{P}}$ \\
    strongly focused & weakly bisimilar & $\trconf{P} = \octx$ where $\rfocus{\octx}{\trproc{P}}$
    \\ \bottomrule
  \end{tabular}
  \caption{A summary of the process embeddings}\label{tab:embedding-summary}
\end{table}
Focusing, be it weak or strong, is essential to these embeddings.
If the unfocused form of ordered rewriting were used, no bisimulation along these lines appears to be possible, not even a weak one.
For instance, in a hypothetical unfocused embedding, we would likely have 
\begin{equation*}
  \trconf{\caseL[\ell \in L]{\ell => P_{\ell}}}
    = {\textstyle \bigwith_{\ell \in L}(\atmR{\ell} \limp \trproc{P_{\ell}})}
    \reduces \atmR{\kay} \limp \trproc{P_{\kay}}
\end{equation*}
if $\kay \in L$, but there is no configuration $\cnf'$ such that $\caseL[\ell \in L]{\ell => P_{\ell}} \Reduces \cnf'$ and $\trconf{\cnf'} = \atmR{\kay} \limp \trproc{P_{\kay}}$.

\newthought{As examples} of the preceding embeddings, let us revisit our two running examples: binary counters and \acp{DFA}.

% \paragraph*{Binary counter}
Recall from \cref{??} that a binary counter can be implemented by the coinductively defined processes
\begin{equation*}
  \sig = \begin{lgathered}[t]
           \bigl(e \defd \caseR{i => \spawn{e}{b_1} | d => \selectR{z}}\bigr) \,, \\
           \bigl(b_0 \defd \caseR{i => b_1 | d => \spawn{\selectL{d}}{b'_0}}\bigr) \,, \\
           \bigl(b_1 \defd \caseR{i => \spawn{\selectL{i}}{b_0} | d => \spawn{b_0}{\selectR{s}}}\bigr) \,, \\
           \bigl(b'_0 \defd \caseL{z => \selectR{z} | s => \spawn{b_1}{\selectR{s}}}\bigr)
         \end{lgathered}
\end{equation*}
What coinductively defined propositions arise from embedding these process definitions?

Under the strongly focused, weakly bisimilar embedding, the process definition for $b_0$ becomes the coinductively defined proposition given by
\begin{equation*}
  \trsig{b_0 \defd \caseR{i => b_1 | d => \spawn{\selectL{d}}{b'_0}}}
    = \bigl(\n{\defp{b}_0} \defd (\up \dn \n{\defp{b}_1} \pmir \atmL{i}) \with (\up (\atmL{d} \fuse \dn \n{\defp{b}'_0{}}) \pmir \atmL{d})\bigr)
  \,.
\end{equation*}
This defined proposition is \emph{exactly} the same as the one produced in \cref{??} as the object-oriented choreography of the binary counter's initial string rewriting specification.
(Here, for completeness, we have included even the minimally necessary shifts that we would usually elide.)
Even the $\up \dn$ double shift in front of $\n{\defp{b}_1}$ is correctly produced for free by the embedding.

The same is true of the other process definitions: embedding them yields exactly the same coinductively defined propositions as in the binary counter's object-oriented choreography shown in \cref{??}.
\begin{equation*}
  \trsig{\sig} =
  \orsig = \begin{lgathered}[t]
             \bigl(\n{\defp{e}} \defd (\up (\dn \n{\defp{e}} \fuse \dn \n{\defp{b}_1}) \pmir \atmL{i}) \with (\up \atmR{z} \pmir \atmL{d})\bigr) \,, \\
             \bigl(\n{\defp{b}_0} \defd (\up \dn \n{\defp{b}_1} \pmir \atmL{i}) \with (\up (\atmL{d} \fuse \dn \n{\defp{b}'_0{}}) \pmir \atmL{d})\bigr) \,, \\
             \bigl(\n{\defp{b}_1} \defd (\up (\atmL{i} \fuse \dn \n{\defp{b}_0}) \pmir \atmL{i}) \with (\up (\dn \n{\defp{b}_0} \fuse \atmR{s}) \pmir \atmL{d})\bigr) \,, \\
             \bigl(\n{\defp{b}'_0{}} \defd (\atmR{z} \limp \up \atmR{z}) \with (\atmR{s} \limp \up (\dn \n{\defp{b}_1} \fuse \atmR{s}))\bigr)
           \end{lgathered}
\end{equation*}
(Actually, the same definitions arise when using the weakly focused, strongly bisimilar embedding because it treats process definitions the same as the strongly focused, weakly bisimilar embedding does.)

Similarly, recall the functional-style process definitions for the binary counter from \cref{??}:
\begin{equation*}
  \sig' = \begin{lgathered}[t]
            \bigl(i \defd \caseL{e => \spawn{\selectR{e}}{\selectR{b}_1}
                               | b_0 => \selectR{b}_1
                               | b_1 => \spawn{i}{\selectR{b}_0}}\bigr) \,,
            \\
            \bigl(d \defd \caseL{e => \selectR{z}
                               | b_0 => \spawn{d}{b'_0}
                               | b_1 => \spawn{\selectR{b}_0}{\selectR{s}}}\bigr) \,,
            \\
            \bigl(b'_0 \defd \caseL{z => \selectR{z}
                                  | s => \spawn{\selectR{b}_1}{\selectR{s}}}\bigr)
\,.
          \end{lgathered}
\end{equation*}
By embedding definitions, we arrive at exactly the same coinductively defined propositions as in the functional choreography of the binary counter shown in \cref{??}:
\begin{equation*}
  \trsig{\sig'} =
  \orsig' = \begin{lgathered}[t]
              \bigl(\n{\defp{\imath}} \defd (\atmR{e} \limp \up (\atmR{e} \fuse \atmR{b}_1)) \with (\atmR{b}_0 \limp \up \atmR{b}_1) \with (\atmR{b}_1 \limp \up (\dn \n{\defp{\imath}} \fuse \atmR{b}_0)) \,, \\
              \bigl(\n{\defp{d}} \defd (\atmR{e} \limp \up \atmR{z}) \with (\atmR{b}_0 \limp \up (\dn \n{\defp{d}} \fuse \dn \n{\defp{b}_0})) \with (\atmR{b}_1 \limp \up (\atmR{b}_0 \fuse \atmR{s})) \,, \\
              \bigl(\n{\defp{b}'_0{}} \defd (\atmR{z} \limp \up \atmR{z}) \with (\atmR{s} \limp \up (\atmR{b}_1 \fuse \atmR{s}))\bigr)
\,.
            \end{lgathered}
\end{equation*}
(Once again, the same definitions arise from the weakly focused, strongly bisimilar embedding.)

% \paragraph*{\aclp*{DFA}}
The same holds true for \acp{DFA}.
Embedding the \ac{DFA} process definitions from \cref{??} yields the same coinductively defined propositions as in the object-oriented and functional choreographies from \cref{??}:
\begin{align*}
  \trsig{q \defd \caseL[a \in \ialph]{a => q'_a | \eow => \selectR{F}(q)}}
    &= \bigl(
         \n{\defp{q}} \defd
           (\atmR{\eow} \limp \up \atmR{F}(q)) \with
           {\textstyle \bigwith_{a \in \ialph}(\atmR{a} \limp \up \dn \n{\defp{q}'_a{}})}
       \bigr)
 \shortintertext{and} 
  \trsig{a \defd \caseR[q \in Q]{q => \selectL{q}'_a}}
    &= \bigl(
         \n{\defp{a}} \defd {\textstyle \bigwith_{q \in Q}(\up \atmL{q}'_a \pmir \atmL{q})}
       \bigr)
  \\
  \trsig{\eow \defd \caseR[q \in Q]{q => \selectR{F}(q)}}
    &= \bigl(
         \n{\defp{\eow}} \defd {\textstyle \bigwith_{q \in Q}(\up \atmR{F}(q) \pmir \atmL{q})}
       \bigr)
  \,,
\end{align*}
respectively.




\clearpage
\section{A session type system for ordered rewriting}

The preceding embeddings describe bisimulations between process expressions and certain ordered propositions; process configurations and certain ordered contexts.
Because the formula-as-process ordered rewriting frameworks are untyped, these embeddings discard type information when translating process expressions and configurations.
But is that really necessary?
Can the bisimilarity witnessed by these embeddings be leveraged to engineer a session type system for formula-as-process ordered rewriting from the session type system for processes?

Consider, for example, the process expression $\spawn{\selectR{\kay}_1}{\selectL{\kay}_2}$.
Although syntactically well-formed, this process expression is not typable: to be typable, the type at the interface between $\selectR{\kay}_1$ and $\selectL{\kay}_2$ would be faced with the impossible task of simultaneously being both an internal and external choice.
Yet even this untypable process expression can be embedded:
\begin{equation*}
  \trproc{\spawn{\selectR{\kay}_1}{\selectL{\kay}_2}}
    = \atmR{\kay}_1 \fuse \atmL{\kay}_2
  \,.
\end{equation*}
But the question is, can the image of \emph{well-typed} process expressions under this embedding be characterized?

\newthought{The idea} is to engineer a session type system for ordered propositions, based on a judgment $\slof{A |- \p{A} : B}$, such that the following \lcnamecref{thm:embed:type-props} will hold.
%
\begin{restatable}[
  label=thm:embed:type-props
]{theorem}{thmembedtypeprops}
  $\slof{A |-_{\orsig} \p{A}_1 : B}$ if, and only if, there exist process definitions $\sig$ and a process expression $P$ such that $\slof{A |-_{\sig} P : B}$ and $\trsig{\sig} = \orsig$ and $\trproc{P} = \p{A}_1$.
\end{restatable}
%
\noindent
In the judgment $\slof{A |- \p{A}_1 : B}$ and this \lcnamecref{thm:embed:type-props}, $A$ and $B$ are (Curry--Howard interpretations of) singleton propositions that functions as session types, whereas $\p{A}_1$ is a formula-as-process positive ordered proposition that functions as an embedded process expression.

To construct such a type system, we can simply apply the embedding to session typing rules for process expressions.
For instance, the typing rule for a process expression $\spawn{P}{Q}$ suggests that ordered conjunctions $\p{A}_1 \fuse \p{A}_2$ be typed with a cut-like rule, as shown in the adjacent \lcnamecref{fig:embed-spawn-typing}.
\begin{marginfigure}
\begin{gather*}
  \left.\trproc{%
    \infer{\slof{A |- \spawn{P}{Q} : C}}{
      \slof{A |- P : B} & \slof{B |- Q : C}}%
  \right}
  \\\rightsquigarrow\\
  \infer{\slof{A |- \trproc{\spawn{P}{Q}} = \trproc{P} \fuse \trproc{Q} : C}}{
    \slof{A |- \trproc{P} : B} & \slof{B |- \trproc{Q} : C}}
\shortintertext{suggests}
  \infer[\jrule{CUT}\smash{^B}]{\slof{A |- \p{A}_1 \fuse \p{A}_2 : C}}{
    \slof{A |- \p{A}_1 : B} & \slof{B |- \p{A}_2 : C}}
\end{gather*}
\caption{The embedding suggests a session-typing rule for ordered conjunction.}\label{fig:embed-spawn-typing}
\end{marginfigure}%
%
By this procedure, we can also reverse-engineer session-typing rules for the other polarized positive propositions from the session-typing rules for process expressions.
We arrive at:
\begin{inferences}
  \infer[\jrule{CUT}\smash{^B}]{\slof{A |- \p{A}_1 \fuse \p{A}_2 : C}}{
    \slof{A |- \p{A}_1 : B} & \slof{B |- \p{A}_2 : C}}
  \and
  \infer[\jrule{ID}\smash{^A}]{\slof{A |- \one : A}}{}
  \\
  \infer[\rrule{\plus}']{\slof{A_{\kay} |- \selectR{\kay} : \plus*[sub=_{\ell \in L}]{\ell:A_{\ell}}}}{
    \text{($\kay \in L$)}}
  \and
  \infer[\lrule{\plus}]{\slof{\plus*[sub=_{\ell \in L}]{\ell:A_{\ell}} |- \dn {\textstyle \bigwith_{\ell \in L} (\selectR{\ell} \limp \up \p{A}_{\ell})} : C}}{
    \multipremise{\ell \in L}{\slof{A_{\ell} |- \p{A}_{\ell} : C}}}
  \\
  \infer[\rrule{\with}]{\slof{A |- \dn {\textstyle \bigwith_{\ell \in L} (\up \p{A}_{\ell} \pmir \selectL{\ell})} : \with*[sub=_{\ell \in L}]{\ell:B_{\ell}}}}{
    \multipremise{\ell \in L}{\slof{A |- \p{A}_{\ell} : B_{\ell}}}}
  \and
  \infer[\lrule{\with}']{\slof{\with*[sub=_{\ell \in L}]{\ell:B_{\ell}} |- \selectL{\kay} : B_{\kay}}}{
    \text{($\kay \in L$)}}
  \\
  \infer[]{\slof{A |-_{\orsig} \dn \n{\defp{p}} : B}}{
    \text{($\slof{A |- \n{\defp{p}} \defd \n{A}_0 : B}) \in \orsig$)}}
\end{inferences}

At first, these rules may be a bit startling.
It's especially surprising to see the proposition $\p{A}_1 \fuse \p{A}_2$ typed without splitting the context and the proposition $\one$ typed with a nonempty context to its left.
But once viewed from the formula-as-process perspective, these rules become quite natural: $\p{A}_1 \fuse \p{A}_2$ is a composition of process expressions and cut-as-composition is familiar; and $\one$ is a forwarding process, so using the identity rule is not that surprising after all.

Now we can prove the adequacy of these typing rules for positive propositions.
%
\thmembedtypeprops*
\begin{proof}
  From left to right, by structural induction on the typing derivation of $\slof{A |-_{\orsig} \p{A}_1 : B}$;
  from right to left, by structural induction on the typing derivation of $\slof{A |-_{\sig} P : B}$.
\end{proof}
%
This would be an extremely strong \lcnamecref{thm:embed:type-props} if not for the specificity of the $\lrule{\plus}$ and $\rrule{\with}$ rules above.
These rules do not assign a type to general negative propositions like $\n{A} \with \n{B}$.
Instead, only the restricted propositions $\dn {\textstyle \bigwith_{\ell \in L}(\up \p{A}_{\ell} \pmir \atmL{\ell})}$ and $\dn {\textstyle \bigwith_{\ell \in L}(\atmR{\ell} \limp \up \p{A}_{\ell})}$ are typable.
This is because we want these reverse-engineered typing rules to correspond to the image of the process expression typing rules.
In general, $\n{A} \with \n{B}$ describes a nondeterministic choice, something that is not present in the syntax of session-typed process expressions.%
\footnote{This suggests the possibility of introducing a well-typed form of nondeterministic choice.
  We discuss this idea further in \cref{??}.}
Even so, the \lcnamecref{thm:embed:type-props} is quite strong.

As a final remark, we would like to point out that the $0$-ary forms of the above $\lrule{\plus}$ and $\rrule{\with}$ rules need careful consideration.
Because we identify a $0$-ary alternative conjunction with $\top$, the $0$-ary forms of these rules can lead to a system in which two different typings of $\top$ are possible: $\slof{\zero |- \top : C}$ and $\slof{A |- \top : \top}$.
For this reason, we disallow $0$-ary alternative conjunctions and require that the label set $L$ is nonempty.

\newthought{Coinductively defined propositions} can also be given session-typing rules.
These take the form of rules verifying that a collection of mutually coinductive definitions is well-typed.
The judgment is $\vdash_{\orsig'} \orsig$, where the definitions $\orsig$ are judged according to the definitions $\orsig'$.
The standard trick of tying the mutually recursive knot by using $\vdash_{\orsig} \orsig$ at the top level is used: $\orsig$ is well-typed if $\vdash_{\orsig} \orsig$ holds.
\begin{inferences}
  \infer{\vdash_{\orsig'} \orsig , (\slof{A |- \n{\defp{p}} \defd \n{A}_0 : B})}{
    \text{($\n{A}_0 \neq \up \p{B}_0$)} &
    \vdash_{\orsig'} \orsig &
    \slof{A |-_{\orsig'} \dn \n{A}_0 : B}}
  \and
  \infer{\vdash_{\orsig'} \orsig , (\slof{A |- \n{\defp{p}} \defd \up \p{A}_0 : B})}{
    \vdash_{\orsig'} \orsig &
    \slof{A |-_{\orsig'} \p{A}_0 : B}}
\end{inferences}
These rules can be derived by the same kind of reasoning as used for the above proposition typing rules.
In particular, the peculiarities of the embedding $\trsig{\sig}$ for process definitions explains why the above two rules distinguish cases on whether or not the definition's body begins with an $\up$ shift.

We may prove the following \lcnamecref{thm:embed:type-sig}.
\begin{theorem}\label{thm:embed:type-sig}
  $\vdash_{\orsig'} \orsig$ if, and only if, there exist process definitions $\sig$ and $\sig'$ such that $\vdash_{\sig'} \sig$ and $\trsig{\sig} = \orsig$ and $\trsig{\sig'} = \orsig'$.
\end{theorem}
\begin{proof}
  From left to right, by structural induction on the typing derivation of $\vdash_{\orsig'} \orsig$;
  from right to left, by structural induction on  $\vdash_{\sig'} \sig$.
\end{proof}


\newthought{Ordered contexts} can be given session-typing rules, too.
Once again, we can derive these rules by applying the embeddings to the configuration typing rules of \cref{??}.
However, the particulars of ordered contexts differ between the weakly focused and strongly focused frameworks: weakly focused contexts are built from positive propositions, whereas strongly focused contexts are built from negative propositions and positive atoms.
So we will actually have two sets of session-typing rules for ordered contexts, using the one that matches the style of focusing in effect.

For the weakly focused framework, the judgment will be $\slcof{A |-_{\orsig} \p{\octx} : B}$, where the definitions $\orsig$ are typically elided because they are passed from conclusion to premises unchanged.
In the weakly focused framework, the strongly bisimilar embedding yields the following three session-typing rules for ordered contexts.
\begin{inferences}
  \infer[\jrule{C-CUT}\smash{^B}]{\slcof{A |- \p{\octx}_1 \oc \p{\octx}_2 : C}}{
    \slcof{A |- \p{\octx}_1 : B} & \slcof{B |- \p{\octx}_2 : C}}
  \and
  \infer[\jrule{C-ID}\smash{^A}]{\slcof{A |- \octxe : A}}{}
  \\
  \infer[\jrule{C-PROC}]{\slcof{A |- \p{A}_1 : B}}{
    \slof{A |- \p{A}_1 : B}}
\end{inferences}
In particular, the final rule can be understood as arising from the weakly focused, strongly bisimilar embedding's clause $\trconf{P} = \trproc{P}$.

\begin{theorem}
  $\slcof{A |-_{\orsig} \p{\octx} : B}$ if, and only if, there exists a configuration $\cnf$ such that $\slcof{A |-_{\sig} \cnf : B}$ and $\trsig{\sig} = \orsig$ and $\trconf{\cnf} = \p{\octx}$.
\end{theorem}
\begin{proof}
  From left to right, by structural induction on the typing derivation of $\slcof{A |-_{\orsig} \p{\octx} : B}$;
  from right to left, by structural induction on the typing derivation of $\slcof{A |-_{\sig} \cnf : B}$.
\end{proof}

If the strongly focused framework is instead being used, then we will use a different set of session-typing rules for ordered contexts, owing to the fact that ordered contexts are now based on negative propositions and positive atoms, not positive propositions.
The strongly focused, weakly bisimilar embedding uses the clause $\trconf{P} = \octx$ where $\rfocus{\octx}{\trproc{P}}$.
Consequently, the following $\jrule{C-PROC}$ rule makes use of the right focus pattern judgment to invert a positive proposition to an ordered context.
\begin{inferences}
  \infer[\jrule{C-CUT}\smash{^B}]{\slcof{A |- \octx_1 \oc \octx_2 : C}}{
    \slcof{A |- \octx_1 : B} & \slcof{B |- \octx_2 : C}}
  \and
  \infer[\jrule{C-ID}\smash{^A}] {\slcof{A |- \octxe : A}}{}
  \\
  \infer[\jrule{C-PROC}]{\slcof{A |- \octx : B}}{
    \rfocus{\octx}{\p{A}} & \slof{A |- \p{A} : B}}
\end{inferences}
With these rules, we can prove a result of the now familiar form.
\begin{theorem}
  $\slcof{A |-_{\orsig} \octx : B}$ if, and only if, there exists a configuration $\cnf$ such that $\slcof{A |-_{\sig} \cnf : B}$ and $\trsig{\sig} = \orsig$ and $\trconf{\cnf} = \octx$.
\end{theorem}
\begin{proof}
  From left to right, by structural induction on the typing derivation of $\slcof{A |-_{\orsig} \octx : B}$;
  from right to left, by structural induction on $\slcof{A |-_{\sig} \cnf : B}$.
\end{proof}




\footnote{Careful with the 0-ary forms because you could end up with $\slof{\zero |- \top : C}$ and \emph{also} $\slof{A |- \top : \top}$.}


%%% Local Variables:
%%% mode: latex
%%% TeX-master: "thesis"
%%% End:

%% \chapter{Session-typed ordered rewriting}\label{ch:session-typed-rewriting}

\begin{itemize}
\item Reverse-engineer session typing rules for ordered specifications
\item The translation is invertible for well-typed ordered contexts
\item Typed bisimilarity is somehow equivalent to ordered bisimilarity, when contexts are typed\footnote{Different chapter?}
\item Type inference for ordered specifications\footnote{Different chapter?}
\end{itemize}

%%% Local Variables:
%%% mode: latex
%%% TeX-master: "thesis"
%%% End:

%% \chapter{Isomorphisms}

\section{An asynchronous sequent calculus}

Here are the propositions and proof terms for singleton logic's asynchronous sequent calculus.
\begin{syntax*}
  Propositions & A,B &
    \alpha \mid \plus*[sub=_{\ell \in L}]{\ell:A_{\ell}}
           \mid \with*[sub=_{\ell \in L}]{\ell:A_{\ell}}
  \\
  Proof terms & P,Q &
    \spawn{P}{Q} \mid \fwd
      \begin{array}[t]{@{{} \mid {}}l@{}}
        \selectR{\kay} \mid \caseL[\ell \in L]{\ell => P_{\kay}} \\
        \caseR[\ell \in L]{\ell => P_{\kay}} \mid \selectL{\kay}
      \end{array}
\end{syntax*}
The inference rules for singleton logic's asynchronous sequent calculus are:
\begin{inferences}
  \infer[\jrule{CUT}^B]{\slof{A |- \spawn{P}{Q} : C}}{
    \slof{A |- P : B} & \slof{B |- Q : C}}
  \and
  \infer[\jrule{ID}]{\slof{A |- \fwd : A}}{}
  \\
  \infer[\rrule{\plus}]{\slof{A_{\kay} |- \selectR{\kay} : \plus*[sub=_{\ell \in L}]{\ell:A_{\ell}}}}{
    \text{($\kay \in L$)}}
  \and
  \infer[\lrule{\plus}]{\slof{\plus*[sub=_{\ell \in L}]{\ell:A_{\ell}} |- \caseL[\ell \in L]{\ell => P_{\ell}} : C}}{
    \multipremise{\ell \in L}{\slof{A_{\ell} |- P_{\ell} : C}}}
  \\
  \infer[\rrule{\with}]{\slof{A |- \caseR[\ell \in L]{\ell => P_{\ell}} : \with*[sub=_{\ell \in L}]{\ell:C_{\ell}}}}{
    \multipremise{\ell \in L}{\slof{A |- P_{\ell} : C_{\ell}}}}
  \and
  \infer[\lrule{\with}]{\slof{\with*[sub=_{\ell \in L}]{\ell:C_{\ell}} |- \selectL{\kay} : C_{\kay}}}{
    \text{($\kay \in L$)}}
\end{inferences}



\section{Normalization of proofs}

\subsection{Normal proofs}

\begin{inferences}
  \infer[\jrule{ID}]{\slof{A |- \fwd : A}}{}
  \\
  \infer[]{\slof{A |- \spawn{N}{Q^{\plus}} : C}}{
    \slof{A |- N : B} & \slofp{B ||- Q^{\plus} : C}}
  \and
  \infer[\lrule{\plus}]{\slof{\plus*[sub=_{\ell \in L}]{\ell:A_{\ell}} |- \caseL[\ell \in L]{\ell => P_{\ell}} : C}}{
    \multipremise{\ell \in L}{\slof{A_{\ell} |- P_{\ell} : C}}}
  \\
  \infer[\rrule{\with}]{\slof{A |- \caseR[\ell \in L]{\ell => P_{\ell}} : \with*[sub=_{\ell \in L}]{\ell:C_{\ell}}}}{
    \multipremise{\ell \in L}{\slof{A |- P_{\ell} : C_{\ell}}}}
  \and
  \infer[]{\slof{A |- \spawn{Q^{\with}}{N} : C}}{
    \slofn{A ||- Q^{\with} : B} & \slof{B |- N : C}}
  \\
  \infer[\rrule{\plus}]{\slofp{A ||- \spawn{Q^{\plus}}{\selectR{\kay}} : \plus*[sub=_{\ell \in L}]{\ell:C_{\ell}}}}{
    \text{($\kay \in L$)} & \slofp{A ||- Q^{\plus} : C_{\kay}}}
  \and
  \infer[\n{\jrule{ID}}]{\slofp{\n{A} ||- \fwd : \n{A}}}{}
  \\
  \infer[\lrule{\with}]{\slofn{\with*[sub=_{\ell \in L}]{\ell:A_{\ell}} ||- \spawn{\selectL{\kay}}{Q^{\with}} : C}}{
    \text{($\kay \in L$)} & \slofn{A_{\kay} ||- Q^{\with} : C}}
  \and
  \infer[\p{\jrule{ID}}]{\slofn{\p{A} ||- \fwd : \p{A}}}{}
\end{inferences}

\subsection{Admissibility of normal cut}

\NewDocumentCommand \nspawn { m m } { #1 \mathbin{\blacksquare} #2 }

\begin{equation*}
  \infer-[\jrule{CUT}]{\slof{A |- \nspawn{M}{N} : C}}{
    \slof{A |- M : B} & \slof{B |- N : C}}
\end{equation*}

\begin{gather*}
  \infer-[\jrule{CUT}^B]{\slof{A |- \nspawn{(\spawn{M_0}{(\spawn{Q^{\plus}}{\selectR{\kay}})})}{\caseL[\ell \in L]{\ell => N_{\ell}}} : C}}{
    \infer[]{\slof{A |- \spawn{M_0}{(\spawn{Q^{\plus}}{\selectR{\kay}})} : \plus*[sub=_{\ell \in L}]{\ell:B_{\ell}}}}{
      \slof{A |- M_0 : A'} &
      \infer[\rrule{\plus}]{\slofp{A' ||- \spawn{Q^{\plus}}{\selectR{\kay}} : \plus*[sub=_{\ell \in L}]{\ell:B_{\ell}}}}{
        \text{($\kay \in L$)} & \slofp{A' ||- Q^{\plus} : B_{\kay}}}} &
    \infer[\lrule{\plus}]{\slof{\plus*[sub=_{\ell \in L}]{\ell:B_{\ell}} |- \caseL[\ell \in L]{\ell => N_{\ell}} : C}}{
      \multipremise{\ell \in L}{\slof{B_{\ell} |- N_{\ell} : C}}}}
  \\=\\
  \infer-[\jrule{CUT}^{B_{\kay}}]{\slof{A |- \nspawn{(\spawn{M_0}{Q^{\plus}})}{N_{\kay}} : C}}{
    \infer[]{\slof{A |- \spawn{M_0}{Q^{\plus}} : B_{\kay}}}{
      \slof{A |- M_0 : A'} & \slofp{A' ||- Q^{\plus} : B_{\kay}}} &
    \slof{B_{\kay} |- N_{\kay} : C}}
\end{gather*}

\begin{gather*}
  \infer-[\jrule{CUT}^{\n{B}}]{\slof{A |- \nspawn{(\spawn{M_0}{\fwd})}{N} : C}}{
    \infer[]{\slof{A |- \spawn{M}{\fwd} : \n{B}}}{
      \slof{A |- M_0 : \n{B}} &
      \infer[\n{\jrule{ID}}]{\slofp{\n{B} ||- \fwd : \n{B}}}{}} &
    \slof{\n{B} |- N : C}}
  \\=\\
  \infer-[\jrule{CUT}^{\n{B}}]{\slof{A |- \nspawn{M_0}{N} : C}}{
    \slof{A |- M_0 : \n{B}} & \slof{\n{B} |- N : C}}
\end{gather*}

\begin{gather*}
  \infer-[\jrule{CUT}^B]{\slof{A |- \nspawn{(\spawn{Q^{\with}}{M_0})}{N} : C}}{
    \infer[]{\slof{A |- \spawn{Q^{\with}}{M_0} : B}}{
      \slofn{A ||- Q^{\with} : B_0} & \slof{B_0 |- M_0 : B}} &
    \slof{B |- N : C}}
  \\=\\
  \infer[]{\slof{A |- \spawn{Q^{\with}}{(\nspawn{M_0}{N})} : C}}{
    \slofn{A ||- Q^{\with} : B_0} &
    \infer-[\jrule{CUT}^B]{\slof{B_0 |- \nspawn{M_0}{N} : B}}{
      \slof{B_0 |- M_0 : B} & \slof{B |- N : C}}}
\end{gather*}

\begin{gather*}
  \infer-[\jrule{CUT}^B]{\slof{A |- \nspawn{\caseL[\ell \in L]{\ell => M_{\ell}}}{(\spawn{Q^{\with}}{N_0})} : C}}{
    \infer[\lrule{\plus}]{\slof{\plus*[sub=_{\ell \in L}]{\ell:A_{\ell}} |- \caseL[\ell \in L]{\ell => M_{\ell}} : B}}{
      \multipremise{\ell \in L}{\slof{A_{\ell} |- M_{\ell} : B}}} &
    \infer[]{\slof{B |- \spawn{Q^{\with}}{N_0} : C}}{
      \slofn{B ||- Q^{\with} : B_0} & \slof{B_0 |- N_0 : C}}}
  \\=\\
  \infer-[\jrule{CUT}^B]{\slof{A |- \nspawn{\caseL[\ell \in L]{\ell => \nspawn{M_{\ell}}{(\spawn{Q^{\with}}{\fwd})}}}{N_0} : C}}{
    \infer[\lrule{\plus}]{\slof{\plus*[sub=_{\ell \in L}]{\ell:A_{\ell}} |- \caseL[\ell \in L]{\ell => \nspawn{M_{\ell}}{(\spawn{Q^{\with}}{\fwd})}} : B}}{
      \multipremise{\ell \in L}{
        \infer-[\jrule{CUT}^B]{\slof{A_{\ell} |- \nspawn{M_{\ell}}{(\spawn{Q^{\with}}{\fwd})} : B_0}}{
          \slof{A_{\ell} |- M_{\ell} : B} &
          \infer[]{\slof{B |- \spawn{Q^{\with}}{\fwd} : B_0}}{
            \slofn{B ||- Q^{\with} : B_0} &
            \infer[\jrule{ID}]{\slof{B_0 |- \fwd : B_0}}{}}}}} &
    \slof{B_0 |- N_0 : C}}
\end{gather*}

\subsection{}

We would like to prove a normalization theorem for the asynchronous sequent calculus.
Unlike the usual, synchronous sequent calculus normalization will not be a process of cut elimination.
For example, the provable sequent $\slseq{\alpha |- \plus*{a:\plus*{b:\alpha}}}$ has no cut-free proof.

Proof terms form a monoid under the $\spawn{}{}$ constructor, with $\fwd$ as the unit element.
Associativity and unit laws hold:
\begin{gather*}
  \infer[\jrule{CUT}^{B_2}]{\slof{A |- \spawn{(\spawn{P}{Q})}{R} : C}}{
    \infer[\jrule{CUT}^{B_1}]{\slof{A |- \spawn{P}{Q} : B_2}}{
      \slof{A |- P : B_1} & \slof{B_1 |- Q : B_2}} &
    \slof{B_2 |- R : C}}
  \equiv
  \infer[\jrule{CUT}^{B_1}]{\slof{A |- \spawn{P}{(\spawn{Q}{R})} : C}}{
    \slof{A |- P : B_1} &
    \infer[\jrule{CUT}^{B_2}]{\slof{B_1 |- \spawn{Q}{R} : C}}{
      \slof{B_1 |- Q : B_2} & \slof{B_2 |- R : C}}}
  \\
  \infer[\jrule{CUT}^A]{\slof{A |- \spawn{\fwd}{P} : C}}{
    \infer[\jrule{ID}^A]{\slof{A |- \fwd : A}}{} &
    \slof{A |- P : C}}
  \equiv
  \slof{A |- P : C}
  \equiv
  \infer[\jrule{CUT}^C]{\slof{A |- \spawn{P}{\fwd} : C}}{
    \slof{A |- P : C} &
    \infer[\jrule{ID}^C]{\slof{C |- \fwd : C}}{}}
\end{gather*}
In this way, the $\jrule{CUT}$ rule can be seen as a form of composition for proof terms.

Modulo these associativity and unit laws, normal proof terms are those given by the following grammar.
\begin{syntax*}
  Normal terms & N &
    \spawn{Q^{\with}}{Q^{\plus}}
      \begin{array}[t]{@{{} \mid {}}l@{}}
        \spawn{Q^{\with}}{\caseL[\ell \in L]{\ell => N_{\ell}}} \\
        \spawn{\caseR[\ell \in L]{\ell => N_{\ell}}}{Q^{\plus}}
      \end{array}
  \\
  Q{ueues} & Q^{\plus} &
    \fwd \mid \spawn{Q^{\plus}_1}{Q^{\plus}_2} \mid \selectR{\kay}
  \\
           & Q^{\with} &
    \fwd \mid \spawn{Q^{\with}_1}{Q^{\with}_2} \mid \selectL{\kay}
\end{syntax*}
We will adopt the following as reductions.
\begin{gather*}
  \spawn{\selectR{\kay}}{\caseL[\ell \in L]{\ell => N_{\ell}}}
    \reduces N_{\kay}
  \\
  \spawn{\caseR[\ell \in L]{\ell => M_{\ell}}}{\selectL{\kay}}
    \reduces M_{\kay}
  \\
  \spawn{\caseL[\ell \in L]{\ell => M_{\ell}}}{N}
    \reduces \caseL[\ell \in L]{\ell => \spawn{M_{\ell}}{N}}
  \\
  \spawn{M}{\caseR[\ell \in L]{\ell => N_{\ell}}}
    \reduces \caseR[\ell \in L]{\ell => \spawn{M}{N_{\ell}}}
\end{gather*}

\begin{lemma}
  For all $\plus$-queues $Q^{\plus}$, either $Q^{\plus} \equiv \fwd$ or $Q^{\plus} \equiv \spawn{Q^{\plus}_0}{\selectR{\kay}}$ for some $\plus$-queue $Q^{\plus}_0$.
  Similarly, for all $\with$-queues $Q^{\with}$, either $Q^{\with} \equiv \fwd$ or $Q^{\with} \equiv \spawn{\selectL{\kay}}{Q^{\with}_0}$ for some $\with$-queue $Q^{\with}_0$.
\end{lemma}

\begin{theorem}[Weak normalization]
  If $\slof{A |- M : B}$ and $\slof{B |- N : C}$ are normal terms, then $\spawn{M}{N} \Reduces N'$ for some normal term $N'$.
\end{theorem}
%
\begin{proof}
  We follow the example of standard cut elimination, using lexicographic induction, first on the principal type and then simultaneously on the given derivations.
  The several cases are organized into identity reductions, principal reductions, and left and right commutative reductions.
  We frequently appeal to the preceding lemma.
  \begin{description}
  \item[Identity cut reductions]
    One case has $M = \spawn{Q^{\with}_1}{Q^{\plus}_1}$ and $N = \spawn{Q^{\with}_2}{Q^{\plus}_2}$.
    More specifically:
    \begin{equation*}
      \infer-[\jrule{CUT}^C]{\slof{A |- \spawn{(\spawn{Q^{\with}_1}{Q^{\plus}_1})}{(\spawn{Q^{\with}_2}{Q^{\plus}_2})} : E}}{
        \infer[\jrule{CUT}^B]{\slof{A |- \spawn{Q^{\with}_1}{Q^{\plus}_1} : C}}{
          \slof{A |- Q^{\with}_1 : B} &
          \slof{B |- Q^{\plus}_1 : C}} &
        \infer[\jrule{CUT}^D]{\slof{C |- \spawn{Q^{\with}_2}{Q^{\plus}_2} : E}}{
          \slof{C |- Q^{\with}_2 : D} &
          \slof{D |- Q^{\plus}_2 : E}}}
    \end{equation*}

    There are two subcases according to the shape of $Q^{\plus}_1$: either $Q^{\plus}_1 \equiv \fwd$ or $Q^{\plus}_1 \equiv \spawn{Q^{\plus}_0}{\selectR{\kay}_1}$ for some $\plus$-queue $Q^{\plus}_0$.
    \begin{itemize}
    \item Consider the subcase in which $Q^{\plus}_1 \equiv \fwd$.
      Using associativity, we have
      \begin{gather*}
        \slof{A |- \spawn{(\spawn{Q^{\with}_1}{Q^{\plus}_1})}{(\spawn{Q^{\with}_2}{Q^{\plus}_2})} : E}
        \\\equiv\\
        \infer[\jrule{CUT}^D]{\slof{A |- \spawn{(\spawn{Q^{\with}_1}{Q^{\with}_2})}{Q^{\plus}_2} : E}}{
          \infer[\jrule{CUT}^C]{\slof{A |- \spawn{Q^{\with}_1}{Q^{\with}_2} : D}}{
            \slof{A |- Q^{\with}_1 : C} &
            \slof{C |- Q^{\with}_2 : D}} &
          \slof{D |- Q^{\plus}_2 : E}}
      \end{gather*}
      which is normal.

    \item Consider the subcase in which $Q^{\plus}_1 \equiv \spawn{Q^{\plus}_0}{\selectR{\kay}_1}$ for some $\plus$-queue $Q^{\plus}_0$.
      In this subcase, we must also have $Q^{\with}_2 \equiv \fwd$, otherwise this term is ill-typed.
      Because $Q^{\with}_2 \equiv \fwd$, this subcase is symmetric to the previous one.
    \end{itemize}

  \item[Principal cut reductions]
    One case has $M = \spawn{Q^{\with}_1}{Q^{\plus}}$ and $N = \spawn{Q^{\with}_2}{\caseL[\ell \in L]{\ell => N_{\ell}}}$.
    More specifically:
    \begin{equation*}
      \infer-[\jrule{CUT}^B]{\slof{A |- \spawn{(\spawn{Q^{\with}_1}{Q^{\plus}})}{(\spawn{Q^{\with}_2}{\caseL[\ell \in L]{\ell => N_{\ell}}})} : D}}{
        \slof{A |- \spawn{Q^{\with}_1}{Q^{\plus}} : B} &
        \infer[\jrule{CUT}^C]{\slof{B |- \spawn{Q^{\with}_2}{\caseL[\ell \in L]{\ell => N_{\ell}}} : D}}{
          \slof{B |- Q^{\with}_2 : \plus*[sub=_{\ell \in L}]{\ell:C_{\ell}}} &
          \infer[\lrule{\plus}]{\slof{\plus*[sub=_{\ell \in L}]{\ell:C_{\ell}} |- \caseL[\ell \in L]{\ell => N_{\ell}} : D}}{
            \multipremise{\ell \in L}{\slof{C_{\ell} |- N_{\ell} : D}}}}}
    \end{equation*}

    There are two subcases according to the shape of $Q^{\plus}$: either $Q^{\plus} \equiv \fwd$ or $Q^{\plus} \equiv \spawn{Q^{\plus}_0}{\selectR{\kay}}$ for some $\plus$-queue $Q^{\plus}_0$.
    \begin{itemize}
    \item Consider the subcase in which $Q^{\plus} \equiv \fwd$.
      In this subcase, $\slof{A |- Q^{\with}_1 : B}$.
      Using associativity, we have:
      \begin{gather*}
        % \infer-[\jrule{CUT}^B]{\slof{A |- \spawn{(\spawn{Q^{\with}_1}{Q^{\plus}})}{(\spawn{Q^{\with}_2}{\caseL[\ell \in L]{\ell => N_{\ell}}})} : D}}{
        % \slof{A |- \spawn{Q^{\with}_1}{Q^{\plus}} : B} &
        % \infer[\jrule{CUT}^C]{\slof{B |- \spawn{Q^{\with}_2}{\caseL[\ell \in L]{\ell => N_{\ell}}} : D}}{
        % \slof{B |- Q^{\with}_2 : \plus*[sub=_{\ell \in L}]{\ell:C_{\ell}}} &
        % \infer[\lrule{\plus}]{\slof{\plus*[sub=_{\ell \in L}]{\ell:C_{\ell}} |- \caseL[\ell \in L]{\ell => N_{\ell}} : D}}{
        % \multipremise{\ell \in L}{\slof{C_{\ell} |- N_{\ell} : D}}}}}
        \slof{A |- \spawn{(\spawn{Q^{\with}_1}{Q^{\plus}})}{(\spawn{Q^{\with}_2}{\caseL[\ell \in L]{\ell => N_{\ell}}})} : D}
        \\\equiv\\
        \slof{A |- \spawn{(\spawn{Q^{\with}_1}{\fwd})}{(\spawn{Q^{\with}_2}{\caseL[\ell \in L]{\ell => N_{\ell}}})} : D}
        \\\equiv\\
        \infer[\jrule{CUT}^C]{\slof{A |- \spawn{(\spawn{Q^{\with}_1}{Q^{\with}_2})}{\caseL[\ell \in L]{\ell => N_{\ell}}} : D}}{
          \infer[\jrule{CUT}^B]{\slof{A |- \spawn{Q^{\with}_1}{Q^{\with}_2} : \plus*[sub=_{\ell \in L}]{\ell:C_{\ell}}}}{
            \slof{A |- Q^{\with}_1 : B} &
            \slof{B |- Q^{\with}_2 : \plus*[sub=_{\ell \in L}]{\ell:C_{\ell}}}} &
          \infer[\lrule{\plus}]{\slof{\plus*[sub=_{\ell \in L}]{\ell:C_{\ell}} |- \caseL[\ell \in L]{\ell => N_{\ell}} : D}}{
            \multipremise{\ell \in L}{\slof{C_{\ell} |- N_{\ell} : D}}}}
      \end{gather*}
      which is normal.

    \item Consider the subcase in which $Q^{\plus} \equiv \spawn{Q^{\plus}_0}{\selectR{\kay}}$ for some $\plus$-queue $Q^{\plus}_0$.
      For this process to be well-typed, we must have $Q^{\with}_2 \equiv \fwd$.
      Using associativity and a principal reduction, we have:
      \begin{gather*}
        % \infer-[\jrule{CUT}^B]{\slof{A |- \spawn{(\spawn{Q^{\with}_1}{Q^{\plus}})}{(\spawn{Q^{\with}_2}{\caseL[\ell \in L]{\ell => N_{\ell}}})} : D}}{
        % \slof{A |- \spawn{Q^{\with}_1}{Q^{\plus}} : B} &
        % \infer[\jrule{CUT}^C]{\slof{B |- \spawn{Q^{\with}_2}{\caseL[\ell \in L]{\ell => N_{\ell}}} : D}}{
        % \slof{B |- Q^{\with}_2 : \plus*[sub=_{\ell \in L}]{\ell:C_{\ell}}} &
        % \infer[\lrule{\plus}]{\slof{\plus*[sub=_{\ell \in L}]{\ell:C_{\ell}} |- \caseL[\ell \in L]{\ell => N_{\ell}} : D}}{
        % \multipremise{\ell \in L}{\slof{C_{\ell} |- N_{\ell} : D}}}}}
        \slof{A |- \spawn{(\spawn{Q^{\with}_1}{Q^{\plus}})}{(\spawn{Q^{\with}_2}{\caseL[\ell \in L]{\ell => N_{\ell}}})} : D}
        \\\equiv\\
        % \infer-[\jrule{CUT}^C]{\slof{A |- \spawn{(\spawn{Q^{\with}_1}{(\spawn{Q^{\plus}_0}{\selectR{\kay}})})}{\caseL[\ell \in L]{\ell => N_{\ell}}} : D}}{
        % \infer[\jrule{CUT}^{B'}]{\slof{A |- \spawn{Q^{\with}_1}{(\spawn{Q^{\plus}_0}{\selectR{\kay}})} : \plus*[sub=_{\ell \in L}]{\ell:C_{\ell}}}}{
        % \slof{A |- Q^{\with}_1 : B'} &
        % \infer[\jrule{CUT}^{C_{\kay}}]{\slof{B' |- \spawn{Q^{\plus}_0}{\selectR{\kay}} : \plus*[sub=_{\ell \in L}]{\ell:C_{\ell}}}}{
        % \slof{B' |- Q^{\plus}_0 : C_{\kay}} &
        % \infer[\rrule{\plus}]{\slof{C_{\kay} |- \selectR{\kay} : \plus*[sub=_{\ell \in L}]{\ell:C_{\ell}}}}{
        % \text{($\kay \in L$)}}}} &
        % \infer[\lrule{\plus}]{\slof{\plus*[sub=_{\ell \in L}]{\ell:C_{\ell}} |- \caseL[\ell \in L]{\ell => N_{\ell}} : D}}{
        % \multipremise{\ell \in L}{\slof{C_{\ell} |- N_{\ell} : D}}}}
        \slof{A |- \spawn{(\spawn{Q^{\with}_1}{(\spawn{Q^{\plus}_0}{\selectR{\kay}})})}{\caseL[\ell \in L]{\ell => N_{\ell}}} : D}
        \\\equiv\reduces\\
        \infer-[\jrule{CUT}^{C_{\kay}}]{\slof{A |- \spawn{(\spawn{Q^{\with}_1}{Q^{\plus}_0})}{N_{\kay}} : D}}{
          \infer[\jrule{CUT}^{B'}]{\slof{A |- \spawn{Q^{\with}_1}{Q^{\plus}_0} : C_{\kay}}}{
            \slof{A |- Q^{\with}_1 : B'} &
            \slof{B' |- Q^{\plus}_0 : C_{\kay}}} &
          \slof{C_{\kay} |- N_{\kay} : D}}
      \end{gather*}
    \end{itemize}
    Notice that the appeal to the inductive hypothesis is made at a smaller type, namely $C_{\kay}$.

    Another case has $M = \spawn{\caseR[\ell \in L]{\ell => M_{\ell}}}{Q^{\plus}_1}$ and $N = \spawn{Q^{\with}}{Q^{\plus}_2}$.
    This case is symmetric to the previous one.

    Another case has $M = \spawn{\caseR[\ell \in L]{\ell => M_{\ell}}}{Q^{\plus}}$ and $N = \spawn{Q^{\with}}{\caseL[i \in I]{i => N_i}}$.
    More specifically: 
    \begin{equation*}
      \infer-[\jrule{CUT}^B]{\slof{A |- \spawn{(\spawn{Q^{\with}_1}{Q^{\plus}})}{(\spawn{Q^{\with}_2}{\caseL[\ell \in L]{\ell => N_{\ell}}})} : D}}{
        \deduce{\slof{A |- \spawn{\caseR[\ell \in L]{\ell => M_{\ell}}}{Q^{\plus}} : B}}{\vdots} &
        \infer[\jrule{CUT}^C]{\slof{B |- \spawn{Q^{\with}}{\caseL[i \in I]{i => N_i}} : D}}{
          \slof{B |- Q^{\with} : \plus*[sub=_{i \in i}]{i:C_i}} &
          \infer[\lrule{\plus}]{\slof{\plus*[sub=_{i \in I}]{i:C_i} |- \caseL[i \in I]{i => N_i} : D}}{
            \multipremise{i \in I}{\slof{C_i |- N_i : D}}}}}
    \end{equation*}
    
    There are two subcases according to the shape of $Q^{\plus}$: either $Q^{\plus} \equiv \fwd$ or $Q^{\plus} \equiv \spawn{Q^{\plus}_0}{\selectR{\kay}}$ for some $\plus$-queue $Q^{\plus}_0$.
    \begin{itemize}
    \item Consider the subcase in which $Q^{\plus} \equiv \fwd$.
      In this subcase, we must have $Q^{\with} \equiv \spawn{\selectL{\kay}}{Q^{\with}_0}$ for some $\with$-queue, otherwise the term will be ill-typed.
      Using associativity and a principal reduction, we have:
      \begin{gather*}
        % \infer-[\jrule{CUT}^B]{\slof{A |- \spawn{\caseR[\ell \in L]{\ell => M_{\ell}}}{(\spawn{(\spawn{\selectL{\kay}}{Q^{\with}_0})}{\caseL[i \in I]{i => N_i}})} : D}}{
        %   \infer[\rrule{\with}]{\slof{A |- \caseR[\ell \in L]{\ell => M_{\ell}} : \with*[sub=_{\ell \in L}]{\ell:B_{\ell}}}}{
        %     \multipremise{\ell \in L}{\slof{A |- M_{\ell} : B_{\ell}}}} &
        %   \infer[\jrule{CUT}^C]{\slof{\with*[sub=_{\ell \in L}]{\ell:B_{\ell}} |- \spawn{(\spawn{\selectL{\kay}}{Q^{\with}_0})}{\caseL[i \in I]{i => N_i}} : D}}{
        %     \infer[\jrule{CUT}^{B_{\kay}}]{\slof{\with*[sub=_{\ell \in L}]{\ell:B_{\ell}} |- \spawn{\selectL{\kay}}{Q^{\with}_0} : \plus*[sub=_{i \in i}]{i:C_i}}}{
        %       \infer[\lrule{\with}]{\slof{\with*[sub=_{\ell \in L}]{\ell:B_{\ell}} |- \selectL{\kay} : B_{\kay}}}{
        %         \text{($\kay \in L$)}} &
        %       \slof{B_{\kay} |- Q^{\with}_0 : \plus*[sub=_{i \in i}]{i:C_i}}} &
        %     \infer[\lrule{\plus}]{\slof{\plus*[sub=_{i \in I}]{i:C_i} |- \caseL[i \in I]{i => N_i} : D}}{
        %       \multipremise{i \in I}{\slof{C_i |- N_i : D}}}}}
        \slof{A |- \spawn{(\spawn{\caseR[\ell \in L]{\ell => M_{\ell}}}{Q^{\plus}})}{(\spawn{Q^{\with}}{\caseL[i \in I]{i => N_i}})} : D}
        \\\equiv\\
        \slof{A |- \spawn{(\spawn{\caseR[\ell \in L]{\ell => M_{\ell}}}{\fwd})}{(\spawn{(\spawn{\selectL{\kay}}{Q^{\with}_0})}{\caseL[i \in I]{i => N_i}})} : D}
        \\\equiv\reduces\\
        \infer-[\jrule{CUT}^{B_{\kay}}]{\slof{A |- \spawn{M_{\kay}}{(\spawn{Q^{\with}_0}{\caseL[i \in I]{i => N_i}})} : D}}{
          \slof{A |- M_{\kay} : B_{\kay}} &
          \infer[\jrule{CUT}^C]{\slof{B_{\kay} |- \spawn{Q^{\with}_0}{\caseL[i \in I]{i => N_i}} : D}}{
            \slof{B_{\kay} |- Q^{\with}_0 : \plus*[sub=_{i \in i}]{i:C_i}} &
            \infer[\lrule{\plus}]{\slof{\plus*[sub=_{i \in I}]{i:C_i} |- \caseL[i \in I]{i => N_i} : D}}{
              \multipremise{i \in I}{\slof{C_i |- N_i : D}}}}}
      \end{gather*}
    Notice that the appeal to the inductive hypothesis is made at a smaller type, namely $B_{\kay}$.

    \item Consider the subcase in which $Q^{\plus} \equiv \spawn{Q^{\plus}_0}{\selectR{\kay}}$ for some $\plus$-queue.
      In this subcase, we must have $Q^{\with} \equiv \fwd$, otherwise the term will be ill-typed.
      This subcase is symmetric to the previous one.
      % \begin{gather*}
      %   \slof{A |- \spawn{(\spawn{\caseR[\ell \in L]{\ell => M_{\ell}}}{Q^{\plus}})}{(\spawn{Q^{\with}}{\caseL[i \in I]{i => N_i}})} : D}
      %   \\\equiv\\
      %   \slof{A |- \spawn{(\spawn{\caseR[\ell \in L]{\ell => M_{\ell}}}{(\spawn{Q^{\plus}_0}{\selectR{\kay}})})}{(\spawn{\fwd}{\caseL[i \in I]{i => N_i}})} : D}
      %   \\\equiv\reduces\\
      %   \slof{A |- \spawn{(\spawn{\caseR[\ell \in L]{\ell => M_{\ell}}}{Q^{\plus}_0})}{N_{\kay}} : D}
      % \end{gather*}
    \end{itemize}

  \item[Left commutative cut reductions]
    One case has $M = \spawn{Q^{\with}}{\caseL[\ell \in L]{\ell => M_{\ell}}}$.
    More specifically:
    \begin{equation*}
      \infer-[\jrule{CUT}^C]{\slof{A |- \spawn{(\spawn{Q^{\with}}{\caseL[\ell \in L]{\ell => M_{\ell}}})}{N} : D}}{
        \infer[\jrule{CUT}^B]{\slof{A |- \spawn{Q^{\with}}{\caseL[\ell \in L]{\ell => M_{\ell}}} : C}}{
          \slof{A |- Q^{\with} : \plus*[sub=_{\ell \in L}]{\ell:B_{\ell}}} &
          \infer[\lrule{\plus}]{\slof{\plus*[sub=_{\ell \in L}]{\ell:B_{\ell}} |- \caseL[\ell \in L]{\ell => M_{\ell}} : C}}{
            \multipremise{\ell \in L}{\slof{B_{\ell} |- M_{\ell} : C}}}} &
        \slof{C |- N : D}}
    \end{equation*}
    Using associativity and a left commutative reduction, we have:
    \begin{gather*}
      \slof{A |- \spawn{(\spawn{Q^{\with}}{\caseL[\ell \in L]{\ell => M_{\ell}}})}{N} : D}
      \\\equiv\reduces\\
      \infer[\jrule{CUT}^B]{\slof{A |- \spawn{Q^{\with}}{\caseL[\ell \in L]{\ell => \spawn{M_{\ell}}{N}}} : D}}{
        \slof{A |- Q^{\with} : \plus*[sub=_{\ell \in L}]{\ell:B_{\ell}}} &
        \infer[\lrule{\plus}]{\slof{\plus*[sub=_{\ell \in L}]{\ell:B_{\ell}} |- \caseL[\ell \in L]{\ell => \spawn{M_{\ell}}{N}} : D}}{
          \multipremise{\ell \in L}{
            \infer-[\jrule{CUT}^C]{\slof{B_{\ell} |- \spawn{M_{\ell}}{N} : D}}{
              \slof{B_{\ell} |- M_{\ell} : C} &
              \slof{C |- N : D}}}}}
    \end{gather*}
    The appeal to the inductive hypothesis for $\spawn{M_{\ell}}{N}$ is justified because the left term is smaller, while the right term is unchanged.

  \item[Right commutative cut reductions]
    One case has $N = \spawn{\caseR[\ell \in L]{\ell => N_{\ell}}}{Q^{\plus}}$.
    This case is symmetric to the above left commutative case.
    \qedhere
  \end{description}
\end{proof}


Notice the following lemma is also provable. 
\begin{lemma}
  If $N$ is a normal term, then either:
  \begin{itemize}[nosep]
  \item $N \equiv \fwd$;
  \item $N \equiv \caseL[\ell \in L]{\ell => N_{\ell}}$ for some normal terms $(N_{\ell})_{\ell \in L}$;
  \item $N \equiv \caseR[\ell \in L]{\ell => N_{\ell}}$ for some normal terms $(N_{\ell})_{\ell \in L}$;
  \item $N \equiv \spawn{\selectL{\kay}}{N_0}$ for some normal term $N_0$; or
  \item $N \equiv \spawn{N_0}{\selectR{\kay}}$ for some normal term $N_0$.
  \end{itemize}
\end{lemma}

I was tempted to use this lemma to reorganize the above proof of normalization.
%
\begin{theorem}[Weak normalization]
  If $\slof{A |- M : B}$ and $\slof{B |- N : C}$ are normal terms, then $\spawn{M}{N} \Reduces N'$ for some normal term $N'$.
\end{theorem}
%
\begin{proof}[Failed proof attempt]
  \begin{description}
  % \item[Identity cut ``reductions'']
  %   In the cases in which $M \equiv \fwd$ or $N \equiv \fwd$, we may follow the example of standard cut elimination.
  %   Here, however, these are not reductions, but equivalent terms.
  %   \begin{gather*}
  %     \infer-[\jrule{CUT}^A]{\slof{A |- \spawn{\fwd}{N} : C}}{
  %       \infer[\jrule{ID}^A]{\slof{A |- \fwd : A}}{} &
  %       \slof{A |- N : C}}
  %     \equiv
  %     \slof{A |- N : C}
  %     \shortintertext{and}
  %     \infer-[\jrule{CUT}^B]{\slof{A |- \spawn{M}{\fwd} : B}}{
  %       \slof{A |- M : B} &
  %       \infer[\jrule{ID}^B]{\slof{B |- \fwd : B}}{}}
  %     \equiv
  %     \slof{A |- M : B}
  %   \end{gather*}

  % \item[Principal cut reductions]
  %   Two cases are like the principal cut reductions found in standard cut elimination.
  %   For these cases, we will adopt the reductions
  %   \begin{gather*}
  %     \spawn{\selectR{\kay}}{\caseL[\ell \in L]{\ell => N_{\ell}}} \reduces N_{\kay} \\
  %     \spawn{\caseR[\ell \in L]{\ell => M_{\ell}}}{\selectL{\kay}} \reduces M_{\kay}
  %   \end{gather*}

  %   One of the principal cases has $M \equiv \spawn{M_0}{\selectR{\kay}}$ and $N \equiv \caseL[\ell \in L]{\ell => N_{\ell}}$.
  %   Using associativity and one of the above reductions, we have 
  %   \begin{gather*}
  %     \infer-[\jrule{CUT}^B]{\slof{A |- \spawn{(\spawn{M_0}{\selectR{\kay}})}{\caseL[\ell \in L]{\ell => N_{\ell}}} : C}}{
  %       \infer[\jrule{CUT}^{B_{\kay}}]{\slof{A |- \spawn{M_0}{\selectR{\kay}} : \plus*[sub=_{\ell \in L}]{\ell:B_{\ell}}}}{
  %         \slof{A |- M_0 : B_{\kay}} &
  %         \infer[\rrule{\plus}]{\slof{B_{\kay} |- \selectR{\kay} : \plus*[sub=_{\ell \in L}]{\ell:B_{\ell}}}}{
  %           \text{($\kay \in L$)}}} &
  %       \infer[\lrule{\plus}]{\slof{\plus*[sub=_{\ell \in L}]{\ell:B_{\ell}} |- \caseL[\ell \in L]{\ell => N_{\ell}} : C}}{
  %         \multipremise{\ell \in L}{\slof{B_{\ell} |- N_{\ell} : C}}}}
  %     \\\equiv\reduces\\
  %     \infer-[\jrule{CUT}^{B_{\kay}}]{\slof{A |- \spawn{M_0}{N_{\kay}} : C}}{
  %       \slof{A |- M_0 : B_{\kay}} &
  %       \slof{B_{\kay} |- N_{\kay} : C}}
  %   \end{gather*}

  %   The other principal case has $M \equiv \caseR[\ell \in L]{\ell => M_{\ell}}$ and $N \equiv \spawn{\selectL{\kay}}{N_0}$.
  %   Similarly using associativity and the other above reduction, we have 
  %   \begin{gather*}
  %     \infer-[\jrule{CUT}^B]{\slof{A |- \spawn{\caseR[\ell \in L]{\ell => M_{\ell}}}{(\spawn{\selectL{\kay}}{N_0})} : C}}{
  %       \infer[\rrule{\with}]{\slof{A |- \caseR[\ell \in L]{\ell => M_{\ell}} : \with*[sub=_{\ell \in L}]{\ell:B_{\ell}}}}{
  %         \multipremise{\ell \in L}{\slof{A |- M_{\ell} : B_{\ell}}}} &
  %       \infer[\jrule{CUT}^{B_{\kay}}]{\slof{\with*[sub=_{\ell \in L}]{\ell:B_{\ell}} |- \spawn{\selectL{\kay}}{N_0} : C}}{
  %         \infer[\lrule{\with}]{\slof{\with*[sub=_{\ell \in L}]{\ell:B_{\ell}} |- \selectL{\kay} : B_{\kay}}}{
  %           \text{($\kay \in L$)}} &
  %         \slof{B_{\kay} |- N_0 : C}}}
  %     \\\equiv\reduces\\
  %     \infer-[\jrule{CUT}^{B_{\kay}}]{\slof{A |- \spawn{M_{\kay}}{N_0} : C}}{
  %       \slof{A |- M_{\kay} : B_{\kay}} &
  %       \slof{B_{\kay} |- N_0 : C}}
  %   \end{gather*}

  \item[Left commutative cut reductions]
    There are two left commutative cut reductions.
    The first of these arises from the case in which $M \equiv \spawn{\selectL{\kay}}{M_0}$.
    Using only associativity, we can push the non-analytic cut up:
    \begin{gather*}
      \infer-[\jrule{CUT}^B]{\slof{\with*[sub=_{\ell \in L}]{\ell:A_{\ell}} |- \spawn{(\spawn{\selectL{\kay}}{M_0})}{N} : C}}{
        \infer[\jrule{CUT}^{A_{\kay}}]{\slof{\with*[sub=_{\ell \in L}]{\ell:A_{\ell}} |- \spawn{\selectL{\kay}}{M_0} : B}}{
          \infer[\lrule{\with}]{\slof{\with*[sub=_{\ell \in L}]{\ell:A_{\ell}} |- \selectL{\kay} : A_{\kay}}}{
            \text{($\kay \in L$)}} &
          \slof{A_{\kay} |- M_0 : B}} &
        \slof{B |- N : C}}
      \\\equiv\\
      \infer[\jrule{CUT}^{A_{\kay}}]{\slof{\with*[sub=_{\ell \in L}]{\ell:A_{\ell}} |- \spawn{\selectL{\kay}}{(\spawn{M_0}{N})} : C}}{
        \infer[\lrule{\with}]{\slof{\with*[sub=_{\ell \in L}]{\ell:A_{\ell}} |- \selectL{\kay} : A_{\kay}}}{
          \text{($\kay \in L$)}} &
        \infer-[\jrule{CUT}^B]{\slof{A_{\kay} |- \spawn{M_0}{N} : C}}{
          \slof{A_{\kay} |- M_0 : B} & \slof{B |- N : C}}}
    \end{gather*}
    Consider the specific instance in which $M = \spawn{\selectL{\kay}}{\fwd}$ and $N = \caseR[\ell \in L]{\ell => N_{\ell}}$.
    The above reasoning suggests that $\spawn{\selectL{\kay}}{\caseR[\ell \in L]{\ell => N_{\ell}}}$ is a normal term.
    But that is not true -- $\spawn{\selectL{\kay}}{\caseR[\ell \in L]{\ell => N_{\ell}}} \reduces \caseR[\ell \in L]{\ell => \spawn{\selectL{\kay}}{N_{\ell}}}$.

    % The other left commutative case occurs when $M \equiv \caseL[\ell \in L]{\ell => M_{\ell}}$.
    % If we adopt
    % \begin{equation*}
    %   \spawn{\caseL[\ell \in L]{\ell => M_{\ell}}}{N} \reduces \caseL[\ell \in L]{\ell => \spawn{M_{\ell}}{N}}
    % \end{equation*}
    % as a reduction, then
    % The non-analytic cut can be pushed up:
    % \begin{gather*}
    %   \infer-[\jrule{CUT}^B]{\slof{\plus*[sub=_{\ell \in L}]{\ell:A_{\ell}} |- \spawn{\caseL[\ell \in L]{\ell => M_{\ell}}}{N} : C}}{
    %     \infer[\lrule{\plus}]{\slof{\plus*[sub=_{\ell \in L}]{\ell:A_{\ell}} |- \caseL[\ell \in L]{\ell => M_{\ell}} : B}}{
    %       \multipremise{\ell \in L}{\slof{A_{\ell} |- M_{\ell} : B}}} &
    %     \slof{B |- N : C}}
    %   \\\reduces\\
    %   \infer[\lrule{\plus}]{\slof{\plus*[sub=_{\ell \in L}]{\ell:A_{\ell}} |- \caseL[\ell \in L]{\ell => \spawn{M_{\ell}}{N}} : B}}{
    %     \multipremise{\ell \in L}{
    %       \infer-[\jrule{CUT}^B]{\slof{A_{\ell} |- \spawn{M_{\ell}}{N} : C}}{
    %         \slof{A_{\ell} |- M_{\ell} : B} &
    %         \slof{B |- N : C}}}}
    % \end{gather*}

  % \item[Right commutative cut reductions]
  %   The two remaining cases are symmetric to the above left commutative cases.
  %   These right commutative cases are resolved by associativity and a new form of reduction:
  %   \begin{gather*}
  %     \spawn{M}{(\spawn{N_0}{\selectR{\kay}})} \equiv \spawn{(\spawn{M}{N_0})}{\selectR{\kay}} \\
  %     \spawn{M}{\caseR[\ell \in L]{\ell => N_{\ell}}} \reduces \caseR[\ell \in L]{\ell => \spawn{M}{N_{\ell}}}
  %     \mathrlap{\,.}
  %     \qedhere
  %   \end{gather*}
  \end{description}
\end{proof}

% What is odd about this proof is that it seems to suggest that normal terms can be generated by the following grammar.
% \begin{syntax*}
%   & N & \fwd
%     \begin{array}[t]{@{{}\mid{}}l@{}}
%       \spawn{N}{\selectR{\kay}} \mid \caseL[\ell \in L]{\ell => N_{\ell}} \\
%       \caseR[\ell \in L]{\ell => N_{\ell}} \mid \spawn{\selectL{\kay}}{N}
%     \end{array}
% \end{syntax*}
% But not all such terms are irreducible -- $\spawn{\selectL{\kay}}{\caseR[\ell \in L]{\ell => N_{\ell}}} \reduces \caseR[\ell \in L]{\ell => \spawn{\selectL{\kay}}{N_{\ell}}}$, for example.




% \section{Circular proofs}



% \section{Isomorphism with transducers}

% \begin{syntax*}
%   Transducers & T &
%     q \mid \spawnR{T}{\kay} \mid \caseL[\ell \in L]{\ell => T_{\ell}} \mid \selectR{\kay}
% \end{syntax*}
% Transducers are isomorphic to the $\eta$-long circular proofs constructed from the above grammar.
% Such a proof is $\eta$-long if the axiom for $\selectR{\kay}$ is used only at atomic type:
% \begin{equation*}
%   \infer[\rrule{\plus}]{\slof{\alpha |- \selectR{\kay} : \plus*[sub=_{\ell \in L}]{\ell:A_{\ell}}}}{
%     \text{($\kay \in L$)} & \text{($A_{\kay} = \alpha$)}}
% \end{equation*}
% From the perspective of a transducer, this means that the $\rrule{\plus}$ rule may be used only for terminal symbols in the output alphabet.



% \section{Circularity}

% \begin{syntax*}
%   Weak-head normal terms & E &
%     \lambda x.e \mid x\,e_1 \dotsm\mkern1mu e_n
% \end{syntax*}


% \begin{syntax*}
%   Weak-head normal terms & N &
%     \begin{array}[t]{@{}l@{}}
%       \fwd \mid \selectR{\kay} \mid \selectL{\kay} \\
%       \mathllap{\mid {}} \caseL[\ell \in L]{\ell => P_{\ell}} \mid \caseR[\ell \in L]{\ell => P_{\ell}} \\
%       \mathllap{\mid {}} \spawn{P}{\selectR{\kay}} \mid \spawn{\selectL{\kay}}{P}
%     \end{array}
% \end{syntax*}

% \section{Normal-form processes}

% The following is a revised spine-like grammar for normal forms.
% \begin{syntax*}
%   Normal-form processes & N &
%     \fwd
%     \begin{array}[t]{@{{} \mid {}}l@{}}
%       Q^{\with} \mid Q^{\plus}
%       \begin{array}[t]{@{{} \mid {}}l@{}}
%         \caseR[\ell \in L]{\ell => N_{\ell}} \\
%         \caseL[\ell \in L]{\ell => N_{\ell}}
%       \end{array}
%     \\
%     \spawn{Q^{\with}}{Q^{\plus}}
%       \begin{array}[t]{@{{} \mid {}}l@{}}
%         \spawn{\caseR[\ell \in L]{\ell => N_{\ell}}}{Q^{\plus}} \\
%         \spawn{Q^{\with}}{\caseL[\ell \in L]{\ell => N_{\ell}}}
%       \end{array}
%     \end{array}
%   \\
%   Right q{}ueues & Q^{\plus} &
%     \spawn{Q^{\plus}}{\selectR{\kay}} \mid \selectR{\kay}
%   \\
%   Left q{}ueues & Q^{\with} &
%     \spawn{\selectL{\kay}}{Q^{\with}} \mid \selectL{\kay}
% \end{syntax*}
% By reasoning similar to that used earlier, I would expect the following results.
% \begin{theorem}
%   If $\slof{A |- P : C}$, then $P \equiv\reduces P'$ for some $P'$, or $P \equiv\mathrel{\mathsf{normal}}$.
%   Also, if $N \;\mathsf{normal}$, then $N \longarrownot\reduces$.
% \end{theorem}

% Consider the following two transducer processes over the alphabet $\ialph = \{a\}$ that map $a^n \mapsto a^{n+1}$.
% \begin{equation*}
%   \begin{lgathered}
%     \finwds{\ialph} \defd \plus*{a:\finwds{\ialph}, \emp:\varepsilon}
%     \\
%     \slof{\finwds{\ialph} |- q : \finwds{\ialph}}
%       \defd \spawn{\caseL{a => q
%                         | \emp => \selectR{\emp}}}{\selectR{a}}
%     \\
%     \slof{\finwds{\ialph} |- s : \finwds{\ialph}}
%       \defd \caseL{a => \spawn{s}{\selectR{a}}
%                  | \emp => \spawn{\selectR{\emp}}{\selectR{a}}}
%   \end{lgathered}
% \end{equation*}
% The process $q$ can be reduced to $s$ by an infinite sequence of reductions:
% \begin{equation*}
%   \slof{\finwds{\ialph} |- q : \finwds{\ialph}}
%     \defd \begin{aligned}[t]
%             \MoveEqLeft[0.5]
%             \spawn{\caseL{a => q | \emp => \selectR{\emp}}}{\selectR{a}}
%             \\
%               &\reduces \caseL{a => \spawn{q}{\selectR{a}} | \emp => \spawn{\selectR{\emp}}{\selectR{a}}}
%             \\
%               &\reduces^\omega \caseL{a => \spawn{s}{\selectR{a}}
%                                    | \emp => \spawn{\selectR{\emp}}{\selectR{a}}}
%       \defd \slof{\finwds{\ialph} |- s : \finwds{\ialph}}
%   \end{aligned}
% \end{equation*}
% In this sense, $s$ is ``more normal'' than $q$.
% But if we interpret the definition of $s$ equirecursively, then is $s$ really normal?
% For instance, we have the reduction
% \begin{align*}
%   s &= \caseL{a => \spawn{\caseL{a => \spawn{s}{\selectR{a}}
%                                | \emp => \spawn{\selectR{\emp}}{\selectR{a}}}}
%                          {\selectR{a}}
%             | \emp => \spawn{\selectR{\emp}}{\selectR{a}}}
%     \\
%     &\reduces \caseL{a => \caseL{a => \spawn{(\spawn{s}{\selectR{a}})}{\selectR{a}}
%                                | \emp => \spawn{(\spawn{\selectR{\emp}}{\selectR{a}})}{\selectR{a}}}
%             | \emp => \spawn{\selectR{\emp}}{\selectR{a}}}
% \end{align*}

% Notice that the left commutative cut reduction
% \begin{gather*}
%   \infer[\jrule{CUT}]{\slof{\plus*[sub=_{\ell \in L}]{\ell:A_{\ell}} |- \spawn{\caseL[\ell \in L]{\ell => P_{\ell}}}{\selectR{\kay}} : \plus*[sub=_{i \in I}]{i:C_i}}}{
%     \infer[\lrule{\plus}]{\slof{\plus*[sub=_{\ell \in L}]{\ell:A_{\ell}} |- \caseL[\ell \in L]{\ell => P_{\ell}} : C_{\kay}}}{
%       \multipremise{\ell \in L}{\slof{A_{\ell} |- P_{\ell} : C_{\kay}}}} &
%     \infer[\rrule{\plus}]{\slof{C_k |- \selectR{\kay} : \plus*[sub=_{i \in I}]{i:C_i}}}{
%       \text{($\kay \in I$)}}}
%   \\\reduces\\
%   \infer[\lrule{\plus}]{\slof{\plus*[sub=_{\ell \in L}]{\ell:A_{\ell}} |- \caseL[\ell \in L]{\ell => \spawn{P_{\ell}}{\selectR{\kay}}} : \plus*[sub=_{i \in I}]{i:C_i}}}{
%     \multipremise{\ell \in L}{
%       \infer[\jrule{CUT}]{\slof{A_{\ell} |- \spawn{P_{\ell}}{\selectR{\kay}} : \plus*[sub=_{i \in I}]{i:C_i}}}{
%         \slof{A_{\ell} |- P_{\ell} : C_{\kay}} &
%         \infer[\rrule{\plus}]{\slof{C_{\kay} |- \selectR{\kay} : \plus*[sub=_{i \in I}]{i:C_i}}}{
%         \text{($\kay \in I$)}}}}}
% \end{gather*}
% is effectively the commuting conversion between the $\rrule{\plus}$ and $\lrule{\plus}$ rules:
% \begin{gather*}
%   \infer[\rrule{\plus}]{\slof{\plus*[sub=_{\ell \in L}]{\ell:A_{\ell}} |- \selectR{\kay}[{\caseL[\ell \in L]{\ell => P_{\ell}}}] : \plus*[sub=_{i \in I}]{i:C_i}}}{
%     \text{($\kay \in I$)} &
%     \infer[\lrule{\plus}]{\slof{\plus*[sub=_{\ell \in L}]{\ell:A_{\ell}} |- \caseL[\ell \in L]{\ell => P_{\ell}} : C_{\kay}}}{
%       \multipremise{\ell \in L}{\slof{A_{\ell} |- P_{\ell} : C_{\kay}}}}}
%   \\\equiv\\
%   \infer[\lrule{\plus}]{\slof{\plus*[sub=_{\ell \in L}]{\ell:A_{\ell}} |- \caseL[\ell \in L]{\ell => \selectR{\kay}[P_{\ell}]} : \plus*[sub=_{i \in I}]{i:C_i}}}{
%     \multipremise{\ell \in L}{
%       \infer[\rrule{\plus}]{\slof{A_{\ell} |- \selectR{\kay}[P_{\ell}] : \plus*[sub=_{i \in I}]{i:C_i}}}{
%         \text{($\kay \in I$)} &
%         \slof{A_{\ell} |- P_{\ell} : C_{\kay}}}}}
% \end{gather*}
% So, should that left commutative cut reduction even be considered a reduction?

% \section{}

% In the APLAS isomorphism, proofs were $\eta$-long so that the $\jrule{ID}$ rule did not appear.
% With the revised $\rrule{\plus}$ rule, there is a similar problem.
% Consider the process
% \begin{equation*}
%   \infer[\rrule{\plus}]{\slof{\finwds{\ialph} |- \selectR{a} : \finwds{\ialph}}}{}
% \end{equation*}
% where $\finwds{\ialph} \defd \plus*[sub=_{a \in \ialph}]{a:\finwds{\ialph}, \emp:\varepsilon}$.
% This process is certainly irreducible and $\eta$-long, but there is no transducer that directly matches the process.
% % \begin{equation*}
% %   \begin{lgathered}
% %     \finwds{\ialph} \defd \plus*[sub=_{a \in \ialph}]{a:\finwds{\ialph}, \emp:\varepsilon}
% %     \\
% %     \slof{\finwds{\ialph} |- n : \finwds{\ialph}}
% %       \defd \fwd
% %     \\
% %     \slof{\finwds{\ialph} |- n : \finwds{\ialph}}
% %       \defd Q^{\plus}
% %     \\
% %     \slof{\finwds{\ialph} |- n : \finwds{\ialph}}
% %       \defd \caseL[a \in \ialph]{a => n'_a | \emp => Q^{\plus}}
% %   \end{lgathered}
% % \end{equation*}
% % This isn't quite right for an isomorphism.
% % We shouldn't have $\fwd$, but that is easy to remove by requiring $\eta$-long form.
% % The main problem is having $Q^{\plus}$ instead of $\spawn{n'}{Q^{\plus}}$.
% % If we had the usual right rule for $\plus$, then $\eta$-long form would rule out such a normal form.
% % But

% The problem is that there is an implicit forward in the ``asynchronous'' right rule:
% \begin{equation*}
%   \infer{\slof{A_{\kay} |- \selectR{\kay} : \plus*[sub=_{\ell \in L}]{\ell:A_{\ell}}}}{
%     \text{($\kay \in L$)}}
%   \qquad\raisebox{0.75\baselineskip}{$\leftrightsquigarrow$}\qquad
%   \infer{\slof{A_{\kay} |- \selectR{\kay}[\fwd] : \plus*[sub=_{\ell \in L}]{\ell:A_{\ell}}}}{
%     \text{($\kay \in L$)} &
%     \infer{\slof{A_{\kay} |- \fwd : A_{\kay}}}{}}
% \end{equation*}
% Simply demanding that $A_{\kay}$ always be atomic is much too strong of a requirement.
% What about giving the types a (more prominent) role in normal forms?
% \begin{inferences}
%   \infer{\slof{\alpha |- \fwd \Downarrow \alpha}}{}
%   \and
%   \infer{\slof{\alpha |- \selectR{\kay} \Downarrow \plus*[sub=_{\ell \in L}]{\ell:A_{\ell}}}}{
%     \text{($\kay \in L$)} & \text{($A_{\kay} = \alpha$)}}
%   \and
%   \infer{\slof{A |- \spawn{P}{\selectR{\kay}} \Downarrow \plus*[sub=_{\ell \in L}]{\ell:B_{\ell}}}}{
%     \text{($\kay \in L$)} & \slof{A |- P : B_{\kay}}}
% \end{inferences}
% But isn't this just $\selectR{\kay}[P]$ in another guise?


\section{}

Terms of the form
\begin{equation*}
  \spawn{\caseL[\ell \in L]{\ell => M_{\ell}}}{\caseR[\kay \in K]{\kay => N_{\kay}}}
\end{equation*}
should not be considered normal because the commutative reductions would enable potential principal reductions between $M_{\ell}$ and $N_{\kay}$.
Consider, for example, the following instance in which the commutative reductions enable a principal reduction.
\begin{align*}
  \MoveEqLeft[0.5]
  \spawn{\caseL[\ell \in L]{\ell => \selectR{\ell}}}{\caseR[\kay \in K]{\kay => \caseL[\ell' \in L]{\ell' => \selectR{\kay}}}} \\
    &\Reduces \caseL[\ell \in L]{\ell => \caseR[\kay \in K]{\kay => \spawn{\selectR{\ell}}{\caseL[\ell' \in L]{\ell' => \selectR{\kay}}}}} \\
    &\reduces \caseL[\ell \in L]{\ell => \caseR[\kay \in K]{\kay => \selectR{\kay}}}
\end{align*}

% \begin{syntax*}
%   Normal terms & N &
%     \spawn{Q^{\with}}{\spawn{S}{Q^{\plus}}}
%   \\
%   Synchronous & S &
%     \fwd
%       \begin{array}[t]{@{{} \mid {}}l@{}}
%         \caseL[\ell \in L]{\ell => N_{\ell}} \\
%         \caseR[\ell \in L]{\ell => N_{\ell}}
%       \end{array}
%   \\
%   Q{ueues} & Q^{\plus} &
%     \fwd \mid \spawn{Q^{\plus}_1}{Q^{\plus}_2} \mid \selectR{\kay}
%   \\
%            & Q^{\with} &
%     \fwd \mid \spawn{Q^{\with}_1}{Q^{\with}_2} \mid \selectL{\kay}
% \end{syntax*}

We will work modulo associativity and unit laws:
\begin{gather*}
  \spawn{P_1}{(\spawn{P_2}{P_3})} \equiv \spawn{(\spawn{P_1}{P_2})}{P_3}
  \\
  \spawn{P}{\fwd} \equiv P \equiv \spawn{\fwd}{P}
\end{gather*}

Let the normal terms be those generated by the following grammar, modulo the above associativity and unit laws.
\begin{syntax*}
  Normal terms & N &
    \begin{array}[t]{@{}l@{}}
      \spawn{Q^{\with}}{N} \mid \spawn{N}{Q^{\plus}} \mid \fwd \\
        \begin{array}[t]{@{\mathllap{\mid {}}}l@{}}
          \caseL[\ell \in L]{\ell => N_{\ell}} \\
          \caseR[\ell \in L]{\ell => N_{\ell}}
        \end{array}
    \end{array}
  \\
  $\plus$-Q{ueues} & Q^{\plus} &
    \fwd \mid \spawn{Q^{\plus}_1}{Q^{\plus}_2} \mid \selectR{\kay}
  \\
  $\with$-Q{ueues} & Q^{\with} &
    \fwd \mid \spawn{Q^{\with}_1}{Q^{\with}_2} \mid \selectL{\kay}
\end{syntax*}

We will also use $Q$ as a metavariable for a queue that may be either a $\plus$- or $\with$-queue.
Queues themselves may be put into normal form.
\begin{lemma}
  For all $\plus$-queues $Q^{\plus}$, either:
  \begin{enumerate*}[label=\emph{(\roman*)}]
  \item $Q^{\plus} \equiv \fwd$; or
  \item $Q^{\plus} \equiv \spawn{Q^{\plus}_0}{\selectR{\kay}}$ for some $\plus$-queue $Q^{\plus}_0$.
  \end{enumerate*}
  Similarly, for all $\with$-queues $Q^{\with}$, either:
  \begin{enumerate*}[label=\emph{(\roman*)}]
  \item $Q^{\with} \equiv \fwd$; or
  \item $Q^{\with} \equiv \spawn{\selectL{\kay}}{Q^{\with}_0}$ for some $\with$-queue $Q^{\with}_0$.
  \end{enumerate*}
\end{lemma}

% We may define the following measure for queues.
% %
% \begin{align*}
%   \qmeas{\fwd} &= 0 \\
%   \qmeas{\selectL{\kay}} = \qmeas{\selectR{\kay}} &= 1 \\
%   % \qmeas{\spawn{Q^{\with}_1}{Q^{\with}_2}} &= \qmeas{Q^{\with}_1} + \qmeas{Q^{\with}_2} \\
%   % \qmeas{\spawn{Q^{\plus}_1}{Q^{\plus}_2}} &= \qmeas{Q^{\plus}_1} + \qmeas{Q^{\plus}_2}
%   \qmeas{\spawn{Q_1}{Q_2}} &= \qmeas{Q_1} + \qmeas{Q_2}
% \end{align*}
% This measure respects the associativity and unit laws:
% \begin{proposition}
%   If $Q \equiv Q'$, then $\qmeas{Q} = \qmeas{Q'}$.
% \end{proposition}
% %
% \begin{proof}[Proof sketch]
%   Notice that
%   \begin{gather*}
%     \qmeas{\spawn{Q}{\fwd}} = \qmeas{Q} + 0 = \qmeas{Q} = 0 + \qmeas{Q} = \qmeas{\spawn{\fwd}{Q}}
%   \shortintertext{and}
%     \begin{aligned}[b]
%       \qmeas{\spawn{Q_1}{(\spawn{Q_2}{Q_3})}}
%         &= \qmeas{Q_1} + (\qmeas{Q_2} + \qmeas{Q_3}) \\
%         &= (\qmeas{Q_1} + \qmeas{Q_2}) + \qmeas{Q_3}
%          = \qmeas{\spawn{(\spawn{Q_1}{Q_2})}{Q_3}}
%       \mathrlap{\,.}
%     \end{aligned}
%     \qedhere
%   \end{gather*}
% \end{proof}

We will adopt the following as reductions.
\begin{gather*}
  \spawn{\caseR[\ell \in L]{\ell => M_{\ell}}}{\selectL{\kay}}
    \reduces M_{\kay}
  \\
  \spawn{\selectR{\kay}}{\caseL[\ell \in L]{\ell => N_{\ell}}}
    \reduces N_{\kay}
  \\
  \spawn{M}{\caseR[\ell \in L]{\ell => N_{\ell}}}
    \reduces \caseR[\ell \in L]{\ell => \spawn{M}{N_{\ell}}}
    \mathrlap{\qquad\text{($M \neq \selectL{\kay}$ and $M \neq \spawn{P_1}{P_2}$ and $M \neq \fwd$)}}
  \\
  \spawn{\caseL[\ell \in L]{\ell => M_{\ell}}}{N}
    \reduces \caseL[\ell \in L]{\ell => \spawn{M_{\ell}}{N}}
    \mathrlap{\qquad\text{($N \neq \selectR{\kay}$ and $N \neq \spawn{P_1}{P_2}$ and $N \neq \fwd$)}}
\end{gather*}
These reductions arise from a proof of weak normalization.
Before presenting that proof, we need to define a termination measure.
%
\DeclarePairedDelimiter{\qmeas}{\lvert}{\rvert}
\DeclarePairedDelimiter{\meas}{\lvert}{\rvert}
%
\begin{align*}
  \meas{\fwd} &= 1 \\
  \meas{\spawn{Q^{\with}}{N}} &= 1 + \qmeas{Q^{\with}} + \meas{N} \\
  \meas{\spawn{N}{Q^{\plus}}} &= 1 + \meas{N} + \qmeas{Q^{\plus}} \\
  \meas{\caseL[\ell \in L]{\ell => N_{\ell}}} &= 1 + {\textstyle \sum}_{\ell \in L}{\meas{N_{\ell}}} \\
  \meas{\caseR[\ell \in L]{\ell => N_{\ell}}} &= 1 + {\textstyle \sum}_{\ell \in L}{\meas{N_{\ell}}}
\end{align*}


\begin{theorem}[Weak normalization]
  If $\slof{A |- M : B}$ and $\slof{B |- N : C}$ are normal terms, then there exists a normal term $\slof{A |- N' : C}$.
\end{theorem}
%
\begin{proof}
  By lexicographic induction, first on the structure of the principal type, and next on the measure of the composition, $\meas{\spawn{M}{N}}$.
  The cases are organized in a manner similar to that of the cut elimination proof for the standard sequent calculus.
  \begin{description}
  \item[Identity cut eliminations]
    Several cases act like identity cut eliminations:
    \begin{itemize}
    \item If either $M = \fwd$ or $N = \fwd$, then $\spawn{M}{N}$ is already normal modulo the unit laws:
      \begin{gather*}
        \spawn{\fwd}{N} \equiv N
        \\
        \spawn{M}{\fwd} \equiv M
      \end{gather*}

    \item If $M = \spawn{M_0}{\fwd}$ or $N = \spawn{\fwd}{N_0}$, then 
      \begin{gather*}
        \spawn{(\spawn{M_0}{\fwd})}{N}
          \equiv \spawn{M_0}{N}
        \\
        \spawn{M}{(\spawn{\fwd}{N_0})}
          \equiv \spawn{M}{N_0}
      \end{gather*}
      The principal type stays the same and the measure strictly decreases:
      \begin{align*}
        \MoveEqLeft[0.5]
        \meas{\spawn{(\spawn{M_0}{\fwd})}{N}} \\
          &= 3 + \meas{M_0} + \meas{N} \\
          &> 1 + \meas{M_0} + \meas{N} \\
          &= \meas{\spawn{M_0}{N}}
      \end{align*}
      and similarly for the case in which $N = \spawn{\fwd}{N_0}$.
    \end{itemize}

  \item[Principal cut reductions]
    Two cases act like principal cut reductions:
    \begin{itemize}
    \item If $M = \caseR[\ell \in L]{\ell => M_{\ell}}$ and $N = \spawn{(\spawn{\selectL{\kay}}{Q^{\with}_0})}{N_0}$, then using associativity and a principal cut reduction, we arrive at:
      \begin{align*}
        \MoveEqLeft[0.5]
        \spawn{\caseR[\ell \in L]{\ell => M_{\ell}}}{(\spawn{(\spawn{\selectL{\kay}}{Q^{\with}_0})}{N_0})} \\
          &\equiv \spawn{(\spawn{\caseR[\ell \in L]{\ell => M_{\ell}}}{\selectL{\kay}})}{(\spawn{Q^{\with}_0}{N_0})} \\
          &\reduces \spawn{M_{\kay}}{(\spawn{Q^{\with}_0}{N_0})}
      \end{align*}
      The outermost, non-analytic cut occurs at a strictly smaller type.

    \item If $M = \spawn{M_0}{(\spawn{Q^{\plus}_0}{\selectR{\kay}})}$ and $N = \caseL[\ell \in L]{\ell => N_{\ell}}$, then using associativity and a principal cut reduction, we arrive at:
      \begin{align*}
        \MoveEqLeft[0.5]
        \spawn{(\spawn{M_0}{(\spawn{Q^{\plus}_0}{\selectR{\kay}})})}{\caseL[\ell \in L]{\ell => N_{\ell}}} \\
          &\equiv \spawn{(\spawn{M_0}{Q^{\plus}_0})}{(\spawn{\selectR{\kay}}{\caseL[\ell \in L]{\ell => N_{\ell}}})} \\
          &\reduces \spawn{(\spawn{M_0}{Q^{\plus}_0})}{N_{\kay}}      
      \end{align*}
      Once again, the outermost, non-analytic cut occurs at a strictly smaller type.
    \end{itemize}

  \item[Left commutative cut reductions]
    Several cases act like left commutative cut reductions.
    \begin{itemize}
    \item If $M = \spawn{Q^{\with}}{M_0}$, then using associativity, we arrive at:
      \begin{equation*}
        \spawn{(\spawn{Q^{\with}}{M_0})}{N}
          \equiv \spawn{Q^{\with}}{(\spawn{M_0}{N})}      
      \end{equation*}
      In the innermost, non-analytic cut, the principal type remains the same and the measure strictly decreases:
      \begin{align*}
        \MoveEqLeft[0.5]
        \meas{\spawn{(\spawn{Q^{\with}}{M_0})}{N}} \\
          &= 2 + \qmeas{Q^{\with}} + \meas{M_0} + \meas{N} \\
          &> 1 + \meas{M_0} + \meas{N} \\
          &= \spawn{M_0}{N}
        \,.
      \end{align*}

    \item If $M = \caseL[\ell \in L]{\ell => M_{\ell}}$ and $N = \spawn{(\spawn{\selectL{\kay}}{Q^{\with}_0})}{N_0}$, then using associativity and a left commutative cut reduction, we arrive at:
      \begin{align*}
        \MoveEqLeft[0.5]
        \spawn{\caseL[\ell \in L]{\ell => M_{\ell}}}{(\spawn{(\spawn{\selectL{\kay}}{Q^{\with}_0})}{N_0})} \\
          &\equiv \spawn{(\spawn{\caseL[\ell \in L]{\ell => M_{\ell}}}{\selectL{\kay}})}{(\spawn{Q^{\with}_0}{N_0})} \\
          &\reduces \spawn{\caseL[\ell \in L]{\ell => \spawn{M_{\ell}}{\selectL{\kay}}}}{(\spawn{Q^{\with}_0}{N_0})}
      \end{align*}
      The outermost, non-analytic cut occurs at a strictly smaller type.
      The innermost, non-analytic cuts occur at the same type, but the measure strictly decreases:
      \begin{align*}
        \MoveEqLeft[0.5]
        \meas{\spawn{\caseL[\ell \in L]{\ell => M_{\ell}}}{(\spawn{Q^{\with}}{N_0})}} \\
          &= 3 + \bigl({\textstyle \sum_{\ell \in L}{\meas{M_{\ell}}}}\bigr) + \qmeas{Q^{\with}} + \meas{N_0} \\
          &> 2 + \meas{M_{\ell}} \\
          &=\meas{\spawn{M_{\ell}}{\selectL{\kay}}}
        \mathrlap{\,.}
      \end{align*}

    \item If $M = \caseL[\ell \in L]{\ell => M_{\ell}}$ and $N = \caseL[\kay \in K]{\kay => N_{\kay}}$, then using a left commutative cut reduction, we arrive at:
      \begin{align*}
        \MoveEqLeft[0.5]
        \spawn{\caseL[\ell \in L]{\ell => M_{\ell}}}
              {\caseL[\kay \in K]{\kay => N_{\kay}}} \\
          &\reduces \caseL[\ell \in L]{\ell => \spawn{M_{\ell}}{\caseL[\kay \in K]{\kay => N_{\kay}}}}
      \end{align*}
      The innermost, non-analytic cuts occur at the same type, but the measure strictly decreases:
      \begin{align*}
        \MoveEqLeft[0.5]
        \meas{\spawn{\caseL[\ell \in L]{\ell => M_{\ell}}}
                    {\caseL[\kay \in K]{\kay => N_{\kay}}}} \\
          &= 2 + \bigl({\textstyle \sum_{\ell \in L} \meas{M_{\ell}}}\bigr) + \meas{\caseL[\kay \in K]{\kay => N_{\kay}}} \\
          &> 1 + \meas{M_{\ell}} + \meas{\caseL[\kay \in K]{\kay => N_{\kay}}} \\
          &= \meas{\spawn{M_{\ell}}{\caseL[\kay \in K]{\kay => N_{\kay}}}}
        \,.
      \end{align*}

    \item The case in which $M = \caseL[\ell \in L]{\ell => M_{\ell}}$ and $N = \caseR[\kay \in K]{\kay => N_{\kay}}$ is similar to the above case:
      \begin{align*}
        \MoveEqLeft[0.5]
        \spawn{\caseL[\ell \in L]{\ell => M_{\ell}}}
              {\caseR[\kay \in K]{\kay => N_{\kay}}} \\
          &\reduces \caseL[\ell \in L]{\ell => \spawn{M_{\ell}}{\caseR[\kay \in K]{\kay => N_{\kay}}}}        
      \end{align*}
    \end{itemize}

  \item[Right commutative cut reductions]
    Symmetric to the above left commutative reductions are the following right commutative cut reductions.
    \begin{gather*}
      \spawn{M}{(\spawn{N_0}{Q^{\plus}})}
        \equiv \spawn{(\spawn{M}{N_0})}{Q^{\plus}}
      \\[2\jot]
      \begin{aligned}
        \MoveEqLeft[0.5]
        \spawn{(\spawn{M_0}{(\spawn{Q^{\plus}_0}{\selectR{\kay}})})}{\caseR[\ell \in L]{\ell => N_{\ell}}} \\
          &\equiv \spawn{(\spawn{M_0}{Q^{\plus}_0})}{(\spawn{\selectR{\kay}}{\caseR[\ell \in L]{\ell => N_{\ell}}})} \\
          &\reduces \spawn{(\spawn{M_0}{Q^{\plus}_0})}{\caseR[\ell \in L]{\ell => \spawn{\selectR{\kay}}{N_{\ell}}}}
      \end{aligned}
      \\[2\jot]
      \begin{aligned}
        \MoveEqLeft[0.5]
        \spawn{\caseR[\kay \in K]{\kay => M_{\kay}}}
              {\caseR[\ell \in L]{\ell => N_{\ell}}} \\
          &\reduces \caseR[\ell \in L]{\ell => \spawn{\caseR[\kay \in K]{\kay => M_{\kay}}}{N_{\ell}}}
      \end{aligned}
      \\[2\jot]
      \begin{aligned}
        \MoveEqLeft[0.5]
        \spawn{\caseL[\kay \in K]{\kay => M_{\kay}}}
              {\caseR[\ell \in L]{\ell => N_{\ell}}} \\
          &\reduces \caseR[\ell \in L]{\ell => \spawn{\caseL[\kay \in K]{\kay => M_{\kay}}}{N_{\ell}}}
      \end{aligned}
    \end{gather*}
  \end{description}
\end{proof}


Consider $\spawn{(\selectL{\kay}[P_0])}{\caseR[\ell \in L]{\ell => Q_{\ell}}}$.
We don't say that this process is in normal form, even when $P_0$ cannot possibly interact with $Q_{\ell}$.
We always allow
\begin{equation*}
  \spawn{(\selectL{\kay}[P_0])}{\caseR[\ell \in L]{\ell => Q_{\ell}}} \reduces \caseR[\ell \in L]{\ell => \spawn{(\selectL{\kay}[P_0])}{Q_{\ell}}}
  \,.
\end{equation*}
Is it that the identity cut reductions are considered interactions?

% \begin{align*}
%   \MoveEqLeft[0.5]
%   \spawn{\caseR[\ell \in L]{\ell => M_{\ell}}}{(\spawn{\fwd}{N})} \\
%     &\equiv \spawn{\caseR[\ell \in L]{\ell => M_{\ell}}}{N}
% \end{align*}


% \begin{align*}
%   \MoveEqLeft[0.5]
%   \spawn{(\spawn{Q^{\with}_1}{\spawn{\fwd}{\fwd}})}
%         {(\spawn{Q^{\with}_2}{\spawn{S_2}{Q^{\plus}_2}})} \\
%     &\equiv \spawn{(\spawn{Q^{\with}_1}{Q^{\with}_2})}
%                   {\spawn{S_2}{Q^{\plus}_2}}
% \end{align*}

% \begin{align*}
%   \MoveEqLeft[0.5]
%   \spawn{(\spawn{Q^{\with}_1}{\spawn{S_1}{(\spawn{Q^{\plus}_0}{\selectR{\kay}})}})}
%         {(\spawn{\fwd}{\spawn{\caseL[\ell \in L]{\ell => N_{\ell}}}{Q^{\plus}_2}})} \\
%     &\hphantom{\reduces {}} \mathllap{\equiv}\;
%        \spawn{(\spawn{Q^{\with}_1}{\spawn{S_1}{Q^{\plus}_0}})}
%              {(\spawn{(\spawn{\selectR{\kay}}{\caseL[\ell \in L]{\ell => N_{\ell}}})}{Q^{\plus}_2})} \\
%     &\reduces \spawn{(\spawn{Q^{\with}_1}{\spawn{S_1}{Q^{\plus}_0}})}
%                     {(\spawn{N_{\kay}}{Q^{\plus}_2})}
% \end{align*}

% \begin{align*}
%   \MoveEqLeft[0.5]
%   \spawn{(\spawn{Q^{\with}_1}{(\spawn{Q^{\plus}_0}{\selectR{\kay}})})}
%         {(\spawn{\fwd}{\caseR[\ell \in L]{\ell => N_{\ell}}})} \\
%     &\hphantom{\reduces {}} \mathllap{\equiv}\;
%        \spawn{(\spawn{Q^{\with}_1}{Q^{\plus}_0})}
%              {(\spawn{\selectR{\kay}}{\caseR[\ell \in L]{\ell => N_{\ell}}})} \\
%     &\reduces \spawn{(\spawn{Q^{\with}_1}{Q^{\plus}_0})}
%                     {\caseR[\ell \in L]{\ell => \spawn{\selectR{\kay}}{N_{\ell}}}}
% \end{align*}

% \begin{align*}
%   \MoveEqLeft[0.5]
%   \spawn{(\spawn{Q^{\with}_1}{\caseR[\ell \in L]{\ell => M_{\ell}}})}
%         {(\spawn{\fwd}{Q^{\plus}})} \\
%     &\equiv \spawn{(\spawn{Q^{\with}_1}{Q^{\with}_2})}
%                   {\caseR[\ell \in L]{\ell => N_{\ell}}}
% \end{align*}


\section{Circular proofs}

\subsection{Standard sequent calculus}

\begin{syntax*}
  Normal terms & N &
    p \mid \fwd
    \begin{array}[t]{@{{} \mid {}}l@{}}
      \selectR{\kay}[N] \mid \caseL[\ell \in L]{\ell => N_{\ell}} \\
      \caseR[\ell \in L]{\ell => N_{\ell}} \mid \selectL{\kay}[N]
    \end{array}
\end{syntax*}

Suppose that we wish to normalize $\spawn{N_1}{N_2}$ for some normal terms $N_1$ and $N_2$.
Assume that all definitions are normal.
% 
% For each pair of definitions $p_i \defd N_i$ and $p_j \defd N_j$, create a definition
% \begin{equation*}
%   \spawn{p_i}{p_j} \defd \spawn{N_i}{N_j}
% \end{equation*}
% and normalize the body.
% 
The potentially troublesome cases are:
\begin{equation*}
  \begin{lgathered}
    \spawn{p_1}{p_2} \:\text{,}\:
    \begin{array}[t]{@{}l@{}}
      \spawn{p}{\caseL[\ell \in L]{\ell => N_{\ell}}} \:\text{,}\:
      \spawn{p}{(\selectL{\kay}[N])} \:\text{,}\: \\
      \spawn{\caseR[\ell \in L]{\ell => N_{\ell}}}{p} \:\text{, and}\;
      \spawn{(\selectR{\kay}[N])}{p}
      \:.
    \end{array}
  \end{lgathered}
\end{equation*}
When one of these cases is encountered, create a new definition $p' \defd N'$, where $N'$ is the result of normalizing the cut after expanding the remaining process variable(s).
For example, if $p \defd N$ and $\spawn{p}{\caseL[\ell \in L]{\ell => N_{\ell}}}$ is encountered, then normalize
\begin{equation*}
  p' \defd \spawn{N}{\caseL[\ell \in L]{\ell => N_{\ell}}}
\end{equation*}
Because there are finitely many definitions and finitely many subterms of a definition's body, this process must eventually terminate.

As an example, consider the transducer that compresses runs of $b$s:
\begin{equation*}
  \begin{lgathered}
    q_0 \defd \caseL{a => \selectR{a}[q_0]
                   | b => \selectR{b}[q_1]}
    \\
    q_1 \defd \caseL{a => \selectR{a}[q_0]
                   | b => q_1}
  \end{lgathered}
\end{equation*}
We would like to give process definitions for the composition of two such transducers.
Following the above procedure, we arrive at the definitions
\begin{equation*}
  \begin{lgathered}
    \spawn{q_0}{q_0} \defd \caseL{a => \selectR{a}[(\spawn{q_0}{q_0})]
                                | b => \selectR{b}[(\spawn{q_1}{q_1})]}
    \\
    \spawn{q_1}{q_1} \defd \caseL{a => \selectR{a}[(\spawn{q_0}{q_0})]
                                | b => \spawn{q_1}{q_1}}
  \end{lgathered}
\end{equation*}
These definitions are equivalent to $q_0$ and $q_1$, so the transducer is idempotent.

\subsection{Asynchronous calculus}

\begin{syntax*}
  Normal terms & N &
    \begin{array}[t]{@{}l@{}}
      p \mid \spawn{Q^{\with}}{N} \mid \spawn{N}{Q^{\plus}} \mid \fwd \\
        \begin{array}[t]{@{\mathllap{\mid {}}}l@{}}
          \caseL[\ell \in L]{\ell => N_{\ell}} \\
          \caseR[\ell \in L]{\ell => N_{\ell}}
        \end{array}
    \end{array}
  \\
  Q{ueues} & Q^{\plus} &
    \fwd \mid \spawn{Q^{\plus}_1}{Q^{\plus}_2} \mid \selectR{\kay}
  \\
           & Q^{\with} &
    \fwd \mid \spawn{Q^{\with}_1}{Q^{\with}_2} \mid \selectL{\kay}
\end{syntax*}

\begin{equation*}
  \begin{lgathered}
    \spawn{p_1}{p_2} \\
    \spawn{p}{(\spawn{(\spawn{\selectL{\kay}}{Q^{\with}})}{N})} \\
    \spawn{p}{\caseL[\ell \in L]{\ell => N_{\ell}}} \\
    \spawn{p}{\caseR[\ell \in L]{\ell => N_{\ell}}} \\
    \spawn{(\spawn{N}{(\spawn{Q^{\with}}{\selectR{\kay}})})}{p} \\
    \spawn{\caseR[\ell \in L]{\ell => N_{\ell}}}{p} \\
    \spawn{\caseL[\ell \in L]{\ell => N_{\ell}}}{p}
  \end{lgathered}
\end{equation*}


\section{$\eta$-long form}

To require terms to be $\eta$-long, we would restrict the identity rule to propositional atoms:
\begin{equation*}
  \infer[\jrule{ID}]{\slof{\alpha |- \fwd : \alpha}}{}
\end{equation*}
This restriction interacts subtly with the characterization of normal forms.

Consider $\slof{A_{\kay} |- \selectR{\kay} : \plus*[sub=_{\ell \in L}]{\ell:A_{\ell}}}$.
Without the $\eta$-long restriction in place, this term is normal because it is equivalent to one of the specified syntatic forms: $\selectR{\kay} \equiv \spawn{\fwd}{\selectR{\kay}}$.
With the $\eta$-long restriction in place, $\slof{A_{\kay} |- \selectR{\kay} : \plus*[sub=_{\ell \in L}]{\ell:A_{\ell}}}$ is typable, but, in general, $\slof{A_{\kay} \not|- \spawn{\fwd}{\selectR{\kay}} : \plus*[sub=_{\ell \in L}]{\ell:A_{\ell}}}$.
In other words, $\selectR{\kay}$ is no longer normal (in general), and the unit laws now hold only at certain types:
\begin{equation*}
  \begin{lgathered}
    \spawn{\fwd}{P} \equiv P \quad\text{if $\slof{\alpha |- P : A}$} \\
    \spawn{P}{\fwd} \equiv P \quad\text{if $\slof{A |- P : \alpha}$}
  \end{lgathered}
\end{equation*}
This seems reasonable, right?

It would seem to be difficult to prevent the typing $\slof{A_{\kay} |- \selectR{\kay} : \plus*[sub=_{\ell \in L}]{\ell:A_{\ell}}}$.
If several messages are sent, the first ones use a compound type for their continuation, so to prevent the typing $\slof{A_{\kay} |- \selectR{\kay} : \plus*[sub=_{\ell \in L}]{\ell:A_{\ell}}}$, it looks like we would need to involve the normal forms in the typing rules:
\begin{inferences}
  \infer[\rrule{\plus}]{\slof{\alpha |- \selectR{\kay} : \plus*[sub=_{\ell \in L}]{\ell:A_{\ell}}}}{
    \text{($\kay \in L$)} & \text{($A_{\kay} = \alpha$)}}
  \and
  \infer[\jrule{$\plus$-CUT}]{\slof{A |- \spawn{P}{\selectR{\kay}} : \plus*[sub=_{\ell \in L}]{\ell:C_{\ell}}}}{
    \text{($\kay \in L$)} &
    \slof{A |- P : C_{\kay}}}
\end{inferences}

%%% Local Variables:
%%% mode: latex
%%% TeX-master: "thesis"
%%% End:

\chapter{Conclusion}\label{ch:conclusion}

In this document, we have explored two proof-theoretic characterizations of concurrency -- proof construction and proof reduction -- in the context of concurrent systems that have chain topologies.

On the proof-construction side, we have shown a new way of systematically deriving a rewriting framework from the ordered sequent calculus~\parencref{ch:ordered-rewriting}, and identified a message-passing fragment of that framework~\parencref{ch:choreographies}.
We have shown how to take string rewriting specifications of concurrent systems~\parencref{ch:string-rewriting} and choreograph them into message-passing ordered rewriting~\parencref{ch:choreographies}.

On the proof-reduction side, we have uncovered a semi-axiomatic sequent calculus for singleton logic, a logic that restricts sequents to have exactly one antecedent~\parencref{ch:singleton-logic}.
We have demonstrated that the semi-axiomatic nature of this calculus gives rise to a clean correspondence between proof normalization and asynchronous message-passing communication~\parencref{ch:process-chains}.

Lastly, we have shown that the asynchronous processes that arise from the semi-axiomatic sequent calculus for singleton logic can be faithfully embedded within the message-passing ordered rewriting framework~\parencref{ch:correspond}.
This has provided a relationship between the proof-construction and proof-reduction characterizations of concurrency.

This document now closes by discussing a few avenues for future work.

\section{Potential avenues for future work}\label{sec:conclusion:nondeterminism}

\subsection{From ordered rewriting to multiset rewriting, singleton processes to linear processes}\label{sec:conclusion:generalize}

One obvious avenue for future work is to extend the ideas in this document to linear logic.
We conjecture that the string rewriting specifications of \cref{ch:string-rewriting} would be replaced with multiset rewriting specifications\autocite{Meseguer:TCS92}; the formula-as-process ordered rewriting of \cref{ch:formula-as-process}, with formula-as-process linear rewriting based on the focused linear sequent calculus\autocites{Miller:ELP92}{Cervesato+Scedrov:IC09}; and the singleton processes of \cref{ch:process-chains}, with \acs{SILL} processes\autocite{Caires+:MSCS16}.

% For the proof construction and rewriting perspective on concurrency, this would involve a shift from ordered rewriting to multiset rewriting, as understood logically in terms of the focused linear sequent calculus.

For instance, in the formula-as-process ordered rewriting choreography of the binary counter~\parencref{sec:formula-as-process:counters-oo}, we used the coinductively defined proposition $\defp{b}_0$ given by 
\begin{equation*}
  \defp{b}_0 \defd (\up \dn \defp{b}_1 \pmir \atmL{i}) \with (\up (\atmL{d} \fuse \dn \defp{b}'_0) \pmir \atmL{d})
  \,,
\end{equation*}
where here all of the shifts have been made explicit.
With a move to rewriting based on linear logic, rather than ordered logic, we would likely use the coinductively defined proposition $\defp{b}_0(x, y)$ given by
\begin{equation*}
  \defp{b}_0(x, y) \defd
    \begin{lgathered}[t]
      \bigl(\forall y'.\, i(y, y') \lolli \up \dn \defp{b}_1(x, y')\bigr) \\
      {} \with \bigl(\forall y'.\, d(y, y') \lolli \up \exists x'.\, d(x, x') \tensor \dn \defp{b}'_0(x', y')\bigr)
    \,,
    \end{lgathered}
\end{equation*}
together with similar defined linear propositions $\defp{e}(x, y)$, $\defp{b}_1(x, y)$, and $\defp{b}'_0(x, y)$.
Here the first-order parameters $x$ and $y$ thread together propositions in the context in a way that maintains the binary counter's essential structure.
In line with work by \textcite{Simmons+Pfenning:HOSC11}, $\defp{b}_0(x, y)$ is the \emph{destination-passing} embedding of $\defp{b}_0$.
In a formula-as-process interpretation, the destinations $x$ and $y$ can be viewed as channels that connect processes.

This example, however, does not fully exploit the expressive power of first-order linearity because the binary counter still has a chain topology.
Because \ac{SILL} processes admit tree topologies, defined propositions that use destinations in a tree topology should be allowed.
But a judgment for checking that destinations do not form cycles would likely be needed, for otherwise a destinations-as-channels interpretation would lead to ill-formed \ac{SILL} processes.

We would also need to characterize a general procedure for choreographing multiset rewriting specifications, in the vein of what was presented in \cref{ch:choreographies} for string rewriting specifications.


\subsection{First-order extension}

Another avenue for future work would be to extend the results contained in this document to first-order, not propositional, ordered rewriting and first-order polymorphic session-typed processes.
For example, we might embed sending and receiving processes by 
\begin{equation*}
  \begin{lgathered}[t]
    \trproc{a \shortleftarrow \mathsf{recvR}; P} = \dn \forall a.\, (\up \trproc{P} \pmir \atmL{\mathsf{tm}}(a)) \\
    \trproc{\mathsf{sendL}\:t} = \atmL{\mathsf{tm}}(t)
  \,,
  \end{lgathered}
\end{equation*}
and similarly for $\mathsf{sendR}$ and $\mathsf{recvL}$.
It would be nice if the extralogical $\atmL{\mathsf{tm}}$ and $\atmR{\mathsf{tm}}$ predicates could be done away with, but that appears to be impossible.
It seems that, even in an asynchronous calculus, the $\atmL{\mathsf{tm}}$ and $\atmR{\mathsf{tm}}$ atoms provide the small but necessary amount of synchronization to ensure that the intended term $t$ is used to instantiate the receiving proposition's universal quantifier.
However, it does seem plausible that terms could be packaged with the transmission of other labels, such as $\atmL{\kay}(t)$, which would hide  $\atmL{\mathsf{tm}}$ and $\atmR{\mathsf{tm}}$.

It would also be interesting to consider how session-typed processes with second-order polymorphism\autocite{Caires+:ESOP13} would be embedded in the rewriting framework.
A conjecture is that second-order polymorphism would be needed in the rewriting framework so that the embedding could be something like 
\begin{equation*}
  \begin{lgathered}[t]
    \trproc{\alpha \shortleftarrow \mathsf{recvR}; P} = \dn \forall \alpha.\, (\up \trproc{P} \pmir \atmL{\mathsf{pr}}(\alpha)) \\
    \trproc{\mathsf{sendL}\:P} = \atmL{\mathsf{pr}}(\trproc{P})
  \,.
  \end{lgathered}
\end{equation*}


\subsection{Session-typed nondeterministic choice}

In \cref{sec:correspond:types}, we leveraged the bisimilarity of process expressions and their embedding within formula-as-process ordered rewriting to reverse-engineer a session type system for ordered rewriting.
In particular, from the session-typing rules for $\caseL[\ell \in L]{\ell => P_{\ell}}$ and $\caseR[\ell \in L]{\ell => P_{\ell}}$, we arrived at rules for typing deterministic choices:
\begin{gather*}
  \infer[\lrule{\plus}]{\slof{\plus*[sub=_{\ell \in L}]{\ell:B_{\ell}} |- \dn {\textstyle \bigwith_{\ell \in L}(\atmR{\ell} \limp \up \p{A}_{\ell})} : C}}{
    \multipremise{\ell \in L}{\slof{B_{\ell} |- \p{A}_{\ell} : C}}}
  \shortintertext{and}
  \infer[\rrule{\with}]{\slof{A |- \dn {\textstyle \bigwith_{\ell \in L}(\up \p{A}_{\ell} \pmir \atmL{\ell})} : \with*[sub=_{\ell \in L}]{\ell:B_{\ell}}}}{
    \multipremise{\ell \in L}{\slof{A |- \p{A}_{\ell} : B_{\ell}}}}
\end{gather*}
Focusing in combination with the left- or right-handed implication ensures that the choices embodied by the alternative conjunction here are deterministic, not nondeterministic, choices.

However, in terms of the ordered propositions alone, it would seem more natural to have a typing rule for $\n{A}_1 \with \n{A}_2$ -- that is, a rule something like
\begin{equation*}
  \infer{\slof{A |- \n{A}_1 \with \n{A}_2 : C}}{\vdots}
  \,.
\end{equation*}
Finding such a rule for the ordered proposition $\n{A}_1 \with \n{A}_2$ might then allow us, by leveraging the ideas behind the bisimilar embedding of \cref{ch:correspond}, to reverse-engineer a process expression for some form of well-behaved, well-typed nondeterministic choice.

\Textfootcite{Stock:JUB20} has begun to look into incorporating nondeterministic choice into session-typed processes, emphasizing the operational considerations.
It would be interesting to consider whether his ideas can be adapted to the formula-as-process ordered rewriting framework.


\subsection{Induction, coinduction, termination, and productivity}

\Textfootcite{Derakhshan+Pfenning:LMCS20} have developed an infinitary calculus in which inductive and coinductive session types can be used to guarantee the termination and productivity of well-typed processes in a Curry--Howard interpretation of singleton logic.
Leveraging types, their validity condition on circular proofs is locally and effectively decidable.
\Textfootcite{Somayyajula+Pfenning:20} have done something similar with sized types.
An interesting question is whether their ideas might be applied to (formula-as-process) ordered rewriting.

Productivity is about observable progress and even untyped ordered rewriting has a notion of observation, as embodied in ordered bisimilarity~\parencref{ch:ordered-bisimilarity}.
So it seems likely that productive ordered rewriting systems can be characterized.

What is unclear is whether there is a locally and effectively decidable condition on ordered propositions that can guarantee productivity.
The local decidability of condition on circular proofs introduced by \citeauthor{Derakhshan+Pfenning:LMCS20}
% Can their validity condition on circular proofs be mapped to a condition on ordered propositions?
very much relies on the unrolling of inductive and coinductive types.
But ordered rewriting is untyped, at least natively, so whether productivity can be characterized in a locally decidable way is unclear.
% Yet productivity
% So it does not seem unreasonable to hope for some kind of validity condition in spite of the absence of types.
Also, unlike proofs, rewriting traces are constructed from open-ended derivations.
Does that open-endedness in any way affect the existence or shape of the productivity condition?

\subsection{Generative invariants and session types}

\Textfootcite{Simmons:CMU12} describes \emph{generative invariants} as a way to express invariants of ordered logical specifications.
These generative invariants generalize context-free grammars, as well as regular worlds from \textsc{lf}.
A generative invariant for a binary counter specification in his framework has similarities to the session type for binary counters given in \cref{sec:process-chains:binary-counter}:
\begin{align*}
  \mathsf{ctr} &\defd e \with (\mathsf{ctr} \fuse b_0) \with (\mathsf{ctr} \fuse b_1) \with (\mathsf{ctr} \fuse i) \\
  \mathsf{ctr}' &\defd z \with (\mathsf{ctr} \fuse s) \with (\mathsf{ctr} \fuse d) \with (\mathsf{ctr}' \fuse b'_0) \\[1ex]
  \ctr &\defd \with*{ i: \ctr , d: \plus*{ z: \ctre , s: \ctr } }
\end{align*}
It would be interesting to investigate whether these similarities can be extrapolated to a correspondence between generative invariants and session types.
If so, generative invariants might serve as a form of session typing for ordered rewriting that is more native than the system presented in \cref{sec:correspond:types}.

% In addition, the generative invariant for the binary counter looks a lot like the object-oriented choreography, but in an abstracted form:
% \begin{align*}
%   \mathsf{ctr} &\defd (\mathsf{ctr} \fuse i) \with \bigl((z \with (\mathsf{ctr} \fuse s)) \fuse d\bigr)
%   \\
%   \defp{e} &\defd (\defp{e} \fuse \defp{b}_1 \pmir \atmL{i}) \with (\atmR{z} \pmir \atmL{d}) \\
%   \defp{b}_0 &\defd (\up \dn \defp{b}_1 \pmir \atmL{i}) \with (\atmR{z} \pmir \atmL{d}) \\
% \end{align*}



%%% Local Variables:
%%% mode: latex
%%% TeX-master: "thesis"
%%% End:


\appendix
\chapter{\acs*{CCS}}

\begin{theorem}
  The following are equivalent.
  \begin{itemize}
  \item If $P \simu{R}\overset{c}{\reduces} Q'$, then $\bar{c} \mid P \overset{\tau}{\Reduces}\simu{R} Q'$.
  \item If $P \simu{R} Q$ and $\bar{c} \mid Q \overset{\tau}{\reduces} Q'$, then $\bar{c} \mid P \overset{\tau}{\Reduces}\simu{R} Q'$.
  \end{itemize}
\end{theorem}
\begin{proof}
  We prove each direction separately.
  \begin{itemize}
  \item 
  Assume that $P \simu{R} Q$ and $\bar{c} \mid Q \overset{\tau}{\reduces} Q'$.
  Either the internal transition on $\bar{c} \mid Q$ is derived from an input on $c$ -- \ie, $Q \overset{c}{\reduces} Q'$ -- or it leaves the message $\bar{c}$ untouched -- \ie, there exists a process $Q'_0$ such that $Q \overset{\tau}{\Reduces} Q'_0$ and $Q' = \bar{c} \mid Q'_0$.
  \begin{itemize}
  \item
    In the former case, we have $P \simu{R}\overset{c}{\reduces} Q'$ and can deduce that $\bar{c} \mid P \overset{\tau}{\Reduces}\simu{R} Q'$.
  \item 
    In the latter case, $P \overset{\tau}{\Reduces}\simu{R} Q'_0$ because $\simu{R}$ is reduction-closed.
    Framing $\bar{c}$ on yields $\bar{c} \mid P \overset{\tau}{\Reduces} \bar{c} \mid P'_0$ and $P'_0 \simu{R} Q'_0$, for some process $P'_0$.
    $\bar{c} \mid P'_0 \Reduces\simu{R} \bar{c} \mid Q'_0 = Q'$.
  \end{itemize}

  \item 
  Assume that $P \simu{R}\overset{c}{\reduces} Q'$ -- \ie, that $P \simu{R} Q$ and $Q \overset{c}{\reduces} Q'$, for some process $Q$.
  It follows that $\bar{c} \mid Q \overset{\tau}{\reduces} Q'$.
  Because $\simu{R}$ is contextual, $\bar{c} \mid P \simu{R} \bar{c} \mid Q$.\fixnote{Except that it isn't!}
  Because $\simu{R}$ is $\overset{\tau}{\Reduces}$-closed, $\bar{c} \mid P \overset{\tau}{\Reduces}\simu{R} Q'$.
  % 
  \qedhere
\end{itemize}
\end{proof}

\begin{theorem}\label{thm:emptiness-bisim-equiv}
  The following are equivalent.
  \begin{itemize}
  \item
    If $\octx \simu{R} (\octxe)$, then: $\atmR{\lctx} \oc \octx \Reduces\rframe{\simu{S}}{\atmR{\lctx}} \atmR{\lctx}$ for all $\atmR{\lctx}$; and $\atmL{\lctx} \oc \octx \Reduces\lframe{\atmL{\lctx}}{\simu{S}} \atmL{\lctx}$ for all $\atmL{\lctx}$.
  \item
    If $\octx \simu{R} (\octxe)$, then $\octx \Reduces (\octxe) \simu{S} (\octxe)$.
  \end{itemize}
\end{theorem}
\begin{proof}
  Because the premises of the two statements are the same, it suffices to prove that their conclusions are equivalent.
  We prove each direction separately.
  \begin{itemize}
  \item 
  Assume that $\atmR{\lctx} \oc \octx \Reduces\rframe{\simu{S}}{\atmR{\lctx}} \atmR{\lctx}$ for all $\atmR{\lctx}$ and $\atmL{\lctx} \oc \octx \Reduces\lframe{\atmL{\lctx}}{\simu{S}} \atmL{\lctx}$ for all $\atmL{\lctx}$.
  Choose an atomic proposition $\atmR{a}$ that does not appear in $\octx$; instantiating the emptiness bisimulation condition with $\atmR{\lctx} = \atmR{a}$, we have $\atmR{a} \oc \octx \Reduces\rframe{\simu{S}}{\atmR{a}} \atmR{a}$.
  We can prove by induction on the reduction sequence that $\octx \Reduces (\octxe) \simu{S} (\octxe)$.
  \begin{itemize}
  \item
    Consider the case in which the reduction sequence is trivial, \ie, the case in which $\atmR{a} \oc \octx \rframe{\simu{S}}{\atmR{a}} \atmR{a}$.
    Because $\atmR{a}$ does not appear in $\octx$, this holds only if $\octx = (\octxe) \simu{S} (\octxe)$.

  \item
    Consider the case in which $\atmR{a} \oc \octx \reduces\Reduces\rframe{\simu{S}}{\atmR{a}} \atmR{a}$.
    Because $\atmR{a}$ does not appear in $\octx$, it cannot participate in the initial reduction, so $\octx \reduces \octx'$ and $\atmR{a} \oc \octx' \Reduces\rframe{\simu{S}}{\atmR{a}} \atmR{a}$, for some context $\octx'$.
    Moreover, because it arises from $\octx$, the context $\octx'$ does not contain any occurances of $\atmR{a}$.
    From the inductive hypothesis it therefore follows that $\octx' \Reduces (\octxe) \simu{S} (\octxe)$;
    prepending the reduction from $\octx$, we conclude that $\octx \Reduces (\octxe) \simu{S} (\octxe)$.
  \end{itemize}

  \item 
  Assume that $\octx \Reduces (\octxe) \simu{S} (\octxe)$.
  Because reduction is closed under framing, $\atmR{\lctx} \oc \octx \Reduces \atmR{\lctx}$.
  Also, $\atmR{\lctx} \rframe{\simu{S}}{\atmR{\lctx}} \atmR{\lctx}$.
  It follows that $\atmR{\lctx} \oc \octx \Reduces\rframe{\simu{S}}{\atmR{\lctx}} \atmR{\lctx}$ for all $\atmR{\lctx}$; and, by symmetric reasoning, that $\octx \oc \atmL{\lctx} \Reduces\lframe{\atmL{\lctx}}{\simu{S}} \atmL{\lctx}$ for all $\atmL{\lctx}$.
  %
  \qedhere
  \end{itemize}
\end{proof}

%%% Local Variables:
%%% mode: latex
%%% TeX-master: "thesis"
%%% End:


%% \chapter{Miscellaneous notes}

\begin{itemize}
\item Negative propositions as processes; positive propositions and contexts as configurations?
\end{itemize}

%%% Local Variables:
%%% mode: latex
%%% TeX-master: "thesis"
%%% End:


\backmatter

\printbibliography

\end{document}
