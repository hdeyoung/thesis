\chapter{From processes to rewriting}

\section{A shallow embedding of processes in ordered rewriting}

\alertnote{Is this a shallow embedding?
Is HOAS deep or shallow?}

Here we give a translation, $\trconf{}$, of process configurations into ordered contexts from the formula-as-process rewriting framework of \cref{ch:formula-as-process}.
% More specifically, we first define two mutually inductive translations, $\trproc[+]{}$ and $\trproc[-]{}$, that map process expressions to positive and negative propositions, respectively, and then define a homomorphism, $\trconf{}$, that maps process configurations to ordered contexts.
For the translation to be adequate, it should serve as a bisimulation between the reduction semantics for process configurations and formula-as-process rewriting for ordered contexts.
Ideally, $\trconf{}$ ought to be a strong bisimulation, so that the operational correspondence is as tight as possible.
Because the translation will be a total function, the strong bisimulation requirement amounts to the diagrams
\begin{equation*}
  \begin{tikzcd}
    \cnf \rar[reduces] \dar[relation][swap]{\trconf{}} & \cnf\mathrlap{'} \dar[relation]{\trconf{}}
    \\
    \trconf{\cnf} \rar[reduces, exists] & \trconf{\cnf'}
  \end{tikzcd}
  \qquad\text{and}\qquad
  \begin{tikzcd}
    \cnf \rar[reduces, exists] \dar[relation][swap]{\trconf{}} & \cnf\mathrlap{'} \dar[relation, exists]{\trconf{}}
    \\
    \trconf{\cnf} \rar[reduces] & \octx\mathrlap{'\,.}
  \end{tikzcd}\hphantom{'\,.}
\end{equation*}

We expect that $\trconf{}$ will be a monoid homomorphism from process configurations to ordered contexts;
consequently, we will need an auxiliary translation, $\trproc{}$, for individual process expressions.
\begin{equation*}
  \begin{aligned}
    \trconf{\cnfe} &= (\octxe) \\
    \trconf{\cnf_1 \cc \cnf_2} &= \trconf{\cnf_1} \oc \trconf{\cnf_2} \\
    \trconf{P} &= \trproc{P}
  \end{aligned}
\end{equation*}
% So that $\trconf{\cnf}$ is a well-formed ordered context in the formula-as-process focused ordered rewriting framework, $\trproc{}$ should map process expressions to either atoms or negative propositions.%
% \footnote{Recall from \cref{??} that formula-as-process ordered contexts may not contain non-atomic positive propositions.}

With this definition for $\trconf{}$ in hand, we can then run each process reduction axiom through the first bisimulation diagram to generate constraints on $\trproc{}$ that must be satisfied for $\trconf{}$ to be a strong bisimulation.
These axioms and induced constraints are summarized in \cref{tbl:trconf-constraints}.%
%
\begin{table}[tb]
  \renewcommand{\arraystretch}{1.2}
  \begin{tabular}{@{}cc@{}}
    \toprule
    \emph{Process reduction} & \emph{Formula-as-process rewriting constraint}
    \\ \midrule
    $\spawn{P}{Q} \reduces P \cc Q$ & $\trproc{\spawn{P}{Q}} \dreduces \trproc{P} \oc \trproc{Q}$\hphantom{\quad($\kay \in L$)}
    \\
    $\fwd \reduces (\cnfe)$ & $\trproc{\fwd} \dreduces (\octxe)$\hphantom{\quad($\kay \in L$)}
    \\
    $\selectR{\kay} \cc \caseL[\ell \in L]{\ell => Q_{\ell}} \reduces Q_{\kay}$ & $\trproc{\selectR{\kay}} \oc \trproc{\caseL[\ell \in L]{\ell => Q_{\ell}}} \dreduces \trproc{Q_{\kay}}$\quad($\kay \in L$)
    \\
    $\caseR[\ell \in L]{\ell => P_{\ell}} \cc \selectL{\kay} \reduces P_{\kay}$ & $\trproc{\caseR[\ell \in L]{\ell => P_{\ell}}} \oc \trproc{\selectL{\kay}} \dreduces \trproc{P_{\kay}}$\quad($\kay \in L$)
    \\ \addlinespace \bottomrule
  \end{tabular}
  \caption{Constraints on $\trproc{}$ that must be satisfied if $\trconf{}$ is to be a strong reduction bisimulation}\label{tbl:trconf-constraints}
\end{table}

For example, the process reduction axiom $\spawn{P}{Q} \reduces P \cc Q$ induces the constraint $\trproc{\spawn{P}{Q}} \dreduces \trproc{P} \oc \trproc{Q}$.
In other words, we would like to find a definition for $\trproc{\spawn{P}{Q}}$, compositional in $\trproc{P}$ and $\trproc{Q}$, such that $\trproc{\spawn{P}{Q}}$  decomposes to $\trproc{P} \oc \trproc{Q}$ in a single step.

In the formula-as-process ordered rewriting framework with the eager inversion of positive propositions that its fully focused nature implies, we can define
\begin{equation*}
  \trproc{\spawn{P}{Q}} = \up (\bigfuse \trproc{P} \fuse \bigfuse \trproc{Q})
    \reduces \trproc{P} \oc \trproc{Q}
\end{equation*}
because $\rfocus{\trproc{P}}{\bigfuse \trproc{P}}$ and, similarly, $\rfocus{\trproc{Q}}{\bigfuse \trproc{Q}}$.

This $\up (\bigfuse \octx_1 \fuse \bigfuse \octx_2)$ pattern looks familiar.
In \cref{??}, it was used in the embedding of weakly focused ordered rewriting into fully focused ordered rewriting.
What if, instead of using the fully focused formula-as-process rewriting framework, we were to move to a \emph{weakly} focused framework?
Then we could simplify the definition of $\trproc{\spawn{P}{Q}}$ to merely
\begin{equation*}
  \trproc{\spawn{P}{Q}} = \trproc{P} \fuse \trproc{Q}
  \,,
\end{equation*}
which indeed takes a complete step to decompose the conjunction: $\trproc{P} \fuse \trproc{Q} \reduces \trproc{P} \oc \trproc{Q}$.
This As an embedding, weakly focused rewriting 

\begin{equation*}
  \trproc{\fwd} = \one
\end{equation*}

Observe that $\trproc{\selectR{\kay}} \oc \dn {\textstyle \bigwith_{\ell \in L}(\trproc{\selectR{\ell}} \limp \up \trproc{Q_{\ell}})} \reduces \trproc{Q_{\kay}}$ is a valid weakly focused formula-as-process rewriting, for all $\kay \in L$, if $\trproc{\selectR{\kay}}$ is a right-directed atom.
\begin{align*}
  \trproc{\selectR{\kay}} &= \atmR{\kay} \\
  \trproc{\caseL[\ell \in L]{\ell => Q_{\ell}}} &= \dn {\textstyle \bigwith_{\ell \in L}(\atmR{\ell} \limp \up \trproc{Q_{\ell}})}
%
\intertext{and, symmetrically,}
%
  \trproc{\caseR[\ell \in L]{\ell => P_{\ell}}} &= \dn {\textstyle \bigwith_{\ell \in L}(\up \trproc{P_{\ell}} \pmir \atmL{\ell})} \\
  \trproc{\selectL{\kay}} &= \atmL{\kay}
\end{align*}

The obvious candidate is to define
\begin{equation*}
  \trproc{\spawn{P}{Q}} = \trproc{P} \fuse \trproc{Q}
  \,,
\end{equation*}
but there are two obstacles to such a definition.
First, $\trproc{P} \fuse \trproc{Q}$ is not a non-atomic positive proposition.
But, second and more troublingly, positive propositions are decomposed all at once during the focusing bipole that constitutes a rewriting step in the formula-as-process \emph{focused} ordered rewriting framework.
This leads to a mismatch between the small-step process reductions and the big-step formula-as-process focused ordered rewriting, as shown in the adjacent \lcnamecref{fig:translation:focused-mismatch}.%
%
\begin{marginfigure}
  \begin{equation*}
    \begin{tikzcd}
      \spawn{(\spawn{P}{Q})}{R} \rar[reduces] \arrow[relation]{dd}[swap]{\trconf{}}
        & (\spawn{P}{Q}) \cc R \dar[relation]{\trconf{}}
      \\
        & (\trproc{P} \fuse \trproc{Q}) \oc \trproc{R} \dar[phantom]{\neq}
      \\[-2ex]
      (\trproc{P} \fuse \trproc{Q}) \fuse \trproc{R} \rar[reduces]
        & (\trproc{P} \oc \trproc{Q}) \oc \trproc{R}
    \end{tikzcd}
  \end{equation*}
  \caption{Mismatch between process reduction and big-step decomposition of positive propositions}\label{fig:translation:focused-mismatch}
\end{marginfigure}%

There are (at least) two solutions.
We could abandon our ideal strong bisimulation and instead settle for a weak bisimulation.

Another possible solution, and the one we pursue here, is to shift the rewriting framework from a fully focused framework to the \emph{weakly focused} formula-as-process ordered rewriting framework described in \cref{??}.
With weak focusing, positive propositions are not decomposed eagerly in a big-step manner -- 
\begin{gather*}
  (\trproc{P} \fuse \trproc{Q}) \fuse \trproc{R} \nreduces (\trproc{P} \oc \trproc{Q}) \oc \trproc{R}
\intertext{but rather}
  (\trproc{P} \fuse \trproc{Q}) \fuse \trproc{R} \reduces (\trproc{P} \fuse \trproc{Q}) \oc \trproc{R}
\end{gather*}
%
\begin{marginfigure}
  \begin{equation*}
    \begin{tikzcd}
      \spawn{(\spawn{P}{Q})}{R} \rar[reduces] \dar[relation][swap]{\trconf{}}
        & (\spawn{P}{Q}) \cc R \dar[relation]{\trconf{}}
      \\
      (\trproc{P} \fuse \trproc{Q}) \fuse \trproc{R} \rar[reduces]
        & (\trproc{P} \fuse \trproc{Q}) \oc \trproc{R}
    \end{tikzcd}
  \end{equation*}
  \caption{Mismatch between process reduction and big-step decomposition of positive propositions}\label{fig:translation:focused-mismatch}
\end{marginfigure}%

\begin{marginfigure}
  \begin{align*}
    \trconf{\cnfe} &= (\octxe) \\
    \trconf{\cnf_1 \cc \cnf_2} &= \trconf{\cnf_1} \oc \trconf{\cnf_2} \\
    \trconf{P} &= \trproc{P}
  %
  \shortintertext{where}
  %
    \trproc{\spawn{P_1}{P_2}}
      &= \trproc{P_1} \fuse \trproc{P_2} \\
    \trproc{\fwd} &= \one
    \\
    \trproc{\selectR{\kay}} &= \atmR{\kay} \\
    \trproc{\caseL[\ell \in L]{\ell => P_{\ell}}}
      &= \dn {\textstyle \bigwith_{\ell \in L}(\atmR{\ell} \limp \up \trproc{P_{\ell}})}
    \\
    \trproc{\caseR[\ell \in L]{\ell => P_{\ell}}}
      &= \dn {\textstyle \bigwith_{\ell \in L}(\up \trproc{P_{\ell}} \pmir \atmL{\ell})} \\
    \trproc{\selectL{\kay}} &= \atmL{\kay}
    \\
    \trproc{\defp{p}} &= \dn \n{\defp{p}}
  \end{align*}
  \caption{A \emph{strongly} bisimilar embedding of process configurations within \emph{weakly} focused formula-as-process ordered rewriting}\label{fig:process-embedding:weak-focus-strong-bisim}
\end{marginfigure}

\begin{theorem}[Adequacy of $\trconf{}$]
  Under weakly focused formula-as-process ordered rewriting, $\trconf{}$ constitutes a strong bisimulation.
  That is:
  \begin{itemize}[nosep]
  \item If $\cnf \reduces \cnf'$, then $\trconf{\cnf} \reduces \trconf{\cnf'}$.
  \item If $\trconf{\cnf} \reduces \octx'$, then there exists $\cnf'$ such that $\cnf \reduces \cnf'$ and $\trconf{\cnf'} = \octx'$.
  \end{itemize}
\end{theorem}
\begin{proof}
  The first part is by induction on the process configuration reduction, $\cnf \reduces \cnf'$; the second part is by induction on the weakly focused formula-as-process ordered rewriting, $\trconf{\cnf} \reduces \octx'$.
\end{proof}

\begin{marginfigure}
  \begin{align*}
    \trconf{\cnfe} &= (\octxe) \\
    \trconf{\cnf_1 \cc \cnf_2} &= \trconf{\cnf_1} \oc \trconf{\cnf_2} \\
    \trconf{P} &= \trproc{P}
  %
  \shortintertext{where}
  %
    \trproc{\spawn{P_1}{P_2}}
      &= \up (\bigfuse \trproc{P_1} \fuse \bigfuse \trproc{P_2}) \\
    \trproc{\fwd} &= \up \one
    \\
    \trproc{\selectR{\kay}} &= \atmR{\kay} \\
    \trproc{\caseL[\ell \in L]{\ell => P_{\ell}}}
      &= {\textstyle \bigwith_{\ell \in L}(\atmR{\ell} \limp \up \bigfuse \trproc{P_{\ell}})}
    \\
    \trproc{\caseR[\ell \in L]{\ell => P_{\ell}}}
      &= {\textstyle \bigwith_{\ell \in L}(\up \bigfuse \trproc{P_{\ell}} \pmir \atmL{\ell})} \\
    \trproc{\selectL{\kay}} &= \atmL{\kay}
    \\
    \trproc{\defp{p}} &= \n{\defp{p}}
  \end{align*}
  \caption{A \emph{strongly} bisimilar embedding of process configurations within \emph{fully} focused formula-as-process ordered rewriting}\label{fig:process-embedding:full-focus-strong-bisim}
\end{marginfigure}


\begin{marginfigure}
  \begin{align*}
    \trconf{\cnfe} &= (\octxe) \\
    \trconf{\cnf_1 \cc \cnf_2} &= \trconf{\cnf_1} \oc \trconf{\cnf_2} \\
    \trconf{P} &= \trproc{P}
  %
  \shortintertext{where}
  %
    \trproc{\spawn{P_1}{P_2}}
      &= \trproc{P_1} \oc \trproc{P_2} \\
    \trproc{\fwd} &= \octxe
    \\
    \trproc{\selectR{\kay}} &= \atmR{\kay} \\
    \trproc{\caseL[\ell \in L]{\ell => P_{\ell}}}
      &= {\textstyle \bigwith_{\ell \in L}(\atmR{\ell} \limp \up \bigfuse \trproc{P_{\ell}})}
    \\
    \trproc{\caseR[\ell \in L]{\ell => P_{\ell}}}
      &= {\textstyle \bigwith_{\ell \in L}(\up \bigfuse \trproc{P_{\ell}} \pmir \atmL{\ell})} \\
    \trproc{\selectL{\kay}} &= \atmL{\kay}
    \\
    \trproc{\defp{p}} &= \n{\defp{p}}
  \end{align*}
  \caption{A \emph{weakly} bisimilar embedding of process configurations within \emph{fully} focused formula-as-process ordered rewriting}\label{fig:process-embedding:full-focus-weak-bisim}
\end{marginfigure}



\begin{equation*}
  \begin{aligned}
    \trconf{\cnfe} &= (\octxe) \\
    \trconf{\cnf_1 \cc \cnf_2} &= \trconf{\cnf_1} \oc \trconf{\cnf_2} \\
    \trconf{P} &= \octx \text{, where $\rfocus{\octx}{\trproc{P}}$}
  \end{aligned}
\end{equation*}

\begin{theorem}
  Under (fully) focused formula-as-process ordered rewriting, $\trconf{}$ constitutes only a \emph{weak} reduction bisimulation.
  That is:
  \begin{itemize}[nosep]
  \item If $\cnf \reduces \cnf'$, then either $\trconf{\cnf} \reduces \trconf{\cnf'}$ or $\trconf{\cnf} = \trconf{\cnf'}$.
  \item If $\trconf{\cnf} \reduces \octx'$, then there exists $\cnf'$ such that $\cnf \Reduces \cnf'$ and $\trconf{\cnf'} = \octx'$.
  \end{itemize}
\end{theorem}



% \begin{align*}
%   \embed*{\bigfuse \trproc{\spawn{P_1}{P_2}}}
%     &= \up (\bigfuse \embed*{\bigfuse \trproc{P_1}} \fuse \bigfuse \embed*{\bigfuse \trproc{P_2}}) \\
%   \embed*{\bigfuse \trproc{\fwd}} &= \up \one
%   \\
%   \embed*{\bigfuse \trproc{\selectR{\kay}}} &= \atmR{\kay} \\
%   \embed*{\bigfuse \trproc{\caseL[\ell \in L]{\ell => P_{\ell}}}} &= {\textstyle \bigwith_{\ell \in L}(\atmR{\ell} \limp \up \bigfuse \embed*{\bigfuse \trproc{P_{\ell}}})}
%   \\
%   \embed*{\bigfuse \trproc{\caseR[\ell \in L]{\ell => P_{\ell}}}} &= {\textstyle \bigwith_{\ell \in L}(\up \bigfuse \embed*{\bigfuse \trproc{P_{\ell}}} \pmir \atmL{\ell})} \\
%   \embed*{\bigfuse \trproc{\selectL{\kay}}} &= \atmL{\kay}
% \end{align*}

\begin{equation*}
  \begin{aligned}
    \trconf{\octxe} &= (\octxe) \\
    \trconf{\cnf_1 \cc \cnf_2} &= \trconf{\cnf_1} \oc \trconf{\cnf_2} \\
    \trconf{P} &= \trproc[\fuse]{P}
  \end{aligned}
  \quad\text{and}\quad
  \begin{aligned}
    \trproc[\fuse]{\spawn{P_1}{P_2}}
      &= \up (\bigfuse \trproc[\fuse]{P_1} \fuse \bigfuse \trproc[\fuse]{P_2}) \\
    \trproc[\fuse]{\fwd} &= \up \one
    \\
    \trproc[\fuse]{\selectR{\kay}} &= \atmR{\kay} \\
    \trproc[\fuse]{\caseL[\ell \in L]{\ell => P_{\ell}}} &= {\textstyle \bigwith_{\ell \in L}(\atmR{\ell} \limp \up \bigfuse \trproc[\fuse]{P_{\ell}})}
    \\
    \trproc[\fuse]{\caseR[\ell \in L]{\ell => P_{\ell}}} &= {\textstyle \bigwith_{\ell \in L}(\up \bigfuse \trproc[\fuse]{P_{\ell}} \pmir \atmL{\ell})} \\
    \trproc[\fuse]{\selectL{\kay}} &= \atmL{\kay} \\
    \trproc[\fuse]{\defp{p}} &= \n{\defp{p}}
  \end{aligned}
\end{equation*}

\begin{theorem}
  $\trproc[\fuse]{P} = \embed*{\trproc{P}}$
\end{theorem}

Except for coinductive definitions?


\begin{equation*}
  \begin{aligned}
    \trconf{\cnfe} &= (\octxe) \\
    \trconf{\cnf_1 \cc \cnf_2} &= \trconf{\cnf_1} \oc \trconf{\cnf_2} \\
    \trconf{P} &= \trproc[+]{P}
  \end{aligned}
  \quad\text{where}\quad
  \begin{aligned}
    \trproc[+]{\spawn{P_1}{P_2}}
      &= \trproc[+]{P_1} \fuse \trproc[+]{P_2} \\
    \trproc[+]{\fwd} &= \one
    \\
    \trproc[+]{\selectR{\kay}} &= \atmR{\kay} \\
    \trproc[+]{\selectL{\kay}} &= \atmL{\kay} \\
    \trproc[+]{P} &= \dn \trproc[-]{P}
    \\
    \trproc[-]{\caseL[\ell \in L]{\ell => P_{\ell}}}
      &= {\textstyle \bigwith_{\ell \in L}(\atmR{\ell} \limp \up \trproc[+]{P_{\ell}})} \\
    \trproc[-]{\caseR[\ell \in L]{\ell => P_{\ell}}}
      &= {\textstyle \bigwith_{\ell \in L}(\up \trproc[+]{P_{\ell}} \pmir \atmL{\ell})} \\
    \trproc[-]{\defp{p}} &= \n{\defp{p}} \\
    \trproc[-]{P} &= \up \trproc[+]{P}
  \end{aligned}
\end{equation*}

\begin{equation*}
  \trconf{p \defd P} = (\n{\defp{p}} \defd \trproc[-]{P})
\end{equation*}




% (k)0 = k0 = k = [k]0
% (P*Q)0 = {(P) * (Q)}0 = ^{(P)+ * (Q)+} = ^{[P]+ * [Q]+} = [P*Q]0
% (=)0 = {1}0 = ^1 = [=]0
% (caseL{P_i})0 = {v&_i {i \ ^(P_i)}}0 = {&_i {i \ ^(P_i)}}- = &_i {i \ ^(P_i)+} = &_i {i \ ^[P_i]+} = [caseL{P_i}]- = [caseL{P_i}]0
% (p)0 = {v p}0 = {p}- = p = [p]- = [p]0

% (k)+ = {k}+ = k = [k]+
% (P)+ = v{(P)0} = v[P]0 = [P]+

% [caseL{P_i}]- = &_i {i \ ^[P_i]+} = &_i {i \ ^{(P_i)}+} = {&_i {i \ ^(P_i)}}- = ... = ^ v &_i = {^ v &_i {i \ ^(P_i)}}- = {^ (caseL{P_i})}-
% [P]- = ^[P]+ = ^{(P)+} = {^(P)}- = 

\begin{align*}
  \trconf{b_0 \defd \caseR{i => b_1 | d => \spawn{\selectL{d}}{b'_0}}}
    &= \bigl(
         \n{\defp{b}_0} \defd
           (\up \dn \n{\defp{b}_1} \pmir \atmL{i}) \with
           (\up (\atmL{d} \fuse \dn \n{\defp{b}'_0{}}) \pmir \atmL{d})
       \bigr)
  \\
    &= \bigl(
         \defp{b}_0 \defd
           (\up \dn \defp{b}_1 \pmir \atmL{i}) \with
           (\atmL{d} \fuse \defp{b}'_0 \pmir \atmL{d})
       \bigr)
\end{align*}

\begin{equation*}
  \begin{lgathered}
    \trconf{e \defd \caseR{i => \spawn{e}{b_1} | d => \selectR{z}}}
      = \bigl(\n{\defp{e}} \defd (\up (\dn \n{\defp{e}} \fuse \dn \n{\defp{b}_1}) \pmir \atmL{i}) \with (\up \atmR{z} \pmir \atmL{d})\bigr)
    \\
    \trconf{b_0 \defd \caseR{i => b_1 | d => \spawn{\selectL{d}}{b'_0}}}
      = \bigl(\n{\defp{b}_0} \defd (\up \dn \n{\defp{b}_1} \pmir \atmL{i}) \with (\up (\atmL{d} \fuse \dn \n{\defp{b}'_0{}}) \pmir \atmL{d})\bigr)
    \\
    \trconf{b_1 \defd \caseR{i => \spawn{\selectL{i}}{b_0} | d => \spawn{b_0}{\selectR{s}}}}
      = \bigl(\n{\defp{b}_1} \defd (\up (\atmL{i} \fuse \dn \n{\defp{b}_0}) \pmir \atmL{i}) \with (\up (\dn \n{\defp{b}_0} \fuse \atmR{s}) \pmir \atmL{d})\bigr)
    \\
    \trconf{b'_0 \defd \caseL{z => \selectR{z} | s => \spawn{b_1}{\selectR{s}}}}
      = \bigl(\n{\defp{b}'_0{}} \defd (\atmR{z} \limp \up \atmR{z}) \with (\atmR{s} \limp \up (\dn \n{\defp{b}_1} \fuse \atmR{s}))\bigr)
  \end{lgathered}
\end{equation*}


\begin{align*}
  \trconf{q \defd \caseL[a \in \ialph]{a => q'_a | \eow => F(q)}}
    &= \bigl(
         \defp{q} \defd
           (\atmR{\eow} \limp \atmR{F}(q)) \with
           {\textstyle \bigwith_{a \in \ialph}(\atmR{a} \limp \up \dn \defp{q}'_a)}
       \bigr)
\end{align*}



\clearpage

Notice that neither this \lcnamecref{??} nor a corresponding weak reduction bisimulation \lcnamecref{??} would hold if the unfocused form of ordered rewriting were used.%
\footnote{For example, $\trconf{\caseL[\ell \in L]{\ell => P_{\ell}}} = \bigwith_{\ell \in L}(\atmR{\ell} \limp \trproc{Q_{\ell}}) \reduces (\atmR{\kay} \limp \trproc{Q_{\kay}})$ if $\kay \in L$, but there is no configuration $\cnf'$ such that $\caseL[\ell \in L]{\ell => P_{\ell}} \Reduces \cnf'$ and $\trconf{\cnf'} = \atmR{\kay} \limp \trproc{Q_{\kay}}$.}




\clearpage
\section{A session type system for ordered rewriting}

With the above 

\begin{inferences}
  \infer[\jrule{CUT}\smash{^B}]{\slof{A |- \p{A}_1 \fuse \p{A}_2 : C}}{
    \slof{A |- \p{A}_1 : B} & \slof{B |- \p{A}_2 : C}}
  \and
  \infer[\jrule{ID}\smash{^A}]{\slof{A |- \one : A}}{}
  \\
  \infer[\rrule{\plus}']{\slof{A_{\kay} |- \selectR{\kay} : \plus*[sub=_{\ell \in L}]{\ell:A_{\ell}}}}{
    \text{($\kay \in L$)}}
  \and
  \infer[\lrule{\plus}]{\slof{\plus*[sub=_{\ell \in L}]{\ell:A_{\ell}} |- \dn \bigwith_{\ell \in L} (\selectR{\ell} \limp \up \p{A}_{\ell}) : C}}{
    \multipremise{\ell \in L}{\slof{A_{\ell} |- \p{A}_{\ell} : C}}}
  \\
  \infer[\rrule{\with}]{\slof{A |- \dn \bigwith_{\ell \in L} (\p{A}_{\ell} \pmir \selectL{\ell}) : \with*[sub=_{\ell \in L}]{\ell:B_{\ell}}}}{
    \multipremise{\ell \in L}{\slof{A |- \p{A}_{\ell} : B_{\ell}}}}
  \and
  \infer[\lrule{\with},]{\slof{\with*[sub=_{\ell \in L}]{\ell:B_{\ell}} |- \selectL{\kay} : B_{\kay}}}{
    \text{($\kay \in L$)}}
  \\
  \infer[\jrule{C-CUT}\smash{^B}]{\slcof{A |- \octx_1 \oc \octx_2 : C}}{
    \slcof{A |- \octx_1 : B} & \slcof{B |- \octx_2 : C}}
  \and
  \infer[\jrule{C-ID}\smash{^A}]{\slcof{A |- \octxe : A}}{}
  \and
  \infer[\jrule{C-PROC}]{\slcof{A |- \p{A} : B}}{
    \slof{A |- \p{A} : B}}
\end{inferences}

\footnote{Careful with the 0-ary forms because you could end up with $\slof{\zero |- \top : C}$ and \emph{also} $\slof{A |- \top : \top}$.}

\begin{theorem}
  If $\slof{A |- P : B}$, then $\slof{A |- \trproc[+]{P} : B}$.
  Conversely, if $\slof{A |- \p{A} : B}$, then $\slof{A |- P : B}$ for some process $P$ such that $\trproc[+]{P} = \p{A}$.

  Similarly, if $\slcof{A |- \cnf : B}$, then $\slcof{A |- \trconf{\cnf} : B}$.
  Conversely, if $\slcof{A |- \octx : B}$, then $\slcof{A |- \cnf : B}$ for some configuration $\cnf$ such that $\trconf{\cnf} = \octx$.
\end{theorem}

%%% Local Variables:
%%% mode: latex
%%% TeX-master: "thesis"
%%% End:
