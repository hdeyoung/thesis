\chapter{From processes to rewriting}

\section{A shallow embedding of processes in ordered rewriting}

\alertnote{Is this a shallow embedding?
Is HOAS deep or shallow?}

Here we give a translation, $\trconf{}$, of process configurations into ordered contexts from the formula-as-process rewriting framework of \cref{ch:formula-as-process}.
% More specifically, we first define two mutually inductive translations, $\trproc[+]{}$ and $\trproc[-]{}$, that map process expressions to positive and negative propositions, respectively, and then define a homomorphism, $\trconf{}$, that maps process configurations to ordered contexts.
For the translation to be adequate, it should serve as a reduction bisimulation between the reduction semantics for process configurations and formula-as-process rewriting for ordered contexts.
Ideally, $\trconf{}$ ought to be a strong reduction bisimulation, so that the operational correspondence is as tight as possible.
Because the translation will be a total function, the strong bisimulation requirement amounts to the diagrams
\begin{equation*}
  \begin{tikzcd}
    \cnf \rar[reduces] \dar[relation][swap]{\trconf{}} & \cnf\mathrlap{'} \dar[relation]{\trconf{}}
    \\
    \trconf{\cnf} \rar[reduces, exists] & \trconf{\cnf'}
  \end{tikzcd}
  \qquad\text{and}\qquad
  \begin{tikzcd}
    \cnf \rar[reduces, exists] \dar[relation][swap]{\trconf{}} & \cnf\mathrlap{'} \dar[relation, exists]{\trconf{}}
    \\
    \trconf{\cnf} \rar[reduces] & \octx\mathrlap{'\,.}
  \end{tikzcd}\hphantom{'\,.}
\end{equation*}

We expect that $\trconf{}$ be a monoid homomorphism from process configurations to ordered contexts;
consequently, we will need an auxiliary translation, $\trproc{}$.
\begin{equation*}
  \begin{aligned}
    \trconf{\cnfe} &= (\octxe) \\
    \trconf{\cnf_1 \cc \cnf_2} &= \trconf{\cnf_1} \oc \trconf{\cnf_2} \\
    \trconf{P} &= \trproc{P}
  \end{aligned}
\end{equation*}
So that $\trconf{\cnf}$ is a well-formed ordered context in the formula-as-process focused ordered rewriting framework, $\trproc{}$ should map process expressions to either atoms or negative propositions.%
\footnote{Recall from \cref{??} that formula-as-process ordered contexts may not contain non-atomic positive propositions.}

With this definition for $\trconf{}$ in hand, we can then run each process reduction axiom through the first bisimulation diagram to generate constraints on $\trproc{}$ that must be satisfied if $\trconf{}$ is to be a strong bisimulation.
These axioms and induced constraints are summarized in \cref{tbl:trconf-constraints}.%
%
\begin{table}[tb]
  \renewcommand{\arraystretch}{1.2}
  \begin{tabular}{@{}cc@{\enspace}l@{}}
    \toprule
    \emph{Process reduction} & \emph{Formula-as-process rewriting constraint}
    \\ \midrule
    $\spawn{P}{Q} \reduces P \cc Q$ & $\trproc{\spawn{P}{Q}} \dreduces \trproc{P} \oc \trproc{Q}$
    \\
    $\fwd \reduces (\cnfe)$ & $\trproc{\fwd} \dreduces (\octxe)$
    \\
    $\selectR{\kay} \cc \caseL[\ell \in L]{\ell => Q_{\ell}} \reduces Q_{\kay}$ & $\trproc{\selectR{\kay}} \oc \trproc{\caseL[\ell \in L]{\ell => Q_{\ell}}} \dreduces \trproc{Q_{\kay}}$ & $\!(\kay \in L)$
    \\
    $\caseR[\ell \in L]{\ell => P_{\ell}} \cc \selectL{\kay} \reduces P_{\kay}$ & $\trproc{\caseR[\ell \in L]{\ell => P_{\ell}}} \oc \trproc{\selectL{\kay}} \dreduces \trproc{P_{\kay}}$ & $\!(\kay \in L)$
    \\ \addlinespace \bottomrule
  \end{tabular}
  \caption{Constraints on $\trproc{}$ that must be satisfied if $\trconf{}$ is to be a strong reduction bisimulation}\label{tbl:trconf-constraints}
\end{table}

For example, the process reduction axiom $\spawn{P}{Q} \reduces P \cc Q$ induces the constraint $\trproc{\spawn{P}{Q}} \dreduces \trproc{P} \oc \trproc{Q}$.
In other words, we would like to find a definition for $\trproc{\spawn{P}{Q}}$, compositional in $\trproc{P}$ and $\trproc{Q}$, such that $\trproc{\spawn{P}{Q}}$  decomposes to $\trproc{P} \oc \trproc{Q}$ in a single step.

The obvious candidate is to define
\begin{equation*}
  \trproc{\spawn{P}{Q}} = \trproc{P} \fuse \trproc{Q}
  \,,
\end{equation*}
but there are two obstacles to such a definition.
First, $\trproc{P} \fuse \trproc{Q}$ is not a non-atomic positive proposition.
But, second and more troublingly, positive propositions are decomposed all at once during the focusing bipole that constitutes a rewriting step in the formula-as-process \emph{focused} ordered rewriting framework.
This leads to a mismatch between the small-step process reductions and the big-step formula-as-process focused ordered rewriting, as shown in the adjacent \lcnamecref{fig:translation:focused-mismatch}.%
%
\begin{marginfigure}
  \begin{equation*}
    \begin{tikzcd}
      \spawn{(\spawn{P}{Q})}{R} \rar[reduces] \arrow[relation]{dd}[swap]{\trconf{}}
        & (\spawn{P}{Q}) \cc R \dar[relation]{\trconf{}}
      \\
        & (\trproc{P} \fuse \trproc{Q}) \oc \trproc{R} \dar[phantom]{\neq}
      \\[-2ex]
      (\trproc{P} \fuse \trproc{Q}) \fuse \trproc{R} \rar[reduces]
        & (\trproc{P} \oc \trproc{Q}) \oc \trproc{R}
    \end{tikzcd}
  \end{equation*}
  \caption{Mismatch between process reduction and big-step decomposition of positive propositions}\label{fig:translation:focused-mismatch}
\end{marginfigure}%


\begin{equation*}
  \trproc{\caseL[\ell \in L]{\ell => Q_{\ell}}} = \dn \bigwith_{\ell \in L} \bigl(\trproc{\selectR{\ell}} \limp \up \trproc{Q_{\ell}}\bigr)
  \,.
\end{equation*}

\begin{equation*}
  \trproc{\selectR{\kay}} = \atmR{\kay}
\end{equation*}


\begin{equation*}
  \!\begin{aligned}[b]
    \trproc{\spawn{P}{Q}} &= \trproc{P} \fuse \trproc{Q} \\
    \trproc{\fwd} &= \one \\
    \trproc{\selectR{\kay}} &= \atmR{\kay} \\
    \trproc{\selectL{\kay}} &= \atmL{\kay} \\
    \trproc{\defp{p}} &= \dn \n{\defp{p}}
  \end{aligned}
  \quad
  \!\begin{aligned}[b]
    \trproc{\caseL[\ell \in L]{\ell => Q_{\ell}}} &= \dn {\textstyle \bigwith_{\ell \in L}\bigl(\atmR{\ell} \limp \up \trproc{Q_{\ell}}\bigr)} \\
    \trproc{\caseR[\ell \in L]{\ell => P_{\ell}}} &= \dn {\textstyle \bigwith_{\ell \in L}\bigl(\up \trproc{P_{\ell}} \pmir \atmL{\ell}\bigr)} \\\\
  \end{aligned}
\end{equation*}


\begin{equation*}
  \begin{lgathered}
    \defp{e} \defd \caseR{i => \spawn{\defp{e}}{\defp{b}_1}
                        | d => \selectR{z}}
    \\
    \defp{b}_0 \defd \caseR{i => \defp{b}_1
                          | d => \spawn{\selectL{d}}{\defp{b}'_0}}
    \\
    \defp{b}_1 \defd \caseR{i => \spawn{\selectL{i}}{\defp{b}_0}
                          | d => \spawn{\defp{b}_0}{\selectR{s}}}
    \\
    \defp{b}'_0 \defd \caseL{z => \selectR{z}
                           | s => \spawn{\defp{b}_1}{\selectR{s}}}
  \end{lgathered}
  \qquad
  \begin{lgathered}
    \n{\defp{e}} \defd \dn \bigl((\up (\dn \n{\defp{e}} \fuse \dn \n{\defp{b}_1}) \pmir \atmL{i}) \with (\up \atmR{z} \pmir \atmL{d})\bigr)
    \\
    \n{\defp{b}_0} \defd \dn \bigl((\up \dn \n{\defp{b}_1} \pmir \atmL{i}) \with (\up (\atmL{d} \fuse \dn \n{\defp{b}'_0{}}) \pmir \atmL{d})\bigr)
    \\
    \n{\defp{b}_1} \defd \dn \bigl((\up (\atmL{i} \fuse \dn \n{\defp{b}_1}) \pmir \atmL{i}) \with (\up (\dn \n{\defp{b}_0} \fuse \atmR{s}) \pmir \atmL{d})\bigr)
    \\
    \n{\defp{b}'_0{}} \defd \dn \bigl((\atmR{z} \limp \up \atmR{z}) \with (\atmR{s} \limp \up (\dn \n{\defp{b}_1} \fuse \atmR{s})\bigr)
  \end{lgathered}
  \qquad
  \begin{lgathered}
    \n{\defp{e}} \defd \dn \bigl(\n{\defp{e}} \fuse \n{\defp{b}_1} \pmir \atmL{i}) \with (\atmR{z} \pmir \atmL{d})\bigr)
    \\
    \n{\defp{b}_0} \defd \dn \bigl((\up \dn \n{\defp{b}_1} \pmir \atmL{i}) \with (\atmL{d} \fuse \n{\defp{b}'_0{}} \pmir \atmL{d})\bigr)
    \\
    \n{\defp{b}_1} \defd \dn \bigl((\atmL{i} \fuse \n{\defp{b}_1} \pmir \atmL{i}) \with (\n{\defp{b}_0} \fuse \atmR{s} \pmir \atmL{d})\bigr)
    \\
    \n{\defp{b}'_0{}} \defd \dn \bigl((\atmR{z} \limp \atmR{z}) \with (\atmR{s} \limp \n{\defp{b}_1} \fuse \atmR{s}\bigr)
  \end{lgathered}
\end{equation*}


\begin{equation*}
  \!\begin{aligned}[b]
    \trproc{\spawn{P}{Q}} &= \trproc{P} \fuse \trproc{Q} \\
    \trproc{\fwd} &= \one \\
    \trproc{\selectR{\kay}} &= \atmR{\kay} \\
    \trproc{\selectL{\kay}} &= \atmL{\kay} \\
    \trproc{P} &= \dn \trproc[-]{P}
  \end{aligned}
  \quad
  \!\begin{aligned}[b]
    \trproc[-]{\defp{p}} &= \n{\defp{p}} \\
    \trproc[-]{\caseL[\ell \in L]{\ell => Q_{\ell}}} &= {\textstyle \bigwith_{\ell \in L}\bigl(\atmR{\ell} \limp \up \trproc{Q_{\ell}}\bigr)} \\
    \trproc[-]{\caseR[\ell \in L]{\ell => P_{\ell}}} &= {\textstyle \bigwith_{\ell \in L}\bigl(\up \trproc{P_{\ell}} \pmir \atmL{\ell}\bigr)} \\
    \trproc[-]{P} &= \up \trproc{P}
  \end{aligned}
\end{equation*}





\begin{equation*}
  \!\begin{aligned}[t]
    \trproc[+]{\spawn{P}{Q}} &= \trproc[+]{P} \fuse \trproc[+]{Q} \\
    \trproc[+]{\fwd} &= \one \\
    \trproc[+]{\selectR{\kay}} &= \atmR{\kay} \\
    \trproc[+]{\selectL{\kay}} &= \atmL{\kay} \\
    \trproc[+]{P} &= \dn \trproc[-]{P}
  \end{aligned}
  \qquad
  \!\begin{aligned}[t]
    \trproc[-]{\caseL[\ell \in L]{\ell => Q_{\ell}}} &= \bigwith_{\ell \in L}\bigl(\atmR{\ell} \limp \up \trproc[+]{Q_{\ell}}\bigr) \\
    \trproc[-]{\caseR[\ell \in L]{\ell => P_{\ell}}} &= \bigwith_{\ell \in L}\bigl(\up \trproc[+]{P_{\ell}} \pmir \atmL{\ell}\bigr) \\
    \trproc[-]{P} &= \up \trproc[+]{P}
  \end{aligned}
  \qquad
  \!\begin{aligned}[t]
    \trproc{P} &= \trproc[-]{\caseL[\ell \in L]{\ell => Q_{\ell}}} &= \bigwith_{\ell \in L}\bigl(\atmR{\ell} \limp \up \trproc[+]{Q_{\ell}}\bigr) \\
    \trproc[-]{\caseR[\ell \in L]{\ell => P_{\ell}}} &= \bigwith_{\ell \in L}\bigl(\up \trproc[+]{P_{\ell}} \pmir \atmL{\ell}\bigr) \\
    \trproc[-]{P} &= \up \trproc[+]{P}
  \end{aligned}
\end{equation*}



\begin{equation*}
  \!\begin{aligned}[t]
    \trproc[+]{\spawn{P}{Q}} &= \trproc[+]{P} \fuse \trproc[+]{Q} \\
    \trproc[+]{\fwd} &= \one \\
    \trproc[+]{\selectR{\kay}} &= \atmR{\kay} \\
    \trproc[+]{\selectL{\kay}} &= \atmL{\kay} \\
    \trproc[+]{P} &= \dn \trproc[-]{P}
  \end{aligned}
  \qquad
  \!\begin{aligned}[t]
    \trproc[-]{\caseL[\ell \in L]{\ell => Q_{\ell}}} &= \bigwith_{\ell \in L}\bigl(\atmR{\ell} \limp \up \trproc[+]{Q_{\ell}}\bigr) \\
    \trproc[-]{\caseR[\ell \in L]{\ell => P_{\ell}}} &= \bigwith_{\ell \in L}\bigl(\up \trproc[+]{P_{\ell}} \pmir \atmL{\ell}\bigr) \\
    \trproc[-]{P} &= \up \trproc[+]{P}
  \end{aligned}
\end{equation*}


\begin{theorem}
  $\trconf{}$ constitutes a (strong) reduction bisimulation.
  That is:
  \begin{itemize}[nosep]
  \item If $\cnf \reduces \cnf'$, then $\trconf{\cnf} \reduces \trconf{\cnf'}$.
  \item If $\trconf{\cnf} = \octx \reduces \octx'$, then $\cnf \reduces \cnf'$ for some $\cnf'$ such that $\trconf{\cnf'} = \octx'$.
  \end{itemize}
\end{theorem}

Notice that neither this \lcnamecref{??} nor a corresponding weak reduction bisimulation \lcnamecref{??} would hold if the unfocused form of ordered rewriting were used.%
\footnote{For example, $\trconf{\caseL[\ell \in L]{\ell => P_{\ell}}} = \bigwith_{\ell \in L}(\atmR{\ell} \limp \trproc{Q_{\ell}}) \reduces (\atmR{\kay} \limp \trproc{Q_{\kay}})$ if $\kay \in L$, but there is no configuration $\cnf'$ such that $\caseL[\ell \in L]{\ell => P_{\ell}} \Reduces \cnf'$ and $\trconf{\cnf'} = \atmR{\kay} \limp \trproc{Q_{\kay}}$.}


\section{A session type system for ordered rewriting}

\begin{inferences}
  \infer[\jrule{CUT}\smash{^B}]{\slof{A |- \p{A}_1 \fuse \p{A}_2 : C}}{
    \slof{A |- \p{A}_1 : B} & \slof{B |- \p{A}_2 : C}}
  \and
  \infer[\jrule{ID}\smash{^A}]{\slof{A |- \one : A}}{}
  \\
  \infer[\rrule{\plus}']{\slof{A_{\kay} |- \selectR{\kay} : \plus*[sub=_{\ell \in L}]{\ell:A_{\ell}}}}{
    \text{($\kay \in L$)}}
  \and
  \infer[\lrule{\plus}]{\slof{\plus*[sub=_{\ell \in L}]{\ell:A_{\ell}} |- \dn \bigwith_{\ell \in L} (\selectR{\ell} \limp \up \p{A}_{\ell}) : C}}{
    \multipremise{\ell \in L}{\slof{A_{\ell} |- \p{A}_{\ell} : C}}}
  \\
  \infer[\rrule{\with}]{\slof{A |- \dn \bigwith_{\ell \in L} (\p{A}_{\ell} \pmir \selectL{\ell}) : \with*[sub=_{\ell \in L}]{\ell:B_{\ell}}}}{
    \multipremise{\ell \in L}{\slof{A |- \p{A}_{\ell} : B_{\ell}}}}
  \and
  \infer[\lrule{\with},]{\slof{\with*[sub=_{\ell \in L}]{\ell:B_{\ell}} |- \selectL{\kay} : B_{\kay}}}{
    \text{($\kay \in L$)}}
  \\
  \infer[\jrule{C-CUT}\smash{^B}]{\slcof{A |- \octx_1 \oc \octx_2 : C}}{
    \slcof{A |- \octx_1 : B} & \slcof{B |- \octx_2 : C}}
  \and
  \infer[\jrule{C-ID}\smash{^A}]{\slcof{A |- \octxe : A}}{}
  \and
  \infer[\jrule{C-PROC}]{\slcof{A |- \p{A} : B}}{
    \slof{A |- \p{A} : B}}
\end{inferences}

\footnote{Careful with the 0-ary forms because you could end up with $\slof{\zero |- \top : C}$ and \emph{also} $\slof{A |- \top : \top}$.}

\begin{theorem}
  If $\slof{A |- P : B}$, then $\slof{A |- \trproc[+]{P} : B}$.
  Conversely, if $\slof{A |- \p{A} : B}$, then $\slof{A |- P : B}$ for some process $P$ such that $\trproc[+]{P} = \p{A}$.

  Similarly, if $\slcof{A |- \cnf : B}$, then $\slcof{A |- \trconf{\cnf} : B}$.
  Conversely, if $\slcof{A |- \octx : B}$, then $\slcof{A |- \cnf : B}$ for some configuration $\cnf$ such that $\trconf{\cnf} = \octx$.
\end{theorem}


\section{Examples}

\begin{equation*}
  \dfa{q} \defd \caseL[a \in \ialph]{a => \dfa{q}'_a | \emp => \dfa{F}(q)}
\end{equation*}
where $\dfa{F}(q) = \selectR{?}$ if $q \in F$ and $\dfa{F}(q) = \selectR{?}$ if $q \notin F$.

\begin{align*}
  \trproc[-]{\dfa{q}}
    &\defd \trproc[-]{\caseL[a \in \ialph]{a => \dfa{q}'_a | \emp => \dfa{F}(q)}} \\
    &= \bigwith_{a \in \ialph} (\atmR{a} \limp \up \trproc[+]{\dfa{q}'_a}) \with (\atmR{\emp} \limp \up \trproc[+]{\dfa{F}(q)}) \\
    &= \bigwith_{a \in \ialph} (\atmR{a} \limp \up \dn \trproc[-]{\dfa{q}'_a}) \with (\atmR{\emp} \limp \up \trproc[+]{\dfa{F}(q)})
\end{align*}


\begin{equation*}
  \begin{lgathered}
    e \defd (e \fuse b_1 \pmir \atmL{i}) \with (\atmR{z} \pmir \atmL{d})
  \end{lgathered}
\end{equation*}

\begin{equation*}
  \begin{lgathered}
    \trproc[-]{e} \defd \bigl(\up (\dn \trproc[-]{e} \fuse \dn \trproc[-]{b_1}) \pmir \atmL{i}\bigr) \with (\up \atmR{z} \pmir \atmL{d})
    \\
    \trproc[-]{b_0} \defd (\up \dn \trproc[-]{b_1} \pmir \atmL{i}) \with \bigl(\up (\atmL{d} \fuse \dn \trproc[-]{b'_0}) \pmir \atmL{d}\bigr)
    \\
    \trproc[-]{b_1} \defd \bigl(\up (\atmL{i} \fuse \dn \trproc[-]{b_0}) \pmir \atmL{i}\bigr) \with \bigl(\up (\dn \trproc[-]{b_0} \fuse \atmR{s}) \pmir \atmL{d}\bigr)
    \\
    \trproc[-]{b'_0} \defd (\atmR{z} \limp \up \atmR{z}) \with \bigl(\atmR{s} \limp \up (\dn \trproc[-]{b_1} \fuse \atmR{s})\bigr)
  \end{lgathered}
  \qquad
  \begin{lgathered}
    e \defd (e \fuse b_1 \pmir \atmL{i}) \with (\atmR{z} \pmir \atmL{d}) \\
    b_0 \defd (\up \dn b_1 \pmir \atmL{i}) \with (\atmL{d} \fuse b'_0 \pmir \atmL{d}) \\
    b_1 \defd (\atmL{i} \fuse b_0 \pmir \atmL{i}) \with (b_0 \fuse \atmR{s} \pmir \atmL{d}) \\
    b'_0 \defd (\atmR{z} \limp \atmR{z}) \with (\atmR{s} \limp b_1 \fuse \atmR{s})
  \end{lgathered}
\end{equation*}


\section{}

\begin{equation*}
  \left\lbag
    \infer{a \oc \dfa{q} \reduces \dfa{q}'_a}{
      (q \dfareduces[a] q'_a)}
  \right\rbag_{(q, a) \in Q \times \ialph}
  \qquad\text{and}\qquad
  \left\lbag
    \infer{\emp \oc \dfa{q} \reduces \dfa{F}(q)}{}
  \right\rbag_{q \in Q}
\end{equation*}

Either $\atmR{a} \oc \dfa{q}$ or $a \oc \atmL{\dfa{q}}$.

\begin{equation*}
  \begin{lgathered}
    a \defd \bigwith_{q \in Q} (\atmL{q}'_a \pmir \atmL{q}) \\
    \emp \defd \bigwith_{q \in Q} (\dfa{F}(q) \pmir \atmL{q})
  \end{lgathered}
\end{equation*}

Previously, we argued that the state-oriented encoding  required adequacy to be judged up to bisimilarity: $\atmR{a} \oc \dfa{q} \reduces \dfa{q}'$ did not imply $q \dfareduces[a] q'$ because the equivirecursive treatment of definitions means that the encoding is not injective.

In the symbol-oriented encoding, adequacy no longer needs to be judged up to bisimilarity: $a \oc \atmL{q} \reduces \atmL{q}'$ does indeed imply $q \dfareduces[a] q'$.
By inversion on the given rewriting, it suffices to show that $q \dfareduces[a] q'_a$ and $\atmL{q}'_a = \atmL{q}'$ together imply $q \dfareduces[a] q'$.
This time, because states are encoded as atoms, which have an uncomplicated notion of equality, the encoding is injective.

These differences in equality of atoms and recursively defined propositions should perhaps not be unexpected given the correspondence between atoms and messages; recursively defined propositions and processes.
Being observable, messages are easy to compare for equality.
But processes' internal structures are hidden, and therefore it shouldn't be possible to compare them for equality,

We could try to reverse engineer an equivalence on input symbols from the rewriting bisimilarity of their encodings.
The encodings of symbols $a$ and $b$ are bisimilar if, and only if:
\begin{itemize}
\item $q \dfareduces[a] q'_a$ implies $q \dfareduces[b] q'_a$, for all $q$ and $q'_a$; and
  % $q \dfareduces[a] q'_a$ implies $b \oc \atmL{q} \Reduces\lframe{\atmL{q}'_a}{\osim} \atmL{q}'_a$
  % which implies $b \oc \atmL{q} \Reduces \atmL{q}'_a$
  % which implies $q \dfareduces[b] q'_a$.
\item $q \dfareduces[b] q'_b$ implies $q \dfareduces[a] q'_b$, for all $q$ and $q'_b$.
  % and vice versa.
\end{itemize}
In other words, symbols $a$ and $b$ have bisimilar encodings exactly when those encodings are equal.


$\atmR{\emp} \oc \rev{\atmR{w}} \oc \dfa{q} \osim \emp \oc \rev{w} \oc \atmL{q}$ for all $w \in \finwds{\ialph}$ and $q \in Q$.

\begin{inferences}
  \infer{\atmR{\emp} \oc \dfa{q} \simu{R} \dfa{F}(q)}{}
  \and 
  \infer{\octx \oc \atmR{a} \oc \dfa{q} \simu{R} \p{A}}{
    \octx \oc \dfa{q}'_a \simu{R} \p{A} & \text{($q \dfareduces[a] q'_a$)}}
  \and
  \infer{\octx \simu{R} \octxe}{
    \octx \simu{R} \one}
\end{inferences}

\begin{inferences}
  \infer{\emp \oc \atmL{q} \simu{S} \dfa{F}(q)}{}
  \and 
  \infer{\lctx \oc a \oc \atmL{q} \simu{S} \p{A}}{
    \lctx \oc \atmL{q}'_a \simu{S} \p{A} & \text{($q \dfareduces[a] q'_a$)}}
  \and
  \infer{\octx \simu{S} \octxe}{
    \octx \simu{S} \one}
\end{inferences}

The relation $\simu{R}\simu{S}^{-1} \union \simu{R} \union \simu{S}^{-1}$ is a labled bisimulation up to reflexivity.

% If $q \dfareduces[a] q'_a$ and $\dfa{q}'_a = \dfa{q}'$, then $q \dfareduces[a] q'$.

% If $a \oc \atmL{q} \reduces \atmL{q}'$, then $q \dfareduces[a] q'$.

% If $q \dfareduces[a] q'_a$ and $\atmL{q}'_a = \atmL{q}'$, then $q \dfareduces[a] q'$.

% \begin{itemize}
% \item $q \dfareduces[a] q'$ if, and only if, $a \oc \atmL{q} \reduces \atmL{q}'$.
% \item $q \dfareduces[w] q'$ if, and only if, $\rev{w} \oc \atmL{q} \Reduces \atmL{q}'$.
% \item $q \in F$ if, and only if, $\emp \oc \atmL{q} \reduces \one$.
% \item If $q \asim s$, then $a \oc \atmL{q} \reduces \atmL{q}'_a$ implies $a \oc \atmL{s} \reduces \atmL{s}'_a$ for some $s'_a$ such that $q'_a \asim s'_a$.
% \end{itemize}

% $a \osim b$ if and only if $b \oc \atmL{q} \Reduces\osim \atmL{q}'_a$ for all $q$.

% $a \osim b$ if and only if $b \oc \atmL{q} \reduces \atmL{q}'_a$ for all $q$.

% $\octx \osim \octxe$ only if $\octx \Reduces \octxe$.
% $\octx \osim \octxe$ only if $\atmR{x} \oc \octx \Reduces\osim \atmR{x}$ only if $\atmR{x} \oc \octx \Reduces \octx' \oc \atmR{x}$ and $\octx' \osim \octxe$ only if $\octx \Reduces \octx' = \octxe$.

% $a \sim b$ if $q \dfareduces[a] q'_a$ if and only if $q \dfareduces[b] q'_a$ for all $q$.


\clearpage
\section{}

% \section{Open bisimilarity}

% \begin{falseclaim}
%   If $\slcof{A |- \cnf \tsim \dnf : B}$, then $\trconf{\cnf} \osim \trconf{\dnf}$.
% \end{falseclaim}

% Consider $\slcof{\plus*[sub=_{\ell \in L}]{\ell:A_{\ell}} |- \cnf = \cnfe : \plus*[sub=_{\ell \in L}]{\ell:A_{\ell}}}$ and $\slcof{\plus*[sub=_{\ell \in L}]{\ell:A_{\ell}} |- \dnf = \caseL[\ell \in L]{\ell => \selectR{\ell}} : \plus*[sub=_{\ell \in L}]{\ell:A_{\ell}}}$.
% We do not have $\octxe \osim \bigwith_{\ell \in L}(\selectR{\ell} \limp \selectR{\ell})$ because of atoms from outside $L$ -- for example, $\selectR{a} \oc \bigwith_{\ell \in L}(\selectR{\ell} \limp \selectR{\ell})$ can not emit $\selectR{a}$ when $a \notin L$.

% \begin{conjecture}
%   If $\slcof{A |- \cnf : B}$ and $\trconf{\cnf} \osim \octx$, then there exists a configuration $\dnf$ such that $\slcof{A |- \cnf \tsim \dnf : B}$ and $\trconf{\dnf} = \octx$.
% \end{conjecture}
% \begin{proof}[Proof idea]
%   By structural induction on $\cnf$, using the following.
%   \begin{quotation}
%     If $\slof{A |- P : B}$ and $\trproc{P} \osim \octx$, then there exists a process $Q$ such that $\slof{A |- P \tsim Q : B}$ and $\trproc{Q} \secudeR \octx$.
%   \end{quotation}

%   Suppose $\slof{A_{\ell} |- P_{\ell} ; B}$ and $\bigwith_{\ell \in L}(\selectR{\ell} \limp \trproc{P_{\ell}}) \osim \octx$; we must exibit a configuation $Q$ such that $\slof{\plus*[sub=_{\ell \in L}]{\ell;A_{\ell}} |- \caseL[\ell \in L]{\ell => P_{\ell}} \tsim Q : B}$ and $\trproc{Q} \secudeR \octx$.
%   By input bisimilarity, $\selectR{\ell} \oc \octx \Reduces\miso \trproc{P_{\ell}}$ for all $\ell \in L$.
%   By the inductive hypothesis, there exist processes $Q_{\ell}$ such that $\slof{A_{\ell} |- P_{\ell} \tsim Q_{\ell} : B}$ and $\trproc{Q_{\ell}} \secudeR \selectR{\ell} \oc \octx$.
%   Is $\slof{\plus*[sub=_{\ell \in L}]{\ell:A_{\ell}} |- \caseL[\ell \in L]{\ell => P_{\ell}} \tsim \caseL[\ell \in L]{\ell => Q_{\ell}} : B}$ true?
%   I think so. 
%   Is $\trproc{\caseL[\ell \in L]{\ell => Q_{\ell}}} = \bigwith_{\ell \in L}(\selectR{\ell} \limp \trproc{Q_{\ell}}) \secudeR \octx$ true?
% \end{proof}


% \begin{conjecture}
%   If $\slcof{A |- \cnf : B}$ and $\slcof{A |- \dnf : B}$ and $\trconf{\cnf} \osim \trconf{\dnf}$, then $\slcof{A |- \cnf \tsim \dnf : B}$.
% \end{conjecture}
% \begin{proof}[Proof idea]
%   By structural induction on $\cnf$, using the following.
%   \begin{quotation}
%     If $\slof{A |- P : B}$ and $\slof{A |- Q : B}$ and $\trproc{P} \osim \trproc{Q}$, then $\slof{A |- P \tsim Q : B}$.
%   \end{quotation}

%   Suppose that $P = \caseL[\ell \in L]{\ell => P_{\ell}}$ and $\slof{A_{\ell} |- P_{\ell} : B}$; we must show that $\slof{\plus*[sub=_{\ell \in L}]{\ell:A_{\ell}} |- \caseL[\ell \in L]{\ell => P_{\ell}} \tsim Q : B}$.
%   By input bisimilarity, we have $\selectR{\ell} \oc \trproc{Q} \Reduces \octx_{\ell} \miso \trproc{P_{\ell}}$.
%   I'm not sure how to continue from here.
% \end{proof}

%%% Local Variables:
%%% mode: latex
%%% TeX-master: "thesis"
%%% End:
