\chapter{From processes to rewriting}\label{ch:correspond}

\fixnote{Introduction}

\section{Embedding process configurations in formula-as-process ordered rewriting}\label{sec:correspond:embed}

Here we give an embedding, $\trconf{}$, of process configurations into ordered contexts from the formula-as-process rewriting framework of \cref{ch:formula-as-process}.
% More specifically, we first define two mutually inductive translations, $\trproc[+]{}$ and $\trproc[-]{}$, that map process expressions to positive and negative propositions, respectively, and then define a homomorphism, $\trconf{}$, that maps process configurations to ordered contexts.
For the embedding to be adequate, it must preserve the dynamics of processes -- $\trconf{}$ must serve as a bisimulation between the reduction semantics for process configurations and the formula-as-process rewriting semantics for ordered contexts.%
\footnote{Because they relate process configuration reductions to ordered rewritings, these bisimulations are not rewriting bisimulations of the kind introduced in \cref{ch:ordered-bisimilarity}.
  Instead, they are ordinary, unlabeled \enquote{reduction bisimulations}, using the terminology of \textcite{Sangiorgi+Walker:CUP03}.}

Ideally, the embedding $\trconf{}$ will be a \emph{strong} bisimulation, so that the correspondence is lockstep.
Because the embedding will be a total function, it will be a strong bisimulation if the diagrams
\begin{equation*}
  \begin{tikzcd}
    \cnf \rar[reduces] \dar[relation][swap]{\trconf{}} & \cnf\mathrlap{'} \dar[relation]{\trconf{}}
    \\
    \trconf{\cnf} \rar[reduces, exists] & \trconf{\cnf'}
  \end{tikzcd}
  \qquad\text{and}\qquad
  \begin{tikzcd}
    \cnf \rar[reduces, exists] \dar[relation][swap]{\trconf{}} & \cnf\mathrlap{'} \dar[relation, exists]{\trconf{}}
    \\
    \trconf{\cnf} \rar[reduces] & \octx\mathrlap{'}% \,.}
  \end{tikzcd}% \hphantom{'\,.}
\end{equation*}
hold.
But strong bisimulations are often elusive, so we will be satisfied to settle for a weak bisimulation if necessary.

In addition, the embedding $\trconf{}$ ought to be a monoid homomorphism from process configurations to ordered contexts, so we define
\begin{equation*}
  \begin{aligned}
    \trconf{\cnf_1 \cc \cnf_2} &= \trconf{\cnf_1} \oc \trconf{\cnf_2} \\
    \trconf{\cnfe} &= (\octxe)
  \end{aligned}
\end{equation*}
with a clause for $\trconf{P}$ that remains to be filled in.

By varying how that $\trconf{P}$ clause is filled in and whether we are working with weak focusing or full focusing (hereafter \emph{strong} focusing), we will arrive at three closely related embeddings: a weakly focused, strongly bisimilar embedding~\parencref{sec:embed:weak-focused-strong-bisim}; a strongly focused, strongly bisimilar embedding~\parencref{sec:embed:strong-focused-strong-bisim}; and a strongly focused, weakly bisimilar embedding~\parencref{sec:embed:strong-focused-weak-bisim}.
Of these three embeddings, we prefer the first and last because they most directly expose an appealing correspondence between process constructors and logical connectives.


\subsection{A weakly focused, strongly bisimilar embedding}\label{sec:embed:weak-focused-strong-bisim}


% Ideally, $\trconf{}$ ought to be a strong bisimulation, so that the operational correspondence is as tight as possible.
% Because the translation will be a total function, the strong bisimulation requirement amounts to the diagrams
% \begin{equation*}
%   \begin{tikzcd}
%     \cnf \rar[reduces] \dar[relation][swap]{\trconf{}} & \cnf\mathrlap{'} \dar[relation]{\trconf{}}
%     \\
%     \trconf{\cnf} \rar[reduces, exists] & \trconf{\cnf'}
%   \end{tikzcd}
%   \qquad\text{and}\qquad
%   \begin{tikzcd}
%     \cnf \rar[reduces, exists] \dar[relation][swap]{\trconf{}} & \cnf\mathrlap{'} \dar[relation, exists]{\trconf{}}
%     \\
%     \trconf{\cnf} \rar[reduces] & \octx\mathrlap{'\,.}
%   \end{tikzcd}\hphantom{'\,.}
% \end{equation*}

% \begin{equation*}
%   \begin{tikzcd}
%     \cnf \rar[reduces] \dar[relation][swap]{\trconf{}} & \cnf\mathrlap{'} \dar[relation]{\trconf{}}
%     \\
%     \trconf{\cnf} \rar[Reduces, exists] & \trconf{\cnf'}
%   \end{tikzcd}
%   \qquad\text{and}\qquad
%   \begin{tikzcd}
%     \cnf \rar[Reduces, exists] \dar[relation][swap]{\trconf{}} & \cnf\mathrlap{'} \dar[relation, exists]{\trconf{}}
%     \\
%     \trconf{\cnf} \rar[reduces] & \octx\mathrlap{'\,.}
%   \end{tikzcd}\hphantom{'\,.}
% \end{equation*}

For the first embedding, we will fill in the $\trconf{P}$ clause as
\begin{equation*}
  \trconf{P} = \trproc{P}
\end{equation*}
where $\trproc{}$ is an auxiliary function that embeds process expressions.
(For the moment, we will ignore coinductively defined process expressions because they introduce a few complications.)
% So that $\trconf{\cnf}$ is a well-formed ordered context in the formula-as-process focused ordered rewriting framework, $\trproc{}$ should map process expressions to either atoms or negative propositions.%
% \footnote{Recall from \cref{??} that formula-as-process ordered contexts may not contain non-atomic positive propositions.}

With this definition for $\trconf{}$ in hand, we can then run each process reduction axiom through the first bisimulation diagram to generate constraints on $\trproc{}$ that must be satisfied if $\trconf{}$ is to be a strong bisimulation.
These axioms and induced constraints are summarized in \cref{tbl:trconf-constraints}.%
%
\begin{table}[tb]
  \renewcommand{\arraystretch}{1.2}
  \begin{tabular}{@{}c@{\quad\ }c@{}}
    \toprule
    \emph{Process reduction} & \emph{Formula-as-process rewriting constraint}
    \\ \midrule
    $\spawn{P}{Q} \reduces P \cc Q$ & $\trproc{\spawn{P}{Q}} \dreduces \trproc{P} \oc \trproc{Q}$\hphantom{\quad($\kay \in L$)}
    \\
    $\fwd \reduces (\cnfe)$ & $\trproc{\fwd} \dreduces (\octxe)$\hphantom{\quad($\kay \in L$)}
    \\
    $\selectR{\kay} \cc \caseL[\ell \in L]{\ell => Q_{\ell}} \reduces Q_{\kay}$ & $\trproc{\selectR{\kay}} \oc \trproc{\caseL[\ell \in L]{\ell => Q_{\ell}}} \dreduces \trproc{Q_{\kay}}$\quad($\kay \in L$)
    \\
    $\caseR[\ell \in L]{\ell => P_{\ell}} \cc \selectL{\kay} \reduces P_{\kay}$ & $\trproc{\caseR[\ell \in L]{\ell => P_{\ell}}} \oc \trproc{\selectL{\kay}} \dreduces \trproc{P_{\kay}}$\quad($\kay \in L$)
    \\ \addlinespace \bottomrule
  \end{tabular}
  \caption{Constraints on $\trproc{}$ that must be satisfied if $\trconf{}$ is to be a strong bisimulation}\label{tbl:trconf-constraints}
\end{table}

For example, the process reduction axiom $\spawn{P}{Q} \reduces P \cc Q$ induces the constraint $\trproc{\spawn{P}{Q}} \dreduces \trproc{P} \oc \trproc{Q}$.
In other words, $\trproc{\spawn{P}{Q}}$ ought to decompose compositionally in a single step.


% \subsection{A weakly focused, strongly bisimilar embedding}

We usually presume that the formula-as-process ordered rewriting framework is strongly focused, as described in \cref{ch:formula-as-process}.
But suppose that we instead move to a \emph{weakly focused} framework, as sketched in \cref{sec:ordered-rewriting:weak-focusing}.
Then we can define 
\begin{equation*}
  \trproc{\spawn{P}{Q}} = \trproc{P} \fuse \trproc{Q}
  \quad\text{which satisfies}\quad
  \trproc{\spawn{P}{Q}} \dreduces \trproc{P} \oc \trproc{Q}
\end{equation*}
because the weak focusing discipline does not invert positive propositions eagerly.
\begin{marginfigure}
  \begin{equation*}
    \begin{tikzcd}
      \spawn{P}{Q} \arrow[reduces]{rrr} \dar[relation][swap]{\trconf{}}
        &[-1.85em] &&[-1.85em] P \cc Q \dar[relation]{\trconf{}}
      \\
      \trconf{\spawn{P}{Q}} \rar[phantom]{=}
        & \trproc{P} \fuse \trproc{Q} \rar[reduces]
        & \trproc{P} \oc \trproc{Q} \rar[phantom]{=}
        & \trconf{P \cc Q}
    \end{tikzcd}
  \end{equation*}
\end{marginfigure}%
% By similar reasoning, we might define
We can also define 
$  \trproc{\fwd} = \one$
%  \quad\text{
which similarly satisfies %}\quad
$  \trproc{\fwd} \dreduces (\octxe)$.
%  \,,
%\end{equation*}
% again because of weak focusing's treatment of positive propositions.
% \begin{marginfigure}
%   \begin{equation*}
%     \begin{tikzcd}
%       \fwd \arrow[reduces]{rrr} \dar[relation][swap]{\trconf{}}
%         &&& (\cnfe) \dar[relation]{\trconf{}}
%       \\
%       \trconf{\fwd} \rar[phantom]{=}
%         & \one \rar[reduces]
%         & (\octxe) \rar[phantom]{=}
%         & \trconf{\cnfe}
%     \end{tikzcd}
%   \end{equation*}
% \end{marginfigure}
% 
Notice that there is a clean and direct correspondence between the process constructor $\spawn{}{}$ and the logical connective $\fuse$ here; moreover, just as $\fwd$ is the unit of $\spawn{}{}$, so is its embedding the unit of the embedding of $\spawn{}{}$.

The constraints on $\trproc{\caseL[\ell \in L]{\ell => Q_{\ell}}}$ can be satisfied by defining
\begin{gather*}
  \trproc{\caseL[\ell \in L]{\ell => Q_{\ell}}}
    = \bigwith_{\ell \in L}\bigl(\trproc{\selectR{\ell}} \limp \up \trproc{Q_{\ell}}\bigr)
  % \intertext{which satisfies}
  % \trproc{\selectR{\kay}} \oc \trproc{\caseL[\ell \in L]{\ell => Q_{\ell}}}
  %   \reduces \trproc{Q_{\kay}}
  \,.
\end{gather*}
% for all $\kay \in L$.
Of course, because the formula-as-process framework restricts left- and right-handed implications to have atomic premises of complementary direction\parencref{sec:formula-as-process:interpretation}, this suggests that we also define
\begin{equation*}
  \trproc{\selectR{\kay}} = \atmR{\kay}
  \,,
\end{equation*}
for otherwise the above left-handed implication will not be a well-formed proposition.
% Indeed, 
% \begin{equation*}
%   \trproc{\selectR{\kay}} \oc \trproc{\caseL[\ell \in L]{\ell => Q_{\ell}}} = \atmR{\kay} \oc {\textstyle \bigl(\bigwith_{\ell \in L}(\atmR{\ell} \limp \up \trproc{Q_{\ell}})\bigr)}
%     \reduces \trproc{Q_{\kay}}
% \end{equation*}
% for all $\kay \in L$.
Symmetrically, we also define 
\begin{align*}
  \trproc{\caseR[\ell \in L]{\ell => P_{\ell}}} &= \bigwith_{\ell \in L}\bigl(\up \trproc{P_{\ell}} \pmir \atmL{\ell}\bigr) \\
  \trproc{\selectL{\kay}} &= \atmL{\kay}
  \,.
\end{align*}

\newthought{Coinductively defined processes} require a bit of extra care.
As described in \cref{sec:ordered-rewriting:weak-focusing}, the weakly focused ordered rewriting framework did not include coinductively defined propositions.
Here we assume that coinductively defined negative propositions, like those of the strongly focused formula-as-process ordered rewriting framework~\parencref{sec:formula-as-process:coinductive}, also exist for the weakly focused framework.%
\footnote{An argument can be made that the weakly focused framework should include coinductively defined propositions that are only \emph{positive}, not negative, in keeping with its partiality toward positive propositions in stable contexts.
  That is, however, at odds with the formula-as-process interpretation of coinductively defined propositions as coinductively defined processes, so we choose to have only negative defined propositions.
  Perhaps this small wrinkle could be ironed out if definitions were treated iso\-re\-cur\-sive\-ly, rather than equi\-re\-cur\-sive\-ly, but we choose not to pursue that here.}

To embed a process call, $\defp{p}$, in the weakly focused formula-as-process ordered rewriting framework, we will translate that process's definition to the definition of a coinductively defined negative proposition, $\n{\defp{p}}$.
Then the process call is simply embedded as 
\begin{equation*}
  \trproc{\defp{p}} = \dn \n{\defp{p}}
  \,.
\end{equation*}
But how should process \emph{definitions} be translated?

A first, natural attempt would be to map each process definition $\defp{p} \defd P$ to a negative proposition definition $\n{\defp{p}} \defd \up \trproc{P}$.
Unfortunately, this doesn't quite work.
It introduces a stutter in the unfolding and rewriting of defined processes like $\defp{p} \defd \caseL[\ell \in L]{\ell => P_{\ell}}$.
On the process side, we have the single-step
\begin{equation*}
  \selectR{\kay} \cc \defp{p} = \selectR{\kay} \cc \caseL[\ell \in L]{\ell => P_{\ell}} \reduces P_{\kay}
  \,.
\end{equation*}
But on the ordered rewriting side, it would take two steps to reach the same point:
\begin{equation*}
  \trconf{\selectR{\kay} \cc \defp{p}}
    = \atmR{\kay} \oc \dn \n{\defp{p}}
    = \atmR{\kay} \oc \dn \up \dn {\textstyle \bigwith_{\ell \in L}(\atmR{\ell} \limp \up \trproc{P_{\kay}})}
    \reduces \atmR{\kay} \oc \dn {\textstyle \bigwith_{\ell \in L}(\atmR{\ell} \limp \up \trproc{P_{\kay}})}
    \reduces \trproc{P_{\kay}}
  \,,
\end{equation*}
for all $\kay \in L$.
The extra step is caused by the $\up \dn$ double shift, with the $\up$ shift arising from the way we are trying to translate definitions, and with the $\dn$ shift arising from $\trproc{\caseL[\ell \in L]{\ell => P_{\ell}}}$.

Fortunately, this stuttering can be eliminated with a more careful translation of process definitions.
Instead of blindly inserting an $\up$ shift in front of $\trproc{P}$, let's construct a negative proposition by stripping off the outermost $\dn$ shift if $\trproc{P}$ has the form $\dn \n{A}$, and otherwise inserting an $\up$ shift:
\begin{align*}
  \trsig{\sig, (\defp{p} \defd P)}
    &= \trsig{\sig} , \left(\n{\defp{p}} \defd
         \begin{cases*}
           \n{A} & if $\trproc{P} = \dn \n{A}$ \\
           \up \trproc{P} & otherwise
         \end{cases*}
                            \right)
  \\
  \trconf{\sige} &= (\orsige)
\end{align*}
Now a defined process like $\defp{p} \defd \caseL[\ell \in L]{\ell => P_{\ell}}$ is translated as
\begin{equation*}
  \trsig{\defp{p} \defd \caseL[\ell \in L]{\ell => P_{\ell}}}
    = \bigl(\n{\defp{p}} \defd {\textstyle \bigwith_{\ell \in L}(\atmR{\ell} \limp \up \trproc{P_{\ell}})}\bigr)
  \,,
\end{equation*}
which does not induce the stutter:
\begin{equation*}
  \trconf{\selectR{\kay} \cc \defp{p}}
    = \atmR{\kay} \oc \dn \n{\defp{p}}
    = \atmR{\kay} \oc \dn {\textstyle \bigwith_{\ell \in L}(\atmR{\ell} \limp \up \trproc{P_{\kay}})}
    \reduces \trproc{P_{\kay}}
  \,,
\end{equation*}
for all $\kay \in L$.

\newthought{%
In total, the embedding} of process configurations and process expressions into a weakly focused formula-as-process ordered rewriting framework is summarized in \cref{fig:process-embedding:weak-focus-strong-bisim}.
\begin{marginfigure}
  \begin{align*}
    \trconf{\cnf_1 \cc \cnf_2} &= \trconf{\cnf_1} \oc \trconf{\cnf_2} \\
    \trconf{\cnfe} &= (\octxe) \\
    \trconf{P} &= \trproc{P}
  %
  \shortintertext{and}
  %
    \trsig{\sig, (\defp{p} \defd P)}
      &= \trsig{\sig} , \left(\n{\defp{p}} \defd
           \begin{cases*}
             \n{A} & if $\trproc{P} = \dn \n{A}$ \\
             \up \trproc{P} & otherwise
           \end{cases*}
         \right)
    \\
    \trconf{\sige} &= (\orsige)
  %
  \shortintertext{where}
  %
    \trproc{\spawn{P_1}{P_2}}
      &= \trproc{P_1} \fuse \trproc{P_2} \\
    \trproc{\fwd} &= \one
    \\
    \trproc{\selectR{\kay}} &= \atmR{\kay} \\
    \trproc{\caseL[\ell \in L]{\ell => P_{\ell}}}
      &= \dn {\textstyle \bigwith_{\ell \in L}(\atmR{\ell} \limp \up \trproc{P_{\ell}})}
    \\
    \trproc{\caseR[\ell \in L]{\ell => P_{\ell}}}
      &= \dn {\textstyle \bigwith_{\ell \in L}(\up \trproc{P_{\ell}} \pmir \atmL{\ell})} \\
    \trproc{\selectL{\kay}} &= \atmL{\kay}
    \\
    \trproc{\defp{p}} &= \dn \n{\defp{p}}
  \end{align*}
  \caption{A \emph{strongly} bisimilar embedding of process configurations within a \emph{weakly} focused formula-as-process ordered rewriting framework}\label{fig:process-embedding:weak-focus-strong-bisim}
\end{marginfigure}

Notice that, with the exception of a small wrinkle in the translation of process definitions, this embedding is quite clean and direct.
Especially appealing is the close correspondence between process constructors and logical connectives noted previously.
% The process constructor $\spawn{}{}$ for spawning a new process cleanly corresponds to the logical connective $\fuse$ for ordered conjunction.
% And, just as the forwarding process, $\fwd$, can be seen as the unit of $\spawn{}{}$, so is the embedding $\trproc{\fwd} = \one$ the unit of the $\fuse$ connective.
%
Moreover, the embedding is a \emph{strong} bisimulation:
%
\begin{theorem}[Adequacy of $\trconf{}$]
  Under the weakly focused formula-as-process ordered rewriting framework, $\trconf{}$ constitutes a strong bisimulation.
  That is:
  \begin{itemize}[nosep]
  \item If\/ $\cnf \reduces \cnf'$, then $\trconf{\cnf} \reduces \trconf{\cnf'}$.
  \item If\/ $\trconf{\cnf} \reduces \p{\octx'{}}$, then there exists a configuration $\cnf'$ such that $\cnf \reduces \cnf'$ and $\trconf{\cnf'} = \p{\octx'{}}$.
  \end{itemize}
\end{theorem}
\begin{proof}
  The first part is by induction on the process configuration reduction, $\cnf \reduces \cnf'$; the second part is by induction on the weakly focused formula-as-process ordered rewriting, $\trconf{\cnf} \reduces \p{\octx'{}}$.
\end{proof}

\subsection{A strongly focused, strongly bisimilar embedding}\label{sec:embed:strong-focused-strong-bisim}

One might object to the preceding embedding's reliance on weak focusing to achieve strong bisimilarity.
Needing to switch from strong focusing to the more exotic weak focusing is admittedly a blemish, but one that is hidden by the appealing correspondence of the process constructors with logical connectives. 
% $\spawn{}{}$ with $\fuse$ and $\fwd$ with $\one$.

If one insists upon strong focusing, there is nevertheless
% two avenues to a bisimilar embedding of process configurations.
a way forward.
% 
% First,
Recall from \cref{sec:ordered-rewriting:weak-focusing} that weak focusing can be embedded within strong focusing by the careful addition of shifts.
The $\embed*{}$ embedding described there could be composed with the $\trproc{}$ embedding of processes described above.
For example, the composed embedding of $\spawn{P_1}{P_2}$ would be
\begin{equation*}
  \embed{\trproc{\spawn{P_1}{P_2}}} = \up \bigl(\bigfuse (\embed{\trproc{P_1}}) \fuse \bigfuse (\embed{\trproc{P_2}})\bigr)
  \,,
\end{equation*}
which satisfies $\embed{\trproc{\spawn{P_1}{P_2}}} \reduces \embed{\trproc{P_1}} \oc \embed{\trproc{P_2}}$ in the strong focusing framework.
% because $\rfocus{\embed{\trproc{P_1}} \oc \embed{\trproc{P_2}}}{\bigfuse (\embed{\trproc{P_1}}) \fuse \bigfuse (\embed{\trproc{P_2}})}$ holds.
Using such a composition via $\trconf{P} = \embed{\trproc{P}}$ would turn $\trconf{}$ into a strongly bisimilar embedding of process configurations within \emph{strongly} focused formula-as-process ordered rewriting framework.
(For a definition of the embedding for process definitions, see \cref{fig:process-embedding:full-focus-strong-bisim}.)
%
\begin{marginfigure}
  \begin{align*}
    \trconf{\cnfe} &= (\octxe) \\
    \trconf{\cnf_1 \cc \cnf_2} &= \trconf{\cnf_1} \oc \trconf{\cnf_2} \\
    \trconf{P} &= \embed{\trproc{P}}
  %
  \shortintertext{and}
  %
    \trsig{\sig, (\defp{p} \defd P)}
      &= \trsig{\sig} , \left(\n{\defp{p}} \defd
           \begin{cases*}
             \embed{\trproc{P}} & if $P \neq \selectL{\kay}$ and $P \neq \selectR{\kay}$ \\
             \up \embed{\trproc{P}} & otherwise
           \end{cases*}
         \right)
    \\
    \trconf{\sige} &= (\orsige)
  % %
  % \shortintertext{where}
  % %
  %   \embed{\trproc{\spawn{P_1}{P_2}}}
  %     &= \up (\bigfuse \embed{\trproc{P_1}} \fuse \bigfuse \embed{\trproc{P_2}}) \\
  %   \embed{\trproc{\fwd}} &= \up \one
  %   \\
  %   \embed{\trproc{\selectR{\kay}}} &= \atmR{\kay} \\
  %   \embed{\trproc{\caseL[\ell \in L]{\ell => P_{\ell}}}}
  %     &= {\textstyle \bigwith_{\ell \in L}(\atmR{\ell} \limp \up \bigfuse \embed{\trproc{P_{\ell}}})}
  %   \\
  %   \embed{\trproc{\caseR[\ell \in L]{\ell => P_{\ell}}}}
  %     &= {\textstyle \bigwith_{\ell \in L}(\up \bigfuse \embed{\trproc{P_{\ell}}} \pmir \atmL{\ell})} \\
  %   \embed{\trproc{\selectL{\kay}}} &= \atmL{\kay}
  %   \\
  %   \embed{\trproc{\defp{p}}} &= \n{\defp{p}}
  \end{align*}
  where $\trproc{P}$ is defined exactly as in \cref{fig:process-embedding:weak-focus-strong-bisim}.
  %
  \vspace{.75\baselineskip}
  %
  \caption{A \emph{strongly} bisimilar embedding of process configurations within the \emph{strongly} focused formula-as-process ordered rewriting framework}\label{fig:process-embedding:full-focus-strong-bisim}
\end{marginfigure}%
%

Explicit weak focusing is not needed, as long as we are satisfied with a somewhat more complex embedding that obscures the correspondence between process constructors and logical connectives.
However, we do not dwell further on this embedding, instead viewing it as of secondary value because we do prefer the more direct correspondence.
Moreover, as we will now show, strong focusing does not exclude a more direct correspondence.

\subsection{A strongly focused, weakly bisimilar embedding}\label{sec:embed:strong-focused-weak-bisim}

If we abandon our desire for a strong bisimulation and content ourselves with a weak bisimulation that operates on the strongly focused rewriting framework, then we can, in fact, retain the direct and appealing correspondence with logical connectives.
Without changing the definition of $\trproc{}$ at all, let us define $\trconf{}$ by 
\begin{align*}
  \trconf{\cnf_1 \cc \cnf_2} &= \trconf{\cnf_1} \oc \trconf{\cnf_2} \\
  \trconf{\cnfe} &= (\octxe) \\
  \trconf{P} &= \octx \text{ where $\rfocus{\octx}{\trproc{P}}$}
  \,.
\end{align*}
%
\begin{marginfigure}
  \begin{align*}
    \trconf{\cnf_1 \cc \cnf_2} &= \trconf{\cnf_1} \oc \trconf{\cnf_2} \\
    \trconf{\cnfe} &= (\octxe) \\
    \trconf{P} &= \octx \text{ where $\rfocus{\octx}{\trproc{P}}$}
  \end{align*}
  and where $\trsig{\sig}$ and $\trproc{P}$ are defined exactly as in \cref{fig:process-embedding:weak-focus-strong-bisim}.
  %
  \vspace{.75\baselineskip}
  %
  \caption{A \emph{weakly} bisimilar embedding of process configurations within the \emph{strongly} focused formula-as-process ordered rewriting framework}\label{fig:process-embedding:strong-focus-weak-bisim}
\end{marginfigure}
%
Decomposing the process expression $\spawn{P}{Q}$ now takes no time at all in its embedded form:
\begin{equation*}
  \begin{tikzcd}[row sep=tiny]
    \spawn{P}{Q} \arrow[reduces]{rr}
      &[-2em] &[-2em] P \cc Q
    \\ & \text{and yet} & \\
    \trconf{\spawn{P}{Q}} \rar[phantom]{=}
      & \trconf{P} \oc \trconf{Q} \rar[phantom]{=}
      & \trconf{P \cc Q}
  \mathrlap{\,.}
  \end{tikzcd}
\end{equation*}
It is this mismatch that makes this version of $\trconf{}$ a weak, not strong, bisimulation.
%
\begin{theorem}
  Under the strongly focused formula-as-process ordered rewriting framework, $\trconf{}$ constitutes only a \emph{weak} bisimulation.
  That is:
  \begin{itemize}[nosep]
  \item
    If\/ $\cnf \reduces \cnf'$, then either $\trconf{\cnf} \reduces \trconf{\cnf'}$ or $\trconf{\cnf} = \trconf{\cnf'}$.
    More specifically, if $\cnf \reduces \cnf'$ arises from the receipt of a message, then $\trconf{\cnf} \reduces \trconf{\cnf'}$; otherwise, $\trconf{\cnf} = \trconf{\cnf'}$.
  \item
    If\/ $\trconf{\cnf} \reduces \octx'$, then there exists a configuration $\cnf'$ such that $\cnf \reduces^{+} \cnf'$ and $\trconf{\cnf'} = \octx'$.
  \end{itemize}
\end{theorem}
\begin{proof}
  The first part is by induction on the process configuration reduction, $\cnf \reduces \cnf'$; the second part is by induction on the strongly focused formula-as-process ordered rewriting, $\trconf{\cnf} \reduces \octx'$.
\end{proof}

By using the $\trproc{}$ function unchanged and only substituting the clause $\trconf{P} = \trproc{P}$ with $\trconf{P} = \octx$ where $\rfocus{\octx}{\trproc{P}}$, it is clear that this new version of the $\trconf{}$ bisimulation is weak only because positive propositions are now eagerly inverted.
That is to say, the weakness of the bisimulation is a purely administrative artifact that is unrelated to the main computational aspects.

% Also, by using the same $\trproc{}$ function here, we can preserve the appealing close correspondence of process constructors with logical connectives, in spite of the move to strong focusing.



% In the formula-as-process ordered rewriting framework with the eager inversion of positive propositions that its fully focused nature implies, we can define
% \begin{equation*}
%   \trproc{\spawn{P}{Q}} = \up (\bigfuse \trproc{P} \fuse \bigfuse \trproc{Q})
%     \reduces \trproc{P} \oc \trproc{Q}
% \end{equation*}
% because $\rfocus{\trproc{P}}{\bigfuse \trproc{P}}$ and, similarly, $\rfocus{\trproc{Q}}{\bigfuse \trproc{Q}}$.

% But this $\up (\bigfuse \octx_1 \fuse \bigfuse \octx_2)$ pattern looks quite familiar.
% In \cref{??}, it was used in the embedding of weakly focused ordered rewriting into fully focused ordered rewriting.
% What if, instead of using the fully focused formula-as-process rewriting framework, we were to move to a \emph{weakly} focused framework?
% Then we could simplify the definition of $\trproc{\spawn{P}{Q}}$ to merely
% \begin{equation*}
%   \trproc{\spawn{P}{Q}} = \trproc{P} \fuse \trproc{Q}
%   \,,
% \end{equation*}
% which indeed takes a complete step to decompose the conjunction: $\trproc{P} \fuse \trproc{Q} \reduces \trproc{P} \oc \trproc{Q}$.
% This As an embedding, weakly focused rewriting 

% \begin{equation*}
%   \trproc{\fwd} = \one
% \end{equation*}

% Observe that $\trproc{\selectR{\kay}} \oc \dn {\textstyle \bigwith_{\ell \in L}(\trproc{\selectR{\ell}} \limp \up \trproc{Q_{\ell}})} \reduces \trproc{Q_{\kay}}$ is a valid weakly focused formula-as-process rewriting, for all $\kay \in L$, if $\trproc{\selectR{\kay}}$ is a right-directed atom.
% \begin{align*}
%   \trproc{\selectR{\kay}} &= \atmR{\kay} \\
%   \trproc{\caseL[\ell \in L]{\ell => Q_{\ell}}} &= \dn {\textstyle \bigwith_{\ell \in L}(\atmR{\ell} \limp \up \trproc{Q_{\ell}})}
% %
% \intertext{and, symmetrically,}
% %
%   \trproc{\caseR[\ell \in L]{\ell => P_{\ell}}} &= \dn {\textstyle \bigwith_{\ell \in L}(\up \trproc{P_{\ell}} \pmir \atmL{\ell})} \\
%   \trproc{\selectL{\kay}} &= \atmL{\kay}
% \end{align*}

% The obvious candidate is to define
% \begin{equation*}
%   \trproc{\spawn{P}{Q}} = \trproc{P} \fuse \trproc{Q}
%   \,,
% \end{equation*}
% but there are two obstacles to such a definition.
% First, $\trproc{P} \fuse \trproc{Q}$ is not a non-atomic positive proposition.
% But, second and more troublingly, positive propositions are decomposed all at once during the focusing bipole that constitutes a rewriting step in the formula-as-process \emph{focused} ordered rewriting framework.
% This leads to a mismatch between the small-step process reductions and the big-step formula-as-process focused ordered rewriting, as shown in the adjacent \lcnamecref{fig:translation:focused-mismatch}.%
% %
% \begin{marginfigure}
%   \begin{equation*}
%     \begin{tikzcd}
%       \spawn{(\spawn{P}{Q})}{R} \rar[reduces] \arrow[relation]{dd}[swap]{\trconf{}}
%         & (\spawn{P}{Q}) \cc R \dar[relation]{\trconf{}}
%       \\
%         & (\trproc{P} \fuse \trproc{Q}) \oc \trproc{R} \dar[phantom]{\neq}
%       \\[-2ex]
%       (\trproc{P} \fuse \trproc{Q}) \fuse \trproc{R} \rar[reduces]
%         & (\trproc{P} \oc \trproc{Q}) \oc \trproc{R}
%     \end{tikzcd}
%   \end{equation*}
%   \caption{Mismatch between process reduction and big-step decomposition of positive propositions}\label{fig:translation:focused-mismatch}
% \end{marginfigure}%

% There are (at least) two solutions.
% We could abandon our ideal strong bisimulation and instead settle for a weak bisimulation.

% Another possible solution, and the one we pursue here, is to shift the rewriting framework from a fully focused framework to the \emph{weakly focused} formula-as-process ordered rewriting framework described in \cref{??}.
% With weak focusing, positive propositions are not decomposed eagerly in a big-step manner -- 
% \begin{gather*}
%   (\trproc{P} \fuse \trproc{Q}) \fuse \trproc{R} \nreduces (\trproc{P} \oc \trproc{Q}) \oc \trproc{R}
% \intertext{but rather}
%   (\trproc{P} \fuse \trproc{Q}) \fuse \trproc{R} \reduces (\trproc{P} \fuse \trproc{Q}) \oc \trproc{R}
% \end{gather*}
% %
% \begin{marginfigure}
%   \begin{equation*}
%     \begin{tikzcd}
%       \spawn{(\spawn{P}{Q})}{R} \rar[reduces] \dar[relation][swap]{\trconf{}}
%         & (\spawn{P}{Q}) \cc R \dar[relation]{\trconf{}}
%       \\
%       (\trproc{P} \fuse \trproc{Q}) \fuse \trproc{R} \rar[reduces]
%         & (\trproc{P} \fuse \trproc{Q}) \oc \trproc{R}
%     \end{tikzcd}
%   \end{equation*}
%   \caption{Mismatch between process reduction and big-step decomposition of positive propositions}\label{fig:translation:focused-mismatch}
% \end{marginfigure}%




% \begin{marginfigure}
%   \begin{align*}
%     \trconf{\cnfe} &= (\octxe) \\
%     \trconf{\cnf_1 \cc \cnf_2} &= \trconf{\cnf_1} \oc \trconf{\cnf_2} \\
%     \trconf{P} &= \trproc{P}
%   %
%   \shortintertext{where}
%   %
%     \trproc{\spawn{P_1}{P_2}}
%       &= \trproc{P_1} \oc \trproc{P_2} \\
%     \trproc{\fwd} &= \octxe
%     \\
%     \trproc{\selectR{\kay}} &= \atmR{\kay} \\
%     \trproc{\caseL[\ell \in L]{\ell => P_{\ell}}}
%       &= {\textstyle \bigwith_{\ell \in L}(\atmR{\ell} \limp \up \bigfuse \trproc{P_{\ell}})}
%     \\
%     \trproc{\caseR[\ell \in L]{\ell => P_{\ell}}}
%       &= {\textstyle \bigwith_{\ell \in L}(\up \bigfuse \trproc{P_{\ell}} \pmir \atmL{\ell})} \\
%     \trproc{\selectL{\kay}} &= \atmL{\kay}
%     \\
%     \trproc{\defp{p}} &= \n{\defp{p}}
%   \end{align*}
%   \caption{A \emph{weakly} bisimilar embedding of process configurations within \emph{fully} focused formula-as-process ordered rewriting}\label{fig:process-embedding:full-focus-weak-bisim}
% \end{marginfigure}



% \begin{equation*}
%   \begin{aligned}
%     \trconf{\cnf_1 \cc \cnf_2} &= \trconf{\cnf_1} \oc \trconf{\cnf_2} \\
%     \trconf{\cnfe} &= (\octxe) \\
%     \trconf{P} &= \octx \text{, where $\rfocus{\octx}{\trproc{P}}$}
%   \end{aligned}
% \end{equation*}





% % \begin{align*}
% %   \embed*{\bigfuse \trproc{\spawn{P_1}{P_2}}}
% %     &= \up (\bigfuse \embed*{\bigfuse \trproc{P_1}} \fuse \bigfuse \embed*{\bigfuse \trproc{P_2}}) \\
% %   \embed*{\bigfuse \trproc{\fwd}} &= \up \one
% %   \\
% %   \embed*{\bigfuse \trproc{\selectR{\kay}}} &= \atmR{\kay} \\
% %   \embed*{\bigfuse \trproc{\caseL[\ell \in L]{\ell => P_{\ell}}}} &= {\textstyle \bigwith_{\ell \in L}(\atmR{\ell} \limp \up \bigfuse \embed*{\bigfuse \trproc{P_{\ell}}})}
% %   \\
% %   \embed*{\bigfuse \trproc{\caseR[\ell \in L]{\ell => P_{\ell}}}} &= {\textstyle \bigwith_{\ell \in L}(\up \bigfuse \embed*{\bigfuse \trproc{P_{\ell}}} \pmir \atmL{\ell})} \\
% %   \embed*{\bigfuse \trproc{\selectL{\kay}}} &= \atmL{\kay}
% % \end{align*}

% \begin{equation*}
%   \begin{aligned}
%     \trconf{\octxe} &= (\octxe) \\
%     \trconf{\cnf_1 \cc \cnf_2} &= \trconf{\cnf_1} \oc \trconf{\cnf_2} \\
%     \trconf{P} &= \trproc[\fuse]{P}
%   \end{aligned}
%   \quad\text{and}\quad
%   \begin{aligned}
%     \trproc[\fuse]{\spawn{P_1}{P_2}}
%       &= \up (\bigfuse \trproc[\fuse]{P_1} \fuse \bigfuse \trproc[\fuse]{P_2}) \\
%     \trproc[\fuse]{\fwd} &= \up \one
%     \\
%     \trproc[\fuse]{\selectR{\kay}} &= \atmR{\kay} \\
%     \trproc[\fuse]{\caseL[\ell \in L]{\ell => P_{\ell}}} &= {\textstyle \bigwith_{\ell \in L}(\atmR{\ell} \limp \up \bigfuse \trproc[\fuse]{P_{\ell}})}
%     \\
%     \trproc[\fuse]{\caseR[\ell \in L]{\ell => P_{\ell}}} &= {\textstyle \bigwith_{\ell \in L}(\up \bigfuse \trproc[\fuse]{P_{\ell}} \pmir \atmL{\ell})} \\
%     \trproc[\fuse]{\selectL{\kay}} &= \atmL{\kay} \\
%     \trproc[\fuse]{\defp{p}} &= \n{\defp{p}}
%   \end{aligned}
% \end{equation*}

% \begin{theorem}
%   $\trproc[\fuse]{P} = \embed*{\trproc{P}}$
% \end{theorem}



\clearpage
\subsection{Examples and other comments}\label{sec:correspond:embed:examples}


In summary, we have three distinct embeddings of process expressions and configurations within focused formula-as-process ordered rewriting.
\begin{table}[tbp]
  \begin{tabular}{@{}lll@{}}
    \toprule
    \emph{Focusing} & \emph{Bisimilarity} & \emph{Key clause}
    \\ \midrule
    weakly focused & strongly bisimilar & $\trconf{P} = \trproc{P}$ \\
    strongly focused & strongly bisimilar & $\trconf{P} = \embed{\trproc{P}}$ \\
    strongly focused & weakly bisimilar & $\trconf{P} = \octx$ where $\rfocus{\octx}{\trproc{P}}$
    \\ \bottomrule
  \end{tabular}
  \caption{A summary of the process embeddings}\label{tab:embedding-summary}
\end{table}
Focusing, be it weak or strong, is essential to these embeddings.
If the unfocused form of ordered rewriting were used, no bisimulation along these lines appears to be possible, not even a weak one.
For instance, in a hypothetical unfocused embedding, we would likely have 
\begin{equation*}
  \trconf{\caseL[\ell \in L]{\ell => P_{\ell}}}
    = {\textstyle \bigwith_{\ell \in L}(\atmR{\ell} \limp \trproc{P_{\ell}})}
    \reduces \atmR{\kay} \limp \trproc{P_{\kay}}
\end{equation*}
if $\kay \in L$, but there is no configuration $\cnf'$ such that $\caseL[\ell \in L]{\ell => P_{\ell}} \Reduces \cnf'$ and $\trconf{\cnf'} = \atmR{\kay} \limp \trproc{P_{\kay}}$.

\newthought{As examples} of the preceding embeddings, let us revisit our two running examples: binary counters and \acp{DFA}.

% \paragraph*{Binary counter}
Recall from \cref{sec:process-chains:binary-counter} that a binary counter can be implemented by the coinductively defined processes
\begin{equation*}
  \sig = \begin{lgathered}[t]
           \bigl(e \defd \caseR{i => \spawn{e}{b_1} | d => \selectR{z}}\bigr) \,, \\
           \bigl(b_0 \defd \caseR{i => b_1 | d => \spawn{\selectL{d}}{b'_0}}\bigr) \,, \\
           \bigl(b_1 \defd \caseR{i => \spawn{\selectL{i}}{b_0} | d => \spawn{b_0}{\selectR{s}}}\bigr) \,, \\
           \bigl(b'_0 \defd \caseL{z => \selectR{z} | s => \spawn{b_1}{\selectR{s}}}\bigr)
         \end{lgathered}
\end{equation*}
What coinductively defined propositions arise from embedding these process definitions?

Under the strongly focused, weakly bisimilar embedding, the process definition for $b_0$ becomes the coinductively defined proposition given by
\begin{equation*}
  \trsig{b_0 \defd \caseR{i => b_1 | d => \spawn{\selectL{d}}{b'_0}}}
    = \bigl(\n{\defp{b}_0} \defd (\up \dn \n{\defp{b}_1} \pmir \atmL{i}) \with (\up (\atmL{d} \fuse \dn \n{\defp{b}'_0{}}) \pmir \atmL{d})\bigr)
  \,.
\end{equation*}
This defined proposition is \emph{exactly} the same as the one produced in \cref{sec:formula-as-process:counters-oo} as the object-oriented choreography of the binary counter's initial string rewriting specification.
(Here, for completeness, we have included even the minimally necessary shifts that we would usually elide.)
Even the $\up \dn$ double shift in front of $\n{\defp{b}_1}$ is correctly produced for free by the embedding.

The same is true of the other process definitions: embedding them yields exactly the same coinductively defined propositions as in the binary counter's object-oriented choreography shown in \cref{sec:formula-as-process:counters-oo}.
\begin{equation*}
  \trsig{\sig} =
  \orsig = \begin{lgathered}[t]
             \bigl(\n{\defp{e}} \defd (\up (\dn \n{\defp{e}} \fuse \dn \n{\defp{b}_1}) \pmir \atmL{i}) \with (\up \atmR{z} \pmir \atmL{d})\bigr) \,, \\
             \bigl(\n{\defp{b}_0} \defd (\up \dn \n{\defp{b}_1} \pmir \atmL{i}) \with (\up (\atmL{d} \fuse \dn \n{\defp{b}'_0{}}) \pmir \atmL{d})\bigr) \,, \\
             \bigl(\n{\defp{b}_1} \defd (\up (\atmL{i} \fuse \dn \n{\defp{b}_0}) \pmir \atmL{i}) \with (\up (\dn \n{\defp{b}_0} \fuse \atmR{s}) \pmir \atmL{d})\bigr) \,, \\
             \bigl(\n{\defp{b}'_0{}} \defd (\atmR{z} \limp \up \atmR{z}) \with (\atmR{s} \limp \up (\dn \n{\defp{b}_1} \fuse \atmR{s}))\bigr)
           \end{lgathered}
\end{equation*}
(Actually, the same definitions arise when using the weakly focused, strongly bisimilar embedding because it treats process definitions the same as the strongly focused, weakly bisimilar embedding does.)

Similarly, recall the functional-style process definitions for the binary counter from \cref{sec:formula-as-process:counters-functional}:
\begin{equation*}
  \sig' = \begin{lgathered}[t]
            \bigl(i \defd \caseL{e => \spawn{\selectR{e}}{\selectR{b}_1}
                               | b_0 => \selectR{b}_1
                               | b_1 => \spawn{i}{\selectR{b}_0}}\bigr) \,,
            \\
            \bigl(d \defd \caseL{e => \selectR{z}
                               | b_0 => \spawn{d}{b'_0}
                               | b_1 => \spawn{\selectR{b}_0}{\selectR{s}}}\bigr) \,,
            \\
            \bigl(b'_0 \defd \caseL{z => \selectR{z}
                                  | s => \spawn{\selectR{b}_1}{\selectR{s}}}\bigr)
\,.
          \end{lgathered}
\end{equation*}
By embedding definitions, we arrive at exactly the same coinductively defined propositions as in the functional choreography of the binary counter shown in \cref{sec:formula-as-process:counters-functional}:
\begin{equation*}
  \trsig{\sig'} =
  \orsig' = \begin{lgathered}[t]
              \bigl(\n{\defp{\imath}} \defd (\atmR{e} \limp \up (\atmR{e} \fuse \atmR{b}_1)) \with (\atmR{b}_0 \limp \up \atmR{b}_1) \with (\atmR{b}_1 \limp \up (\dn \n{\defp{\imath}} \fuse \atmR{b}_0)) \,, \\
              \bigl(\n{\defp{d}} \defd (\atmR{e} \limp \up \atmR{z}) \with (\atmR{b}_0 \limp \up (\dn \n{\defp{d}} \fuse \dn \n{\defp{b}_0})) \with (\atmR{b}_1 \limp \up (\atmR{b}_0 \fuse \atmR{s})) \,, \\
              \bigl(\n{\defp{b}'_0{}} \defd (\atmR{z} \limp \up \atmR{z}) \with (\atmR{s} \limp \up (\atmR{b}_1 \fuse \atmR{s}))\bigr)
\,.
            \end{lgathered}
\end{equation*}
(Once again, the same definitions arise from the weakly focused, strongly bisimilar embedding.)

% \paragraph*{\aclp*{DFA}}
The same holds true for \acp{DFA}.\fixnote{But process definitions for \acp{DFA} were not introduced -- only for sequential transducers!}
Embedding the \ac{DFA} process definitions from \cref{??} yields the same coinductively defined propositions as in the object-oriented and functional choreographies from \cref{??}:
\begin{align*}
  \trsig{q \defd \caseL[a \in \ialph]{a => q'_a | \eow => \selectR{F}(q)}}
    &= \bigl(
         \n{\defp{q}} \defd
           (\atmR{\eow} \limp \up \atmR{F}(q)) \with
           {\textstyle \bigwith_{a \in \ialph}(\atmR{a} \limp \up \dn \n{\defp{q}'_a{}})}
       \bigr)
 \shortintertext{and} 
  \trsig{a \defd \caseR[q \in Q]{q => \selectL{q}'_a}}
    &= \bigl(
         \n{\defp{a}} \defd {\textstyle \bigwith_{q \in Q}(\up \atmL{q}'_a \pmir \atmL{q})}
       \bigr)
  \\
  \trsig{\eow \defd \caseR[q \in Q]{q => \selectR{F}(q)}}
    &= \bigl(
         \n{\defp{\eow}} \defd {\textstyle \bigwith_{q \in Q}(\up \atmR{F}(q) \pmir \atmL{q})}
       \bigr)
  \,,
\end{align*}
respectively.


\begin{align*}
  \trsig{\defp{q} \defd \caseL[a \in \ialph]{a => \spawn{\defp{q}'}{\rev{\selectR{w}}}}}
    &= \bigl(
         \n{\defp{q}} \defd
           {\textstyle \bigwith_{a \in \ialph}(\atmR{a} \limp \up (\dn \n{\defp{q}'{}} \fuse \rev{\atmR{w}}))}
       \bigr)
 \shortintertext{and} 
  \trsig{\defp{a} \defd \caseR[q \in Q]{q => \spawn{\selectL{q}'}{\rev{\defp{w}{}}}}}
    &= \bigl(
         \n{\defp{a}} \defd {\textstyle \bigwith_{q \in Q}(\up (\atmL{q}' \fuse \rev{\defp{w}}) \pmir \atmL{q})}
       \bigr)
  \,,
\end{align*}




\clearpage
\section{A session type system for ordered rewriting}\label{sec:correspond:types}

The preceding embeddings describe bisimulations between process expressions and certain ordered propositions; process configurations and certain ordered contexts.
Because the formula-as-process ordered rewriting frameworks are untyped, these embeddings discard type information when translating process expressions and configurations.
But is that really necessary?
Can the bisimilarity witnessed by these embeddings be leveraged to engineer a session type system for formula-as-process ordered rewriting from the session type system for processes?

Consider, for example, the process expression $\spawn{\selectR{\kay}_1}{\selectL{\kay}_2}$.
Although syntactically well-formed, this process expression is not typable: to be typable, the type at the interface between $\selectR{\kay}_1$ and $\selectL{\kay}_2$ would be faced with the impossible task of simultaneously being both an internal and external choice.
Yet even this untypable process expression can be embedded:
\begin{equation*}
  \trproc{\spawn{\selectR{\kay}_1}{\selectL{\kay}_2}}
    = \atmR{\kay}_1 \fuse \atmL{\kay}_2
  \,.
\end{equation*}
But the question is, can the image of \emph{well-typed} process expressions under this embedding be characterized?

\newthought{The idea} is to engineer a session type system for ordered propositions, based on a judgment $\slof{A |- \p{A} : B}$, such that the following \lcnamecref{thm:embed:type-props} will hold.
%
\begin{restatable}[
  label=thm:embed:type-props
]{theorem}{thmembedtypeprops}
  $\slof{A |-_{\orsig} \p{A}_1 : B}$ if, and only if, there exist process definitions $\sig$ and a process expression $P$ such that $\slof{A |-_{\sig} P : B}$ and $\trsig{\sig} = \orsig$ and $\trproc{P} = \p{A}_1$.
\end{restatable}
%
\noindent
In the judgment $\slof{A |- \p{A}_1 : B}$ and this \lcnamecref{thm:embed:type-props}, $A$ and $B$ are (Curry--Howard interpretations of) singleton propositions that functions as session types, whereas $\p{A}_1$ is a formula-as-process positive ordered proposition that functions as an embedded process expression.

To construct such a type system, we can simply apply the embedding to session typing rules for process expressions.
For instance, the typing rule for a process expression $\spawn{P}{Q}$ suggests that ordered conjunctions $\p{A}_1 \fuse \p{A}_2$ be typed with a cut-like rule, as shown in the adjacent \lcnamecref{fig:embed-spawn-typing}.
\begin{marginfigure}
\begin{gather*}
  \left.\trproc{%
    \infer{\slof{A |- \spawn{P}{Q} : C}}{
      \slof{A |- P : B} & \slof{B |- Q : C}}%
  \right}
  \\\rightsquigarrow\\
  \infer{\slof{A |- \trproc{\spawn{P}{Q}} = \trproc{P} \fuse \trproc{Q} : C}}{
    \slof{A |- \trproc{P} : B} & \slof{B |- \trproc{Q} : C}}
\shortintertext{suggests}
  \infer[\jrule{CUT}\smash{^B}]{\slof{A |- \p{A}_1 \fuse \p{A}_2 : C}}{
    \slof{A |- \p{A}_1 : B} & \slof{B |- \p{A}_2 : C}}
\end{gather*}
\caption{The embedding suggests a session-typing rule for ordered conjunction.}\label{fig:embed-spawn-typing}
\end{marginfigure}%
%
By this procedure, we can also reverse-engineer session-typing rules for the other polarized positive propositions from the session-typing rules for process expressions.
We arrive at:
\begin{inferences}
  \infer[\jrule{CUT}\smash{^B}]{\slof{A |- \p{A}_1 \fuse \p{A}_2 : C}}{
    \slof{A |- \p{A}_1 : B} & \slof{B |- \p{A}_2 : C}}
  \and
  \infer[\jrule{ID}\smash{^A}]{\slof{A |- \one : A}}{}
  \\
  \infer[\rrule{\plus}']{\slof{A_{\kay} |- \selectR{\kay} : \plus*[sub=_{\ell \in L}]{\ell:A_{\ell}}}}{
    \text{($\kay \in L$)}}
  \and
  \infer[\lrule{\plus}]{\slof{\plus*[sub=_{\ell \in L}]{\ell:A_{\ell}} |- \dn {\textstyle \bigwith_{\ell \in L} (\selectR{\ell} \limp \up \p{A}_{\ell})} : C}}{
    \multipremise{\ell \in L}{\slof{A_{\ell} |- \p{A}_{\ell} : C}}}
  \\
  \infer[\rrule{\with}]{\slof{A |- \dn {\textstyle \bigwith_{\ell \in L} (\up \p{A}_{\ell} \pmir \selectL{\ell})} : \with*[sub=_{\ell \in L}]{\ell:B_{\ell}}}}{
    \multipremise{\ell \in L}{\slof{A |- \p{A}_{\ell} : B_{\ell}}}}
  \and
  \infer[\lrule{\with}']{\slof{\with*[sub=_{\ell \in L}]{\ell:B_{\ell}} |- \selectL{\kay} : B_{\kay}}}{
    \text{($\kay \in L$)}}
  \\
  \infer[]{\slof{A |-_{\orsig} \dn \n{\defp{p}} : B}}{
    \text{($\slof{A |- \n{\defp{p}} \defd \n{A}_0 : B}) \in \orsig$)}}
\end{inferences}

At first, these rules may be a bit startling.
It's especially surprising to see the proposition $\p{A}_1 \fuse \p{A}_2$ typed without splitting the context and the proposition $\one$ typed with a nonempty context to its left.
But once viewed from the formula-as-process perspective, these rules become quite natural: $\p{A}_1 \fuse \p{A}_2$ is a composition of process expressions and cut-as-composition is familiar; and $\one$ is a forwarding process, so using the identity rule is not that surprising after all.

Now we can prove the adequacy of these typing rules for positive propositions.
%
\thmembedtypeprops*
\begin{proof}
  From left to right, by structural induction on the typing derivation of $\slof{A |-_{\orsig} \p{A}_1 : B}$;
  from right to left, by structural induction on the typing derivation of $\slof{A |-_{\sig} P : B}$.
\end{proof}
%
This would be an extremely strong \lcnamecref{thm:embed:type-props} if not for the specificity of the $\lrule{\plus}$ and $\rrule{\with}$ rules above.
These rules do not assign a type to general negative propositions like $\n{A} \with \n{B}$.
Instead, only the restricted propositions $\dn {\textstyle \bigwith_{\ell \in L}(\up \p{A}_{\ell} \pmir \atmL{\ell})}$ and $\dn {\textstyle \bigwith_{\ell \in L}(\atmR{\ell} \limp \up \p{A}_{\ell})}$ are typable.
This is because we want these reverse-engineered typing rules to correspond to the image of the process expression typing rules.
In general, $\n{A} \with \n{B}$ describes a nondeterministic choice, something that is not present in the syntax of session-typed process expressions.%
\footnote{This suggests the possibility of introducing a well-typed form of nondeterministic choice.
  We discuss this idea further in \cref{sec:conclusion:nondeterminism}.}
Even so, the \lcnamecref{thm:embed:type-props} is quite strong.

As a final remark, we would like to point out that the $0$-ary forms of the above $\lrule{\plus}$ and $\rrule{\with}$ rules need careful consideration.
Because we identify a $0$-ary alternative conjunction with $\top$, the $0$-ary forms of these rules can lead to a system in which two different typings of $\top$ are possible: $\slof{\zero |- \top : C}$ and $\slof{A |- \top : \top}$.
For this reason, we disallow $0$-ary alternative conjunctions and require that the label set $L$ is nonempty.

\newthought{Coinductively defined propositions} can also be given session-typing rules.
These take the form of rules verifying that a collection of mutually coinductive definitions is well-typed.
The judgment is $\vdash_{\orsig'} \orsig$, where the definitions $\orsig$ are judged according to the definitions $\orsig'$.
The standard trick of tying the mutually recursive knot by using $\vdash_{\orsig} \orsig$ at the top level is used: $\orsig$ is well-typed if $\vdash_{\orsig} \orsig$ holds.
\begin{inferences}
  \infer{\vdash_{\orsig'} \orsig , (\slof{A |- \n{\defp{p}} \defd \n{A}_0 : B})}{
    \text{($\n{A}_0 \neq \up \p{B}_0$)} &
    \vdash_{\orsig'} \orsig &
    \slof{A |-_{\orsig'} \dn \n{A}_0 : B}}
  \and
  \infer{\vdash_{\orsig'} \orsig , (\slof{A |- \n{\defp{p}} \defd \up \p{A}_0 : B})}{
    \vdash_{\orsig'} \orsig &
    \slof{A |-_{\orsig'} \p{A}_0 : B}}
\end{inferences}
These rules can be derived by the same kind of reasoning as used for the above proposition typing rules.
In particular, the peculiarities of the embedding $\trsig{\sig}$ for process definitions explains why the above two rules distinguish cases on whether or not the definition's body begins with an $\up$ shift.

We may prove the following \lcnamecref{thm:embed:type-sig}.
\begin{theorem}\label{thm:embed:type-sig}
  $\vdash_{\orsig'} \orsig$ if, and only if, there exist process definitions $\sig$ and $\sig'$ such that $\vdash_{\sig'} \sig$ and $\trsig{\sig} = \orsig$ and $\trsig{\sig'} = \orsig'$.
\end{theorem}
\begin{proof}
  From left to right, by structural induction on the typing derivation of $\vdash_{\orsig'} \orsig$;
  from right to left, by structural induction on  $\vdash_{\sig'} \sig$.
\end{proof}


\newthought{Ordered contexts} can be given session-typing rules, too.
Once again, we can derive these rules by applying the embeddings to the configuration typing rules of \cref{sec:process-chains:typed-chains}.
However, the particulars of ordered contexts differ between the weakly focused and strongly focused frameworks: weakly focused contexts are built from positive propositions, whereas strongly focused contexts are built from negative propositions and positive atoms.
So we will actually have two sets of session-typing rules for ordered contexts, using the one that matches the style of focusing in effect.

For the weakly focused framework, the judgment will be $\slcof{A |-_{\orsig} \p{\octx} : B}$, where the definitions $\orsig$ are typically elided because they are passed from conclusion to premises unchanged.
In the weakly focused framework, the strongly bisimilar embedding yields the following three session-typing rules for ordered contexts.
\begin{inferences}
  \infer[\jrule{C-CUT}\smash{^B}]{\slcof{A |- \p{\octx}_1 \oc \p{\octx}_2 : C}}{
    \slcof{A |- \p{\octx}_1 : B} & \slcof{B |- \p{\octx}_2 : C}}
  \and
  \infer[\jrule{C-ID}\smash{^A}]{\slcof{A |- \octxe : A}}{}
  \\
  \infer[\jrule{C-PROC}]{\slcof{A |- \p{A}_1 : B}}{
    \slof{A |- \p{A}_1 : B}}
\end{inferences}
In particular, the final rule can be understood as arising from the weakly focused, strongly bisimilar embedding's clause $\trconf{P} = \trproc{P}$.

\begin{theorem}
  $\slcof{A |-_{\orsig} \p{\octx} : B}$ if, and only if, there exists a configuration $\cnf$ such that $\slcof{A |-_{\sig} \cnf : B}$ and $\trsig{\sig} = \orsig$ and $\trconf{\cnf} = \p{\octx}$.
\end{theorem}
\begin{proof}
  From left to right, by structural induction on the typing derivation of $\slcof{A |-_{\orsig} \p{\octx} : B}$;
  from right to left, by structural induction on the typing derivation of $\slcof{A |-_{\sig} \cnf : B}$.
\end{proof}

If the strongly focused framework is instead being used, then we will use a different set of session-typing rules for ordered contexts, owing to the fact that ordered contexts are now based on negative propositions and positive atoms, not positive propositions.
The strongly focused, weakly bisimilar embedding uses the clause $\trconf{P} = \octx$ where $\rfocus{\octx}{\trproc{P}}$.
Consequently, the following $\jrule{C-PROC}$ rule makes use of the right focus pattern judgment to invert a positive proposition to an ordered context.
\begin{inferences}
  \infer[\jrule{C-CUT}\smash{^B}]{\slcof{A |- \octx_1 \oc \octx_2 : C}}{
    \slcof{A |- \octx_1 : B} & \slcof{B |- \octx_2 : C}}
  \and
  \infer[\jrule{C-ID}\smash{^A}] {\slcof{A |- \octxe : A}}{}
  \\
  \infer[\jrule{C-PROC}]{\slcof{A |- \octx : B}}{
    \rfocus{\octx}{\p{A}} & \slof{A |- \p{A} : B}}
\end{inferences}
With these rules, we can prove a result of the now familiar form.
\begin{theorem}
  $\slcof{A |-_{\orsig} \octx : B}$ if, and only if, there exists a configuration $\cnf$ such that $\slcof{A |-_{\sig} \cnf : B}$ and $\trsig{\sig} = \orsig$ and $\trconf{\cnf} = \octx$.
\end{theorem}
\begin{proof}
  From left to right, by structural induction on the typing derivation of $\slcof{A |-_{\orsig} \octx : B}$;
  from right to left, by structural induction on $\slcof{A |-_{\sig} \cnf : B}$.
\end{proof}




\footnote{Careful with the 0-ary forms because you could end up with $\slof{\zero |- \top : C}$ and \emph{also} $\slof{A |- \top : \top}$.}


%%% Local Variables:
%%% mode: latex
%%% TeX-master: "thesis"
%%% End:
