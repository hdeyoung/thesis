\chapter{Weak normalization}

\section{}

\begin{equation*}
  \infer-[\jrule{N-CUT}]{\slof{A |- \nspawn{M}{N} : C}}{
    \slof{A |- M : B} & \slof{B |- N : C}}
\end{equation*}

\subsection{Queues as possibly empty lists}

\begin{syntax*}
  Normal terms & M,N &
    \begin{array}[t]{@{}l@{}}
      \spawn{N}{Q^{\plus}} \mid \caseL[\ell \in L]{\ell => N_{\ell}} \\
      \mathllap{\mid {}} \caseR[\ell \in L]{\ell => N_{\ell}} \mid \spawn{Q^{\with}}{N} \\
      \mathllap{\mid {}} \fwd
    \end{array}
  \\
  {Q}ueues & Q^{\plus} &
    \fwd \mid \spawn{Q^{\plus}}{\selectR{\kay}}
  \\
  & Q^{\with} &
    \fwd \mid \spawn{\selectL{\kay}}{Q^{\with}}
\end{syntax*}

\begin{description}
\item[Identity cuts]
There are several cases involving identity.
\begin{equation*}
  \begin{lgathered}
    \nspawn{\fwd}{N} = N \\
    \nspawn{M}{\fwd} = M \\
    \nspawn{(\spawn{M_0}{\fwd})}{N} = \nspawn{M_0}{N} \\
    \nspawn{M}{(\spawn{\fwd}{N_0})} = \nspawn{M}{N_0}
  \end{lgathered}
\end{equation*}

\item[Principal cuts]
In the principal cuts, the type becomes smaller.
\begin{equation*}
  \begin{lgathered}
    \nspawn{(\spawn{M_0}{(\spawn{Q^{\plus}}{\selectR{\kay}})})}
           {\caseL[\ell \in L]{\ell => N_{\ell}}}
      = \nspawn{(\spawn{M_0}{Q^{\plus}})}{N_{\kay}} \\
    \nspawn{\caseR[\ell \in L]{\ell => M_{\ell}}}
           {(\spawn{(\spawn{\selectL{\kay}}{Q^{\with}})}{N_0})}
      = \nspawn{M_{\kay}}{(\spawn{Q^{\with}}{N_0})}
  \end{lgathered}
\end{equation*}

\item[Left commutative cuts]
There are also several left commutative cuts:
\begin{equation*}
  \begin{lgathered}
    \nspawn{(\spawn{Q^{\with}}{M_0})}{N}
      = \spawn{Q^{\with}}{(\nspawn{M_0}{N})} \\
    \begin{aligned}[t]
      \MoveEqLeft[.75]
      \nspawn{\caseL[\ell \in L]{\ell => M_{\ell}}}
             {\caseL[\kay \in K]{\kay => N_{\kay}}} \\[-\jot]
        &= \caseL[\ell \in L]{\ell =>
             \nspawn{M_{\ell}}{\caseL[\kay \in K]{\kay => N_{\kay}}}}
      \\
      \MoveEqLeft[.75]
      \nspawn{\caseL[\ell \in L]{\ell => M_{\ell}}}
             {\caseR[\kay \in K]{\kay => N_{\kay}}} \\[-\jot]
        &= \caseL[\ell \in L]{\ell =>
             \nspawn{M_{\ell}}{\caseR[\kay \in K]{\kay => N_{\kay}}}}
    \end{aligned}
    \\
    \nspawn{\caseL[\ell \in L]{\ell => M_{\ell}}}
           {(\spawn{(\spawn{\selectL{\kay}}{Q^{\with}})}{N_0})}
      = \caseL[\ell \in L]{\ell =>
          \nspawn{M_{\ell}}{(\spawn{(\spawn{\selectL{\kay}}{Q^{\with}})}{N_0})}}
  \end{lgathered}
\end{equation*}
If we don't care about determinism, the final case could be replaced with
\begin{equation*}
  \nspawn{\caseL[\ell \in L]{\ell => M_{\ell}}}
         {(\spawn{Q^{\with}}{N_0})}
    = \caseL[\ell \in L]{\ell =>
        \nspawn{M_{\ell}}{(\spawn{Q^{\with}}{N_0})}}
\end{equation*}

\item[Right commutative cuts]
These cuts are symmetric to the left commutative cuts, so we omit a listing of them.
\end{description}

Also possible are the following.
\begin{equation*}
  \begin{lgathered}
    \nspawn{(\spawn{\fwd}{M_0})}{N} = \nspawn{M_0}{N} \\
    \nspawn{M}{(\spawn{N_0}{\fwd})} = \nspawn{M}{N_0}
  \end{lgathered}
\end{equation*}
These overlap with two of the commutative cases.

Notice that temination of this function can be established by a lexicographic induction, first on the structure of the principal type, and then simultaneously on the given proofs.

\NewDocumentCommand \wnorm { m } { \mathit{wn}(#1) }

We may normalize arbitrary proofs as follows.
\begin{equation*}
  \begin{lgathered}
    \wnorm{\spawn{P_1}{P_2}} = \nspawn{\wnorm{P_1}}{\wnorm{P_2}} \\
    \wnorm{\fwd} = \fwd
    \\
    \wnorm{\selectR{\kay}} = \spawn{\fwd}{(\spawn{\fwd}{\selectR{\kay}})} \\
    \wnorm{\caseL[\ell \in L]{\ell => P_{\ell}}}
      = \caseL[\ell \in L]{\ell => \wnorm{P_{\ell}}}
    \\
    \wnorm{\caseR[\ell \in L]{\ell => P_{\ell}}}
      = \caseR[\ell \in L]{\ell => \wnorm{P_{\ell}}} \\
    \wnorm{\selectL{\kay}} = \spawn{(\spawn{\selectL{\kay}}{\fwd})}{\fwd}
  \end{lgathered}
\end{equation*}
This weak normalization occurs by structural induction on the given proof.

\subsection{Queues as strictly nonempty lists}

\begin{syntax*}
  Normal terms & M,N &
    \begin{array}[t]{@{}l@{}}
      \spawn{N}{Q^{\plus}} \mid \caseL[\ell \in L]{\ell => N_{\ell}} \\
      \mathllap{\mid {}} \caseR[\ell \in L]{\ell => N_{\ell}} \mid \spawn{Q^{\with}}{N} \\
      \mathllap{\mid {}} \fwd
    \end{array}
  \\
  {Q}ueues & Q^{\plus} &
    \selectR{\kay} \mid \spawn{Q^{\plus}}{\selectR{\kay}}
  \\
  & Q^{\with} &
    \selectL{\kay} \mid \spawn{\selectL{\kay}}{Q^{\with}}
\end{syntax*}

Accordingly, two of the identity cuts no longer occur, but two more principal cuts are added.
\begin{equation*}
  \begin{lgathered}
    \nspawn{(\spawn{M_0}{\selectR{\kay}})}
           {\caseL[\ell \in L]{\ell => N_{\ell}}}
      = \nspawn{M_0}{N_{\kay}} \\
    \nspawn{\caseR[\ell \in L]{\ell => M_{\ell}}}
           {(\spawn{\selectL{\kay}}{N_0})}
      = \nspawn{M_{\kay}}{N_0}
  \end{lgathered}
\end{equation*}
A left commutative cut is also added, along with its symmetric right commutative counterpart.
\begin{equation*}
  \begin{lgathered}
  \nspawn{\caseL[\ell \in L]{\ell => M_{\ell}}}
         {(\spawn{\selectL{\kay}}{N_0})}
    = \caseL[\ell \in L]{\ell =>
        \nspawn{M_{\ell}}{(\spawn{\selectL{\kay}}{N_0})}} \\
  \nspawn{(\spawn{M_0}{\selectR{\kay}})}
         {\caseR[\ell \in L]{\ell => N_{\ell}}}
    = \caseR[\ell \in L]{\ell =>
        \nspawn{(\spawn{M_0}{\selectR{\kay}})}{N_{\ell}}}
  \end{lgathered}
\end{equation*}

We normalize proofs as before, with two revisions.
\begin{equation*}
  \begin{lgathered}
    \wnorm{\selectR{\kay}} = \spawn{\fwd}{\selectR{\kay}} \\
    \wnorm{\selectL{\kay}} = \spawn{\selectL{\kay}}{\fwd}
  \end{lgathered}
\end{equation*}

\subsection{Queues as normal terms, directly}

\begin{syntax*}
  Normal terms & M,N &
    \begin{array}[t]{@{}l@{}}
      \spawn{N}{Q^{\plus}} \mid \caseL[\ell \in L]{\ell => N_{\ell}} \\
      \mathllap{\mid {}} \caseR[\ell \in L]{\ell => N_{\ell}} \mid \spawn{Q^{\with}}{N} \\
      \mathllap{\mid {}} \fwd \mid Q^{\plus} \mid Q^{\with}
    \end{array}
\end{syntax*}

The advantage of this grammar is that weak normalization is slightly simplified:
\begin{equation*}
  \begin{lgathered}
    \wnorm{\selectR{\kay}} = \selectR{\kay} \\
    \wnorm{\selectL{\kay}} = \selectL{\kay}
  \end{lgathered}
\end{equation*}
The disadvantage is that several more cases must be added to $\nspawn{}{}$.

\section{}

\begin{syntax*}
  Normal terms & M,N &
    \begin{array}[t]{@{}l@{}}
      \spawn{N}{Q^{\plus}} \mid \caseL[\ell \in L]{\ell => N_{\ell}} \\
      \mathllap{\mid {}} \caseR[\ell \in L]{\ell => N_{\ell}} \mid \spawn{Q^{\with}}{N} \\
      \mathllap{\mid {}} \fwd
    \end{array}
  \\
  {Q}ueues & Q^{\plus} &
    \fwd \mid \spawn{Q^{\plus}_2}{Q^{\plus}_1} \mid \selectR{\kay}
  \\
  & Q^{\with} &
    \fwd \mid \spawn{Q^{\with}_1}{Q^{\with}_2} \mid \selectL{\kay}
\end{syntax*}

\NewDocumentCommand \qpspawn { o m m } { #2 \mathbin{\blacktriangleright} #3 }
\NewDocumentCommand \qwspawn { o m m } { #2 \mathbin{\blacktriangleleft} #3 }

\begin{inferences}
  \infer-[\jrule{N-CUT}]{\slof{A |- \nspawn{M}{N} : C}}{
    \slof{A |- M : B} & \slof{B |- N : C}}
  \\
  \infer-[\jrule{Q$^{\plus}$-CUT}]{\slof{A |- \qpspawn{Q^{\plus}}{N} : C}}{
    \slof{A |- Q^{\plus} : B} & \slof{B |- N : C}}
  \and
  \infer-[\jrule{Q$^{\with}$-CUT}]{\slof{A |- \qwspawn{M}{Q^{\with}} : C}}{
    \slof{A |- M : B} & \slof{B |- Q^{\with} : C}}
\end{inferences}

The cut between normal terms is defined by:
\begin{equation*}
  \begin{lgathered}
    \nspawn{\fwd}{N} = N \\
    \nspawn{M}{\fwd} = M \\
    \nspawn{(\spawn{M_0}{Q^{\plus}})}{N}
      = \nspawn{M_0}{(\qpspawn{Q^{\plus}}{N})} \\
    \nspawn{M}{(\spawn{Q^{\with}}{N_0})}
      = \nspawn{(\qwspawn{M}{Q^{\with}})}{N_0} \\
    \nspawn{(\spawn{Q^{\with}}{M_0})}{N}
      = \spawn{Q^{\with}}{(\nspawn{M_0}{N})} \\
    \nspawn{M}{(\spawn{N_0}{Q^{\plus}})}
      = \spawn{(\nspawn{M}{N_0})}{Q^{\plus}} \\
    \nspawn{\caseL[\ell \in L]{\ell => M_{\ell}}}
           {\caseL[\kay \in K]{\kay => N_{\kay}}} \\[-.75\jot]\quad
      = \caseL[\ell \in L]{\ell =>
          \nspawn{M_{\ell}}{\caseL[\kay \in K]{\kay => N_{\kay}}}} \\
    \nspawn{\caseL[\ell \in L]{\ell => M_{\ell}}}
           {\caseR[\kay \in K]{\kay => N_{\kay}}} \\[-.75\jot]\quad
      = \caseL[\ell \in L]{\ell =>
          \nspawn{M_{\ell}}{\caseR[\kay \in K]{\kay => N_{\kay}}}} \\
    \nspawn{\caseR[\ell \in L]{\ell => M_{\ell}}}
           {\caseR[\kay \in K]{\kay => N_{\kay}}} \\[-.75\jot]\quad
      = \caseR[\kay \in K]{\kay =>
          \nspawn{\caseR[\ell \in L]{\ell => M_{\ell}}}{N_{\kay}}}
  \end{lgathered}
\end{equation*}
We would then define the queue cuts as follows.
\begin{equation*}
  \begin{lgathered}
    \qpspawn{\fwd}{N} = N \\
    \qpspawn{(\spawn{Q^{\plus}_2}{Q^{\plus}_1})}{N}
      = \qpspawn{Q^{\plus}_2}{(\qpspawn{Q^{\plus}_1}{N})} \\
    \qpspawn{Q^{\plus}}{\fwd} = \spawn{\fwd}{Q^{\plus}} \\
    \qpspawn{\selectR{\kay}}{\caseL[\ell \in L]{\ell => N_{\ell}}} = N_{\kay} \\
    \qpspawn{Q^{\plus}}{(\spawn{N_0}{Q^{\plus}_0})}
      = \spawn{(\qpspawn{Q^{\plus}}{N_0})}{Q^{\plus}_0} \\
    \qpspawn{Q^{\plus}}{\caseR[\ell \in L]{\ell => N_{\ell}}}
      = \caseR[\ell \in L]{\ell => \qpspawn{Q^{\plus}}{N_{\ell}}} \\
  \end{lgathered}
\end{equation*}
and
\begin{equation*}
  \begin{lgathered}
    \qwspawn{M}{\fwd} = M \\
    \qwspawn{M}{(\spawn{Q^{\with}_1}{Q^{\with}_2})}
      = \qwspawn{(\qwspawn{M}{Q^{\with}_1})}{Q^{\with}_2} \\
    \qwspawn{\fwd}{Q^{\with}} = \spawn{Q^{\with}}{\fwd} \\
    \qwspawn{\caseR[\ell \in L]{\ell => M_{\ell}}}{\selectL{\kay}} = M_{\kay} \\
    \qwspawn{(\spawn{Q^{\with}_0}{M_0})}{Q^{\with}}
      = \spawn{Q^{\with}_0}{(\qwspawn{M_0}{Q^{\with}})} \\
    \qwspawn{\caseL[\ell \in L]{\ell => M_{\ell}}}{Q^{\with}}
      = \caseL[\ell \in L]{\ell => \qwspawn{M_{\ell}}{Q^{\with}}} \\
  \end{lgathered}
\end{equation*}
The problem with this approach comes in proving termination.  
In the two principal cases for $\nspawn{}{}$, the type does not necessarily get smaller.
However, if we rule out empty queues, this approach could work. 

$\wnorm{-}$ is defined as above.

\section{}

\begin{syntax*}
  Normal terms & M,N &
    \begin{array}[t]{@{}l@{}}
      \spawn{N}{\selectR{\kay}} \mid \caseL[\ell \in L]{\ell => N_{\ell}} \\
      \mathllap{\mid {}} \caseR[\ell \in L]{\ell => N_{\ell}} \mid \spawn{\selectL{\kay}}{N} \\
      \mathllap{\mid {}} \fwd
    \end{array}
\end{syntax*}

\begin{inferences}
  \infer-[\jrule{N-CUT}]{\slof{A |- \nspawn{M}{N} : C}}{
    \slof{A |- M : B} & \slof{B |- N : C}}
  \\
  \infer-[\jrule{$\plus$-CUT}]{\slof{A_{\kay} |- \qpspawn{\selectR{\kay}}{N} : C}}{
    \slof{\plus*[sub=_{\ell \in L}]{\ell:A_{\ell}} |- N : C} &
    \text{($\kay \in L$)}}
  \and
  \infer-[\jrule{$\with$-CUT}]{\slof{A |- \qwspawn{M}{\selectL{\kay}} : C_{\kay}}}{
    \slof{A |- M : \with*[sub=_{\ell \in L}]{\ell:C_{\ell}}} &
    \text{($\kay \in L$)}}
\end{inferences}

The cut between normal terms is defined by:
\begin{equation*}
  \begin{lgathered}
    \nspawn{\fwd}{N} = N \\
    \nspawn{M}{\fwd} = M \\
    \nspawn{(\spawn{M_0}{\selectR{\kay}})}{N}
      = \nspawn{M_0}{(\qpspawn{\selectR{\kay}}{N})} \\
    \nspawn{M}{(\spawn{\selectL{\kay}}{N_0})}
      = \nspawn{(\qwspawn{M}{\selectL{\kay}})}{N_0} \\
    \nspawn{(\spawn{\selectL{\kay}}{M_0})}{N}
      = \spawn{\selectL{\kay}}{(\nspawn{M_0}{N})} \\
    \nspawn{M}{(\spawn{N_0}{\selectR{\kay}})}
      = \spawn{(\nspawn{M}{N_0})}{\selectR{\kay}} \\
    \nspawn{\caseL[\ell \in L]{\ell => M_{\ell}}}
           {\caseL[\kay \in K]{\kay => N_{\kay}}} \\[-.75\jot]\quad
      = \caseL[\ell \in L]{\ell =>
          \nspawn{M_{\ell}}{\caseL[\kay \in K]{\kay => N_{\kay}}}} \\
    \nspawn{\caseL[\ell \in L]{\ell => M_{\ell}}}
           {\caseR[\kay \in K]{\kay => N_{\kay}}} \\[-.75\jot]\quad
      = \caseL[\ell \in L]{\ell =>
          \nspawn{M_{\ell}}{\caseR[\kay \in K]{\kay => N_{\kay}}}} \\
    \nspawn{\caseR[\ell \in L]{\ell => M_{\ell}}}
           {\caseR[\kay \in K]{\kay => N_{\kay}}} \\[-.75\jot]\quad
      = \caseR[\kay \in K]{\kay =>
          \nspawn{\caseR[\ell \in L]{\ell => M_{\ell}}}{N_{\kay}}}
  \end{lgathered}
\end{equation*}
We would then define the message cuts as follows.
\begin{equation*}
  \begin{lgathered}
    \qpspawn{\selectR{\kay}}{\fwd} = \spawn{\fwd}{\selectR{\kay}} \\
    \qpspawn{\selectR{\kay}}{\caseL[\ell \in L]{\ell => N_{\ell}}} = N_{\kay} \\
    \qpspawn{\selectR{\kay}}{(\spawn{N_0}{\selectR{\kay}_0})}
      = \spawn{(\qpspawn{\selectR{\kay}}{N_0})}{\selectR{\kay}_0} \\
    \qpspawn{\selectR{\kay}}{\caseR[\ell \in L]{\ell => N_{\ell}}}
      = \caseR[\ell \in L]{\ell => \qpspawn{\selectR{\kay}}{N_{\ell}}} \\
  \end{lgathered}
\end{equation*}
and
\begin{equation*}
  \begin{lgathered}
    \qwspawn{\fwd}{\selectL{\kay}} = \spawn{\selectL{\kay}}{\fwd} \\
    \qwspawn{\caseR[\ell \in L]{\ell => M_{\ell}}}{\selectL{\kay}} = M_{\kay} \\
    \qwspawn{(\spawn{\selectL{\kay}_0}{M_0})}{\selectL{\kay}}
      = \spawn{\selectL{\kay}_0}{(\qwspawn{M_0}{\selectL{\kay}})} \\
    \qwspawn{\caseL[\ell \in L]{\ell => M_{\ell}}}{\selectL{\kay}}
      = \caseL[\ell \in L]{\ell => \qwspawn{M_{\ell}}{\selectL{\kay}}}
  \end{lgathered}
\end{equation*}

$\wnorm{-}$ is defined as above.


\section{A pattern-based focused calculus for singleton logic}

\subsection{With atomic suspensions}

\begin{syntax*}
  % Propositions & A &
  %   \plus*[sub=_{\ell \in L}]{\ell:A_{\ell}} \mid
  %   \with*[sub=_{\ell \in L}]{\ell:A_{\ell}} \mid \p{p}
  % \\
  Positive propositions & \p{A} &
    \plus*[sub=_{\ell \in L}]{\ell:\p{A}_{\ell}} \mid \p{p} \mid \dn \n{A}
  \\
  Negative propositions & \n{A} &
    \with*[sub=_{\ell \in L}]{\ell:\n{A}_{\ell}} \mid \up \p{A} \mid \n{p}
  \\
  Antecedents & \delta & \n{A} \mid \psusp{\p{p}}
  \\
  Conse{q}uents & \gamma & \p{A} \mid \nsusp{\n{p}}
\end{syntax*}

\ExplSyntaxOn
\RenewDocumentCommand \slofp { >{ \SplitArgument {1} { ||- } } m } { \use_i:nn #1 \Vdash [ \use_ii:nn #1 ] }
\ExplSyntaxOff

\paragraph{Pattern judgments}
The meanings of the logical connectives are given by right- and left-pattern judgments for propositions.
The right-pattern judgment, $\slofp{\delta ||- \p{A}}$, is a total relation from positive propositions to antecedents.
\begin{inferences}
  \infer[\rrule{\plus}]{\slofp{\delta ||- \plus*[sub=_{\ell \in L}]{\ell:\p{A}_{\ell}}}}{
    \text{($\kay \in L$)} & \slofp{\delta ||- \p{A}_{\kay}}}
  \and
  \infer[\rrule{\dn}]{\slofp{\n{A} ||- \dn \n{A}}}{}
  \and
  \infer[\p{\jrule{ID}}]{\slofp{\psusp{\p{p}} ||- \p{p}}}{}
\end{inferences}
Dually, the left-pattern judgment, $\slofn{\n{A} ||- \gamma}$, is a total relation on negative propositions to consequents.
\begin{inferences}
  \infer[\lrule{\with}]{\slofn{\with*[sub=_{\ell \in L}]{\ell:A_{\ell}} ||- \gamma}}{
    \text{($\kay \in L$)} & \slofn{A_{\kay} ||- \gamma}}
  \and
  \infer[\lrule{\up}]{\slofn{\up \p{A} ||- \p{A}}}{}
  \and
  \infer[\n{\jrule{ID}}]{\slofn{\n{p} ||- \nsusp{\n{p}}}}{}
\end{inferences}

\paragraph{Stable se{q}uents}
We use the judgment $\slof{\delta |- \gamma}$ for stable sequents.
In a stable sequent, we may focus on either the right or left.
This is accomplished by filling in the holes of an appropriate pattern.
\begin{inferences}
  \infer[\rrule{+}]{\slof{\delta |- \p{A}}}{
    \slof{\delta |- \delta'} & \slofp{\delta' ||- \p{A}}}
  \and
  \infer[\lrule{-}]{\slof{\n{A} |- \gamma}}{
    \slofn{\n{A} ||- \gamma'} & \slof{\gamma' |- \gamma}}
\end{inferences}

\paragraph{Inversion seq\-uents}
Inversion may take place on the right and left with the judgments $\slof{\delta |- \delta'}$ and $\slof{\gamma' |- \gamma}$, respectively.
Invertible propositions may be established by considering all possible patterns; suspensions must be immediate.
\begin{inferences}
  \infer[\rrule{-}]{\slof{\delta |- \n{A}}}{
    \slofn{\n{A} ||- \gamma} \longrightarrow \slof{\delta |- \gamma}}
  \and
  \infer[\p{\jrule{SUSP}}]{\slof{\psusp{\p{p}} |- \psusp{\p{p}}}}{}
  \\
  \infer[\lrule{+}]{\slof{\p{A} |- \gamma}}{
    \slofp{\delta ||- \p{A}} \longrightarrow \slof{\delta |- \gamma}}
  \and
  \infer[\n{\jrule{SUSP}}]{\slof{\nsusp{\n{p}} |- \nsusp{\n{p}}}}{}
\end{inferences}
Notice that negative propositions are not invertible on the left, not even trivially; dually, positive propositions are not even trivially invertible on the right.

\subsection{Weakly focused}

\begin{syntax*}
  Positive propositions & \p{A} &
    \plus*[sub=_{\ell \in L}]{\ell:\p{A}_{\ell}} \mid \p{p} \mid \dn \n{A}
  \\
  Negative propositions & \n{A} &
    \with*[sub=_{\ell \in L}]{\ell:\n{A}_{\ell}} \mid \n{p} \mid \up \p{A}
  \\
  Negative antecedents & \n{\delta} & \n{A} \mid \psusp{\p{p}}
  \\
  Positive conse{q}uents & \p{\gamma} & \p{A} \mid \nsusp{\n{p}}
  \\
  Antecedents & \delta & \n{\delta} \mid \p{A}
  \\
  Conse{q}uents & \gamma & \p{\gamma} \mid \n{A}
\end{syntax*}

\ExplSyntaxOn
\RenewDocumentCommand \slofp { >{ \SplitArgument {1} { ||- } } m } { \use_i:nn #1 \Vdash [ \use_ii:nn #1 ] }
\ExplSyntaxOff

\paragraph{Pattern judgments}
The meanings of the logical connectives are given by right- and left-pattern judgments for propositions.
The right-pattern judgment, $\slofp{\n{\delta} ||- Q^{\plus} : \p{A}}$, is a total relation from positive propositions to negative antecedents.
\begin{inferences}
  \infer[\rrule{\plus}]{\slofp{\n{\delta} ||- \spawn{Q^{\plus}}{\selectR{\kay}} : \plus*[sub=_{\ell \in L}]{\ell:\p{A}_{\ell}}}}{
    \text{($\kay \in L$)} & \slofp{\n{\delta} ||- Q^{\plus} : \p{A}_{\kay}}}
  \\
  \infer[\rrule{\dn}]{\slofp{\n{A} ||- \mathsf{dnR} : \dn \n{A}}}{}
  \and
  \infer[\p{\jrule{ID}}]{\slofp{\psusp{\p{p}} ||- \closeR : \p{p}}}{}
\end{inferences}
Dually, the left-pattern judgment, $\slofn{\n{A} ||- Q^{\with} : \p{\gamma}}$, is a total relation on negative propositions to positive consequents.
\begin{inferences}
  \infer[\lrule{\with}]{\slofn{\with*[sub=_{\ell \in L}]{\ell:\n{A}_{\ell}} ||- \spawn{\selectL{\kay}}{Q^{\with}} : \p{\gamma}}}{
    \text{($\kay \in L$)} & \slofn{\n{A}_{\kay} ||- Q^{\with} : \p{\gamma}}}
  \\
  \infer[\lrule{\up}]{\slofn{\up \p{A} ||- \mathsf{upL} : \p{A}}}{}
  \and
  \infer[\n{\jrule{ID}}]{\slofn{\n{p} ||- \closeL : \nsusp{\n{p}}}}{}
\end{inferences}

\paragraph{Stable se{q}uents}
We use the judgment $\slof{\delta |- N : \gamma}$ for stable sequents.
In a stable sequent, we may focus on either the right or left.
This is accomplished by filling in the holes of an appropriate pattern.
\begin{inferences}
  \infer[\rrule{+}]{\slof{\delta |- \spawn{R}{Q^{\plus}} : \p{A}}}{
    \slof{\delta |- R : \n{\delta}_A} & \slofp{\n{\delta}_A ||- Q^{\plus} : \p{A}}}
  \and
  \infer[\lrule{-}]{\slof{\n{A} |- \spawn{Q^{\with}}{L} : \gamma}}{
    \slofn{\n{A} ||- Q^{\with} : \p{\gamma}_A} & \slof{\p{\gamma}_A |- L : \gamma}}
\end{inferences}
Because this is a weakly focused calculus, the invertible rules apply to stable sequents.
\begin{inferences}
  \infer[\lrule{\plus}]{\slof{\plus*[sub=_{\ell \in L}]{\ell:\p{A}_{\ell}} |- \caseL[\ell \in L]{\ell => N_{\ell}} : \gamma}}{
    \multipremise{\ell \in L}{\slof{\p{A}_{\ell} |- N_{\ell} : \gamma}}}
  \\
  \infer[\p{\eta}]{\slof{\p{p} |- \waitL{N} : \gamma}}{
    \slof{\psusp{\p{p}} |- N : \gamma}}
  \and
  \infer[\lrule{\dn}]{\slof{\dn \n{A} |- \mathsf{dnL}; N : \gamma}}{
    \slof{\n{A} |- N : \gamma}}
  \\
  \infer[\rrule{\with}]{\slof{\delta |- \caseR[\ell \in L]{\ell => N_{\ell}} : \with*[sub=_{\ell \in L}]{\ell:\n{A}_{\ell}}}}{
    \multipremise{\ell \in L}{\slof{\delta |- N_{\ell} : \n{A}_{\ell}}}}
  \\
  \infer[\n{\eta}]{\slof{\delta |- \waitR{N} : \n{p}}}{
    \slof{\delta |- N : \nsusp{\n{p}}}}
  \and
  \infer[\rrule{\up}]{\slof{\delta |- \mathsf{upR}; N : \up \p{A}}}{
    \slof{\delta |- N : \p{A}}}
\end{inferences}

\paragraph{Inversion seq\-uents}
Although this calculus is weakly focused, weak inversion sequents are still needed to ensure that the $\rrule{+}$ and $\lrule{-}$ rules behave correctly.
The judgments are $\slof{\delta |- R : \n{\delta}}$ and $\slof{\p{\gamma} |- L : \gamma}$, respectively.
Invertible propositions may be established by a stable sequent; suspensions must be immediate.
\begin{inferences}
  \infer[\rrule{-}]{\slof{\delta |- \iota_R(N) : \n{A}}}{
    \slof{\delta |- N : \n{A}}}
  \and
  \infer[\p{\jrule{SUSP}}]{\slof{\psusp{\p{p}} |- \fwd : \psusp{\p{p}}}}{}
  \\
  \infer[\lrule{+}]{\slof{\p{A} |- \iota_L(N) : \gamma}}{
    \slof{\p{A} |- L : \gamma}}
  \and
  \infer[\n{\jrule{SUSP}}]{\slof{\nsusp{\n{p}} |- \fwd : \nsusp{\n{p}}}}{}
\end{inferences}


\subsection{Admissibility of cut}

\NewDocumentCommand \nspawn { o m m } { #2 \mathbin{\blacklozenge} \IfValueT{#1}{^{#1}} #3 }
\NewDocumentCommand \qpspawn { o m m } { #2 \mathbin{\blacktriangleright} \IfValueT{#1}{^{#1}} #3 }
\NewDocumentCommand \qwspawn { o m m } { #2 \mathbin{\blacktriangleleft} \IfValueT{#1}{^{#1}} #3 }
\NewDocumentCommand \spawnp { o m m } { \llbracket#2\rrbracket #3 }
\NewDocumentCommand \spawnn { o m m } { \langle#2\rangle #3 }

\begin{inferences}
  \infer-[\p{\jrule{CUT}}]{\slof{\n{\delta} |- \qpspawn{Q^{\plus}}{N} : \gamma}}{
    \slofp{\n{\delta} ||- Q^{\plus} : \p{A}} & \slof{\p{A} |- N : \gamma}}
  \and
  \infer-[\n{\jrule{CUT}}]{\slof{\delta |- \qwspawn{N}{Q^{\with}} : \p{\gamma}}}{
    \slof{\delta |- N : \n{A}} & \slofn{\n{A} ||- Q^{\with} : \p{\gamma}}}
\end{inferences}

\begin{equation*}
  \begin{lgathered}
    \qpspawn{(\spawn{Q^{\plus}}{\selectR{\kay}})}{\caseL[\ell \in L]{\ell => N_{\ell}}}
      = \qpspawn{Q^{\plus}}{N_{\kay}}
    \\
    \qpspawn{\closeR}{(\waitL{N})} = N
    \\
    \qpspawn{\mathsf{dnR}}{(\mathsf{dnL}; N)} = N
    \\
    \qpspawn{Q^{\plus}}{\caseR[\ell \in L]{\ell => N_{\ell}}}
      = \caseR[\ell \in L]{\ell => \qpspawn{Q^{\plus}}{N_{\ell}}}
    \\
    \qpspawn{Q^{\plus}}{(\waitR{N})}
      = \waitR{(\qpspawn{Q^{\plus}}{N})}
    \\
    \qpspawn{Q^{\plus}}{(\mathsf{upR}; N)}
      = \mathsf{upR}; (\qpspawn{Q^{\plus}}{N})
    \\
    \qpspawn{Q^{\plus}_1}{(\spawn{\iota_R(N)}{Q^{\plus}_2})}
      = \spawn{\iota_R(\qpspawn{Q^{\plus}_1}{N})}{Q^{\plus}_2}
  \end{lgathered}
\end{equation*}



\begin{inferences}
  \infer-[\jrule{CUT}]{\slof{\delta |- \nspawn[A]{M}{N} : \gamma}}{
    \slof{\delta |- M : A} & \slof{A |- N : \gamma}}
\end{inferences}


\begin{equation*}
  \begin{lgathered}
    \nspawn{(\spawn{R}{Q^{\plus}})}{N}
      = \nspawn{R}{(\qpspawn{Q^{\plus}}{N})}
  \end{lgathered}
\end{equation*}

% \begin{gather*}
%   \infer-[\p{\jrule{CUT}}]{\slof{\delta |- \p{C}}}{
%     \infer[\rrule{+}]{\slof{\delta |- \p{A}}}{
%       \slof{\delta |- \delta'} & \slofp{\delta' ||- \p{A}}} &
%     \infer[\lrule{+}]{\slof{\p{A} |- \p{C}}}{
%       \slofp{\delta'' ||- \p{A}} \longrightarrow \slof{\delta'' |- \p{C}}}}
%   \\=\\
%   \infer-[\n{\jrule{CUT}}]{\slof{\delta |- \p{C}}}{
%     \slof{\delta |- \delta'} &
%     \infer{\slof{\delta' |- \p{C}}}{
%       \slofp{\delta' ||- \p{A}} & \slofp{\delta'' ||- \p{A}} \longrightarrow \slof{\delta'' |- \p{C}}}}
% \end{gather*}

% \begin{gather*}
%   \infer-[\n{\jrule{CUT}}]{\slof{\delta |- \p{C}}}{
%     \infer[\rrule{-}]{\slof{\delta |- \n{A}}}{
%       \slofn{\n{A} ||- \p{B}} \longrightarrow \slof{\delta |- \p{B}}} &
%     \infer[\lrule{-}]{\slof{\n{A} |- \p{C}}}{
%       \slofn{\n{A} ||- \p{C}_0} & \slof{\p{C}_0 |- \p{C}}}}
%   \\=\\
%   \infer-[\p{\jrule{CUT}}]{\slof{\delta |- \p{C}}}{
%     \infer{\slof{\delta |- \p{C}_0}}{
%       \slofn{\n{A} ||- \p{C}_0} & \slofn{\n{A} ||- \p{B}} \longrightarrow \slof{\delta |- \p{B}}} &
%     \slof{\p{C}_0 |- \p{C}}}
% \end{gather*}

% \begin{gather*}
%   \infer-[\n{\jrule{CUT}}]{\slof{\delta |- \p{C}}}{
%     \infer[\p{\jrule{SUSP}}]{\slof{\psusp{\p{p}} |- \psusp{\p{p}}}}{} &
%     \slof{\psusp{\p{p}} |- \p{C}}}
%   \\=\\
%   \slof{\psusp{\p{p}} |- \p{C}}
% \end{gather*}

% \begin{gather*}
%   \infer-[\p{\jrule{CUT}}]{\slof{\n{B} |- \p{C}}}{
%     \infer[\lrule{-}]{\slof{\n{B} |- \p{A}}}{
%       \slofn{\n{B} ||- \p{A}_0} & \slof{\p{A}_0 |- \p{A}}} &
%     \infer[\lrule{+}]{\slof{\p{A} |- \p{C}}}{
%       \slofp{\delta'' ||- \p{A}} \longrightarrow \slof{\delta'' |- \p{C}}}}
%   \\=\\
%   \infer-[\n{\jrule{CUT}}]{\slof{\delta |- \p{C}}}{
%     \slof{\delta |- \delta'} &
%     \infer{\slof{\delta' |- \p{C}}}{
%       \slofp{\delta' ||- \p{A}} & \slofp{\delta'' ||- \p{A}} \longrightarrow \slof{\delta'' |- \p{C}}}}
% \end{gather*}

\subsection{Substitution}

\begin{equation*}
  \infer-[\p{\jrule{SUBST}}]{\slof{\n{\delta} |- [Q^{\plus}](\spawn{\fwd}{Q^{\plus}_0}) : \gamma}}{
    \slofp{\n{\delta} ||- Q^{\plus} : \p{A}} &
    \infer[\rrule{+}]{\slof{\psusp{\p{A}} |- \spawn{\fwd}{Q^{\plus}_0} : \p{C}}}{
      \infer[\p{\jrule{SUSP}}]{\slof{\psusp{\p{A}} |- \fwd : \psusp{\p{A}}}}{} &
      \slofp{\psusp{\p{A}} ||- Q^{\plus}_0 : \p{C}}}}
  =
  ?
\end{equation*}

\subsection{Identity expansion}

\begin{inferences}
  \infer-[\p{\eta}]{\slof{\p{A} |- \eta^{\p{A}}(N) : \gamma}}{
    \slof{\psusp{\p{A}} |- N : \gamma}}
  \and
  \infer-[\n{\eta}]{\slof{\delta |- \eta^{\n{A}}(N) : \n{A}}}{
    \slof{\delta |- N : \nsusp{\n{A}}}}
\end{inferences}

\begin{equation*}
  \infer-[\p{\eta}]{\slof{\plus*[sub=_{\ell \in L}]{\ell:\p{A}_{\ell}} |- \eta^{\plus}(N) : \gamma}}{
    \slof{\psusp{\plus*[sub=_{\ell \in L}]{\ell:\p{A}_{\ell}}} |- N : \gamma}}
  =
  \infer[\lrule{\plus}]{\slof{\plus*[sub=_{\ell \in L}]{\ell:\p{A}_{\ell}} |- \caseL[\ell \in L]{\ell => \eta^{\p{A}_{\ell}}([\spawn{\closeR}{\selectR{\ell}}]N)} : \gamma}}{
    \multipremise{\ell \in L}{
      \infer-[\p{\eta}]{\slof{\p{A}_{\ell} |- \eta^{\p{A}_{\ell}}([\spawn{\closeR}{\selectR{\ell}}]N) : \gamma}}{
        \infer-[\p{\jrule{SUBST}}]{\slof{\psusp{\p{A}_{\ell}} |- [\spawn{\closeR}{\selectR{\ell}}]N : \gamma}}{
          \infer[\rrule{\plus}]{\slofp{\psusp{\p{A}_{\ell}} ||- \spawn{\closeR}{\selectR{\ell}} : \plus*[sub=_{\ell' \in L}]{\ell':\p{A}_{\ell'}}}}{
            \text{($\ell \in L$)} &
            \infer[\p{\jrule{ID}}]{\slofp{\psusp{\p{A}_{\ell}} ||- \closeR : \p{A}_{\ell}}}{}} &
          \slof{\psusp{\plus*[sub=_{\ell' \in L}]{\ell':\p{A}_{\ell'}}} |- N : \gamma}}}}}
\end{equation*}

\begin{equation*}
  \infer-[\p{\eta}]{\slof{\p{p} |- \eta^{\p{p}}(N) : \gamma}}{
    \slof{\psusp{\p{p}} |- N : \gamma}}
  =
  \infer[\p{\eta}]{\slof{\p{p} |- \waitL{N} : \gamma}}{
    \slof{\psusp{\p{p}} |- N : \gamma}}
\end{equation*}

\begin{equation*}
  \infer-[\p{\eta}]{\slof{\dn \n{A} |- \eta^{\dn \n{A}}(N) : \gamma}}{
    \slof{\psusp{\dn \n{A}} |- N : \gamma}}
  =
  \infer[\lrule{\dn}]{\slof{\dn \n{A} |- \mathsf{dnL}; [\mathsf{dnR}]N : \gamma}}{
    \infer-[\p{\jrule{SUBST}}]{\slof{\n{A} |- [\mathsf{dnR}]N : \gamma}}{
      \infer[\rrule{\dn}]{\slofp{\n{A} ||- \mathsf{dnR} : \dn \n{A}}}{} &
      \slof{\psusp{\dn \n{A}} |- N : \gamma}}}
\end{equation*}

\section{}

\subsection{Toward weak normalization: Completeness of focusing}

\begin{equation*}
  \infer[\lrule{\plus}]{\slof{\plus*[sub=_{\ell \in L}]{\ell:A_{\ell}} |- \caseL[\ell \in L]{\ell => P_{\ell}} : C}}{
    \multipremise{\ell \in L}{\slof{A_{\ell} |- P : C}}}
\end{equation*}

\begin{equation*}
  \infer[\rrule{\plus}]{\slof{A |- \selectR{\kay}[P] : \plus*[sub=_{\ell \in L}]{\ell:C_{\ell}}}}{
    \text{($\kay \in L$)} & \slof{A |- P : C_{\kay}}}
\end{equation*}

\begin{equation*}
  \infer-{\slof{A |- W(\selectR{\kay}[P]) : \plus*[sub=_{\ell \in L}]{\ell:C_{\ell}}}}{
    \infer[\rrule{\plus}]{\slof{A |- \selectR{\kay}[P] : \plus*[sub=_{\ell \in L}]{\ell:C_{\ell}}}}{
      \text{($\kay \in L$)} & \slof{A |- P : C_{\kay}}}}
  =
  \infer[\rrule{\plus}]{\slof{A |- \selectR{\kay}[P] : \plus*[sub=_{\ell \in L}]{\ell:C_{\ell}}}}{
    \text{($\kay \in L$)} &
    \infer-{\slof{A |- W(P) : C_{\kay}}}{
      \slof{A |- P : C_{\kay}}}}
\end{equation*}



\section{}

\subsection{Weakly focused, with atomic suspensions}

\begin{syntax*}
  Propositions & A &
    \plus*[sub=_{\ell \in L}]{\ell:A_{\ell}} \mid
    \with*[sub=_{\ell \in L}]{\ell:A_{\ell}} \mid \p{p}
  \\
  Positive propositions & \p{A} &
    \plus*[sub=_{\ell \in L}]{\ell:A_{\ell}} \mid \p{p}
  \\
  Negative propositions & \n{A} &
    \with*[sub=_{\ell \in L}]{\ell:A_{\ell}}
  \\
  Antecedents & \n{\delta} & \n{A} \mid \psusp{\p{p}}
  \\
  Weak antecedents & \delta & A \mid \psusp{\p{p}}
\end{syntax*}

\ExplSyntaxOn
\RenewDocumentCommand \slofp { >{ \SplitArgument {1} { ||- } } m } { \use_i:nn #1 \Vdash [ \use_ii:nn #1 ] }
\ExplSyntaxOff

\paragraph{Pattern judgments}
The meanings of the logical connectives are given by right- and left-pattern judgments for propositions.
The right-pattern judgment for propositions, $\slofp{\n{\delta} ||- A}$, is a total relation from propositions to antecedents.
\begin{inferences}
  \infer[\rrule{\plus}]{\slofp{\n{\delta} ||- \spawn{Q^{\plus}}{\selectR{\kay}} : \plus*[sub=_{\ell \in L}]{\ell:A_{\ell}}}}{
    \text{($\kay \in L$)} & \slofp{\n{\delta} ||- Q^{\plus} : A_{\kay}}}
  \\
  \infer[\jrule{R-BLUR}]{\slofp{\n{A} ||- \fwd : \n{A}}}{}
  \and
  \infer[\p{\jrule{ID}}]{\slofp{\psusp{\p{p}} ||- \closeR : \p{p}}}{}
\end{inferences}
Dually, the left-pattern judgment for propositions, $\slofn{A ||- \p{C}}$, is a total binary relation on propositions.
\begin{inferences}
  \infer[\lrule{\with}]{\slofn{\with*[sub=_{\ell \in L}]{\ell:A_{\ell}} ||- \spawn{\selectL{\kay}}{Q^{\with}} : \p{C}}}{
    \text{($\kay \in L$)} & \slofn{A_{\kay} ||- Q^{\with} : \p{C}}}
  \and
  \infer[\jrule{L-BLUR}]{\slofn{\p{A} ||- \fwd : \p{A}}}{}
\end{inferences}

\paragraph{Stable se{q}uents}
We use the judgment $\slof{\delta |- N : A}$ for stable sequents.
In a stable sequent, we may focus on either the right or left.
This is accomplished by filling in the holes of an appropriate pattern.
\begin{inferences}
  \infer[\rrule{+}]{\slof{\delta |- \spawn{I}{Q^{\plus}} : \p{A}}}{
    \slof{\delta |- I : \n{\delta}} & \slofp{\n{\delta} ||- Q^{\plus} : \p{A}}}
  \and
  \infer[\lrule{-}]{\slof{\n{A} |- \spawn{Q^{\with}}{N} : C}}{
    \slofn{\n{A} ||- Q^{\with} : \p{B}} & \slof{\p{B} |- N : C}}
  \\
  \infer[\lrule{\plus}]{\slof{\plus*[sub=_{\ell \in L}]{\ell:A_{\ell}} |- \caseL[\ell \in L]{\ell => N_{\ell}} : C}}{
    \multipremise{\ell \in L}{\slof{A_{\ell} |- N_{\ell} : C}}}
  \and
  \infer[\p{\eta}]{\slof{\p{p} |- \waitL{N} : C}}{
    \slof{\psusp{\p{p}} |- N : C}}
  \\
  \infer[\rrule{\with}]{\slof{\delta |- \caseR[\ell \in L]{\ell => N_{\ell}} : \with*[sub=_{\ell \in L}]{\ell:A_{\ell}}}}{
    \multipremise{\ell \in L}{\slof{\delta |- N_{\ell} : A_{\ell}}}}
\end{inferences}

\paragraph{Inversion seq\-uents}
Although this is a weakly focused calculus, we still need a so-called right-inversion judgment, $\slof{\delta |- I : \n{\delta}}$.
Otherwise, focusing on $\p{p}$ would incorrectly (?) allow invertible rules to be moved inside the focusing phase.
\begin{inferences}
  \infer[\rrule{-}]{\slof{\delta |- \iota(N) : \n{A}}}{
    \slof{\delta |- N : \n{A}}}
  \and
  \infer[\psusp{}]{\slof{\psusp{\p{p}} |- \fwd : \psusp{\p{p}}}}{}
\end{inferences}


\subsection{Weakly focused, with general suspensions}

Except for atomic propositions, our propositions are left unpolarized.
\begin{syntax*}
  Propositions & A &
    \begin{array}[t]{@{}l@{}}
      \plus*[sub=_{\ell \in L}]{\ell:A_{\ell}} \mid \p{p} \\
      \mathllap{\mid {}}
        \with*[sub=_{\ell \in L}]{\ell:A_{\ell}} \mid \n{p}
    \end{array}
\end{syntax*}
We may identify the positive and negative propositions as those with a positive or negative connective, respectively, at the top level.
\begin{syntax*}
  Positive propositions & \p{A} &
    \plus*[sub=_{\ell \in L}]{\ell:A_{\ell}} \mid \p{p}
  \\
  Negative propositions & \n{A} &
    \with*[sub=_{\ell \in L}]{\ell:A_{\ell}} \mid \n{p}
\end{syntax*}
Notice that polarity is shallow and not inherited by the immediate subformulas.

Building on propositions, we also have antecedents and consequents.
\begin{syntax*}
  Antecedents & \n{\delta} & \n{A} \mid \psusp{\p{A}}
  \\
  Consequents & \p{\gamma} & \p{A} \mid \nsusp{\n{A}}
  \\
  Weak antecedents & \delta & A \mid \psusp{\p{A}}
  \\
  Weak consequents & \gamma & A \mid \nsusp{\n{A}}
\end{syntax*}

\paragraph{Pattern judgments}
The meanings of the logical connectives are given by right- and left-pattern judgments for propositions.
The right-pattern judgment for propositions, $\slofp{\n{\delta} ||- Q^{\plus} : A}$, is a total relation from propositions to antecedents.
\begin{inferences}
  \infer[\rrule{\plus}]{\slofp{\n{\delta} ||- \spawn{Q^{\plus}}{\selectR{\kay}} : \plus*[sub=_{\ell \in L}]{\ell:A_{\ell}}}}{
    \text{($\kay \in L$)} & \slofp{\n{\delta} ||- Q^{\plus} : A_{\kay}}}
  \\
  \infer[\jrule{R-BLUR}]{\slofp{\n{A} ||- \fwd : \n{A}}}{}
  \and
  \infer[\p{\jrule{ID}}]{\slofp{\psusp{\p{A}} ||- \closeR : \p{A}}}{}
\end{inferences}
Dually, the left-pattern judgment for propositions, $\slofn{A ||- Q^{\with} : \p{\gamma}}$, is a total binary relation on propositions.
\begin{inferences}
  \infer[\lrule{\with}]{\slofn{\with*[sub=_{\ell \in L}]{\ell:A_{\ell}} ||- \spawn{\selectL{\kay}}{Q^{\with}} : \p{\gamma}}}{
    \text{($\kay \in L$)} & \slofn{A_{\kay} ||- Q^{\with} : \p{\gamma}}}
  \\
  \infer[\jrule{L-BLUR}]{\slofn{\p{A} ||- \fwd : \p{A}}}{}
  \and
  \infer[\n{\jrule{ID}}]{\slofn{\n{A} ||- \closeL : \nsusp{\n{A}}}}{}
\end{inferences}

\paragraph{Stable se{q}uents}
We use the judgment $\slof{\delta |- N : \gamma}$ for stable sequents.
In a stable sequent, we may focus on either the right or left.
This is accomplished by filling in the holes of an appropriate pattern.
\begin{inferences}
  \infer[\rrule{+}]{\slof{\delta |- \spawn{R}{Q^{\plus}} : \p{A}}}{
    \slof{\delta |- R : \n{\delta}} & \slofp{\n{\delta} ||- Q^{\plus} : \p{A}}}
  \and
  \infer[\lrule{-}]{\slof{\n{A} |- \spawn{Q^{\with}}{L} : \gamma}}{
    \slofn{\n{A} ||- Q^{\with} : \p{\gamma}} & \slof{\p{\gamma} |- L : \gamma}}
  \\
  \infer[\lrule{\plus}]{\slof{\plus*[sub=_{\ell \in L}]{\ell:A_{\ell}} |- \caseL[\ell \in L]{\ell => N_{\ell}} : \gamma}}{
    \multipremise{\ell \in L}{\slof{A_{\ell} |- N_{\ell} : \gamma}}}
  \and
  \infer[\p{\eta}]{\slof{\p{p} |- \waitL{N} : \gamma}}{
    \slof{\psusp{\p{p}} |- N : \gamma}}
  \\
  \infer[\rrule{\with}]{\slof{\delta |- \caseR[\ell \in L]{\ell => N_{\ell}} : \with*[sub=_{\ell \in L}]{\ell:A_{\ell}}}}{
    \multipremise{\ell \in L}{\slof{\delta |- N_{\ell} : A_{\ell}}}}
  \and
  \infer[\n{\eta}]{\slof{\delta |- \waitR{N} : \n{p}}}{
    \slof{\delta |- N : \nsusp{\n{p}}}}
\end{inferences}

\paragraph{Inversion seq\-uents}
Although this is a weakly focused calculus, we still need so-called right- and left-inversion judgments, $\slof{\delta |- R : \n{\delta}}$ and $\slof{\p{\gamma} |- L : \gamma}$.
Otherwise, focusing on $\p{p}$ would incorrectly (?) allow invertible rules to be moved inside the focusing phase.
\begin{inferences}
  \infer[\rrule{-}]{\slof{\delta |- \rho(N) : \n{A}}}{
    \slof{\delta |- N : \n{A}}}
  \and
  \infer[\p{\jrule{SUSP}}]{\slof{\psusp{\p{A}} |- \fwd : \psusp{\p{A}}}}{}
  \\
  \infer[\lrule{+}]{\slof{\p{A} |- \lambda(N) : \gamma}}{
    \slof{\p{A} |- N : \gamma}}
  \and
  \infer[\n{\jrule{SUSP}}]{\slof{\nsusp{\n{A}} |- \fwd : \nsusp{\n{A}}}}{}
\end{inferences}

\subsection{Admissibility of cut}

\NewDocumentCommand \nspawn { o m m } { #2 \mathbin{\blacktriangleright} #3 }
\NewDocumentCommand \rsubst { o m m } { [#2] #3 }

\begin{equation*}
  \infer-[\jrule{CUT}]{\slof{\delta |- \nspawn{M}{N} : \gamma}}{
    \slof{\delta |- M : A} & \slof{A |- N : \gamma}}
\end{equation*}

\begin{equation*}
  \infer-[\jrule{R-SUBST}]{\slof{\n{\delta} |- \rsubst{M}{N} : \gamma}}{
    \slofp{\n{\delta} ||- Q^{\plus} : A} & \slof{A |- N : \gamma}}
\end{equation*}

\begin{equation*}
  \infer-[\jrule{CUT}]{\slof{\delta |- \nspawn{(\spawn{R}{Q^{\plus}})}{N} : \gamma}}{
    \infer[\rrule{+}]{\slof{\delta |- \spawn{R}{Q^{\plus}} : \p{A}}}{
      \slof{\delta |- R : \n{\delta}} & \slofp{\n{\delta} ||- Q^{\plus} : \p{A}}} &
    \slof{\p{A} |- N : \gamma}}
\end{equation*}



\subsection{Queue substitution}

We have the following substitution principle for suspensions of positive propositions.
\begin{inferences}
  \infer-{\slof{\n{\delta}_{\p{A}} |- [Q^{\plus}]N : \gamma}}{
    \slofp{\n{\delta}_{\p{A}} ||- Q^{\plus} : \p{A}} & \slof{\psusp{\p{A}} |- N : \gamma}}
\end{inferences}

We may attempt to prove the admissibility of this principle by structural induction on the normal term $N$.
There are three possible cases -- one for each of the $\rrule{+}$, $\rrule{\with}$, and $\n{\eta}$ rules.
For the case involving $\rrule{\with}$, the queue substitution commutes with the $\mathsf{caseR}$:
\begin{gather*}
  \infer-{\slof{\n{\delta}_{\p{A}} |- [Q^{\plus}]\caseR[\ell \in L]{\ell => N_{\ell}} : \with*[sub=_{\ell \in L}]{\ell:C_{\ell}}}}{
    \slofp{\n{\delta}_{\p{A}} ||- Q^{\plus} : \p{A}} &
    \infer[\rrule{\with}]{\slof{\psusp{\p{A}} |- \caseR[\ell \in L]{\ell => N_{\ell}} : \with*[sub=_{\ell \in L}]{\ell:C_{\ell}}}}{
      \multipremise{\ell \in L}{\slof{\psusp{\p{A}} |- N_{\ell} : C_{\ell}}}}}
  \\=\\
  \infer[\rrule{\with}]{\slof{\n{\delta}_{\p{A}} |- \caseR[\ell \in L]{\ell => [Q^{\plus}]N_{\ell}} : \with*[sub=_{\ell \in L}]{\ell:C_{\ell}}}}{
    \multipremise{\ell \in L}{
      \infer-{\slof{\n{\delta}_{\p{A}} |- [Q^{\plus}]N_{\ell} : C_{\ell}}}{
        \slofp{\n{\delta}_{\p{A}} ||- Q^{\plus} : \p{A}} & \slof{\psusp{\p{A}} |- N_{\ell} : C_{\ell}}}}}
\end{gather*}
The case involving $\n{\eta}$ is similar.
Which leaves the case involving $\rrule{+}$.
\begin{gather*}
  \infer-{\slof{\n{\delta}_{\p{A}} |- [Q^{\plus}](\spawn{R}{Q^{\plus}_{\p{C}}}) : \p{C}}}{
    \slofp{\n{\delta}_{\p{A}} ||- Q^{\plus} : \p{A}} &
    \infer[\rrule{+}]{\slof{\psusp{\p{A}} |- \spawn{R}{Q^{\plus}_{\p{C}}} : \p{C}}}{
      \slof{\psusp{\p{A}} |- R : \n{\delta}_{\p{C}}} &
      \slofp{\n{\delta}_{\p{C}} ||- Q^{\plus}_{\p{C}} : \p{C}}}}
  \\=\\
\end{gather*}

\begin{gather*}
  \infer-{\slof{\n{\delta}_{\p{A}} |- [Q^{\plus}](\spawn{R}{Q^{\plus}_{\p{C}}}) : \p{C}}}{
    \slofp{\n{\delta}_{\p{A}} ||- Q^{\plus} : \p{A}} &
    \infer[\rrule{+}]{\slof{\psusp{\p{A}} |- \spawn{R}{Q^{\plus}_{\p{C}}} : \p{C}}}{
      \infer[\rrule{-}]{\slof{\psusp{\p{A}} |- R : \n{\delta}_{\p{C}}}}{
        } &
      \slofp{\n{\delta}_{\p{C}} ||- Q^{\plus}_{\p{C}} : \p{C}}}}
  \\=\\
\end{gather*}

\begin{gather*}
  \infer-{\slof{\n{\delta}_{\p{A}} |- [Q^{\plus}](\spawn{\fwd}{Q^{\plus}_{\p{C}}}) : \p{C}}}{
    \slofp{\n{\delta}_{\p{A}} ||- Q^{\plus} : \p{A}} &
    \infer[\rrule{+}]{\slof{\psusp{\p{A}} |- \spawn{\fwd}{Q^{\plus}_{\p{C}}} : \p{C}}}{
      \infer[\p{\jrule{SUSP}}]{\slof{\psusp{\p{A}} |- \fwd : \psusp{\p{A}}}}{} &
      \slofp{\psusp{\p{A}} ||- Q^{\plus}_{\p{C}} : \p{C}}}}
  \\=\\
  ?
\end{gather*}

\subsection{Weak normalization}

\begin{gather*}
  \infer-[\jrule{WN}]{\slof{\plus*[sub=_{\ell \in L}]{\ell:A_{\ell}} |- n\bigl(\caseL[\ell \in L]{\ell => P_{\ell}}\bigr) : C}}{
    \infer[\lrule{\plus}]{\slof{\plus*[sub=_{\ell \in L}]{\ell:A_{\ell}} |- \caseL[\ell \in L]{\ell => P_{\ell}} : C}}{
      \multipremise{\ell \in L}{\slof{A_{\ell} |- P_{\ell} : C}}}}
  \\=\\
  \infer[\lrule{\plus}]{\slof{\plus*[sub=_{\ell \in L}]{\ell:A_{\ell}} |- \caseL[\ell \in L]{\ell => n(P_{\ell})} : C}}{
    \multipremise{\ell \in L}{
      \infer-[\jrule{WN}]{\slof{A_{\ell} |- n(P_{\ell}) : C}}{
        \slof{A_{\ell} |- P_{\ell} : C}}}}
\end{gather*}

\begin{gather*}
  \infer-[\jrule{WN}]{\slof{A |- n(\selectR{\kay}[P]) : \plus*[sub=_{\ell \in L}]{\ell:C_{\ell}}}}{
    \infer[\rrule{\plus}]{\slof{A |- \selectR{\kay}[P] : \plus*[sub=_{\ell \in L}]{\ell:C_{\ell}}}}{
      \text{($\kay \in L$)} & \slof{A |- P : C_{\kay}}}}
  \\=\\
  \infer[\rrule{+}]{\slof{A |- \spawn{\rho\bigl(n(P)\bigr)}{(\spawn{\fwd}{\selectR{\kay}})} : \plus*[sub=_{\ell \in L}]{\ell:C_{\ell}}}}{
    \infer[\rrule{-}]{\slof{A |- \rho\bigl(n(P)\bigr) : \n{C}_{\kay}}}{
      \infer-[\jrule{WN}]{\slof{A |- n(P) : \n{C}_{\kay}}}{
        \slof{A |- P : \n{C}_{\kay}}}} &
    \infer[\rrule{\plus}]{\slofp{\n{C}_{\kay} ||- \spawn{\fwd}{\selectR{\kay}} : \plus*[sub=_{\ell \in L}]{\ell:C_{\ell}}}}{
      \text{($\kay \in L$)} &
      \infer[\jrule{R-BLUR}]{\slofp{\n{C}_{\kay} ||- \fwd : \n{C}_{\kay}}}{}}}
\end{gather*}

%%% Local Variables:
%%% mode: latex
%%% TeX-master: "thesis"
%%% End:
