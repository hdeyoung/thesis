\chapter{Ordered rewriting}\label{ch:ordered-rewriting}

In this \lcnamecref{ch:ordered-rewriting}, we develop a rewriting interpretation of the ordered sequent calculus from the previous \lcnamecref{ch:ordered-logic}.

In \citeyear{Lambek:AMM58}, \citeauthor{Lambek:AMM58} developed a syntactic calculus, now known as the Lambek calculus, for formally describing the structure of sentences.\autocite{Lambek:AMM58}
Words are assigned syntactic types, which roughly correspond to grammatical parts of speech.
From a logical perspective, the Lambek calculus can [also] be viewed as a precursor to (and generalization of) \citeauthor{Girard:TCS87}'s linear logic\autocite{Girard:TCS87}\relax.\autocites{Polakow+Pfenning:MFPS99}{Polakow+Pfenning:TLCA99}
Implicit in \citeauthor{Lambek:AMM58}'s original article is a third perspective of the calculus: string rewriting.

In this \lcnamecref{ch:ordered-rewriting}, we review the Lambek calculus from a [string] rewriting perspective.








\section{Introduction}

In the previous \lcnamecref{ch:ordered-logic}, we saw that the ordered sequent calculus can be given a resource interpretation in which sequents $\oseq{\octx |- A}$ may be read as \enquote{From resources $\octx$, resource goal $A$ is achievable.}
For instance, the left rule for ordered conjunction ($\lrule{\fuse}$, see adjacent display)%
\marginnote{%
  $\infer[\lrule{\fuse}]{\oseq{\octx'_L \oc (A \fuse B) \oc \octx'_R |- C}}{
     \oseq{\octx'_L \oc A \oc B \oc \octx'_R |- C}}$%
}
was read \enquote{Goal $C$ is achievable from resource $A \fuse B$ if it is achievable from the separate resources $A \oc B$.}

As alluded in the previous \lcnamecref{ch:ordered-logic}'s discussion of ordered conjunction\footnote{See \cpageref{p:ordered-logic:ordered-conjunction}.}, this $\lrule{\fuse}$ rule is essentially a rule of resource decomposition: it decomposes [the resource] $A \fuse B$ into the separate resources $A \oc B$ and relegates the unchanged goal $C$ to a secondary role.

\newthought{%
This \lcnamecref{ch:ordered-rewriting}%
}
begins by exploring a refactoring of the ordered sequent calculus's left rules around this idea of resource decomposition~\parencref{sec:ordered-rewriting:??}.
Most of the left rules can be easily refactored in this way, although a few will prove resistant to the change.

Emphasizing resource decomposition naturally leads us to a rewriting interpretation of (a fragment of) ordered logic~\parencref{sec:ordered-rewriting:??}.
This rewriting system is closely related to traditional notions of string rewriting\autocite{??}, but simultaneously restricts and generalizes [...] along distinct axes.

The connection of ordered logic and the Lambek calculus to rewriting is certainly not new.
\Citeauthor{Lambek:AMM58}'s original article\autocite{Lambek:AMM58}

This development borrows from \citeauthor{Cervesato+Scedrov:IC09}'s work on intuitionistic linear logic as an expressive rewriting framework that generalizes traditional notions of multiset rewriting.\autocite{Cervesato+Scedrov:IC09}



\newthought{Most} of the left rules could be seen as decomposing resources.
The left rules were seen as decomposing resources, such as the $\lrule{\fuse}$~rule%
\marginnote{%
  $\infer[\lrule{\fuse}]{\oseq{\octx'_L \oc (A \fuse B) \oc \octx'_R |- C}}{
     \oseq{\octx'_L \oc A \oc B \oc \octx'_R |- C}}$%
}
decomposing $A \fuse B$ into the resources $A \oc B$.
The right rules, on the other hand, were seen as ...

Replacing the left rules with a single, common rule ... and a new judgment, $\octx \reduces \octx'$, that exposes [makes [more] explicit] the decomposition of resources/state transformation aspect.


\section{Most left rules decompose ordered resources}

Recall two of the ordered sequent calculus's left rules: $\lrule{\fuse}$ and $\lrule{\with}_1$.
\begin{inferences}
  \infer[\lrule{\fuse}]{\oseq{\octx'_L \oc (A \fuse B) \oc \octx'_R |- C}}{
    \oseq{\octx'_L \oc A \oc B \oc \octx'_R |- C}}
  \and
  \infer[\lrule{\with}_1]{\oseq{\octx'_L \oc (A \with B) \oc \octx'_R |- C}}{
    \oseq{\octx'_L \oc A \oc \octx'_R |- C}}
\end{inferences}
Both rules decompose the principal resource: in the $\lrule{\fuse}$ rule, $A \fuse B$ into the separate resources $A \oc B$; and, in the $\lrule{\with}_1$ rule, $A \with B$ into $A$.
However, in both cases, the resource decomposition is somewhat obscured by boilerplate.
The framed contexts $\octx'_L$ and $\octx'_R$ and goal $C$ serve to enable the rules to be applied anywhere [in the string of resources], without restriction;
these concerns are not specific to the $\lrule{\fuse}$ and $\lrule{\with}_1$ rules, but are general boilerplate that arguably should be factored out.

To decouple the resource decomposition from the surrounding boilerplate, we will introduce a new judgment: $\octx \reduces \octx'$, meaning \enquote{Resources $\octx$ may be decomposed into [resources] $\octx'$.}
With this judgment in hand, the boilerplate can be factored into a uniform left rule, $\lrule{\star}$:
\begin{equation*}
  \infer[\lrule{\star}]{\oseq{\octx'_L \oc \octx \oc \octx'_R |- C}}{
    \octx \reduces \octx' & \oseq{\octx'_L \oc \octx' \oc \octx'_R |- C}}
  .
\end{equation*}

For many of the ordered logical connectives, this approach  works perfectly.
The decomposition of $A \fuse B$ into $A \oc B$ is, for example, captured by
\begin{equation*}
  \infer[\lrule{\fuse}']{A \fuse B \reduces A \oc B}{}
  ,
\end{equation*}
so that the ordered sequent calculus's standard $\lrule{\fuse}$ rule
% left rule for multiplicative conjunction
is then derivable from the uniform left rule:
\begin{equation*}
  \infer[\lrule{\fuse}]{\oseq{\octx'_L \oc (A \fuse B) \oc \octx'_R |- C}}{
    \oseq{\octx'_L \oc A \oc B \oc \octx'_R |- C}}
  %
  \enspace\leftrightsquigarrow\enspace
  %
  \infer[\lrule{\star}]{\oseq{\octx'_L \oc (A \fuse B) \oc \octx'_R |- C}}{
    \infer[\lrule{\fuse}']{A \fuse B \reduces A \oc B}{} &
    \oseq{\octx'_L \oc A \oc B \oc \octx'_R |- C}}
  .
\end{equation*}
The left rules for $\one$ and $A \with B$ can be refactored in a similar way.
Despite their additional, minor premises, even the left rules for left- and right-handed implications can be refactored in this way.
\begin{inferences}
  \infer[\lrule{\limp}']{\octx \oc (A \limp B) \reduces B}{
    \oseq{\octx |- A}}
  \and
  \infer[\lrule{\pmir}']{(B \pmir A) \oc \octx \reduces B}{
    \oseq{\octx |- A}}
\end{inferences}

\begin{equation*}
  \infer[\lrule{\limp}]{\oseq{\octx'_L \oc \octx \oc (A \limp B) \oc \octx'_R |- C}}{
    \oseq{\octx |- A} &
    \oseq{\octx'_L \oc B \oc \octx'_R |- C}}
  %
  \enspace\leftrightsquigarrow\enspace
  %
  \infer[\lrule{\star}]{\oseq{\octx'_L \oc \octx \oc (A \limp B) \oc \octx'_R |- C}}{
    \infer[\lrule{\limp}']{\octx \oc (A \limp B) \reduces B}{
      \oseq{\octx |- A}} &
    \oseq{\octx'_L \oc B \oc \octx'_R |- C}}
\end{equation*}


Unfortunately, the left rules for neither additive disjunction, $A \plus B$,%
\marginpar{%
  $\infer[\lrule{\plus}]{\oseq{\octx'_L \oc (A \plus B) \oc \octx'_R |-  C}}{
     \oseq{\octx'_L \oc A \oc \octx'_R |-  C} &
     \oseq{\octx'_L \oc B \oc \octx'_R |-  C}}$%
}
nor its unit, $\zero$, can be refactored in this way.
The difficulty with additive disjunction isn't that its left rule, $\lrule{\plus}$, doesn't decompose the resource $A \plus B$.
The $\lrule{\plus}$ rule certainly does decompose $A \plus B$, but it does so [...].
[...] retain the standard $\lrule{\plus}$ and $\lrule{\zero}$ rules.

\begin{figure}[tbp]
  \begin{inferences}
    \infer[\jrule{CUT}\smash{^A}]{\oseq{\octx'_L \oc \octx \oc \octx'_R |- C}}{
      \oseq{\octx |- A} & \oseq{\octx'_L \oc A \oc \octx'_R |- C}}
    \and 
    \kern-.05em
    \infer[\jrule{ID}\smash{^A}]{\oseq{A |- A}}{}
    \kern-.05em
    \and
    \infer[\lrule{\star}]{\oseq{\octx'_L \oc \octx \oc \octx'_R |- C}}{
      \octx \reduces \octx' & \oseq{\octx'_L \oc \octx' \oc \octx'_R |- C}}
    \\
    \infer[\rrule{\fuse}]{\oseq{\octx_1 \oc \octx_2 |- A \fuse B}}{
      \oseq{\octx_1 |- A} & \oseq{\octx_2 |- B}}
    \and
    \infer[\lrule{\fuse}']{A \fuse B \reduces A \oc B}{}
    \\
    \infer[\rrule{\one}]{\oseq{\octxe |- \one}}{}
    \and
    \infer[\lrule{\one}']{\one \reduces \octxe}{}
    \\
    \infer[\rrule{\with}]{\oseq{\octx |- A \with B}}{
      \oseq{\octx |- A} & \oseq{\octx |- B}}
    \and
    \infer[\lrule{\with}_1']{A \with B \reduces A}{}
    \and
    \infer[\lrule{\with}_2']{A \with B \reduces B}{}
    \\
    \infer[\rrule{\top}]{\oseq{\octx |- \top}}{}
    \and
    \text{(no $\lrule{\top}'$ rule)}
    \\
    \infer[\rrule{\limp}]{\oseq{\octx |- A \limp B}}{
      \oseq{A \oc \octx |- B}}
    \and
    \infer[\lrule{\limp}']{\octx \oc (A \limp B) \reduces B}{
      \oseq{\octx |- A}}
    \\
    \infer[\rrule{\pmir}]{\oseq{\octx |- B \pmir A}}{
      \oseq{\octx \oc A |- B}}
    \and
    \infer[\lrule{\pmir}']{(B \pmir A) \oc \octx \reduces B}{
      \oseq{\octx |- A}}
    \\
    \infer[\rrule{\plus}_1]{\oseq{\octx |- A \plus B}}{
      \oseq{\octx |- A}}
    \and
    \infer[\rrule{\plus}_2]{\oseq{\octx |- A \plus B}}{
      \oseq{\octx |- B}}
    \and
    \infer[\lrule{\plus}]{\oseq{\octx'_L \oc (A \plus B) \oc \octx'_R |- C}}{
      \oseq{\octx'_L \oc A \oc \octx'_R |- C} &
      \oseq{\octx'_L \oc B \oc \octx'_R |- C}}
    \\
    \text{(no $\rrule{\zero}$ rule)}
    \and
    \infer[\lrule{\zero}]{\oseq{\octx'_L \oc \zero \oc \octx'_R |- C}}{}
  \end{inferences}
  \caption{A refactoring of the ordered sequent calculus to emphasize that many left rules amount to resource decomposition}\label{fig:ordered-rewriting:decompose-seq-calc}
\end{figure}

\Cref{fig:ordered-rewriting:decompose-seq-calc} presents the fully refactored sequent calculus for ordered logic.

\begin{theorem}
  $\oseq{\octx |- A}$ is derivable in the refactored calculus of \cref{fig:ordered-rewriting:decompose-seq-calc} if, and only if, $\oseq{\octx |- A}$ is derivable in the ordered sequent calculus of \cref{fig:ordered-logic:sequent-calculus}.
\end{theorem}
%
\begin{proof}
  In each direction, by structural induction on the given derivation.

  To prove the left-to-right direction, the inductive hypothesis is generalized to include: 
  \begin{quotation}
    If $\octx \reduces \octx'$, then $\oseq{\octx |- \bigfuse \octx'}$ in the standard ordered sequent calculus.
  \end{quotation}
  where $\bigfuse \octx'$ is a proposition that reifies the context $\octx'$:
  \begin{equation*}
    \begin{aligned}
      \textstyle \bigfuse (\octx_1 \oc \octx_2)
        &= \textstyle (\bigfuse \octx_1) \fuse (\bigfuse \octx_2) \\
      \textstyle \bigfuse (\octxe) &= \one \\
      \textstyle \bigfuse A &= A
    \end{aligned}
  \end{equation*}
  A simple lemma about this operation is also used:
  If $\oseq{\octx'_L \oc \octx \oc \octx'_R |- C}$, then $\oseq{\octx'_L \oc (\bigfuse \octx) \oc \octx'_R |- C}$.

  As an example, consider the case of the uniform left rule, $\lrule{\star}$:
  \begin{equation*}
    \infer[\lrule{\star}]{\oseq{\octx'_L \oc \octx \oc \octx'_R |- C}}{
      \octx \reduces \octx' &
      \oseq{\octx'_L \oc \octx' \oc \octx'_R |- C}}
  \end{equation*}
  By the inductive hypothesis, both $\oseq{\octx |- \bigfuse \octx'}$ and $\oseq{\octx'_L \oc \octx' \oc \octx'_R |- C}$ in the standard sequent calculus.
  We may then construct
  \begin{equation*}
    \infer[\jrule{CUT}\smash{^{\bigfuse \octx'}}]{\oseq{\octx'_L \oc \octx \oc \octx'_R |- C}}{
      \oseq{\octx |- \bigfuse \octx'} &
      \infer-{\oseq{\octx'_L \oc (\bigfuse \octx') \oc \octx'_R |- C}}{
        \oseq{\octx'_L \oc \octx' \oc \octx'_R |- C}}}
    \qedhere
  \end{equation*}
\end{proof}


\section{Decomposition as rewriting}

Instead of taking the resource decomposition rules as a basis for a reconfigured sequent calculus, we can view them as the foundation of a rewriting system.
For example, the decomposition of resource $A \fuse B$, 
\begin{equation*}
  \infer[\lrule{\fuse}']{A \fuse B \reduces A \oc B}{}
\end{equation*}
can be seen as rewriting $A\fuse B$ into $A \oc B$.

Viewing the resource decomposition rules for left- and right-handed implications as rewriting rules is slightly problematic, however.%
\marginnote{%
  \begin{gather*}
    \infer[\lrule{\limp}']{\octx \oc (A \limp B) \reduces B}{
      \oseq{\octx |- A}}
    \\
    \infer[\lrule{\pmir}']{(B \pmir A) \oc \octx \reduces B}{
      \oseq{\octx |- A}}
  \end{gather*}
}
Notice that the premises of these rules both require proofs of $\oseq{\octx |-  A}$.
In the refactored sequent calculus of \cref{fig:ordered-rewriting:decompose-seq-calc}, that dependence of judgments is fine.
But for a rewriting system, including arbitrary[/general] proofs would be odd -- rewriting should be a syntax-directed process and should not depend on provability.

Fortunately, there is a simple restriction that can be made: demand that 
\begin{inferences}
  \infer[\lrule{\limp}']{A \oc (A \limp B) \reduces B}{}
  \and
  \infer[\lrule{\pmir}']{(B \pmir A) \oc A \reduces B}{}
\end{inferences}

\begin{equation*}
  \infer[\jrule{CUT}\smash{^A}]{\oseq{\octx'_L \oc \octx \oc (A \limp B) \oc \octx'_R |- C}}{
    \oseq{\octx |- A} &
    \infer[\lrule{\star}]{\oseq{\octx'_L \oc A \oc (A \limp B) \oc \octx'_R |- C}}{
      \infer[\lrule{\limp}']{A \oc (A \limp B) \reduces B}{} &    
      \oseq{\octx'_L \oc B \oc \octx'_R |- C}}}
\end{equation*}

\begin{figure}[tbp]
  \begin{inferences}
    \infer{\octx_1 \oc \octx_2 \reduces \octx'_1 \oc \octx_2}{
      \octx_1 \reduces \octx'_1}
    \and
    \infer{\octx_1 \oc \octx_2 \reduces \octx_1 \oc \octx'_2}{
      \octx_2 \reduces \octx'_2}
    \\
    \infer[\lrule{\fuse}']{A \fuse B \reduces A \oc B}{}
    \and
    \infer[\lrule{\one}']{\one \reduces \octxe}{}
    \\
    \infer[\lrule{\with}_1']{A \with B \reduces A}{}
    \and
    \infer[\lrule{\with}_2']{A \with B \reduces B}{}
    \and
    \text{(no $\lrule{\top}'$ rule)}
    \\
    \infer[\lrule{\limp}']{A \oc (A \limp B) \reduces B}{}
    \and
    \infer[\lrule{\pmir}']{(B \pmir A) \oc A \reduces B}{}
    \\
    \text{(no $\lrule{\plus}'$ and $\lrule{\zero}'$ rules)}
  \end{inferences}

  \begin{inferences}
    \infer{\octx \Reduces \octx}{}
    \and
    \infer{\octx \Reduces \octx''}{
      \octx \Reduces \octx' & \octx' \Reduces \octx''}
    \and
    \infer{\octx \Reduces \octx'}{
      \octx \reduces \octx'}
  \end{inferences}
  \caption{A rewriting fragment of ordered logic, based on resource decomposition}\label{fig:ordered-rewriting:rewriting}
\end{figure}





\section{Propositional ordered rewriting}

In this \lcnamecref{sec:ordered-rewriting:general}, we develop a rewriting interpretation of the ordered sequent calculus from the previous \lcnamecref{ch:ordered-logic}.
This development closely follows \citeauthor{Cervesato+Scedrov:IC09}'s work on intuitionistic linear logic as a multiset rewriting framework.\autocite{Cervesato+Scedrov:IC09}

Just as their linear logical rewriting framework is more expressive than multiset rewriting, ordered rewriting framework presented in this chapter can be seen as an extension of traditional notions of string rewriting.


\begin{equation*}
  \infer*{\oseq{\octx |- A}}{
    \oseq{\octx' |- A'}}
\end{equation*}


Many of the ordered sequent calculus's left rules consist of a single major premise with the same consequent as in the rule's conclusion [sequent], as well as a minor premise in the case of the $\lrule{\limp}$ and $\lrule{\pmir}$ rules.
\begin{inferences}
  \infer[\lrule{\fuse}]{\oseq{\octx'_L \oc (A \fuse B) \oc \octx'_R |- C}}{
    \oseq{\octx'_L \oc A \oc B \oc \octx'_R |- C}}
  \and
  \infer[\lrule{\with}_1]{\oseq{\octx'_L \oc (A \with B) \oc \octx'_R |- C}}{
    \oseq{\octx'_L \oc A \oc \octx'_R |- C}}
\end{inferences}
Both rules, at their core, decompose resources -- the resource $A \fuse B$ into the separate resources $A \oc B$; and the resource $A \with B$ into the resource $A$.
The resource decomposition is somewhat obscured 
Notice that much of these two rules is devoted to shared scaffolding/boilerplate -- the framing contexts $\octx'_L$ and $\octx'_R$, and goal consequent $C$ that remain unchanged from conclusion to premise.

Because so many rules share this scaffolding, it might be worthwhile to restructure the ordered sequent calculus to expose this shared scaffolding.
\begin{equation*}
  \infer{\oseq{\octx |- C}}{
    \octx \reduces \octx' & \oseq{\octx' |- C}}
\end{equation*}
For instance, if $\octx_L \oc (A \fuse B) \oc \octx_R \reduces \octx_L \oc A \oc B \oc \octx_R$ holds, then the usual $\lrule{\fuse}$ rule is a derivable instance of this generalized left rule.


\begin{theorem}
  $\oseq{\octx |- A}$ in ... if and only if $\oseq{\octx |- A}$ in ...
\end{theorem}
\begin{proof}
  The two directions are proved separately, each by induction on the structure of the given derivation.
  \begin{gather*}
    \infer[\lrule{\with}_1]{\oseq{\octx'_L \oc (A \with B) \oc \octx'_R |- C}}{
      \oseq{\octx'_L \oc A \oc \octx'_R |- C}}
    \\\rightsquigarrow\\
    \infer[]{\oseq{\octx'_L \oc (A \with B) \oc \octx'_R |- C}}{
      \infer[]{\octx'_L \oc (A \with B) \oc \octx'_R \reduces \octx'_L \oc A \oc \octx'_R}{
        \infer[]{(A \with B) \oc \octx'_R \reduces A \oc \octx'_R}{
        \infer[\lrule{\with}'_1]{A \with B \reduces A}{}}} &
      \oseq{\octx'_L \oc A \oc \octx'_R |- C}}
  \end{gather*}

  \begin{equation*}
    \begin{lgathered}
      \bigfuse (\octx_1 \oc \octx_2) = (\bigfuse \octx_1) \fuse (\bigfuse \octx_2) \\
      \bigfuse (\octxe) = \one \\
      \bigfuse A = A
    \end{lgathered}
  \end{equation*}

  \begin{lemma}
    If\/ $\octx \reduces \octx'$, then $\oseq{\octx |- \bigfuse \octx'}$.
    $\oseq{\octx' |- \bigfuse \octx'}$ for all $\octx'$.
  \end{lemma}
\end{proof}

\begin{theorem}
  If $\oseq{\octx |- A}$ and $\octx'_L \oc A \oc \octx'_R \reduces \octx'$, then $\oseq{\octx'_L \oc \octx \oc \octx'_R |- \bigfuse \octx'}$.
\end{theorem}

\begin{syntax*}
  Propositions &
    A & p \mid A \limp B \mid B \pmir A
          \mid A \fuse B \mid \one
          \mid A \with B \mid \top
  \\
  Ordered contexts & 
    \octx & \octxe \mid \octx_1 \oc \octx_2 \mid A
\end{syntax*}

\begin{itemize}
\item Lambek calculus and rewriting; compare to multiset rewriting; compare to string rewriting
\item Explain why $\plus$ and $\zero$ (and $\bot$) are undesirable here.
\item Connections to left rules
\end{itemize}

The rewriting relation is the smallest compatible relation that satisfies:
\begin{inferences}
  \infer{A \oc (A \limp B) \reduces B}{}
  \and
  \infer{(B \pmir A) \oc A \reduces B}{}
  \\
  \infer{A \with B \reduces A}{}
  \and
  \infer{A \with B \reduces B}{}
  \and
  \text{(no rule for $\top$)}
  \\
  \infer{A \fuse B \reduces A \oc B}{}
  \and
  \infer{\one \reduces \octxe}{}
\end{inferences}
We will also refer to this relation as \vocab{reduction}%
\footnote{Input transitions are postponed to \cref{ch:ordered-bisimilarity}.}%
.

$\Reduces$ is the reflexive-transitive closure of $\reduces$


\begin{equation*}
  \infer[\lrule{\with}_1]{\oseq{\octx'_L \oc (A \with B) \oc \octx'_R |- \gamma}}{
    \oseq{\octx'_L \oc A \oc \octx'_R |- \gamma}}
  \leftrightsquigarrow
  \infer{\octx'_L \oc (A \with B) \oc \octx'_R \reduces \octx'_L \oc A \oc \octx'_R}{
    \infer{A \with B \reduces A}{}}
\end{equation*}

\begin{equation*}
  \infer[\lrule{\limp}]{\oseq{\octx'_L \oc \octx \oc (A \limp B) \oc \octx'_R |- \gamma}}{
    \oseq{\octx |- A} & \oseq{\octx'_L \oc B \oc \octx'_R |- \gamma}}
  \rightsquigarrow
  \infer[\lrule{\limp}']{\oseq{\octx'_L \oc A \oc (A \limp B) \oc \octx'_R |- \gamma}}{
    \oseq{\octx'_L \oc B \oc \octx'_R |- \gamma}}
  \leftrightsquigarrow
  \infer{\octx'_L \oc A \oc (A \limp B) \oc \octx'_R \reduces \octx'_L \oc B \oc \octx'_R}{
    \infer{A \oc (A \limp B) \reduces B}{}}
\end{equation*}


\subsection{Definitions}

\begin{itemize}
\item not very interesting without recursion
\end{itemize}

\subsection{Examples}

\paragraph*{Automata and transducers}

\begin{equation*}
  \begin{lgathered}[t]
    q_0 \defd (a \limp q_0) \with (b \limp q_0 \with q_1) \\
    q_1 \defd (a \limp q_2) \with (b \limp q_2) \with (\emp \limp \one) \\
    q_2 \defd (a \limp q_2) \with (b \limp q_2)
  \end{lgathered}
  \qquad
  \begin{lgathered}[t]
    s_0 \defd (a \limp s_0) \with (b \limp s_1) \\
    s_1 \defd (a \limp s_0) \with (b \limp s_1) \with (\emp \limp \one)
  \end{lgathered}
\end{equation*}

\begin{equation*}
  \nfa{q} \defd \bigwith_{a \in \ialph} \bigl({\textstyle a \limp \bigwith_{q'_a} \nfa{q}'_a}\bigr)
\end{equation*}

\begin{theorem}
  Let $\aut{A} = (Q, \mathord{\nfareduces}, F)$ be \iac{NFA} over an input alphabet $\ialph$.
  Then:
  \begin{itemize}[nosep]
  \item $q \nfareduces[a] q'$ if and only if $\atm{a} \oc \nfa{q} \Reduces \nfa{q}'$.
  \item $q \in F$ if and only if $\atm{\emp} \oc \nfa{q} \Reduces \octxe$.
  \item $q \notin F$ if and only if $\atm{\emp} \oc \nfa{q} \longarrownot\reduces$.\alertnote{Careful -- depends on focusing!}
  \end{itemize}
  % \item
  %   If $\atm{a} \oc \nfa{q} \Reduces \nfa{q}'$, then $q \nfareduces[a] q'$.
  %   If $\atm{\emp} \oc \nfa{q} \Reduces \octxe$, then $q \in F$.
\end{theorem}


\paragraph*{Binary counter}

\begin{equation*}
  \begin{lgathered}
    e \defd (e \fuse b_1 \pmir i) \with (z \pmir d) \\
    b_0 \defd (b_1 \pmir i) \with (d \fuse b'_0 \pmir d) \\
    b_1 \defd (i \fuse b_0 \pmir i) \with (b_0 \fuse s \pmir d) \\
    b'_0 \defd (z \limp z) \with (s \limp b_1 \fuse s)
  \end{lgathered}
\end{equation*}

\begin{itemize}
\item Concurrency that is not message-passing
\item Alternative choreography -- how are these related?
\begin{equation*}
  \begin{lgathered}
    i \defd (e \limp e \fuse b_1) \with (b_0 \limp b_1) \with (b_1 \limp i \fuse b_0) \\
    d \defd (e \limp z) \with (b_0 \limp d \fuse b'_0) \with (b_1 \limp b_0 \fuse s) \\
    b'_0 \defd (z \limp z) \with (s \limp b_1 \fuse s)
  \end{lgathered}
\end{equation*}

\begin{equation*}
  \begin{lgathered}
    i \defd (e \limp e \fuse b_1) \with (b_0 \limp b_1) \with (b_1 \limp i \fuse b_0) \\
    d \defd (e \limp z) \with (b_0 \limp d \fuse b'_0) \with (b_1 \limp b_0 \fuse s) \\
    z \defd z \pmir b'_0 \\
    s \defd b_1 \fuse s \pmir b'_0
  \end{lgathered}
\end{equation*}
\end{itemize}


\section{}



%%% Local Variables:
%%% mode: latex
%%% TeX-master: "thesis"
%%% End:
